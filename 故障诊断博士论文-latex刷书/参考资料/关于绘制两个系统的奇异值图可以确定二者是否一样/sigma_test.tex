% This LaTeX document needs to be compiled with XeLaTeX.
\documentclass[10pt]{article}

\begin{document}
\maketitle





\usepackage{amsmath,amssymb}
\usepackage{tcolorbox}
\tcbuselibrary{skins, breakable}










\subsection{Verification of Left Coprime Factorization via Singular Value Analysis}

For a multi-input multi-output (MIMO) system $G(s)$, the left coprime factorization (LCF) is expressed as
\begin{equation}
G(s) = M_l^{-1}(s) N_l(s),
\end{equation}
where both $M_l(s)$ and $N_l(s)$ are stable, minimal, and proper transfer matrices.  
They satisfy the normalization condition
\begin{equation}
M_l(s)M_l^*(s) + N_l(s)N_l^*(s) = I,
\end{equation}
which ensures that $\begin{bmatrix} M_l & N_l \end{bmatrix}$ forms an \emph{inner system}, i.e.,
\begin{equation}
\begin{bmatrix} M_l(j\omega) & N_l(j\omega) \end{bmatrix}^*
\begin{bmatrix} M_l(j\omega) & N_l(j\omega) \end{bmatrix} = I.
\end{equation}
Thus, this decomposition is ``normalized.''

\subsubsection{Verification Principle}

To verify that the computed factors satisfy $G(s)=M_l^{-1}(s)N_l(s)$, one can compare the singular value plots of both systems:
\begin{equation}
\sigma\big(G(j\omega)\big) \quad \text{and} \quad 
\sigma\big(M_l^{-1}(j\omega)N_l(j\omega)\big),
\end{equation}
where $\sigma(\cdot)$ denotes the vector of singular values of a matrix.

If the two sets of singular values coincide over the entire frequency range, it implies that
\begin{equation}
\sigma_i(G(j\omega)) = \sigma_i(M_l^{-1}(j\omega)N_l(j\omega)),
\quad \forall i,\;\forall \omega,
\end{equation}
which confirms that the two systems exhibit identical gain characteristics across all input–output directions.  
Because the LCF guarantees consistent phase behavior (through the unitary property of the inner factor), this equality of singular values is sufficient to conclude
\begin{equation}
G(s) = M_l^{-1}(s)N_l(s).
\end{equation}

\subsubsection{Physical Interpretation}

For a MIMO transfer matrix $G(j\omega)$, each singular value $\sigma_i(G(j\omega))$ represents the gain of the system along a specific input–output direction at frequency $\omega$.  
Hence, matching singular value spectra between $G$ and $M_l^{-1}N_l$ means the two systems amplify all directions identically at every frequency.

In MATLAB, this verification is typically performed as:
\begin{verbatim}
sigma(sys,'b-', Ml\Nl,'r--')
\end{verbatim}
where the blue line represents the original system $G(s)$ and the red dashed line represents the reconstructed system $M_l^{-1}(s)N_l(s)$.  
If the two curves completely overlap, the factorization is confirmed to be correct.

\subsubsection{Summary}

\begin{tcolorbox}[colback=blue!5!white,colframe=blue!60!black,title=\textbf{Conclusion}]
Comparing the singular value spectra of $G(s)$ and $M_l^{-1}(s)N_l(s)$ provides a numerical validation of the left coprime factorization.
Because the normalized property ensures phase consistency, identical singular values across all frequencies imply
\[
G(s) = M_l^{-1}(s)N_l(s).
\]
\end{tcolorbox}


\end{document}