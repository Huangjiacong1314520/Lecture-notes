\documentclass[10pt]{article}
\usepackage{researchnotes}
\usepackage[utf8]{inputenc}
\usepackage[T1]{fontenc}
\usepackage{amsmath}
\usepackage{amsfonts}
\usepackage{amssymb}
\usepackage{array}
\usepackage{ctex}
\usepackage[version=4]{mhchem}
\usepackage{stmaryrd}
\usepackage{hyperref}
\hypersetup{colorlinks=true, linkcolor=blue, filecolor=magenta, urlcolor=cyan,}
\urlstyle{same}
\usepackage{graphicx}
\usepackage[export]{adjustbox}
\graphicspath{ {./images/} }
\usepackage{caption}
\usepackage{mathrsfs}
\usepackage{bbold}
\usepackage{tikz}
\usepackage{listings}
\usetikzlibrary{shapes,arrows,positioning}

\title{稳定性的逻辑}

\author{黄嘉聪}
\date{}



\begin{document}
\maketitle
\captionsetup{singlelinecheck=false}


\begin{abstract}
主要介绍了QK的作用和怎么想到构造这玩意的
\end{abstract}

KEY WORDS: robust control; $\mu$; skew $\mu$; lower bound

\section{\textbf{$(Q_K)$} 的作用和"为什么会想到构造它"}

很好,这里把 \textbf{$(Q_K)$} 的作用和"为什么会想到构造它"讲成一条完整链路:从 $\mu$ 的定义 $\rightarrow$ 同质性与单位化 $\rightarrow$ "极点/边界原则" $\rightarrow$ 分块极点的具体形态 $\rightarrow$ 计算与算法的好处。你读完就能明白:$(Q_K)$ 不是拍脑袋的集合,它是从问题结构里"被逼出来"的。

\subsection*{1) 从 $\mu$ 定义到"只需关注方向"的单位化}

混合-$\mu$ 的"奇异化"定义是
\[
\mu_K(M)=\Big(\min_{\Delta\in X_K}\{\|\Delta\| : \det(I-\Delta M)=0\}\Big)^{-1}.
\]

\begin{itemize}
    \item 这个优化是\textbf{同质}的:如果 $\Delta$ 使 $I-\Delta M$ 奇异,那么对任何 $t>0$,"放缩" $t\Delta$ 也只是在尺度上改变。
    \item 因此可把 $\Delta$ 写成"半径×方向"的极坐标:$\Delta=t\widehat{\Delta}$,其中 $\|\widehat{\Delta}\|=1$。
    \item 把"最小半径"问题改写为"在单位壳上挑方向"的问题,得到
    \[
    \mu_K(M)=\max_{\widehat{\Delta}\in X_K,\ \|\widehat{\Delta}\|\le 1}\ \rho_R(\widehat{\Delta}M).
    \]
    也就是说,\textbf{只需要在单位球(单位壳)上挑"方向"},把 $\widehat{\Delta}M$ 的实特征值推到最大即可。
\end{itemize}

\subsection*{2) 极点/边界原则:最优方向在"单位球的极点/边界"}

在线性-分式的目标(或由特征值/瑞利商诱导的目标)下,带范数约束的最优化常常在\textbf{单位球的极点(extreme points)或边界}达到最优:

\begin{itemize}
    \item 对标量块:$\max\{|q|:\ |q|\le 1\}$ 的最优必在 $|q|=1$ 或端点;
    \item 对矩阵块:$\max\{\operatorname{Re}\langle \Delta,X\rangle:\ \|\Delta\|_\sigma\le 1\}$ 的最优由\textbf{极点}取得,等价于"把 $\Delta$ 与 $X$ 的奇异向量对齐",并把奇异值推到边界(见冯-诺依曼迹不等式、极点表征)。
\end{itemize}

于是我们自然会把"单位球"进一步缩到它的\textbf{边界极点集合}上搜索。这就是 $(Q_K)$ 的雏形。

\subsection*{3) 分块到极点:$(Q_K)$ 的具体构造}

$X_K$ 是块对角不确定性集合(实重复标量、复重复标量、全复块等的直和)。把"极点/边界原则"逐块应用:

\begin{itemize}
    \item \textbf{复重复标量块} $\delta I_r$:单位球 $|\delta|\le 1$ 的极点在\textbf{单位圆}上,故取 $|\delta|=1\Rightarrow \delta=e^{\mathrm j\phi}$。
    $\checkmark$ 这给出 $(Q_K)$ 中"复标量块相位在单位圆上"。
    
    \item \textbf{实重复标量块} $\delta I_r$:单位球 $|\delta|\le 1$ 的边界是区间端点和内点;在极值问题里通常"卡在端点" $\delta=\pm 1$(若一阶最优性迫使留在内点,$\delta\in[-1,1]$ 也被 $(Q_K)$ 覆盖)。
    $\checkmark$ 这给出 $(Q_K)$ 中"实标量块在 $[-1,1]$"。
    
    \item \textbf{全复块} $\Delta_F$(方阵,$\|\Delta_F\|_\sigma\le 1$):把 $\Delta_F=U\Sigma V^*$(SVD),当与目标矩阵对齐时,最优会把可用的奇异值推到边界 $\Sigma=I$(或一个部分等距算子)。当块是方阵时,可取\textbf{酉矩阵}作为极点代表($\Delta_F\in\mathbb C^{p\times p}\Rightarrow \Delta_F$ 选酉 $U V^*$)。
    $\checkmark$ 这给出 $(Q_K)$ 中"全复块取酉矩阵"。
\end{itemize}

把三类块的"边界代表"做块对角直和,就得到
\[
Q_K=\operatorname{diag}\big(q_1 I,\dots,\ e^{\mathrm j\phi_\ell}I,\ \ U_1,\dots\big),
\]
这正是你看到的那组"相位/符号/酉"的集合。\textbf{它只保留'方向/相位',不含'半径'};半径由最终的特征值 $\lambda$(或放缩 $t$)承担。

\subsection*{4) $(Q_K)$ 的三个关键作用}

\subsubsection*{作用 A:把 $\mu$ 的定义变成"边界上的实特征值最大化"}

有了单位化与极点化,得到
\[
\mu_K(M)=\max_{\Delta\in B_{X_K}}\rho_R(\Delta M)
\quad\Longrightarrow\quad
\mu_K(M)=\max_{Q\in Q_K}\rho_R(QM).
\]
这一步是论文里的核心结论:\textbf{只在 $(Q_K)$ 上搜索就能达到 $\mu$}。直观上:把"允许的半径"统一提到边界而不丧失可达性,剩下只调"方向"。

\subsubsection*{作用 B:把"块范数约束"变成"相位/符号/酉"的\textbf{离散(或低维)参数化}}

\begin{itemize}
    \item 复标量只剩一个相位 $\phi$;
    \item 实标量只剩一个符号(或内点标量 $q\in[-1,1]$);
    \item 全复块只剩一个酉矩阵(方向)。
\end{itemize}
这极大简化了搜索与迭代(不再在半径上纠结)。

\subsubsection*{作用 C:给"幂迭代"提供\textbf{可投影的闭式更新}}

Young–Doyle 下界的幂迭代之所以能落地,就是因为对每个块都有\textbf{显式的"投影/对齐"规则}:

\begin{itemize}
    \item 复标量块:取 $q=\dfrac{w^*a}{|w^*a|}$ 的相位;
    \item 实标量块:按 $\operatorname{sign}(\Re(a^*w))$ 卡到 $\pm 1$,或保持内点;
    \item 全复块:把 $a,w$ 的能量配平,等价于选取一个等距算子/酉使两端"对齐"。
\end{itemize}
这些正是 $(Q_K)$ 结构带来的\textbf{块级闭式解},使得每一步投影都简单、稳定。

\subsection*{5) 为什么会"想到"构造 $(Q_K)$?}

\textbf{这是从三个一线索自然推演出来的:}

\begin{enumerate}
    \item \textbf{同质性与单位化}:
    $\mu$ 的定义本身是"放缩不变"的 $\Rightarrow$ 把半径抽掉,只看方向 $\Rightarrow$ 搜索单位壳。
    
    \item \textbf{极点原则}(凸分析/算子理论的常识):
    在"范数$\le 1$"的约束下,线性或谱型目标的最优往往在\textbf{极点/边界} $\Rightarrow$ 只看单位壳的"最外层"。
    
    \item \textbf{结构可分性}(块对角):
    结构化集合是直和 $\Rightarrow$ "极点"也可逐块表征;
    对标量块,"极点"就是"相位或符号";对全复块,"极点"是"等距/酉"。
    于是诞生了 $(Q_K)$:\textbf{"单位边界上的复块 + 实块在 $[-1,1]$ 的相位/符号刻度 + 全复块的酉"}。
\end{enumerate}

这三点合在一起,$(Q_K)$ 并不是人为设定的"奇技淫巧",而是从 $\mu$ 的数学结构\textbf{推演得到的"边界方向集"};它既有\textbf{理论最优性}(最大化到 $\mu$),又有\textbf{数值可实现性}(幂迭代时易投影)。

\subsection*{6) 小结(一句话版)}

\textbf{$(Q_K)$} 是把"结构化单位球"收缩到"各块的边界极点"的集合;
它保留了让 $\mu$ 取到最大值所需的\textbf{方向信息},同时把"幅度/半径"交给特征值本身;
因此既保证\textbf{理论上不损失最优},又让\textbf{算法上每步更新有闭式投影},是混合-$\mu$ 下界计算的关键角色。






\section{为什么只在$Q_K$​(复块取单位圆相位、实块在[−1,1][−1,1],全复块取酉)的“边界集”上搜索,就能达到真正的$\mu$}


好的,我把"III. 下界即最大化"这一段里的核心论证完整拆开讲:为什么
\[
\boxed{\mu_K(M)=\max_{Q\in Q_K}\rho_R(QM)}
\]
也就是——只在 $Q_K$(复块取单位圆相位、实块在 $[-1,1]$,全复块取酉)的"边界集"上搜索,就能达到真正的 $\mu$。

\subsection{起点:把"奇异化的最小范数"化成"单位球上的最大实谱半径"}

混合 $\mu$ 的定义是
\[
\mu_K(M)=\big(\min_{\Delta\in X_K}\{\|\Delta\|:\det(I-\Delta M)=0\}\big)^{-1}.
\]
把 $\Delta=t\widehat{\Delta}$($\|\widehat{\Delta}\|=1$)代入,并用
\[
\det(I-\Delta M)=0
\iff \exists v\neq 0:\ \Delta M v=v
\iff 1\in\sigma(\Delta M),
\]
就得到
\[
\mu_K(M)=\max_{\Delta\in B_{X_K}}\rho_R(\Delta M),
\qquad
B_{X_K}:=\{\Delta\in X_K:\|\Delta\|\le 1\}.
\]
含义:只需要在"结构化单位球"的\textbf{壳}上挑方向,把 $\Delta M$ 的\textbf{实}特征值推到最大即可。

\subsection{关键问题:能否把搜索域从 $B_{X_K}$ 缩到"边界极点" $Q_K$ 而不损失最优?}

我们要证明
\[
\max_{\Delta\in B_{X_K}}\rho_R(\Delta M)=\max_{Q\in Q_K}\rho_R(QM).
\]
难点在于:$B_{X_K}$ 里允许"复标量块的模 $<1$"、"全复块的奇异值 $\Sigma\prec I$"等\textbf{内点};而 $Q_K$ 强制"复标量块在单位圆上、全复块取酉"。为什么把这些块\textbf{顶到边界}不会让最优值变差,反而能达到等号?

\subsection{块极点原则:把"可缩放的块"顶到边界不会变差}

把 $\Delta$ 分块写成
\[
\Delta=\operatorname{diag}(\underbrace{q_1 I,\dots,q_{n_R}I}_{\text{实重复标量}},
\underbrace{\alpha_1e^{j\phi_1}I,\dots,\alpha_{n_C}e^{j\phi_{n_C}}I}_{\text{复重复标量}},
\underbrace{U_1\Sigma_1V_1^*,\dots,U_{n_F}\Sigma_{n_F}V_{n_F}^*}_{\text{全复块}}),
\]
其中 $|q_i|\le 1$、$0\le \alpha_\ell\le 1$、$\Sigma_j=\operatorname{diag}(\sigma_{j,k})\in[0,1]$。

\textbf{要点}:在保持"$(I-\Delta M)$ 可奇异"的同时,顺着每个"可缩放"的复方向把模(或奇异值)\textbf{推到 1},最坏情形不会变差:

\begin{itemize}
    \item 复重复标量块:把 $\alpha_\ell\uparrow 1$,并适配相位 $\phi_\ell$;
    \item 全复块:把 $\Sigma_j\uparrow I$,并把 $U_j,V_j$ 调成"对齐"的等距算子(酉)。
\end{itemize}

这背后是两个标准事实:

\begin{enumerate}
    \item \textbf{线性-谱型最优化在范数球的极点/边界达到极值}(冯–诺依曼迹不等式、KKT 条件、SVD 对齐)。
    \item \textbf{多块同时"上边界"是可行的}:可以沿着"径向放大 + 相位/单位向量对齐"的路径保持(或恢复)$\det(I-\Delta M)=0$。直观地:一旦某个 $\Delta$ 使 $\Delta M$ 有实特征值 $\lambda\in(0,1]$,就能把该块的幅度推到 1,同时按左右特征向量的相位/奇异向量方向做\textbf{配平},把"可达的 $\lambda$"不减反增。极限状态正好就是 $Q\in Q_K$。
\end{enumerate}

于是,对任何 $\Delta\in B_{X_K}$ 都可构造一个 $Q(\Delta)\in Q_K$ 使
\[
\rho_R(Q(\Delta) M)\ \ge\ \rho_R(\Delta M).
\]
对左边再取最大,就得到
\[
\max_{Q\in Q_K}\rho_R(QM)\ \ge\ \max_{\Delta\in B_{X_K}}\rho_R(\Delta M).
\]
而因为 $Q_K\subset B_{X_K}$,反向不等式也成立,故两者相等。

\begin{quote}
直观版:$\mu$ 关心的是"把实特征值推到 1 的\textbf{最小尺度}",而复/全复块自带"幅度自由度"。\textbf{把这些自由度全用满(冲到边界)},再靠相位/酉来\textbf{调整方向},一定不比"留在内部"更差。极限形态就是 $Q\in Q_K$。
\end{quote}

\subsection{为什么实块允许留在 $[-1,1]$ 而不强制到 $\pm 1$?}

对实重复标量块 $q\in[-1,1]$,KKT 条件会出现两类情况:

\begin{itemize}
    \item \textbf{端点}:若方向导数指向外,则最优落在 $q=\pm 1$;
    \item \textbf{内点}:若一阶必要条件 $\Re(a^*w)=0$(由左右特征向量的"对齐条件"导出)成立,则可以停在内点 $q\in(-1,1)$。
\end{itemize}

因此 $Q_K$ 对实块写成 $[-1,1]$(而不是只取 $\pm 1$)正是为了\textbf{覆盖内点极值}的情形。

\subsection{结论与意义}

\begin{itemize}
    \item \textbf{理论结论}:$\displaystyle \mu_K(M)=\max_{Q\in Q_K}\rho_R(QM)$。
    $\mu$ 真正的值等于在"边界集 $Q_K$"上把实特征值推到最大的结果。
    \item \textbf{算法意义}:把"范数球上的非凸最大化"化成"\textbf{块极点集}上的非凸最大化"。这一步让 Young–Doyle 的\textbf{幂迭代}可落地:每一步都只需按块做\textbf{闭式投影}(复块取相位、全复块做等距配平、实块卡端点或内点),于是能快速给出\textbf{紧的下界}。
    \item \textbf{工程用法}:频率逐点算 $\underline{\mu}(\omega)=\max_{Q\in Q_K}\rho_R(QM(j\omega))$,再配合 D-scaling 上界形成夹逼;在 FDI/$\mu$-综合里定位最坏频段并做设计。
\end{itemize}

\subsection{小提示:如何把这段思想"看见"}

\begin{itemize}
    \item 把一个任意 $\Delta$ 的复块奇异值从 $<1$ 连续推到 $=1$,同时把相位/酉对齐到当前的左右特征向量方向;
    \item 观察 $\rho_R(\Delta M)$ 不减(通常变大);
    \item 极限点就是某个 $Q\in Q_K$。
\end{itemize}

这就是"把单位球的内点推到边界极点"背后的几何图像,也是为什么"只搜 $Q_K$"足以达到真正 $\mu$ 的根本原因。














\end{document}