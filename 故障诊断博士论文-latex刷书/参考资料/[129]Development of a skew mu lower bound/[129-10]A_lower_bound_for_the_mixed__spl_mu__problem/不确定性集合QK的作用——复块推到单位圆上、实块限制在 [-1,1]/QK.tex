\documentclass[10pt]{article}
\usepackage{researchnotes}
\usepackage[utf8]{inputenc}
\usepackage[T1]{fontenc}
\usepackage{amsmath}
\usepackage{amsfonts}
\usepackage{amssymb}
\usepackage{array}
\usepackage{ctex}
\usepackage[version=4]{mhchem}
\usepackage{stmaryrd}
\usepackage{hyperref}
\hypersetup{colorlinks=true, linkcolor=blue, filecolor=magenta, urlcolor=cyan,}
\urlstyle{same}
\usepackage{graphicx}
\usepackage[export]{adjustbox}
\graphicspath{ {./images/} }
\usepackage{caption}
\usepackage{mathrsfs}
\usepackage{bbold}
\usepackage{tikz}
\usepackage{listings}
\usetikzlibrary{shapes,arrows,positioning}

\title{稳定性的逻辑}

\author{黄嘉聪}
\date{}



\begin{document}
\maketitle
\captionsetup{singlelinecheck=false}


\begin{abstract}
主要介绍了稳定性的相关知识,重点在于理解:稳定性条件(传统SISO和传统MIMO和鲁棒稳定背后有什么共性?有界输入有界输出稳定和内稳定)为什么鲁棒稳定的条件是det(I-M\delta)
\end{abstract}

KEY WORDS: robust control; $\mu$; skew $\mu$; lower bound

























\end{document}