\documentclass[10pt]{article}
\usepackage{researchnotes}
\usepackage[utf8]{inputenc}
\usepackage[T1]{fontenc}
\usepackage{amsmath}
\usepackage{amsfonts}
\usepackage{amssymb}
\usepackage{array}
\usepackage{ctex}
\usepackage[version=4]{mhchem}
\usepackage{stmaryrd}
\usepackage{hyperref}
\hypersetup{colorlinks=true, linkcolor=blue, filecolor=magenta, urlcolor=cyan,}
\urlstyle{same}
\usepackage{graphicx}
\usepackage[export]{adjustbox}
\graphicspath{ {./images/} }
\usepackage{caption}
\usepackage{mathrsfs}
\usepackage{bbold}
\usepackage{tikz}
\usepackage{listings}
\usetikzlibrary{shapes,arrows,positioning}

\title{稳定性的逻辑}

\author{黄嘉聪}
\date{}



\begin{document}
\maketitle
\captionsetup{singlelinecheck=false}


\begin{abstract}
主要介绍了稳定性的相关知识,重点在于理解:稳定性条件(传统SISO和传统MIMO和鲁棒稳定背后有什么共性?有界输入有界输出稳定和内稳定)为什么鲁棒稳定的条件是det(I-M\delta)
\end{abstract}

KEY WORDS: robust control; $\mu$; skew $\mu$; lower bound

\section*{1. INTRODUCTION}
\begin{researchnote}[author=JC, date=2025-10-23]
    \subsubsection*{几何直觉理解$\mu$的含义}
    $M$是一个标称系统的传递函数复矩阵,$\Delta$是不确定性集合中的其中一个情况。我们现在判断含不确定性的系统矩阵$\Delta M$是否稳定。
    要判断一个系统是否稳定,就要看该系统的矩阵的特征值的实部是否小于0,即矩阵的特征方程特征多项式不存在右半平面零点。
    矩阵$\Delta M$的特征值$\lambda$决定系统是否碰到边界,$\lambda$是复数,如果$\lambda$的模小于1,则该
    \begin{itemize}
        \item 若所有的特征值都在
    \end{itemize}
\end{researchnote}

\begin{researchnote}[author=JC, date=2025-10-23]
  \subsubsection*{两种稳定性}
  \begin{itemize}
    \item BIBO稳定(有界输入有界输出稳定) —— 基于传递函数矩阵本身
      \begin{itemize}
          \item 定义:任何一个有界的输入信号,都只能产生有界的输出信号。
          \item 判据:对于传递函数矩阵$G(s)$,其所有传递函数元的所有极点都必须位于复平面左半开平面,即所有极点都具有负实部。
          \item 这是一个相对“较弱”的稳定性概念,只保证输出不会发散,但不保证系统内部状态不发散。
          \item 如果系统存在零极点对消,这种方法可能会漏掉被消掉的不稳定极点。被消掉的极点虽然在外部传递函数中看不到,但它在系统内部可能依然存在并导致内部不稳定。因此,这种方法严格来说只保证BIBO稳定。
      \end{itemize}
    \item 内部稳定 ———— 基于状态空间实现   更严格 更可靠
    \begin{itemize}
          \item 定义:系统在零输入条件下的状态响应是有界的,并且渐近趋于零。这意味着系统的所有模态(特征值)都是衰减的。
          \item 判据:系统矩阵$A$(状态空间表示中的)的所有特征值都必须位于复平面左半开平面。
          \item 这是一个“更强”的稳定性概念,它保证了系统内部状态和外部输出都是稳定的。对于最小实现系统,BIBO稳定等价于内部稳定。
      \end{itemize}
  \end{itemize}
\end{researchnote}

\begin{researchnote}[author=JC, date=2025-10-23]
  \subsubsection*{判断BIBO稳定的方法————基于传递函数矩阵本身}

\end{researchnote}

\begin{researchnote}[author=JC, date=2025-10-23]
  \subsubsection*{判断内部稳定的方法————基于状态空间实现}

\end{researchnote}






\begin{researchnote}[author=JC, date=2025-10-23]
  \subsubsection*{在鲁棒控制中的稳定性}
  鲁棒控制中,小增益定理是先假设了
    \begin{itemize}
      \item M(s) 是标称闭环系统的传递函数矩阵(稳定且属于$\mathcal{RH}_\infty$)。
      \item Δ(s) 表示不确定性(也属于$\mathcal{RH}_\infty$)
    \end{itemize}
  鲁棒稳定性分析框架(鲁棒控制基础的小增益定理中的图8.8)这个图不是控制框图?这是两个系统反馈互联,
\end{researchnote}









\section*{核心概念:两种稳定性}

\begin{enumerate}
    \item \textbf{BIBO稳定(有界输入有界输出稳定)}
    \begin{itemize}
        \item \textbf{定义}:任何一个有界的输入信号,都只能产生有界的输出信号。
        \item \textbf{判据}:对于传递函数矩阵 \( G(s) \),其所有传递函数元的所有极点都必须位于\textbf{复平面左半开平面},即所有极点都具有\textbf{负实部}。
        \item \textbf{特点}:这是一个相对"较弱"的稳定性概念,只保证输出不会发散,但不保证系统内部状态不发散。
    \end{itemize}

    \item \textbf{内部稳定}
    \begin{itemize}
        \item \textbf{定义}:系统在零输入条件下的状态响应是有界的,并且渐近趋于零。这意味着系统的所有模态(特征值)都是衰减的。
        \item \textbf{判据}:系统矩阵 \( A \)(状态空间表示中的)的所有特征值都必须位于\textbf{复平面左半开平面}。
        \item \textbf{特点}:这是一个"更强"的稳定性概念,它保证了系统内部状态和外部输出都是稳定的。对于最小实现系统,BIBO稳定等价于内部稳定。
    \end{itemize}
\end{enumerate}

\section*{判断方法与步骤}

假设我们有一个传递函数矩阵 \( G(s) \):

\[
G(s) = \begin{bmatrix}
g_{11}(s) & g_{12}(s) & \cdots \\
g_{21}(s) & \ddots &  \\
\vdots &  & 
\end{bmatrix}
\]

\subsection*{方法一:基于传递函数矩阵本身(判断BIBO稳定)}

这种方法适用于快速判断系统的外部稳定性。

\begin{enumerate}
    \item \textbf{步骤一:找到所有传递函数元}
    \begin{itemize}
        \item 将矩阵 \( G(s) \) 中的每一个元素 \( g_{ij}(s) \) 都识别出来。每个 \( g_{ij}(s) \) 都是一个单输入单输出(SISO)的传递函数。
    \end{itemize}

    \item \textbf{步骤二:计算每个传递函数元的极点}
    \begin{itemize}
        \item 将每个 \( g_{ij}(s) \) 化为严格真分式形式(分子阶数小于分母阶数)。
        \item 令每个 \( g_{ij}(s) \) 的分母多项式等于零,解出所有的极点 \( p_k \)(其中 \( k=1,2,\ldots \))。
    \end{itemize}

    \item \textbf{步骤三:检查所有极点的实部}
    \begin{itemize}
        \item 检查步骤二中求出的\textbf{所有极点}。
        \item \textbf{稳定性判据}:
        \begin{itemize}
            \item 如果\textbf{所有}极点 \( p_k \) 都满足 \( \text{Re}(p_k) < 0 \)(即都在左半开平面),则该系统是 \textbf{BIBO稳定} 的。
            \item 只要存在\textbf{一个}极点满足 \( \text{Re}(p_k) \geq 0 \)(即在虚轴上或右半平面),则该系统是 \textbf{BIBO不稳定} 的。
            \begin{itemize}
                \item 在虚轴上有单极点 $\rightarrow$ 临界稳定(输出有界但不衰减)。
                \item 在虚轴上有重极点或在右半平面有极点 $\rightarrow$ 不稳定(输出无界)。
            \end{itemize}
        \end{itemize}
    \end{itemize}
\end{enumerate}

\textbf{注意事项}:
\begin{itemize}
    \item 如果系统存在零极点对消,这种方法可能会漏掉被消掉的不稳定极点。被消掉的极点虽然在外部传递函数中看不到,但它在系统内部可能依然存在并导致内部不稳定。因此,这种方法严格来说只保证BIBO稳定。
\end{itemize}

\subsection*{方法二:基于状态空间实现(判断内部稳定)}

这是更严格、更可靠的方法,因为它揭示了系统的全部动态特性。

\begin{enumerate}
    \item \textbf{步骤一:找到系统的一个状态空间实现}
    \begin{itemize}
        \item 将传递函数矩阵 \( G(s) \) 转化为状态空间形式:
        \[
        \begin{align*}
        \dot{x}(t) &= A x(t) + B u(t) \\
        y(t) &= C x(t) + D u(t)
        \end{align*}
        \]
        \item 这个实现最好是\textbf{最小实现}(即完全能控且完全能观的)。最小实现保证了没有零极点对消,传递函数矩阵包含了系统所有的动态信息。
    \end{itemize}

    \item \textbf{步骤二:计算系统矩阵 \( A \) 的特征值}
    \begin{itemize}
        \item 求解特征方程 \( \det(\lambda I - A) = 0 \),得到系统矩阵 \( A \) 的所有特征值 \( \lambda_k \)。这些特征值也被称为系统的"极点"。
    \end{itemize}

    \item \textbf{步骤三:检查所有特征值的实部}
    \begin{itemize}
        \item \textbf{稳定性判据}:
        \begin{itemize}
            \item 如果系统矩阵 \( A \) 的\textbf{所有特征值} \( \lambda_k \) 都满足 \( \text{Re}(\lambda_k) < 0 \),则该系统是\textbf{内部稳定}的(也是BIBO稳定的)。
            \item 只要存在\textbf{一个}特征值满足 \( \text{Re}(\lambda_k) \geq 0 \),则该系统是\textbf{内部不稳定}的。
        \end{itemize}
    \end{itemize}
\end{enumerate}

\textbf{为什么方法二更受青睐?}
因为它直接检查系统的内部动态。即使传递函数矩阵因为零极点对消而"看起来"稳定,状态空间方法也能通过检查 \( A \) 矩阵发现那些被隐藏的不稳定模式。

\section*{总结与对比}

\begin{center}
\begin{tabular}{|p{0.25\textwidth}|p{0.3\textwidth}|p{0.3\textwidth}|}
\hline
\textbf{特性} & \textbf{方法一:基于传递函数(BIBO稳定)} & \textbf{方法二:基于状态空间(内部稳定)} \\
\hline
\textbf{检查对象} & 传递函数矩阵 \( G(s) \) 的每一个元素的所有极点 & 系统矩阵 \( A \) 的所有特征值 \\
\hline
\textbf{稳定性类型} & 输入-输出稳定性(BIBO稳定) & 内部稳定性 \\
\hline
\textbf{可靠性} & 较低。如果存在\textbf{不稳定零极点对消},会得到错误结论。 & 非常高。揭示了系统的全部动态。 \\
\hline
\textbf{复杂度} & 相对简单,直接计算极点。 & 需要获得状态空间实现,并计算矩阵特征值。 \\
\hline
\textbf{适用场景} & 快速评估、初步设计、不存在零极点对消的简单系统。 & 严格分析、控制器综合、可靠性要求高的复杂系统。 \\
\hline
\end{tabular}
\end{center}

\textbf{黄金法则}:
对于一个控制系统,我们通常追求的是\textbf{内部稳定}。因此,\textbf{最推荐的方法是方法二}:将传递函数矩阵转化为\textbf{最小实现},然后检查其系统矩阵 \( A \) 的特征值是否全部具有负实部。这是判断线性系统稳定性的最根本、最可靠的手段。








\section{反馈互联}
根据您提供的定理描述和问题,我注意到您使用了"串联连接"这个术语。在控制理论中,特别是小增益定理(Theorem 8.1)的上下文中,\textbf{M 和 Δ 并不是串联连接,而是反馈互联}。这是一个关键区别,让我详细解释一下。

\subsection*{1. 正确的连接方式:反馈互联}
\begin{itemize}
    \item 在标准的鲁棒稳定性分析中,如图 8.8 所示的"M-Δ loop"是一个\textbf{反馈互联系统}。其典型结构如下:
    \begin{itemize}
        \item M(s) 是标称闭环系统的传递函数矩阵(稳定且属于 $\mathcal{RH}_\infty$)。
        \item Δ(s) 表示不确定性(也属于 $\mathcal{RH}_\infty$)。
        \item 输出信号从 M(s) 反馈到 Δ(s) 的输入,再从 Δ(s) 的输出反馈回 M(s) 的输入,形成一个闭环。
    \end{itemize}
    \item 这种连接方式通常用框图表示,其中 M 和 Δ 构成一个反馈环,而不是简单的串联。
\end{itemize}

\subsection*{2. 为什么是反馈互联而不是串联?}
\begin{itemize}
    \item \textbf{串联连接}(Series Connection)指的是两个系统一个接一个连接,没有反馈路径。例如,如果信号先经过 M 再经过 Δ,那么这是串联。
    \item 但在小增益定理中,系统是\textbf{反馈互联}(Feedback Interconnection),因为不确定性 Δ 被建模为与标称系统 M 形成闭环。这种结构允许我们分析不确定性对系统稳定性的影响。
    \item 具体来说,反馈互联的方程通常为:
    \[
    y = M(u) \quad \text{和} \quad u = \Delta(y) + w
    \]
    其中 w 是外部输入,这导致了闭环特征方程 $\det(I - M\Delta) = 0$ 或 $\det(I - \Delta M) = 0$。
\end{itemize}

\subsection*{3. 为什么标称系统 M 和不确定性 Δ 能这样连接?}
\begin{itemize}
    \item 这种反馈互联是鲁棒控制中的标准设置,用于捕捉不确定性对系统的影响。M(s) 通常是从标称系统推导出的传递矩阵,它包含了控制器和标称模型的交互。
    \item Δ(s) 表示模型不确定性,例如参数变化、未建模动态等。通过将 Δ 放在反馈路径中,我们可以模拟不确定性对系统的"干扰"或"反馈"效应。
    \item \textbf{小增益定理}的核心思想是:如果 M 和 Δ 都是稳定的(即属于 $\mathcal{RH}_\infty$),且它们的 H∞ 范数乘积小于 1(即 $\|M\|_\infty < \gamma$ 且 $\|\Delta\|_\infty \leq 1/\gamma$),那么反馈互联系统仍然是内部稳定的。这是因为小增益条件保证了闭环信号不会放大,从而防止不稳定。
\end{itemize}

\subsection*{4. 澄清"串联"的误解}
\begin{itemize}
    \item 如果您从图像中看到 M 和 Δ 直接连接,可能是因为框图简化了反馈路径。在典型图中,M 和 Δ 通常被画成两个方块,带有箭头形成环状,这就是反馈互联。
    \item 如果误解为串联,可能会错误地分析稳定性。例如,串联系统的稳定性只需要每个组件稳定即可,但反馈互联需要更严格的条件(如小增益定理)。
\end{itemize}

\subsection*{5. 实际应用中的重要性}
\begin{itemize}
    \item 这种反馈互联结构在μ分析、H∞控制中广泛使用。它允许我们处理结构化或非结构化的不确定性。
    \item 例如,在飞行器控制中,M 表示标称控制器和飞机模型,Δ 表示气动参数的不确定性,通过反馈互联可以设计鲁棒控制器。
\end{itemize}

总结来说,在小增益定理中,M 和 Δ 是反馈互联的,而不是串联连接。这种连接方式能有效分析鲁棒稳定性,因为不确定性被建模为反馈扰动,从而可以通过特征方程 $\det(I - M\Delta) = 0$ 来检验稳定性。如果您有更多细节或图像,我可以进一步解释具体连接方式。





\section{鲁棒控制中的反馈互联和经典控制理论中的控制框图反馈连接有什么区别?}


这是一个非常深刻且重要的问题。鲁棒控制中的反馈互联和经典控制理论中的控制框图反馈连接,在\textbf{基本框图形态上是相似的},但它们在\textbf{建模思想、分析目标、和处理不确定性方式上存在根本性的区别}。

我们可以从以下几个维度来理解它们的区别:

\subsection{1. 核心目的与哲学思想}

\begin{center}
\begin{tabular}{|p{0.4\textwidth}|p{0.25\textwidth}|p{0.25\textwidth}|}
\hline
\textbf{维度} & \textbf{经典控制理论反馈连接} & \textbf{鲁棒控制反馈互联} \\
\hline
\textbf{核心目的} & \textbf{性能设计}。通过反馈来使一个\textbf{已知的、固定的}被控对象达到期望的性能指标,如稳定性、响应速度、稳态误差、抗干扰性。 & \textbf{稳健性分析}。分析和保证一个\textbf{已经设计好的}闭环系统,在面临\textbf{未知的、变化的}模型不确定性时,依然能保持稳定和满足性能。 \\
\hline
\textbf{哲学思想} & \textbf{``塑造''动态}。认为被控对象是确定的,通过设计控制器来``塑造''闭环系统的动态特性。 & \textbf{``免疫''扰动}。承认模型是不精确的,将不确定性视为对标称系统的``攻击'',系统需要具备对这种攻击的``免疫力''。 \\
\hline
\end{tabular}
\end{center}

\textbf{简单比喻:}
\begin{itemize}
    \item \textbf{经典控制}:像一个工程师在精心调试一台机器,通过调整旋钮(控制器参数)让机器运行得又快又准。
    \item \textbf{鲁棒控制}:像在测试这台机器的坚固程度,问:``如果机器的某个零件老化(参数变化),或者受到风吹(外部扰动),它会不会散架?''
\end{itemize}

\subsection{2. 系统组件的角色与含义}

\begin{center}
\begin{tabular}{|p{0.4\textwidth}|p{0.25\textwidth}|p{0.25\textwidth}|}
\hline
\textbf{维度} & \textbf{经典控制理论反馈连接} & \textbf{鲁棒控制反馈互联} \\
\hline
\textbf{前向通道} & 通常由\textbf{控制器 (C)} 和\textbf{被控对象 (P)} 串联组成。即 \( G = C \cdot P \) 或类似形式。 & \textbf{标称闭环系统 (M)}。它是一个\textbf{已经包含了控制器和标称被控对象}的、稳定的传递函数矩阵。它代表了从不确定性注入点到取出点的闭环动态。 \\
\hline
\textbf{反馈通道} & \textbf{传感器或反馈增益}。通常是-1(负反馈)或一个测量环节,目的是将输出信号引回与输入比较。 & \textbf{不确定性块 ($\Delta$)}。它代表了系统的模型不确定性,可以是参数变化、未建模动态等。它是对\textbf{标称模型不精确性}的封装。 \\
\hline
\end{tabular}
\end{center}

\textbf{关键区别}:
在经典控制框图中,反馈路径是系统\textbf{固有的一部分},是\textbf{为了实现控制目的而故意引入的}。
在鲁棒控制的M-$\Delta$结构中,反馈路径是\textbf{为了描述和分析问题而抽象出来的},它代表了我们对系统认知的\textbf{缺陷和不足}。

\subsection{3. 对不确定性的处理方式}

\begin{center}
\begin{tabular}{|p{0.4\textwidth}|p{0.25\textwidth}|p{0.25\textwidth}|}
\hline
\textbf{维度} & \textbf{经典控制理论反馈连接} & \textbf{鲁棒控制反馈互联} \\
\hline
\textbf{不确定性} & 通常是\textbf{隐式的、后验的}。通过\textbf{稳定裕度}(相位裕度、幅值裕度)来\textbf{间接地、粗略地}反映系统对不确定性的容忍度。 & \textbf{显式的、先验的}。不确定性 $\Delta$ 被\textbf{明确地建模}出来,并作为系统的一个组成部分。分析是在\textbf{一族系统}上进行的。 \\
\hline
\textbf{分析目标} & 对于\textbf{一个}确定的系统,判断它是否稳定,性能是否达标。 & 对于\textbf{所有}满足 \( \|\Delta\| \le 1 \) 的不确定性,判断整个互联系统是否\textbf{都}稳定。 \\
\hline
\end{tabular}
\end{center}

\subsection{4. 数学工具与分析范式}

\begin{center}
\begin{tabular}{|p{0.4\textwidth}|p{0.25\textwidth}|p{0.25\textwidth}|}
\hline
\textbf{维度} & \textbf{经典控制理论反馈连接} & \textbf{鲁棒控制反馈互联} \\
\hline
\textbf{数学工具} & \textbf{频域分析}:奈奎斯特曲线、伯德图、根轨迹。\newline
\textbf{特征方程}:\( 1 + L(s) = 0 \),其中 \( L(s) \) 是开环传递函数。 & \textbf{泛函分析、线性代数}:H∞范数、结构奇异值 ($\mu$)。\newline
\textbf{特征方程}:\( \det(I - M\Delta) = 0 \)。 \\
\hline
\textbf{分析范式} & \textbf{单点测试}。在复平面上画出一条曲线(如奈奎斯特曲线),看它是否包围(-1, j0)点。 & \textbf{集合测试}。检查标称系统 M 的 H∞ 范数是否小于1(小增益定理),或者计算 $\mu(M)$ 是否小于1。这保证了\textbf{整个不确定性集合}都不会引起失稳。 \\
\hline
\end{tabular}
\end{center}

\subsection{一个直观的例子对比}

假设我们要控制一个质量-弹簧-阻尼系统。

\begin{itemize}
    \item \textbf{经典控制视角}:
    \begin{itemize}
        \item \textbf{框图}:`[R] $\rightarrow$ [Comparator] $\rightarrow$ [Controller C(s)] $\rightarrow$ [Plant P(s)] $\rightarrow$ [Y]` $\rightarrow$ (反馈到Comparator)
        \item \textbf{问题}:设计 C(s)(例如一个PID控制器),使得输出 Y 能快速、平稳地跟踪参考输入 R。
        \item \textbf{分析}:画出 \( C(s)P(s) \) 的奈奎斯特图,检查相位裕度是否大于45度。如果够了,就认为设计是``鲁棒''的。
    \end{itemize}

    \item \textbf{鲁棒控制视角}:
    \begin{itemize}
        \item \textbf{前提}:我们已经有了一个设计好的控制器 C(s)。但我们知道弹簧刚度 k 和阻尼系数 c 可能在一定范围内变化。
        \item \textbf{重构框图}:我们将不确定性(k, c的变化)提取出来,封装成一个块 $\Delta$。然后通过框图变换,将整个系统(包含 C(s) 和标称的 P(s))重写为一个 M-$\Delta$ 的\textbf{上三角反馈互联}形式。
        \item \textbf{问题}:对于所有可能的 $\Delta$(满足 \( \|\Delta\|_{\infty} \le 1 \)),这个 M-$\Delta$ 互联系统是否都稳定?
        \item \textbf{分析}:计算 M(s) 的 H∞ 范数。如果 \( \|M(s)\|_{\infty} < 1 \),那么根据小增益定理,系统对于所有这类不确定性都是鲁棒稳定的。
    \end{itemize}
\end{itemize}

\subsection{总结}

\begin{center}
\begin{tabular}{|p{0.25\textwidth}|p{0.35\textwidth}|p{0.35\textwidth}|}
\hline
\textbf{特性} & \textbf{经典控制反馈连接} & \textbf{鲁棒控制反馈互联} \\
\hline
\textbf{核心} & \textbf{设计系统} & \textbf{分析系统的稳健性} \\
\hline
\textbf{对象模型} & 确定的、单一的 & 带不确定性的、一个集合 \\
\hline
\textbf{反馈路径角色} & 实现控制目标的必要手段 & 描述模型不确定性的抽象工具 \\
\hline
\textbf{不确定性处理} & 隐式、间接(稳定裕度) & 显式、直接(不确定性块 $\Delta$) \\
\hline
\textbf{数学基础} & 复变函数、频域响应 & 泛函分析、矩阵理论 \\
\hline
\end{tabular}
\end{center}

简而言之,经典控制的反馈是 \textbf{``设计的舞台''} ,而鲁棒控制的反馈互联是 \textbf{``分析的战场''} 。前者关乎如何创造一个好的系统,后者关乎如何证明这个系统在非理想条件下依然可靠。鲁棒控制可以看作是经典控制思想在面对模型不确定性时的一种深化和精化。


\section{如何将带有不确定性的经典控制系统转换为标准的M-$\Delta$分析框架}

这是一个非常核心的鲁棒控制概念!您描述的是\textbf{如何将带有不确定性的经典控制系统转换为标准的M-$\Delta$分析框架}。让我详细解释这个转换过程。

\subsection{原系统:经典反馈控制系统}

首先,我们有一个标准的反馈控制系统:
\begin{itemize}
    \item 控制器:C(s)
    \item 被控对象:P(s, $\theta$),其中$\theta$代表不确定参数(如弹簧刚度k、阻尼系数c)
    \item 参考输入:r
    \item 输出:y
\end{itemize}

标准框图:
\begin{verbatim}
r --> [+] --> C(s) --> P(s,θ) --> y
       ^                      |
       |                      |
       +----------------------+
\end{verbatim}

\subsection{步骤1:分离标称模型和不确定性}

我们将不确定的被控对象分解为:
\[ \mathbf{P(s,\theta) = P_0(s) + \Delta P(s)} \]

其中:
\begin{itemize}
    \item \textbf{P₀(s)}:标称模型(在标称参数值下的模型)
    \item \textbf{$\Delta$P(s)}:由于参数变化引起的模型误差
\end{itemize}

\subsection{步骤2:将不确定性提取为独立模块}

现在我们重新绘制框图,将$\Delta$P明确表示为独立的模块:

\begin{verbatim}
r --> [+] --> C(s) --> P₀(s) --> [+] --> y
       ^                      |         ^
       |                      |         |
       |                      +-- ΔP <--+
       |                              |
       +------------------------------+
\end{verbatim}

\subsection{步骤3:框图变换得到M-$\Delta$结构}

这是最关键的一步!通过框图代数变换,我们将系统重写为\textbf{上三角反馈互联形式}:

\subsubsection{变换后的M-$\Delta$结构:}
\begin{verbatim}
      +-----+       +-----+
 w -->|  M  |------>|  z  |
      |     |       |     |
      +-----+       +-----+
         |             |
         |             |
         |    +-----+  |
         +----|  Δ  |<--
              +-----+
\end{verbatim}

其中:
\begin{itemize}
    \item \textbf{$\Delta$}:不确定性块(封装了k, c的变化)
    \item \textbf{M}:标称闭环系统(包含C(s)和P₀(s))
    \item \textbf{w}:外部输入到不确定性块的信号
    \item \textbf{z}:不确定性块输出到系统的信号
\end{itemize}

\subsection{步骤4:数学推导}

\subsubsection{原始系统方程:}
\begin{enumerate}
    \item \( u = C(s)(r - y) \)
    \item \( y = P(s,\theta)u = [P_0(s) + \Delta P(s)]u \)
\end{enumerate}

\subsubsection{定义新变量:}
\begin{itemize}
    \item 令 \( z = u \)(控制器的输出)
    \item 令 \( w = \Delta P(s) \cdot z \)(不确定性的输出)
\end{itemize}

\subsubsection{重写方程:}
\begin{enumerate}
    \item \( z = C(s)(r - y) \)
    \item \( y = P_0(s)z + w \)
    \item \( w = \Delta P(s) \cdot z \)
\end{enumerate}

通过代数消去中间变量,我们可以得到:
\begin{itemize}
    \item 从w到z的传递函数就是\textbf{M(s)}
    \item 从z到w的传递函数就是\textbf{$\Delta$(s)}
\end{itemize}

\subsection{步骤5:得到最终的M-$\Delta$形式}

最终我们得到标准的反馈互联:
\begin{verbatim}
      +-----+
      |     |
 w -->|  M  |--> z
      |     |
      +-----+
        |   |
        |   |
        |   |
        +-->|  Δ  |--> w
            +-----+
\end{verbatim}

其中:
\begin{itemize}
    \item \textbf{M(s)} = 从w到z的传递函数 = \( -C(s)(I + P_0(s)C(s))^{-1} \)
    \item \textbf{$\Delta$(s)} = 从z到w的传递函数 = $\Delta$P(s)
\end{itemize}

\subsection{为什么这个转换很重要?}

\begin{enumerate}
    \item \textbf{统一分析框架}:将各种不确定性(参数变化、未建模动态等)统一表示为$\Delta$块
    \item \textbf{应用鲁棒理论}:现在可以应用小增益定理:如果$\|M\|_\infty \cdot \|\Delta\|_\infty < 1$,则系统鲁棒稳定
    \item \textbf{结构化分析}:对于多个不确定参数,可以构建块对角结构的$\Delta$
    \item \textbf{$\mu$分析}:可以计算结构奇异值$\mu$来获得更精确的鲁棒稳定性条件
\end{enumerate}

\subsection{实际示例}

假设质量-弹簧-阻尼系统:
\begin{itemize}
    \item 标称参数:m=1, k₀=2, c₀=0.5
    \item 不确定参数:k ∈ [1.8, 2.2], c ∈ [0.4, 0.6]
\end{itemize}

通过这种转换:
\begin{itemize}
    \item P₀(s) = 1/(s² + 0.5s + 2)
    \item $\Delta$P(s) 封装了k和c的变化
    \item M(s) 包含了控制器C(s)和P₀(s)
    \item 现在可以分析系统对所有允许的(k,c)组合是否保持稳定
\end{itemize}

这种\textbf{框图重构}是鲁棒控制理论的基石,它让我们能够系统地分析和处理系统的不确定性!



\section{框图重构正是线性分式变换LFT的具体实现}

这个框图重构\textbf{正是线性分式变换(LFT)的具体实现}!您问到了关键点上。让我详细解释这个重要关系。

\subsection{线性分式变换(LFT)的定义}

线性分式变换是一种描述两个系统通过反馈互联的通用数学框架。它有两种形式:

\subsubsection{1. 下线性分式变换(Lower LFT)}
\begin{verbatim}
      +-----+
      |     |
u1 -->|  P  |--> y1
      |     |
      +-----+
        |  |
        |  |
        v  v
      +-----+
      |     |
u2 <--|  K  |--> y2
      +-----+
\end{verbatim}
数学表示:\textbf{Fₗ(P, K)}

\subsubsection{2. 上线性分式变换(Upper LFT)}
\begin{verbatim}
      +-----+
      |     |
u1 -->|  P  |--> y1
      |     |
      +-----+
        |  |
        |  |
        v  v
      +-----+
      |     |
y2 <--|  Δ  |--> u2
      +-----+
\end{verbatim}
数学表示:\textbf{Fᵤ(P, $\Delta$)}

\subsection{框图重构就是构建上LFT的过程}

您之前描述的框图重构,\textbf{正是将系统转换为上线性分式变换 Fᵤ(M, $\Delta$) 的标准过程}!

\subsubsection{重构步骤与LFT的对应关系}

\textbf{原始系统}:
\begin{verbatim}
r --> [+] --> C(s) --> P(s,θ) --> y
       ^                      |
       |                      |
       +----------------------+
\end{verbatim}

\textbf{重构为上LFT形式}:
\begin{verbatim}
      +-----+
      |     |
 w -->|  M  |--> z
      |     |
      +-----+
        |  |
        |  |
        v  v
      +-----+
      |     |
z  -->|  Δ  |--> w
      +-----+
\end{verbatim}

\subsubsection{数学上的LFT表示}

在数学上,这个系统可以表示为:

\textbf{Fᵤ(M, $\Delta$) = M₂₂ + M₂₁$\Delta$(I - M₁₁$\Delta$)⁻¹M₁₂}

其中M被分块为:
\[
M = \begin{bmatrix}
M_{11} & M_{12} \\
M_{21} & M_{22}
\end{bmatrix}
\]

\subsection{具体实现:如何构建LFT}

\subsubsection{步骤1:定义广义植物P}

首先,我们需要构建一个广义植物,包含所有信号:
\begin{itemize}
    \item 外部输入:参考信号r、扰动等
    \item 控制输入:u
    \item 测量输出:y
    \item 性能输出:需要优化的信号
\end{itemize}

\subsubsection{步骤2:提取不确定性通道}

在广义植物中,我们明确标识不确定性通道:
\begin{itemize}
    \item \textbf{z}:不确定性块的输入(从系统到不确定性)
    \item \textbf{w}:不确定性块的输出(从不确定性到系统)
\end{itemize}

\subsubsection{步骤3:构建分块矩阵M}

通过框图代数,我们将系统重写为:
\[
\begin{bmatrix}
z \\ y
\end{bmatrix}
= M \begin{bmatrix}
w \\ u
\end{bmatrix}
=
\begin{bmatrix}
M_{11} & M_{12} \\
M_{21} & M_{22}
\end{bmatrix}
\begin{bmatrix}
w \\ u
\end{bmatrix}
\]

\subsubsection{步骤4:闭合不确定性环路}

应用上LFT公式:
\textbf{y = Fᵤ(M, $\Delta$)u = [M₂₂ + M₂₁$\Delta$(I - M₁₁$\Delta$)⁻¹M₁₂]u}

\subsection{实际示例:质量-弹簧-阻尼系统}

让我们用具体例子说明:

\subsubsection{系统方程:}
\[
m\ddot{x} + c\dot{x} + kx = u
\]
\[
y = x
\]

\subsubsection{参数不确定性:}
\begin{itemize}
    \item 标称值:m₀, c₀, k₀
    \item 实际值:m = m₀(1 + δₘ), c = c₀(1 + δ꜀), k = k₀(1 + δₖ)
\end{itemize}

\subsubsection{构建LFT过程:}

\begin{enumerate}
    \item \textbf{改写动力学方程}:
    \[
    m₀\ddot{x} + c₀\dot{x} + k₀x = u - (δₘm₀\ddot{x} + δ꜀c₀\dot{x} + δₖk₀x)
    \]
    
    \item \textbf{定义不确定性输出}:
    \[
    w = δₘm₀\ddot{x} + δ꜀c₀\dot{x} + δₖk₀x
    \]
    
    \item \textbf{定义不确定性输入}:
    \[
    z = \begin{bmatrix} \ddot{x} \\ \dot{x} \\ x \end{bmatrix}
    \]
    
    \item \textbf{构建分块矩阵M}:
    通过拉普拉斯变换,我们可以得到:
    \[
    \begin{bmatrix}
    z \\ y
    \end{bmatrix}
    = M \begin{bmatrix}
    w \\ u
    \end{bmatrix}
    \]
    其中:
    \[
    M = \begin{bmatrix}
    M_{11} & M_{12} \\
    M_{21} & M_{22}
    \end{bmatrix}
    \]
    
    \item \textbf{不确定性矩阵$\Delta$}:
    \[
    \Delta = \begin{bmatrix}
    δₘ & 0 & 0 \\
    0 & δ꜀ & 0 \\
    0 & 0 & δₖ
    \end{bmatrix}
    \]
\end{enumerate}

\subsection{LFT在鲁棒控制中的优势}

\subsubsection{1. 统一框架}
\begin{itemize}
    \item 将所有类型的不确定性(参数变化、未建模动态、延迟等)统一表示为$\Delta$块
    \item 提供标准化的分析方法
\end{itemize}

\subsubsection{2. 结构化分析}
对于多个不确定参数,$\Delta$具有块对角结构:
\[
\Delta = \text{diag}(δ₁I_{r₁}, δ₂I_{r₂}, \ldots, \Delta₁, \Delta₂, \ldots)
\]

\subsubsection{3. 便于应用鲁棒理论}
\begin{itemize}
    \item \textbf{小增益定理}:如果$\|M_{11}\|_\infty \cdot \|\Delta\|_\infty < 1$,则系统鲁棒稳定
    \item \textbf{$\mu$分析}:计算结构奇异值$\mu(M_{11})$来获得精确的鲁棒稳定性边界
\end{itemize}

\subsubsection{4. 控制器综合}
在H∞控制中,控制器的设计问题可以表述为:
\textbf{找到控制器K,使得$\|Fₗ(P, K)\|_\infty$最小}

\subsection{总结}

\textbf{框图重构就是构建线性分式变换的具体过程}。通过这种转换:

\begin{enumerate}
    \item 我们将复杂的含不确定性系统转换为标准的LFT形式
    \item 不确定性被隔离在$\Delta$块中
    \item 标称动态被封装在M矩阵中
    \item 使得我们可以系统性地应用鲁棒控制理论
\end{enumerate}

这种LFT框架是现代鲁棒控制的理论基石,它使得H∞控制、$\mu$分析等高级方法成为可能。您之前描述的框图重构过程,本质上就是在实践线性分式变换的理论!

\section{为什么稳定性由det(I - MΔ) = 0决定}
这是一个非常深刻的问题,触及了鲁棒控制理论的核心。让我们一步步深入分析。
\hl{重点在于这个鲁棒控制框图结构的输入输出是循环的,一圈一圈的,那你只要一圈大于1,就会无限变大下去。而之前经典控制理论是对系统的输入和输出,只有一圈,反馈回路返回的是k(r-y)}

\subsection{为什么稳定性由 det(I - M$\Delta$) = 0 决定?}

\subsubsection{从信号流角度理解}

考虑M-$\Delta$反馈系统:
\begin{verbatim}
      +-----+
      |     |
 w -->|  M  |--> z
      |     |
      +-----+
        |  |
        |  |
        v  v
      +-----+
      |     |
z  -->|  Δ  |--> w
      +-----+
\end{verbatim}

系统方程:
\begin{itemize}
    \item z = M · w
    \item w = $\Delta$ · z
\end{itemize}

代入得到:\textbf{z = M$\Delta$z} 或 \textbf{(I - M$\Delta$)z = 0}

\textbf{det(I - M$\Delta$) = 0} 意味着:
\begin{itemize}
    \item 矩阵 (I - M$\Delta$) 奇异
    \item 存在非零向量 z 使得 (I - M$\Delta$)z = 0
    \item 系统有非零解而不需要外部激励 $\rightarrow$ \textbf{自激振荡}
    \item 这对应于闭环系统有特征值在稳定边界上
\end{itemize}

\subsubsection{与经典控制理论的类比}

在经典SISO控制中,闭环特征方程为:
\textbf{1 + L(s) = 0},其中 L(s) 是开环传递函数

在MIMO系统中,这推广为:
\textbf{det(I + L(s)) = 0}

在M-$\Delta$框架中,环路增益是 M$\Delta$,因此特征方程为:
\textbf{det(I - M$\Delta$) = 0}

\subsection{M和$\Delta$相乘的物理意义}

\subsubsection{M$\Delta$ 不是简单的串联!}

\textbf{物理意义:环路增益矩阵}

M$\Delta$ 表示信号在闭环中循环一次的总体效果:
\begin{itemize}
    \item 信号从M的输入开始
    \item 经过M的动力学
    \item 进入$\Delta$的不确定性
    \item 再反馈回M的输入
\end{itemize}

\textbf{M$\Delta$ 捕获了反馈环路的总增益}

\subsubsection{具体物理解释}

考虑质量-弹簧-阻尼系统的例子:
\begin{itemize}
    \item \textbf{M}:包含控制器和标称物理模型的闭环动力学
    \item \textbf{$\Delta$}:参数不确定性(k, c的变化)
    \item \textbf{M$\Delta$}:表示"不确定性如何通过系统动力学被放大或衰减"
\end{itemize}

如果 $\|$M$\Delta$$\|$ > 1,意味着:
\begin{itemize}
    \item 不确定性在环路循环中被放大
    \item 系统可能变得不稳定
\end{itemize}

\subsection{M$\Delta$ 与系统矩阵 A 的关系}

\subsubsection{关键区别:}
\begin{itemize}
    \item \textbf{系统矩阵 A}:描述\textbf{状态}的动态演化(内部稳定性)
    \item \textbf{M$\Delta$}:描述\textbf{输入-输出}关系(外部信号增益)
\end{itemize}

\subsubsection{但它们通过状态空间实现相关联}

假设 M 和 $\Delta$ 都有状态空间实现:

M: \( (A_M, B_M, C_M, D_M) \)
$\Delta$: \( (A_\Delta, B_\Delta, C_\Delta, D_\Delta) \)

那么整个闭环系统的\textbf{真实系统矩阵}是:

\[
A_{cl} = \begin{bmatrix}
A_M + B_M D_\Delta (I - D_M D_\Delta)^{-1} C_M & B_M (I - D_\Delta D_M)^{-1} C_\Delta \\
B_\Delta (I - D_M D_\Delta)^{-1} C_M & A_\Delta + B_\Delta D_M (I - D_\Delta D_M)^{-1} C_\Delta
\end{bmatrix}
\]

\textbf{det(I - M$\Delta$) = 0} 实际上等价于 \textbf{det(sI - A\_cl) 在 s=0 时有根}(在频域分析中,这对应于虚轴上的极点)。

\subsection{严格的数学推导}

\subsubsection{从状态空间到特征方程}

整个闭环系统的状态方程:
\[
\begin{aligned}
\dot{x}_M &= A_M x_M + B_M w \\
\dot{x}_\Delta &= A_\Delta x_\Delta + B_\Delta z \\
z &= C_M x_M + D_M w \\
w &= C_\Delta x_\Delta + D_\Delta z
\end{aligned}
\]

消去代数变量 z, w,得到纯微分方程:
\[
\begin{bmatrix}
\dot{x}_M \\
\dot{x}_\Delta
\end{bmatrix}
= A_{cl}
\begin{bmatrix}
x_M \\
x_\Delta
\end{bmatrix}
\]

特征方程:\textbf{det(sI - A\_cl) = 0}

通过框图代数,可以证明这个特征方程等价于:
\textbf{det(I - M(s)$\Delta$(s)) = 0}

\subsubsection{频域解释}

在频域中,闭环传递函数为:
\[
T(s) = F_u(M(s), \Delta(s)) = M_{22} + M_{21}\Delta(I - M_{11}\Delta)^{-1}M_{12}
\]

闭环极点出现在 \textbf{det(I - M\_{11}(s)$\Delta$(s)) = 0} 时。

\subsection{直观的物理解释}

\subsubsection{"不确定性放大"观点}

把系统想象成一个音频反馈系统:
\begin{itemize}
    \item \textbf{M}:音响系统和房间的声学特性
    \item \textbf{$\Delta$}:麦克风的位置和增益
    \item \textbf{M$\Delta$}:从扬声器到麦克风再回来的总增益
\end{itemize}

当 \textbf{det(I - M$\Delta$) = 0} 时:
\begin{itemize}
    \item 相当于环路增益为1
    \item 任何微小噪声都会被无限放大
    \item 系统产生自激振荡(啸叫)
\end{itemize}

\subsubsection{"虚拟短路"观点}

det(I - M$\Delta$) = 0 意味着存在信号模式可以"免费"地在环路中循环,不需要外部能量输入——这正是振荡的数学定义。

\subsection{在鲁棒稳定性分析中的应用}

小增益定理的核心思想:
\begin{itemize}
    \item 如果 $\|$M$\|_\infty$ · $\|$$\Delta$$\|_\infty$ < 1
    \item 那么对于所有 s = j$\omega$,有 $\|$M(j$\omega$)$\Delta$(j$\omega$)$\|$ < 1
    \item 这意味着 det(I - M(j$\omega$)$\Delta$(j$\omega$)) $\neq$ 0 对所有 $\omega$ 成立
    \item 因此系统没有虚轴极点,保持稳定
\end{itemize}

\subsection{总结}

\begin{enumerate}
    \item \textbf{det(I - M$\Delta$) = 0} 是闭环系统的\textbf{特征方程在频域的表示}
    \item \textbf{M$\Delta$} 是\textbf{环路增益矩阵},描述信号循环一次的总效果
    \item \textbf{M$\Delta$ $\neq$ 系统矩阵A},但通过状态空间实现相关联
    \item 这个条件能判断稳定性是因为它直接对应闭环极点的位置
    \item 物理上,它表示系统存在能自我维持的振荡模式
\end{enumerate}

这种分析方法的美妙之处在于:\textbf{它将复杂的内部稳定性问题转化为可计算的外部增益检验},使得我们能够系统性地分析不确定性对稳定性的影响。





\section{经典SISO控制系统中的闭环特征方程}

好的,这是一个非常基础且核心的问题。我们来详细推导经典SISO(单输入单输出)控制系统中的闭环特征方程。

\subsection{基本的闭环控制系统结构}

考虑一个典型的负反馈系统框图:

\begin{verbatim}
R(s) --> [+] --> [控制器 C(s)] --> [被控对象 P(s)] -------+---> Y(s)
         ^                                          |
         |                                          |
         +-----------------[反馈通道]-----------------+
\end{verbatim}

其中:
\begin{itemize}
    \item \( R(s) \):参考输入(拉普拉斯变换后)
    \item \( Y(s) \):系统输出
    \item \( C(s) \):控制器传递函数
    \item \( P(s) \):被控对象传递函数
    \item 反馈通道通常为单位反馈(增益为1),如果不是单位反馈,则还会有一个 \( H(s) \) 模块。
\end{itemize}

\subsection{推导过程(代数法)}

我们从框图中直接写出各个信号的关系:

\subsubsection{前向通道的输入}
这是比较点(求和点)的输出,即误差信号 \( E(s) \)。
\[
E(s) = R(s) - Y(s) \quad \text{(对于单位负反馈)}
\]

\subsubsection{前向通道的输出}
前向通道的输入 \( E(s) \) 经过控制器 \( C(s) \) 和被控对象 \( P(s) \)。
\[
Y(s) = [C(s)P(s)] \cdot E(s)
\]
我们定义 \textbf{开环传递函数} \( L(s) \) 为前向通道所有环节的乘积:
\[
L(s) = C(s)P(s)
\]
因此上式可写为:
\[
Y(s) = L(s) \cdot E(s)
\]

\subsubsection{建立方程}
将方程 \( E(s) = R(s) - Y(s) \) 代入上式。
\[
Y(s) = L(s) [R(s) - Y(s)]
\]

\subsubsection{求解闭环传递函数}
将包含 \( Y(s) \) 的项移到一边。
\[
Y(s) = L(s)R(s) - L(s)Y(s)
\]
\[
Y(s) + L(s)Y(s) = L(s)R(s)
\]
\[
Y(s) [1 + L(s)] = L(s)R(s)
\]
最终得到\textbf{闭环传递函数} \( T(s) \):
\[
T(s) = \frac{Y(s)}{R(s)} = \frac{L(s)}{1 + L(s)}
\]

\subsection{特征方程的出现}

闭环传递函数 \( T(s) \) 的分母为 \( 1 + L(s) \)。

\begin{itemize}
    \item \textbf{系统的极点}:传递函数的极点就是使其分母等于零的 \( s \) 的值。
    \item \textbf{稳定性}:一个线性系统稳定的充要条件是\textbf{其传递函数的所有极点都必须具有负实部}(即都位于复平面的左半部分)。
\end{itemize}

因此,令闭环传递函数的分母为零,就得到了决定系统稳定性和动态特性的根本方程——\textbf{闭环特征方程}:
\[
\boxed{1 + L(s) = 0}
\]
其中 \( L(s) = C(s)P(s) \) 是开环传递函数。

\textbf{解这个方程得到的根 \( s \),就是闭环系统的极点。}

\subsection{推导过程(框图简化法)}

我们也可以通过框图的等效变换来理解。上述系统是一个简单的反馈环,其等效闭环传递函数可以直接用公式表示:

\textbf{闭环传递函数 = \frac{前向通道传递函数}{1 + 环路增益}}

其中"环路增益"是指信号沿着环路走一圈的总增益,在这个标准负反馈系统中,就是 \( L(s) = C(s)P(s) \)。

\begin{itemize}
    \item 前向通道传递函数:\( L(s) \)
    \item 环路增益:\( L(s) \)
\end{itemize}

所以:
\[
T(s) = \frac{L(s)}{1 + L(s)}
\]
同样,令分母 \( 1 + L(s) = 0 \) 即得特征方程。

\subsection{如果包含传感器动力学 \( H(s) \)}

如果反馈通道不是1,而是一个传感器模型 \( H(s) \),框图如下:

\begin{verbatim}
R(s) --> [+] --> [C(s)] --> [P(s)] ----> Y(s)
         ^                             |
         |                             |
         +------------[H(s)]<-----------+
\end{verbatim}

此时:
\begin{itemize}
    \item 开环传递函数 \( L(s) \) 定义为从误差信号 \( E(s) \) 开始,绕环路一周回到误差信号位置的总传递函数。即:
    \[
    L(s) = C(s)P(s)H(s)
    \]
    \item 闭环传递函数 \( T(s) = \frac{Y(s)}{R(s)} \) 变为:
    \[
    T(s) = \frac{C(s)P(s)}{1 + C(s)P(s)H(s)} = \frac{\text{前向通道}}{1 + \text{环路增益}}
    \]
    \item 此时,\textbf{闭环特征方程}为:
    \[
    1 + L(s) = 0 \quad \text{或} \quad \boxed{1 + C(s)P(s)H(s) = 0}
    \]
\end{itemize}

\subsection{总结}

\begin{itemize}
    \item \textbf{核心思想}:闭环系统的动态特性(稳定性、响应速度等)由其极点决定。
    \item \textbf{推导方法}:通过分析闭环系统的信号关系,求出闭环传递函数,其\textbf{分母多项式}即为系统特征多项式。
    \item \textbf{特征方程}:令闭环传递函数的分母等于零,\( 1 + L(s) = 0 \),此方程称为闭环特征方程。
    \item \textbf{物理意义}:该方程决定了系统在无外部输入 (\( R(s)=0 \)) 时,其内部状态的自由运动模式(衰减、振荡或发散)。
\end{itemize}

这个简单的方程 \( 1 + L(s) = 0 \) 是整个经典控制理论的基石,奈奎斯特判据、根轨迹法等都是基于这个方程来分析系统稳定性的





\section{推广到MIMO系统中}


非常好!将 SISO 系统的特征方程推广到 MIMO 系统是一个重要的概念飞跃。让我们详细推导这个过程。

\subsection{MIMO 系统的基本结构}

考虑一个多输入多输出的闭环系统:

\begin{verbatim}
R(s) --> [+] --> [控制器 C(s)] --> [被控对象 P(s)] -------+---> Y(s)
         ^                                              |
         |                                              |
         +----------------------------------------------+
\end{verbatim}

其中:
\begin{itemize}
    \item \( R(s) \):参考输入向量(m×1)
    \item \( Y(s) \):输出向量(p×1)  
    \item \( C(s) \):控制器传递函数矩阵(m×p)
    \item \( P(s) \):被控对象传递函数矩阵(p×m)
    \item 这里假设单位负反馈
\end{itemize}

\subsection{从 SISO 到 MIMO 的数学推广}

\subsubsection{SISO 系统的推导回顾}
对于 SISO 系统,我们有:
\[
Y(s) = L(s)E(s) = L(s)[R(s) - Y(s)]
\]
\[
Y(s) = L(s)R(s) - L(s)Y(s)
\]
\[
Y(s) + L(s)Y(s) = L(s)R(s)
\]
\[
[1 + L(s)]Y(s) = L(s)R(s)
\]
特征方程:\( 1 + L(s) = 0 \)

\subsubsection{MIMO 系统的推导}

在 MIMO 系统中,所有量都变成了矩阵/向量:

\begin{enumerate}
    \item \textbf{系统方程}:
    \[
    Y(s) = P(s)U(s)
    \]
    \[
    U(s) = C(s)E(s) = C(s)[R(s) - Y(s)]
    \]

    \item \textbf{代入得到}:
    \[
    Y(s) = P(s)C(s)[R(s) - Y(s)]
    \]

    \item \textbf{定义开环传递函数矩阵}:
    \[
    L(s) = P(s)C(s) \quad (\text{p} \times \text{p 矩阵})
    \]
    那么:
    \[
    Y(s) = L(s)[R(s) - Y(s)]
    \]

    \item \textbf{重新排列项}:
    \[
    Y(s) = L(s)R(s) - L(s)Y(s)
    \]
    \[
    Y(s) + L(s)Y(s) = L(s)R(s)
    \]

    \item \textbf{关键步骤:提取公因子}  
    在矩阵运算中,不能简单地把 1 提出来。我们需要使用单位矩阵 I:
    \[
    IY(s) + L(s)Y(s) = L(s)R(s)
    \]
    \[
    [I + L(s)]Y(s) = L(s)R(s)
    \]

    \item \textbf{得到闭环传递函数矩阵}:
    \[
    Y(s) = [I + L(s)]^{-1}L(s)R(s)
    \]
    闭环传递函数矩阵为:
    \[
    T(s) = [I + L(s)]^{-1}L(s)
    \]
\end{enumerate}

\subsection{MIMO 特征方程的推导}

\subsubsection{极点位置的决定因素}

在 MIMO 系统中,闭环传递函数矩阵 \( T(s) \) 的极点由什么决定?

观察:
\[
T(s) = [I + L(s)]^{-1}L(s)
\]

\begin{itemize}
    \item 矩阵的逆 \( [I + L(s)]^{-1} \) 可以写成:
    \[
    [I + L(s)]^{-1} = \frac{\text{adj}(I + L(s))}{\det(I + L(s))}
    \]
    其中 adj(·) 表示伴随矩阵。

    \item 因此:
    \[
    T(s) = \frac{\text{adj}(I + L(s))L(s)}{\det(I + L(s))}
    \]
\end{itemize}

\textbf{关键结论}:闭环传递函数矩阵 \( T(s) \) 的所有元素都有相同的分母多项式 \( \det(I + L(s)) \)。

因此,MIMO 系统的\textbf{闭环特征方程}为:
\[
\boxed{\det(I + L(s)) = 0}
\]

\subsection{数学上的严格证明}

\subsubsection{从状态空间角度}

假设开环系统的状态空间实现为:
\[
\begin{aligned}
\dot{x} &= Ax + Bu \\
y &= Cx + Du
\end{aligned}
\]

加上单位负反馈 \( u = r - y \),得到闭环系统:
\[
\begin{aligned}
\dot{x} &= Ax + B(r - y) \\
&= Ax + Br - B(Cx + Du) \\
&= (A - B(I + D)^{-1}C)x + B(I + D)^{-1}r
\end{aligned}
\]

闭环系统矩阵为:
\[
A_{cl} = A - B(I + D)^{-1}C
\]

闭环特征方程为:
\[
\det(sI - A_{cl}) = 0
\]

可以证明,这个方程等价于:
\[
\det(I + C(sI - A)^{-1}B + D) = 0
\]
而 \( C(sI - A)^{-1}B + D = P(s)C(s) = L(s) \),所以:
\[
\det(I + L(s)) = 0
\]

\subsection{物理意义解释}

\subsubsection{为什么是 det(I + L(s))?}

在 SISO 中,\( 1 + L(s) = 0 \) 意味着环路增益为 -1,系统处于稳定边界。

在 MIMO 中,\( \det(I + L(s)) = 0 \) 意味着:

\begin{enumerate}
    \item \textbf{矩阵 \( I + L(s) \) 是奇异的}
    \item \textbf{存在非零向量 v 使得 \( [I + L(s)]v = 0 \)}
    \item \textbf{即 \( L(s)v = -v \)}
\end{enumerate}

\textbf{物理解释}:存在某种输入模式 v,使得绕环路一周后,信号被精确地反向并保持幅度不变——这正是振荡的条件。

\subsubsection{与特征值的关系}

特征方程 \( \det(I + L(s)) = 0 \) 的解,给出了使闭环系统矩阵有特征值为 0 的 s 值。在频域分析中,我们关心的是这些解是否在右半平面。

\subsection{实际应用中的注意事项}

\subsubsection{反馈结构的推广}

对于更一般的反馈结构:
\begin{verbatim}
R(s) --> [+] --> [C(s)] --> [P(s)] ----> Y(s)
         ^                             |
         |                             |
         +------------[H(s)]<----------+
\end{verbatim}

开环传递函数矩阵为:
\[
L(s) = P(s)C(s)H(s)
\]
特征方程仍为:
\[
\det(I + L(s)) = 0
\]

\subsubsection{计算考虑}

在实际中,我们通常:
\begin{enumerate}
    \item 在多个频率点 \( s = j\omega \) 计算 \( L(j\omega) \)
    \item 检查 \( \det(I + L(j\omega)) \) 是否包围原点(推广的奈奎斯特判据)
    \item 或者直接求解 \( \det(I + L(s)) = 0 \) 的根
\end{enumerate}

\subsection{与鲁棒控制的联系}

在鲁棒控制的 M-Δ 框架中,特征方程为:
\[
\det(I - M\Delta) = 0
\]

这可以看作是:
\begin{itemize}
    \item 开环传递函数 \( L = -M\Delta \)
    \item 因此 \( \det(I + L) = \det(I - M\Delta) = 0 \)
\end{itemize}

负号的出现是因为 M-Δ 结构通常表示为正反馈。

\subsection{总结}

从 SISO 到 MIMO 的推广:

\begin{center}
\begin{tabular}{|l|l|l|}
\hline
\textbf{系统类型} & \textbf{开环传递函数} & \textbf{特征方程} \\
\hline
SISO & \( L(s) \) (标量) & \( 1 + L(s) = 0 \) \\
\hline
MIMO & \( L(s) \) (矩阵) & \( \det(I + L(s)) = 0 \) \\
\hline
\end{tabular}
\end{center}

这个推广是自然且优美的:\textbf{将标量的 1 替换为单位矩阵 I,将普通的加法替换为矩阵加法,将等式检验替换为行列式检验}。

MIMO 特征方程 \( \det(I + L(s)) = 0 \) 是分析多变量系统稳定性的理论基础,也是现代鲁棒控制理论的起点。

\section{通过 det(I + L(s)) = 0 判断的稳定性是 BIBO 稳定性,而不是内部稳定性}
\subsection{但是在鲁棒控制中,小增益定理通常假设标称系统 M 是内部稳定的,不确定性 Δ 是稳定的,在这些假设下,det(I - MΔ) = 0 判断的鲁棒稳定性就是内部稳定性,因为标称系统已经是内部稳定的,只要反馈互联不引入新的不稳定模式,整个系统就内部稳定}


您提出了一个非常关键且深刻的问题!确实,\textbf{通过 det(I + L(s)) = 0 判断的稳定性是 BIBO 稳定性,而不是内部稳定性}。您的质疑完全正确,这种方法在某些情况下确实不可靠。

让我详细解释这个问题。

\subsection{BIBO稳定 vs 内部稳定}

\subsubsection{BIBO稳定(有界输入有界输出稳定)}
\begin{itemize}
    \item \textbf{定义}:任何有界输入都产生有界输出
    \item \textbf{判据}:闭环传递函数的所有极点都在左半开平面
    \item \textbf{检查方法}:det(I + L(s)) = 0 的根都在左半开平面
\end{itemize}

\subsubsection{内部稳定}
\begin{itemize}
    \item \textbf{定义}:系统在零输入条件下的状态响应有界且渐近趋于零
    \item \textbf{判据}:闭环系统矩阵的所有特征值都在左半开平面
    \item \textbf{要求}:所有内部信号都是有界的
\end{itemize}

\subsection{为什么 det(I + L(s)) = 0 只能保证 BIBO 稳定?}

\subsubsection{问题根源:零极点对消}

考虑一个系统:
\begin{itemize}
    \item 控制器:\( C(s) = \frac{s-1}{s+2} \)
    \item 被控对象:\( P(s) = \frac{1}{s-1} \)
\end{itemize}

\textbf{开环传递函数}:
\[
L(s) = P(s)C(s) = \frac{1}{s-1} \cdot \frac{s-1}{s+2} = \frac{1}{s+2}
\]

\textbf{闭环特征方程}:
\[
\det(I + L(s)) = 1 + \frac{1}{s+2} = \frac{s+3}{s+2} = 0
\]
极点:s = -3(在左半平面)$\rightarrow$ \textbf{BIBO稳定}

\textbf{但是},系统内部发生了什么?

\subsubsection{状态空间分析}

P(s) 的状态空间实现:
\[
\dot{x}_p = x_p + u_p, \quad y_p = x_p
\]
有一个不稳定的极点 s = 1

C(s) 的状态空间实现:  
\[
\dot{x}_c = -2x_c + u_c, \quad y_c = -3x_c + u_c
\]

闭环后,\textbf{整个系统的系统矩阵}有一个特征值在 s = 1(不稳定!)

\textbf{发生了什么?}
\begin{itemize}
    \item 不稳定的极点 s = 1 在开环传递函数 L(s) 中被对消了
    \item 从外部看(BIBO),系统是稳定的
    \item 但从内部看,不稳定的模态依然存在,只是不可控或不可观
\end{itemize}

\subsection{这种方法的不可靠性}

\subsubsection{不可靠的情况:}
\begin{enumerate}
    \item \textbf{不稳定的零极点对消}
    \begin{itemize}
        \item 如上例,隐藏的不稳定模式不会被 det(I + L(s)) = 0 检测到
        \item 系统 BIBO 稳定但内部不稳定
    \end{itemize}
    
    \item \textbf{非最小实现系统}
    \begin{itemize}
        \item 系统有不可控或不可观模式
        \item det(I + L(s)) 只反映可控可观部分的动力学
    \end{itemize}
    
    \item \textbf{传感器/执行器动力学}
    \begin{itemize}
        \item 如果反馈通道 H(s) 有不稳定极点,可能在对消中被隐藏
    \end{itemize}
\end{enumerate}

\subsubsection{可靠的情况:}
当系统是\textbf{最小实现}(完全能控且完全能观)时:
\begin{itemize}
    \item BIBO 稳定 $\Leftrightarrow$ 内部稳定
    \item det(I + L(s)) = 0 可以可靠判断稳定性
\end{itemize}

\subsection{如何可靠地判断稳定性?}

\subsubsection{方法1:状态空间方法(推荐)}
直接分析闭环系统矩阵 \( A_{cl} \) 的特征值:

\[
\text{系统内部稳定} \iff \text{所有 } \lambda_i(A_{cl}) \text{ 都有负实部}
\]

这是最可靠的方法,能检测所有隐藏的不稳定模式。

\subsubsection{方法2:广义特征方程}
检查所有可能的输入-输出对的传递函数:

对于系统:
\begin{verbatim}
R --> [+] --> C(s) --> P(s) --> Y
       ^                      |
       |                      |
       +-------- H(s) <--------
\end{verbatim}

需要检查以下四个传递函数都稳定:
\begin{enumerate}
    \item \( [I + PCH]^{-1} \) - 灵敏度函数
    \item \( [I + PCH]^{-1}PC \)  
    \item \( [I + PCH]^{-1}P \)
    \item \( C[I + PCH]^{-1} \)
\end{enumerate}

如果这四个传递函数都稳定,则系统内部稳定。

\subsubsection{方法3:使用广义植物框架}
在鲁棒控制的广义植物框架中,内部稳定性要求:
\begin{itemize}
    \item 所有从外部输入到内部输出的传递函数都稳定
    \item 这自动避免了不稳定的零极点对消问题
\end{itemize}

\subsection{在鲁棒控制中的处理}

在鲁棒控制的 M-$\Delta$ 框架中,小增益定理通常假设:

\begin{enumerate}
    \item \textbf{标称系统 M 是内部稳定的}
    \item \textbf{不确定性 $\Delta$ 是稳定的}
\end{enumerate}

在这些假设下:
\begin{itemize}
    \item det(I - M$\Delta$) = 0 判断的鲁棒稳定性就是内部稳定性
    \item 因为标称系统已经是内部稳定的,只要反馈互联不引入新的不稳定模式,整个系统就内部稳定
\end{itemize}

\subsection{实际工程中的建议}

\subsubsection{设计原则:}
\begin{enumerate}
    \item \textbf{避免不稳定的零极点对消}
    \begin{itemize}
        \item 在控制器设计中,不要对消被控对象的不稳定极点
        \item 使用其他方法(如状态反馈)来稳定不稳定模式
    \end{itemize}
    
    \item \textbf{使用状态空间方法进行最终验证}
    \begin{itemize}
        \item 在设计完成后,通过状态空间模型验证内部稳定性
    \end{itemize}
    
    \item \textbf{在频域设计中保持警惕}
    \begin{itemize}
        \item 如果使用频域方法,确保系统是最小实现
        \item 检查所有可能的输入-输出通道
    \end{itemize}
\end{enumerate}

\subsubsection{具体步骤:}
\begin{lstlisting}[language=Python]
# 伪代码示例:可靠的稳定性检验
def check_internal_stability(P, C, H=None):
    # 方法1:状态空间分析(最可靠)
    A_cl = closed_loop_system_matrix(P, C, H)
    eigenvalues = np.linalg.eigvals(A_cl)
    if all(np.real(eigenvalues) < 0):
        return True  # 内部稳定
    else:
        return False  # 内部不稳定
    
    # 方法2:检查所有传递函数
    # 这也能发现不稳定的零极点对消
\end{lstlisting}

\subsection{总结}

您的质疑完全正确:

\begin{enumerate}
    \item \textbf{det(I + L(s)) = 0 只能保证 BIBO 稳定},不能保证内部稳定
    \item \textbf{当存在不稳定的零极点对消时,这种方法会给出错误的结论}
    \item \textbf{最可靠的稳定性判断方法是状态空间分析},直接检查闭环系统矩阵的特征值
    \item \textbf{在鲁棒控制中,我们通常假设标称系统已经是内部稳定的},然后分析不确定性对稳定性的影响
\end{enumerate}

因此,在严格的系统设计中,我们应该:
\begin{itemize}
    \item 使用状态空间方法进行最终的稳定性验证
    \item 在频域设计中保持对零极点对消的警惕
    \item 遵循"不对消不稳定极点"的设计原则
\end{itemize}

这确保了我们的稳定性判断既是 BIBO 稳定的,也是内部稳定的。

\section{\hl{如果标称系统都不内部稳定的话,那就没有必要考虑鲁棒稳定性了}}
因为鲁棒是又在标称系统上加入了不确定性,你标称系统都不稳定,加了不确定性之后肯定更难以稳定了







\end{document}