\documentclass[10pt]{article}
\usepackage{researchnotes}
\usepackage[utf8]{inputenc}
\usepackage[T1]{fontenc}
\usepackage{amsmath}
\usepackage{amsfonts}
\usepackage{amssymb}
\usepackage{array}
\usepackage{ctex}
\usepackage[version=4]{mhchem}
\usepackage{stmaryrd}
\usepackage{hyperref}
\hypersetup{colorlinks=true, linkcolor=blue, filecolor=magenta, urlcolor=cyan,}
\urlstyle{same}
\usepackage{graphicx}
\usepackage[export]{adjustbox}
\graphicspath{ {./images/} }
\usepackage{caption}
\usepackage{mathrsfs}
\usepackage{bbold}
\usepackage{tikz}
\usepackage{listings}
\usepackage{booktabs}
\usetikzlibrary{shapes,arrows,positioning}

\title{稳定性的逻辑}

\author{黄嘉聪}
\date{}



\begin{document}
\maketitle
\captionsetup{singlelinecheck=false}


\begin{abstract}
主要介绍了稳定性的相关知识,重点在于理解:稳定性条件(传统SISO和传统MIMO和鲁棒稳定背后有什么共性?有界输入有界输出稳定和内稳定)为什么鲁棒稳定的条件是det(I-M\delta)
\end{abstract}

KEY WORDS: robust control; $\mu$; skew $\mu$; lower bound

\section{复单位圆指的是特征值的取值范围?单位球是$\Delta$的取值范围?}

非常好,你的问题切到了 \textbf{$\mu$-分析几何图景的核心}:
"复单位圆"和"单位球"这两个'单位'确实是两个不同层面的概念——一个在\textbf{特征值空间($\lambda$-平面)}里,一个在\textbf{矩阵空间($\Delta$-空间)}里。下面我们分层详细解释:

\subsection{两个"空间"要分清}

\begin{center}
\begin{tabular}{|p{0.2\textwidth}|p{0.15\textwidth}|p{0.15\textwidth}|p{0.25\textwidth}|}
\hline
\textbf{概念} & \textbf{所属空间} & \textbf{几何意义} & \textbf{出现位置} \\
\hline
\textbf{复单位圆} $\{\lambda\in\mathbb{C}:|\lambda|=1\}$ & 特征值平面($\lambda$-plane) & 描述矩阵 $\Delta M$ 的\textbf{特征值位置} & 用来判断系统"临界稳定"或"奇异" \\
\hline
\textbf{结构化单位球} $B_{X_K}=\{\Delta\in X_K:\bar\sigma(\Delta)\le1\}$ & 不确定性矩阵空间($\Delta$-space) & 描述\textbf{允许的不确定性方向/幅度范围} & $\mu$ 的定义域,搜索"最坏方向" \\
\hline
\end{tabular}
\end{center}

\subsection{复单位圆:特征值层面的边界}

在 $\mu$ 定义或稳定性分析里,矩阵 $\Delta M$ 的特征值 $\lambda$ 决定系统是否"碰到边界":

\begin{itemize}
    \item \textbf{离散时间系统}:若所有特征值都在复单位圆 \textbf{内}($|\lambda|<1$),系统稳定;
    \item \textbf{达到奇异或失稳}:有特征值恰好在单位圆 \textbf{上}($|\lambda|=1$);
    \item \textbf{$\mu$ 定义中的关键点}:
    当我们找到一个 $\Delta$ 使得
    \[
    \det(I-\Delta M)=0,
    \]
    这就意味着 $1$ 是 $\Delta M$ 的\textbf{特征值},而 1 就在复单位圆上($|1|=1$)。
    所以我们也可以说:当某个特征值"撞到复单位圆"时,系统到达鲁棒稳定的边界。
\end{itemize}

因此:

\begin{quote}
\textbf{复单位圆}是特征值的"允许范围"的边界。
在 $\mu$-分析里,$\mu$ 反映要让特征值从原本的稳定区域(单位圆内)推到单位圆上的最小缩放倍数。
\end{quote}

\subsection{结构化单位球:$\Delta$ 的可选范围}

另一方面,$\mu$ 是在 \textbf{$\Delta$-空间} 里搜索"哪一个方向最容易让系统变坏"。
定义:
\[
B_{X_K}=\{\Delta\in X_K:\bar\sigma(\Delta)\le1\}.
\]

\begin{itemize}
    \item 这是所有允许结构(块对角)的 $\Delta$ 的\textbf{单位球(或单位壳)};
    \item 它规定了每个不确定性块的\textbf{范数约束}(复块 $|\Delta|\le1$、实块 $|\Delta|\le1$、全复块 $|\Delta|\le1$);
    \item $\mu$ 的定义相当于问:
\end{itemize}

\begin{quote}
需要把 $\Delta$ 的整体"半径"放大多少倍,才能让特征值撞到复单位圆?
\end{quote}

即:
\[
\text{最小 }t>0\ \text{使存在 } \Delta\in B_{X_K},\ \det(I-t\Delta M)=0.
\]
由此得到
\[
\mu_K(M)=\frac{1}{t_{\min}}=\max_{\Delta\in B_{X_K}}\rho_R(\Delta M).
\]
这就是为什么"单位球"出现——它是 $\Delta$ 的取值范围,而"单位圆"是 $\lambda$ 的取值范围。二者通过方程
\[
(I-\Delta M)v=0 \Leftrightarrow \lambda=1
\]
彼此关联。

\subsection{\hl{联系起来的几何直觉}}

把两层放在一起想:

\begin{enumerate}
    \item \textbf{$\Delta$ 空间(不确定性方向)}:
    在单位球 $B_{X_K}$ 中找一个方向 $\Delta$,使得 $\Delta M$ 的特征值 $\lambda$ 尽量大。
    \item \textbf{$\lambda$ 空间(特征值位置)}:
    对应的 $\lambda$ 从原来单位圆内(安全区)向外移动;
    当 $\lambda=1$(或 $|\lambda|=1$)时,系统刚好变得奇异/临界稳定。
\end{enumerate}

因此:

\begin{itemize}
    \item \textbf{单位球}:$\Delta$ 的"可用能量范围";
    \item \textbf{复单位圆}:系统特征值的"稳定边界";
    \item \textbf{$\mu$}:这两个空间的"桥梁",描述要把 $\Delta$ 的半径放大到多少,使 $\lambda$ 到达单位圆。
\end{itemize}

\subsection{图形比喻}

可以想成两个同心圆:

\begin{itemize}
    \item \textbf{内圈(单位圆)}:$\lambda$ 平面上特征值的允许区域;
    \item \textbf{外圈($\Delta$ 单位球)}:$\Delta$ 空间里不确定性的允许大小。
\end{itemize}

$\mu$ 问的是:

\begin{quote}
需要把 $\Delta$-球放大多少倍,才能在 $\lambda$-平面上让某个特征值刚好碰到单位圆?
\end{quote}

\subsection{总结一句话}

\begin{itemize}
    \item \textbf{复单位圆} $\rightarrow$ 是特征值 $\lambda$ 的几何边界(判别系统或 $(I-\Delta M)$ 是否奇异);
    \item \textbf{结构化单位球} $\rightarrow$ 是不确定性矩阵 $\Delta$ 的范数约束集合;
    \item $\mu$ 连接两者:它告诉我们在结构化单位球上放大多少倍,才能让 $\Delta M$ 的特征值撞到复单位圆。
\end{itemize}






\section{不确定性都结构化了为什么还能是个球形?}

这个问题问得非常深刻——你已经抓到了 \textbf{$\mu$-分析几何直觉的精髓}。
看似"结构化"的不确定性,为什么我们还称它的取值集合为"球形单位球"?
下面我从三个层次(\textbf{数学层面、几何层面、结构层面})来完整解释,让你真正理解:

\subsection{数学层面:"球形"来自范数约束,不是形状几何意义的球体}

当我们写
\[
B_{X_K} = \{\Delta \in X_K : \|\Delta\| \le 1\},
\]
这里的"球"其实是指在一个向量空间上以某个\textbf{矩阵范数}定义的"单位球体"集合。
它并不是在三维欧式空间里的圆球那种形状,而是范数意义上的"等半径集合"。

例如,在 $\mathbb{R}^2$ 上,用不同的范数定义"单位球":

\begin{itemize}
    \item 2-范数(欧几里得)$\rightarrow$ 圆形;
    \item 1-范数 $\rightarrow$ 菱形;
    \item $\infty$-范数 $\rightarrow$ 正方形。
\end{itemize}

它们都叫"unit ball",只是形状不同。

矩阵空间的"球"同理。
我们说"单位球",是因为它满足
\[
\|\Delta\| \le 1,
\]
而不是说它真的几何上是"球体"。

\subsection{几何层面:结构化 $\neq$ 非球,而是球的笛卡尔积}

当我们引入\textbf{结构化不确定性},即
\[
\Delta = \mathrm{diag}(\Delta_1,\Delta_2,\ldots,\Delta_k),
\]
每个子块 $\Delta_i$ 自己也有一个局部"单位球"定义:

\begin{center}
\begin{tabular}{|p{0.3\textwidth}|p{0.3\textwidth}|p{0.3\textwidth}|}
\hline
\textbf{不确定性类型} & \textbf{局部集合} & \textbf{几何直觉} \\
\hline
实标量 $\delta_i\in\mathbb{R}$ & $|\delta_i| \le 1$ & 实轴上的区间 $[-1,1]$ \\
\hline
复标量 $\delta_i\in\mathbb{C}$ & $|\delta_i| \le 1$ & 复平面上的圆盘 \\
\hline
全复块 $\Delta_i\in\mathbb{C}^{r_i\times r_i}$ & $\bar\sigma(\Delta_i)\le1$ & 高维空间中的"矩阵单位球" \\
\hline
\end{tabular}
\end{center}

把这些\textbf{局部单位球}按块拼成对角块对角结构:
\[
B_{X_K} = \left\{\mathrm{diag}(\Delta_1,\dots,\Delta_k):\ \bar\sigma(\Delta_i)\le1\right\}.
\]

几何上,这个集合不是一个"单个球",而是各个子球的\textbf{笛卡尔积 (direct product)}。
我们叫它"结构化单位球(structured unit ball)"。

$\Rightarrow$ 所以严格说,这个集合是一个\textbf{块乘积形球体},
每个块的"球形半径"为1,但块与块之间没有耦合。

\subsection{结构层面:"结构化"体现的是方向约束,不是形状约束}

当你在 $\mu$-分析中结构化不确定性时,你其实是说:

\begin{quote}
"$\Delta$ 的非零元素只能出现在这些块中(方向约束),
但在每个块内部,大小依然按范数约束 $\le 1$。"
\end{quote}

也就是说:

\begin{itemize}
    \item \textbf{结构化} 限制了 $\Delta$ 的"方向";
    \item \textbf{球形} 限制了 $\Delta$ 的"长度"。
\end{itemize}

举个直观比喻:

\begin{quote}
你规定:一个人只能在若干固定的街区(块结构)里走路,
但每个街区内他能走的距离(长度)不超过 1。
那么所有可能的行走位置集合仍然是"每个街区的球形区域"的笛卡尔积——也可以叫"结构化单位球"。
\end{quote}

\subsection{形式上为什么还是"球"的表达式}

因为 $\mu$ 的定义是:
\[
\mu_K(M)^{-1} = \min_{\Delta\in X_K} \{ \|\Delta\|:\ \det(I-\Delta M)=0\}.
\]

这在形式上与\textbf{非结构化不确定性}完全一致,只是约束 $\Delta \in X_K$。
因此自然定义
\[
B_{X_K} := \{ \Delta\in X_K : \|\Delta\| \le 1\},
\]
沿用了"unit ball"的名字。
只不过这个"球"在数学上是"限制在结构化子空间上的球体切片 (section of the full unit ball)"。

\subsection{一个直观例子:二维图像}

假设
\[
\Delta = \begin{bmatrix}
\delta_1 & 0 \\
0 & \delta_2
\end{bmatrix}, \qquad
\delta_1,\delta_2\in\mathbb{R},\ |\delta_i|\le 1.
\]

那么:

\begin{itemize}
    \item 完全非结构化的单位球是 $\{\Delta:\|\Delta\|_2\le1\}$,是一个真正的四维"球"。
    \item 而这个结构化集合等价于 $\{(\delta_1,\delta_2):|\delta_1|\le1,|\delta_2|\le1\}$,
    几何上是一个\textbf{二维正方形}。
\end{itemize}

虽然它的形状不是"圆的",我们仍然称它为"结构化单位球"——
因为它是"以 $\|\Delta\|\le1$" 为半径约束的子集。

\subsection{小结一句话}

\begin{quote}
"结构化"决定了 \textbf{哪些方向(子空间)} 可以取值;
"单位球"决定了 \textbf{这些方向上长度(范数)$\le1$}。
\end{quote}

所以:

\begin{itemize}
    \item \textbf{单位球} $\neq$ "$\Delta$ 完全自由的球形区域",
    \item \textbf{而是} "在特定结构方向上截取的、范数半径为1的球体切片"。
\end{itemize}

数学上:
\[
B_{X_K} = X_K \cap \{\Delta:\|\Delta\|\le1\}
\]
这就是 \textbf{结构化单位球} 的精确定义与"球形"意义。

\subsection{总结}

\begin{center}
\begin{tabular}{|p{0.3\textwidth}|p{0.6\textwidth}|}
\hline
\textbf{概念} & \textbf{实际含义} \\
\hline
"单位球" & 指所有范数 $\le 1$ 的矩阵集合(形状可不是几何球) \\
\hline
"结构化" & 限制 $\Delta$ 的非零分块位置和重复模式 \\
\hline
"结构化单位球" & 是在结构化子空间中截取出的范数$\le1$的集合 \\
\hline
"为什么还能是球" & 因为范数约束仍然是"半径型"的,只是局限在结构化方向上 \\
\hline
\end{tabular}
\end{center}


































\section{谱半径:最大特征值模}


下面把"谱半径(spectral radius)"讲清楚:它的精确定义、与特征值/范数的关系、常用性质和在系统/鲁棒分析里的用法,并给几个小例子帮你快速判断。

\subsection{定义(核心一句话)}

对任意方阵 $A\in\mathbb{C}^{n\times n}$,\textbf{谱半径}定义为
\[
\rho(A) = \max\{|\lambda|:\ \lambda\in\sigma(A)\},
\]
也就是 \textbf{A 的全部特征值模长(绝对值)里最大的那个}。

\begin{quote}
补:有时会区分"实谱半径" $\rho_R(A)$(仅在实特征值里取最大值;若无实特征值则记为 0)。你在混合-$\mu$推导里遇到的 $\rho_R$ 就是这个量,它与标准谱半径 $\rho(A)$ 不同。
\end{quote}

\subsection{与特征值/范数的基本关系}

\subsubsection{相似不变性}
\[
\rho(SAS^{-1})=\rho(A)\quad(\forall\ S\ \text{可逆})
\]
换一组基(相似变换)不改变谱半径。这也说明谱半径只取决于特征值本身。

\subsubsection{标量缩放}
\[
\rho(\alpha A)=|\alpha|\rho(A)\quad(\alpha\in\mathbb{C})
\]

\subsubsection{与任何次乘性矩阵范数的关系}
对任一矩阵范数 $\|\cdot\|$(满足 $\|AB\|\le\|A\|\|B\|$),都有
\[
\rho(A) \le \|A\|.
\]
并且著名的 \textbf{Gelfand 公式}(非常重要):
\[
\boxed{\rho(A) = \lim_{k\to\infty} \|A^k\|^{1/k} = \inf_{k\ge 1} \|A^k\|^{1/k}}
\]
这告诉你:无论选什么范数,取高次幂的"k 次方根极限",都会收敛到谱半径。

\subsubsection{与奇异值(2-范数)的关系}

\begin{itemize}
    \item 一般情形:$\rho(A) \le \|A\|_2 = \bar\sigma(A)$(最大奇异值)。
    \item \textbf{若 $A$ 正规(normal)}(与其共轭转置可交换,如厄米/酉/对称矩阵),则
    \[
    \|A\|_2 = \max_i |\lambda_i(A)| = \rho(A).
    \]
    即正规矩阵时,谱半径就等于 2-范数。
\end{itemize}

\subsubsection{与特征多项式/根的关系}
若 $p_A(\lambda)=\det(\lambda I-A)=\prod_{i=1}^n(\lambda-\lambda_i)$,则
\[
\rho(A)=\max_i |\lambda_i|.
\]
因此它是"特征多项式所有根的最大模"。

\subsubsection{Gershgorin 圆盘给界}
若 $A=[a_{ij}]$,以第 $i$ 行为中心 $a_{ii}$、半径 $R_i=\sum_{j\ne i}|a_{ij}|$ 的 Gershgorin 圆盘覆盖所有特征值,于是
\[
\rho(A) \le \max_i \big(|a_{ii}|+R_i\big).
\]
这是估计谱半径的一个简单上界。

\subsection{与稳定性/动力学的关系}

\begin{itemize}
    \item \textbf{离散时间系统} $x_{k+1}=Ax_k$:
    若 $A$ 可对角化且 $\rho(A)<1$,则 $A^k\to 0$,即\textbf{渐近稳定}。更广泛地,谱半径控制了 $A^k$ 的增长/衰减速率(见 Gelfand 公式)。
    
    \item \textbf{连续时间系统} $\dot x=Ax$:
    稳定性由 \textbf{谱幅}(spectral abscissa,最大实部)而不是谱半径决定:$\alpha(A):=\max_i \Re\lambda_i(A)<0$ 则稳定。这里容易和谱半径混淆:连续时间看"实部",离散时间看"模长"。
    
    \item \textbf{非负矩阵(Perron–Frobenius)}
    若 $A\ge 0$(元素非负)且不可约,则 $\rho(A)$ 是 A 的最大实特征值,对应\textbf{正}的左右特征向量。这在网络/马尔可夫链/图论中很常见。
\end{itemize}

\subsection{在鲁棒控制与 $\mu$-分析中的角色}

在你之前的问题中,我们用到了
\[
\mu_K(M) = \max_{\Delta\in B_{X_K}}\rho_R(\Delta M),
\]
这里的 $\rho_R$ 是"实谱半径"。原因是奇异条件 $\det(I-\Delta M)=0$ 等价于"$\Delta M$ 存在实特征值 1",于是把"找最小范数使奇异"等价成"在单位球上把 $\Delta M$ 的实特征值推到最大"。这就是谱半径(或实谱半径)在 $\mu$ 下界公式中的关键位置。

\subsection{快速例子}

\begin{enumerate}
    \item \textbf{对角矩阵} $A=\mathrm{diag}(2,-3,1+i)$
    特征值模 $\{2,3,\sqrt{2}\}$,$\rho(A)=3$。
    
    \item \textbf{幂零矩阵} $N=\begin{bmatrix}0&1\\0&0\end{bmatrix}$
    特征值全是 0,$\rho(N)=0$;但 $\|N\|_2>0$。说明谱半径与范数不同。
    
    \item \textbf{正规矩阵} $U$ 酉阵
    $|\lambda_i(U)|=1$,$\rho(U)=1=\|U\|_2$。
    
    \item \textbf{离散稳定}
    若 $\rho(A)=0.8$,通常 $A^k$ 以 $0.8^k$ 量级衰减(精确速率受是否正规及条件数影响,但 Gelfand 公式保证极限根在 0.8)。
\end{enumerate}

\subsection{小结}

\begin{itemize}
    \item 谱半径 $\rho(A)$ 是"最大的特征值模"。
    \item 它与任何矩阵范数满足 $\rho(A)\le\|A\|$,并由 Gelfand 公式与幂次极限紧密相连。
    \item 正规矩阵时 $\rho(A)=\|A\|_2$。
    \item 离散系统稳定看 $\rho(A)<1$;连续系统稳定看谱幅 $\alpha(A)<0$。
    \item 在 $\mu$-分析里,谱半径(尤其"实谱半径")把"奇异化问题"转成"特征值最大化问题",是下界算法的核心桥梁。
\end{itemize}



\begin{questionnote}[author=JC, date=2025-10-23]
    $\Delta M$如果不是一个方阵,那应该没有特征值才对啊?你说的这个是不是有问题?
    \begin{figure}[H]
    \begin{center}
        \includegraphics[width=\textwidth]{实谱半径和μ下界的关系}
        \captionsetup{labelformat=empty}
        % \caption{}
    \end{center}
    \end{figure}
\end{questionnote}


\section{LFT保证了$\Delta M$是方阵}


非常好,这个问题非常关键 —— 你已经注意到了很多初学 $\mu$-分析时容易忽略的一个"隐藏假设":
\textbf{$\Delta M$ 必须是方阵},否则谈特征值确实是没有意义的。
让我们仔细澄清这一点。

\subsection{你的直觉完全正确}

在一般矩阵理论中:

\begin{itemize}
    \item 如果 $A\in\mathbb{C}^{m\times n}$,且 $m\neq n$,那么 $A$ 不是方阵;
    \item 此时 $\det(I-A)$ 没有定义(因为 $I$ 只能定义在方阵空间);
    \item 所以自然也就\textbf{没有特征值的概念}(特征值只对方阵定义)。
\end{itemize}

因此,如果 $\mu$ 定义中出现了
\[
\det(I-\Delta M)=0,
\]
就必须隐含一个前提:\textbf{$\Delta M$ 是方阵。}

\subsection{在 $\mu$-分析框架中,这个前提是成立的}

在 $\mu$ 理论里,矩阵 $M$ 与不确定性 $\Delta$ 的维度并不是随便的,而是通过 \textbf{线性分式变换 (LFT)} 精确匹配好的。
通常我们写:
\[
M = \begin{bmatrix}
M_{11} & M_{12} \\
M_{21} & M_{22}
\end{bmatrix},
\qquad
\text{并定义闭环: } F_u(M,\Delta)=M_{22}+M_{21}\Delta(I-M_{11}\Delta)^{-1}M_{12}.
\]

而 $\mu$ 分析是在"不确定性反馈环节"内部进行的,定义的是:
\[
\mu_K(M) = \frac{1}{\min\{\|\Delta\|:\ \det(I-\Delta M)=0,\ \Delta\in X_K\}}.
\]

此处的 $M$ 并不是任意矩阵,而是经过系统分块后处于 LFT 反馈通道里的那一块。
维度选择保证了乘积 $\Delta M$ \textbf{总是方阵},因为它代表的是反馈通道从"$\Delta$输出$\rightarrow$$\Delta$输入"的映射。

具体来说:

\begin{itemize}
    \item 若 $\Delta$ 映射的是"$\Delta$输入 $\rightarrow$ $\Delta$输出",那么 $M$ 的相应块 $M_{22}$ 映射的是"$\Delta$输出 $\rightarrow$ $\Delta$输入";
    \item 于是 $\Delta M_{22}$ 的维度自然是 $n_\Delta\times n_\Delta$。
\end{itemize}

$\checkmark$ \textbf{因此在 $\mu$ 定义中,乘积 $\Delta M$ 一定是方阵。}

\subsection{为什么理论必须要求"方阵反馈通道"}

这是鲁棒控制的核心逻辑之一:

\begin{itemize}
    \item 反馈闭环 $I-\Delta M$ 代表系统整体的内部映射;
    \item 稳定性要求这个反馈环节可逆;
    \item "奇异"意味着闭环不再可逆(等价于有特征值 1);
    \item 这必须建立在反馈通道\textbf{输入输出维度相等}的前提下。
\end{itemize}

否则根本无法定义"$I-\Delta M$"或"$\det(\cdots)$"。

\subsection{如果 $\Delta M$ 不是方阵怎么办?}

如果真的有某个分析环节出现 $M\in\mathbb{C}^{p\times q}$,$\Delta\in\mathbb{C}^{q\times p}$,使 $p\neq q$,那就要重新构造分析矩阵使其方阵化。

常见做法有两种:

\subsubsection{扩维(augmentation)}

构造一个等价的方阵系统块,例如
\[
\begin{bmatrix}
0 & M \\
I & 0
\end{bmatrix}
\begin{bmatrix}
\Delta & 0 \\
0 & I
\end{bmatrix},
\]
这样整体就变成方阵,再分析其谱性质。

\subsubsection{通过LFT重构}

在标准 LFT 框架中,系统的上/下互联已经自然保证
$\Delta M$ 方阵 —— 如果没方阵,那说明你取错了分析块。

\subsection{小结(重点)}

\begin{center}
\begin{tabular}{|p{0.2\textwidth}|p{0.7\textwidth}|}
\hline
\textbf{项目} & \textbf{正确理解} \\
\hline
\textbf{$\mu$ 定义} & $\displaystyle \mu_K(M)=\frac{1}{\min_{\Delta\in X_K}\{\|\Delta\|:\det(I-\Delta M)=0\}}$ \\
\hline
\textbf{要求} & $\Delta M$ 必须是方阵;否则 det 和特征值都没定义 \\
\hline
\textbf{解释} & 在 $\mu$-分析里 $M$ 取的是闭环不确定性反馈通道,维度天然匹配 \\
\hline
\textbf{因此} & 讨论"$\Delta M$ 的实特征值 1"是完全合法的前提下说的 \\
\hline
\textbf{若不方阵} & 需要通过扩维或重新定义反馈路径使其方阵化 \\
\hline
\end{tabular}
\end{center}

$\checkmark$ \textbf{一句话总结:}

\begin{quote}
在 $\mu$-分析中,$\Delta M$ 代表的是"不确定性反馈通道"的闭环传递矩阵。
由于反馈系统输入输出必须匹配,所以它总是方阵。
因此我们才能合法地谈"特征值""谱半径""奇异条件 $\det(I-\Delta M)=0$"这些概念。
\end{quote}




















\begin{researchnote}[author=JC, date=2025-10-23]
    \subsection*{\hl{为什么最小的扰动对应最大的特征值——以标量不确定性举例}}
    \subsubsection*{情况一:$\Delta = \{\delta I : \delta \in \mathbb{C}\}$}

        这时不确定性是标量乘单位矩阵。

        \textbf{为什么 $\mu_{\Delta}(M) = \rho(M)$?}

        \begin{itemize}
            \item 特征方程:$\det(I - M(\delta I)) = \det(I - \delta M) = 0$
            \item 这等价于 $\det(\frac{1}{\delta}I - M) = 0$,即 $1/\delta$ 是 $M$ 的特征值
            \item 最小的 $|\delta|$ 对应最大的 $|1/\delta|$,即 $M$ 的谱半径 $\rho(M)$
            \item 因此 $\mu_{\Delta}(M) = 1 / \min|\delta| = \rho(M)$
        \end{itemize}
\end{researchnote}






\section{结构奇异值的等价定义————谱半径}


    \begin{figure}[H]
    \begin{center}
        \includegraphics[width=\textwidth]{结构奇异值等价定义}
        \captionsetup{labelformat=empty}
        % \caption{}
    \end{center}
    \end{figure}

这个引理是结构奇异值($\mu$)理论中的核心结果。让我详细解释为什么这个引理成立。

\subsection{结构奇异值 $\mu$ 的基本定义}

首先回顾结构奇异值 $\mu$ 的定义:
\[
\mu_{\Delta}(M) = \frac{1}{\min\{\bar{\sigma}(\Delta) : \Delta \in \Delta, \det(I - M\Delta) = 0\}}
\]
如果没有这样的 $\Delta$ 存在,则 $\mu_{\Delta}(M) = 0$。

换句话说:\textbf{$\mu$ 是使系统变得不稳定的最小不确定性 $\Delta$ 的范数的倒数}。

\subsection{引理 10.1 的重新表述}

引理声称:
\[
\mu_{\Delta}(M) = \max_{\Delta \in B\Delta} \rho(M\Delta)
\]
其中 $B\Delta = \{\Delta \in \Delta : \bar{\sigma}(\Delta) \leq 1\}$ 是单位球内的不确定性集合。

\subsection{为什么这个等式成立?}

\subsubsection{直观理解}

考虑特征方程 $\det(I - M\Delta) = 0$,这等价于说 \textbf{1 是 $M\Delta$ 的特征值}。

现在,谱半径 $\rho(M\Delta)$ 是 $M\Delta$ 的特征值的最大模。如果 $\rho(M\Delta) \geq 1$,意味着存在某个特征值的模至少为 1。

\textbf{关键洞察}:
\begin{itemize}
    \item 如果 $\rho(M\Delta) < 1$,那么所有特征值的模都小于 1,系统稳定
    \item 如果 $\rho(M\Delta) = 1$,那么系统处于稳定边界
    \item 如果 $\rho(M\Delta) > 1$,那么系统不稳定
\end{itemize}

因此,\textbf{最大化 $\rho(M\Delta)$ 就相当于找到最坏情况的不确定性,使系统最接近不稳定}。

\subsubsection{数学证明思路}

从 $\mu$ 的定义出发:
\[
\mu_{\Delta}(M) = \frac{1}{\min\{\bar{\sigma}(\Delta) : \det(I - M\Delta) = 0\}}
\]

设 $\Delta_0$ 是达到这个最小值的 $\Delta$,即:
\[
\bar{\sigma}(\Delta_0) = \min\{\bar{\sigma}(\Delta) : \det(I - M\Delta) = 0\}
\]
那么:
\[
\mu_{\Delta}(M) = \frac{1}{\bar{\sigma}(\Delta_0)}
\]

现在考虑缩放后的不确定性 $\tilde{\Delta} = \frac{\Delta_0}{\bar{\sigma}(\Delta_0)}$,则:
\begin{itemize}
    \item $\bar{\sigma}(\tilde{\Delta}) = 1$,所以 $\tilde{\Delta} \in B\Delta$
    \item $\det(I - M\tilde{\Delta}) = \det\left(I - \frac{M\Delta_0}{\bar{\sigma}(\Delta_0)}\right) = 0$
\end{itemize}

由于 $\det(I - M\tilde{\Delta}) = 0$,存在特征值 $\lambda$ 使得 $M\tilde{\Delta}v = \lambda v$ 且 $|\lambda| = 1$。

因此:
\[
\rho(M\tilde{\Delta}) \geq 1
\]

实际上,可以证明对于这个特定的 $\tilde{\Delta}$,有 $\rho(M\tilde{\Delta}) = 1$。

反过来,对于任意 $\Delta \in B\Delta$,如果 $\rho(M\Delta) = c$,那么通过缩放可以构造一个使系统不稳定的 $\Delta'$,其范数为 $1/c$。

因此,\textbf{最大谱半径正好等于 $\mu$}。

\subsection{两个极端情况的解释}

\subsubsection{情况一:$\Delta = \{\delta I : \delta \in \mathbb{C}\}$}

这时不确定性是标量乘单位矩阵。

\textbf{为什么 $\mu_{\Delta}(M) = \rho(M)$?}

\begin{itemize}
    \item 特征方程:$\det(I - M(\delta I)) = \det(I - \delta M) = 0$
    \item 这等价于 $\det(\frac{1}{\delta}I - M) = 0$,即 $1/\delta$ 是 $M$ 的特征值
    \item 最小的 $|\delta|$ 对应最大的 $|1/\delta|$,即 $M$ 的谱半径 $\rho(M)$
    \item 因此 $\mu_{\Delta}(M) = 1 / \min|\delta| = \rho(M)$
\end{itemize}

\subsubsection{情况二:$\Delta = \mathbb{C}^{n \times n}$}

这时不确定性是任意复数矩阵。

\textbf{为什么 $\mu_{\Delta}(M) = \bar{\sigma}(M)$?}

\begin{itemize}
    \item 根据小增益定理,最坏情况的不确定性沿着 $M$ 的最大增益方向
    \item 最小的 $\bar{\sigma}(\Delta)$ 使得 $\det(I - M\Delta) = 0$ 是 $1/\bar{\sigma}(M)$
    \item 因此 $\mu_{\Delta}(M) = \bar{\sigma}(M)$
\end{itemize}

\subsection{不等式关系的几何解释}

\[
\{\delta I_n : \delta \in \mathbb{C}\} \subset \Delta \subset \mathbb{C}^{n \times n}
\]

这个包含关系意味着:
\begin{itemize}
    \item 标量不确定性是最受限制的情况
    \item 完全不确定性是最不受限制的情况
    \item 一般的结构不确定性介于两者之间
\end{itemize}

因此谱半径和最大奇异值给出了 $\mu$ 的下界和上界:

\[
\rho(M) \leq \mu_{\Delta}(M) \leq \bar{\sigma}(M)
\]

\subsection{为什么这个引理重要?}

\subsubsection{理论意义:}
\begin{enumerate}
    \item \textbf{连接了不同概念}:将结构奇异值与谱半径、奇异值联系起来
    \item \textbf{提供了计算框架}:虽然直接计算 $\mu$ 很困难,但这个引理提供了上下界
    \item \textbf{揭示了本质}:$\mu$ 衡量的是"在最坏结构不确定性下的最大增益"
\end{enumerate}

\subsubsection{实际应用:}
\begin{enumerate}
    \item \textbf{鲁棒稳定性分析}:如果 $\mu_{\Delta}(M) < 1$,系统对所有允许的不确定性鲁棒稳定
    \item \textbf{控制器设计}:设计控制器使 $\mu$ 最小化
    \item \textbf{性能鲁棒性}:类似的框架可用于分析性能鲁棒性
\end{enumerate}

\subsection{计算考虑}

虽然这个引理给出了 $\mu$ 的表达式,但实际计算仍然困难:
\begin{itemize}
    \item 右边的最大化是非凸优化问题
    \item 对于一般结构 $\Delta$,计算 $\mu$ 是 NP-hard 问题
    \item 实践中使用上下界逼近:
    \begin{itemize}
        \item 下界:通过幂迭代等方法估计
        \item 上界:通过求解 LMI 等问题得到
    \end{itemize}
\end{itemize}

\subsection{总结}

这个引理成立的原因是:
\begin{enumerate}
    \item \textbf{$\mu$ 本质上衡量的是最坏情况下的环路增益}
    \item \textbf{谱半径 $\rho(M\Delta)$ 直接关联到稳定性边界}
    \item \textbf{最大化谱半径等价于找到最危险的不确定性方向}
    \item \textbf{极端情况对应着不确定性结构的极端限制}
\end{enumerate}

这个结果奠定了 $\mu$ 分析的理论基础,使得我们能够系统性地分析结构化不确定性对系统稳定性的影响。





\end{document}