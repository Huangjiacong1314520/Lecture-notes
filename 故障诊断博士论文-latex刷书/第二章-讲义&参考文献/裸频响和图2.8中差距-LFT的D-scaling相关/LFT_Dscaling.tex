% \documentclass{article}





% This LaTeX document needs to be compiled with XeLaTeX.
\documentclass[10pt]{article}

\usepackage{amsmath}
\usepackage{amssymb}
\usepackage{graphicx}
\usepackage{float}
\usepackage{lipsum}


\usepackage[utf8]{inputenc}
\usepackage{graphicx}
\usepackage[export]{adjustbox}
\usepackage{hyperref}
\hypersetup{colorlinks=true, linkcolor=blue, filecolor=magenta, urlcolor=cyan,}
\urlstyle{same}
\usepackage{amsmath}
\usepackage{amsfonts}
\usepackage{amssymb}
\usepackage[version=4]{mhchem}
\usepackage{stmaryrd}
\usepackage{caption}
\usepackage{bbold}
\usepackage{fixltx2e}
\usepackage[fallback]{xeCJK}
\usepackage{polyglossia}
\usepackage{fontspec}
\usepackage{newunicodechar}





\begin{document}

\section*{问题分析与解答}

\subsection*{第一段内容}

你看到的这个“幅值差了几十 dB”的现象,不是缩放问题,而是比的对象不一样:

\textbf{你画的是}
\[ \tilde{G}_d(s, 0) = 7 \cdot 0.25 \, G_1 + 3 \cdot 1 \, G_2 + 3 \cdot 0.75 \, G_3 \]
(或者把 $\delta_1$, $\Delta_1$ 代进去后的裸频响的 Bode 图)。

\textbf{论文图 (Fig. 2.8)} 画的是
\[ \tilde{G}_{d,\text{init}}(s) = \mathbf{F}_u(D(s), \, \text{blkdiag}(\delta_1, \Delta_1(s))), \]
并与“下界” $L(\omega)$ 以及“插值模型”对比。这里的 $\mathbf{F}_u(\cdot)$ 是 \textbf{上线性分式变换 (upper LFT)},$D(s)$ 是 skewed-$\mu$ 幂迭代 (或 D-K) 得到的频率依赖的 D-scaling。这一项会把幅值成百上千倍地重塑 (+20~+60 dB 很常见)。再加上有时还会乘以前后权重 $W_o$, $W_i$ 去做 $\mu$ 规格化,最终曲线当然会让“裸的 $\tilde{G}_d$”高很多。

更具体地说:

\begin{enumerate}
    \item $\Delta_1(s)$ 是“全通”型 (all-pass)
    
    你截图给出的
    \[ \Delta_1(s) = \frac{s^2 - 2.388 \times 10^4 s + 3.069 \times 10^6}{s^2 + 2.388 \times 10^4 s + 3.069 \times 10^6} \]
    是单位增益的全通块 ($|\Delta_1(j\omega)| = 1$)。再乘上 $1 + W(s)\Delta_1(s)$ 也只能带来个位数 dB 的变化 ($W$ 的横值在 0.1~0.2 左右),绝不可能凭它把曲线抬到 +20 dB。
    
    所以真正把曲线抬高几十 dB 的,是 $D(s)$ (和可能的 $W_o$, $W_i$),而不是 $\Delta_1$。
    
    \item 论文图是“被 D-scaling 后的系统”,你的是“未 scaling 的名义系统”
    
    Fig. 2.8 的文字里明确写了 “$\tilde{G}_{d,\text{init}}(s) = \mathbf{F}_u(D(s), \, \text{blkdiag}(\delta_1, \Delta_1(s)))$ has the largest gain ...”。
    
    这就是说他们先用 skewed-$\mu$ 幂迭代 (或 D-K) 在每个频率点求出频率依赖的放大器 $D(j\omega)$,再把它作为上 LFT 的“外环”,生成“边界模型”。这个 $D(s)$ 的作用就是把在 $\mu$ 意义下“危险”的方向大幅放大,幅值自然能到 +20 dB 甚至更高。你代码里没这一步,当然对不上。
    
    \item 坐标轴单位也不同 (Hz vs rad/s),但这只影响横坐标位置
    
    论文横轴标的是 Hz,MATLAB bode 默认是 rad/s。这会导致峰位看起来“挪了 $2\pi$ 倍”,但不会造成你说的“幅值差几十 dB”。幅值的大差距还是第 2 点的原因。
\end{enumerate}

\subsection*{第二段内容}

我画的确实是裸频啊,可是图中的标称系统 The nominal system is shown “$G_d(s, 0)$”不就是裸频啊吗,也就是图中蓝色的线条,为什么也那么高?它不是不含不确定性吗?标称也可以接到上 LFT 里吗?

非常好的追问——这正是很多人第一次读 Claessens (2024) 这类 $\mu$-分析论文时会混淆的地方。我们来把“标称系统 $ G_d(s, 0) $”的删除蓝线为什么也会在 +10 ~ +20 dB 这一量级讲清楚。

\begin{enumerate}
    \item “标称系统”并不是原始的 $ 7G_1 + 3G_2 + 3G_3 $
    
    文中的 $ G_d(s, 0) $ 已经不是最初那三个基础环节的直接线性组合,而是经过整个 LFT 封装后提取出来的名义通道。
    
    也就是说:
    \[ G_d(s, \Delta) = F_u(D(s), \text{blkdiag}(\delta_1, \Delta_1(s))) \]
    
    当 $\Delta = 0$ 时得到的 $ G_d(s, 0) $(蓝线)仍然处在 LFT 结构里,包含了 $D(s)$ 这一串联放大器的效应。
    
    在 $\mu$-框架下,$ D(s) $ 被设计为频率依赖的标度矩阵(scaling),它的模值可能在某些频段大于 10 ~ 30 倍。
    
    因此即便不含不确定性块,名义系统也会被 $ D(s) $ “抬高”几十 dB。
    
    \item 为什么要这样定义“标称”
    
    在 $\mu$-综合或 skewed-$\mu$ 算法代算法中,我们的目标是找到一个 $ D(s) $,使得:
    \[ \mu_\Delta(D(j\omega) G_d(j\omega) D^{-1}(j\omega)) = 1 \]
    
    在这个过程中,D 相当于一个频率依赖的放大因子,让系统在“$\mu = 1$”的边界上衡量增益。
    
    然后再把 D 固定下来,连同 G 一起形成 $ G_d $。所以论文里的 “nominal $ G_d(s, 0) $”其实是经 D-scaling 后的标称模型,而不是你手里未缩放的“原始标称系统”。
    
    \item 蓝线为什么还叫“nominal”
    
    因为:
    \begin{itemize}
        \item 它对应 $\delta_1 = 0, \Delta_1 = 0$,即不含不确定性扰动;
        \item 但仍保留了设计时的标度结构 $ D(s) $;
        \item 这样便于和“最坏情况”(橙线)或“插值解”(紫线)进行统一比较。
    \end{itemize}
    
    论文作者通常把这种“已经被入 D(s) 的版本”称为 nominal,以区分于原始未缩放模型。
\end{enumerate}

\end{document}