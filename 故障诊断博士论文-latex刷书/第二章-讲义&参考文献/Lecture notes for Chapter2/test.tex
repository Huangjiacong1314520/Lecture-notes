[cite_start]Remark 14 详细解释了在**多输出系统 (MIMO, $n_y > 1$)** 的**鲁棒故障检测滤波器 (Robust Fault Detection Filter)** 设计中,用于约束不确定性影响的**条件 (42)** 难以验证的原因 [cite: 1535]。

### 1. 条件 (42) 的背景与含义

条件 (42) 为:
$$\tilde{G}_d(j\omega, \Delta) \tilde{G}_d^H(j\omega, \Delta) \leq \bar{G}_d(j\omega) \bar{G}_d^H(j\omega) \quad \text{(42)}$$
* [cite_start]**目的:** 该条件是确保包含不确定性的干扰模型 $\tilde{G}_d(s, \Delta)$,其对残差信号的影响能够被一个**固定的上界模型** $\bar{G}_d(s)$ 所完全包络 (或捕获) [cite: 1508, 1522, 1624]。
* **鲁棒性:** 满足此条件是设计**后置滤波器 $R(s)$** 的关键,以保证对于所有不确定性 $\Delta$,残差信号对干扰的敏感度始终低于设计阈值 $\gamma$,即 $\left\| [cite_start]R(s) \tilde{G}_d(s, \Delta) \right\|_\infty \leq \gamma$ [cite: 1535]。

### 2. 单输出系统 (SISO/MISO) 的特殊性

[cite_start]对于**单输出系统** ($n_y = 1$),例如 MISO (多输入单输出) 系统,**奇异值检查是足够的** [cite: 1535]。
* [cite_start]**充分条件:** 此时,只需验证最大奇异值满足 $\sigma_1( \tilde{G}_d(j\omega, \Delta)) \leq \sigma_1( \bar{G}_d(j\omega))$ 即可 [cite: 1535]。
* **原因:** 对于单输出系统,输出是一个标量,其不确定性空间(即“椭圆”)退化为一个线段(即幅值)。此时,最大的奇异值 $\sigma_1$ **完全代表了不确定性对输出的影响大小**,因此直接比较大小就足以确保包络关系成立。

### 3. 多输出系统的核心问题:奇异值不能直接比较的原因

[cite_start]对于**多输出系统** ($n_y > 1$),仅比较奇异值 **不具备充分性**,因为它**没有考虑图 8 中椭圆的旋转** [cite: 1535]。

#### 几何意义(椭圆的旋转)
* [cite_start]在多输出系统中,一个传输矩阵 $A(j\omega)$ 的奇异值 $\sigma_i$ 描述了其在频域上的**增益大小**,即其输出超椭球的**半轴长度** [cite: 1511, 1525]。
* [cite_start]该矩阵的左奇异向量(Left Singular Vectors)描述了输出超椭球在输出空间中的**方向**或**旋转** [cite: 1511, 1525]。
* [cite_start]**问题所在:** 随着不确定性 $\Delta$ 的变化,**不仅不确定系统的奇异值 ($\tilde{G}_d$ 椭圆的大小) 会改变,其左奇异向量 (椭圆的方向/旋转) 也会改变** [cite: 1535]。

#### 实际影响
* [cite_start]直接比较 $\sigma_1( \tilde{G}_d(j\omega, \Delta)) \leq \sigma_1( \bar{G}_d(j\omega))$ **只保证**不确定性椭圆的**最长半轴**没有超过上界椭圆的最长半轴 [cite: 1535]。
* [cite_start]**但是,**如果 $\tilde{G}_d(j\omega, \Delta)$ 的椭圆**旋转**了一个角度,即使它的 $\sigma_1$ 小于 $\bar{G}_d$ 的 $\sigma_1$,该旋转后的椭圆仍然有可能**部分超出了** $\bar{G}_d$ 所定义的固定椭圆边界 [cite: 1508, 1522]。
* **后果:** 一旦发生这种情况,条件 (42) 就会失效,进而可能导致最终设计的滤波器 $R(s)$ 无法满足鲁棒性要求 $\left\| [cite_start]R(s) \tilde{G}_d(s, \Delta) \right\|_\infty \leq \gamma$,从而产生**不希望的结果**(例如,残差中出现大量干扰泄漏,导致误报) [cite: 1535]。

### 4. 论文中采用的保守方法

[cite_start]由于缺乏针对多输出系统的自动检查算法和指导方针,论文采用了**保守的方法**来构造上界模型 $\bar{G}_d(s)$ [cite: 1535]。
* [cite_start]**方法:** 仅通过补偿**最大奇异值** $\sigma_1(\tilde{G}_d(s, \Delta))$ 来构造 $\bar{G}_d(s)$ [cite: 1535]。
* [cite_start]**结果:** 这种方法使得上界模型 $\bar{G}_d(s)$ 具有一个**对角线后置滤波器**,且奇异值相等 [cite: 1535][cite_start]。在几何上,这相当于在图 8 中,用一个能够捕获所有不确定性椭圆的**圆形**(而不是紧凑的椭圆)作为上界 [cite: 1535]。
* [cite_start]**意义:** 圆形在所有方向上都具有相等的增益,因此能够**无条件地包络**所有可能旋转的不确定性椭圆,从而满足约束条件,代价是设计结果更加保守(即留有更大的裕度) [cite: 1535]。