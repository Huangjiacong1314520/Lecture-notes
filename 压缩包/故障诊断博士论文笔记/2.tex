\chapter{不确定系统的闭环最优故障检测}\label{ch:environment}
注意只是检测不是诊断,因此只是包含故障检测,并不包含分离和隔离,更不包括检测出故障之后
怎么对故障进行处理。
《2024博士论文》
Both data-driven methods [61, 94, 137] and 
model-based methods [35, 63, 94, 137, 262] have shown considerable progress.
本章是基于观测器的方法[35, 79],属于基于模型的方法。
在这些方法的基础上,已经开发出用于故障检测与隔离(FDI)的综合策略[35,63,262]。
[35]J. Chen and R. J. Patton, Robust Model-Based Fault Diagnosis for Dynamic Systems, Boston, MA: Springer US, 1999, pages 1089–1091.


%——————————————————————————————————%

\section{研究对象}

\subsection{问题}
开环/闭环配置下 ||  连续时间 || 线性时不变 || 不确定系统 的 鲁棒故障检测滤波器设计问题
\subsubsection{为什么要研究这个问题}

为什么要单独研究不确定性系统?因为传统方法都是针对标称系统进行故障诊断的(如Steven x Ding)
而对包含不确定性的系统,是否故障诊断会没有区别?
在不考虑不确定性时,开环和闭环的残差生成结果是一直的(见Ding的书,7.9)
那么在含不确定性情况下,开环和闭环是否会有区别?
在这里并没有提到光刻机等应用场景,因此这应该是普适的问题,对应普适的解决方法。

\subsubsection{为什么研究光刻机的故障诊断问题需要研究这样一个问题?}
现实世界系统中固有的可变性和不确定性,导致系统一定会发生故障,那问题就在于何时会发生故障。

\subsubsection{研究这样一个问题有什么意义}
及时检测和识别故障对于降低性能下降、损坏和对人员安全威胁的风险至关重要。
从诊断系统中获得的知识可用于优化维护计划,从而减少风险和停机时间。
在此背景下,有效的故障检测方法在确保复杂工程系统的可靠性和安全性方面起着至关重要的作用。

\subsection{故障检测的关键挑战}
故障检测(FD)的关键挑战在于区分故障与未知扰动。

\subsection{基于模型的故障诊断方法总结与存在的问题}
基于模型的故障诊断系统要取得令人满意的性能,需要在故障敏感性和扰动抑制之间取得微妙的平衡[62]。
\subsubsection{针对LTI系统的故障诊断方法}
基于因式分解的技术[62,142],这些技术通常通过求解黎卡提方程来实现[157]。
此外,还采用了H−/H∞技术,并运用线性矩阵不等式(LMI)综合法[134,156,248,270]。
如果扰动和系统模型完全已知,这些方法在使残差对故障尽可能敏感的意义上是最优的。
\subsubsection{用于解决建模不确定性问题的鲁棒方法}
许多方法利用$$H/infinit$$准则,并通过$$miu$$综合对其进行优化[216,242]。
其他方法涉及H模型匹配技术,通过线性矩阵不等式(LMI)优化来解决[285]。
此外,还采用H-/H无穷准则,并通过LMI解决方案来处理[125,269],有时会与μg分析相结合[178]。
\subsubsection{方法的缺陷(待理解)}
与针对无不确定性的线性时不变系统的方法[62,157,248]不同,
现有文献中的鲁棒方法往往要么过于保守,要么应用复杂。
例如,仅依赖H准则的方法不能直接考虑H故障灵敏度,而必须进行后验分析。
模型匹配技术通常无法保证H故障灵敏度方面的最优性,
因为其有效性在很大程度上取决于参考模型。
此外,这些方法通常难以扩展用于故障隔离目的。
一般而言,许多故障诊断方法是针对开环系统开发的,因此并不适用于闭环系统。
通过一个激励示例表明,考虑闭环对于获得可靠结果至关重要。


\subsection{设计目标(同时也是故障检测的关键挑战)}
设计目标:将故障与扰动、固有建模不确定性区分开

\subsection{方法}
本章是基于观测器的方法
\subsubsection{老实说,没有很理解,需要对每个名词、动作进行释义以及解释作用}
该解决方案基于不确定性和扰动模型的上界,通过单个黎卡提方程求解,实现了扰动抑制、建模不确定性与故障灵敏度之间的最优折衷。
基于最坏情况扰动和不确定性模型,通过求解单个Riccati方程,
提供了一种统一的方法来处理参数和动态不确定性。
这种最坏情况模型是通过非线性优化和边界Nevanlinna-Pick方法的应用获得的。
\subsubsection{上界模型-相关参考文献}、
上界模型源于具有混合不确定性系统的最坏情况增益分析,其中混合不确定性既包含动态不确定性,也包含参数不确定性。已知该问题是NP难问题[26]。为解决此问题,可采用倾斜μ幂迭代[12,129,212]和基于D-G标度的凸优化[12,130,188,212]等技术计算下界和上界。倾斜μ幂迭代使用启发式方法确定特定频率下对应最坏情况下界的参数或复矩阵值。通过插值法可构建稳定的线性时不变(LTI)样本[286]。基于这一概念,[195,196]中提出的方法构建了在多个频率上最大化增益的最坏情况混合不确定性样本,该方法利用非线性优化和边界Nevanlinna-Pick(BNP)插值[14]。这种方法生成一个稳定、范数有界的LTI不确定性样本,该样本对一组矩阵样本进行插值,为不确定性和扰动模型提供最坏情况下的上界。

\subsection{如何验证:实验}
下一代光刻用掩膜板工作台原型
(最终不能脱离实际,还是要回到光刻机上来,不然就是纯为了做研究而做研究了)

\subsection{结论/效果}
结果表明,该方法在故障敏感性与对建模不确定性和扰动的抑制之间取得了最优折衷。
这种能力能够在残差中清晰地区分故障和不良影响,从而提高故障检测的可靠性,最终有助于提升安全性和性能。






%——————————————————————————————————%
