% 测试跨页功能
\documentclass[12pt]{article}
\usepackage{researchnotes}
\usepackage{lipsum}  % 用于生成测试文本

\begin{document}

\section{跨页测试}

这是正文内容。

% 测试1:短笔记(不跨页)
\begin{rn}[author=JC, date=2025-10-20]
\textbf{短笔记测试}

这是一个简短的笔记,应该不会跨页。
\end{rn}

% 测试2:长笔记(会跨页)
\begin{rn}[author=JC, date=2025-10-20]
\textbf{长笔记测试 - 会跨页}

\lipsum[1-5]  % 生成5段测试文本

这是一些数学公式:

当\hli{n \to \infty}时,我们有:
\[
\hlm{\lim_{n \to \infty} \frac{1}{n} = 0}
\]

\lipsum[6-10]  % 再生成5段测试文本

\textbf{重要结论:}
\begin{itemize}
    \item 第一点:\hl{这很重要}
    \item 第二点:\hlr{这非常重要}
    \item 第三点:\hlg{这是补充说明}
\end{itemize}

\lipsum[11-12]  % 继续生成文本
\end{rn}

% 测试3:疑问笔记(也会跨页)
\begin{qn}[author=张三, date=2025-10-20]
\textbf{这是一个很长的疑问}

\lipsum[1-3]

\textbf{数学推导:}

假设\hli{x > 0},则:
\[
\hlmr{\int_0^x f(t)dt = F(x) - F(0)}
\]

\lipsum[4-6]

\textbf{待解决的问题:}
\begin{enumerate}
    \item \hlp{问题1}:如何证明?
    \item \hlp{问题2}:是否有反例?
    \item \hlp{问题3}:能否推广?
\end{enumerate}

\lipsum[7-8]
\end{qn}

\section{后续内容}

笔记已经结束,这是正文的后续内容。

\end{document}