\chapter{问题背景}

%——————————————————————————————————%

\section{问题背景与现有解决方案的缺陷-时变系统辨识}


想象一下,你要教一个机器人反复完成同一个动作,比如在传送带上抓取一个盒子。

“时不变”系统(简单情况): 如果每次盒子的重量、位置都一模一样,机器人很容易通过几次尝试就学会最佳动作。

“时变”系统(复杂情况): 但如果传送带上的盒子时轻时重(参数在变化),机器人上次学会的动作,这次可能就不适用了。这就是一个 “时变系统”。

时变系统辨识领域的两个亟待解决的问题:精度和跟踪速度。

\subsection{传统方法存在的问题-最小二乘与遗忘因子}

传统方法的困境:
最小二乘法:
其收敛速度较低, 估计值无法及时跟踪真实值的变化。
科学家们常用一种叫“最小二乘法”的工具来帮机器“学习”系统参数。它就像是一个“金鱼脑”的学徒,只记得最近一点信息,然后就把之前的都忘了。对于变化的系统,它学得慢,而且总是“慢半拍”,等它学会,系统早就变了。

引入遗忘因子:
无法完全跟踪时变参数的变化,存在估计延迟。
有人提出了“遗忘因子”:让这个学徒忘得更快一点,以便更快地学习新东西。但这又带来了新问题:忘得太快,就容易把噪声(测量误差)也当成了真知识,结果学得不准。

\newpage


%——————————————————————————————————%

\section{本文引入的解决方法——迭代轴}


论文的核心思想非常巧妙,它引入了一个 “迭代轴” 的概念。我们把它理解为 “时空穿梭”学习法。

时间轴 (k): 我们熟悉的时间流逝,比如从第1秒到第40秒。

迭代轴 (j): 系统重复运行的次数。比如让机器人反复执行40秒的抓取任务,第一次运行是j=1,第二次是j=2...

关键假设(这个方法的基石):

尽管盒子的重量随时间(k)在变,但在每天的同一时刻,盒子的重量是一样的。
比如,每天上午10:00整,传送带上的盒子总是1公斤重。

这意味着什么?
我们把一个在时间轴上变化的参数,映射到了迭代轴上,变成了一个在同一时刻下不变的参数。这样,我们就可以在“同一个时刻点”上,通过反复迭代(多次实验),来学习这个“不变”的参数。

比喻:
研究一个每天都在重复的、有固定剧本的舞台剧。虽然剧情(时间轴)在推进,但每天(迭代轴)的晚上8点15分,主角都会说同一句台词。我们的方法就是反复观看不同天的“晚上8点15分”这一时刻,来精确学习这句台词是什么。

\subsection{方法根本}
将时变系统辨识问题转化为迭代学习问题


\subsection{对于该方法的疑问}
机械臂不可能一天都是固定剧本吧?那么多不同重量的货物,不可能每天都按照固定顺序去排列夹取吧?

%——————————————————————————————————%

\section{系统模型与假设}

\subsection{系统模型}

\[  y(k)= \frac{B(k,z)}{A(k,z)}u(k)+v(k)    \]
其中

\subsection{关键假设}

迭代不变性:同一时刻的参数在不同迭代中保持不变。

噪声特性:v(k) 是白噪声,与输入不相关。

系统阶次已知,参数有界。

\begin{enumerate}
    \item 迭代不变性:同一时刻的参数在不同迭代中保持不变。
    \item 噪声特性:v(k) 是白噪声,与输入不相关。
    \item 系统阶次已知,参数有界。
\end{enumerate}


%——————————————————————————————————%

\section{核心算法设计-三件套}
\subsection{代价函数设计-基于二次型优化的迭代学习辨识}
每次迭代学习时,代价函数的目标:
\begin{enumerate}
    \item 估计误差平方和:输出误差要小
    \item 参数估计值的变化量:参数变化惩罚项,增强鲁棒性(我这次估算的参数,和上次的参数不能差别太大)
\end{enumerate}
通过平衡这两个目标,学徒既能紧跟系统变化,又不会因为一点噪声就疑神疑鬼、大幅修改参数,变得非常稳健。

\subsection{协方差矩阵-基于SVD的协方差矩阵更新}
考虑到舍入误差
以及输入信号非持续激励等因素可能会导致协方差
矩阵失去正定性甚至奇异,导致估计误差变大甚至
算法不收敛,本文提出了一种基于奇异值分解
(SVD)的协方差矩阵迭代更新方法,提高了算法
的数值稳定性。
学徒脑子里有一个“记忆矩阵”(协方差矩阵 P_j(k)),用来记录它学到了多少知识。在传统方法里,这个矩阵在更新时可能会出问题,比如出现“负数记忆”,导致计算崩溃(数值不稳定)。

论文的解决方案是使用 SVD(奇异值分解)。

你可以把它想象成一种“标准化记忆法”。就像我们把杂乱的书房按照“重要性”和“类别”重新整理一遍。

每次更新记忆时,都用SVD这个“标准化流程”来整理,确保记忆矩阵永远是“健康、正面的”,从而保证了整个计算过程的稳定性,不会算着算着就崩溃。

\subsection{噪声影响-偏差补偿方法}

由于理论分析得到该方法是有偏的,因此需要进行补偿
由于测量中永远存在噪声,学徒的估计值会有一个固定的偏差(就像照片永远偏黄一点)。

考虑到噪声的存在会使参数
估计值与真实值之间存在偏差,本文设计了一种偏
差补偿方法,提高了时变参数的估计精度。

论文的解决方法是:

诊断偏差: 首先从数学上证明,这个偏差的大小与噪声的强度 (σ²) 成正比。

估计噪声: 利用那个“小本本”(代价函数)的值,反过来估算出噪声的强度。

实施补偿: 最后,从有偏差的估计结果中,减掉这个计算出来的偏差量。

经过这一步“美颜”,参数的估计结果就变得更加精确、无偏了。

\subsubsection{偏差分析}
\subsubsection{偏差补偿项}
\subsubsection{噪声方差估计}
修炼三:偏差补偿(给结果“美颜”)


%——————————————————————————————————%

\section{仿真验证}
包括应用场景验证和其他方法对比
\subsection{例1:慢变与快变参数跟踪}
为什么要这么验证:针对应用场景的?
系统参数包括正弦、线性、分段常数等多种变化形式。

结果显示:

偏差补偿显著提高估计精度;

SVD更新提升数值稳定性;

算法能完全跟踪参数变化,无延迟。
\subsection{例2:与FFBCRLS方法对比}
FFBCRLS(带遗忘因子的偏差补偿递推最小二乘)存在明显估计延迟;

本文方法由于引入迭代轴,无延迟,估计误差更小;

迭代方法在有限时间系统中具有优势,但计算量较大。


%——————————————————————————————————%







