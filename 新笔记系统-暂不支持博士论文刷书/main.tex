% main.tex - 使用示例
\documentclass[12pt]{article}

% 中文支持
\usepackage[UTF8]{ctex}
\usepackage{amsmath, amssymb}

% 导入笔记宏包
\usepackage{researchnotes}

% 如果想隐藏所有笔记,取消下面一行的注释
% \hidenotes

% 如果之前隐藏了笔记,想重新显示,使用
% \shownotes

\begin{document}

\section{示例章节}

\hl{这是正文内容。下面是一个研究笔记的例子。}这里是不支持中文高亮的表现,返回看代码你就会发现左边是中文。
\hl{important} 

\begin{researchnote}[author=JC, date=2025-10-16, class=]
\textbf{包含关系:}
\[
Q_d \subset \{\tilde{\Delta}_d(j\omega) : \|\tilde{\Delta}_d(s)\|_\infty \leq 1\}
\]

但关键是:在计算最坏情况增益时,这个包含关系实际上是"充分"的。

\textbf{等价性定理:}

对于最坏情况增益计算,有:
\[
\sup_{\tilde{\Delta}_d \in \Delta_d} \sigma(\mathcal{F}_u(D(j\omega), \tilde{\Delta}_d(j\omega))) 
= \sup_{Q_d \in Q_d} \sigma(\mathcal{F}_u(D(j\omega), Q_d))
\]

这意味着:
\begin{itemize}
\item 虽然 $Q_d$ 是 $\tilde{\Delta}_d(j\omega)$ 的一个子集
\item 但在寻找最大增益时,我们不会丢失任何信息
\item 因为本质上这个 $Q_d$ 就是所有最坏增益对应的不确定性的集合
\end{itemize}
\end{researchnote}

medhighmedlow

\begin{questionnote}[author=JC, date=2025-10-16, class=]
如何找到这个最坏增益对应的不确定性的集合 $Q_d$?其实本质上也就是找 $Q_d$ 中的各个元素 $Q_j$
\end{questionnote}

\subsection{Example 2.24}

Consider the uncertain system
\begin{align*}
\tilde{G}_d(s,\Delta) &= (7G_1(s)(0.25+\delta_1) + 3G_2(s)(1-\delta_1) \\
&\quad + 3G_3(s)(0.75-\delta_1))(1+W(s)\Delta_1(s))
\end{align*}

with $G_1 = \frac{1}{(s+2)^2}$, $G_2 = \frac{10\pi s}{(s+200\pi)^2}$, and $G_3 = \frac{100\pi s}{(s+200\pi)^2}$, dynamic weight $W(s) = 0.2\frac{s+10\pi}{s+0.1\pi}$, and uncertain parameters $\delta_1 \in \mathbb{R}$ with $|\delta_1| \leq 1$, and $\Delta_1(s)$ with $\|\Delta_1\|_\infty \leq 1$.

% 更多示例笔记
\begin{researchnote}[author=张三, date=2025-10-18]
这是另一个研究笔记的例子,展示了如何使用不同的作者和日期。

可以包含:
\begin{itemize}
\item 数学公式
\item 列表
\item 各种文本格式
\end{itemize}
\end{researchnote}





% 高亮使用示例
\section{高亮功能演示}

\subsection{1. 文本高亮(中英文)}

\begin{rn}
\textbf{不同颜色的文本高亮:}

\begin{itemize}
    \item 这是\hl{黄色高亮(默认)}的文字
    \item 这是\hlr{红色高亮}的文字,用于标注错误或重要警告
    \item 这是\hlg{绿色高亮}的文字,用于标注正确或补充
    \item 这是\hlb{蓝色高亮}的文字,用于标注定义或概念
    \item 这是\hlo{橙色高亮}的文字,用于标注需要注意
    \item 这是\hlp{粉色高亮}的文字,用于标注问题
    \item 这是\hlgr{灰色高亮}的文字,用于标注次要信息
\end{itemize}

\textbf{英文高亮示例:}

This is \hl{important} information. The \hlr{critical error} needs attention.
\end{rn}

\subsection{2. 行内数学公式高亮}

\begin{rn}
\textbf{行内公式高亮:}

当\hli{n \to \infty}时,级数\hlig{\sum_{i=1}^n \frac{1}{i^2}}收敛到\hlib{\frac{\pi^2}{6}}。

这个不等式\hlir{x^2 + y^2 \geq 2xy}是\hlio{柯西不等式}的特殊情况。

注意:\hlip{\delta > 0}这个条件很重要!
\end{rn}

\subsection{3. 独立数学公式高亮}

\begin{qn}
\textbf{如何证明欧拉公式?}

欧拉公式:\hlm{e^{i\pi} + 1 = 0}

泰勒展开:
\[
\hlmg{e^{ix} = \sum_{n=0}^{\infty} \frac{(ix)^n}{n!} = \cos x + i\sin x}
\]

当\hlmo{x = \pi}时,得到:
\[
\hlmr{e^{i\pi} = \cos\pi + i\sin\pi = -1}
\]

因此:\hlmb{e^{i\pi} + 1 = 0}
\end{qn}

\subsection{4. 混合使用示例}

\begin{rn}[author=张三, date=2025-10-20]
\textbf{包含关系:}

对于\hl{最坏情况分析},我们有\hli{Q_d \subset \{\tilde{\Delta}_d(j\omega)\}}。

\textbf{关键定理:}

\hlr{等价性定理}告诉我们:
\[
\hlm{\sup_{\tilde{\Delta}_d \in \Delta_d} \sigma(\mathcal{F}_u) = \sup_{Q_d \in Q_d} \sigma(\mathcal{F}_u)}
\]

这意味着:
\begin{itemize}
    \item 虽然\hli{Q_d}是\hlg{真子集}
    \item 但在\hlo{寻找最大增益}时,\hlb{不会丢失信息}
    \item 因为\hlp{$Q_d$就是所有最坏情况的集合}
\end{itemize}
\end{rn}

\subsection{5. 复杂公式高亮}

\begin{rn}
考虑不确定系统:
\[
\tilde{G}_d(s,\Delta) = \hlm{(7G_1(s)(0.25+\delta_1) + 3G_2(s)(1-\delta_1))}
\]

其中传递函数定义为:
\begin{align*}
G_1 &= \hlmg{\frac{1}{(s+2)^2}} \\
G_2 &= \hlmb{\frac{10\pi s}{(s+200\pi)^2}}
\end{align*}

不确定参数满足\hlir{|\delta_1| \leq 1}和\hlio{\|\Delta_1\|_\infty \leq 1}。
\end{rn}

\subsection{6. 实用技巧}

\begin{qn}
\textbf{高亮使用建议:}

\begin{enumerate}
    \item \hlr{红色}:错误、警告、最重要的内容
    \item \hlo{橙色}:需要注意、有疑问的地方
    \item \hl{黄色}:一般重要内容(默认)
    \item \hlg{绿色}:补充说明、正确的答案
    \item \hlb{蓝色}:定义、概念、术语
    \item \hlp{粉色}:问题、待解决的事项
    \item \hlgr{灰色}:次要信息、参考资料
\end{enumerate}

\textbf{注意事项:}
\begin{itemize}
    \item 不要过度使用高亮,\hlr{太多高亮等于没有高亮}
    \item 保持\hlg{同一种颜色表示同一类信息}
    \item 对于\hlb{长公式},只高亮\hlo{关键部分}
\end{itemize}
\end{qn}

\end{document}