\documentclass[cn,10pt,math=newtx,citestyle=gb7714-2015,bibstyle=gb7714-2015]{0-figure/elegantbook}

\usepackage{CJKfntef}
\usepackage{amssymb}
\usepackage{array}
\usepackage[utf8]{inputenc}
%\usepackage{caption}
\usepackage{chemformula}
\usepackage{svg}
\usepackage{extarrows}
%\usepackage{0-figure/extrarelation}
\usepackage{tikz-3dplot}
% 加载你自己的笔记宏包
\usepackage{notes}  % 直接用 .sty 名字,不用写扩展名



% \usepackage[utf8]{inputenc}
% \usepackage{xcolor}
% \usepackage{tcolorbox}
% \usepackage[colorinlistoftodos]{todonotes}  % 保留用于列表

% % 重新定义笔记类型 - 使用美观柔和的色彩
% \newcommand{\researchnote}[1]{%
%     \begin{tcolorbox}[
%         colback=teal!8!white,
%         colframe=teal!50!black,
%         title=研究笔记,
%         fonttitle=\scriptsize\bfseries\color{teal!70!black},
%         coltitle=teal!70!black,
%         arc=3pt,
%         boxrule=0.8pt,
%         left=6pt,
%         right=6pt,
%         top=4pt,
%         bottom=4pt
%     ]
%         \scriptsize\color{gray!60!black} #1
%     \end{tcolorbox}%
% }

% \newcommand{\methodnote}[1]{%
%     \begin{tcolorbox}[
%         colback=blue!8!white,
%         colframe=blue!50!black,
%         title=方法笔记,
%         fonttitle=\scriptsize\bfseries\color{blue!70!black},
%         coltitle=blue!70!black,
%         arc=3pt,
%         boxrule=0.8pt,
%         left=6pt,
%         right=6pt,
%         top=4pt,
%         bottom=4pt
%     ]
%         \scriptsize\color{gray!60!black} #1
%     \end{tcolorbox}%
% }

% \newcommand{\resultnote}[1]{%
%     \begin{tcolorbox}[
%         colback=orange!8!white,
%         colframe=orange!45!brown,
%         title=结果笔记,
%         fonttitle=\scriptsize\bfseries\color{orange!60!brown},
%         coltitle=orange!60!brown,
%         arc=3pt,
%         boxrule=0.8pt,
%         left=6pt,
%         right=6pt,
%         top=4pt,
%         bottom=4pt
%     ]
%         \scriptsize\color{gray!60!black} #1
%     \end{tcolorbox}%
% }

% \newcommand{\futurework}[1]{%
%     \begin{tcolorbox}[
%         colback=violet!8!white,
%         colframe=violet!50!black,
%         title=后续工作,
%         fonttitle=\scriptsize\bfseries\color{violet!70!black},
%         coltitle=violet!70!black,
%         arc=3pt,
%         boxrule=0.8pt,
%         left=6pt,
%         right=6pt,
%         top=4pt,
%         bottom=4pt
%     ]
%         \scriptsize\color{gray!60!black} #1
%     \end{tcolorbox}%
% }

% % % 同时保留原始todo命令用于列表(可选)
% % \newcommand{\researchnotetodo}[1]{\todo[color=teal!25]{\scriptsize 研究: #1}}
% % \newcommand{\methodnotetodo}[1]{\todo[color=blue!25]{\scriptsize 方法: #1}}
% % \newcommand{\resultnotetodo}[1]{\todo[color=orange!25]{\scriptsize 结果: #1}}
% % \newcommand{\futureworktodo}[1]{\todo[color=violet!25]{\scriptsize 后续: #1}}
% \documentclass{article}
% \usepackage[utf8]{inputenc}
% \usepackage{xcolor}
% \usepackage{tcolorbox}
% \usepackage[colorinlistoftodos]{todonotes}

% % 改良学术风格配色
% \definecolor{notegreen}{RGB}{214,243,226}
% \definecolor{notegreenframe}{RGB}{46,139,87}

% \definecolor{noteblue}{RGB}{222,235,247}
% \definecolor{noteblueframe}{RGB}{23,74,117}

% \definecolor{noteyellow}{RGB}{255,249,219}
% \definecolor{noteyellowframe}{RGB}{193,141,0}

% \definecolor{notepurple}{RGB}{237,230,246}
% \definecolor{notepurpleframe}{RGB}{109,76,147}

% % 改进后的笔记环境
% \newcommand{\researchnote}[1]{%
%     \begin{tcolorbox}[colback=notegreen,colframe=notegreenframe,title=研究笔记,fonttitle=\scriptsize]
%         \scriptsize #1
%     \end{tcolorbox}%
% }

% \newcommand{\methodnote}[1]{%
%     \begin{tcolorbox}[colback=noteblue,colframe=noteblueframe,title=方法笔记,fonttitle=\scriptsize]
%         \scriptsize #1
%     \end{tcolorbox}%
% }

% \newcommand{\resultnote}[1]{%
%     \begin{tcolorbox}[colback=noteyellow,colframe=noteyellowframe,title=结果笔记,fonttitle=\scriptsize]
%         \scriptsize #1
%     \end{tcolorbox}%
% }

% \newcommand{\futurework}[1]{%
%     \begin{tcolorbox}[colback=notepurple,colframe=notepurpleframe,title=后续工作,fonttitle=\scriptsize]
%         \scriptsize #1
%     \end{tcolorbox}%
% }

% % 仍保留todo列表功能
% \newcommand{\researchnotetodo}[1]{\todo[color=notegreen]{\scriptsize 研究: #1}}
% \newcommand{\methodnotetodo}[1]{\todo[color=noteblue]{\scriptsize 方法: #1}}
% \newcommand{\resultnotetodo}[1]{\todo[color=noteyellow]{\scriptsize 结果: #1}}
% \newcommand{\futureworktodo}[1]{\todo[color=notepurple]{\scriptsize 后续: #1}}


\maketitle

\section{引言}
这是引言部分。

\researchnote{需要添加更多背景文献引用,特别是近三年的研究成果。}

\section{方法}
我们采用了先进的研究方法。

\methodnote{考虑与其他方法对比,特别是与传统方法的优缺点分析。}

\section{结果}
实验结果如图1所示。

\resultnote{异常数据点需要检查,可能由于设备校准问题导致。}

\section{讨论}
基于以上结果,我们得出重要结论。

\futurework{需要长期跟踪研究,建议设置6个月和12个月的随访点。}

% 显示所有笔记的汇总列表(如果需要)
\newpage
\section*{研究笔记汇总}
\listoftodos

% 默认:如果作者或日期留空,宏会自动补上 \today
\researchnote[JC][]{high}{需要补充最近三年的关键综述,建议查阅 IEEE T-... 的相关文章。}

\section{方法}
简要描述方法。

\methodnote[Li]{2025-02-01}[med]{建议在实验对比中加入基线 B,并说明参数选择理由。}

\section{结果}
实验结果如图所示。

\resultnote[][2025-02-02][high]{某些试验存在离群点,怀疑为设备标定问题,请复测并记录环境温度。}

\section{讨论}
讨论与结论。

\futurework[JC][\today][low]{长期随访:建议 6 个月、12 个月进行性能回归测试。}

% 也可以留空作者/日期,优先级可写成 high/med/low(大小写不敏感)
\researchnote{无需作者信息的快速备注示例,默认会标注为"作者: "并使用当天日期。}



\usetikzlibrary{cd}
\cover{0-figure/cover.jpg}
\logo{0-figure/science-and-mind.jpg}
\input{newcommand}

\title{LaTeX+Overleaf教程}
\author{Maki's Lab}
\date{\today}

\begin{document}
\maketitle
\frontmatter

\newpage

\tableofcontents
\mainmatter


\chapter{问题背景}

%——————————————————————————————————%

\section{问题背景与现有解决方案的缺陷-时变系统辨识}


想象一下,你要教一个机器人反复完成同一个动作,比如在传送带上抓取一个盒子。

“时不变”系统(简单情况): 如果每次盒子的重量、位置都一模一样,机器人很容易通过几次尝试就学会最佳动作。

“时变”系统(复杂情况): 但如果传送带上的盒子时轻时重(参数在变化),机器人上次学会的动作,这次可能就不适用了。这就是一个 “时变系统”。

时变系统辨识领域的两个亟待解决的问题:精度和跟踪速度。

\subsection{传统方法存在的问题-最小二乘与遗忘因子}

传统方法的困境:
最小二乘法:
其收敛速度较低, 估计值无法及时跟踪真实值的变化。
科学家们常用一种叫“最小二乘法”的工具来帮机器“学习”系统参数。它就像是一个“金鱼脑”的学徒,只记得最近一点信息,然后就把之前的都忘了。对于变化的系统,它学得慢,而且总是“慢半拍”,等它学会,系统早就变了。

引入遗忘因子:
无法完全跟踪时变参数的变化,存在估计延迟。
有人提出了“遗忘因子”:让这个学徒忘得更快一点,以便更快地学习新东西。但这又带来了新问题:忘得太快,就容易把噪声(测量误差)也当成了真知识,结果学得不准。

\newpage


%——————————————————————————————————%

\section{本文引入的解决方法——迭代轴}


论文的核心思想非常巧妙,它引入了一个 “迭代轴” 的概念。我们把它理解为 “时空穿梭”学习法。

时间轴 (k): 我们熟悉的时间流逝,比如从第1秒到第40秒。

迭代轴 (j): 系统重复运行的次数。比如让机器人反复执行40秒的抓取任务,第一次运行是j=1,第二次是j=2...

关键假设(这个方法的基石):

尽管盒子的重量随时间(k)在变,但在每天的同一时刻,盒子的重量是一样的。
比如,每天上午10:00整,传送带上的盒子总是1公斤重。

这意味着什么?
我们把一个在时间轴上变化的参数,映射到了迭代轴上,变成了一个在同一时刻下不变的参数。这样,我们就可以在“同一个时刻点”上,通过反复迭代(多次实验),来学习这个“不变”的参数。

比喻:
研究一个每天都在重复的、有固定剧本的舞台剧。虽然剧情(时间轴)在推进,但每天(迭代轴)的晚上8点15分,主角都会说同一句台词。我们的方法就是反复观看不同天的“晚上8点15分”这一时刻,来精确学习这句台词是什么。

\subsection{方法根本}
将时变系统辨识问题转化为迭代学习问题


\subsection{对于该方法的疑问}
机械臂不可能一天都是固定剧本吧?那么多不同重量的货物,不可能每天都按照固定顺序去排列夹取吧?

%——————————————————————————————————%

\section{系统模型与假设}

\subsection{系统模型}

\[  y(k)= \frac{B(k,z)}{A(k,z)}u(k)+v(k)    \]
其中

\subsection{关键假设}

迭代不变性:同一时刻的参数在不同迭代中保持不变。

噪声特性:v(k) 是白噪声,与输入不相关。

系统阶次已知,参数有界。

\begin{enumerate}
    \item 迭代不变性:同一时刻的参数在不同迭代中保持不变。
    \item 噪声特性:v(k) 是白噪声,与输入不相关。
    \item 系统阶次已知,参数有界。
\end{enumerate}


%——————————————————————————————————%

\section{核心算法设计-三件套}
\subsection{代价函数设计-基于二次型优化的迭代学习辨识}
每次迭代学习时,代价函数的目标:
\begin{enumerate}
    \item 估计误差平方和:输出误差要小
    \item 参数估计值的变化量:参数变化惩罚项,增强鲁棒性(我这次估算的参数,和上次的参数不能差别太大)
\end{enumerate}
通过平衡这两个目标,学徒既能紧跟系统变化,又不会因为一点噪声就疑神疑鬼、大幅修改参数,变得非常稳健。

\subsection{协方差矩阵-基于SVD的协方差矩阵更新}
考虑到舍入误差
以及输入信号非持续激励等因素可能会导致协方差
矩阵失去正定性甚至奇异,导致估计误差变大甚至
算法不收敛,本文提出了一种基于奇异值分解
(SVD)的协方差矩阵迭代更新方法,提高了算法
的数值稳定性。
学徒脑子里有一个“记忆矩阵”(协方差矩阵 P_j(k)),用来记录它学到了多少知识。在传统方法里,这个矩阵在更新时可能会出问题,比如出现“负数记忆”,导致计算崩溃(数值不稳定)。

论文的解决方案是使用 SVD(奇异值分解)。

你可以把它想象成一种“标准化记忆法”。就像我们把杂乱的书房按照“重要性”和“类别”重新整理一遍。

每次更新记忆时,都用SVD这个“标准化流程”来整理,确保记忆矩阵永远是“健康、正面的”,从而保证了整个计算过程的稳定性,不会算着算着就崩溃。

\subsection{噪声影响-偏差补偿方法}

由于理论分析得到该方法是有偏的,因此需要进行补偿
由于测量中永远存在噪声,学徒的估计值会有一个固定的偏差(就像照片永远偏黄一点)。

考虑到噪声的存在会使参数
估计值与真实值之间存在偏差,本文设计了一种偏
差补偿方法,提高了时变参数的估计精度。

论文的解决方法是:

诊断偏差: 首先从数学上证明,这个偏差的大小与噪声的强度 (σ²) 成正比。

估计噪声: 利用那个“小本本”(代价函数)的值,反过来估算出噪声的强度。

实施补偿: 最后,从有偏差的估计结果中,减掉这个计算出来的偏差量。

经过这一步“美颜”,参数的估计结果就变得更加精确、无偏了。

\subsubsection{偏差分析}
\subsubsection{偏差补偿项}
\subsubsection{噪声方差估计}
修炼三:偏差补偿(给结果“美颜”)


%——————————————————————————————————%

\section{仿真验证}
包括应用场景验证和其他方法对比
\subsection{例1:慢变与快变参数跟踪}
为什么要这么验证:针对应用场景的?
系统参数包括正弦、线性、分段常数等多种变化形式。

结果显示:

偏差补偿显著提高估计精度;

SVD更新提升数值稳定性;

算法能完全跟踪参数变化,无延迟。
\subsection{例2:与FFBCRLS方法对比}
FFBCRLS(带遗忘因子的偏差补偿递推最小二乘)存在明显估计延迟;

本文方法由于引入迭代轴,无延迟,估计误差更小;

迭代方法在有限时间系统中具有优势,但计算量较大。


%——————————————————————————————————%








\newpage

\chapter{不确定系统的闭环最优故障检测}\label{ch:environment}
注意只是检测不是诊断,因此只是包含故障检测,并不包含分离和隔离,更不包括检测出故障之后
怎么对故障进行处理。
《2024博士论文》
Both data-driven methods [61, 94, 137] and 
model-based methods [35, 63, 94, 137, 262] have shown considerable progress.
本章是基于观测器的方法[35, 79],属于基于模型的方法。
在这些方法的基础上,已经开发出用于故障检测与隔离(FDI)的综合策略[35,63,262]。
[35]J. Chen and R. J. Patton, Robust Model-Based Fault Diagnosis for Dynamic Systems, Boston, MA: Springer US, 1999, pages 1089–1091.


%——————————————————————————————————%

\section{研究对象}

\subsection{问题}
开环/闭环配置下 ||  连续时间 || 线性时不变 || 不确定系统 的 鲁棒故障检测滤波器设计问题
\subsubsection{为什么要研究这个问题}

为什么要单独研究不确定性系统?因为传统方法都是针对标称系统进行故障诊断的(如Steven x Ding)
而对包含不确定性的系统,是否故障诊断会没有区别?
在不考虑不确定性时,开环和闭环的残差生成结果是一直的(见Ding的书,7.9)
那么在含不确定性情况下,开环和闭环是否会有区别?
在这里并没有提到光刻机等应用场景,因此这应该是普适的问题,对应普适的解决方法。

\subsubsection{为什么研究光刻机的故障诊断问题需要研究这样一个问题?}
现实世界系统中固有的可变性和不确定性,导致系统一定会发生故障,那问题就在于何时会发生故障。

\subsubsection{研究这样一个问题有什么意义}
及时检测和识别故障对于降低性能下降、损坏和对人员安全威胁的风险至关重要。
从诊断系统中获得的知识可用于优化维护计划,从而减少风险和停机时间。
在此背景下,有效的故障检测方法在确保复杂工程系统的可靠性和安全性方面起着至关重要的作用。

\subsection{故障检测的关键挑战}
故障检测(FD)的关键挑战在于区分故障与未知扰动。

\subsection{基于模型的故障诊断方法总结与存在的问题}
基于模型的故障诊断系统要取得令人满意的性能,需要在故障敏感性和扰动抑制之间取得微妙的平衡[62]。
\subsubsection{针对LTI系统的故障诊断方法}
基于因式分解的技术[62,142],这些技术通常通过求解黎卡提方程来实现[157]。
此外,还采用了H−/H∞技术,并运用线性矩阵不等式(LMI)综合法[134,156,248,270]。
如果扰动和系统模型完全已知,这些方法在使残差对故障尽可能敏感的意义上是最优的。
\subsubsection{用于解决建模不确定性问题的鲁棒方法}
许多方法利用$$H/infinit$$准则,并通过$$miu$$综合对其进行优化[216,242]。
其他方法涉及H模型匹配技术,通过线性矩阵不等式(LMI)优化来解决[285]。
此外,还采用H-/H无穷准则,并通过LMI解决方案来处理[125,269],有时会与μg分析相结合[178]。
\subsubsection{方法的缺陷(待理解)}
与针对无不确定性的线性时不变系统的方法[62,157,248]不同,
现有文献中的鲁棒方法往往要么过于保守,要么应用复杂。
例如,仅依赖H准则的方法不能直接考虑H故障灵敏度,而必须进行后验分析。
模型匹配技术通常无法保证H故障灵敏度方面的最优性,
因为其有效性在很大程度上取决于参考模型。
此外,这些方法通常难以扩展用于故障隔离目的。
一般而言,许多故障诊断方法是针对开环系统开发的,因此并不适用于闭环系统。
通过一个激励示例表明,考虑闭环对于获得可靠结果至关重要。


\subsection{设计目标(同时也是故障检测的关键挑战)}
设计目标:将故障与扰动、固有建模不确定性区分开

\subsection{方法}
本章是基于观测器的方法
\subsubsection{老实说,没有很理解,需要对每个名词、动作进行释义以及解释作用}
该解决方案基于不确定性和扰动模型的上界,通过单个黎卡提方程求解,实现了扰动抑制、建模不确定性与故障灵敏度之间的最优折衷。
基于最坏情况扰动和不确定性模型,通过求解单个Riccati方程,
提供了一种统一的方法来处理参数和动态不确定性。
这种最坏情况模型是通过非线性优化和边界Nevanlinna-Pick方法的应用获得的。
\subsubsection{上界模型-相关参考文献}、
上界模型源于具有混合不确定性系统的最坏情况增益分析,其中混合不确定性既包含动态不确定性,也包含参数不确定性。已知该问题是NP难问题[26]。为解决此问题,可采用倾斜μ幂迭代[12,129,212]和基于D-G标度的凸优化[12,130,188,212]等技术计算下界和上界。倾斜μ幂迭代使用启发式方法确定特定频率下对应最坏情况下界的参数或复矩阵值。通过插值法可构建稳定的线性时不变(LTI)样本[286]。基于这一概念,[195,196]中提出的方法构建了在多个频率上最大化增益的最坏情况混合不确定性样本,该方法利用非线性优化和边界Nevanlinna-Pick(BNP)插值[14]。这种方法生成一个稳定、范数有界的LTI不确定性样本,该样本对一组矩阵样本进行插值,为不确定性和扰动模型提供最坏情况下的上界。

\subsection{如何验证:实验}
下一代光刻用掩膜板工作台原型
(最终不能脱离实际,还是要回到光刻机上来,不然就是纯为了做研究而做研究了)

\subsection{结论/效果}
结果表明,该方法在故障敏感性与对建模不确定性和扰动的抑制之间取得了最优折衷。
这种能力能够在残差中清晰地区分故障和不良影响,从而提高故障检测的可靠性,最终有助于提升安全性和性能。






%——————————————————————————————————%

\newpage

\input{3}
\newpage
\input{4}
\newpage

\input{5}
\newpage

\printbibliography
\end{document}
