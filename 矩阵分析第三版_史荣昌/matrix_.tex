% This LaTeX document needs to be compiled with XeLaTeX.
\documentclass[10pt]{article}
\usepackage[utf8]{inputenc}
\usepackage{amsmath}
\usepackage{amsfonts}
\usepackage{amssymb}
\usepackage[version=4]{mhchem}
\usepackage{stmaryrd}
\usepackage{mathrsfs}
\usepackage{graphicx}
\usepackage[export]{adjustbox}
\graphicspath{ {./images/} }
\usepackage{arydshln}
\usepackage{bbold}
\usepackage[fallback]{xeCJK}
\usepackage{polyglossia}
\usepackage{fontspec}
\IfFontExistsTF{Noto Serif CJK SC}
{\setCJKmainfont{Noto Serif CJK SC}}
{\IfFontExistsTF{STSong}
  {\setCJKmainfont{STSong}}
  {\IfFontExistsTF{Droid Sans Fallback}
    {\setCJKmainfont{Droid Sans Fallback}}
    {\setCJKmainfont{SimSun}}
}}

\setmainlanguage{english}
\IfFontExistsTF{CMU Serif}
{\setmainfont{CMU Serif}}
{\IfFontExistsTF{DejaVu Sans}
  {\setmainfont{DejaVu Sans}}
  {\setmainfont{Georgia}}
}

\title{別路拉拉行 \\
 第3版 }

\author{史荣肙 魏 丰 编著}
\date{}


\begin{document}
\maketitle


\section*{矩 阵 分 析 }
(第3版)

史荣昌 魏 丰 编著

\section*{内 容 简 介}
本书是作者根据 20 多年教学实践经历 3 个版本使用编写而成,主要介绍线性空间和线性变换,$\lambda$-矩阵与矩阵的 Jordan 标准形,矩阵的有理标准形,内积空间、正规矩阵、Hermite 矩阵,矩阵分解,范数、序列、级数,矩阵函数,函数矩阵与矩阵微分方程,矩阵广义逆,Kronecker 积。

本书适合高等院校学生、工学硕士、工程硕士研究生应用。

\section*{版权专有 侵权必究}
\section*{图书在版编目(CIP)数据}
矩阵分析/史荣昌,魏丰编著.-3 版.—北京:北京理工大学出版社, 2010. 6

ISBN 978-7-5640-3189-3\\
I.(1)矩… II.(1)史…(2)魏… III.(1)矩阵分析-高等学校-教材 IV.(1)0151.21

中国版本图书馆 CIP 数据核字(2010)第085033 号

出版发行/北京理工大学出版社\\
社 址/北京市海淀区中关村南大街5号\\
邮 编/ 100081\\
电 话/( 010 )68914775(办公室)68944990(批销中心)68911084(读者服务部)\\
网 址/http://www.bitpress.com.cn\\
经 销/全国各地新华书店\\
印 刷/天津紫阳印刷有限公司\\
开 本/710毫米 $\times 1000$ 毫米 1/16\\
印 张/ 18\\
字 数/338千字\\
版 次/2010年6月第3版 2010年6月第12次印刷\\
印 数/33001~37000册 责任校对/陈玉梅\\
定 价/29.00元 责任印制/边心超

图书出现印装质量问题,本社负责调换

\section*{前 言}
本书是根据矩阵理论的基本内容,以及作者 20 余年讲授的经验,并吸收了许多同仁的建议编写而成。

作者认为,一本合适的工学硕士、工程硕士研究生的教材,除了具备一定的理论深度、广度之外,行文应该深入浅出,简洁、易读,适于自学。与其余同类教科书相比,本书注意配备相当数量的例题、习题,使得读者易于理解、掌握基本理论的内容、方法。在选择例题以及讲解例题的细节上又有特色。为了帮助学生自学及较顺利演算习题,掌握重点,作者还编写了与本书配套的《矩阵分析学习指导》一书,该书中有本书习题的全部解答。

本书与作者以往相同教材相比,除了增加例题、习题以外,本版应许多读者要求还适当增加了部分内容,但是风格特色没有改变。

本书适宜 $50 \sim 60$ 学时教学之用。教师可以根据具体情况选用。某些章节内容用*标示,属较难部分。

在本书的编写过程中得到了华东理工大学谢国瑞教授、北京理工大学尤定华教授、吴惠彬副教授的帮助、指教,在此表示衷心的感谢。

本书难免有许多缺点、错误,祈望读者批评指正。

\section*{目 录}
第一章 线性空间和线性变换 ..... (1)\\
§ 1.1 线性空间 ..... (1)\\
§ 1.2 基与坐标、坐标变换 ..... (5)\\
§ 1.3 线性子空间 ..... (12)\\
§ 1.4 线性映射 ..... (18)\\
§ 1.5 线性映射的值域、核 ..... (25)\\
§ 1.6 线性变换的矩阵与线性变换的运算 ..... (29)\\
§ $1.7 n$ 维线性空间的同构 ..... (33)\\
§ 1.8 线性变换的特征值与特征向量 ..... (35)\\
§ 1.9 线性变换的不变子空间 ..... (41)\\
§ 1.10 矩阵的相似对角形 ..... (45)\\
习题 ..... (52)\\
第二章 $\lambda$-矩阵与矩阵的 Jordan 标准形 ..... (55)\\
§ $2.1 \lambda$-矩阵及标准形 ..... (55)\\
§ 2.2 初等因子与相似条件 ..... (66)\\
§ 2.3 矩阵的 Jordan 标准形 ..... (75)\\
§ 2.4 矩阵的有理标准形 ..... (86)\\
习题 ..... (90)\\
第三章 内积空间、正规矩阵、Hermite 矩阵 ..... (93)\\
§3.1 欧氏空间、酉空间 ..... (93)\\
§3.2 标准正交基、Schmidt 方法 ..... (99)\\
§3.3 酉变换、正交变换 ..... (102)\\
§ 3.4 幂等矩阵、正交投影 ..... (106)\\
§3.5 对称与反对称变换 ..... (113)\\
§ 3.6 Schur 引理、正规矩阵 ..... (115)\\
§ 3. 7 Hermite 变换、正规变换 ..... (125)\\
§ 3.8 Hermite 矩阵、Hermite 二次齐式 ..... (129)\\
§3.9 正定二次齐式、正定 Hermite 矩阵 ..... (134)\\
§ 3. 10 Hermite 矩阵偶在复相合下的标准形 ..... (139)\\
§ 3. 11 Rayleigh 商 ..... (144)\\
知阵分析(第3版)\\
习题 ..... (147)\\
第四章 矩阵分解 ..... (151)\\
§4.1 矩阵的满秩分解 ..... (151)\\
§4.2 矩阵的正交三角分解( $U R 、 Q R$ 分解) ..... (154)\\
§4.3 矩阵的奇异值分解 ..... (157)\\
§4.4 矩阵的极分解 ..... (162)\\
§4.5 矩阵的谱分解 ..... (164)\\
习题 ..... (174)\\
第五章 范数、序列、级数 ..... (177)\\
§5.1 向量范数 ..... (177)\\
§ 5.2 矩阵范数 ..... (180)\\
§5.3 诱导范数(算子范数) ..... (183)\\
§5.4 矩阵序列与极限 ..... (187)\\
§ 5.5 矩阵幂级数 ..... (192)\\
§5.6 矩阵的测度 ..... (199)\\
习题 ..... (202)\\
第六章 矩阵函数 ..... (204)\\
§6.1 矩阵多项式、最小多项式 ..... (204)\\
§6.2 矩阵函数及其 Jordan 表示 ..... (209)\\
§6.3 矩阵函数的内插多项式表示与多项式表示 ..... (212)\\
§ 6.4 矩阵函数的幂级数表示 ..... (217)\\
§6.5 矩阵指数函数与矩阵三角函数 ..... (222)\\
习题 ..... (225)\\
第七章 函数矩阵与矩阵微分方程 ..... (229)\\
§7.1 函数矩阵对纯量的导数与积分 ..... (229)\\
§7.2 函数向量的线性相关性 ..... (234)\\
§7.3 矩阵微分方程 $\frac{\mathrm{d} \boldsymbol{X}(t)}{\mathrm{d} t}=\boldsymbol{A}(t) \boldsymbol{X}(t)$ ..... (238)\\
§7.4 线性向量微分方程 $\frac{\mathrm{d} \boldsymbol{x}(t)}{\mathrm{d} t}=\boldsymbol{A}(t) \boldsymbol{x}(t)+\boldsymbol{f}(t)$ ..... (240)\\
习题 ..... (243)\\
第八章 矩阵的广义逆 ..... (244)\\
§8.1 广义逆矩阵 ..... (244)\\
§8.2 伪逆矩阵 ..... (249)\\
§8.3 广义逆与线性方程组 ..... (252)\\
习题 ..... (258)\\
第九章 Kronecker 积 ..... (259)\\
§9.1 Kronecker 积的定义与性质 ..... (259)\\
§ 9.2 函数矩阵对矩阵的导数 ..... (264)\\
§ 9.3 Kronecker 积的特征值 ..... (270)\\
§9.4 矩阵的列展开与行展开 ..... (271)\\
§9.5 线性矩阵代数方程 ..... (272)\\
符号说明 ..... (275)\\
参考文献 ..... (277)

\section*{线性空间和线性变换}
本章所介绍的基本概念是矩阵理论的基础,线性空间、线性子空间、线性变换及其矩阵表示,线性变换的核 $\mathscr{A}^{-1}(0)$ 与值域 $\mathscr{B}(V)$ ,线性变换的特征值与特征向量等是本书的重要概念。

\section*{§1.1 线性空间}
线性空间是线性代数最基本的概念。我们将简要地介绍线性空间,所考虑的数域是实数域(记为 $\mathbf{R}$ )和复数域(记为 $\mathbf{C}$ ),统称数域 $F$ 。

\section*{一、线性空间概念}
先举几个例子。\\
例1.1.1 $n$ 阶实方阵集合 $R^{n \times n}=\{\boldsymbol{A}, \boldsymbol{B}, \boldsymbol{C}, \cdots\}$ ,由线性代数知矩阵有加法运算与数乘矩阵运算。矩阵的加法运算具有四条性质:\\
(1)加法交换律 $\boldsymbol{A}+\boldsymbol{B}=\boldsymbol{B}+\boldsymbol{A}$ ;\\
(2)加法结合律 $(\boldsymbol{A}+\boldsymbol{B})+\boldsymbol{C}=\boldsymbol{A}+(\boldsymbol{B}+\boldsymbol{C})$ ;\\
(3)存在零矩阵" 0 ",满足 $\boldsymbol{A}+0=0+\boldsymbol{A}=\boldsymbol{A}$ ;\\
(4)对于 $R^{n \times n}$ 中任何一个矩阵 $A=\left(a_{i j}\right)_{n \times n}$ ,都有对应的负矩阵 $-A= \left(-a_{i j}\right)_{n \times n}$ ,满足 $\boldsymbol{A}+(-\boldsymbol{A})=0$ 。

数乘矩阵运算也具有四条性质:\\
(5) $1 \cdot \boldsymbol{A}=\boldsymbol{A}$ ;\\
$(6) k(l \boldsymbol{A})=(k l) \boldsymbol{A} \quad(k, l$ 为数 $)$ ;\\
(7)$(k+l) \boldsymbol{A}=k \boldsymbol{A}+l \boldsymbol{A}$ ;\\
(8)$k(\boldsymbol{A}+\boldsymbol{B})=k \boldsymbol{A}+k \boldsymbol{B}$ .\\
例1.1.2 $\boldsymbol{n}$ 元实向量集合 $R^{n}=\{\boldsymbol{\alpha}, \boldsymbol{\beta}, \boldsymbol{\nu}, \cdots\}$ ,由线性代数知向量有加法运算与数乘向量运算。向量的加法运算具有四条性质:\\
(1)加法交换律 $\boldsymbol{\alpha}+\boldsymbol{\beta}=\boldsymbol{\beta}+\boldsymbol{\alpha} \quad\left(\boldsymbol{\alpha}, \boldsymbol{\beta} \in R^{n}\right)$\\
(2)加法结合律 $(\boldsymbol{\alpha}+\boldsymbol{\beta})+\boldsymbol{\nu}=\boldsymbol{\alpha}+(\boldsymbol{\beta}+\boldsymbol{\nu}) \quad\left(\boldsymbol{\alpha}, \boldsymbol{\beta}, \boldsymbol{\nu} \in R^{n}\right)$\\
(3)存在零向量" 0 ",满足 $\boldsymbol{\alpha}+0=0+\boldsymbol{\alpha}=\boldsymbol{\alpha}$ ;\\
(4)对于 $R^{n}$ 中任何一个实向量 $\boldsymbol{\alpha}=\left(a_{1}, a_{2}, \cdots, a_{n}\right)$ 都有对应的负向量 $-\boldsymbol{\alpha}= \left(-a_{1},-a_{2}, \cdots,-a_{n}\right)$ ,满足 $\boldsymbol{\alpha}+(-\boldsymbol{\alpha})=0$ 。

数乘向量运算也具有四条性质:\\
(5) $1 \cdot \boldsymbol{\alpha}=\boldsymbol{\alpha}$\\
(6)$k(l \boldsymbol{\alpha})=(k l) \boldsymbol{\alpha}$\\
(7)$(k+l) \boldsymbol{\alpha}=k \boldsymbol{\alpha}+l \boldsymbol{\alpha}$\\
(8)$k(\boldsymbol{\alpha}+\boldsymbol{\beta})=k \boldsymbol{\alpha}+k \boldsymbol{\beta}$ .\\
例1.1.3 设 $R[x]_{n}$ 表示次数小于 $n$ 的变量 $x$ 的实系数多项式 $f(x)=a_{0}+ a_{1} x+a_{2} x^{2}+\cdots+a_{n-1} x^{n-1}$ 的集合.$R[x]_{n}$ 中的两个多项式 $f(x)$ 与 $g(x)$ 的加法与通常多项式加法运算一样,数乘多项式 $f(x)$ 与通常数乘多项式运算一样。则多项式的加法运算具有四条性质:\\
(1)加法交换律 $f(x)+g(x)=g(x)+f(x)$\\
(2)加法结合律

$$
[f(x)+g(x)]+h(x)=f(x)+[g(x)+h(x)]
$$

(3)存在零多项式 0 ,满足 $f(x)+0=0+f(x)=f(x)$\\
(4)对于 $R[x]_{n}$ 中任何一个多项式 $f(x)=a_{0}+a_{1} x+a_{2} x^{2}+\cdots+a_{n-1} x^{n-1}$ ,都有对应的负多项式 $-f(x)=-a_{0}-a_{1} x-a_{2} x^{2}-\cdots-a_{n-1} x^{n-1}$ ,满足 $f(x)+ (-f(x))=0$.

数乘多项式的运算也具有四条性质:\\
(5) $1 \cdot f(x)=f(x)$\\
(6)$k(l f(x))=(k l) f(x)$\\
(7)$(k+l) f(x)=k f(x)+l f(x)$\\
(8)$k(f(x)+g(x))=k f(x)+k g(x)$ .\\
从这些例子中看到,所涉及的对象(集合)虽然不同,但是它们有共同之处。它们都可定义加法和数乘这两种运算,而且关于这两种运算都具有八条性质。还可以举出许多例子,都有此三例的共同之处。为此,我们把它们统一起来加以研究,引人线性空间的概念。

定义1.1.1 设 $V$ 是一个非空集合,$F$ 是一个数域,在集合 $V$ 的元素之间定义了加法运算。即对于 $V$ 中任意两个元素 $\alpha$ 与 $\beta$ ,在 $V$ 中都有唯一的元素 $\nu$ 与它们相对应,称之为 $\alpha$ 与 $\beta$ 的和,记为 $\nu=\alpha+\beta$ ,并且加法运算满足下面四条法则:\\
(1)交换律 $\alpha+\beta=\beta+\alpha$\\
(2)结合律 $\alpha+(\beta+\xi)=(\alpha+\beta)+\xi$\\
(3)零元素 在 $V$ 中有一元素 0 (称作零元素),对于 $V$ 中任一元素 $\alpha$ 都有 $\alpha+0=\alpha$\\
(4)负元素 对于 $V$ 中每一个元素 $\alpha$ ,都有 $V$ 中的元素 $\beta$ ,使得 $\alpha+\beta=0$ .

在集合 $V$ 的元素与数域 $F$ 中的数之间还定义了一种运算,叫做数乘。即对于 $V$ 中任一元素 $\boldsymbol{\alpha}$ 与 $\boldsymbol{F}$ 中任一数 $k$ ,在 $V$ 中有唯一的一个元素 $\eta$ 与它们对应,称为 $k$与 $\alpha$ 的数乘,记为 $\eta=k \cdot \alpha=k \alpha$ ,并且数乘运算满足下面四条法则:\\
(1) $1 \cdot \alpha=\alpha$\\
(2)$k(l \alpha)=(k l) \alpha$\\
(3)$(k+l) \alpha=k \alpha+l \alpha$\\
(4)$k(\alpha+\beta)=k \alpha+k \beta$ .\\
其中 $k, l$ 表示数域 $F$ 中的任意数,$\alpha, \beta$ 表示 $V$ 中任意元素。\\
称这样的集合 $V$ 为数域 $F$ 上的线性空间。\\
显然,例1.1.1~例1.1.3都是实数域 $\mathbf{R}$ 上的线性空间。现在再举几例。\\
例1.1.4 设 $\boldsymbol{A}$ 为实数(或复数)$m \times n$ 矩阵,易证:齐次线性方程组 $\boldsymbol{A x}=0$的所有解(包括零解)的集合构成实(或复)数域 $\mathbf{R}$(或 $\mathbf{C}$ )上的线性空间。这个空间为方程组 $\boldsymbol{A x}=0$ 的解空间,也称为矩阵 $\boldsymbol{A}$ 的核(或零)空间,常用 $N(\boldsymbol{A})$表示。

例1.1.5 设 $\boldsymbol{A}$ 为实数(或复数)$m \times n$ 矩阵, $\boldsymbol{x}$ 为 $n$ 维列向量,则 $m$ 维列向量集合

$$
V=\left\{\boldsymbol{y} \in R^{m}\left(C^{m}\right) \mid \boldsymbol{y}=\boldsymbol{A} \boldsymbol{x}, \boldsymbol{x} \in R^{n}\left(C^{n}\right), \boldsymbol{A} \in R^{m \times n}\left(C^{m \times n}\right)\right\}
$$

构成实(或复)数域 $\mathbf{R}$(或 $\mathbf{C}$ )上的线性空间,称为 $\boldsymbol{A}$ 的列空间或 $\boldsymbol{A}$ 的值域,常用 $R$ (A)表示。\\
*例1.1.6 设 $X$ 为任意一个非空集合,$F$ 为任意一个数域,定义从 $X$ 到 $F$ 的每一个映射 $f$ 为 $X$ 上的一个 $F$ 值函数,将 $X$ 上的所有 $F$ 值函数构成的集合记做 $F^{X}$ 。在 $F^{X}$ 中定义加法与数乘运算如下:对于任意 $f, g \in F^{X}, k \in F$ ,规定

$$
\begin{array}{cc}
(f+g)(x)=f(x)+g(x), & \forall x \in X \\
(k f)(x)=k(f(x)), & \forall x \in X
\end{array}
$$

那么,容易验证 $F^{X}$ 为数域 $F$ 上的一个线性空间。\\
例1.1.7 由上例可知,$R^{R}$ 为实数域 $\boldsymbol{R}$ 上的线性空间,$C^{C}$ 为复数域 $\mathbf{C}$ 上的线性空间。

例1.1.8 $R^{+}$表示所有正实数集合,在 $R^{+}$中定义加法 $\oplus$ 与数量乘法 $\odot$ 分别为

$$
\begin{aligned}
a \oplus b & =a b, & \forall a, b \in \mathbf{R}^{+} \\
k \odot a & =a^{k}, & \forall a \in \mathbf{R}^{+}, k \in \mathbf{R}
\end{aligned}
$$

可以验证, $\mathbf{R}^{+}$构成实数域 $\mathbf{R}$ 上的线性空间.\\
上述几个例子都是在一个集合 $V$ 上定义加法与数域 $F$ 的数量乘法,构成一个线性空间,但问题并不总是那么简单。下面举两个例子,虽然定义了加法与数量乘法,但 $V$ 并不构成一个线性空间。

例1.1.9 设 $V$ 是由系数在实数域 $\mathbf{R}$ 上,次数为 $n$ 的 $n$ 次多项式 $f(x)$ 构成的集合,其加法运算与数乘运算按照通常规定,则 $V$ 不是 $\mathbf{R}$ 上的线性空间。

例1.1.10 设线性非齐次方程组 $\boldsymbol{A} \boldsymbol{X}=\boldsymbol{b}$ 有解,其通解表达式为

$$
\boldsymbol{\xi}+k_{1} \boldsymbol{\alpha}_{1}+k_{2} \boldsymbol{\alpha}_{2}+\cdots+k_{n-r} \boldsymbol{\alpha}_{n-r} \text {, 其中 } k_{1}, \cdots, k_{n-r} \text { 为任意数. }
$$

可以证明:解向量组 $\boldsymbol{\xi}, \boldsymbol{\xi}+\boldsymbol{\alpha}_{1}, \boldsymbol{\xi}+\boldsymbol{\alpha}_{2}, \cdots, \boldsymbol{\xi}+\boldsymbol{\alpha}_{n-r}$ 是方程组 $\boldsymbol{A} \boldsymbol{X}=\boldsymbol{b}$ 解集合的线性极大无关组(作为练习请读者自己证明),但是该解集合不构成线性空间。

\section*{二、向量的线性相关性}
线性空间的元素 $\boldsymbol{\alpha}, \boldsymbol{\beta}, \boldsymbol{\nu}, \cdots$ 称为向量,这里所指的向量比线性代数中 $n$ 元向量 $\boldsymbol{\alpha}=\left(a_{1}, a_{2}, \cdots, a_{n}\right)^{\mathrm{T}}$ 的含义更为广泛。

向量的线性相关性概念与结论在叙述形式上与线性代数相同,下面只作简单论述。

定义1.1.2 设 $V$ 是数域 $F$ 上的线性空间,$\alpha_{1}, \alpha_{2}, \cdots, \alpha_{r}(r \geqslant 1)$ 是 $V$ 中一组向量,$k_{1}, k_{2}, \cdots, k_{r}$ 是数域 $F$ 中一组数,若向量 $\boldsymbol{\alpha}$ 可以表示成

$$
\boldsymbol{\alpha}=k_{1} \boldsymbol{\alpha}_{1}+k_{2} \boldsymbol{\alpha}_{2}+\cdots+k_{r} \boldsymbol{\alpha}_{r}
$$

则称 $\boldsymbol{\alpha}$ 可由 $\boldsymbol{\alpha}_{1}, \boldsymbol{\alpha}_{2}, \cdots, \boldsymbol{\alpha}_{r}$ 线性表示(出),也称 $\boldsymbol{\alpha}$ 是 $\boldsymbol{\alpha}_{1}, \boldsymbol{\alpha}_{2}, \cdots, \boldsymbol{\alpha}_{r}$ 的线性组合.\\
例如,因为

$$
\left[\begin{array}{ll}
1 & 2 \\
3 & 4
\end{array}\right]=1 \cdot\left[\begin{array}{ll}
1 & 0 \\
0 & 0
\end{array}\right]+2 \cdot\left[\begin{array}{ll}
0 & 1 \\
0 & 0
\end{array}\right]+3 \cdot\left[\begin{array}{ll}
0 & 0 \\
1 & 0
\end{array}\right]+4 \cdot\left[\begin{array}{ll}
0 & 0 \\
0 & 1
\end{array}\right]
$$

所以,矩阵 $\left[\begin{array}{ll}1 & 2 \\ 3 & 4\end{array}\right]$ 是矩阵 $\left[\begin{array}{ll}1 & 0 \\ 0 & 0\end{array}\right],\left[\begin{array}{ll}0 & 1 \\ 0 & 0\end{array}\right],\left[\begin{array}{ll}0 & 0 \\ 1 & 0\end{array}\right]$ 与 $\left[\begin{array}{ll}0 & 0 \\ 0 & 1\end{array}\right]$ 的线性组合.\\
定义1.1.3 设 $\boldsymbol{\alpha}_{1}, \boldsymbol{\alpha}_{2}, \cdots, \boldsymbol{\alpha}_{r}(r \geqslant 1)$ ,是线性空间 $V$ 中一组向量。如果在数域 $F$ 中有 $r$ 个不全为零的数 $k_{1}, k_{2}, \cdots, k_{r}$ ,使得


\begin{equation*}
k_{1} \boldsymbol{\alpha}_{1}+k_{2} \boldsymbol{\alpha}_{2}+\cdots+k_{r} \boldsymbol{\alpha}_{r}=0 \tag{1.1.1}
\end{equation*}


则称 $\boldsymbol{\alpha}_{1}, \boldsymbol{\alpha}_{2}, \cdots, \boldsymbol{\alpha}_{r}$ 线性相关。如果一组向量 $\boldsymbol{\alpha}_{1}, \boldsymbol{\alpha}_{2}, \cdots, \boldsymbol{\alpha}_{r}$ 不线性相关,就称为线性无关。换言之,若

$$
k_{1} \boldsymbol{\alpha}_{1}+k_{2} \boldsymbol{\alpha}_{2}+\cdots+k_{r} \boldsymbol{\alpha}_{r}=0
$$

则只有 $k_{1}=k_{2}=\cdots=k_{r}=0$ ,便称 $\boldsymbol{\alpha}_{1}, \boldsymbol{\alpha}_{2}, \cdots, \boldsymbol{\alpha}_{r}$ 线性无关。\\
一组向量 $\boldsymbol{\alpha}_{1}, \boldsymbol{\alpha}_{2}, \cdots, \boldsymbol{\alpha}_{r}$ 要么线性相关,要么线性无关,非此即彼。\\
对于任何一组向量 $\boldsymbol{\alpha}_{1}, \boldsymbol{\alpha}_{2}, \cdots, \boldsymbol{\alpha}_{r}$ ,都能满足等式

$$
k_{1} \boldsymbol{\alpha}_{1}+k_{2} \boldsymbol{\alpha}_{2}+\cdots+k_{r} \boldsymbol{\alpha}_{r}=\mathbf{0}
$$

因此考察一组向量是否线性相关,关键不在于向量是否满足式(1.1.1),而在于满足式(1.1.1)中 $k_{1}, k_{2}, \cdots, k_{r}$ 是否只能全为零。若只能全为零,则向量组 $\boldsymbol{\alpha}_{1}$ , $\boldsymbol{\alpha}_{2}, \cdots, \boldsymbol{\alpha}_{r}$ 线性无关。若满足式(1.1.1)中的 $k_{1}, k_{2}, \cdots, k_{r}$ 除了全为零的情况以外,还有不全为零的情况,则向量组线性相关。

例1.1.11 试证:$R^{2 \times 2}$ 中的一组向量(矩阵)

$$
E_{11}=\left[\begin{array}{ll}
1 & 0 \\
0 & 0
\end{array}\right], E_{12}=\left[\begin{array}{ll}
0 & 1 \\
0 & 0
\end{array}\right], E_{21}=\left[\begin{array}{ll}
0 & 0 \\
1 & 0
\end{array}\right], E_{22}=\left[\begin{array}{ll}
0 & 0 \\
0 & 1
\end{array}\right]
$$

是线性无关的.\\
证明 若

$$
k_{1} \cdot\left[\begin{array}{ll}
1 & 0 \\
0 & 0
\end{array}\right]+k_{2} \cdot\left[\begin{array}{ll}
0 & 1 \\
0 & 0
\end{array}\right]+k_{3} \cdot\left[\begin{array}{ll}
0 & 0 \\
1 & 0
\end{array}\right]+k_{4} \cdot\left[\begin{array}{ll}
0 & 0 \\
0 & 1
\end{array}\right]=0
$$

即

$$
\left[\begin{array}{ll}
k_{1} & k_{2} \\
k_{3} & k_{4}
\end{array}\right]=0
$$

所以

$$
k_{1}=k_{2}=k_{3}=k_{4}=0
$$

因此满足

$$
k_{1} E_{11}+k_{2} E_{12}+k_{3} E_{21}+k_{4} E_{22}=0
$$

的 $k_{1}, k_{2}, k_{3}, k_{4}$ 只能全为零,于是 $\boldsymbol{E}_{11}, \boldsymbol{E}_{12}, \boldsymbol{E}_{21}, \boldsymbol{E}_{22}$ 线性无关。\\
例1.1.12 试证:$R^{2 \times 2}$ 中的向量(矩阵)组

$$
\boldsymbol{\alpha}_{1}=\left[\begin{array}{ll}
1 & 1 \\
0 & 0
\end{array}\right], \quad \boldsymbol{\alpha}_{2}=\left[\begin{array}{ll}
1 & 1 \\
0 & 1
\end{array}\right], \quad \boldsymbol{\alpha}_{3}=\left[\begin{array}{ll}
0 & 0 \\
0 & 1
\end{array}\right]
$$

是线性相关的。\\
证明 容易验证等式

$$
\alpha_{1}-\alpha_{2}+\alpha_{3}=0
$$

所以 $\boldsymbol{\alpha}_{1}, \boldsymbol{\alpha}_{2}, \boldsymbol{\alpha}_{3}$ 线性相关。\\
定理1.1.1 设线空间 $V$ 中向量组 $\boldsymbol{\alpha}_{1}, \boldsymbol{\alpha}_{2}, \cdots, \boldsymbol{\alpha}_{m}$ 线性无关,且向量组 $\boldsymbol{\alpha}_{1}$ , $\boldsymbol{\alpha}_{2}, \cdots, \boldsymbol{\alpha}_{m}, \boldsymbol{\beta}$ 线性相关,则 $\boldsymbol{\beta}$ 可由 $\boldsymbol{\alpha}_{1}, \boldsymbol{\alpha}_{2}, \cdots, \boldsymbol{\alpha}_{m}$ 线性表出,且表出是唯一的(证明留给读者)。

线性代数中向量组的极大线性无关组,向量组的秩等概念在线性空间中也可自然地引进,不再一一赘述。

\section*{§1.2 基与坐标、坐标变换}
\section*{一、基与维数、坐标}
定义1.2.1 设数域 $F$ 上线性空间 $V$ 中有 $n$ 个线性无关向量 $\boldsymbol{\alpha}_{1}, \boldsymbol{\alpha}_{2}, \cdots, \boldsymbol{\alpha}_{n}$ ,而且 $V$ 中任何一个向量 $\boldsymbol{\alpha}$ 都可由 $\boldsymbol{\alpha}_{1}, \boldsymbol{\alpha}_{2}, \cdots, \boldsymbol{\alpha}_{n}$ 线性表出


\begin{equation*}
\boldsymbol{\alpha}=k_{1} \boldsymbol{\alpha}_{1}+k_{2} \boldsymbol{\alpha}_{2}+\cdots+k_{n} \boldsymbol{\alpha}_{n} \tag{1.2.1}
\end{equation*}


则称 $\boldsymbol{\alpha}_{1}, \boldsymbol{\alpha}_{2}, \cdots, \boldsymbol{\alpha}_{n}$ 为 $V$ 的一个基,$\left(k_{1}, k_{2}, \cdots, k_{n}\right)^{\mathrm{T}}$ 为 $\boldsymbol{\alpha}$ 在基 $\boldsymbol{\alpha}_{1}, \boldsymbol{\alpha}_{2}, \cdots, \boldsymbol{\alpha}_{n}$ 下的坐标。这时,就称 $V$ 为 $n$ 维线性空间,并记 $\operatorname{dim} V=n$ 。

显然, $\boldsymbol{\alpha}_{1}, \boldsymbol{\alpha}_{2}, \cdots, \boldsymbol{\alpha}_{n}$ 在基 $\boldsymbol{\alpha}_{1}, \boldsymbol{\alpha}_{2}, \cdots, \boldsymbol{\alpha}_{n}$ 下的坐标分别为 $(1,0,0, \cdots, 0)^{\mathrm{T}}$ ,\\
$(0,1,0, \cdots, 0)^{\mathrm{T}}, \cdots,(0,0, \cdots, 0,1)^{\mathrm{T}}$ 。即

$$
\boldsymbol{\alpha}_{i}=\left(\boldsymbol{\alpha}_{1}, \boldsymbol{\alpha}_{2}, \cdots, \boldsymbol{\alpha}_{n}\right)\left[\begin{array}{c}
0 \\
\vdots \\
0 \\
1 \\
0 \\
\vdots \\
0
\end{array}\right]-i \text { 行 } \quad(i=1, \cdots, n)
$$

根据定理1.1.1知,向量 $\boldsymbol{\alpha}$ 在基 $\boldsymbol{\alpha}_{1}, \boldsymbol{\alpha}_{2}, \cdots, \boldsymbol{\alpha}_{n}$ 下的坐标 $\left(k_{1}, k_{2}, \cdots, k_{n}\right)^{\mathrm{T}}$ 是唯一的。将式(1.2.1)改写为

\[
\boldsymbol{\alpha}=k_{1} \boldsymbol{\alpha}_{1}+k_{2} \boldsymbol{\alpha}_{2}+\cdots+k_{n} \boldsymbol{\alpha}_{n}=\left(\boldsymbol{\alpha}_{1}, \boldsymbol{\alpha}_{2}, \cdots, \boldsymbol{\alpha}_{n}\right)\left[\begin{array}{c}
k_{1}  \tag{1.2.2}\\
k_{2} \\
\vdots \\
k_{n}
\end{array}\right]
\]

式中 $1 \times n$ 分块矩阵 $\left(\boldsymbol{\alpha}_{1}, \boldsymbol{\alpha}_{2}, \cdots, \boldsymbol{\alpha}_{n}\right)$ 的元素 $\boldsymbol{\alpha}_{1}, \boldsymbol{\alpha}_{2}, \cdots, \boldsymbol{\alpha}_{n}$ 不是数字而是向量。在作矩阵分块运算时可把它看成是一个数来处理

例如,$n$ 维线性空间 $R^{n \times n}$ 中的元素(向量)组 $\left\{\boldsymbol{E}_{i j}\right\}$ ,其中 $n$ 阶矩阵 $\boldsymbol{E}_{i j}$ 是第 $i$行、第 $j$ 列处元素为 $\mathbf{1}$ ,其余元素全为零的矩阵,即

$$
\boldsymbol{E}_{i j}=\left[\begin{array}{ccccccc} 
& 0 & & 0 & & 0 & \\
& \vdots & 0 & 0 & & & \\
0 & \vdots & & 0 & \cdots & 0 \\
& 0 & & 0 & & & \\
& & & 0 & & 0 &
\end{array}\right] \longrightarrow_{i} i \text { 行 } \quad(i, j=1,2, \cdots, n)
$$

是 $R^{n \times n}$ 的一组基。\\
例1.2.1 试证:线性空间

$$
R[x]_{n}=\left\{a_{0}+a_{1} x+a_{2} x^{2}+\cdots+a_{n-1} x^{n-1} \mid a_{i} \in \mathbf{R}\right\}
$$

是 $n$ 维的,并求 $a_{0}+a_{1} x+a_{2} x^{2}+\cdots+a_{n-1} x^{n-1}$ 在基 $1, x-a,(x-a)^{2}, \cdots,(x-$ a)${ }^{n-1}$ 下的坐标。

证 先证 $R[x]_{n}$ 中元素

$$
1, x, x^{2}, \cdots, x^{n-1}
$$

是线性无关的.设

$$
k_{0} \cdot 1+k_{1} \cdot x+k_{2} \cdot x^{2}+\cdots+k_{n-1} x^{n-1}=0
$$

由于 $R[x]_{n}$ 中 $x$ 是变量,所以欲使上式对于任何 $x$ 都成立的充分必要条件是

$$
k_{0}=k_{1}=\cdots=k_{n-1}=0
$$

于是 $1, x, x^{2}, \cdots, x^{n-1}$ 线性无关。\\
对于 $R[x]_{n}$ 中任何一个向量(多项式)

$$
f(x)=a_{0}+a_{1} x+a_{2} x^{2}+\cdots+a_{n-1} x^{n-1} \in R[x]_{n}
$$

均可由 $1, x, x^{2}, \cdots, x^{n-1}$ 线性表出,这表明: $1, x, x^{2}, \cdots, x^{n-1}$ 是 $R[x]_{n}$ 的基,于是 $R [x]_{n}$ 是 $n$ 维的。

不难验证: $1, x-a,(x-a)^{2}, \cdots,(x-a)^{n-1}$ 也是 $R[x]_{n}$ 的一组基。因为

$$
\begin{aligned}
f(x)= & f(a)+f^{\prime}(a)(x-a)+\frac{f^{\prime \prime}(a)}{2!}(x-a)^{2}+\cdots+ \\
& \frac{f^{(n-1)}(a)}{(n-1)!}(x-a)^{n-1}
\end{aligned}
$$

故 $f(x)$ 在这组基下的坐标为

$$
f(a), f^{\prime}(a), \frac{f^{\prime \prime}(a)}{2!}, \cdots, \frac{f^{(n-1)}(a)}{(n-1)!}
$$

例1.2.2 已知

$$
\boldsymbol{A}=\left[\begin{array}{rrrr}
1 & 0 & 2 & 1 \\
-1 & 2 & 1 & 3 \\
1 & 2 & 5 & 5 \\
2 & -2 & 1 & -2
\end{array}\right]
$$

试求 $\boldsymbol{A}$ 的核空间的两组基.\\
解 $\boldsymbol{A}$ 的核空间就是 $\boldsymbol{A} \boldsymbol{x}=0$ 的解空间(见例1.1.4),所以 $\boldsymbol{A} x=0$ 的基础解系就是核空间的基。对 $\boldsymbol{A}$ 作初等行变换后得

$$
\boldsymbol{A}=\left[\begin{array}{rrrr}
1 & 0 & 2 & 1 \\
-1 & 2 & 1 & 3 \\
1 & 2 & 5 & 5 \\
2 & -2 & 1 & -2
\end{array}\right] \rightarrow\left[\begin{array}{cccc}
1 & 0 & 2 & 1 \\
0 & 1 & 3 / 2 & 2 \\
0 & 0 & 0 & 0 \\
0 & 0 & 0 & 0
\end{array}\right]
$$

因此 $\boldsymbol{A} x=0$ 的解为

$$
\left\{\begin{array}{l}
x_{1}=-2 x_{3}-x_{4} \\
x_{2}=-\frac{3}{2} x_{3}-2 x_{4}
\end{array}\right.
$$

其中 $x_{3}, x_{4}$ 为自由变量.不难知 $\boldsymbol{A x}=0$ 的基础解系可以取为

$$
\left\{\begin{array} { l } 
{ \boldsymbol { \alpha } _ { 1 } = ( - 4 , - 3 , 2 , 0 ) ^ { \mathrm { T } } } \\
{ \boldsymbol { \alpha } _ { 2 } = ( - 1 , - 2 , 0 , 1 ) ^ { \mathrm { T } } }
\end{array} \text { 或 } \left\{\begin{array}{l}
\boldsymbol{\alpha}_{1}^{\prime}=(-4,-3,2,0)^{\mathrm{T}} \\
\boldsymbol{\alpha}_{2}^{\prime}=(-6,-7,2,2)^{\mathrm{T}}
\end{array}\right.\right.
$$

它们都可以作为 $\boldsymbol{A}$ 的核空间的基,核空间是二维的。\\
例1.2.3 在 $R^{4}$ 中,求向量 $\boldsymbol{\alpha}=(1,2,1,1)^{\mathrm{T}}$ 在基

$$
\boldsymbol{\alpha}_{1}=(1,1,1,1)^{\mathrm{T}}, \quad \boldsymbol{\alpha}_{2}=(1,1,-1,-1)^{\mathrm{T}}
$$

$$
\boldsymbol{\alpha}_{3}=(1,-1,1,-1)^{\mathrm{T}}, \quad \boldsymbol{\alpha}_{4}=(1,-1,-1,1)^{\mathrm{T}}
$$

下的坐标。\\
解 设 $\boldsymbol{\alpha}=(1,2,1,1)^{\mathrm{T}}$ 在所给基 $\boldsymbol{\alpha}_{1}, \boldsymbol{\alpha}_{2}, \boldsymbol{\alpha}_{3}, \boldsymbol{\alpha}_{4}$ 下的坐标为 $k_{1}, k_{2}, k_{3}, k_{4}$ ,故

$$
\boldsymbol{\alpha}=k_{1} \boldsymbol{\alpha}_{1}+k_{2} \boldsymbol{\alpha}_{2}+k_{3} \boldsymbol{\alpha}_{3}+k_{4} \boldsymbol{\alpha}_{4}
$$

即

$$
\begin{aligned}
(1,2,1,1)^{\mathrm{T}}= & k_{1}(1,1,1,1)^{\mathrm{T}}+k_{2}(1,1,-1,-1)^{\mathrm{T}}+ \\
& k_{3}(1,-1,1,-1)^{\mathrm{T}}+k_{4}(1,-1,-1,1)^{\mathrm{T}} \\
= & \left(k_{1}+k_{2}+k_{3}+k_{4}, k_{1}+k_{2}-k_{3}-k_{4}, k_{1}-k_{2}+\right. \\
& \left.k_{3}-k_{4}, k_{1}-k_{2}-k_{3}+k_{4}\right)^{\mathrm{T}}
\end{aligned}
$$

于是有

$$
\left\{\begin{array}{l}
k_{1}+k_{2}+k_{3}+k_{4}=1 \\
k_{1}+k_{2}-k_{3}-k_{4}=2 \\
k_{1}-k_{2}+k_{3}-k_{4}=1 \\
k_{1}-k_{2}-k_{3}+k_{4}=1
\end{array}\right.
$$

解之得

$$
k_{1}=\frac{5}{4}, \quad k_{2}=\frac{1}{4}, \quad k_{3}=-\frac{1}{4}, \quad k_{4}=-\frac{1}{4}
$$

所以 $\boldsymbol{\alpha}$ 在所给基 $\boldsymbol{\alpha}_{1}, \boldsymbol{\alpha}_{2}, \boldsymbol{\alpha}_{3}, \boldsymbol{\alpha}_{4}$ 下的坐标为 $\left(\frac{5}{4}, \frac{1}{4},-\frac{1}{4},-\frac{1}{4}\right)^{\mathrm{T}}$ .\\
例1.2.4 在 $R^{2 \times 2}$ 中,求 $A=\left[\begin{array}{ll}1 & 2 \\ 1 & 0\end{array}\right]$ 在基 $\left[\begin{array}{ll}1 & 1 \\ 1 & 1\end{array}\right],\left[\begin{array}{ll}1 & 1 \\ 1 & 0\end{array}\right],\left[\begin{array}{ll}1 & 1 \\ 0 & 1\end{array}\right],\left[\begin{array}{ll}1 & 0 \\ 1 & 1\end{array}\right]$ 下的坐标。

解 设

$$
\begin{aligned}
{\left[\begin{array}{ll}
1 & 2 \\
1 & 0
\end{array}\right] } & =k_{1}\left[\begin{array}{ll}
1 & 1 \\
1 & 1
\end{array}\right]+k_{2}\left[\begin{array}{ll}
1 & 1 \\
1 & 0
\end{array}\right]+k_{3}\left[\begin{array}{ll}
1 & 1 \\
0 & 1
\end{array}\right]+k_{4}\left[\begin{array}{ll}
1 & 0 \\
1 & 1
\end{array}\right] \\
& =\left[\begin{array}{cc}
k_{1}+k_{2}+k_{3}+k_{4} & k_{1}+k_{2}+k_{3} \\
k_{1}+k_{2}+k_{4} & k_{1}+k_{3}+k_{4}
\end{array}\right]
\end{aligned}
$$

于是有

$$
\left\{\begin{array}{l}
k_{1}+k_{2}+k_{3}+k_{4} \\
k_{1}+k_{2}+k_{3} \\
k_{1}+k_{2} \\
k_{1}
\end{array}+k_{4}=1=12+k_{3}+k_{4}=0 .\right.
$$

解之得

$$
k_{1}=1, \quad k_{2}=1, \quad k_{3}=0, \quad k_{4}=-1
$$

所以 $\boldsymbol{A}$ 在已给基下的坐标为 $(1,1,0,-1)^{\mathrm{T}}$ .

\section*{二、基变换与坐标变换}
非零线性空间的基不是唯一的,一个向量在不同基下的坐标也是不同的,它们之间的关系就是下面要研究的问题。

设 $\boldsymbol{\alpha}_{1}, \boldsymbol{\alpha}_{2}, \cdots, \boldsymbol{\alpha}_{n}$ 与 $\boldsymbol{\beta}_{1}, \boldsymbol{\beta}_{2}, \cdots, \boldsymbol{\beta}_{n}$ 是 $V$ 中两组基,它们之间的关系是

$$
\begin{aligned}
\boldsymbol{\beta}_{i} & =a_{1 i} \alpha_{1}+a_{2 i} \alpha_{2}+\cdots+a_{n i} \alpha_{n} \\
& =\left(\alpha_{1}, \alpha_{2}, \cdots, \alpha_{n}\right)\left(\begin{array}{c}
a_{1 i} \\
a_{2 i} \\
\vdots \\
a_{n i}
\end{array}\right) \quad(i=1,2, \cdots, n)
\end{aligned}
$$

将这 $n$ 个关系式用矩阵记号可以表示成

\[
\left(\boldsymbol{\beta}_{1}, \boldsymbol{\beta}_{2}, \cdots, \boldsymbol{\beta}_{n}\right)=\left(\boldsymbol{\alpha}_{1}, \boldsymbol{\alpha}_{2}, \cdots, \boldsymbol{\alpha}_{n}\right)\left[\begin{array}{cccc}
a_{11} & a_{12} & \cdots & a_{1 n}  \tag{1.2.3}\\
a_{21} & a_{22} & \cdots & a_{2 n} \\
\vdots & \vdots & & \vdots \\
a_{n 1} & a_{n 2} & \cdots & a_{n n}
\end{array}\right]
\]

称 $\boldsymbol{n}$ 阶方阵

\[
\boldsymbol{P}=\left[\begin{array}{cccc}
a_{11} & a_{12} & \cdots & a_{1 n}  \tag{1.2.4}\\
a_{21} & a_{22} & \cdots & a_{2 n} \\
\vdots & \vdots & & \vdots \\
a_{n 1} & a_{n 2} & \cdots & a_{n n}
\end{array}\right]
\]

是由基 $\boldsymbol{\alpha}_{1}, \boldsymbol{\alpha}_{2}, \cdots, \boldsymbol{\alpha}_{n}$ 到基 $\boldsymbol{\beta}_{1}, \boldsymbol{\beta}_{2}, \cdots, \boldsymbol{\beta}_{n}$ 的过渡矩阵。则式(1.2.3)可以写成


\begin{equation*}
\left(\boldsymbol{\beta}_{1}, \boldsymbol{\beta}_{2}, \cdots, \boldsymbol{\beta}_{n}\right)=\left(\boldsymbol{\alpha}_{1}, \boldsymbol{\alpha}_{2}, \cdots, \boldsymbol{\alpha}_{n}\right) \boldsymbol{P} \tag{1.2.5}
\end{equation*}


读者容易证明下述定理。\\
定理1.2.1 过渡矩阵 $\boldsymbol{P}$ 是可逆的。\\
下面建立 $V$ 中任意一个向量在不同基下坐标间的关系,即导出坐标变换公式。\\
设 $\xi \in V$ ,若 $\xi$ 在基 $\boldsymbol{\alpha}_{1}, \boldsymbol{\alpha}_{2}, \cdots, \boldsymbol{\alpha}_{n}$ 与 $\boldsymbol{\beta}_{1}, \boldsymbol{\beta}_{2}, \cdots, \boldsymbol{\beta}_{n}$ 下的坐标分别为 $\left(x_{1}, x_{2}, \cdots\right.$ , $\left.x_{n}\right)^{\mathrm{T}}$ 与 $\left(y_{1}, y_{2}, \cdots, y_{n}\right)^{\mathrm{T}}$ ,即若

与

$$
\begin{aligned}
& \boldsymbol{\xi}=\left(\boldsymbol{\alpha}_{1}, \boldsymbol{\alpha}_{2}, \cdots, \boldsymbol{\alpha}_{n}\right)\left(\begin{array}{c}
x_{1} \\
x_{2} \\
\vdots \\
x_{n}
\end{array}\right) \\
& \boldsymbol{\xi}=\left(\boldsymbol{\beta}_{1}, \boldsymbol{\beta}_{2}, \cdots, \boldsymbol{\beta}_{n}\right)\left(\begin{array}{c}
y_{1} \\
y_{2} \\
\vdots \\
y_{n}
\end{array}\right)
\end{aligned}
$$

于是有

$$
\left(\boldsymbol{\alpha}_{1}, \boldsymbol{\alpha}_{2}, \cdots, \boldsymbol{\alpha}_{n}\right)\left(\begin{array}{c}
x_{1} \\
x_{2} \\
\vdots \\
x_{n}
\end{array}\right)=\left(\boldsymbol{\beta}_{1}, \boldsymbol{\beta}_{2}, \cdots, \boldsymbol{\beta}_{n}\right)\left(\begin{array}{c}
y_{1} \\
y_{2} \\
\vdots \\
y_{n}
\end{array}\right)
$$

将式(1.2.5)代人上式右端得

$$
\left(\boldsymbol{\alpha}_{1}, \boldsymbol{\alpha}_{2}, \cdots, \boldsymbol{\alpha}_{n}\right)\left(\begin{array}{c}
x_{1} \\
x_{2} \\
\vdots \\
x_{n}
\end{array}\right)=\left(\boldsymbol{\alpha}_{1}, \boldsymbol{\alpha}_{2}, \cdots, \boldsymbol{\alpha}_{n}\right) \boldsymbol{P}\left(\begin{array}{c}
y_{1} \\
y_{2} \\
\vdots \\
y_{n}
\end{array}\right)
$$

由于 $\boldsymbol{\alpha}_{1}, \boldsymbol{\alpha}_{2}, \cdots, \boldsymbol{\alpha}_{n}$ 是线性无关的,所以

\[
\left(\begin{array}{c}
x_{1}  \tag{1.2.6}\\
x_{2} \\
\vdots \\
x_{n}
\end{array}\right)=\boldsymbol{P}\left(\begin{array}{c}
y_{1} \\
y_{2} \\
\vdots \\
y_{n}
\end{array}\right)
\]

或

\[
\left(\begin{array}{c}
y_{1}  \tag{1.2.7}\\
y_{2} \\
\vdots \\
y_{n}
\end{array}\right)=\boldsymbol{P}^{-1}\left(\begin{array}{c}
x_{1} \\
x_{2} \\
\vdots \\
x_{n}
\end{array}\right)
\]

将式(1.2.6)与式(1.2.7)称为坐标变换公式。\\
例1.2.5 在 $R[x]_{n}$ 中, $1, x, x^{2}, \cdots, x^{n-1}$ 与 $1, x-a,(x-a)^{2}, \cdots,(x-a)^{n-1}$ 为两组基,求前一组基到后一组基的过渡矩阵。

解 因为

$$
\begin{aligned}
x-a= & (-a) \cdot 1+1 \cdot x \\
(x-a)^{2}= & (-a)^{2} \cdot 1-2 a \cdot x+1 \cdot x^{2} \\
(x-a)^{3}= & (-a)^{3} \cdot 1+3 a^{2} \cdot x-3 a \cdot x^{2}+x^{3} \\
\cdots \cdots & \\
(x-a)^{n-1}= & (-a)^{n-1} \cdot 1+(n-1)(-a)^{n-2} \cdot x+ \\
& \frac{(n-1)(n-2)}{2}(-a)^{n-3} \cdot x^{2}+\cdots+x^{n-1}
\end{aligned}
$$

故由 $1, x, x^{2}, \cdots, x^{n-1}$ 到 $1, x-a,(x-a)^{2}, \cdots,(x-a)^{n-1}$ 的过渡矩阵为

$$
\left[\begin{array}{cccccc}
1 & -a & (-a)^{2} & (-a)^{3} & \cdots & (-a)^{n-1} \\
0 & 1 & 2(-a) & 3(-a)^{2} & \cdots & (n-1)(-a)^{n-2} \\
0 & 0 & 1 & 3(-a) & \cdots & \frac{(n-1)(n-2)}{2}(-a)^{n-3} \\
\vdots & \vdots & \vdots & \vdots & & \vdots \\
0 & 0 & 0 & 0 & \cdots & 1
\end{array}\right]
$$

例1.2.6 在 $R^{4}$ 中,求由基 $\boldsymbol{\alpha}_{1}, \boldsymbol{\alpha}_{2}, \boldsymbol{\alpha}_{3}, \boldsymbol{\alpha}_{4}$ 到 $\boldsymbol{\beta}_{1}, \boldsymbol{\beta}_{2}, \boldsymbol{\beta}_{3}, \boldsymbol{\beta}_{4}$ 的过渡矩阵,其中

$$
\left\{\begin{array} { l } 
{ \boldsymbol { \alpha } _ { 1 } = ( 1 , 2 , - 1 , 0 ) ^ { \mathrm { T } } } \\
{ \boldsymbol { \alpha } _ { 2 } = ( 1 , - 1 , 1 , 1 ) ^ { \mathrm { T } } } \\
{ \boldsymbol { \alpha } _ { 3 } = ( - 1 , 2 , 1 , 1 ) ^ { \mathrm { T } } } \\
{ \boldsymbol { \alpha } _ { 4 } = ( - 1 , - 1 , 0 , 1 ) ^ { \mathrm { T } } }
\end{array} \quad \text { 与 } \left\{\begin{array}{l}
\boldsymbol{\beta}_{1}=(2,1,0,1)^{\mathrm{T}} \\
\boldsymbol{\beta}_{2}=(0,1,2,2)^{\mathrm{T}} \\
\boldsymbol{\beta}_{3}=(-2,1,1,2)^{\mathrm{T}} \\
\boldsymbol{\beta}_{4}=(1,3,1,2)^{\mathrm{T}}
\end{array}\right.\right.
$$

并求向量 $\boldsymbol{\xi}=\left(\boldsymbol{x}_{1}, \boldsymbol{x}_{2}, \boldsymbol{x}_{3}, \boldsymbol{x}_{4}\right)^{\mathrm{T}}$ 在 $\boldsymbol{\beta}_{1}, \boldsymbol{\beta}_{2}, \boldsymbol{\beta}_{3}, \boldsymbol{\beta}_{4}$ 下的坐标。\\
解 将矩阵 $\left[\boldsymbol{\alpha}_{1}, \boldsymbol{\alpha}_{2}, \boldsymbol{\alpha}_{3}, \boldsymbol{\alpha}_{4} \mid \boldsymbol{\beta}_{1}, \boldsymbol{\beta}_{2}, \boldsymbol{\beta}_{3}, \boldsymbol{\beta}_{4}\right]$ 作初等行变换得

$$
\begin{aligned}
& {\left[\boldsymbol{\alpha}_{1}, \boldsymbol{\alpha}_{2}, \boldsymbol{\alpha}_{3}, \boldsymbol{\alpha}_{4} \mid \boldsymbol{\beta}_{1}, \boldsymbol{\beta}_{2}, \boldsymbol{\beta}_{3}, \boldsymbol{\beta}_{4}\right] } \\
= & {\left[\begin{array}{rrrr|rrrr}
1 & 1 & -1 & -1 & 2 & 0 & -2 & 1 \\
2 & -1 & 2 & -1 & 1 & 1 & 1 & 3 \\
-1 & 1 & 1 & 0 & 0 & 2 & 1 & 1 \\
0 & 1 & 1 & 1 & 1 & 2 & 2 & 2
\end{array}\right] } \\
\longrightarrow & {\left[\begin{array}{llll|llll}
1 & 0 & 0 & 0 & 1 & 0 & 0 & 1 \\
0 & 1 & 0 & 0 & 1 & 1 & 0 & 1 \\
0 & 0 & 1 & 0 & 0 & 1 & 1 & 1 \\
0 & 0 & 0 & 1 & 0 & 0 & 1 & 0
\end{array}\right] }
\end{aligned}
$$

上式表明由基 $\boldsymbol{\alpha}_{1}, \boldsymbol{\alpha}_{2}, \boldsymbol{\alpha}_{3}, \boldsymbol{\alpha}_{4}$ 到基 $\boldsymbol{\beta}_{1}, \boldsymbol{\beta}_{2}, \boldsymbol{\beta}_{3}, \boldsymbol{\beta}_{4}$ 的关系为(为什么?)

$$
\left(\boldsymbol{\beta}_{1}, \boldsymbol{\beta}_{2}, \boldsymbol{\beta}_{3}, \boldsymbol{\beta}_{4}\right)=\left(\boldsymbol{\alpha}_{1}, \boldsymbol{\alpha}_{2}, \boldsymbol{\alpha}_{3}, \boldsymbol{\alpha}_{4}\right)\left[\begin{array}{cccc}
1 & 0 & 0 & 1 \\
1 & 1 & 0 & 1 \\
0 & 1 & 1 & 1 \\
0 & 0 & 1 & 0
\end{array}\right]
$$

所以由 $\boldsymbol{\alpha}_{1}, \boldsymbol{\alpha}_{2}, \boldsymbol{\alpha}_{3}, \boldsymbol{\alpha}_{4}$ 到 $\boldsymbol{\beta}_{1}, \boldsymbol{\beta}_{2}, \boldsymbol{\beta}_{3}, \boldsymbol{\beta}_{4}$ 的过渡矩阵为

$$
\left[\begin{array}{llll}
1 & 0 & 0 & 1 \\
1 & 1 & 0 & 1 \\
0 & 1 & 1 & 1 \\
0 & 0 & 1 & 0
\end{array}\right]
$$

设 $\boldsymbol{\xi}=\left(x_{1}, x_{2}, x_{3}, x_{4}\right)$ 在 $\boldsymbol{\beta}_{1}, \boldsymbol{\beta}_{2}, \boldsymbol{\beta}_{3}, \boldsymbol{\beta}_{4}$ 下的坐标为 $y_{1}, y_{2}, y_{3}, y_{4}$ ,即

$$
\boldsymbol{\xi}=\left(\varepsilon_{1}, \varepsilon_{2}, \varepsilon_{3}, \varepsilon_{4}\right)\left(\begin{array}{l}
x_{1} \\
x_{2} \\
x_{3} \\
x_{4}
\end{array}\right)=\left(\boldsymbol{\beta}_{1}, \boldsymbol{\beta}_{2}, \boldsymbol{\beta}_{3}, \boldsymbol{\beta}_{4}\right)\left(\begin{array}{l}
y_{1} \\
y_{2} \\
y_{3} \\
y_{4}
\end{array}\right)
$$

其中 $\varepsilon_{1}=(1,0,0,0)^{\mathrm{T}}, \varepsilon_{2}=(0,1,0,0)^{\mathrm{T}}, \varepsilon_{3}=(0,0,1,0)^{\mathrm{T}}, \varepsilon_{4}=(0,0,0,1)^{\mathrm{T}}$ 则

$$
\xi=\left(\varepsilon_{1}, \varepsilon_{2}, \varepsilon_{3}, \varepsilon_{4}\right)\left(\begin{array}{l}
x_{1} \\
x_{2} \\
x_{3} \\
x_{4}
\end{array}\right)=\left(\varepsilon_{1}, \varepsilon_{2}, \varepsilon_{3}, \varepsilon_{4}\right)\left[\begin{array}{rrrr}
2 & 0 & -2 & 1 \\
1 & 1 & 1 & 3 \\
0 & 2 & 1 & 1 \\
1 & 2 & 2 & 2
\end{array}\right]\left(\begin{array}{l}
y_{1} \\
y_{2} \\
y_{3} \\
y_{4}
\end{array}\right)
$$

于是

$$
\begin{aligned}
\left(\begin{array}{l}
y_{1} \\
y_{2} \\
y_{3} \\
y_{4}
\end{array}\right) & =\left[\begin{array}{rrrr}
2 & 0 & -2 & 1 \\
1 & 1 & 1 & 3 \\
0 & 2 & 1 & 1 \\
1 & 2 & 2 & 2
\end{array}\right]^{-1}\left(\begin{array}{l}
x_{1} \\
x_{2} \\
x_{3} \\
x_{4}
\end{array}\right) \\
& =\left[\begin{array}{cccc}
\frac{4}{13} & -\frac{6}{13} & -\frac{8}{13} & \frac{11}{13} \\
\frac{2}{13} & -\frac{3}{13} & \frac{9}{13} & -\frac{1}{13} \\
-\frac{3}{13} & -\frac{2}{13} & -\frac{7}{13} & \frac{8}{13} \\
-\frac{1}{13} & \frac{8}{13} & \frac{2}{13} & -\frac{6}{13}
\end{array}\right]\left(\begin{array}{l}
x_{1} \\
x_{2} \\
x_{3} \\
x_{4}
\end{array}\right) \\
& =\left[\begin{array}{c}
\frac{4}{13} x_{1}-\frac{6}{13} x_{2}-\frac{8}{13} x_{3}+\frac{11}{13} x_{4} \\
\frac{2}{13} x_{1}-\frac{3}{13} x_{2}+\frac{9}{13} x_{3}-\frac{1}{13} x_{4} \\
-\frac{3}{13} x_{1}-\frac{2}{13} x_{2}-\frac{7}{13} x_{3}+\frac{8}{13} x_{4} \\
-\frac{1}{13} x_{1}+\frac{8}{13} x_{2}+\frac{2}{13} x_{3}-\frac{6}{13} x_{4}
\end{array}\right]
\end{aligned}
$$

\section*{§1.3 线性子空间}
\section*{一、线性子空间的概念}
在通常的三维几何空间中,过原点的共面向量集,按几何向量的加法与数乘运

算构成一个向量空间,类似地,过原点的共线向量集也可构成一个向量空间,这些向量空间都可看成是三维几何空间的子空间。在 $n$ 维线性空间中,可以引进子空间的概念。

设 $A \in R^{m \times n}$ ,则 $A x=0$ 的解是 $n$ 维向量。因此核空间 $N(A)$(参阅例1.1.4)中的元素全是 $n$ 维向量,它也是线性空间 $R^{n}$ 中的元素。显然,$N(A) \subset R^{n}$ 。因此线性空间 $N(\boldsymbol{A})$ 可以称为线性空间 $R^{n}$ 的子空间。

定义1.3.1 设 $W$ 为数域 $F$ 上的 $n$ 维线性空间 $V$ 的子集合,若 $W$ 中元素满足\\
(1)若 $\boldsymbol{\alpha}, \boldsymbol{\beta} \in W$ ,则 $\boldsymbol{\alpha}+\boldsymbol{\beta} \in W$ ;\\
(2)若 $\boldsymbol{\alpha} \in W, \lambda \in F$ ,则 $\lambda \boldsymbol{\alpha} \in W$ .\\
则容易证明:$W$ 也构成数域 $F$ 上的线性空间。称 $W$ 是线性空间 $V$ 的一个线性子空间,简称子空间。

线性子空间本身也是一个线性空间,因此子空间也有维数、基、坐标等概念,不再一一赘述。

因为子空间 $W$ 中不可能有比 $V$ 更多的线性无关的向量,所以子空间 $W$ 的维数不能超过 $V$ 的维数,即 $\operatorname{dim} W \leqslant \operatorname{dim} V$ 。

例1.3.1 在线性空间 $V$ 中,由单个零向量" 0 "构成的集合是一个线性子空间,称为 $V$ 的零子空间。在线性空间 $V$ 中,$V$ 本身也可看成是一个线性子空间。这两个子空间称为 $V$ 的平凡子空间。一般都讨论非平凡的子空间。

在线性子空间中,十分重要的一个特例是生成子空间。设 $\alpha_{1}, \alpha_{2}, \cdots, \alpha_{s}$ 是线性空间 $V$ 中一组向量,则集合


\begin{equation*}
\operatorname{span}\left\{\boldsymbol{\alpha}_{1}, \boldsymbol{\alpha}_{2}, \cdots, \boldsymbol{\alpha}_{s}\right\}=\left\{k_{1} \boldsymbol{\alpha}_{1}+k_{2} \boldsymbol{\alpha}_{2}+\cdots+k_{s} \boldsymbol{\alpha}_{s} \mid \forall k_{i} \in \boldsymbol{F}\right\} \tag{1.3.1}
\end{equation*}


是非空集合,不难证明。\\
定理1.3.1 $\operatorname{span}\left\{\boldsymbol{\alpha}_{1}, \boldsymbol{\alpha}_{2}, \cdots, \boldsymbol{\alpha}_{s}\right\}$ 是 $V$ 的线性子空间。\\
定义1.3.2 称非空子集 $\operatorname{span}\left\{\boldsymbol{\alpha}_{1}, \boldsymbol{\alpha}_{2}, \cdots, \boldsymbol{\alpha}_{s}\right\}$ 是由向量 $\boldsymbol{\alpha}_{1}, \boldsymbol{\alpha}_{2}, \cdots, \boldsymbol{\alpha}_{s}$ 生成的生成子空间。

根据生成子空间概念及有关向量组理论,显然有下述两个定理。\\
定理1.3.2 dimspan $\left\{\boldsymbol{\alpha}_{1}, \boldsymbol{\alpha}_{2}, \cdots, \boldsymbol{\alpha}_{s}\right\}=\operatorname{rank}\left\{\boldsymbol{\alpha}_{1}, \boldsymbol{\alpha}_{2}, \cdots, \boldsymbol{\alpha}_{s}\right\}$ ,其中 $\operatorname{rank}\left\{\boldsymbol{\alpha}_{1}\right.$ , $\left.\boldsymbol{\alpha}_{2}, \cdots, \boldsymbol{\alpha}_{s}\right\}$ 是向量组 $\boldsymbol{\alpha}_{1}, \boldsymbol{\alpha}_{2}, \cdots, \boldsymbol{\alpha}_{s}$ 的秩(即 $\boldsymbol{\alpha}_{1}, \boldsymbol{\alpha}_{2}, \cdots, \boldsymbol{\alpha}_{s}$ 中极大线性无关组中向量的个数)。向量组 $\boldsymbol{\alpha}_{1}, \boldsymbol{\alpha}_{2}, \cdots, \boldsymbol{\alpha}_{s}$ 的任何一个极大线性无关组均可作为 $\operatorname{span}\left\{\boldsymbol{\alpha}_{1}\right.$ , $\left.\boldsymbol{\alpha}_{2}, \cdots, \boldsymbol{\alpha}_{s}\right\}$ 的一个基。

定理1.3.3 若 $\boldsymbol{\alpha}_{1}, \boldsymbol{\alpha}_{2}, \cdots, \boldsymbol{\alpha}_{s}$ 与 $\boldsymbol{\beta}_{1}, \boldsymbol{\beta}_{2}, \cdots, \boldsymbol{\beta}_{t}$ 都是 $n$ 维向量组,则 $\operatorname{span}\left\{\boldsymbol{\alpha}_{1}\right.$ , $\left.\boldsymbol{\alpha}_{2}, \cdots, \boldsymbol{\alpha}_{s}\right\}=\operatorname{span}\left\{\boldsymbol{\beta}_{1}, \boldsymbol{\beta}_{2}, \cdots, \boldsymbol{\beta}_{t}\right\} \Longleftrightarrow \boldsymbol{\alpha}_{1}, \boldsymbol{\alpha}_{2}, \cdots, \boldsymbol{\alpha}_{s}$ 与 $\boldsymbol{\beta}_{1}, \boldsymbol{\beta}_{2}, \cdots, \boldsymbol{\beta}_{t}$ 等价,(即 $\boldsymbol{\alpha}_{1}$ , $\boldsymbol{\alpha}_{2}, \cdots, \boldsymbol{\alpha}_{s}$ 与 $\boldsymbol{\beta}_{1}, \boldsymbol{\beta}_{2}, \cdots, \boldsymbol{\beta}_{t}$ 可以互相线性表示).

例1.3.2 设 $\boldsymbol{\alpha}_{1}=(1,2,-1,0)^{\mathrm{T}}, \boldsymbol{\alpha}_{2}=(0,1,2,3)^{\mathrm{T}}, \boldsymbol{\alpha}_{3}=(2,3,-4,-3)^{\mathrm{T}}$ .求 $\operatorname{span}\left\{\alpha_{1}, \alpha_{2}, \alpha_{3}\right\}$ 的基与维数。

解 不难验证: $\boldsymbol{\alpha}_{1}, \boldsymbol{\alpha}_{2}$ 是线性无关的,且

$$
\alpha_{3}=2 \alpha_{1}-\alpha_{2}
$$

所以 $\boldsymbol{\alpha}_{1}, \boldsymbol{\alpha}_{2}$ 为 $\operatorname{span}\left\{\boldsymbol{\alpha}_{1}, \boldsymbol{\alpha}_{2}, \boldsymbol{\alpha}_{3}\right\}$ 的基,dimspan $\left\{\boldsymbol{\alpha}_{1}, \boldsymbol{\alpha}_{2}, \boldsymbol{\alpha}_{3}\right\}=2$ ,显然 $\operatorname{span}\left\{\boldsymbol{\alpha}_{1}, \boldsymbol{\alpha}_{2}, \boldsymbol{\alpha}_{3}\right\}=\operatorname{span}\left\{\boldsymbol{\alpha}_{1}, \boldsymbol{\alpha}_{2}\right\}$.

\section*{二、子空间的交、和}
定义1.3.3 设 $V_{1}, V_{2}$ 是线性空间 $V$ 的两个子空间,命

$$
V_{1} \cap V_{2}=\left\{\boldsymbol{\alpha} \mid \boldsymbol{\alpha} \in V_{1} \text { 且 } \boldsymbol{\alpha} \in V_{2}\right\}
$$

可以验证:$V_{1} \cap V_{2}$ 构成 $V$ 的线性子空间.称 $V_{1} \cap V_{2}$ 为 $V_{1}$ 与 $V_{2}$ 的交空间.\\
命

$$
\left.V_{1}+V_{2}=\left|\boldsymbol{\alpha}=\boldsymbol{\alpha}_{1}+\boldsymbol{\alpha}_{2}\right| \boldsymbol{\alpha}_{1} \in V_{1} \text { 且 } \boldsymbol{\alpha}_{2} \in V_{2}\right\}
$$

可以验证:$V_{1}+V_{2}$ 构成 $V$ 的线性子空间.称 $V_{1}+V_{2}$ 为 $V_{1}$ 与 $V_{2}$ 的和空间.\\
例1.3.3 设 $\boldsymbol{A} \in C^{m \times n}, \boldsymbol{B} \in C^{p \times n}$ ,则 $N(\boldsymbol{A}) \cap N(\boldsymbol{B})$ 是方程组

$$
\binom{\boldsymbol{A}}{\boldsymbol{B}} \boldsymbol{x}=0
$$

的解空间.\\
例1.3.4 在三维几何空间中,用 $V_{1}$ 表示过原点与某给定向量共线的向量集合, $V_{2}$ 表示过原点并与 $V_{1}$ 垂直的共面向量集合。则 $V_{1}+V_{2}$ 是整个空间,且 $V_{1} \cap V_{2}=\{0\}$ 。

关于两个生成子空间的和空间有下面定理:\\
定理1.3.4 设 $V_{1}=\operatorname{span}\left\{\boldsymbol{\alpha}_{1}, \boldsymbol{\alpha}_{2}, \cdots, \boldsymbol{\alpha}_{s}\right\}$

$$
V_{2}=\operatorname{span}\left\{\boldsymbol{\beta}_{1}, \boldsymbol{\beta}_{2}, \cdots, \boldsymbol{\beta}_{t}\right\}
$$

则

$$
V_{1}+V_{2}=\operatorname{span}\left\{\boldsymbol{\alpha}_{1}, \cdots, \boldsymbol{\alpha}_{s}, \boldsymbol{\beta}_{1}, \cdots, \boldsymbol{\beta}_{t}\right\}
$$

例1.3.5 设

$$
\begin{aligned}
& \boldsymbol{\alpha}_{1}=(2,1,3,1)^{\mathrm{T}}, \boldsymbol{\alpha}_{2}=(-1,1,-3,1)^{\mathrm{T}}, \boldsymbol{\beta}_{1}=(4,5,3,-1)^{\mathrm{T}}, \\
& \boldsymbol{\beta}_{2}=(1,5,-3,1)^{\mathrm{T}}, V_{1}=\operatorname{span}\left\{\boldsymbol{\alpha}_{1}, \boldsymbol{\alpha}_{2}\right\}, V_{2}=\operatorname{span}\left\{\boldsymbol{\beta}_{1}, \boldsymbol{\beta}_{2}\right\}
\end{aligned}
$$

试求:(1)$V_{1}+V_{2}$ 的基与维数;\\
(2)$V_{1} \cap V_{2}$ 的基与维数.\\
解(1)由定理 1.3.3 知

$$
V_{1}+V_{2}=\operatorname{span}\left\{\boldsymbol{\alpha}_{1}, \boldsymbol{\alpha}_{2}, \boldsymbol{\beta}_{1}, \boldsymbol{\beta}_{2}\right\}
$$

$\boldsymbol{\alpha}_{1}, \boldsymbol{\alpha}_{2}, \boldsymbol{\beta}_{1}$ 是向量组 $\boldsymbol{\alpha}_{1}, \boldsymbol{\alpha}_{2}, \boldsymbol{\beta}_{1}, \boldsymbol{\beta}_{2}$ 的极大无关组,故它是 $V_{1}+V_{2}$ 的基, $\operatorname{dim}\left(V_{1}+V_{2}\right)=3$.\\
(2)设 $\boldsymbol{\alpha} \in V_{1} \cap V_{2}$ ,即 $\boldsymbol{\alpha} \in V_{1}$ 且 $\boldsymbol{\alpha} \in V_{2}$ ,于是

$$
\boldsymbol{\alpha}=k_{1} \boldsymbol{\alpha}_{1}+k_{2} \boldsymbol{\alpha}_{2}=k_{3} \boldsymbol{\beta}_{1}+k_{4} \boldsymbol{\beta}_{2}
$$

将 $\boldsymbol{\alpha}_{1}, \boldsymbol{\alpha}_{2}, \boldsymbol{\beta}_{1}, \boldsymbol{\beta}_{2}$ 的坐标代人上式,解之得

$$
k_{1}=0, \quad k_{2}=\frac{5}{3} k_{4}, \quad k_{3}=-\frac{2}{3} k_{4}
$$

于是

$$
\boldsymbol{\alpha}=k_{1} \boldsymbol{\alpha}_{1}+k_{2} \boldsymbol{\alpha}_{2}=k_{4}\left(-\frac{5}{3}, \frac{5}{3},-5, \frac{5}{3}\right)^{\mathrm{T}}
$$

所以 $V_{1} \cap V_{2}$ 的基为 $\left(-\frac{5}{3}, \frac{5}{3},-5, \frac{5}{3}\right)^{\mathrm{T}}$ ,维数为 1 。\\
又解交空间 $V_{1} \cap V_{2}$ 的向量实质上就是求在 $V_{2}$ 中向量 $k_{1} \boldsymbol{\beta}_{1}+k_{2} \boldsymbol{\beta}_{2}$ 也能由 $\boldsymbol{\alpha}_{1}$ , $\alpha_{2}$ 线性表示的这部分向量,即确定 $k_{1}, k_{2}$ 使得

$$
\text { 秩 }\left(\boldsymbol{\alpha}_{1}, \boldsymbol{\alpha}_{2}, k_{1} \boldsymbol{\beta}_{1}+k_{2} \boldsymbol{\beta}_{2}\right)=\text { 秩 }\left(\boldsymbol{\alpha}_{1}, \boldsymbol{\alpha}_{2}\right)
$$

此即

$$
\left[\begin{array}{rrc}
2 & -1 & 4 k_{1}+k_{2} \\
1 & 1 & 5 k_{1}+5 k_{2} \\
3 & -3 & 3 k_{1}-3 k_{2} \\
1 & 1 & -k_{1}+k_{2}
\end{array}\right] \longrightarrow\left[\begin{array}{ccc}
1 & 1 & 5 k_{1}+5 k_{2} \\
0 & 1 & 2 k_{1}+3 k_{2} \\
0 & 0 & 3 k_{1}+2 k_{2} \\
0 & 0 & 0
\end{array}\right]
$$

于是

$$
3 k_{1}+2 k_{2}=0, k_{1}=-\frac{2}{3} k_{2}
$$

代人

$$
\begin{aligned}
k_{1} \boldsymbol{\beta}_{1}+k_{2} \boldsymbol{\beta}_{2} & =k_{2}\left(-\frac{2}{3} \boldsymbol{\beta}_{1}+\boldsymbol{\beta}_{2}\right) \\
& =k_{2}\left(-\frac{5}{3}, \frac{5}{3},-5, \frac{5}{3}\right)^{\mathrm{T}}
\end{aligned}
$$

所以 $V_{1} \cap V_{2}$ 的基为 $\left(-\frac{5}{3}, \frac{5}{3},-5, \frac{5}{3}\right)^{\mathrm{T}}, \operatorname{dim}\left(V_{1} \cap V_{2}\right)=1$ .\\
例1.3.6 已知 $V_{1}$ 与 $V_{2}$ 分别是方程组(I)与方程组(II)的解空间:


\begin{gather*}
\left\{\begin{array}{l}
x_{1}+x_{2}-3 x_{4}-x_{5}=0 \\
x_{1}-x_{2}+2 x_{3}-x_{4}=0
\end{array}\right.  \tag{I}\\
\left\{\begin{array}{l}
4 x_{1}-2 x_{2}+6 x_{3}+3 x_{4}-4 x_{5}=0 \\
2 x_{1}+4 x_{2}-2 x_{3}+4 x_{4}-7 x_{5}=0
\end{array}\right. \tag{II}
\end{gather*}


求交空间 $V_{1} \cap V_{2}$ 的基与维数。\\
解 方程组(I)与(II)的交空间就是这两个方程组的所有公共解所构成的空间,此即方程组

$$
\left\{\begin{array}{r}
x_{1}+x_{2}-3 x_{4}-x_{5}=0 \\
x_{1}-x_{2}+2 x_{3}-x_{4}=0 \\
4 x_{1}-2 x_{2}+6 x_{3}+3 x_{4}-4 x_{5}=0 \\
2 x_{1}+4 x_{2}-2 x_{3}+4 x_{4}-7 x_{5}=0
\end{array}\right.
$$

的解空间。容易求得该方程组的基础解系为 $(-1,1,1,0,0)^{\mathrm{T}},(12,0,-5,2,6)^{\mathrm{T}}$ ,

它就是所求 $V_{1} \cap V_{2}$ 的基, $\operatorname{dim}\left(V_{1} \cap V_{2}\right)=2$ .\\
例1.3.7 设

$$
\begin{array}{ll}
\boldsymbol{\alpha}_{1}=(1,0,2,0)^{\mathrm{T}}, & \boldsymbol{\alpha}_{2}=(0,1,-1,1)^{\mathrm{T}} \\
\boldsymbol{\beta}_{1}=(1,2,0,0)^{\mathrm{T}}, & \boldsymbol{\beta}_{2}=(0,3,-3,1)^{\mathrm{T}} \\
V_{1}=\operatorname{span}\left\{\boldsymbol{\alpha}_{1}, \boldsymbol{\alpha}_{2}\right\}, & V_{2}=\operatorname{span}\left\{\boldsymbol{\beta}_{1}, \boldsymbol{\beta}_{2}\right\}
\end{array}
$$

试求:(1)$V_{1} \cap V_{2}$ 的基与维数;(2)$V_{1}+V_{2}$ 的基与维数。\\
解(1)不难看出 $\boldsymbol{\alpha}_{1}, \boldsymbol{\alpha}_{2}$ 是线性齐次方程组(I)

\[
\left\{\begin{array}{l}
x_{3}=2 x_{1}-x_{2}  \tag{I}\\
x_{4}=x_{2}
\end{array}\right.
\]

的基础解系,方程组(I)的解空间为 $V_{1}$ .而 $\boldsymbol{\beta}_{1}, \boldsymbol{\beta}_{2}$ 是线性齐次方程组(II)

\[
\left\{\begin{array}{l}
x_{2}=2 x_{1}+3 x_{4}  \tag{II}\\
x_{3}=-3 x_{4}
\end{array}\right.
\]

的基础解系,方程组(II)的解空间为 $V_{2}$ .\\
交空间 $V_{1} \cap V_{2}$ 实质上是(I)与(II)公共解的解空间,即方程组

\[
\left\{\begin{array}{l}
x_{3}=2 x_{1}-x_{2}  \tag{III}\\
x_{4}=x_{2} \\
x_{2}=2 x_{1}+3 x_{4} \\
x_{3}=-3 x_{4}
\end{array}\right.
\]

的解空间。不难求得方程组(III)的基础解系为 $(-1,1,-3,1)^{\mathrm{T}}$ ,此即 $V_{1} \cap V_{2}$ 的基,维数为 1 .\\
(2)$V_{1}+V_{2}=\operatorname{span}\left\{\boldsymbol{\alpha}_{1}, \boldsymbol{\alpha}_{2}, \boldsymbol{\beta}_{1}, \boldsymbol{\beta}_{2}\right\}=\operatorname{span}\left\{\boldsymbol{\alpha}_{1}, \boldsymbol{\alpha}_{2}, \boldsymbol{\beta}_{1}\right\}$

$$
=\operatorname{span}\left\{\boldsymbol{\alpha}_{1}, \boldsymbol{\alpha}_{2}, \boldsymbol{\beta}_{2}\right\}=\operatorname{span}\left\{\boldsymbol{\alpha}_{2}, \boldsymbol{\beta}_{1}, \boldsymbol{\beta}_{2}\right\}
$$

所以 $\operatorname{dim}\left(V_{1}+V_{2}\right)=3$ ,基为 $\boldsymbol{\alpha}_{1}, \boldsymbol{\alpha}_{2}, \boldsymbol{\beta}_{1}$ .\\
定理1.3.5(维数公式)设 $V_{1}$ 与 $V_{2}$ 是线性空间 $V$ 的两个子空间,则


\begin{equation*}
\operatorname{dim} V_{1}+\operatorname{dim} V_{2}=\operatorname{dim}\left(V_{1}+V_{2}\right)+\operatorname{dim}\left(V_{1} \cap V_{2}\right) . \tag{1.3.2}
\end{equation*}


证明 设 $\operatorname{dim} V_{1}=n_{1}, \operatorname{dim} V_{2}=n_{2}, \operatorname{dim}\left(V_{1} \cap V_{2}\right)=m$ .取 $V_{1} \cap V_{2}$ 的一组基

$$
\boldsymbol{\alpha}_{1}, \boldsymbol{\alpha}_{2}, \cdots, \boldsymbol{\alpha}_{m}
$$

它可以扩充成 $V_{1}$ 的一组基

$$
\boldsymbol{\alpha}_{1}, \boldsymbol{\alpha}_{2}, \cdots, \boldsymbol{\alpha}_{m}, \boldsymbol{\beta}_{1}, \boldsymbol{\beta}_{2}, \cdots, \boldsymbol{\beta}_{n_{1}-m}
$$

也可以扩充成 $V_{2}$ 的一组基

$$
\boldsymbol{\alpha}_{1}, \boldsymbol{\alpha}_{2}, \cdots, \boldsymbol{\alpha}_{m}, \boldsymbol{\nu}_{1}, \boldsymbol{\nu}_{2}, \cdots, \boldsymbol{\nu}_{n_{2}-m}
$$

此即

$$
\begin{aligned}
& V_{1}=\operatorname{span}\left\{\boldsymbol{\alpha}_{1}, \boldsymbol{\alpha}_{2}, \cdots, \boldsymbol{\alpha}_{m}, \boldsymbol{\beta}_{1}, \boldsymbol{\beta}_{2}, \cdots, \boldsymbol{\beta}_{n_{1}-m}\right\} \\
& V_{2}=\operatorname{span}\left\{\boldsymbol{\alpha}_{1}, \boldsymbol{\alpha}_{2}, \cdots, \boldsymbol{\alpha}_{m}, \boldsymbol{\nu}_{1}, \boldsymbol{\nu}_{2}, \cdots, \boldsymbol{\nu}_{n_{2}-m}\right\}
\end{aligned}
$$

设

$$
V_{1}+V_{2}=\operatorname{span}\left\{\boldsymbol{\alpha}_{1}, \boldsymbol{\alpha}_{2}, \cdots, \boldsymbol{\alpha}_{m}, \boldsymbol{\beta}_{1}, \boldsymbol{\beta}_{2}, \cdots, \boldsymbol{\beta}_{n_{1}-m}, \boldsymbol{\nu}_{1}, \boldsymbol{\nu}_{2}, \cdots, \boldsymbol{\nu}_{n_{2}-m}\right\}
$$

命

$$
\begin{gathered}
k_{1} \boldsymbol{\alpha}_{1}+\cdots+k_{m} \boldsymbol{\alpha}_{m}+p_{1} \boldsymbol{\beta}_{1}+\cdots+p_{n_{1}-m} \boldsymbol{\beta}_{n_{1}-m}+ \\
q_{1} \boldsymbol{\nu}_{1}+\cdots+q_{n_{2}-m} \boldsymbol{\nu}_{n_{2}-m}=0
\end{gathered}
$$

命

$$
\begin{aligned}
\boldsymbol{\xi} & =k_{1} \boldsymbol{\alpha}_{1}+\cdots+k_{m} \boldsymbol{\alpha}_{m}+p_{1} \boldsymbol{\beta}_{1}+\cdots+p_{n_{1}-m} \boldsymbol{\beta}_{n_{1}-m} \\
& =-q_{1} \boldsymbol{\nu}_{1}-\cdots-q_{n_{2}-m} \boldsymbol{\nu}_{n_{2}-m}
\end{aligned}
$$

由第一个等式知 $\boldsymbol{\xi} \in V_{1}$ ,由第二个等式知 $\boldsymbol{\xi} \in V_{2}$ ,于是 $\boldsymbol{\xi} \in V_{1} \cap V_{2}$ ,故可令

$$
\boldsymbol{\xi}=l_{1} \boldsymbol{\alpha}_{1}+\cdots+l_{m} \boldsymbol{\alpha}_{m}
$$

因此

$$
l_{1} \boldsymbol{\alpha}_{1}+\cdots+l_{m} \boldsymbol{\alpha}_{m}=-q_{1} \boldsymbol{\nu}_{1}-\cdots-q_{n_{2}-m} \boldsymbol{\nu}_{n_{2}-m}
$$

即

$$
l_{1} \boldsymbol{\alpha}_{1}+\cdots+l_{m} \boldsymbol{\alpha}_{m}+q_{1} \boldsymbol{\nu}_{1}+\cdots+q_{n_{2}-m} \boldsymbol{\nu}_{n_{2}-m}=0
$$

由于 $\boldsymbol{\alpha}_{1}, \boldsymbol{\alpha}_{2}, \cdots, \boldsymbol{\alpha}_{m}, \boldsymbol{\nu}_{1}, \cdots, \boldsymbol{\nu}_{n_{2}-m}$ 线性无关,所以

$$
l_{1}=\cdots=l_{m}=q_{1}=\cdots=q_{n_{2}-m}=0
$$

因而 $\boldsymbol{\xi}=0$ ,从而有

$$
k_{1} \boldsymbol{\alpha}_{1}+\cdots+k_{m} \boldsymbol{\alpha}_{m}+p_{1} \boldsymbol{\beta}_{1}+\cdots+p_{n_{1}-m} \boldsymbol{\beta}_{n_{1}-m}=0
$$

由于 $\boldsymbol{\alpha}_{1}, \cdots, \boldsymbol{\alpha}_{m}, \boldsymbol{\beta}_{1}, \cdots, \boldsymbol{\beta}_{n_{1}-m}$ 线性无关,又得

$$
k_{1}=\cdots=k_{m}=p_{1}=\cdots=p_{n_{1}-m}=0
$$

这就证明了 $\boldsymbol{\alpha}_{1}, \cdots, \boldsymbol{\alpha}_{m}, \boldsymbol{\beta}_{1}, \cdots, \boldsymbol{\beta}_{n_{1}-m}, \boldsymbol{\nu}_{1}, \cdots, \boldsymbol{\nu}_{n_{2}-m}$ 线性无关,因而它是 $V_{1}+V_{2}$ 的一组基,$V_{1}+V_{2}$ 的维数为 $n_{1}+n_{2}-m$ .于是维数公式成立.

\section*{三、子空间的直和、补子空间}
定义1.3.4 设 $V_{1}, V_{2}$ 是线性空间 $V$ 的两个子空间,若 $V_{1} \cap V_{2}=\{0\}$ ,则称 $V_{1}$与 $V_{2}$ 的和空间 $V_{1}+V_{2}$ 是直和,并用记号 $V_{1} \oplus V_{2}$ 表示。

定理1.3.6 设 $V_{1}, V_{2}$ 是线性空间 $V$ 的两个子空间,则下列命题是等价的:\\
(1)$V_{1}+V_{2}$ 是直和\\
(2) $\operatorname{dim}\left(V_{1}+V_{2}\right)=\operatorname{dim} V_{1}+\operatorname{dim} V_{2}$\\
(3)设 $\boldsymbol{\alpha}_{1}, \cdots, \boldsymbol{\alpha}_{n_{1}}$ 是 $V_{1}$ 的一组基, $\boldsymbol{\beta}_{1}, \boldsymbol{\beta}_{2}, \cdots, \boldsymbol{\beta}_{n_{2}}$ 是 $V_{2}$ 的一组基,则 $\boldsymbol{\alpha}_{1}, \cdots$ , $\boldsymbol{\alpha}_{n_{1}}, \boldsymbol{\beta}_{1}, \cdots, \boldsymbol{\beta}_{n_{2}}$ 是 $V_{1}+V_{2}$ 的一组基.

证明 $(1) \Longleftrightarrow(2) \quad$ 显然。\\
(2)$\Rightarrow$(3)设 $\operatorname{dim}\left(V_{1}+V_{2}\right)=\operatorname{dim} V_{1}+\operatorname{dim} V_{2}=n_{1}+n_{2}$ ,由定理1.3.4知

$$
V_{1}+V_{2}=\operatorname{span}\left\{\boldsymbol{\alpha}_{1}, \cdots, \boldsymbol{\alpha}_{n_{1}}, \boldsymbol{\beta}_{1}, \cdots, \boldsymbol{\beta}_{n_{2}}\right\}
$$

又由定理1.3.2知

$$
\operatorname{rank}\left\{\boldsymbol{\alpha}_{1}, \cdots, \boldsymbol{\alpha}_{n_{1}}, \boldsymbol{\beta}_{1}, \cdots, \boldsymbol{\beta}_{n_{2}}\right\}=\operatorname{dim}\left(V_{1}+V_{2}\right)=n_{1}+n_{2}
$$

因此 $\boldsymbol{\alpha}_{1}, \cdots, \boldsymbol{\alpha}_{n_{1}}, \boldsymbol{\beta}_{1}, \cdots, \boldsymbol{\beta}_{n_{2}}$ 线性无关,所以它构成 $V_{1}+V_{2}$ 的一组基。\\
(3)$\Rightarrow$(2)因为 $\boldsymbol{\alpha}_{1}, \cdots, \boldsymbol{\alpha}_{n_{1}}, \boldsymbol{\beta}_{1}, \cdots, \boldsymbol{\beta}_{n_{2}}$ 构成 $V_{1}+V_{2}$ 的一组基,故

$$
\operatorname{rank}\left(\boldsymbol{\alpha}_{1}, \cdots, \boldsymbol{\alpha}_{n_{1}}, \boldsymbol{\beta}_{1}, \cdots, \boldsymbol{\beta}_{n_{2}}\right)=n_{1}+n_{2}
$$

于是

$$
\operatorname{dim}\left(V_{1}+V_{2}\right)=\operatorname{dim} \operatorname{span}\left\{\boldsymbol{\alpha}_{1}, \cdots, \boldsymbol{\alpha}_{n_{1}}, \boldsymbol{\beta}_{1}, \cdots, \boldsymbol{\beta}_{n_{2}}\right\}=n_{1}+n_{2}
$$

根据维数公式得

$$
\operatorname{dim}\left(V_{1} \cap V_{2}\right)=0
$$

于是 $V_{1}+V_{2}$ 是直和.\\
定义1.3.5 设 $W, W_{1}, W_{2}$ 是线性空间 $V$ 的三个子空间,且

$$
W=W_{1} \oplus W_{2}
$$

则称 $W$ 有一个直和分解。\\
特别地,若

$$
W=V=W_{1} \oplus W_{2}
$$

便称 $W_{1}$ 和 $W_{2}$ 是线性空间 $V$ 一对互补的子空间,或称 $W_{1}$ 是 $W_{2}$ 的代数补(也可称 $W_{2}$ 是 $W_{1}$ 的代数补)。

定理1.3.7 设 $U$ 是线性空间 $V$ 的一个子空间,则一定存在 $U$ 的代数补子空间 $W$ ,使得

$$
V=U \oplus W
$$

例1.3.8 子空间 $U$ 的代数补不是唯一的.例如,若

$$
\boldsymbol{\alpha}_{1}=(1,0,0)^{\mathrm{T}}, \quad \boldsymbol{\alpha}_{2}=(0,1,0)^{\mathrm{T}}
$$

显然,$U=\operatorname{span}\left\{\boldsymbol{\alpha}_{1}, \boldsymbol{\alpha}_{2}\right\}$ 是 $R^{3}$ 的一个子空间,若令 $\boldsymbol{\alpha}_{3}=(0,0,1)^{\mathrm{T}}$ ,或 $\boldsymbol{\alpha}_{4}=(0,1,1)^{\mathrm{T}}$则 $\operatorname{span}\left\{\boldsymbol{\alpha}_{3}\right\}$ 或 $\operatorname{span}\left\{\boldsymbol{\alpha}_{4}\right\}$ 就是 $U$ 的两个不同代数补。请读者对此例作几何解释。

\section*{§1.4 线 性 映 射}
\section*{一、线性映射定义}
定义1.4.1 设 $V_{1}, V_{2}$ 是数域 $F$ 上两个线性空间,映射, $\mathscr{A}: V_{1} \rightarrow V_{2}$ ,如果对于任何两个向量 $\boldsymbol{\alpha}_{1}, \boldsymbol{\alpha}_{2} \in V_{1}$ 和任何数 $\lambda \in F$ ,都有

$$
\begin{aligned}
& \mathscr{A}\left(\boldsymbol{\alpha}_{1}+\boldsymbol{\alpha}_{2}\right)=\mathscr{B}\left(\boldsymbol{\alpha}_{1}\right)+\mathscr{A}\left(\boldsymbol{\alpha}_{2}\right) ; \\
& \mathscr{A}\left(\lambda \boldsymbol{\alpha}_{1}\right)=\lambda \mathscr{B}\left(\boldsymbol{\alpha}_{1}\right)
\end{aligned}
$$

便称映射 $\mathscr{A}$ 是由 $V_{1}$(定义域)到 $V_{2}$(像集)的线性映射.称 $\boldsymbol{\alpha}_{1}$ 为 $\mathscr{A}\left(\boldsymbol{\alpha}_{1}\right)$ 的原像, $\mathscr{B}\left(\alpha_{1}\right)$ 为 $\alpha_{1}$ 的像。

例1.4.1 设映射 $\mathscr{A}: ~ V \rightarrow V$ 由下式确定

$$
\mathscr{A}(\alpha)=\alpha \in V, \quad \forall \alpha \in V
$$

易证 $\mathscr{A}$ 是线性映射,称它为恒等映射,用 $E$ 记之。\\
设映射 $\mathscr{A}: V_{1} \rightarrow V_{2}$ 由下式确定

$$
\mathscr{A}(\alpha)=0, \quad \forall \alpha \in V_{1}
$$

易证 $\mathscr{A}$ 是线性映射,称它为零映射,用 0 记之。\\
例1.4.2 设 $\boldsymbol{B}=\left(b_{i j}\right)$ 是 $m \times n$ 实矩阵,若映射 $\mathscr{A}: R^{n} \rightarrow R^{m}$ 由下式确定

$$
\mathscr{A}(\boldsymbol{\alpha})=\boldsymbol{B} \boldsymbol{\alpha}, \forall \boldsymbol{\alpha} \in R^{n}
$$

则不难验证 $\mathscr{A}$ 是线性映射。\\
例1.4.3 设映射 $\mathscr{D}: R[x]_{n+1} \rightarrow R[x]_{n}$ 由下式确定

$$
\mathscr{D}(f(x))=\frac{\mathrm{d}}{\mathrm{~d} x} f(x), \quad \forall f(x) \in R[x]_{n+1} .
$$

不难验证, $\mathscr{D}$ 是线性映射。\\
例1.4.4 设映射 $S: R[x]_{n} \rightarrow R[x]_{n+1}$ 由下式确定

$$
S(f(x))=\int_{0}^{x} f(t) \mathrm{d} t, \quad f(x) \in R[x]_{n}
$$

不难验证,$S$ 是线性映射。\\
线性映射简单性质:\\
(1) $\mathscr{B}(0)=0$\\
(2) $\mathscr{A}\left(\sum_{i=1}^{s} k_{i} \boldsymbol{\alpha}_{i}\right)=\sum_{i=1}^{s} k_{i}, \mathscr{A}\left(\boldsymbol{\alpha}_{i}\right), \quad\left(\boldsymbol{\alpha}_{i} \in V, k_{i} \in F\right)$\\
(3)设 $\boldsymbol{\alpha}_{1}, \boldsymbol{\alpha}_{2}, \cdots, \boldsymbol{\alpha}_{s} \in V_{1}, \boldsymbol{\alpha}_{1}, \boldsymbol{\alpha}_{2}, \cdots, \boldsymbol{\alpha}_{s}$ 线性相关,则 $\mathscr{A}\left(\boldsymbol{\alpha}_{1}\right), \mathscr{A}\left(\boldsymbol{\alpha}_{2}\right), \cdots$ , $\mathscr{A}\left(\boldsymbol{\alpha}_{s}\right)$ 也线性相关。

前两个性质证明从略.现证明性质(3)。\\
事实上,由于 $\boldsymbol{\alpha}_{1}, \boldsymbol{\alpha}_{2}, \cdots, \boldsymbol{\alpha}_{s}$ 线性相关,不失一般性,不妨设

$$
\boldsymbol{\alpha}_{s}=k_{1} \boldsymbol{\alpha}_{1}+k_{2} \boldsymbol{\alpha}_{2}+\cdots+k_{s-1} \boldsymbol{\alpha}_{s-1}
$$

于是

$$
\mathscr{A}\left(\boldsymbol{\alpha}_{s}\right)=k_{1} \mathscr{A}\left(\boldsymbol{\alpha}_{1}\right)+k_{2} \mathscr{A}\left(\boldsymbol{\alpha}_{2}\right)+\cdots+k_{s-1} \mathscr{B}\left(\boldsymbol{\alpha}_{s-1}\right)
$$

因此, $\mathscr{A}\left(\boldsymbol{\alpha}_{1}\right), \mathscr{A}\left(\boldsymbol{\alpha}_{2}\right), \cdots, \mathscr{A}\left(\boldsymbol{\alpha}_{s-1}\right), \mathscr{B}\left(\boldsymbol{\alpha}_{s}\right)$ 线性相关.\\
特别要注意,若 $\boldsymbol{\alpha}_{1}, \boldsymbol{\alpha}_{2}, \cdots, \boldsymbol{\alpha}_{s} \in V_{1}$ 线性无关,则 $\mathscr{B}\left(\boldsymbol{\alpha}_{1}\right), \mathscr{A}\left(\boldsymbol{\alpha}_{2}\right), \cdots, \mathscr{B}\left(\boldsymbol{\alpha}_{s}\right)$不一定线性无关。

例1.4.5 设线性映射 $P: R^{3} \rightarrow R^{2}$ 由下式确定

$$
\boldsymbol{\alpha}=\left(a_{1}, a_{2}, a_{3}\right)^{\mathrm{T}} \in R^{3} \rightarrow P(\boldsymbol{\alpha})=\left(a_{1}, a_{2}\right)^{\mathrm{T}} \in R^{2}
$$

容易验证:$R^{3}$ 中的三个线性无关向量

$$
\boldsymbol{\alpha}_{1}=(1,1,1)^{\mathrm{T}}, \boldsymbol{\alpha}_{2}=(1,1,0)^{\mathrm{T}}, \boldsymbol{\alpha}_{3}=(1,0,0)^{\mathrm{T}}
$$

的像

$$
P\left(\boldsymbol{\alpha}_{1}\right)=(1,1)^{\mathrm{T}}, P\left(\boldsymbol{\alpha}_{2}\right)=(1,1)^{\mathrm{T}}, P\left(\boldsymbol{\alpha}_{3}\right)=(1,0)^{\mathrm{T}}
$$

是线性相关的.

\section*{二、线性映射的矩阵表示}
设 $\boldsymbol{\alpha}_{1}, \boldsymbol{\alpha}_{2}, \cdots, \boldsymbol{\alpha}_{n}$ 是 $V_{1}$ 的一组基, $\boldsymbol{\beta}_{1}, \boldsymbol{\beta}_{2}, \cdots, \boldsymbol{\beta}_{m}$ 是 $V_{2}$ 的一组基. $\mathscr{A}$ 是 $V_{1} \rightarrow V_{2}$的一个线性映射,则

$$
\mathscr{A}\left(\boldsymbol{\alpha}_{j}\right)=\sum_{i=1}^{m} a_{i j} \boldsymbol{\beta}_{i} \quad(j=1,2, \cdots, n)
$$

或写成


\begin{align*}
& \mathscr{A}\left(\boldsymbol{\alpha}_{1}, \boldsymbol{\alpha}_{2}, \cdots, \boldsymbol{\alpha}_{n}\right) \\
= & \left(\mathscr{A}\left(\boldsymbol{\alpha}_{1}\right), \mathscr{A}\left(\boldsymbol{\alpha}_{2}\right), \cdots, \mathscr{B}\left(\boldsymbol{\alpha}_{n}\right)\right) \\
= & \left(\sum_{i=1}^{m} a_{i 1} \boldsymbol{\beta}_{i}, \sum_{i=1}^{m} a_{i 2} \boldsymbol{\beta}_{i}, \cdots, \sum_{i=1}^{m} a_{i n} \boldsymbol{\beta}_{i}\right) \\
= & \left(\boldsymbol{\beta}_{1}, \boldsymbol{\beta}_{2}, \cdots, \boldsymbol{\beta}_{m}\right)\left[\begin{array}{cccc}
a_{11} & a_{12} & \cdots & a_{1 n} \\
a_{21} & a_{22} & \cdots & a_{2 n} \\
\vdots & \vdots & & \vdots \\
a_{m 1} & a_{m 2} & \cdots & a_{m n}
\end{array}\right] \tag{1.4.1}
\end{align*}


令

\[
\boldsymbol{A}=\left[\begin{array}{cccc}
a_{11} & a_{12} & \cdots & a_{1 n}  \tag{1.4.2}\\
a_{21} & a_{22} & \cdots & a_{2 n} \\
\vdots & \vdots & & \vdots \\
a_{m 1} & a_{m 2} & \cdots & a_{m n}
\end{array}\right]
\]

把它代人式(1.4.1)得


\begin{equation*}
\mathscr{B}\left(\boldsymbol{\alpha}_{1}, \boldsymbol{\alpha}_{2}, \cdots, \boldsymbol{\alpha}_{n}\right)=\left(\boldsymbol{\beta}_{1}, \boldsymbol{\beta}_{2}, \cdots, \boldsymbol{\beta}_{m}\right) \boldsymbol{A} \tag{1.4.3}
\end{equation*}


矩阵 $\boldsymbol{A}$ 称为线性映射 $\mathscr{A}$ 在基 $\left(\boldsymbol{\alpha}_{1}, \boldsymbol{\alpha}_{2}, \cdots, \boldsymbol{\alpha}_{n}\right)$ 与基 $\left(\boldsymbol{\beta}_{1}, \boldsymbol{\beta}_{2}, \cdots, \boldsymbol{\beta}_{m}\right)$ 下的矩阵表示。\\
显然, $\boldsymbol{A}$ 在确定一对基下的矩阵表示 $\boldsymbol{A}$ 是唯一的。在不同基下的矩阵表示是不一样的。

有了线性映射 $\mathscr{A}$ 在一对基下的矩阵表示 $\boldsymbol{A}$ 之后,可以得到定义域 $V_{1}$ 中向量 $\boldsymbol{\alpha}$ 与它在像空间 $V_{2}$ 中的像, $\mathscr{B}(\boldsymbol{\alpha})$ 之间的坐标关系。

设 $\alpha \in V_{1}$ ,故

$$
\alpha=\left(\alpha_{1}, \alpha_{2}, \cdots, \alpha_{n}\right)\left(\begin{array}{c}
x_{1} \\
x_{2} \\
\vdots \\
x_{n}
\end{array}\right)
$$

它的像 $\mathscr{B}(\boldsymbol{\alpha}) \in V_{2}$ ,可写为

$$
\mathscr{B}(\boldsymbol{\alpha})=\sum_{j=1}^{m} y_{j} \boldsymbol{\beta}_{j}=\left(\boldsymbol{\beta}_{1}, \boldsymbol{\beta}_{2}, \cdots, \boldsymbol{\beta}_{m}\right)\left(\begin{array}{c}
y_{1} \\
y_{2} \\
\vdots \\
y_{m}
\end{array}\right)
$$

又

$$
\mathscr{B}(\boldsymbol{\alpha})=\mathscr{A}\left(\boldsymbol{\alpha}_{1}, \boldsymbol{\alpha}_{2}, \cdots, \boldsymbol{\alpha}_{n}\right)\left(\begin{array}{c}
x_{1} \\
x_{2} \\
\vdots \\
x_{n}
\end{array}\right)
$$

$$
=\left(\boldsymbol{\beta}_{1}, \boldsymbol{\beta}_{2}, \cdots, \boldsymbol{\beta}_{m}\right) \boldsymbol{A}\left(\begin{array}{c}
x_{1} \\
x_{2} \\
\vdots \\
x_{n}
\end{array}\right)
$$

根据 $\mathscr{A}(\boldsymbol{\alpha})$ 坐标的唯一性,得

$$
y_{j}=\sum_{i=1}^{n} a_{j i} x_{i} \quad(j=1,2, \cdots, m)
$$

写成矩阵形式为

\[
\left[\begin{array}{c}
y_{1}  \tag{1.4.4}\\
y_{2} \\
\vdots \\
y_{m}
\end{array}\right]=\left[\begin{array}{cccc}
a_{11} & a_{12} & \cdots & a_{1 n} \\
a_{21} & a_{22} & \cdots & a_{2 n} \\
\vdots & \vdots & & \vdots \\
a_{m 1} & a_{m 2} & \cdots & a_{m n}
\end{array}\right]\left[\begin{array}{c}
x_{1} \\
x_{2} \\
\vdots \\
x_{n}
\end{array}\right]
\]

式(1.4.4)称为线性映射在给定基 $\left(\boldsymbol{\alpha}_{1}, \cdots, \boldsymbol{\alpha}_{n}\right)$ 与 $\left(\boldsymbol{\beta}_{1}, \cdots, \boldsymbol{\beta}_{m}\right)$ 下向量坐标变换公式(原像与像的坐标关系)。

线性映射 $\mathscr{A}$ 在给定基下的矩阵表示 $\boldsymbol{A}$ 是唯一的,它的逆问题就是下述定理。\\
定理1.4.1 设 $V_{1}$ 的基为 $\boldsymbol{\alpha}_{1}, \boldsymbol{\alpha}_{2}, \cdots, \boldsymbol{\alpha}_{n}, V_{2}$ 的基为 $\boldsymbol{\beta}_{1}, \boldsymbol{\beta}_{2}, \cdots, \boldsymbol{\beta}_{m}$ ,给定 $m \times n$矩阵 $\boldsymbol{A}=\left(a_{i j}\right)_{m \times n}$ ,则存在唯一的线性映射 $\mathscr{A}$ ,它在这两个基下的矩阵表示为 $\boldsymbol{A}$ 。

证明 任给 $\alpha \in V_{1}$ ,都有

$$
\begin{aligned}
& \boldsymbol{\alpha}=\left(\boldsymbol{\alpha}_{1}, \boldsymbol{\alpha}_{2}, \cdots, \boldsymbol{\alpha}_{n}\right)\left(\begin{array}{c}
x_{1} \\
x_{2} \\
\vdots \\
x_{n}
\end{array}\right) \\
& \boldsymbol{\beta}=\left(\boldsymbol{\beta}_{1}, \boldsymbol{\beta}_{2}, \cdots, \boldsymbol{\beta}_{m}\right) \boldsymbol{A}\left(\begin{array}{c}
x_{1} \\
x_{2} \\
\vdots \\
x_{n}
\end{array}\right) \in V_{2}
\end{aligned}
$$

取

$$
\mathscr{B}(\boldsymbol{\alpha})=\boldsymbol{\beta}
$$

令变换 $\mathscr{A}: V_{1} \rightarrow V_{2}$ ,它由下式确定

容易验证 $\mathscr{A}$ 是 $V_{1} \rightarrow V_{2}$ 的线性映射.事实上

$$
\mathscr{B}\left(\boldsymbol{\alpha}+\boldsymbol{\alpha}^{\prime}\right)=\left(\boldsymbol{\beta}_{1}, \boldsymbol{\beta}_{2}, \cdots, \boldsymbol{\beta}_{m}\right) \boldsymbol{A}\left(\begin{array}{c}
x_{1}+x_{1}^{\prime} \\
x_{2}+x_{2}^{\prime} \\
\vdots \\
x_{n}+x_{n}^{\prime}
\end{array}\right)
$$

$$
\begin{aligned}
& =\left(\boldsymbol{\beta}_{1}, \boldsymbol{\beta}_{2}, \cdots, \boldsymbol{\beta}_{m}\right) \boldsymbol{A}\left(\begin{array}{c}
x_{1} \\
x_{2} \\
\vdots \\
x_{n}
\end{array}\right)+\left(\boldsymbol{\beta}_{1}, \boldsymbol{\beta}_{2}, \cdots, \boldsymbol{\beta}_{m}\right) \boldsymbol{A}\left(\begin{array}{c}
x_{1}^{\prime} \\
x_{2}^{\prime} \\
\vdots \\
x_{n}^{\prime}
\end{array}\right) \\
& =\mathscr{A}(\boldsymbol{\alpha})+\mathscr{B}\left(\boldsymbol{\alpha}^{\prime}\right) \\
\mathscr{A}(\lambda \boldsymbol{\alpha}) & =\left(\boldsymbol{\beta}_{1}, \boldsymbol{\beta}_{2}, \cdots, \boldsymbol{\beta}_{m}\right) \boldsymbol{A}\left(\begin{array}{c}
\lambda x_{1} \\
\lambda x_{2} \\
\vdots \\
\lambda x_{n}
\end{array}\right) \\
& =\lambda\left(\boldsymbol{\beta}_{1}, \boldsymbol{\beta}_{2}, \cdots, \boldsymbol{\beta}_{m}\right) \boldsymbol{A}\left(\begin{array}{c}
x_{1} \\
x_{2} \\
\vdots \\
x_{n}
\end{array}\right)=\lambda \mathscr{A}(\boldsymbol{\alpha})
\end{aligned}
$$

又因为

$$
\begin{aligned}
\boldsymbol{\alpha}_{1} & =\left(\boldsymbol{\alpha}_{1}, \boldsymbol{\alpha}_{2}, \cdots, \boldsymbol{\alpha}_{n}\right)\left(\begin{array}{c}
1 \\
0 \\
\vdots \\
0
\end{array}\right), \boldsymbol{\alpha}_{2}=\left(\boldsymbol{\alpha}_{1}, \boldsymbol{\alpha}_{2}, \cdots, \boldsymbol{\alpha}_{n}\right)\left(\begin{array}{c}
0 \\
1 \\
0 \\
\vdots \\
0
\end{array}\right), \cdots, \boldsymbol{\alpha}_{n} \\
& =\left(\boldsymbol{\alpha}_{1}, \boldsymbol{\alpha}_{2}, \cdots, \boldsymbol{\alpha}_{n}\right)\left(\begin{array}{c}
0 \\
\vdots \\
0 \\
1
\end{array}\right)
\end{aligned}
$$

于是

$$
\begin{aligned}
& \mathscr{A}\left(\boldsymbol{\alpha}_{1}, \boldsymbol{\alpha}_{2}, \cdots, \boldsymbol{\alpha}_{n}\right)=\left(\mathscr{A}\left(\boldsymbol{\alpha}_{1}\right), \mathscr{A}\left(\boldsymbol{\alpha}_{2}\right), \cdots, \mathscr{A}\left(\boldsymbol{\alpha}_{n}\right)\right) \\
= & \left(\boldsymbol{\beta}_{1}, \boldsymbol{\beta}_{2}, \cdots, \boldsymbol{\beta}_{m}\right) \boldsymbol{A}\left[\begin{array}{cccc}
1 & 0 & \cdots & 0 \\
0 & 1 & \cdots & 0 \\
\vdots & \vdots & & \vdots \\
0 & 0 & \cdots & 1
\end{array}\right]=\left(\boldsymbol{\beta}_{1}, \boldsymbol{\beta}_{2}, \cdots, \boldsymbol{\beta}_{m}\right) \boldsymbol{A}
\end{aligned}
$$

最后证明唯一性.若还有 $\mathscr{A}_{1}: V_{1} \rightarrow V_{2}$ ,且

$$
\mathscr{A}_{1}\left(\boldsymbol{\alpha}_{1}, \boldsymbol{\alpha}_{2}, \cdots, \boldsymbol{\alpha}_{n}\right)=\left(\boldsymbol{\beta}_{1}, \boldsymbol{\beta}_{2}, \cdots, \boldsymbol{\beta}_{m}\right) \boldsymbol{A}
$$

于是

$$
\mathscr{A}\left(\boldsymbol{\alpha}_{1}, \boldsymbol{\alpha}_{2}, \cdots, \boldsymbol{\alpha}_{n}\right)=\mathscr{A}_{1}\left(\boldsymbol{\alpha}_{1}, \boldsymbol{\alpha}_{2}, \cdots, \boldsymbol{\alpha}_{n}\right)
$$

因此,$\forall \boldsymbol{\alpha}_{i}(i=1,2, \cdots, n)$ ,都有

$$
\mathscr{B}\left(\boldsymbol{\alpha}_{i}\right)=\mathscr{A}_{1}\left(\boldsymbol{\alpha}_{i}\right) \quad(i=1,2, \cdots, n)
$$

因为 $\forall \boldsymbol{\alpha} \in V_{1}$ ,有

$$
\begin{aligned}
\mathscr{A}(\boldsymbol{\alpha}) & =\mathscr{A}\left(\boldsymbol{\alpha}_{1}, \boldsymbol{\alpha}_{2}, \cdots, \boldsymbol{\alpha}_{n}\right)\left(\begin{array}{c}
x_{1} \\
x_{2} \\
\vdots \\
x_{n}
\end{array}\right) \\
& =\mathscr{A}_{1}\left(\boldsymbol{\alpha}_{1}, \boldsymbol{\alpha}_{2}, \cdots, \boldsymbol{\alpha}_{n}\right)\left(\begin{array}{c}
x_{1} \\
x_{2} \\
\vdots \\
x_{n}
\end{array}\right)=\mathscr{A}_{1}(\boldsymbol{\alpha})
\end{aligned}
$$

所以

$$
\mathscr{A}=\mathscr{A}_{1}
$$

由此可知,在给定基以后, $\mathscr{A}$ 与矩阵表示 $\boldsymbol{A}$ 是一一对应的。\\
例1.4.6 恒等映射的矩阵表示是单位矩阵,零映射的矩阵表示是零矩阵。 (见例1.4.1),其矩阵阶数是线性空间的维数。

例1.4.7 设 $B=\left[\begin{array}{ll}1 & 2 \\ 1 & 1 \\ 0 & 1\end{array}\right]$ ,映射 $\mathscr{A}: R^{2} \rightarrow R^{3}$ 由下式确定

$$
\mathscr{A}(\alpha)=B \alpha, \quad \alpha \in R^{2}
$$

试求 $\mathscr{A}$ 在基 $\boldsymbol{\alpha}_{1}=(1,0)^{\mathrm{T}}, \boldsymbol{\alpha}_{2}=(0,1)^{\mathrm{T}}$ 与基 $\boldsymbol{\beta}_{1}=(1,0,0)^{\mathrm{T}}, \boldsymbol{\beta}_{2}=(0,1,0)^{\mathrm{T}}, \boldsymbol{\beta}_{3}= (0,0,1)^{\mathrm{T}}$ 下的矩阵表示 $\boldsymbol{A}$ 。

解 $\mathscr{A}\left(\boldsymbol{\alpha}_{1}\right)=(1,1,0)^{\mathrm{T}}=\boldsymbol{\beta}_{1}+\boldsymbol{\beta}_{2}, \mathscr{B}\left(\boldsymbol{\alpha}_{2}\right)=(2,1,1)^{\mathrm{T}}=2 \boldsymbol{\beta}_{1}+\boldsymbol{\beta}_{2}+\boldsymbol{\beta}_{3}$ 于是所求矩阵为

$$
A=\left[\begin{array}{ll}
1 & 2 \\
1 & 1 \\
0 & 1
\end{array}\right]_{3 \times 2}
$$

例1.4.8 求线性映射 $\mathscr{D}: R[x]_{n+1} \rightarrow R[x]_{n}$

$$
\mathscr{D}(f(x))=\frac{\mathrm{d}}{\mathrm{~d} x} f(x)
$$

在基 $1, x, x^{2}, \cdots, x^{n}$ 与基 $1, x, x^{2}, \cdots, x^{n-1}$ 下的矩阵表示 $\boldsymbol{D}$ .\\
解 $\mathscr{D}(1)=0, \mathscr{X}(x)=1, \mathscr{D}\left(x^{2}\right)=2 x, \cdots, \mathscr{D}\left(x^{n}\right)=n x^{n-1}$ ,于是所求矩阵为

$$
\boldsymbol{D}=\left[\begin{array}{ccccc}
0 & 1 & 0 & \cdots & 0 \\
0 & 0 & 2 & \cdots & 0 \\
\vdots & \vdots & \vdots & & \vdots \\
0 & 0 & 0 & \cdots & n
\end{array}\right]_{n \times(n+1)}
$$

注 对于线性映射 $\mathscr{D}: R[x]_{n+1} \rightarrow R[x]_{n+1}$

$$
\mathscr{D}(f(x))=\frac{\mathrm{d}}{\mathrm{~d} x} f(x)
$$

在基 $1, x, x^{2}, \cdots, x^{n}$ 与基 $1, x, x^{2}, \cdots, x^{n}$ 下的矩阵表示为

$$
\boldsymbol{D}=\left[\begin{array}{ccccc}
0 & 1 & 0 & \cdots & 0 \\
0 & 0 & 2 & \cdots & 0 \\
\vdots & \vdots & \vdots & & \vdots \\
0 & 0 & 0 & \cdots & n \\
0 & 0 & 0 & \cdots & 0
\end{array}\right]_{(n+1) \times(n+1)}
$$

例1.4.9 求线性映射 $S: R[x]_{n} \rightarrow R[x]_{n+1}$

$$
S(f(x))=\int_{0}^{x} f(t) \mathrm{d} t, \quad f(t) \in R[x]_{n}
$$

在基 $1, x, x^{2}, \cdots, x^{n-1}$ 与基 $1, x, x^{2}, \cdots, x^{n-1}, x^{n}$ 下的矩阵表示.\\
解 $S(1)=\int_{0}^{x} \mathrm{~d} t=x, S(x)=\int_{0}^{x} t \mathrm{~d} t=\frac{1}{2} x^{2}$ ,

$$
\begin{aligned}
& S\left(x^{2}\right)=\int_{0}^{x} t^{2} \mathrm{~d} t=\frac{1}{3} x^{3}, \cdots \\
& S\left(x^{n-1}\right)=\int_{0}^{x} t^{n-1} \mathrm{~d} t=\frac{1}{n} x^{n}
\end{aligned}
$$

于是所求矩阵为

$$
S=\left[\begin{array}{cccc}
0 & 0 & \cdots & 0 \\
1 & 0 & \cdots & 0 \\
0 & \frac{1}{2} & \cdots & 0 \\
\vdots & \vdots & & \vdots \\
0 & 0 & \cdots & \frac{1}{n}
\end{array}\right]_{(n+1) \times n}
$$

在指定了空间 $V_{1}$ 与 $V_{2}$ 的基之后,便可以求得线性映射 $\mathscr{A}_{:} V_{1} \rightarrow V_{2}$ 在指定一对基下的矩阵表示。但是空间基是不唯一的,自然应该考虑下列两个问题:\\
(1)线性映射在不同对基下的矩阵表示之间有什么关系?\\
(2)对一个线性映射,能否选择一对基,使它的矩阵表示最简单。\\
先来回答第一个问题,第二个问题将在 §1.9与 § 1. 10 节中讨论。\\
定理1.4.2 设 $\mathscr{A}$ 是 $V_{1} \rightarrow V_{2}$ 的一个线性映射, $\boldsymbol{\alpha}_{1}, \boldsymbol{\alpha}_{2}, \cdots, \boldsymbol{\alpha}_{n}$ 与 $\boldsymbol{\alpha}_{1}^{\prime}, \boldsymbol{\alpha}_{2}^{\prime}, \cdots, \boldsymbol{\alpha}_{n}^{\prime}$是 $V_{1}$ 的两组基,由 $\boldsymbol{\alpha}_{i}$ 到 $\boldsymbol{\alpha}_{i}^{\prime}$ 的过渡矩阵为 $\boldsymbol{P}$ .设 $\boldsymbol{\beta}_{1}, \boldsymbol{\beta}_{2}, \cdots, \boldsymbol{\beta}_{m}$ 与 $\boldsymbol{\beta}_{1}^{\prime}, \boldsymbol{\beta}_{2}^{\prime}, \cdots, \boldsymbol{\beta}_{m}^{\prime}$ 是 $V_{2}$ 的两组基,由 $\boldsymbol{\beta}_{j}$ 到 $\boldsymbol{\beta}_{j}^{\prime}$ 的过渡矩阵为 $\boldsymbol{Q}$ .线性映射 $\mathcal{A}$ 在基 $\boldsymbol{\alpha}_{1}, \boldsymbol{\alpha}_{2}, \cdots, \boldsymbol{\alpha}_{n}$ 与 $\boldsymbol{\beta}_{1}$ , $\boldsymbol{\beta}_{2}, \cdots, \boldsymbol{\beta}_{m}$ 下的矩阵表示为 $\boldsymbol{A}$ ,在基 $\boldsymbol{\alpha}_{1}^{\prime}, \boldsymbol{\alpha}_{2}^{\prime}, \cdots, \boldsymbol{\alpha}_{n}^{\prime}$ 与 $\boldsymbol{\beta}_{1}^{\prime}, \boldsymbol{\beta}_{2}^{\prime}, \cdots, \boldsymbol{\beta}_{m}^{\prime}$ 下的矩阵表示为 $\boldsymbol{B}$ ,则


\begin{equation*}
B=Q^{-1} A P \tag{1.4.5}
\end{equation*}


证明 由假设条件知


\begin{align*}
& \mathscr{A}\left(\boldsymbol{\alpha}_{1}, \boldsymbol{\alpha}_{2}, \cdots, \boldsymbol{\alpha}_{n}\right)=\left(\boldsymbol{\beta}_{1}, \boldsymbol{\beta}_{2}, \cdots, \boldsymbol{\beta}_{m}\right) \boldsymbol{A},  \tag{1}\\
& \mathscr{A}\left(\boldsymbol{\alpha}_{1}^{\prime}, \boldsymbol{\alpha}_{2}^{\prime}, \cdots, \boldsymbol{\alpha}_{n}^{\prime}\right)=\left(\boldsymbol{\beta}_{1}^{\prime}, \boldsymbol{\beta}_{2}^{\prime}, \cdots, \boldsymbol{\beta}_{m}^{\prime}\right) \boldsymbol{B}, \tag{2}
\end{align*}


\begin{center}
\includegraphics[max width=\textwidth]{2025_10_21_263f47d9e71ddd3e429ag-032}
\end{center}


\begin{gather*}
\boldsymbol{V}_{2}\left\{\boldsymbol{\beta}_{i}\right\} \xrightarrow{\boldsymbol{Q}}\left\{\boldsymbol{\beta}_{i}^{\prime}\right\} \\
\left(\boldsymbol{\alpha}_{1}^{\prime}, \boldsymbol{\alpha}_{2}^{\prime}, \cdots, \boldsymbol{\alpha}_{n}^{\prime}\right)=\left(\boldsymbol{\alpha}_{1}, \boldsymbol{\alpha}_{2}, \cdots, \boldsymbol{\alpha}_{n}\right) \boldsymbol{P}  \tag{3}\\
\left(\boldsymbol{\beta}_{1}^{\prime}, \boldsymbol{\beta}_{2}^{\prime}, \cdots, \boldsymbol{\beta}_{m}^{\prime}\right)=\left(\boldsymbol{\beta}_{1}, \boldsymbol{\beta}_{2}, \cdots, \boldsymbol{\beta}_{m}\right) \boldsymbol{Q} \tag{4}
\end{gather*}


将式(3)与式(4)代入式(2),得


\begin{equation*}
\mathscr{A}\left(\boldsymbol{\alpha}_{1}, \boldsymbol{\alpha}_{2}, \cdots, \boldsymbol{\alpha}_{n}\right) \boldsymbol{P}=\left(\boldsymbol{\beta}_{1}, \boldsymbol{\beta}_{2}, \cdots, \boldsymbol{\beta}_{m}\right) \boldsymbol{Q B} \tag{5}
\end{equation*}


将式(1)代人式(5)得

$$
\left(\boldsymbol{\beta}_{1}, \boldsymbol{\beta}_{2}, \cdots, \boldsymbol{\beta}_{m}\right) \boldsymbol{A} \boldsymbol{P}=\left(\boldsymbol{\beta}_{1}, \boldsymbol{\beta}_{2}, \cdots, \boldsymbol{\beta}_{m}\right) \boldsymbol{Q B}
$$

因为 $\boldsymbol{\beta}_{1}, \boldsymbol{\beta}_{2}, \cdots, \boldsymbol{\beta}_{m}$ 线性无关,故

$$
A P=Q B
$$

由于 $\boldsymbol{Q}$ 是满秩方阵(因为过渡矩阵都是满秩的)。所以

$$
B=Q^{-1} A P
$$

注 若 $V_{1}$ 到 $V_{2}$ 有几个线性映射 $\mathscr{A}, \mathscr{B}, \mathscr{B}, \mathscr{D}, \cdots$ ,则可以定义线性映射的加法,数乘线性映射等运算,其相应的矩阵表示对应于矩阵的加法,数乘矩阵。(可参阅§1.6)。

定义1.4.2 设 $\boldsymbol{A}, \boldsymbol{B} \in F^{m \times n}$ ,若存在 $\boldsymbol{Q} \in F_{m}^{m \times m}, \boldsymbol{P} \in F_{n}^{n \times n}$ ,满足

$$
B=Q A P
$$

则称 $\boldsymbol{B}$ 与 $\boldsymbol{A}$ 等价.\\
一个线性映射 $\mathscr{A}: V_{n} \rightarrow V_{m}$ 有一系列的 $m \times n$ 矩阵表示: $\boldsymbol{A}, \boldsymbol{B}, \cdots$ .由定理1.4.2知它们之间是互相等价的.反之,互相等价的 $m \times n$ 矩阵代表同一个线性映射。原像 $\boldsymbol{\alpha}$ 的坐标 $\left(x_{1}, x_{2}, \cdots, x_{n}\right)^{\mathrm{T}}$ 与像 $\mathscr{A}(\boldsymbol{\alpha})$ 的坐标 $\left(y_{1}, y_{2}, \cdots, y_{m}\right)^{\mathrm{T}}$ 之间满足式 (1.4.4)。式(1.4.4)也揭示 $m \times n$ 矩阵 $\boldsymbol{A}$ 是 $C^{n} \rightarrow C^{m}$ 的一个线性映射,它与线性映射 $\mathscr{A}: V_{n} \rightarrow V_{m}$ 是对应的。因此,一般的线性空间 $V_{n}$ 与特殊的向量空间 $C^{n}$(或 $\left.R^{n}\right)$ 同构,线性映射 $A$ 可用矩阵 $A$ 代表。

\section*{§1.5 线性映射的值域、核}
定义1.5.1 设 $\mathscr{A}$ 是线性空间 $V_{1}$ 到 $V_{2}$ 的线性映射,命

$$
\mathscr{A}\left(V_{1}\right)=\left\{\boldsymbol{\beta}=\mathscr{A}(\boldsymbol{\alpha}) \in V_{2} \mid \forall \boldsymbol{\alpha} \in V_{1}\right\}
$$

容易证明: $\mathscr{A}\left(V_{1}\right)$ 是 $V_{2}$ 的线性子空间。称 $\mathscr{A}\left(V_{1}\right)$ 是线性映射 $\mathscr{A}$ 的值域,记之为 $R(\mathscr{A})$ 。称 $\operatorname{dim} R(\mathscr{A})$ 为 $\mathscr{A}$ 的秩,记之为 $\operatorname{rank} \mathscr{A}$ 。

定理1.5.1 设, $\mathscr{B}$ 是线性空间 $V_{1}$ 到 $V_{2}$ 的线性映射。 $\boldsymbol{\alpha}_{1}, \boldsymbol{\alpha}_{2}, \cdots, \boldsymbol{\alpha}_{n}$ 是 $V_{1}$ 的基, $\boldsymbol{\beta}_{1}, \boldsymbol{\beta}_{2}, \cdots, \boldsymbol{\beta}_{m}$ 是 $V_{2}$ 的基.$~ \mathscr{B}$ 在该对基下的矩阵表示为 $\boldsymbol{A}=\left(a_{i j}\right)$ ,则\\
(1)$R(\mathscr{A})=\operatorname{span}\left\{\mathscr{A}\left(\boldsymbol{\alpha}_{1}\right), \mathscr{A}\left(\boldsymbol{\alpha}_{2}\right), \cdots, \mathscr{A}\left(\boldsymbol{\alpha}_{n}\right)\right\}$\\
(2) $\operatorname{rank} \mathscr{A}=\operatorname{rank} A$\\
证明(1)因为 $\forall \boldsymbol{\alpha} \in V_{1}$ ,有

故

$$
\begin{aligned}
\boldsymbol{\beta}= & \mathscr{A}(\boldsymbol{\alpha})=\mathscr{A}\left(x_{1} \boldsymbol{\alpha}_{1}+x_{2} \boldsymbol{\alpha}_{2}+\cdots+x_{n} \boldsymbol{\alpha}_{n}\right) \\
= & x_{1} \mathscr{A}\left(\boldsymbol{\alpha}_{1}\right)+x_{2} \mathscr{B}\left(\boldsymbol{\alpha}_{2}\right)+\cdots+x_{n} \mathscr{A}\left(\boldsymbol{\alpha}_{n}\right) \\
& R(\mathscr{A})=\operatorname{span}\left\{\mathscr{B}\left(\boldsymbol{\alpha}_{1}\right), \mathscr{B}\left(\boldsymbol{\alpha}_{2}\right), \cdots, \mathscr{A}\left(\boldsymbol{\alpha}_{n}\right)\right\}
\end{aligned}
$$

(2)由 $\boldsymbol{A}$ 的定义式(1.4.3)可知

$$
\begin{aligned}
\left(\mathscr{A}\left(\boldsymbol{\alpha}_{1}\right), \mathscr{A}\left(\boldsymbol{\alpha}_{2}\right), \cdots, \mathscr{A}\left(\boldsymbol{\alpha}_{n}\right)\right) & =\mathscr{A}\left(\boldsymbol{\alpha}_{1}, \boldsymbol{\alpha}_{2}, \cdots, \boldsymbol{\alpha}_{n}\right) \\
& =\left(\boldsymbol{\beta}_{1}, \boldsymbol{\beta}_{2}, \cdots, \boldsymbol{\beta}_{m}\right) \boldsymbol{A}
\end{aligned}
$$

于是

$$
R(\mathscr{A})=\operatorname{span}\left\{\sum_{k=1}^{m} a_{k 1} \boldsymbol{\beta}_{k}, \sum_{k=1}^{m} a_{k 2} \boldsymbol{\beta}_{k}, \cdots, \sum_{k=1}^{m} a_{k m} \boldsymbol{\beta}_{k}\right\}
$$

而向量组 $\sum_{k=1}^{m} a_{k 1} \boldsymbol{\beta}_{k}, \sum_{k=1}^{m} a_{k 2} \boldsymbol{\beta}_{k}, \cdots, \sum_{k=1}^{m} a_{k m} \boldsymbol{\beta}_{k}$ 的秩等于 $\boldsymbol{A}$ 的列秩,因此

$$
\operatorname{rank} \mathscr{A}=\operatorname{dim} R(\mathscr{A})=\operatorname{rank} A
$$

定义1.5.2 设 $\mathscr{A}$ 是线性空间 $V_{1}$ 到 $V_{2}$ 的线性映射,命

$$
N(\mathscr{A})=\mathscr{A}^{-1}(0)=\left\{\alpha \in V_{1} \mid \mathscr{B}(\alpha)=0\right\} .
$$

容易证明:$N(\mathscr{A})$ 是 $V_{1}$ 的线性子空间,称 $N(\mathscr{A})$ 是线性映射 $\mathscr{A}$ 的核子空间。 $\operatorname{dim} N(\mathscr{A})$ 为 $\mathscr{A}$ 的零度.

可以证明:若 $\operatorname{dim} N(\mathscr{A})=0$ ,则线性无关向量组 $\boldsymbol{\alpha}_{1}, \boldsymbol{\alpha}_{2}, \cdots, \boldsymbol{\alpha}_{r} \in V_{1}$ 的像 $\mathscr{B}\left(\boldsymbol{\alpha}_{1}\right), \mathscr{A}\left(\boldsymbol{\alpha}_{2}\right), \cdots, \mathscr{B}\left(\boldsymbol{\alpha}_{r}\right) \in V_{2}$ 也线性无关。

定理1.5.2 设, $\mathscr{A}$ 是 $n$ 维线性空间 $V_{1}$ 到 $m$ 维线性空间 $V_{2}$ 的线性映射,则

$$
\operatorname{dim} N(\mathscr{A})+\operatorname{dim} R(. \mathscr{A})=n
$$

证明 设 $\operatorname{dim} N(\mathscr{A})=r, \boldsymbol{\alpha}_{1}, \boldsymbol{\alpha}_{2}, \cdots, \boldsymbol{\alpha}_{r}$ 是 $N(\mathscr{A})$ 的基,把它扩充成 $V_{1}$ 的基 $\boldsymbol{\alpha}_{1}$ , $\boldsymbol{\alpha}_{2}, \cdots, \boldsymbol{\alpha}_{r}, \boldsymbol{\alpha}_{r+1}, \cdots, \boldsymbol{\alpha}_{n}$ .则有

$$
\begin{aligned}
R(\mathscr{A}) & =\operatorname{span}\left\{\mathscr{B}\left(\boldsymbol{\alpha}_{1}\right), \mathscr{A}\left(\boldsymbol{\alpha}_{2}\right), \cdots, \mathscr{A}\left(\boldsymbol{\alpha}_{r}\right), \mathscr{A}\left(\boldsymbol{\alpha}_{r+1}\right), \cdots, \mathscr{A}\left(\boldsymbol{\alpha}_{n}\right)\right\} \\
& =\operatorname{span}\left\{0,0, \cdots, 0, \mathscr{A}\left(\boldsymbol{\alpha}_{r+1}\right), \cdots, \mathscr{A}\left(\boldsymbol{\alpha}_{n}\right)\right\} \\
& =\operatorname{span}\left\{\mathscr{A}\left(\boldsymbol{\alpha}_{r+1}\right), \cdots, \mathscr{B}\left(\boldsymbol{\alpha}_{n}\right)\right\}
\end{aligned}
$$

现在证明: $\mathscr{B}\left(\boldsymbol{\alpha}_{r+1}\right), \cdots, \mathscr{A}\left(\boldsymbol{\alpha}_{n}\right)$ 线性无关。\\
设

$$
\sum_{i=r+1}^{n} k_{i} \mathscr{B}\left(\boldsymbol{\alpha}_{i}\right)=0
$$

即

$$
A B\left(\sum_{i=r+1}^{n} k_{i} \boldsymbol{\alpha}_{i}\right)=\mathbf{0}
$$

故 $\sum_{i=r+1}^{n} k_{i} \boldsymbol{\alpha}_{i} \in N(\mathscr{A})$ ,因此

$$
\sum_{i=r+1}^{n} k_{i} \boldsymbol{\alpha}_{i}=\sum_{j=1}^{r} l_{j} \boldsymbol{\alpha}_{j}
$$

根据 $\boldsymbol{\alpha}_{1}, \cdots, \boldsymbol{\alpha}_{n}$ 线性无关,得到

$$
l_{j}=0, \quad k_{i}=0 \quad(j=1,2, \cdots, r ; i=r+1, \cdots, n)
$$

因此 $\mathscr{B}\left(\boldsymbol{\alpha}_{r+1}\right), \cdots, \mathscr{B}\left(\boldsymbol{\alpha}_{n}\right)$ 线性无关.于是

即

$$
\begin{aligned}
& \operatorname{dim} R(\mathscr{A})=n-r \\
& \operatorname{dim} R(\mathscr{B})+\operatorname{dim} N(\mathscr{A})=n
\end{aligned}
$$

注 若 $m \times n$ 矩阵 $\boldsymbol{A}$ 的秩为 $r$ ,那么齐次线性方程组 $\boldsymbol{A} \boldsymbol{X}=0$ 的解空间 $N(\boldsymbol{A})$ 有 $\operatorname{dim} N(\boldsymbol{A})=n-r, \boldsymbol{A}$ 的列空间 $R(\boldsymbol{A})$ 满足 $\operatorname{dim} R(\boldsymbol{A})=r$ ,于是 $\operatorname{dim} N(\boldsymbol{A})+\operatorname{dim} R(\boldsymbol{A})=n$ .

例1.5.1 设 $\mathscr{B}$ 是 $n$ 维线性空间 $V_{1}$ 到 $m$ 维线性空间 $V_{2}$ 的线性映射,$\alpha_{1}$ , $\boldsymbol{\alpha}_{2}, \cdots, \boldsymbol{\alpha}_{n}$ 是 $V_{1}$ 的一组基, $\boldsymbol{\beta}_{1}, \boldsymbol{\beta}_{2}, \cdots, \boldsymbol{\beta}_{m}$ 是 $V_{2}$ 的一组基。线性映射 $\mathscr{A}$ 在这组基下的矩阵表示是 $m \times n$ 矩阵 $\boldsymbol{A}=\left(\boldsymbol{A}_{1}, \boldsymbol{A}_{2}, \cdots, \boldsymbol{A}_{n}\right)$ ,其中 $\boldsymbol{A}_{i}=\left(a_{1 i}, a_{2 i}, \cdots, a_{m i}\right)^{\mathrm{T}}$ 是 $m$ 行列矩阵,$i=1,2, \cdots, n$ 。于是

$$
\mathscr{A}\left(\boldsymbol{\alpha}_{1}, \boldsymbol{\alpha}_{2}, \cdots, \boldsymbol{\alpha}_{n}\right)=\left(\boldsymbol{\beta}_{1}, \boldsymbol{\beta}_{2}, \cdots, \boldsymbol{\beta}_{m}\right) \boldsymbol{A}
$$

故

$$
\mathscr{A}\left(\boldsymbol{\alpha}_{i}\right)=\left(\boldsymbol{\beta}_{1}, \boldsymbol{\beta}_{2}, \cdots, \boldsymbol{\beta}_{m}\right) \boldsymbol{A}_{i} \quad(i=1,2, \cdots, n)
$$

因此 $\mathscr{B}$ 的值域

$$
\begin{aligned}
R(\mathscr{A}) & =\operatorname{span}\left\{\mathscr{A}\left(\boldsymbol{\alpha}_{1}\right), \mathscr{A}\left(\boldsymbol{\alpha}_{2}\right), \cdots, \mathscr{B}\left(\boldsymbol{\alpha}_{n}\right)\right\} \\
& =\operatorname{span}\left\{\left(\boldsymbol{\beta}_{1}, \boldsymbol{\beta}_{2}, \cdots, \boldsymbol{\beta}_{m}\right) \boldsymbol{A}_{1}, \cdots,\left(\boldsymbol{\beta}_{1}, \boldsymbol{\beta}_{2}, \cdots, \boldsymbol{\beta}_{m}\right) \boldsymbol{A}_{n}\right\}
\end{aligned}
$$

由例1.1.5知矩阵 $\boldsymbol{A}$ 的值域

$$
R(A)=\left\{y \mid A x=y, \quad x \in R^{n}\right\}
$$

若取 $\boldsymbol{x}_{i}=(0, \cdots, 0,1,0, \cdots, 0)^{\mathrm{T}}$ 则 $\boldsymbol{A} \boldsymbol{x}_{i}=\boldsymbol{A}_{i} \quad(i=1,2, \cdots, n)$ ,所以

$$
R(\boldsymbol{A})=\operatorname{span}\left\{\boldsymbol{A}_{1}, \boldsymbol{A}_{2}, \cdots, \boldsymbol{A}_{n}\right\}
$$

由上述 $R(A B)$ 与 $R(\boldsymbol{A})$ 的表达式可见 $\mathscr{A}$ 的值域与 $\boldsymbol{A}$ 的值域是一致的,只要把 $\boldsymbol{A}$ 的值域引进"基"以后就与 $\mathscr{A}$ 的值域完全相同。

现在研究 $\mathscr{A}$ 的核 $N(\mathscr{A})$ 与矩阵 $\boldsymbol{A}$ 的核 $N(\boldsymbol{A})$ 。\\
设 $X \in V_{1}$ ,

$$
\boldsymbol{X}=\left(\boldsymbol{\alpha}_{1}, \boldsymbol{\alpha}_{2}, \cdots, \boldsymbol{\alpha}_{n}\right)\left[\begin{array}{c}
x_{1} \\
x_{2} \\
\vdots \\
x_{n}
\end{array}\right]
$$

列向量 $\left(x_{1}, x_{2}, \cdots, x_{n}\right)^{\mathrm{T}}$ 是 $V_{1}$ 中向量 $\boldsymbol{X}$(在基 $\boldsymbol{\alpha}_{1}, \boldsymbol{\alpha}_{2}, \cdots, \boldsymbol{\alpha}_{n}$ 下)的坐标向量。\\
$N(\mathscr{A})$ 中向量 $\boldsymbol{X}$ 必满足

$$
\begin{aligned}
& \mathscr{A}\left(\boldsymbol{\alpha}_{1}, \boldsymbol{\alpha}_{2}, \cdots, \boldsymbol{\alpha}_{n}\right)\left[\begin{array}{c}
x_{1} \\
x_{2} \\
\vdots \\
x_{n}
\end{array}\right]=0 \\
& \left(\boldsymbol{\beta}_{1}, \boldsymbol{\beta}_{2}, \cdots, \boldsymbol{\beta}_{m}\right) \boldsymbol{A}\left[\begin{array}{c}
x_{1} \\
x_{2} \\
\vdots \\
x_{n}
\end{array}\right]=0
\end{aligned}
$$

此即

$$
\boldsymbol{A}\left[\begin{array}{c}
x_{1} \\
x_{2} \\
\vdots \\
x_{n}
\end{array}\right]=0
$$

根据矩阵 $\boldsymbol{A}$ 的核(例1.1.4)知,它就是 $\boldsymbol{A}$ 的核 $\left(x_{1}, x_{2}, \cdots, x_{n}\right)^{\mathrm{T}}$ 所满足的方程式。由上述分析可见, $\mathscr{A}$ 的核 $N(\mathscr{A})$ 中向量 $\boldsymbol{X}$ 的(在基 $\boldsymbol{\alpha}_{1}, \boldsymbol{\alpha}_{2}, \cdots, \boldsymbol{\alpha}_{n}$ )坐标向量( $x_{1}$ , $\left.x_{2}, \cdots, x_{n}\right)^{\mathrm{T}}$ 满足矩阵 $\boldsymbol{A}$ 的核向量所满足的方程。因此 $\mathscr{A}$ 的核与 $\boldsymbol{A}$ 的核是一致的,只要把 $\boldsymbol{A}$ 的核引进"基"以后就与 $\mathscr{A}$ 的核完全相同。

例1.5.2 设线性映射 $\mathscr{A}: R^{3} \rightarrow R^{2}$ 在基 $\boldsymbol{\alpha}_{1}=(-1,1,1)^{\mathrm{T}}, \boldsymbol{\alpha}_{2}=(1,0,-1)^{\mathrm{T}}$ , $\boldsymbol{\alpha}_{3}=(0,1,1)^{\mathrm{T}}$ 与基 $\boldsymbol{\beta}_{1}=(1,1)^{\mathrm{T}}, \boldsymbol{\beta}_{2}=(0,2)^{\mathrm{T}}$ 的矩阵表示为

$$
A=\left[\begin{array}{rrr}
1 & 1 & -1 \\
0 & 1 & 2
\end{array}\right]
$$

求:(1) $\mathscr{A}$ 的核子空间 $N(\mathscr{A})$ 的基与维数。\\
(2) $\mathscr{A}$ 的值域 $R(\mathscr{A})$ 的基与维数.\\
解(1)核子空间就是求 $\boldsymbol{X} \in R^{3}$ 满足 $\mathscr{A}(\boldsymbol{X})=0$ ,由于 $\boldsymbol{X} \in R^{3}$ 。故

$$
\boldsymbol{X}=\left(\boldsymbol{\alpha}_{1}, \boldsymbol{\alpha}_{2}, \boldsymbol{\alpha}_{3}\right)\left[\begin{array}{l}
x_{1} \\
x_{2} \\
x_{3}
\end{array}\right]
$$

于是

$$
\mathscr{A}(\boldsymbol{X})=\mathscr{B}\left(\boldsymbol{\alpha}_{1}, \boldsymbol{\alpha}_{2}, \boldsymbol{\alpha}_{3}\right)\left[\begin{array}{l}
x_{1} \\
x_{2} \\
x_{3}
\end{array}\right]=\left(\boldsymbol{\beta}_{1}, \boldsymbol{\beta}_{2}\right) \boldsymbol{A}\left[\begin{array}{l}
x_{1} \\
x_{2} \\
x_{3}
\end{array}\right]
$$

所以所求 $\boldsymbol{X}$ 的坐标 $x_{1}, x_{2}, x_{3}$ 应是齐次方程组

$$
\left[\begin{array}{rrr}
1 & 1 & -1 \\
0 & 1 & 2
\end{array}\right]\left[\begin{array}{l}
x_{1} \\
x_{2} \\
x_{3}
\end{array}\right]=0
$$

的解空间,求得它的基础解系为

$$
x_{1}=3, x_{2}=-2, x_{3}=1
$$

因此核子空间 $N(\mathscr{A})$ 的基是 $x_{1} \boldsymbol{\alpha}_{1}+x_{2} \boldsymbol{\alpha}_{2}+x_{3} \boldsymbol{\alpha}_{3}=3 \boldsymbol{\alpha}_{1}-2 \boldsymbol{\alpha}_{2}+\boldsymbol{\alpha}_{3}=(-5,4,4)^{\mathrm{T}}$ , $\operatorname{dim} N(\mathscr{B})=1$.

注:$N(\mathscr{A})$ 的基不是 $(3,-2,1)^{\mathrm{T}}$ 。而是 $3 \boldsymbol{\alpha}_{1}-2 \boldsymbol{\alpha}_{2}+\boldsymbol{\alpha}_{3}$ 。为什么?$N(\boldsymbol{A})$ 的基是 $(3,-2,1)^{\mathrm{T}}$ .\\
(2) $\mathscr{A}$ 的值域

$$
\begin{aligned}
R(\mathscr{A}) & =\operatorname{span}\left\{\mathscr{A}\left(\boldsymbol{\alpha}_{1}\right), \mathscr{A}\left(\boldsymbol{\alpha}_{2}\right), \mathscr{A}\left(\boldsymbol{\alpha}_{3}\right)\right\} \\
& =\operatorname{span}\left\{\boldsymbol{\beta}_{1}, \boldsymbol{\beta}_{1}+\boldsymbol{\beta}_{2},-\boldsymbol{\beta}_{1}+2 \boldsymbol{\beta}_{2}\right\} \\
& =\operatorname{span}\left\{\boldsymbol{\beta}_{1}, \boldsymbol{\beta}_{1}+\boldsymbol{\beta}_{2}\right\} \\
& =\operatorname{span}\left\{\boldsymbol{\beta}_{1}, \boldsymbol{\beta}_{2}\right\}=R^{2}
\end{aligned}
$$

\section*{§ 1.6 线性变换的矩阵与线性变换的运算}
\section*{一、线性变换的矩阵表示}
本节及以下几节的线性映射 $\mathscr{A}$ 都是指线性空间 $V$ 到线性空间 $V$ 的映射,特称这样的 $\mathscr{A}$ 为线性空间 $V$ 的线性变换。由于线性变换是线性空间 $V$ 到它自身的映射,所以只需取 $V$ 的一组基 $\boldsymbol{\alpha}_{1}, \boldsymbol{\alpha}_{2}, \cdots, \boldsymbol{\alpha}_{n}$ 即可。

设 $\mathscr{A}$ 是线性空间 $V$ 的线性变换, $\boldsymbol{\alpha}_{1}, \boldsymbol{\alpha}_{2}, \cdots, \boldsymbol{\alpha}_{n}$ 是 $V$ 的一组基,若

$$
\mathscr{A}\left(\boldsymbol{\alpha}_{j}\right)=\sum_{i=1}^{n} a_{i j} \boldsymbol{\alpha}_{i} \quad(j=1,2, \cdots, n)
$$

则 $\mathscr{A}\left(\boldsymbol{\alpha}_{1}, \boldsymbol{\alpha}_{2}, \cdots, \boldsymbol{\alpha}_{n}\right)=\left(\boldsymbol{\alpha}_{1}, \boldsymbol{\alpha}_{2}, \cdots, \boldsymbol{\alpha}_{n}\right)\left[\begin{array}{cccc}a_{11} & a_{12} & \cdots & a_{1 n} \\ a_{21} & a_{22} & \cdots & a_{2 n} \\ \vdots & \vdots & & \vdots \\ a_{n 1} & a_{n 2} & \cdots & a_{n n}\end{array}\right]$


\begin{equation*}
=\left(\boldsymbol{\alpha}_{1}, \boldsymbol{\alpha}_{2}, \cdots, \boldsymbol{\alpha}_{n}\right) \boldsymbol{A} \tag{1.6.1}
\end{equation*}


所以 $\mathscr{B}$ 在 $\boldsymbol{\alpha}_{1}, \boldsymbol{\alpha}_{2}, \cdots, \boldsymbol{\alpha}_{n}$ 下的矩阵表示 $\boldsymbol{A}$ 是 $n$ 阶方阵。\\
设 $\boldsymbol{\alpha}=\left(\boldsymbol{\alpha}_{1}, \boldsymbol{\alpha}_{2}, \cdots, \boldsymbol{\alpha}_{n}\right)\left(\begin{array}{c}x_{1} \\ x_{2} \\ \vdots \\ x_{n}\end{array}\right) \in V$ ,若

$$
\mathscr{B}(\boldsymbol{\alpha})=\left(\boldsymbol{\alpha}_{1}, \boldsymbol{\alpha}_{2}, \cdots, \boldsymbol{\alpha}_{n}\right)\left(\begin{array}{c}
y_{1} \\
y_{2} \\
\vdots \\
y_{n}
\end{array}\right)
$$

则原像 $\boldsymbol{\alpha}$ 与像 $\mathscr{B}(\boldsymbol{\alpha})$ 的坐标变换公式为

\[
\left(\begin{array}{c}
y_{1}  \tag{1.6.2}\\
y_{2} \\
\vdots \\
y_{n}
\end{array}\right)=\boldsymbol{A}\left(\begin{array}{c}
x_{1} \\
x_{2} \\
\vdots \\
x_{n}
\end{array}\right)
\]

例1.6.1 设 $R^{3}$ 中线性变换 $\mathscr{A}$ 将基

$$
\alpha_{1}=\left(\begin{array}{r}
1 \\
1 \\
-1
\end{array}\right), \quad \alpha_{2}=\left(\begin{array}{r}
0 \\
2 \\
-1
\end{array}\right), \quad \alpha_{3}=\left(\begin{array}{r}
1 \\
0 \\
-1
\end{array}\right)
$$

变为基

$$
\alpha_{1}^{\prime}=\left(\begin{array}{r}
1 \\
-1 \\
0
\end{array}\right), \quad \alpha_{2}^{\prime}=\left(\begin{array}{r}
0 \\
1 \\
-1
\end{array}\right), \quad \alpha_{3}^{\prime}=\left(\begin{array}{r}
0 \\
3 \\
-2
\end{array}\right)
$$

(1)求 $\mathscr{A}$ 在基 $\boldsymbol{\alpha}_{1}, \boldsymbol{\alpha}_{2}, \boldsymbol{\alpha}_{3}$ 下的矩阵表示 $\boldsymbol{A}$ ;\\
(2)求向量 $\boldsymbol{\xi}=(1,2,3)^{\mathrm{T}}$ 及 $\mathscr{B}(\boldsymbol{\xi})$ 在基 $\boldsymbol{\alpha}_{1}, \boldsymbol{\alpha}_{2}, \boldsymbol{\alpha}_{3}$ 下的坐标;\\
(3)求向量 $\boldsymbol{\xi}$ 及 $\mathcal{B}(\boldsymbol{\xi})$ 在基 $\boldsymbol{\alpha}_{1}{ }^{\prime}, \boldsymbol{\alpha}_{2}{ }^{\prime}, \boldsymbol{\alpha}_{3}{ }^{\prime}$ 下的坐标。\\
解(1)不难求得

$$
\begin{aligned}
& \mathscr{A}\left(\boldsymbol{\alpha}_{1}\right)=\boldsymbol{\alpha}_{1}^{\prime}=\boldsymbol{\alpha}_{1}-\boldsymbol{\alpha}_{2} \\
& \mathscr{A}\left(\boldsymbol{\alpha}_{2}\right)=\boldsymbol{\alpha}_{2}^{\prime}=-\boldsymbol{\alpha}_{1}+\boldsymbol{\alpha}_{2}+\boldsymbol{\alpha}_{3} \\
& \mathscr{A}\left(\boldsymbol{\alpha}_{3}\right)=\boldsymbol{\alpha}_{3}^{\prime}=-\boldsymbol{\alpha}_{1}+2 \boldsymbol{\alpha}_{2}+\boldsymbol{\alpha}_{3}
\end{aligned}
$$

因此 $\mathscr{A}$ 在 $\boldsymbol{\alpha}_{1}, \boldsymbol{\alpha}_{2}, \boldsymbol{a}_{3}$ 下矩阵表示为

$$
A=\left(\begin{array}{rrr}
1 & -1 & -1 \\
-1 & 1 & 2 \\
0 & 1 & 1
\end{array}\right)
$$

(2)设 $\boldsymbol{\xi}=\left(\boldsymbol{\alpha}_{1}, \boldsymbol{\alpha}_{2}, \boldsymbol{\alpha}_{3}\right)\left(\begin{array}{l}k_{1} \\ k_{2} \\ k_{3}\end{array}\right)$ ,即

$$
\left(\begin{array}{l}
1 \\
2 \\
3
\end{array}\right)=\left(\begin{array}{rrr}
1 & 0 & 1 \\
1 & 2 & 0 \\
-1 & -1 & -1
\end{array}\right)\left(\begin{array}{l}
k_{1} \\
k_{2} \\
k_{3}
\end{array}\right)
$$

解之得

$$
k_{1}=10, \quad k_{2}=-4, \quad k_{3}=-9
$$

所以 $\boldsymbol{\xi}$ 在基 $\boldsymbol{\alpha}_{1}, \boldsymbol{\alpha}_{2}, \boldsymbol{\alpha}_{3}$ 下坐标为 $(10,-4,-9)^{\mathbf{T}}$ 。\\
$\mathscr{B}(\boldsymbol{\xi})$ 在基 $\boldsymbol{\alpha}_{1}, \boldsymbol{\alpha}_{2}, \boldsymbol{\alpha}_{3}$ 下坐标可由式(1.6.2)得

$$
\left(\begin{array}{l}
y_{1} \\
y_{2} \\
y_{3}
\end{array}\right)=\left(\begin{array}{rrr}
1 & -1 & -1 \\
-1 & 1 & 2 \\
0 & 1 & 1
\end{array}\right)\left(\begin{array}{r}
10 \\
-4 \\
-9
\end{array}\right)=\left(\begin{array}{r}
23 \\
-32 \\
-13
\end{array}\right)
$$

(3) $\boldsymbol{\xi}$ 在基 $\boldsymbol{\alpha}_{1}{ }^{\prime}, \boldsymbol{\alpha}_{2}{ }^{\prime}, \boldsymbol{\alpha}_{3}{ }^{\prime}$ 下坐标为

$$
A^{-1}\left(\begin{array}{r}
10 \\
-4 \\
-9
\end{array}\right)=\left(\begin{array}{rrr}
1 & 0 & 1 \\
-1 & -1 & 1 \\
1 & 1 & 0
\end{array}\right)\left(\begin{array}{r}
10 \\
-4 \\
-9
\end{array}\right)=\left(\begin{array}{r}
1 \\
-15 \\
6
\end{array}\right)
$$

$\mathscr{A}(\xi)$ 在基 $\boldsymbol{\alpha}_{1}{ }^{\prime}, \boldsymbol{\alpha}_{2}{ }^{\prime}, \boldsymbol{\alpha}_{3}{ }^{\prime}$ 下坐标为

$$
A^{-1}\left(\begin{array}{r}
23 \\
-32 \\
-13
\end{array}\right)=\left(\begin{array}{rrr}
1 & 0 & 1 \\
-1 & -1 & 1 \\
1 & 1 & 0
\end{array}\right)\left(\begin{array}{r}
23 \\
-32 \\
-13
\end{array}\right)=\left(\begin{array}{r}
10 \\
-4 \\
-9
\end{array}\right)
$$

在 $\mathscr{A}$ 是线性变换情况下,定理1.4.2可改写为:\\
定理1.6.1 设 $\mathscr{A}$ 是 $V$ 到 $V$ 的线性变换, $\boldsymbol{\alpha}_{1}, \boldsymbol{\alpha}_{2}, \cdots, \boldsymbol{\alpha}_{n}$ 与 $\boldsymbol{\alpha}_{1}{ }^{\prime}, \boldsymbol{\alpha}_{2}{ }^{\prime}, \cdots, \boldsymbol{\alpha}_{n}{ }^{\prime}$ 是 $V$ 的两组基。由 $\boldsymbol{\alpha}_{i}$ 到 $\boldsymbol{\alpha}_{i}{ }^{\prime}$ 的过渡矩阵为 $\boldsymbol{P}$ ,线性变换 $\mathscr{A}$ 在基 $\boldsymbol{\alpha}_{1}, \boldsymbol{\alpha}_{2}, \cdots, \boldsymbol{\alpha}_{n}$ 下的矩阵表示为 $\boldsymbol{A}$ ,在基 $\boldsymbol{\alpha}_{1}{ }^{\prime}, \boldsymbol{\alpha}_{2}{ }^{\prime}, \cdots, \boldsymbol{\alpha}_{n}{ }^{\prime}$ 下的矩阵表示为 $\boldsymbol{B}$ ,则


\begin{equation*}
B=P^{-1} A P \tag{1.6.3}
\end{equation*}


定义1.6.1 设 $\boldsymbol{A}, \boldsymbol{B} \in F^{n \times n}$ ,若存在 $\boldsymbol{P} \in F_{n}^{n \times n}$ ,满足

$$
\boldsymbol{B}=\boldsymbol{P}^{-1} \boldsymbol{A P}
$$

则称 $\boldsymbol{B}$ 与 $\boldsymbol{A}$ 相似,记之为 $\boldsymbol{B} \sim \boldsymbol{A}$ 。\\
相似有下述简单性质:\\
(1)自反性: $\boldsymbol{A} \sim \boldsymbol{A}$ ;\\
(2)对称性:若 $\boldsymbol{B} \sim \boldsymbol{A}$ ,则 $\boldsymbol{A} \sim \boldsymbol{B}$ ;\\
(3)传递性:若 $\boldsymbol{A} \sim \boldsymbol{B}, \boldsymbol{B} \sim \boldsymbol{C}$ ,则 $\boldsymbol{A} \sim \boldsymbol{C}$ .

\section*{二、线性变换的运算}
设 $\mathscr{A}, \mathscr{B}$ 是线性空间 $V$ 的两个线性变换,定义它们的乘积 $\mathscr{A B}$ 为

$$
\mathscr{A} \mathscr{B}(\boldsymbol{\alpha})=\mathscr{B}(\mathscr{B}(\boldsymbol{\alpha})) \quad(\boldsymbol{\alpha} \in V)
$$

定义线性变换的加法 $(\mathscr{A}+\mathscr{B})$ 为

$$
(\mathscr{B}+\mathscr{B})(\boldsymbol{\alpha})=\mathscr{C}(\boldsymbol{\alpha})+\mathscr{B}(\boldsymbol{\alpha})
$$

定义数量乘法 $k \notin \mathscr{B}$ 为

$$
(k, \mathscr{C})(\boldsymbol{\alpha})=k, \mathscr{A}(\boldsymbol{\alpha})
$$

$V$ 的变换 $\mathscr{A}$ 称为可逆的,如果有 $V$ 的变换 $\mathscr{B}$ 存在,满足

$$
\mathscr{A B}=\mathscr{B} A B=E
$$

其中 $\boldsymbol{E}$ 是恒等变换,这时变换 $\mathscr{B}$ 称为 $\mathscr{A}$ 的逆变换,记为 $\mathscr{A}^{-1}$ 。\\
不难验证,上述定义的 $\mathscr{A}+\mathscr{B}, \mathscr{A} \mathscr{B}, k \mathscr{B}$ 与 $\mathscr{A}^{-1}$ 都是线性变换。\\
在 $n$ 维线性空间中取定一组基后,其上的一个线性变换, $\mathscr{A}$ 就与一个 $n$ 阶矩阵一一对应。这个对应且保持在线性变换的运算上。此即

定理1.6.2 设 $\boldsymbol{\alpha}_{1}, \boldsymbol{\alpha}_{2}, \cdots, \boldsymbol{\alpha}_{n}$ 是 $n$ 维线性空间 $\boldsymbol{V}$ 的一组基,在这组基下,线性变换 $\mathscr{A}$ 对应有一个 $n$ 阶矩阵 $\boldsymbol{A}$ ,线性变换 $\mathscr{B}$ 对应有一个 $n$ 阶矩阵 $\boldsymbol{B}$ 。这个对应还具有以下的几个性质:\\
(1)线性变换 $\mathscr{A}$ 与 $\mathscr{B}$ 的和 $\mathscr{B}+\mathscr{B}$ 对应于矩阵 $\boldsymbol{A}$ 与 $\boldsymbol{B}$ 的和 $\boldsymbol{A}+\boldsymbol{B}$ 。\\
(2)线性变换 $\mathscr{A}$ 的数量乘积 $k \mathscr{A}$ 对应于矩阵 $\boldsymbol{A}$ 的数量乘积 $k \boldsymbol{A}$ 。\\
(3)线性变换 $\mathscr{A}$ 与 $\mathscr{B}$ 的积 $\mathscr{A} \mathscr{B}$ 对应于矩阵 $\boldsymbol{A}$ 与 $\boldsymbol{B}$ 的积 $\boldsymbol{A B}$ 。\\
(4)若线性变换 $\mathscr{A}$ 可逆(即 $\mathscr{A}^{-1}$ 存在),则 $\mathscr{A}$ 对应的矩阵 $\boldsymbol{A}$ 可逆,且 $\mathscr{A}$ 的逆变换 $\mathscr{A}^{-1}$ 对应于矩阵 $\boldsymbol{A}$ 的逆矩阵 $\boldsymbol{A}^{-1}$ 。

证明 由已知条件知

$$
\begin{aligned}
& \mathscr{A}\left(\boldsymbol{\alpha}_{1}, \boldsymbol{\alpha}_{2}, \cdots, \boldsymbol{\alpha}_{n}\right)=\left(\boldsymbol{\alpha}_{1}, \boldsymbol{\alpha}_{2}, \cdots, \boldsymbol{\alpha}_{n}\right) \boldsymbol{A} \\
& \mathscr{B}\left(\boldsymbol{\alpha}_{1}, \boldsymbol{\alpha}_{2}, \cdots, \boldsymbol{\alpha}_{n}\right)=\left(\boldsymbol{\alpha}_{1}, \boldsymbol{\alpha}_{2}, \cdots, \boldsymbol{\alpha}_{n}\right) \boldsymbol{B}
\end{aligned}
$$

(1)$(\mathscr{A}+\mathscr{B})\left(\boldsymbol{\alpha}_{1}, \boldsymbol{\alpha}_{2}, \cdots, \boldsymbol{\alpha}_{n}\right)$

$$
\begin{aligned}
& =\mathscr{A}\left(\boldsymbol{\alpha}_{1}, \boldsymbol{\alpha}_{2}, \cdots, \boldsymbol{\alpha}_{n}\right)+\mathscr{B}\left(\boldsymbol{\alpha}_{1}, \boldsymbol{\alpha}_{2}, \cdots, \boldsymbol{\alpha}_{n}\right) \\
& =\left(\boldsymbol{\alpha}_{1}, \boldsymbol{\alpha}_{2}, \cdots, \boldsymbol{\alpha}_{n}\right) \boldsymbol{A}+\left(\boldsymbol{\alpha}_{1}, \boldsymbol{\alpha}_{2}, \cdots, \boldsymbol{\alpha}_{n}\right) \boldsymbol{B} \\
& =\left(\boldsymbol{\alpha}_{1}, \boldsymbol{\alpha}_{2}, \cdots, \boldsymbol{\alpha}_{n}\right)(\boldsymbol{A}+\boldsymbol{B})
\end{aligned}
$$

(2)$(k \mathscr{A})\left(\boldsymbol{\alpha}_{1}, \boldsymbol{\alpha}_{2}, \cdots, \boldsymbol{\alpha}_{n}\right)$

$$
\begin{aligned}
& =k\left[\mathscr{A}\left(\boldsymbol{\alpha}_{1}, \boldsymbol{\alpha}_{2}, \cdots, \boldsymbol{\alpha}_{n}\right)\right] \\
& =k\left(\boldsymbol{\alpha}_{1}, \boldsymbol{\alpha}_{2}, \cdots, \boldsymbol{\alpha}_{n}\right) \boldsymbol{A} \\
& =\left(\boldsymbol{\alpha}_{1}, \boldsymbol{\alpha}_{2}, \cdots, \boldsymbol{\alpha}_{n}\right)(k \boldsymbol{A})
\end{aligned}
$$

(3)$(\mathscr{A B} \mathscr{B})\left(\boldsymbol{\alpha}_{1}, \boldsymbol{\alpha}_{2}, \cdots, \boldsymbol{\alpha}_{n}\right)$

$$
\begin{aligned}
& =\mathscr{B}\left[\mathscr{B}\left(\boldsymbol{\alpha}_{1}, \boldsymbol{\alpha}_{2}, \cdots, \boldsymbol{\alpha}_{n}\right)\right] \\
& =\mathscr{A}\left[\left(\boldsymbol{\alpha}_{1}, \boldsymbol{\alpha}_{2}, \cdots, \boldsymbol{\alpha}_{n}\right) \boldsymbol{B}\right] \\
& =\left[\mathscr{B}\left(\boldsymbol{\alpha}_{1}, \boldsymbol{\alpha}_{2}, \cdots, \boldsymbol{\alpha}_{n}\right)\right] \boldsymbol{B} \\
& =\left(\boldsymbol{\alpha}_{1}, \boldsymbol{\alpha}_{2}, \cdots, \boldsymbol{\alpha}_{n}\right) \boldsymbol{A} \boldsymbol{B}
\end{aligned}
$$

(4) $\mathscr{A}^{-1}\left(\boldsymbol{\alpha}_{1}, \boldsymbol{\alpha}_{2}, \cdots, \boldsymbol{\alpha}_{n}\right)=\left(\boldsymbol{\alpha}_{1}, \boldsymbol{\alpha}_{2}, \cdots, \boldsymbol{\alpha}_{n}\right) \boldsymbol{X}$

$$
\begin{aligned}
\mathscr{A} . \mathscr{A}^{-1}\left(\boldsymbol{\alpha}_{1}, \boldsymbol{\alpha}_{2}, \cdots, \boldsymbol{\alpha}_{n}\right) & =\mathscr{A}\left(\boldsymbol{\alpha}_{1}, \boldsymbol{\alpha}_{2}, \cdots, \boldsymbol{\alpha}_{n}\right) \boldsymbol{X} \\
\left(\boldsymbol{\alpha}_{1}, \boldsymbol{\alpha}_{2}, \cdots, \boldsymbol{\alpha}_{n}\right) & =\left(\boldsymbol{\alpha}_{1}, \boldsymbol{\alpha}_{2}, \cdots, \boldsymbol{\alpha}_{n}\right) A \boldsymbol{X}
\end{aligned}
$$

于是 $\boldsymbol{X}=\boldsymbol{A}^{-1}$ .

\section*{§ $1.7 n$ 维线性空间的同构}
在线性空间 $V$ 中取基 $\boldsymbol{\alpha}_{1}, \boldsymbol{\alpha}_{2}, \cdots, \boldsymbol{\alpha}_{n}$ ,则 $V$ 中任何一个向量 $\boldsymbol{\alpha}$ 可以表示成

$$
\boldsymbol{\alpha}=x_{1} \boldsymbol{\alpha}_{1}+x_{2} \boldsymbol{\alpha}_{2}+\cdots+x_{n} \boldsymbol{\alpha}_{n}=\left(\boldsymbol{\alpha}_{1}, \boldsymbol{\alpha}_{2}, \cdots, \boldsymbol{\alpha}_{n}\right)\left[\begin{array}{c}
x_{1} \\
x_{2} \\
\vdots \\
x_{n}
\end{array}\right]
$$

此即 $\boldsymbol{\alpha}$ 对应于一个 $n$ 维列向量空间中(坐标)向量 $\left(x_{1}, x_{2}, \cdots, x_{n}\right)^{\mathrm{T}}$ 。由于 $\boldsymbol{\alpha}$ 的坐标是唯一的,所以这就给出了 $V$ 到 $F^{n}$ 的一个一一对应:

$$
\boldsymbol{\alpha} \longleftrightarrow\left(x_{1}, x_{2}, \cdots, x_{n}\right)^{\mathbf{T}}
$$

若设 $\boldsymbol{\alpha} \boldsymbol{\beta} \in V$ ,且有

$$
\boldsymbol{\alpha} \longleftrightarrow\left(x_{1}, x_{2}, \cdots, x_{n}\right)^{\mathrm{T}}, \boldsymbol{\beta} \longleftrightarrow\left(y_{1}, y_{2}, \cdots, y_{n}\right)^{\mathrm{T}}
$$

则由

$$
\begin{gathered}
\boldsymbol{\alpha}+\boldsymbol{\beta}=\left(x_{1}+y_{1}\right) \boldsymbol{\alpha}_{1}+\cdots+\left(x_{n}+y_{n}\right) \boldsymbol{\alpha}_{n} \\
\lambda \boldsymbol{\alpha}=\left(\lambda x_{1}\right) \boldsymbol{\alpha}_{1}+\cdots+\left(\lambda x_{n}\right) \boldsymbol{\alpha}_{n}
\end{gathered}
$$

可知

$$
\begin{gathered}
\boldsymbol{\alpha}+\boldsymbol{\beta} \longleftrightarrow\left(x_{1}+y_{1}, x_{2}+y_{2}, \cdots, x_{n}+y_{n}\right)^{\mathrm{T}} \\
\lambda \boldsymbol{\alpha} \longleftrightarrow\left(\lambda x_{1}, \lambda x_{2}, \cdots, \lambda x_{n}\right)^{\mathrm{T}}
\end{gathered}
$$

这说明上述 $n$ 维线性空间 $V$ 中向量 $\boldsymbol{\alpha}$ 与 $F^{n}$ 中列向量 $\left(x_{1}, x_{2}, \cdots, x_{n}\right)^{\mathrm{T}}$ 之间一一对应关系,使得 $V$ 中向量之和与数乘向量对应于 $F^{n}$ 中列向量之和与数乘列向量。将这个性质抽象化便是同构的概念。

定义1.7.1 若两个线性空间 $V_{1}$ 与 $V_{2}$ ,存在 $V_{1}$ 到 $V_{2}$ 上的一个一一对应 $\sigma$ ,使得对于所有向量 $\boldsymbol{\alpha}, \boldsymbol{\beta} \in V_{1}$ ,数 $\lambda$ 都有\\
(1)$\sigma(\boldsymbol{\alpha}+\boldsymbol{\beta})=\sigma(\boldsymbol{\alpha})+\sigma(\boldsymbol{\beta})$ ,\\
(2)$\sigma(\lambda \boldsymbol{\alpha})=\lambda \sigma(\boldsymbol{\alpha})$ .\\
则此一一对应 $\sigma$ 称为 $V_{1}$ 到 $V_{2}$ 的同构映射,称 $V_{1}$ 与 $V_{2}$ 是同构的。\\
由此定义可见,上述 $n$ 维线性空间 $V$ 取定一组基 $\boldsymbol{\alpha}_{1}, \boldsymbol{\alpha}_{2}, \cdots, \boldsymbol{\alpha}_{n}$ ,则 $V$ 中向量和它的坐标 $\left(x_{1}, x_{2}, \cdots, x_{n}\right)^{\mathrm{T}}$ 之间给出的一一对应是 $V$ 到 $F^{n}$ 上的一个同构映射。于是数域 $F$ 上任一个 $n$ 维线性空间都与 $n$ 维列向量空间 $F^{n}$ 同构。

\section*{同构映射具有下列基本性质}
(1)$\sigma(0)=0, \sigma(-\alpha)=-\sigma(\alpha)$ .\\
(2)$\sigma\left(k_{1} \boldsymbol{\alpha}_{1}+k_{2} \boldsymbol{\alpha}_{2}+\cdots+k_{s} \boldsymbol{\alpha}_{s}\right)=k_{1} \sigma\left(\boldsymbol{\alpha}_{1}\right)+k_{2} \sigma\left(\boldsymbol{\alpha}_{2}\right)+\cdots+k_{s} \sigma\left(\boldsymbol{\alpha}_{s}\right)$ .\\
(3)$V$ 中向量组 $\boldsymbol{\alpha}_{1}, \boldsymbol{\alpha}_{2}, \cdots, \boldsymbol{\alpha}_{s}$ 线性相(无)关 $\Leftrightarrow$ 像 $\boldsymbol{\sigma}\left(\boldsymbol{\alpha}_{1}\right), \sigma\left(\boldsymbol{\alpha}_{2}\right), \cdots, \sigma\left(\boldsymbol{\alpha}_{s}\right)$线性相(无)关。\\
(4)如果 $V_{1}$ 是 $V$ 的一个子空间,则 $V_{1}$ 在 $\sigma$ 下的像集合 $\sigma\left(V_{1}\right)=|\sigma(\alpha)| \alpha \in \left.V_{1}\right\}$ 是 $\sigma(V)$ 的子空间,并且 $V_{1}$ 与 $\sigma\left(V_{1}\right)$ 维数相同。

定理1.7.1 数域 $F$ 上两个有限维线性空间同构 $\Longleftrightarrow$ 两个线性空间有相同的维数。

在研究线性空间时,不关心线性空间(例如 $R^{n \times n}, R^{n}, R[x]_{n}, \cdots$ )的元素( $n$ 阶矩阵,$n$ 维列向量,次数不大于 $n$ 的多项式,$\cdots$ )是什么,也不关心其中运算是怎样定义的,而只关心线性空间在所定义运算下的代数性质。于是,同构的线性空间是可以不加区别的。换句话说,维数是有限维线性空间唯一的本质特征。因此,$n$ 维向量空间中的一些结论在 $n$ 维线性空间中也是成立的,不必一一重新证明。

例1.7.1 已知 $\boldsymbol{\alpha}_{1}=\left[\begin{array}{ll}1 & 0 \\ 1 & 1\end{array}\right], \boldsymbol{\alpha}_{2}=\left[\begin{array}{ll}0 & 1 \\ 1 & 1\end{array}\right], \boldsymbol{\alpha}_{3}=\left[\begin{array}{ll}1 & 1 \\ 0 & 2\end{array}\right], \boldsymbol{\alpha}_{4}=\left[\begin{array}{ll}1 & 3 \\ 1 & 0\end{array}\right]$ 是 $R^{2 \times 2}$ 中一组基。 $R^{2 \times 2}$ 中线性变换 $\mathscr{A}$ 满足, $\mathscr{B}\left(\boldsymbol{\alpha}_{1}\right)=\left[\begin{array}{ll}1 & 1 \\ 0 & 0\end{array}\right], \mathscr{A}\left(\boldsymbol{\alpha}_{2}\right)=\left[\begin{array}{ll}0 & 0 \\ 0 & 0\end{array}\right], \mathscr{A}\left(\boldsymbol{\alpha}_{3}\right)= \left[\begin{array}{ll}0 & 0 \\ 1 & 1\end{array}\right], \mathscr{A}\left(\boldsymbol{\alpha}_{4}\right)=\left[\begin{array}{ll}0 & 1 \\ 0 & 1\end{array}\right]$ .求 $\mathscr{A}$ 在 $\boldsymbol{\alpha}_{1}, \boldsymbol{\alpha}_{2}, \boldsymbol{\alpha}_{3}, \boldsymbol{\alpha}_{4}$ 下的矩阵表示。

解 $R^{2 \times 2}$ 是 4 维线性空间,利用同构的概念,可把题中矩阵写成向量形式

$$
\begin{aligned}
\boldsymbol{\alpha}_{1} & =(1,0,1,1)^{\mathrm{T}}, \boldsymbol{\alpha}_{2}=(0,1,1,1)^{\mathrm{T}}, \\
\boldsymbol{\alpha}_{3} & =(1,1,0,2)^{\mathrm{T}}, \boldsymbol{\alpha}_{4}=(1,3,1,0)^{\mathrm{T}} \\
\mathscr{A}\left(\boldsymbol{\alpha}_{1}\right) & =(1,1,0,0)^{\mathrm{T}}, \mathscr{A}\left(\boldsymbol{\alpha}_{2}\right)=(0,0,0,0)^{\mathrm{T}} \\
\mathscr{A}\left(\boldsymbol{\alpha}_{3}\right) & =(0,0,1,1)^{\mathrm{T}}, \mathscr{A}\left(\boldsymbol{\alpha}_{4}\right)=(0,1,0,1)^{\mathrm{T}}
\end{aligned}
$$

于是

$$
\begin{aligned}
\mathscr{B}\left(\boldsymbol{\alpha}_{1}, \boldsymbol{\alpha}_{2}, \boldsymbol{\alpha}_{3}, \boldsymbol{\alpha}_{4}\right) & =\left(\mathscr{A}\left(\boldsymbol{\alpha}_{1}\right), \mathscr{A}\left(\boldsymbol{\alpha}_{2}\right), \mathscr{A}\left(\boldsymbol{\alpha}_{3}\right), \mathscr{A}\left(\boldsymbol{\alpha}_{4}\right)\right) \\
& =\left[\begin{array}{llll}
1 & 0 & 0 & 0 \\
1 & 0 & 0 & 1 \\
0 & 0 & 1 & 0 \\
0 & 0 & 1 & 1
\end{array}\right]=\left(\boldsymbol{\alpha}_{1}, \boldsymbol{\alpha}_{2}, \boldsymbol{\alpha}_{3}, \boldsymbol{\alpha}_{4}\right) \boldsymbol{A} \\
& =\left[\begin{array}{llll}
1 & 0 & 1 & 1 \\
0 & 1 & 1 & 3 \\
1 & 1 & 0 & 1 \\
1 & 1 & 2 & 0
\end{array}\right] \boldsymbol{A}
\end{aligned}
$$

于是

$$
A=\left[\begin{array}{llll}
1 & 0 & 1 & 1 \\
0 & 1 & 1 & 3 \\
1 & 1 & 0 & 1 \\
1 & 1 & 2 & 0
\end{array}\right]^{-1}\left[\begin{array}{llll}
1 & 0 & 0 & 0 \\
1 & 0 & 0 & 1 \\
0 & 0 & 1 & 0 \\
0 & 0 & 1 & 1
\end{array}\right]
$$

$$
=\left[\begin{array}{rrrr}
\frac{1}{4} & 0 & \frac{3}{8} & -1 \\
-\frac{3}{4} & 0 & \frac{7}{8} & 1 \\
\frac{1}{4} & 0 & -\frac{1}{8} & 1 \\
\frac{1}{2} & 0 & -\frac{1}{4} & 0
\end{array}\right]
$$

注 根据同构映射的定义,$R^{2 \times 2}$ 中矩阵 $\left[\begin{array}{ll}a_{11} & a_{12} \\ a_{21} & a_{22}\end{array}\right]$ 可以看做 $R^{4}$ 中向量 $\left(a_{11}\right.$ , $\left.a_{12}, a_{21}, a_{22}\right)^{\mathrm{T}}$.

\section*{§1.8 线性变换的特征值与特征向量}
线性变换的特征值与特征向量是非常重要的概念,它在物理、力学、工程中都有实际意义与应用。

\section*{一、线性变换的特征值与特征向量}
定义1.8.1 设 $\mathscr{A}$ 是数域 $F$ 上的 $n$ 维线性空间 $V$ 的线性变换,如果在 $V$ 中存在一个非零向量 $\boldsymbol{\alpha}$ 使得


\begin{equation*}
\mathscr{A}(\boldsymbol{\alpha})=\lambda_{0} \boldsymbol{\alpha}, \boldsymbol{\lambda}_{0} \in \boldsymbol{F} \tag{1.8.1}
\end{equation*}


那么称 $\lambda_{0}$ 是 $\mathscr{A}$ 的一个特征值,称 $\boldsymbol{\alpha}$ 是 $\mathscr{A}$ 的属于特征值 $\lambda_{0}$ 的一个特征向量。\\
从几何上看,变换前后的特征向量仍然共线,或者方向不变( $\lambda_{0}>0$ ),或者方向相反 $\left(\lambda_{0}<0\right)$ ,或者变为零向量 $\left(\lambda_{0}=0\right) \ldots A$ 对 $\boldsymbol{\alpha}$ 的作用是将 $\boldsymbol{\alpha}$ 拉长(或缩短) $\lambda_{0}$ 倍,这个倍数 $\lambda_{0}$ 即为 $\mathscr{A}$ 的一个特征值。

下面介绍线性变换 $\mathcal{A}$ 的特征值和特征向量的计算。\\
设 $\boldsymbol{\alpha}_{1}, \boldsymbol{\alpha}_{2}, \cdots, \boldsymbol{\alpha}_{n}$ 是 $n$ 维线性空间 $V$ 的一组基,线性变换 $\mathscr{A}$ 在这组基下的矩阵表示是 $\boldsymbol{A}$ 。若设 $\lambda_{0}$ 是 $\mathscr{A}$ 的一个特征值,它的一个特征向量 $\boldsymbol{\alpha}$ 在基 $\boldsymbol{\alpha}_{1}, \boldsymbol{\alpha}_{2}, \cdots, \boldsymbol{\alpha}_{n}$下的坐标是 $\left(x_{1}, x_{2}, \cdots, x_{n}\right)^{\mathrm{T}}$ ,即

\[
\boldsymbol{\alpha}=\left(\boldsymbol{\alpha}_{1}, \boldsymbol{\alpha}_{2}, \cdots, \boldsymbol{\alpha}_{n}\right)\left[\begin{array}{c}
x_{1}  \tag{1.8.2}\\
x_{2} \\
\vdots \\
x_{n}
\end{array}\right]
\]

把式(1.8.2)代人式(1.8.1)得

$$
\mathscr{B}\left(\boldsymbol{\alpha}_{1}, \boldsymbol{\alpha}_{2}, \cdots, \boldsymbol{\alpha}_{n}\right)\left[\begin{array}{c}
x_{1} \\
x_{2} \\
\vdots \\
x_{n}
\end{array}\right]=\lambda_{0}\left(\boldsymbol{\alpha}_{1}, \boldsymbol{\alpha}_{2}, \cdots, \boldsymbol{\alpha}_{n}\right)\left[\begin{array}{c}
x_{1} \\
x_{2} \\
\vdots \\
x_{n}
\end{array}\right]
$$

此即

$$
\left(\boldsymbol{\alpha}_{1}, \boldsymbol{\alpha}_{2}, \cdots, \boldsymbol{\alpha}_{n}\right) \boldsymbol{A}\left[\begin{array}{c}
x_{1} \\
x_{2} \\
\vdots \\
x_{n}
\end{array}\right]=\lambda_{0}\left(\boldsymbol{\alpha}_{1}, \boldsymbol{\alpha}_{2}, \cdots, \boldsymbol{\alpha}_{n}\right)\left[\begin{array}{c}
x_{1} \\
x_{2} \\
\vdots \\
x_{n}
\end{array}\right]
$$

由 $\boldsymbol{\alpha}_{1}, \boldsymbol{\alpha}_{2}, \cdots, \boldsymbol{\alpha}_{n}$ 线性无关得

\[
\boldsymbol{A}\left[\begin{array}{c}
x_{1}  \tag{1.8.3}\\
x_{2} \\
\vdots \\
x_{n}
\end{array}\right]=\lambda_{0}\left[\begin{array}{c}
x_{1} \\
x_{2} \\
\vdots \\
x_{n}
\end{array}\right]
\]

式(1.8.3)是 $\mathscr{A}$ 的特征向量 $\boldsymbol{\alpha}$ 在基 $\boldsymbol{\alpha}_{1}, \boldsymbol{\alpha}_{2}, \cdots, \boldsymbol{\alpha}_{n}$ 下的坐标 $\left(x_{1}, x_{2}, \cdots, x_{n}\right)^{\mathrm{T}}$ 满足的关系式,坐标 $\left(x_{1}, x_{2}, \cdots, x_{n}\right)^{\mathrm{T}}$ 是齐次线性方程组


\begin{equation*}
\left(\lambda_{0} \boldsymbol{E}_{n}-\boldsymbol{A}\right) \boldsymbol{X}=0 \tag{1.8.4}
\end{equation*}


的非零解,它有非零解的充分必要条件是


\begin{equation*}
\left|\lambda_{0} \boldsymbol{E}-\boldsymbol{A}\right|=0 \tag{1.8.5}
\end{equation*}


$n$ 阶行列式


\begin{align*}
\left|\lambda_{0} \boldsymbol{E}-\boldsymbol{A}\right|= & \lambda_{0}^{n}-\boldsymbol{E}_{1}(\boldsymbol{A}) \lambda_{0}^{n-1}+\boldsymbol{E}_{2}(\boldsymbol{A}) \lambda_{0}^{n-2}+\cdots+ \\
& (-1)^{n} \boldsymbol{E}_{n}(\boldsymbol{A})=0 \tag{1.8.6}
\end{align*}


式中 $\boldsymbol{E}_{1}(\boldsymbol{A}), \boldsymbol{E}_{2}(\boldsymbol{A}), \cdots, \boldsymbol{E}_{n}(\boldsymbol{A})$ 依次是 $\boldsymbol{A}$ 的 $1,2, \cdots, n$ 阶子行列式之和。特别地,

$$
\begin{aligned}
& \boldsymbol{E}_{1}(\boldsymbol{A})=\operatorname{tr} \boldsymbol{A}=a_{11}+a_{22}+\cdots+a_{n n} \\
& \boldsymbol{E}_{n}(\boldsymbol{A})=|\boldsymbol{A}|
\end{aligned}
$$

定义1.8.2 设 $\boldsymbol{A}$ 是数域 $F$ 上的 $n$ 阶矩阵,$\lambda$ 是一个数字,矩阵 $\lambda \boldsymbol{E}-\boldsymbol{A}$ 称为 $\boldsymbol{A}$的特征矩阵,行列式

\[
\left|\lambda \boldsymbol{E}_{n}-\boldsymbol{A}\right|=\left|\begin{array}{ccccc}
\lambda-a_{11} & -a_{12} & -a_{13} & \cdots & -a_{1 n}  \tag{1.8.7}\\
-a_{21} & \lambda-a_{22} & -a_{23} & \cdots & -a_{2 n} \\
\vdots & \vdots & \vdots & & \vdots \\
-a_{n 1} & -a_{n 2} & -a_{n 3} & \cdots & \lambda-a_{n n}
\end{array}\right|
\]

称为 $\boldsymbol{A}$ 的特征多项式.$n$ 次代数方程 $\left|\lambda \boldsymbol{E}_{n}-\boldsymbol{A}\right|=0$ 称为 $\boldsymbol{A}$ 的特征方程,它的根称为 $\boldsymbol{A}$ 的特征根(或特征值)。以 $\boldsymbol{A}$ 的特征值 $\lambda_{0}$ 代人方程组(1.8.4)所得的非零解,称为 $\boldsymbol{A}$ 的对应于(属于)特征值 $\lambda_{0}$ 的特征向量。矩阵 $\boldsymbol{A}$ 的特征多项式在复数范围

内有 $n$ 个根。因此一个 $n$ 阶方阵有 $n$ 个特征值(重根应计及重数)。矩阵 $\boldsymbol{A}$ 的所有特征值的全体称为 $\boldsymbol{A}$ 的谱,用 $\boldsymbol{\lambda}(\boldsymbol{A})$ 表示。

注 $n$ 阶矩阵 $\boldsymbol{A}$ 的特征值、特征向量的计算本书不介绍,读者可参阅线性代数。

由上面分析可以看到,计算线性变换 $\mathscr{A}$ 的特征值和特征向量变成计算矩阵 $\boldsymbol{A}$ (线性变换在某一个基下的矩阵)的特征值和特征向量。即\\
(1)$\lambda_{0}$ 是 $\mathscr{A}$ 的一个特征值 $\Longleftrightarrow \lambda_{0}$ 是 $\boldsymbol{A}$ 的一个特征值。\\
(2) $\boldsymbol{\alpha}$ 是 $\mathscr{A}$ 的属于特征值 $\lambda_{0}$ 的一个特征向量 $\Longleftrightarrow \boldsymbol{\alpha}$ 的坐标 $\left(x_{1}, x_{2}, \cdots, x_{n}\right)^{\mathrm{T}}$是 $\boldsymbol{A}$ 的属于特征值 $\lambda_{0}$ 的特征向量。

由此可知,求线性变换 $\mathscr{A}$ 的特征值与特征向量只需在 $V$ 的某一个基底上,求 $\mathscr{A}$ 的矩阵表示 $\boldsymbol{A}$ 的特征值与特征向量,需要注意的是由 $\boldsymbol{A}$ 得到的特征向量,它是 $\mathscr{A}$ 的特征向量 $\boldsymbol{\alpha}$ 的坐标(在基下的)向量。

线性变换 $\mathscr{A}$ 在不同基下的矩阵表示是不同的,那么在计算 $\mathscr{A}$ 的特征值和特征向量时采用某一个基下的矩阵表示的特征值和特征向量是否有意义?这是下面要回答的内容。

定理1.8.1 相似矩阵有相同的特征值(证略)。\\
由定理1.8.1可知, $\mathscr{A}$ 的特征值可以通过 $\mathscr{A}$ 的任何一个矩阵表示来计算,特征值都是相同的。其余的问题是用线性变换 $A$ 的不同矩阵 $A, B \cdots$ 表示出来的特征向量是否代表 $\mathscr{A}$ 的特征向量?

若 $\mathscr{A}$ 在基 $\boldsymbol{\alpha}_{1}, \boldsymbol{\alpha}_{2}, \cdots, \boldsymbol{\alpha}_{n}$ 下的矩阵表示为 $\boldsymbol{A}$ ,在基 $\boldsymbol{\beta}_{1}, \boldsymbol{\beta}_{2}, \cdots, \boldsymbol{\beta}_{n}$ 下的矩阵表示为 $\boldsymbol{B}$ ,且

$$
\left(\boldsymbol{\beta}_{1}, \boldsymbol{\beta}_{2}, \cdots, \boldsymbol{\beta}_{n}\right)=\left(\boldsymbol{\alpha}_{1}, \boldsymbol{\alpha}_{2}, \cdots, \boldsymbol{\alpha}_{n}\right) \boldsymbol{P}
$$

由定理1.6.1可知

$$
B=P^{-1} A P
$$

定理1.8.2 若 $\xi=\left(x_{1}, x_{2}, \cdots, x_{n}\right)^{\mathrm{T}}$ 是 $n$ 阶矩阵 $A$ 的属于特征值 $\lambda$ 的特征向量, $\boldsymbol{B}=\boldsymbol{P}^{-1} \boldsymbol{A P}$ ,则 $\boldsymbol{P}^{-1} \boldsymbol{\xi}$ 是 $\boldsymbol{B}$ 的属于特征值 $\lambda$ 的特征向量。

由定理1.8.2可知, $\boldsymbol{A}$ 或 $\boldsymbol{B}$ 的特征向量便是 $\mathscr{A}$ 的特征向量,因为 $\boldsymbol{A}$ 的特征向量 $\left(x_{1}, x_{2}, \cdots, x_{n}\right)^{\mathrm{T}}$ 是 $\mathscr{A}$ 的特征向量 $\boldsymbol{\alpha}$ 的坐标向量,且

$$
\boldsymbol{\alpha}=\left(\boldsymbol{\alpha}_{1}, \boldsymbol{\alpha}_{2}, \cdots, \boldsymbol{\alpha}_{n}\right)\left[\begin{array}{c}
x_{1} \\
x_{2} \\
\vdots \\
x_{n}
\end{array}\right]
$$

$\boldsymbol{B}$ 的特征向量 $\boldsymbol{P}^{-1}\left(x_{1}, x_{2}, \cdots, x_{n}\right)^{\mathrm{T}}$ 是对应的 $\mathscr{A}$ 的特征向量 $\boldsymbol{\beta}$ 的坐标向量,且

$$
\begin{aligned}
\left(\boldsymbol{\beta}_{1}, \boldsymbol{\beta}_{2}, \cdots, \boldsymbol{\beta}_{n}\right) \boldsymbol{P}^{-1}\left[\begin{array}{c}
x_{1} \\
x_{2} \\
\vdots \\
x_{n}
\end{array}\right] & =\left(\boldsymbol{\alpha}_{1}, \boldsymbol{\alpha}_{2}, \cdots, \boldsymbol{\alpha}_{n}\right) \boldsymbol{P} \cdot \boldsymbol{P}^{-1}\left[\begin{array}{c}
x_{1} \\
x_{2} \\
\vdots \\
x_{n}
\end{array}\right] \\
& =\left(\boldsymbol{\alpha}_{1}, \boldsymbol{\alpha}_{2}, \cdots, \boldsymbol{\alpha}_{n}\right)\left[\begin{array}{c}
x_{1} \\
x_{2} \\
\vdots \\
x_{n}
\end{array}\right]=\boldsymbol{\alpha}
\end{aligned}
$$

因此,求 $\mathscr{A}$ 的特征向量可以通过 $\mathscr{A}$ 的任何一个矩阵表示的特征向量而得到。

例1.8.1 已知 $\boldsymbol{\alpha}_{1}=(0,1,1)^{\mathrm{T}}, \boldsymbol{\alpha}_{2}=(1,2,0)^{\mathrm{T}}, \boldsymbol{\alpha}_{3}=(0,0,1)^{\mathrm{T}}$ 是线性空间 $R^{3}$的一组基,$R^{3}$ 中线性变换 $\mathscr{A}$ 满足 $\mathscr{A}\left(\boldsymbol{\alpha}_{1}\right)=(2,5,-2)^{\mathrm{T}}, \mathscr{A}\left(\boldsymbol{\alpha}_{2}\right)=(4,7,-4)^{\mathrm{T}}$ , $\mathscr{A}\left(\boldsymbol{\alpha}_{3}\right)=(-2,-3,6)^{\mathrm{T}}$ .求线性变换 $\mathscr{A}$ 的特征值与特征向量.

解 由题意可知

$$
\begin{aligned}
& \mathscr{A}\left(\alpha_{1}, \alpha_{2}, \alpha_{3}\right)=\left(\mathscr{A}\left(\alpha_{1}\right), \mathscr{B}\left(\alpha_{2}\right), \mathscr{A}\left(\alpha_{3}\right)\right) \\
= & {\left[\begin{array}{rrr}
2 & 4 & -2 \\
5 & 7 & -3 \\
-2 & -4 & 6
\end{array}\right]=\left(\alpha_{1}, \alpha_{2}, \alpha_{3}\right) A } \\
= & {\left[\begin{array}{lll}
0 & 1 & 0 \\
1 & 2 & 0 \\
1 & 0 & 1
\end{array}\right] A }
\end{aligned}
$$

于是, $\mathscr{A}$ 在 $\boldsymbol{\alpha}_{1}, \boldsymbol{\alpha}_{2}, \boldsymbol{\alpha}_{3}$ 下的矩阵 $\boldsymbol{A}$ 表示为

$$
\begin{aligned}
A & =\left[\begin{array}{lll}
0 & 1 & 0 \\
1 & 2 & 0 \\
1 & 0 & 1
\end{array}\right]^{-1}\left[\begin{array}{rrr}
2 & 4 & -2 \\
5 & 7 & -3 \\
-2 & -4 & 6
\end{array}\right] \\
& =\left[\begin{array}{rrr}
1 & -1 & 1 \\
2 & 4 & -2 \\
-3 & -3 & 5
\end{array}\right]
\end{aligned}
$$

$\boldsymbol{A}$ 的特征多项式

$$
|\lambda \boldsymbol{E}-\boldsymbol{A}|=(\lambda-2)^{2}(\lambda-6)
$$

$\boldsymbol{A}$ 的特征值

$$
\lambda_{1}=\lambda_{2}=2, \lambda_{3}=6
$$

$\boldsymbol{A}$ 的属于特征值 2 有两个线性无关特征向量

$$
\boldsymbol{\xi}_{1}=(-1,1,0)^{\mathrm{T}}, \boldsymbol{\xi}_{2}=(1,0,1)^{\mathrm{T}}
$$

$\boldsymbol{A}$ 的属于特征值 6 有特征向量

$$
\xi_{3}=(1,-2,3)^{\mathrm{T}}
$$

所以 $\mathscr{A}$ 的特征值 $\lambda_{1}=\lambda_{2}=2, \lambda_{3}=6$ .\\
$\mathscr{A}$ 的属于特征值 2 有两个线性无关特征向量 $\boldsymbol{\eta}_{1}=\left(\boldsymbol{\alpha}_{1}, \boldsymbol{\alpha}_{2}, \boldsymbol{\alpha}_{3}\right)\left[\begin{array}{r}-1 \\ 1 \\ 0\end{array}\right]=-\boldsymbol{\alpha}_{1}+\boldsymbol{\alpha}_{2}= (1,1,-1)^{\mathrm{T}}, \boldsymbol{\eta}_{2}=\left(\boldsymbol{\alpha}_{1}, \boldsymbol{\alpha}_{2}, \boldsymbol{\alpha}_{3}\right)\left[\begin{array}{l}1 \\ 0 \\ 1\end{array}\right]=\boldsymbol{\alpha}_{1}+\boldsymbol{\alpha}_{3}=(0,1,2)^{\mathrm{T}}$ .于是 $\mathscr{A}$ 的属于特征值 2的全部特征向量为 $k_{1} \boldsymbol{\eta}_{1}+k_{2} \boldsymbol{\eta}_{2}$ ,其中 $k_{1}, k_{2}$ 是不全为零的数。\\
$\mathscr{A}$ 的属于特征值 6 有特征向量 $\boldsymbol{\eta}_{3}=\left(\boldsymbol{\alpha}_{1}, \boldsymbol{\alpha}_{2}, \boldsymbol{\alpha}_{3}\right)\left[\begin{array}{r}1 \\ -2 \\ 3\end{array}\right]=\boldsymbol{\alpha}_{1}-2 \boldsymbol{\alpha}_{2}+3 \boldsymbol{\alpha}_{3}= (-2,-3,4)^{\mathrm{T}}$ .于是 $\mathscr{A}$ 的属于特征值 6 的全部特征向量为 $k_{3} \boldsymbol{\eta}_{3}$ ,其中 $k_{3}$ 为非零数。

例1.8.2 在线性空间 $R[x]_{n}$ 中取一个基

$$
1, x, \frac{1}{2!} x^{2}, \cdots, \frac{1}{(n-1)!} x^{n-1} .
$$

容易求出微分变换 $\mathscr{D}(f(x))=f^{\prime}(x)$ 在该基下的矩阵是

$$
\boldsymbol{D}=\left[\begin{array}{ccccc}
0 & 1 & 0 & \cdots & 0 \\
0 & 0 & 1 & \cdots & 0 \\
\vdots & \vdots & \vdots & & \vdots \\
0 & 0 & 0 & \cdots & 1 \\
0 & 0 & 0 & \cdots & 0
\end{array}\right]
$$

其特征多项式为

$$
|\lambda \boldsymbol{E}-\boldsymbol{D}|=\left|\begin{array}{ccccc}
\lambda & -1 & 0 & \cdots & 0 \\
0 & \lambda & -1 & \cdots & 0 \\
\vdots & \vdots & \vdots & & \vdots \\
0 & 0 & 0 & \cdots & -1 \\
0 & 0 & 0 & \cdots & \lambda
\end{array}\right|=\lambda^{n}
$$

$\boldsymbol{D}$ 的特征值为 $\lambda_{1}=\lambda_{2}=\cdots=\lambda_{n}=0$ ,它所对应的特征向量:$(1,0, \cdots, 0)^{\mathrm{T}}$ 。所以 $\mathscr{P}$的特征值为 $\lambda_{1}=\lambda_{2}=\cdots=\lambda_{n}=0$ ,它所对应的特征向量为

$$
\alpha=1 \cdot 1+0 \cdot x+\cdots+0 \cdot \frac{1}{(n-1)!} x^{n-1}=1 .
$$

它所对应的全体特征向量为 $k \boldsymbol{\alpha}$( $k$ 为所有非零常数).这与微积分学中所有常数的导数为零相一致。

由定理1.6.1知,线性变换 $\mathscr{A}$ 在不同基下对应的矩阵是相似的,而相似矩阵

有相同的特征多项式、特征值、行列式、秩、迹。于是可以将其称为线性变换 $A$ 的特征多项式、特征值、行列式、秩、迹。

矩阵 $\boldsymbol{A}$ 的特征值的全体称为 $\boldsymbol{A}$ 的谱记为 $\lambda(\boldsymbol{A})$ ,它也可称为线性变换 $\mathscr{A}$ 的谱。

\section*{二、特征值、特征向量的性质}
由上述分析可见,线性变换的特征值、特征向量的性质可由讨论矩阵特征值、特征向量的性质得到。\\
$n$ 阶方阵 $A$ 有 $n$ 个特征根(值),对于每一个特征值 $\lambda_{i}$ 代人式(1.8.4)可以求得相应的特征向量,这些特征向量加上零向量构成 $n$ 维向量空间的一个子空间,称为特征子空间,用 $V_{\lambda_{i}}$ 表示。

定义1.8.3 设 $\boldsymbol{A}$ 是 $n$ 阶方阵,它的 $r$ 个互不相同的特征值为 $\lambda_{1}, \lambda_{2}, \cdots, \lambda_{r}$ ,对应的重根数分别为 $p_{1}, p_{2}, \cdots, p_{r}$ ,则称 $p_{i}$ 为 $\lambda_{i}$ 的代数重复度。特征子空间 $V_{\lambda_{i}}$ 的维数 $q_{i}$ 称为 $\lambda_{i}$ 的几何重复度。

关于 $\boldsymbol{A}$ 的特征向量有下面两个结论。\\
定理1.8.3 设 $\lambda_{1}, \lambda_{2}, \cdots \lambda_{r}$ 是 $\boldsymbol{A}$ 的 $r$ 个互不相同的特征值, $\boldsymbol{\alpha}_{i}$ 是对应于 $\lambda_{i}$的特征向量 $(i=1,2, \cdots, r)$ ,则 $\boldsymbol{\alpha}_{1}, \boldsymbol{\alpha}_{2}, \cdots, \boldsymbol{\alpha}_{r}$ 线性无关。

定理1.8.4 设 $\lambda_{1}, \lambda_{2}, \cdots, \lambda_{r}$ 是 $\boldsymbol{A}$ 的 $r$ 个互不相同的特征值,$q_{i}$ 是 $\lambda_{i}$ 的几何重复度, $\boldsymbol{\alpha}_{i 1}, \boldsymbol{\alpha}_{i 2}, \cdots, \boldsymbol{\alpha}_{i q_{i}}$ 是对应于 $\lambda_{i}$ 的 $q_{i}$ 个线性无关的特征向量,则 $\boldsymbol{A}$ 的所有这些特征向量 $\boldsymbol{\alpha}_{11}, \boldsymbol{\alpha}_{12}, \cdots, \boldsymbol{\alpha}_{1 q_{1}} ; \boldsymbol{\alpha}_{21}, \boldsymbol{\alpha}_{22}, \cdots, \boldsymbol{\alpha}_{2 q_{2}}, \cdots ; \boldsymbol{\alpha}_{r 1}, \boldsymbol{\alpha}_{r 2}, \cdots, \boldsymbol{\alpha}_{r q_{r}}$ ,仍然线性无关。

这两个定理的证明可参阅线性代数的有关教材。\\
由定理1.8.4可见,$r$ 个特征子空间的和是直和,即

$$
V_{\lambda_{1}} \oplus V_{\lambda_{2}} \oplus \cdots \oplus V_{\lambda_{r}}
$$

$\boldsymbol{A}$ 的每一个特征值 $\lambda_{i}$ 的代数重复度 $p_{i}$ 与几何重复度 $q_{i}$ 之间有如下关系。\\
定理 1.8.5 矩阵 $\boldsymbol{A}$ 的任一特征值 $\lambda_{i}$ 的几何重复度 $q_{i}$ 不大于它的代数重复度 $p_{i}$ .

证明 设 $\lambda_{i}$ 为 $\boldsymbol{A}$ 的一个特征值,它的对应的线性无关的特征向量为 $\boldsymbol{\alpha}_{i 1}$ , $\boldsymbol{\alpha}_{i 2}, \cdots, \boldsymbol{\alpha}_{i q_{i}}$ ,取 $\boldsymbol{\beta}_{1}, \boldsymbol{\beta}_{2}, \cdots, \boldsymbol{\beta}_{n-q_{i}}$ ,使得

$$
\boldsymbol{\alpha}_{i 1}, \boldsymbol{\alpha}_{i 2}, \cdots, \boldsymbol{\alpha}_{i q_{i}}, \boldsymbol{\beta}_{1}, \boldsymbol{\beta}_{2}, \cdots, \boldsymbol{\beta}_{n-q_{i}}
$$

构成 $n$ 维向量空间的一组基,命

$$
\boldsymbol{P}=\left(\boldsymbol{\alpha}_{i 1}, \boldsymbol{\alpha}_{i 2}, \cdots, \boldsymbol{\alpha}_{i q_{i}}, \boldsymbol{\beta}_{1}, \boldsymbol{\beta}_{2}, \cdots, \boldsymbol{\beta}_{n-q_{i}}\right)
$$

故

$$
\boldsymbol{P}^{-1} \boldsymbol{P}=\left(\boldsymbol{P}^{-1} \boldsymbol{\alpha}_{i 1}, \boldsymbol{P}^{-1} \boldsymbol{\alpha}_{i 2}, \cdots, \boldsymbol{P}^{-1} \boldsymbol{\alpha}_{i q_{i}}, \boldsymbol{P}^{-1} \boldsymbol{\beta}_{1}, \boldsymbol{P}^{-1} \boldsymbol{\beta}_{2}, \cdots, \boldsymbol{P}^{-1} \boldsymbol{\beta}_{n-q_{i}}\right)=\boldsymbol{E} .
$$

因此

$$
\boldsymbol{P}^{-1} \boldsymbol{\alpha}_{i 1}=(1,0, \cdots, 0)^{\mathrm{T}}, \boldsymbol{P}^{-1} \boldsymbol{\alpha}_{i 2}=(0,1,0, \cdots, 0)^{\mathrm{T}}, \cdots
$$

$$
\boldsymbol{P}^{-1} \boldsymbol{\beta}_{n-q_{i}}=(0, \cdots, 0,1)^{\mathrm{T}}
$$

于是

$$
\begin{aligned}
\boldsymbol{P}^{-1} \boldsymbol{A} \boldsymbol{P} & =\boldsymbol{P}^{-1} \boldsymbol{A}\left(\boldsymbol{\alpha}_{i 1}, \boldsymbol{\alpha}_{i 2}, \cdots, \boldsymbol{\alpha}_{i q_{i}}, \boldsymbol{\beta}_{1}, \boldsymbol{\beta}_{2}, \cdots, \boldsymbol{\beta}_{n-q_{i}}\right) \\
& =\boldsymbol{P}^{-1}\left(\boldsymbol{A} \boldsymbol{\alpha}_{i 1}, \boldsymbol{A} \boldsymbol{\alpha}_{i 2}, \cdots, \boldsymbol{A} \boldsymbol{\alpha}_{i q_{i}}, \boldsymbol{A} \boldsymbol{\beta}_{1}, \boldsymbol{A} \boldsymbol{\beta}_{2}, \cdots, \boldsymbol{A} \boldsymbol{\beta}_{n-q_{i}}\right) \\
& =\boldsymbol{P}^{-1}\left(\lambda_{i} \boldsymbol{\alpha}_{i 1}, \lambda_{i} \boldsymbol{\alpha}_{i 2}, \cdots, \lambda_{i} \boldsymbol{\alpha}_{i q_{i}}, \boldsymbol{A} \boldsymbol{\beta}_{1}, \boldsymbol{A} \boldsymbol{\beta}_{2}, \cdots, \boldsymbol{A} \boldsymbol{\beta}_{n-q_{i}}\right) \\
& =\left(\lambda_{i} \boldsymbol{P}^{-1} \boldsymbol{\alpha}_{i 1}, \lambda_{i} \boldsymbol{P}^{-1} \boldsymbol{\alpha}_{i 2}, \cdots, \lambda_{i} \boldsymbol{P}^{-1} \boldsymbol{\alpha}_{i q_{i}}, \boldsymbol{P}^{-1} \boldsymbol{A} \boldsymbol{\beta}_{1}, \boldsymbol{P}^{-1} \boldsymbol{A} \boldsymbol{\beta}_{2}, \cdots, \boldsymbol{P}^{-1} \boldsymbol{A} \boldsymbol{\beta}_{n-q_{i}}\right) \\
& =\left[\begin{array}{cc:c}
\lambda_{i} & & \\
& \ddots & * \\
\hdashline 0 & \lambda_{i} & \\
\hdashline & \boldsymbol{A}_{1}
\end{array}\right]
\end{aligned}
$$

式中 0 表示 $\left(n-q_{i}\right) \times q_{i}$ 零矩阵,$*$ 表示 $q_{i} \times\left(n-q_{i}\right)$ 矩阵,$A_{1}$ 表示 $\left(n-q_{i}\right)$ 阶矩阵。

因此

$$
\left|\lambda \boldsymbol{E}_{n}-\boldsymbol{A}\right|=\left|\lambda \boldsymbol{E}_{n}-\boldsymbol{P}^{-1} \boldsymbol{A} \boldsymbol{P}\right|=\left(\lambda-\lambda_{i}\right)^{q_{i}}\left|\lambda E_{n-q_{i}}-\boldsymbol{A}_{1}\right|
$$

这说明 $\lambda_{i}$ 的几何重复度(它所对应的特征向量个数)不大于 $\lambda_{i}$ 的代数重复度(特征多项式 $\left|\lambda E_{n}-A\right|$ 中因子 $\lambda-\lambda_{i}$ 的次数)。

\section*{§1.9 线性变换的不变子空间}
线性变换在不同基下的矩阵表示 $\boldsymbol{A}, \boldsymbol{B}, \boldsymbol{C} \cdots$ 是互不相同的,而这些矩阵表示之间都是彼此相似的。因此,如何选择恰当的基使得线性变换在该基下的矩阵表示尽可能简单(例如对角形矩阵、准对角形矩阵等)。为此需要讨论线性变换的不变子空间。

定义1.9.1 设 $\mathscr{A}$ 是线性空间 $V$ 的线性变换,$W$ 是 $V$ 的子空间,如果对于任意向量 $\boldsymbol{\alpha} \in W$ 都有 $\mathscr{A}(\boldsymbol{\alpha}) \in W$ ,则称 $W$ 是 $\mathscr{A}$ 的不变子空间。并且, $\mathscr{A}$ 可以看做子空间 $W$ 上的一个线性变换,称为 $\mathscr{A}$ 在 $W$ 上的限制,记做 $\mathscr{A} I_{W}$ ,而且

$$
\left.\mathscr{A}\right|_{W}(\alpha)=\mathscr{A}(\alpha), \forall \alpha \in W
$$

例 1.9.1 线性空间 $V$ 和零子空间都是 $V$ 的任何一个线性变换的不变子空间。

例1.9.2 线性变换 $\mathscr{A}$ 的核 $N(\mathscr{A})$ 与值域 $R(. \mathscr{A})$ 都是 $\mathscr{A}$ 的不变子空间。\\
例1.9.3 若 $W$ 是线性变换 $\mathscr{A}$ 的不变子空间,那么 $\mathscr{A}(W)=\{\mathscr{A}(\boldsymbol{\alpha}) \mid \boldsymbol{\alpha} \in W\}$ 也是 $\mathscr{A}$ 的不变子空间,且 $\mathscr{A}(W) \subseteq W$ .

例1.9.4 设 $0 \neq \alpha \in V$ ,那么 $\operatorname{span}\{\alpha\}$ 是线性变换 $\mathscr{A}$ 的不变子空间的充分必

要条件 $\boldsymbol{\alpha}$ 是 $\mathcal{A}$ 的特征向量。\\
例1.9.5 设 $n$ 次多项式 $f(x) \in F[x]$ ,且

$$
f(x)=a_{n} x^{n}+a_{n-1} x^{n-1}+\cdots+a_{1} x+a_{0}
$$

$\mathscr{A}$ 是 $V$ 的一个线性变换定义( $E$ 为恒等变换)

$$
f(\mathscr{A})=a_{n} \mathscr{A}^{n}+a_{n-1} \mathscr{A}^{n-1}+\cdots+a_{1} \mathscr{A}+a_{0} E
$$

容易验证 $f(\mathscr{A})$ 仍是 $V$ 的线性变换,那么 $R(f(\mathscr{A}))$ 与 $N(f(\mathscr{A}))$ 都是 $\mathscr{A}$ 的不变子空间,$f(\mathscr{A})$ 的特征子空间也是 $\mathscr{A}$ 的不变子空间。

定理1.9.1 线性变换 $\mathscr{A}$ 的特征子空间 $V_{\lambda}$ 是 $\mathscr{A}$ 的不变子空间。\\
定理1.9.2 设 $\mathscr{A}, \mathscr{B}$ 是 $V$ 的两个线性变换,而且 $\mathscr{A} \mathscr{B}=\mathscr{B} \mathscr{A}$ 。则(1) $\mathscr{A}$ 的值域 $R(\mathscr{A})$ 与核 $N(\mathscr{A})$ 都是 $\mathscr{B}$ 的不变子空间。\\
(2) $\mathscr{A}$ 的特征子空间是 $\mathscr{B}$ 的不变子空间。\\
证明(1)设 $\boldsymbol{\alpha} \in N(\mathscr{B})$ ,则 $\mathscr{B}(\boldsymbol{\alpha})=0$ .于是

$$
\mathscr{B}(\mathscr{B}(\boldsymbol{\alpha}))=\mathscr{B}(\mathscr{B}(\boldsymbol{\alpha}))=\mathscr{B}(0)=0
$$

这表明 $\mathscr{B}(\boldsymbol{\alpha}) \in N(\mathscr{A})$ ,因此,$N(\mathscr{A})$ 是 $\mathscr{B}$ 的不变子空间。\\
若 $\boldsymbol{\alpha} \in R(\mathscr{A})$ ,则存在 $\boldsymbol{\eta} \in V$ ,使得 $\mathcal{A}(\boldsymbol{\eta})=\boldsymbol{\alpha}$ .于是

$$
\mathscr{B}(\boldsymbol{\alpha})=\mathscr{B}(\mathscr{B}(\boldsymbol{\eta}))=\mathscr{B}(\mathscr{B}(\boldsymbol{\eta})) \in R(\mathscr{A})
$$

这表明 $R(\mathscr{A})$ 是 $\mathscr{B}$ 的不变子空间。\\
(2)记 $V_{\lambda}$ 表示线性变换 $\mathscr{A}$ 的特征值为 $\lambda$ 的特征子空间,若 $\boldsymbol{\alpha} \in V_{\lambda}$ ,则 $\mathscr{B}(\boldsymbol{\alpha})=\lambda \boldsymbol{\alpha}$ .于是

$$
\mathscr{B}(\mathscr{B}(\boldsymbol{\alpha}))=\mathscr{B}(\mathscr{B}(\boldsymbol{\alpha}))=\mathscr{B}(\lambda \boldsymbol{\alpha})=\lambda \mathscr{B}(\boldsymbol{\alpha}) .
$$

这表明 $\mathscr{B}(\boldsymbol{\alpha})$ 是 $\mathscr{A}$ 的特征值为 $\lambda$ 的特征向量,即 $\mathscr{B}(\boldsymbol{\alpha}) \in V_{\lambda}$ 。所以 $V_{\lambda}$ 是 $\mathscr{B}$ 的不变子空间。

\section*{不变子空间的简单性质}
(1)线性变换 $\mathscr{A}$ 的不变子空间的和与交仍然是 $\mathscr{A}$ 的不变子空间。\\
证明 设 $W_{1}, W_{2}, \cdots, W_{s}$ 是 $\mathscr{A}$ 的不变子空间,在和 $\sum_{i=1}^{s} W_{i}$ 中任取向量 $\sum_{i=1}^{s} \boldsymbol{a}_{i}$ , $\left(\boldsymbol{\alpha}_{i} \in W_{i}, i=1,2, \cdots, s\right)$ .于是

$$
\mathscr{A}\left(\sum_{i=1}^{s} \boldsymbol{\alpha}_{i}\right)=\sum_{i=1}^{s} \mathscr{A}\left(\boldsymbol{\alpha}_{i}\right)
$$

由于 $\forall i, \mathscr{A}\left(\boldsymbol{\alpha}_{i}\right) \in W_{i}$ .故 $\sum_{i=1}^{s} \mathscr{A}\left(\boldsymbol{\alpha}_{i}\right) \in \sum_{i=1}^{s} W_{i}$ ,这表明 $\sum_{i=1}^{s} W_{i}$ 是 $\mathscr{A}$ 的不变子空间.\\
不变子空间的交也是 $\mathscr{A}$ 的不变子空间,请读者证明。\\
(2)设 $W=\operatorname{span}\left\{\boldsymbol{\alpha}_{1}, \boldsymbol{\alpha}_{2}, \cdots, \boldsymbol{\alpha}_{s}\right\}$ ,则 $W$ 是 $\mathscr{A}$ 的不变子空间的充分必要条件是 $\mathscr{B}\left(\boldsymbol{\alpha}_{i}\right) \in W,(1 \leqslant i \leqslant s)$ 。

证明 必要性显然.现证充分性.\\
设 $\mathscr{A}\left(\boldsymbol{\alpha}_{i}\right) \in W \quad(i=1,2, \cdots, s)$ 对于任意 $\xi \in W$ ,都有

$$
\xi=k_{1} \alpha_{1}+k_{2} \alpha_{2}+\cdots+k_{s} \alpha_{s}
$$

于是

$$
\mathscr{B}(\xi)=k_{1} \mathscr{B}\left(\boldsymbol{\alpha}_{1}\right)+k_{2} \mathscr{B}\left(\boldsymbol{\alpha}_{2}\right)+\cdots+k_{s} \mathscr{B}\left(\boldsymbol{\alpha}_{3}\right) \in W
$$

因此,$W$ 是 $\mathscr{A}$ 的不变子空间。\\
(3)$V$ 的任何一个子空间都是数乘变换的不变子空间,由数乘变换的定义立刻可以证明。

下面讨论线性变换 $\mathscr{A}$ 的矩阵表示与 $\mathscr{A}$ 的不变子空间之间的关系。\\
(1)设 $W$ 是 $n$ 维线性空间 $V$ 的线性变换。 $A$ 的不变子空间。 $\alpha_{1}, \alpha_{2}, \cdots, \alpha_{r}$ 是 $W$ 的一组基,$\alpha_{1}, \alpha_{2}, \cdots, \alpha_{r}, \alpha_{r+1}, \cdots, \alpha_{n}$ 是 $V$ 的一组基,现求 $\mathscr{A}$ 在基 $\alpha_{1}, \alpha_{2}, \cdots, \alpha_{r}$ , $\boldsymbol{\alpha}_{r+1}, \cdots, \boldsymbol{\alpha}_{n}$ 下的矩阵表示.

由于 $\mathscr{A}(W) \subset W$ ,故 $\mathscr{A}\left(\boldsymbol{\alpha}_{1}\right), \mathscr{A}\left(\boldsymbol{\alpha}_{2}\right), \cdots, \mathscr{B}\left(\boldsymbol{\alpha}_{r}\right)$ 是 $\boldsymbol{\alpha}_{1}, \boldsymbol{\alpha}_{2}, \cdots, \boldsymbol{\alpha}_{r}$ 的线性组合,而 $\mathscr{A}\left(\boldsymbol{\alpha}_{r+1}\right), \mathscr{A}\left(\boldsymbol{\alpha}_{r+2}\right), \cdots, \mathscr{B}\left(\boldsymbol{\alpha}_{n}\right)$ 是 $\boldsymbol{\alpha}_{1}, \boldsymbol{\alpha}_{2}, \cdots, \boldsymbol{\alpha}_{r}, \cdots, \boldsymbol{\alpha}_{n}$ 的线性组合,即

$$
\begin{aligned}
& \mathscr{A}\left(\boldsymbol{\alpha}_{1}\right)=a_{11} \boldsymbol{\alpha}_{1}+a_{21} \boldsymbol{\alpha}_{2}+\cdots+a_{r 1} \boldsymbol{\alpha}_{r} \\
& \mathscr{A}\left(\boldsymbol{\alpha}_{2}\right)=a_{12} \boldsymbol{\alpha}_{1}+a_{22} \boldsymbol{\alpha}_{2}+\cdots+a_{r 2} \boldsymbol{\alpha}_{r} \\
& \cdots \cdots \\
& \mathscr{A}\left(\boldsymbol{\alpha}_{r}\right)=a_{1 r} \boldsymbol{\alpha}_{1}+a_{2 r} \boldsymbol{\alpha}_{2}+\cdots+a_{r r} \boldsymbol{\alpha}_{r} \\
& \mathscr{A}\left(\boldsymbol{\alpha}_{r+1}\right)=a_{1 r+1} \boldsymbol{\alpha}_{1}+a_{2 r+1} \boldsymbol{\alpha}_{2}+\cdots+a_{r r+1} \boldsymbol{\alpha}_{r}+ \\
& \quad a_{r+1 r+1} \boldsymbol{\alpha}_{r+1}+\cdots+a_{n r+1} \boldsymbol{\alpha}_{n} \\
& \cdots \cdots \\
& \mathscr{A}\left(\boldsymbol{\alpha}_{n}\right)=a_{1 n} \boldsymbol{\alpha}_{1}+a_{2 n} \boldsymbol{\alpha}_{2}+\cdots+a_{r n} \boldsymbol{\alpha}_{r}+a_{r+1 n} \boldsymbol{\alpha}_{r}+\cdots+a_{n n} \boldsymbol{\alpha}_{n}
\end{aligned}
$$

所以, $\mathscr{B}$ 在 $\boldsymbol{\alpha}_{1}, \boldsymbol{\alpha}_{2}, \cdots, \boldsymbol{\alpha}_{r}, \cdots, \boldsymbol{\alpha}_{n}$ 下的矩阵表示为


\begin{align*}
\boldsymbol{A} & =\left[\begin{array}{ccccccc}
a_{11} & a_{12} & \cdots & a_{1 r} & a_{1 r+1} & \cdots & a_{1 n} \\
a_{21} & a_{22} & \cdots & a_{2 r} & a_{2 r+1} & \cdots & a_{2 n} \\
\vdots & \vdots & & \vdots & \vdots & & \vdots \\
a_{r 1} & a_{r 2} & \cdots & a_{r r} & a_{r+1} & \cdots & a_{r n} \\
& & & & a_{(r+1) r+1} & \cdots & a_{r+1 n} \\
& 0 & & & \vdots & & \vdots \\
& & & & a_{n r+1} & \cdots & a_{n n}
\end{array}\right] \\
& =\left[\begin{array}{cc}
A_{1} & A_{2} \\
0 & A_{3}
\end{array}\right] \tag{1.9.1}
\end{align*}


反之,若 $\mathscr{A}$ 在 $\boldsymbol{\alpha}_{1}, \boldsymbol{\alpha}_{2}, \cdots, \boldsymbol{\alpha}_{r}, \boldsymbol{\alpha}_{r+1}, \cdots, \boldsymbol{\alpha}_{n}$ 下的矩阵表示为式(1.9.1)。容易验证,由 $\alpha_{1}, \cdots, \alpha_{r}$ 生成的子空间是 $\mathscr{B}$ 的不变子空间。

我们已讨论了线性变换的矩阵表示是准三角形和不变子空间的关系。\\
(2)现在讨论 $\mathscr{A}$ 的矩阵表示是准对角形和不变子空间的关系,这就是下

述定理。\\
定理1.9.3 设 $\mathscr{A}$ 是线性空间 $V$ 的线性变换,则 $V$ 可以分解为 $\mathscr{A}$ 的不变子空间的直和

$$
V=W_{1} \oplus W_{2} \oplus \cdots \oplus W_{3}
$$

的充分必要条件是 $\mathscr{A}$ 在某组基下的矩阵是准对角矩阵

$$
\operatorname{diag}\left\{A_{1}, A_{2}, \cdots, A_{s}\right\}
$$

其中 $A_{i}$ 为 $\left.A\right|_{w_{i}}$ 在相应基下对应的矩阵。\\
证明 必要性 设

$$
V=W_{1} \oplus W_{2} \oplus \cdots \oplus W_{s}
$$

令 $W_{i}=\operatorname{span}\left\{\alpha_{i 1}, \alpha_{i 2}, \cdots, \alpha_{i n_{i}}\right\}$ 。则

$$
\mathscr{A}\left(\alpha_{i 1}, \alpha_{i 2}, \cdots, \alpha_{i n_{i}}\right)=\left(\alpha_{i 1}, \alpha_{i 2}, \cdots, \alpha_{i n_{i}}\right) A_{i}
$$

$(i=1,2, \cdots, s)$ .其中 $A_{i}$ 为 $n_{i}$ 阶矩阵,$n_{1}+n_{2}+\cdots+n_{s}=n$ .所以

$$
\begin{aligned}
& \mathscr{A}\left(\boldsymbol{\alpha}_{11}, \cdots, \boldsymbol{\alpha}_{1 n_{1}}, \cdots, \boldsymbol{\alpha}_{s 1}, \cdots, \boldsymbol{\alpha}_{s n_{s}}\right) \\
= & \left(\boldsymbol{\alpha}_{11}, \cdots, \boldsymbol{\alpha}_{1 n_{1}}, \cdots, \boldsymbol{\alpha}_{s 1}, \cdots, \boldsymbol{\alpha}_{s_{n}}\right)\left[\begin{array}{llll}
\boldsymbol{A}_{1} & & & \\
& \boldsymbol{A}_{2} & & \\
& & \ddots & \\
& & & \boldsymbol{A}_{s}
\end{array}\right]
\end{aligned}
$$

充分性 设 $\mathscr{A}$ 在 $V$ 的某组基 $\alpha_{11}, \cdots, \alpha_{1 n_{1}}, \cdots, \alpha_{s 1}, \cdots, \alpha_{s n_{s}}\left(n_{1}+n_{2}+\cdots+n_{s}=n\right)$下的矩阵为

$$
\left[\begin{array}{llll}
A_{1} & & & \\
& A_{2} & & \\
& & \ddots & \\
& & & A_{s}
\end{array}\right]
$$

其中 $\boldsymbol{A}_{i}$ 是 $n_{i}$ 阶矩阵。设 $W_{i}=\operatorname{span}\left\{\boldsymbol{\alpha}_{i 1}, \boldsymbol{\alpha}_{i 2}, \cdots, \boldsymbol{\alpha}_{i n_{i}}\right\} \quad(i=1,2, \cdots, s)$ 。那么

$$
\mathscr{A}\left(\alpha_{i 1}, \alpha_{i 2}, \cdots, \alpha_{i n_{i}}\right)=\left(\alpha_{i 1}, \alpha_{i 2}, \cdots, \alpha_{i n_{i}}\right) A_{i}
$$

从而 $\mathcal{A}\left(\boldsymbol{\alpha}_{i j}\right) \in W_{i} \quad\left(j=1,2, \cdots, n_{i}\right)$ ,此即 $W_{i}$ 是 $\mathscr{A}$ 的不变子空间。且

$$
V=W_{1} \oplus W_{2} \oplus \cdots \oplus W_{s}
$$

下面继续介绍线性变换的不变子空间中一类特殊而且非常重要的子空间——线性变换的根子空间。

定理1.9.4 设线性空间 $V$ 的线性变换 $\mathscr{A}$ 的特征多项式为 $f(\lambda)$ ,它可以分解为一次因式的乘积

$$
f(\lambda)=\left(\lambda-\lambda_{1}\right)^{r_{1}}\left(\lambda-\lambda_{2}\right)^{r_{2}} \cdots\left(\lambda-\lambda_{s}\right)^{r_{s}}
$$

( $r_{1}, r_{2}, \cdots, r_{s}$ 为正整数)\\
则线性空间 $V$ 可以分解成不变子空间的直和

$$
V=R_{\lambda_{1}}(\mathscr{A}) \oplus R_{\lambda_{2}}(\mathscr{A}) \oplus \cdots \oplus R_{\lambda_{3}}(\mathscr{A})
$$

其中 $\mathscr{A}$ 的不变子空间

$$
R_{\lambda_{i}}(\mathscr{A})=\left\{\xi \mid\left(\mathscr{A}-\lambda_{i} E\right)^{r_{i}}(\xi)=0, \xi \in V\right\}
$$

称 $R_{\lambda_{i}}(\mathscr{A})$ 是 $\mathscr{A}$ 的属于 $\lambda_{i}$ 的根子空间。\\
定理1.9.5 设线性空间 $V$ 的线性变换 $\mathscr{A}$ 的最小多项式为

$$
\psi_{\lambda}=P_{1}(\lambda)^{d_{1}} P_{2}(\lambda)^{d_{2}} \cdots P_{s}(\lambda)^{d_{s}}
$$

其中 $P_{i}(\lambda)(i=1,2, \cdots, s)$ 是首项系数为 1 的且彼此互不相同的不可约多项式,那么\\
(1)$V$ 可分解成一些不变子空间的直和

$$
V=W_{1} \oplus W_{2} \oplus \cdots \oplus W_{s} ;
$$

(2) $\mathscr{A}_{i}=\left.\mathscr{A}\right|_{W_{i}}$ 的最小多项式为 $P_{i}(\lambda)^{d_{i}}, 1 \leqslant i \leqslant s$ ;\\
(3)如果线性变换 $\mathscr{B}$ 与 $\mathscr{A}$ 可交换,那么 $\boldsymbol{W}_{i}$ 也是 $\mathscr{B}$ 的不变子空间 $(1 \leqslant i \leqslant s)$ ,其中 $W_{i}=\left\{\xi \mid P_{i}(\mathscr{A})^{d_{i}}(\xi)=0, \xi \in V\right\}$ 。

\section*{§1.10 矩阵的相似对角形}
线性变换理论要研究的一个主要问题是:对于 $n$ 维线性空间 $V$ 上的线性变换 $\mathscr{A}$ ,是否存在 $V$ 的一个基使得 $\mathscr{A}$ 在这个基下的矩阵为对角矩阵。

定义1.10.1 数域 $F$ 上的 $n$ 维线性空间 $V$ 的线性变换 $\mathscr{A}$ 称为可对角化的,如果 $V$ 中存在一个基,使得 $\mathscr{A}$ 在这个基下的矩阵为对角矩阵。

定义1.10.2 若 $n$ 阶矩阵 $\boldsymbol{A}$ 与对角矩阵相似,则称 $\boldsymbol{A}$ 可对角化,也称 $\boldsymbol{A}$ 是单纯矩阵。

设 $\mathscr{A}$ 是 $n$ 维线性空间 $V$ 的线性变换, $\mathscr{A}$ 在基 $\boldsymbol{\alpha}_{1}, \boldsymbol{\alpha}_{2}, \cdots, \boldsymbol{\alpha}_{n}$ 下的矩阵表示为 $\boldsymbol{A}$ ,即

$$
\mathscr{B}\left(\alpha_{1}, \alpha_{2}, \cdots, \alpha_{n}\right)=\left(\alpha_{1}, \alpha_{2}, \cdots, \alpha_{n}\right) A
$$

不难证明:\\
定理1.10.1 线性变换 $\mathscr{A}$ 可对角的充分必要条件是 $A$ 可对角化。(证略)由此可见,我们只需研究矩阵的可对角化问题即可。

\section*{一、矩阵A可对角化条件}
定理1.10.2 $n$ 阶矩阵 $A$ 可对角化的充要条件 $A$ 有 $n$ 个线性无关的特征向量。证明 必要性:设满秩矩阵 $\boldsymbol{P}$ ,满足


\begin{equation*}
\boldsymbol{P}^{-1} \boldsymbol{A} \boldsymbol{P}=\operatorname{diag}\left(\lambda_{1}, \lambda_{2}, \cdots, \lambda_{n}\right) \tag{1.10.1}
\end{equation*}


把 $\boldsymbol{P}$ 按列向量进行分块


\begin{equation*}
\boldsymbol{P}=\left(\boldsymbol{\alpha}_{1}, \boldsymbol{\alpha}_{2}, \cdots, \boldsymbol{\alpha}_{n}\right) \tag{1.10.2}
\end{equation*}


将式(1.10.2)代人式(1.10.1)得

$$
\boldsymbol{A}\left(\boldsymbol{\alpha}_{1}, \cdots, \boldsymbol{\alpha}_{n}\right)=\left(\boldsymbol{\alpha}_{1}, \cdots, \boldsymbol{\alpha}_{n}\right) \operatorname{diag}\left(\lambda_{1}, \lambda_{1}, \cdots, \lambda_{n}\right)
$$

于是


\begin{equation*}
\boldsymbol{A} \boldsymbol{\alpha}_{i}=\lambda_{i} \boldsymbol{\alpha}_{i} \quad(i=1,2, \cdots, n) \tag{1.10.3}
\end{equation*}


因为 $\boldsymbol{P}$ 是满秩的,所以 $\boldsymbol{\alpha}_{1}, \boldsymbol{\alpha}_{2}, \cdots, \boldsymbol{\alpha}_{n}$ 是线性无关的。从而由式(1.10.1)知, $\boldsymbol{A}$ 有 $n$个线性无关的特征向量。

充分性:设 $\boldsymbol{A}$ 有 $n$ 个线性无关的特征向量 $\boldsymbol{\alpha}_{1}, \cdots, \boldsymbol{\alpha}_{n}$ ,即 $\boldsymbol{A} \boldsymbol{\alpha}_{i}=\lambda_{i} \boldsymbol{\alpha}_{i} \quad(i= 1,2, \cdots, n)$ 。命

$$
\boldsymbol{P}=\left(\boldsymbol{\alpha}_{1}, \boldsymbol{\alpha}_{2}, \cdots, \boldsymbol{\alpha}_{n}\right)
$$

显然 $\boldsymbol{P}$ 是满秩的.故

即

$$
\begin{aligned}
\boldsymbol{A} \boldsymbol{P}= & \boldsymbol{A}\left(\boldsymbol{\alpha}_{1}, \boldsymbol{\alpha}_{2}, \cdots, \boldsymbol{\alpha}_{n}\right) \\
= & \left(\boldsymbol{A} \boldsymbol{\alpha}_{1}, \boldsymbol{A} \boldsymbol{\alpha}_{2}, \cdots, \boldsymbol{A} \boldsymbol{\alpha}_{n}\right) \\
= & \left(\lambda_{1} \boldsymbol{\alpha}_{1}, \lambda_{2} \boldsymbol{\alpha}_{2}, \cdots, \lambda_{n} \boldsymbol{\alpha}_{n}\right) \\
= & \left(\boldsymbol{\alpha}_{1}, \boldsymbol{\alpha}_{2}, \cdots, \boldsymbol{\alpha}_{n}\right) \operatorname{diag}\left(\lambda_{1}, \lambda_{2}, \cdots, \lambda_{n}\right) \\
= & \boldsymbol{P} \operatorname{diag}\left(\lambda_{1}, \lambda_{2}, \cdots, \lambda_{n}\right) \\
& \boldsymbol{P}^{-1} \boldsymbol{A} \boldsymbol{P}=\operatorname{diag}\left(\lambda_{1}, \lambda_{2}, \cdots, \lambda_{n}\right)
\end{aligned}
$$

推论 设 $\boldsymbol{P}^{-1} \boldsymbol{A} \boldsymbol{P}=\operatorname{diag}\left(\lambda_{1}, \lambda_{2}, \cdots, \lambda_{n}\right)$ ,则 $\lambda_{1}, \lambda_{2}, \cdots, \lambda_{n}$ 是 $\boldsymbol{A}$ 的 $n$ 个特征值, $\boldsymbol{P}$ 的第 $i$ 个列向量是 $\boldsymbol{A}$ 的属于 $\lambda_{i}$ 的特征向量。

由定理1.10.2可见,并不是任何一个线性变换都存在一个基,使其在该基下的矩阵表示呈现对角形。若一个线性变换在某组基下的矩阵表示是对角形,便称这线性变换是可对角化变换。

例1.10.1 已知线性微分方程组

\[
\left\{\begin{array}{l}
\frac{\mathrm{d} x_{1}}{\mathrm{~d} t}=a_{11} x_{1}+a_{12} x_{2}+\cdots+a_{1 n} x_{n}  \tag{1}\\
\frac{\mathrm{~d} x_{2}}{\mathrm{~d} t}=a_{21} x_{1}+a_{22} x_{2}+\cdots+a_{2 n} x_{n} \\
\vdots \\
\frac{\mathrm{~d} x_{n}}{\mathrm{~d} t}=a_{n 1} x_{1}+a_{n 2} x_{2}+\cdots+a_{n n} x_{n}
\end{array}\right.
\]

令

$$
\boldsymbol{X}=\left(\begin{array}{c}
x_{1} \\
x_{2} \\
\vdots \\
x_{n}
\end{array}\right), \quad \frac{\mathrm{d} \boldsymbol{X}}{\mathrm{~d} t}=\left(\begin{array}{c}
\frac{\mathrm{d} x_{1}}{\mathrm{~d} t} \\
\vdots \\
\frac{\mathrm{~d} x_{n}}{\mathrm{~d} t}
\end{array}\right), \quad \boldsymbol{A}=\left[\begin{array}{cccc}
a_{11} & a_{12} & \cdots & a_{1 n} \\
a_{21} & a_{22} & \cdots & a_{2 n} \\
\vdots & \vdots & & \vdots \\
a_{n 1} & a_{n 2} & \cdots & a_{n n}
\end{array}\right]
$$

则方程组(1)的矩阵形式为


\begin{equation*}
\frac{\mathrm{d} \boldsymbol{X}}{\mathrm{~d} t}=\boldsymbol{A} \boldsymbol{X} \tag{2}
\end{equation*}


若 $A$ 可对角化,即存在 $P \in C_{n}^{n \times n}$ ,使得

$$
\boldsymbol{P}^{-1} \boldsymbol{A P}=\boldsymbol{\Lambda}=\operatorname{diag}\left(\lambda_{1}, \cdots, \lambda_{n}\right)
$$

命


\begin{equation*}
X=P Y \tag{3}
\end{equation*}


其中 $\boldsymbol{Y}=\left(\begin{array}{c}y_{1} \\ \vdots \\ y_{n}\end{array}\right)$ ,把式(3)代入式(2)得

$$
\frac{\mathrm{d}(\boldsymbol{P} \boldsymbol{Y})}{\mathrm{d} t}=\boldsymbol{A} \boldsymbol{P} \boldsymbol{Y},
$$

即

$$
\boldsymbol{P} \frac{\mathrm{d} \boldsymbol{Y}}{\mathrm{~d} t}=\boldsymbol{A} \boldsymbol{P} \boldsymbol{Y}
$$

以 $\boldsymbol{P}^{-1}$ 左乘上式两端得


\begin{equation*}
\frac{\mathrm{d} \boldsymbol{Y}}{\mathrm{~d} t}=\boldsymbol{P}^{-1} \boldsymbol{A P Y}=\boldsymbol{\Lambda} \boldsymbol{Y} \tag{4}
\end{equation*}


因此

$$
\left\{\begin{array}{l}
\frac{\mathrm{d} y_{1}}{\mathrm{~d} t}=\lambda_{1} y_{1} \\
\frac{\mathrm{~d} y_{2}}{\mathrm{~d} t}=\lambda_{2} y_{2} \\
\vdots \\
\frac{\mathrm{~d} y_{n}}{\mathrm{~d} t}=\lambda_{n} y_{n}
\end{array}\right.
$$

经过积分得

$$
y_{1}=c_{1} \mathrm{e}^{\lambda_{1} t}, \quad y_{2}=c_{2} \mathrm{e}^{\lambda_{2} t}, \cdots, y_{n}=c_{n} \mathrm{e}^{\lambda_{n} t}
$$

代人方程组(3)求得微分方程解 $x_{1}, x_{2}, \cdots, x_{n}$\\
定理1.10.3 矩阵 $\boldsymbol{A}$ 可对角化的充要条件是 $\boldsymbol{A}$ 的每一个特征值的几何重复度等于代数重复度。

证明 设 $n$ 阶矩阵的谱为 $\left\{\lambda_{1}, \lambda_{2}, \cdots, \lambda_{r}\right\} . \lambda_{i}$ 的代数重复度为 $p_{i}$ ,几何重复度为 $q_{i} \quad(i=1,2, \cdots, r)$ 。则

$$
p_{1}+p_{2}+\cdots+p_{r}=n
$$

由定理1.8.5知

$$
q_{1}+q_{2}+\cdots+q_{r} \leqslant p_{1}+p_{2}+\cdots+p_{r}=n
$$

由定理1.10.2知

$$
q_{1}+q_{2}+\cdots+q_{r}=n
$$

故得

$$
q_{1}=p_{1}, \quad q_{2}=p_{2}, \quad \cdots, \quad q_{r}=p_{r}
$$

推论 若矩阵 $\boldsymbol{A}$ 的特征根全是单根,则 $\boldsymbol{A}$ 可对角化。\\
定理1.10.4 设 $n$ 阶矩阵 $\boldsymbol{A}$ 的谱为 $\left\{\lambda_{1}, \lambda_{2}, \cdots, \lambda_{r}\right\}$ ,特征值 $\lambda_{i}$ 的代数重复度为 $p_{i} \quad(i=1,2, \cdots, r)$ ,则 $\boldsymbol{A}$ 与对角矩阵相似的充要条件是 $\lambda_{i}$ 的代数重复度 $p_{i}= n-\operatorname{rank}\left(\lambda_{i} E-A\right) \quad(i=1,2, \cdots, r)$ 。

证明 由定理1.10.3知 $\lambda_{i}$ 的代数重复度 $p_{i}$ 等于它的几何重复度 $q_{i}$ ,而 $\lambda_{i}$ 的几何重复度就是线性齐次方程组 $\left(\lambda_{i} \boldsymbol{E}-\boldsymbol{A}\right) x=0$ 的基础解系向量个数,即 $\lambda_{i}$ 的几何重复度等于 $n-\operatorname{rank}\left(\lambda_{i} E-A\right)$ 。

二、可交换情况 $\boldsymbol{A B}=\boldsymbol{B A}$\\
一般而言,若 $\boldsymbol{A}, \boldsymbol{B} \in C^{n \times n}$ ,未必能有


\begin{equation*}
A B=B A \tag{1.10.4}
\end{equation*}


若 $\boldsymbol{A B}=\boldsymbol{B A}$ ,便称 $\boldsymbol{A}$ 与 $\boldsymbol{B}$(乘法)可交换.\\
定理1.10.5 若 $\boldsymbol{A}$ 与 $\boldsymbol{B}$ 乘法可交换,则 $\boldsymbol{A}$ 的任何特征子空间都是 $\boldsymbol{B}$ 的不变子空间。

注:定理1.10.5是定理1.9.2的另一种说法。并且可知,$B$ 的任何特征子空间也是 $\boldsymbol{A}$ 的不变子空间。

定理1.10.6 若 $\boldsymbol{A}$ 与 $\boldsymbol{B}$ 乘法可交换,则 $\boldsymbol{A}$ 的任何特征子空间中都有 $\boldsymbol{B}$ 的特征向量。

证明 设 $V_{\lambda_{0}}$ 是 $\boldsymbol{A}$ 的特征值为 $\lambda_{0}$ 的特征子空间, $\boldsymbol{\alpha}_{1}, \boldsymbol{\alpha}_{2}, \cdots, \boldsymbol{\alpha}_{s}$ 是 $V_{\lambda_{0}}$ 的一个基,由定理1.10.5知 $V_{\lambda_{0}}$ 是 $\boldsymbol{B}$ 的不变子空间。所以


\begin{equation*}
\boldsymbol{B} \boldsymbol{\alpha}_{i}=c_{1 i} \boldsymbol{\alpha}_{1}+c_{2 i} \boldsymbol{\alpha}_{2}+\cdots+c_{s i} \boldsymbol{\alpha}_{s} \quad(i=1,2, \cdots, s) \tag{1.10.5}
\end{equation*}


命

\[
M=\left[\begin{array}{cccc}
c_{11} & c_{12} & \cdots & c_{1 s}  \tag{1.10.6}\\
c_{21} & c_{22} & \cdots & c_{2 s} \\
\vdots & \vdots & & \vdots \\
c_{s 1} & c_{s 2} & \cdots & c_{s s}
\end{array}\right]
\]

设 $X \in V_{\lambda_{0}}$ ,则有


\begin{equation*}
\boldsymbol{X}=l_{1} \boldsymbol{\alpha}_{1}+l_{2} \boldsymbol{\alpha}_{2}+\cdots+l_{s} \boldsymbol{\alpha}_{s} \tag{1.10.7}
\end{equation*}


欲使 $\boldsymbol{X}$ 是 $V_{\lambda 0}$ 的向量,只需 $\boldsymbol{B X}=\boldsymbol{\mu} \boldsymbol{X}$ .于是结合式(1.10.5)有

$$
\begin{aligned}
\boldsymbol{B X}= & l_{1} \boldsymbol{B} \boldsymbol{\alpha}_{1}+l_{2} \boldsymbol{B} \boldsymbol{\alpha}_{2}+\cdots+l_{s} \boldsymbol{B} \boldsymbol{\alpha}_{s} \\
= & \left(l_{1} c_{11}+l_{2} c_{12}+\cdots+l_{s} c_{1 s}\right) \boldsymbol{\alpha}_{1}+ \\
& \left(l_{1} c_{21}+l_{2} c_{22}+\cdots+l_{s} c_{2 s}\right) \boldsymbol{\alpha}_{2}+\cdots+ \\
& \left(l_{1} c_{s 1}+l_{2} c_{s 2}+\cdots+l_{s} c_{s s}\right) \boldsymbol{\alpha}_{s} \\
\mu \boldsymbol{X}= & \mu l_{1} \boldsymbol{\alpha}_{1}+\mu l_{2} \boldsymbol{\alpha}_{2}+\cdots+\mu l_{s} \boldsymbol{\alpha}_{s}
\end{aligned}
$$

把 $\boldsymbol{B X}$ 与 $\boldsymbol{\mu X}$ 的表达式代人


\begin{equation*}
\boldsymbol{B} \boldsymbol{X}=\mu \boldsymbol{X} \tag{1.10.8}
\end{equation*}


并根据 $\boldsymbol{\alpha}_{1}, \boldsymbol{\alpha}_{2}, \cdots, \boldsymbol{\alpha}_{s}$ 线性无关,得到 $l_{1}, l_{2}, \cdots, l_{s}$ 满足方程组

$$
\left\{\begin{array}{cccc}
l_{1}\left(c_{11}-\mu\right)+l_{2} c_{12}+\cdots+l_{s} c_{1 s} & =0 \\
l_{1} c_{21}+l_{2}\left(c_{22}-\mu\right)+\cdots+l_{s} c_{2 s} & =0 \\
\vdots & \vdots & \vdots & \vdots \\
l_{1} c_{s 1}+l_{2} c_{s 2}+\cdots+l_{s}\left(c_{s s}-\mu\right) & =0
\end{array}\right.
$$

此即 $\left(l_{1}, l_{2}, \cdots, l_{s}\right)^{\mathrm{T}}$ 是 $s$ 阶矩阵 $\boldsymbol{M}$ 的特征向量,它总是存在的。因此在 $V_{\lambda_{0}}$ 中至少存在一组数 $l_{1}, l_{2}, \cdots, l_{s}$ 使得 $X=l_{1} \alpha_{1}+l_{2} \alpha_{2}+\cdots+l_{s} \alpha_{s}$ 满足式 $(1.10 .8)$ ,即 $X$ 是 $B$的一个特征向量。

推论1.10.1 若 $\boldsymbol{A}$ 与 $\boldsymbol{B}$ 乘法可交换,则 $\boldsymbol{A}$ 与 $\boldsymbol{B}$ 必有公共的特征向量。\\
推论 1.10.2 若 $\boldsymbol{A}$ 与 $\boldsymbol{B}$ 乘法可交换,$\lambda_{1}, \lambda_{2}, \cdots, \lambda_{k}$ 是 $\boldsymbol{A}$ 的 $k$ 个相异特征值,则 $\boldsymbol{A}$ 与 $\boldsymbol{B}$ 至少有 $k$ 个线性无关的公共特征向量。

\section*{三、同时对角化}
引理1.10.1 设 $A \in C^{n \times n}, B \in C^{m \times m}$ ,且 $D=\left[\begin{array}{ll}A & 0 \\ 0 & B\end{array}\right]$ ,则 $D$ 可以对角化的充要条件是 $\boldsymbol{A}, \boldsymbol{B}$ 都可以对角化。

证明 充分性 若 $A, B$ 都可以对角化,存在 $S_{1} \in C_{n}^{n \times n}, S_{2} \in C_{m}^{m \times m}$ ,满足

$$
\begin{gathered}
S_{1}^{-1} A S_{1}=A_{1}=\text { 对角形 } \\
S_{2}^{-1} B S_{2}=A_{2}=\text { 对角形 } \\
S=\left[\begin{array}{cc}
S_{1} & 0 \\
0 & S_{2}
\end{array}\right]
\end{gathered}
$$

令

则

$$
\begin{aligned}
S^{-1} D S & =\left[\begin{array}{cc}
S_{1}^{-1} & 0 \\
0 & S_{2}^{-1}
\end{array}\right]\left[\begin{array}{cc}
A & 0 \\
0 & B
\end{array}\right]\left[\begin{array}{cc}
S_{1} & 0 \\
0 & S_{2}
\end{array}\right] \\
& =\left[\begin{array}{cc}
S_{1}^{-1} A S_{1} & 0 \\
0 & S_{2}^{-1} B S_{2}
\end{array}\right]=\left[\begin{array}{cc}
A_{1} & 0 \\
0 & A_{2}
\end{array}\right]=A=\text { 对角形 }
\end{aligned}
$$

必要性 若 $D$ 可以对角化,存在 $S \in C_{n+m}^{(n+m) \times(n+m)}$ ,满足

命

$$
\begin{gathered}
S^{-1} D S=A=\operatorname{diag}\left(\lambda_{1}, \lambda_{2}, \cdots, \lambda_{n}, \lambda_{n+1}, \cdots, \lambda_{n+m}\right) \\
S=\left(\alpha_{1}, \alpha_{2}, \cdots, \alpha_{n}, \alpha_{n+1}, \cdots, \alpha_{n+m}\right) .
\end{gathered}
$$

其中

$$
\alpha_{i}=\left[\begin{array}{c}
\xi_{i} \\
\eta_{i}
\end{array}\right] \in C^{n+m}, \quad \xi_{i} \in C^{n}, \quad \eta_{i} \in C^{m} \quad(i=1,2, \cdots, n+m)
$$

因为 $\boldsymbol{D S}=\operatorname{Sdiag}\left(\lambda_{1}, \lambda_{2}, \cdots, \lambda_{n}, \lambda_{n+1}, \cdots, \lambda_{n+m}\right)$ ,所以

$$
\begin{aligned}
& \boldsymbol{D}\left(\boldsymbol{\alpha}_{1}, \boldsymbol{\alpha}_{2}, \cdots, \boldsymbol{\alpha}_{n}, \cdots, \boldsymbol{\alpha}_{n+m}\right) \\
= & \left(\boldsymbol{\alpha}_{1}, \boldsymbol{\alpha}_{2}, \cdots, \boldsymbol{\alpha}_{n}, \cdots, \boldsymbol{\alpha}_{n+m}\right) \times\left[\begin{array}{llll}
\lambda_{1} & & & \\
& \lambda_{2} & & \\
& & \ddots & \\
& & \lambda_{n+m}
\end{array}\right] \\
= & \left(\lambda_{1} \boldsymbol{\alpha}_{1}, \lambda_{2} \boldsymbol{\alpha}_{2}, \cdots, \lambda_{n} \boldsymbol{\alpha}_{n}, \cdots, \lambda_{n+m} \boldsymbol{\alpha}_{n+m}\right)
\end{aligned}
$$

比较上式两端得

$$
\begin{gathered}
\boldsymbol{D} \boldsymbol{\alpha}_{i}=\lambda_{i} \boldsymbol{\alpha}_{i} \quad(i=1,2, \cdots, n+m) \\
{\left[\begin{array}{cc}
\boldsymbol{A} & \mathbf{0} \\
\mathbf{0} & \boldsymbol{B}
\end{array}\right]\left[\begin{array}{r}
\xi_{i} \\
\boldsymbol{\eta}_{i}
\end{array}\right]=\lambda_{i}\left[\begin{array}{r}
\xi_{i} \\
\boldsymbol{\eta}_{i}
\end{array}\right] \quad(i=1,2, \cdots, n+m)}
\end{gathered}
$$

即\\
比较上式两端得

$$
\boldsymbol{A} \boldsymbol{\xi}_{i}=\lambda_{i} \boldsymbol{\xi}_{i}, \boldsymbol{B} \boldsymbol{\eta}_{i}=\lambda_{i} \boldsymbol{\eta}_{i} \quad(i=1,2, \cdots, n+m)
$$

这说明 $\boldsymbol{\xi}_{i}$ 是 $\boldsymbol{A}$ 的特征向量, $\boldsymbol{\eta}_{i}$ 是 $\boldsymbol{B}$ 的特征向量。现在将要证明 $(n+m)$ 个 $\boldsymbol{\xi}_{i}$ 中仅有 $n$ 个是线性无关的,$(n+m)$ 个 $\boldsymbol{\eta}_{i}$ 中仅有 $m$ 个是线性无关的。

因为

$$
\boldsymbol{S}=\left[\begin{array}{cccccc}
\boldsymbol{\xi}_{1}, & \boldsymbol{\xi}_{2}, & \cdots, & \boldsymbol{\xi}_{n}, & \cdots, & \boldsymbol{\xi}_{n+m} \\
\boldsymbol{\eta}_{1}, & \boldsymbol{\eta}_{2}, & \cdots, & \boldsymbol{\eta}_{n}, & \cdots, & \boldsymbol{\eta}_{n+m}
\end{array}\right] \in C_{n+m}^{(n+m) \times(n+m)}
$$

所以 $S$ 的 $(n+m)$ 个行向量线性无关,于是矩阵 $\left(\boldsymbol{\xi}_{1}, \boldsymbol{\xi}_{2}, \cdots, \boldsymbol{\xi}_{n}, \cdots, \boldsymbol{\xi}_{n+m}\right) \in C^{n \times(n+m)}$ 的 $n$ 个行向量线性无关,$\left(\boldsymbol{\eta}_{1}, \boldsymbol{\eta}_{2}, \cdots, \boldsymbol{\eta}_{n}, \cdots, \boldsymbol{\eta}_{n+m}\right) \in C^{m \times(n+m)}$ 的 $m$ 个行向量线性无关。因此

$$
\begin{gathered}
\operatorname{rank}\left(\boldsymbol{\xi}_{1}, \boldsymbol{\xi}_{2}, \cdots, \boldsymbol{\xi}_{n}, \cdots, \boldsymbol{\xi}_{n+m}\right)=n \\
\operatorname{rank}\left(\boldsymbol{\eta}_{1}, \boldsymbol{\eta}_{2}, \cdots, \boldsymbol{\eta}_{n}, \cdots, \boldsymbol{\eta}_{n+m}\right)=m
\end{gathered}
$$

此即 $(n+m)$ 个 $\xi_{i}$ 中仅有 $n$ 个线性无关,$(n+m)$ 个 $\eta_{i}$ 中仅有 $m$ 个线性无关。所以 $A, B$ 均可对角化.

定理1.10.7 设 $A, B \in C^{n \times n}$ 都可以对角化,则 $A, B$ 同时对角化的充要条件是 $A B=B A$ .

证明 必要性:若存在 $P \in C_{n}^{n \times n}$ ,满足

$$
\begin{aligned}
& \boldsymbol{P}^{-1} \boldsymbol{A} \boldsymbol{P}=\operatorname{diag}\left(\lambda_{1}, \lambda_{2}, \cdots, \lambda_{n}\right) \\
& \boldsymbol{P}^{-1} \boldsymbol{B} \boldsymbol{P}=\operatorname{diag}\left(\mu_{1}, \mu_{2}, \cdots, \mu_{n}\right)
\end{aligned}
$$

则

$$
\begin{aligned}
\left(\boldsymbol{P}^{-1} \boldsymbol{A} \boldsymbol{P}\right)\left(\boldsymbol{P}^{-1} \boldsymbol{B} \boldsymbol{P}\right) & =\operatorname{diag}\left(\lambda_{1}, \lambda_{2}, \cdots, \lambda_{n}\right) \operatorname{diag}\left(\mu_{1}, \mu_{2}, \cdots, \mu_{n}\right) \\
& =\operatorname{diag}\left(\mu_{1}, \mu_{2}, \cdots, \mu_{n}\right) \operatorname{diag}\left(\lambda_{1}, \lambda_{2}, \cdots, \lambda_{n}\right) \\
& =\left(\boldsymbol{P}^{-1} \boldsymbol{B} \boldsymbol{P}\right)\left(\boldsymbol{P}^{-1} \boldsymbol{A} \boldsymbol{P}\right)
\end{aligned}
$$

此即

$$
A B=B A
$$

充分性:分两步论述。先假定 $\boldsymbol{A}$ 为对角形矩阵

$$
A=\left[\begin{array}{llll}
\lambda_{1} E_{1} & & & \\
& \lambda_{2} E_{2} & & \\
& & \ddots & \\
& & & \lambda_{h} E_{h}
\end{array}\right]
$$

其中 $\boldsymbol{E}_{i}$ 是单位矩阵,其阶数为 $\lambda_{i}$ .对 $\boldsymbol{B}$ 实施分块,其分法使之与 $\boldsymbol{A}$ 能相乘

$$
\boldsymbol{B}=\left[\begin{array}{cccc}
\boldsymbol{B}_{11} & \boldsymbol{B}_{12} & \cdots & \boldsymbol{B}_{1 h} \\
\boldsymbol{B}_{21} & \boldsymbol{B}_{22} & \cdots & \boldsymbol{B}_{2 h} \\
\vdots & \vdots & & \vdots \\
\boldsymbol{B}_{h 1} & \boldsymbol{B}_{h 2} & \cdots & \boldsymbol{B}_{h h}
\end{array}\right],
$$

其中 $\boldsymbol{B}_{i j}$ 的行数与 $\boldsymbol{E}_{i}$ 阶数相同,列数与 $\boldsymbol{E}_{j}$ 的阶数相同.由于 $\boldsymbol{A B}=\boldsymbol{B A}$ ,所以 $\boldsymbol{B}_{i j}=0 (i \neq j)$ ,即

$$
\boldsymbol{B}=\left[\begin{array}{llll}
\boldsymbol{B}_{11} & & & \\
& \boldsymbol{B}_{22} & & \\
& & \ddots & \\
& & & \boldsymbol{B}_{h h}
\end{array}\right]
$$

其中 $\boldsymbol{B}_{i i}$ 均为方阵,由引理1.10.1知, $\boldsymbol{B}_{11}, \boldsymbol{B}_{22}, \cdots, \boldsymbol{B}_{h h}$ 都是可对角化矩阵。即存在满秩方阵 $\boldsymbol{T}_{i}$ ,使得 $\boldsymbol{T}_{i}^{-1} \boldsymbol{B}_{i i} \boldsymbol{T}_{i}$ 是对角形矩阵 $(i=1,2, \cdots, h)$ 。命

$$
\boldsymbol{T}=\left[\begin{array}{llll}
\boldsymbol{T}_{1} & & & \\
& \boldsymbol{T}_{2} & & \\
& & \ddots & \\
& & & \boldsymbol{T}_{h}
\end{array}\right]
$$

则 $\boldsymbol{T}^{-1} \boldsymbol{A} \boldsymbol{T}, \boldsymbol{T}^{-1} \boldsymbol{B} \boldsymbol{T}$ 均为对角形矩阵。\\
现设 $A$ 可以对角化,则存在 $S \in C_{n}^{n \times n}$ ,满足

$$
\boldsymbol{S}^{-1} \boldsymbol{A} \boldsymbol{S}=\left[\begin{array}{llll}
\lambda_{1} & & & \\
& \lambda_{2} & & \\
& & \ddots & \\
& & & \lambda_{n}
\end{array}\right]=\widetilde{\boldsymbol{A}}
$$

且 $\boldsymbol{S}^{-1} \boldsymbol{B S}=\widetilde{\boldsymbol{B}}$ 也是可以对角化矩阵。根据 $\boldsymbol{A B}=\boldsymbol{B A}$ ,可得 $\widetilde{\boldsymbol{A}} \widetilde{\boldsymbol{B}}=\widetilde{\boldsymbol{B}} \widetilde{\boldsymbol{A}}$ .由前面论述知,对于 $\widetilde{A}, \widetilde{B}$ 可以同时对角化,即存在 $T \in C_{n}^{n \times n}$ 满足

$$
T^{-1} \tilde{A} T=\Lambda_{1} \quad \text { 与 } \quad T^{-1} \tilde{B} T=\Lambda_{2}
$$

所以

$$
(S T)^{-1} A(S T)=\Lambda_{1} \quad \text { 与 } \quad(S T)^{-1} B(S T)=\Lambda_{2}
$$

此即 $A, B$ 可以同时对角化.

\section*{习 题}
1-1 试证:所有 $n$ 阶对称矩阵组成 $\frac{n(n+1)}{2}$ 维线性空间;所有的 $n$ 阶反对称矩阵组成 $\frac{n(n-1)}{2}$ 维线性空间.

1-2 在 $R^{4}$ 中,求向量 $\boldsymbol{\alpha}=(1,2,1,1)^{\mathrm{T}}$ 在基

$$
\begin{array}{ll}
\alpha_{1}=(1,1,1,1)^{\mathrm{T}}, & \alpha_{2}=(1,1,-1,-1)^{\mathrm{T}} \\
\alpha_{3}=(1,-1,1,-1)^{\mathrm{T}}, & \alpha_{4}=(1,-1,-1,1)^{\mathrm{T}}
\end{array}
$$

下的坐标。\\
1-3 在 $R^{2 \times 2}$ 中求矩阵

$$
A=\left[\begin{array}{ll}
1 & 2 \\
0 & 3
\end{array}\right]
$$

在基 $E_{1}=\left[\begin{array}{ll}1 & 1 \\ 1 & 1\end{array}\right], E_{2}=\left[\begin{array}{ll}1 & 1 \\ 1 & 0\end{array}\right], E_{3}=\left[\begin{array}{ll}1 & 1 \\ 0 & 0\end{array}\right], E_{4}=\left[\begin{array}{ll}1 & 0 \\ 0 & 0\end{array}\right]$ 下的坐标。\\
1-4 试证:在 $R^{2 \times 2}$ 中矩阵

$$
\alpha_{1}=\left[\begin{array}{ll}
1 & 1 \\
1 & 1
\end{array}\right], \alpha_{2}=\left[\begin{array}{ll}
1 & 1 \\
0 & 1
\end{array}\right], \alpha_{3}=\left[\begin{array}{ll}
1 & 1 \\
1 & 0
\end{array}\right], \alpha_{4}=\left[\begin{array}{ll}
1 & 0 \\
1 & 1
\end{array}\right]
$$

线性无关,并求 $\boldsymbol{\alpha}=\left[\begin{array}{ll}a & b \\ c & d\end{array}\right]$ 在基 $\boldsymbol{\alpha}_{1}, \boldsymbol{\alpha}_{2}, \boldsymbol{\alpha}_{3}, \boldsymbol{\alpha}_{4}$ 下的坐标。\\
1-5 设 $R[x]_{4}$ 是所有次数小于 4 的实系数多项式组成的线性空间,求多项式 $p(x)=1+2 x^{3}$ 在基 $1, x-1,(x-1)^{2},(x-1)^{3}$ 下的坐标。

1-6 已知 $R^{4}$ 中的两组基

$$
\begin{array}{ll}
\boldsymbol{\alpha}_{1}=(1,-1,0,0)^{\mathrm{T}}, & \boldsymbol{\alpha}_{2}=(0,1,-1,0)^{\mathrm{T}} \\
\boldsymbol{\alpha}_{3}=(0,0,1,-1)^{\mathrm{T}}, & \boldsymbol{\alpha}_{4}=(1,0,0,1)^{\mathrm{T}}
\end{array}
$$

与

$$
\begin{array}{ll}
\boldsymbol{\beta}_{1}=(2,1,-1,1)^{\mathrm{T}}, & \boldsymbol{\beta}_{2}=(0,3,1,0)^{\mathrm{T}} \\
\boldsymbol{\beta}_{3}=(5,3,2,1)^{\mathrm{T}}, & \boldsymbol{\beta}_{4}=(6,6,1,3)^{\mathrm{T}}
\end{array}
$$

求:由基 $\boldsymbol{\alpha}_{1}, \boldsymbol{\alpha}_{2}, \boldsymbol{\alpha}_{3}, \boldsymbol{\alpha}_{4}$ 到基 $\boldsymbol{\beta}_{1}, \boldsymbol{\beta}_{2}, \boldsymbol{\beta}_{3}, \boldsymbol{\beta}_{4}$ 的过渡矩阵,并求向量 $\xi=\left[x_{1}, x_{2}, x_{3}\right.$ , $\left.x_{4}\right]^{\mathrm{T}}$ 在基 $\boldsymbol{\beta}_{1}, \boldsymbol{\beta}_{2}, \boldsymbol{\beta}_{3}, \boldsymbol{\beta}_{4}$ 下的坐标。

1-7 已知

$$
\begin{array}{ll}
\boldsymbol{\alpha}_{1}=(1,2,1,0)^{\mathrm{T}}, & \boldsymbol{\alpha}_{2}=(-1,1,1,1)^{\mathrm{T}} \\
\boldsymbol{\beta}_{1}=(2,-1,0,1)^{\mathrm{T}}, & \boldsymbol{\beta}_{2}=(1,-1,3,7)^{\mathrm{T}}
\end{array}
$$

求 $\operatorname{span}\left\{\alpha_{1}, \alpha_{2}\right\}$ 与 $\operatorname{span}\left\{\beta_{1}, \beta_{2}\right\}$ 的和与交的基和维数。\\
1-8 已知 $V_{1}$ 是齐次线性方程组\\
(I)$\left\{\begin{array}{l}x_{1}-2 x_{2}-x_{3}-x_{4}=0 \\ 5 x_{1}-10 x_{2}-6 x_{3}-4 x_{4}=0\end{array}\right.$\\
的解空间,$V_{2}$ 是齐次线性方程组\\
( II )$x_{1}-x_{2}+x_{3}+2 x_{4}=0$\\
的解空间,试求:\\
(1)$V_{1}$ 与 $V_{2}$ 的基与维数\\
(2)$V_{1} \cap V_{2}$ 的基与维数\\
(3)$V_{1}+V_{2}$ 的基与维数\\
1-9 已知 $V_{1}$ 是齐次线性方程组

$$
\left\{\begin{array}{l}
x_{1}-x_{2}+5 x_{3}-x_{4}=0 \\
x_{1}+x_{2}-2 x_{3}+3 x_{4}=0
\end{array}\right.
$$

的解空间 $V_{2}=\operatorname{span}\left\{\alpha_{1}, \alpha_{2}, \alpha_{3}\right\}$ ,其中 $\alpha_{1}=(-3,-1,1,0)^{\mathrm{T}}, \alpha_{2}=(4,3,-1$ ,1)${ }^{\mathrm{T}}, \alpha_{3}=(-2,1,1,-1)^{\mathrm{T}}$ ,试求:\\
(1)$V_{1}$ 与 $V_{2}$ 的基与维数\\
(2)$V_{1} \cap V_{2}$ 的基与维数\\
(3)$V_{1}+V_{2}$ 的基与维数\\
1-10 已知 $V_{1}=\operatorname{span}\left\{\alpha_{1}, \alpha_{2}, \alpha_{3}\right\}, V_{2}=\operatorname{span}\left\{\beta_{1}, \beta_{2}\right\}$\\
试求:(1)$V_{1}$ 与 $V_{2}$ 的基与维数\\
(2)$V_{1} \cap V_{2}$ 的基与维数\\
(3)$V_{1}+V_{2}$ 的基与维数\\
其中 $\quad \alpha_{1}=(1,2,1,0)^{\mathrm{T}} \quad \alpha_{2}=(-1,1,0,1)^{\mathrm{T}} \quad \alpha_{3}=(1,5,2,1)^{\mathrm{T}}$

$$
\beta_{1}=(0,1,2,1)^{\mathrm{T}} \quad \beta_{2}=(0,5,10,5)^{\mathrm{T}}
$$

1-11 设 $\mathscr{A}$ 是 $n$ 维线性空间 $V$ 的一个线性变换,对某个 $\xi \in V$ 有 $\mathscr{A}^{k-1}(\xi) \neq 0, \mathscr{A}^{k}(\xi)=0$ ,试证:$\xi, \mathcal{B}(\xi), \mathscr{A}^{2}(\xi), \cdots, \mathscr{A}^{k-1}(\xi)$ 线性无关。

1-12 若 $n$ 维线性空间中线性变换 $\mathscr{A}$ 使得对 $V$ 中的任何向量 $\xi$ 都有 $\mathscr{A}^{n-1} (\xi) \neq 0, \mathscr{A}^{n}(\xi)=0$ ,求 $\mathscr{A}$ 在某一基下的矩阵表示。

1-13 设 $\boldsymbol{\beta}_{1}, \boldsymbol{\beta}_{2}, \cdots, \boldsymbol{\beta}_{m}$ 线性无关,且

$$
\begin{aligned}
\xi_{i} & =a_{1 i} \beta_{1}+a_{2 i} \beta_{2}+\cdots+a_{m i} \beta_{m} \\
& =\left(\beta_{1}, \beta_{2}, \cdots, \beta_{m}\right)\left[\begin{array}{c}
a_{1 i} \\
a_{2 i} \\
\vdots \\
a_{m i}
\end{array}\right] \quad(i=1,2, \cdots, s)
\end{aligned}
$$

试证:向量组 $\xi_{1}, \xi_{2}, \cdots, \xi_{s}$ 的秩 $=$ 矩阵 $\left(a_{i j}\right)_{m \times s}$ 的秩。\\
1-14 设 $\mathscr{B}$ 是线性空间 $R^{3}$ 的线性变换,它在 $R^{3}$ 中基 $\boldsymbol{\alpha}_{1}, \boldsymbol{\alpha}_{2}, \boldsymbol{\alpha}_{3}$ 下的矩阵表示为

$$
A=\left[\begin{array}{rrr}
1 & 2 & 3 \\
-1 & 0 & 3 \\
2 & 1 & 5
\end{array}\right]
$$

(1)求 $\mathscr{A}$ 在基 $\boldsymbol{\beta}_{1}=\boldsymbol{\alpha}_{1}, \boldsymbol{\beta}_{2}=\boldsymbol{\alpha}_{1}+\boldsymbol{\alpha}_{2}, \boldsymbol{\beta}_{3}=\boldsymbol{\alpha}_{1}+\boldsymbol{\alpha}_{2}+\boldsymbol{\alpha}_{3}$ 下的矩阵表示。\\
(2)求 $\mathscr{A}$ 在基 $\boldsymbol{\alpha}_{1}, \boldsymbol{\alpha}_{2}, \boldsymbol{\alpha}_{3}$ 下的核与值域。\\
1-15 设线性变换 $\mathscr{A}$ 在基 $\boldsymbol{\alpha}_{1}=(-1,1,1)^{\mathrm{T}}, \boldsymbol{\alpha}_{2}=(1,0,-1)^{\mathrm{T}}, \boldsymbol{\alpha}_{3}=(0,1$ , 1)${ }^{\mathrm{T}}$ 下的矩阵表示是

$$
A=\left[\begin{array}{rrr}
1 & 0 & -1 \\
1 & 1 & 0 \\
-1 & 2 & 3
\end{array}\right]
$$

(1)求 $\mathscr{A}$ 在基 $\varepsilon_{1}=(1,0,0)^{\mathrm{T}}, \varepsilon_{2}=(0,1,0)^{\mathrm{T}}, \varepsilon_{3}=(0,0,1)^{\mathrm{T}}$ 下的矩阵表示。\\
(2)求 $A$ 的核与值域。\\
1-16 求矩阵 $\boldsymbol{A}$ 的列空间 $R(\boldsymbol{A})$ 与核空间 $N(\boldsymbol{A})$ .\\
(1) $\boldsymbol{A}=\left[\begin{array}{lll}1 & 1 & 6 \\ 0 & 4 & 2 \\ 1 & 2 & 6\end{array}\right]$\\
(2) $\boldsymbol{A}=\left[\begin{array}{rrr}0 & 2 & -4 \\ -1 & -4 & 5 \\ 3 & 1 & 7 \\ 6 & 5 & -10\end{array}\right]$

1-17.试证: $\operatorname{tr}(\boldsymbol{A B})^{k}=\operatorname{tr}(\boldsymbol{B A})^{k}, k=1,2, \cdots$ .其中 $\operatorname{tr}(\boldsymbol{A})$ 表示矩阵 $\boldsymbol{A}$ 的迹.\\
1-18 试证: $\operatorname{tr}\left(\boldsymbol{A}^{k}\right)=\sum_{i=1}^{n} \lambda_{i}^{k}, \lambda_{i}$ 是 $\boldsymbol{A}$ 的特征值.\\
1-19 设 $A^{2}=E$ ,试证:$A$ 的特征值只能是 +1 或 -1 .\\
1-20 设 $A^{2}=A$ ,试证:$A$ 的特征值只能是 0 或1.\\
1-21 已知可逆矩阵 $\boldsymbol{A}$ 的特征值和特征向量,试求 $\boldsymbol{A}^{-1}$ 的特征值和特征向量.

1-22 已知 $A \sim B$ ,试证: $\operatorname{tr}(A)=\operatorname{tr}(B)$ .\\
1-23 求矩阵 $\boldsymbol{A}$ 的特征值与特征向量。\\
(1)$A=\left[\begin{array}{rrr}0 & 1 & 0 \\ -4 & 4 & 0 \\ -2 & 1 & 2\end{array}\right]$\\
(2)$A=\left[\begin{array}{lll}0 & 1 & 1 \\ 1 & 0 & 1 \\ 1 & 1 & 0\end{array}\right]$\\
(3)$A=\left[\begin{array}{lll}0 & 0 & 1 \\ 0 & 1 & 0 \\ 1 & 0 & 0\end{array}\right]$\\
(4)$A=\left[\begin{array}{rrr}0 & 2 & 1 \\ -2 & 0 & 3 \\ -1 & -3 & 0\end{array}\right]$

\section*{第二章}
\section*{$\boldsymbol{\lambda}$-矩阵与矩阵的 Jordan 标准形}
本章简单介绍 $\lambda$-矩阵的基本结论。矩阵在相似意义下的标准形中,Jordan 标准形是最重要的一种,它不仅在理论上有重要作用,而且在物理、力学、工程上有广泛应用。

这一章讨论 $\lambda$-矩阵和数字矩阵的相似标准形:内容包括 $\lambda$-矩阵及其标准形;初等因子及数字矩阵相似的条件;矩阵的 Jordan 标准形。

\section*{§ 2.1 入-矩阵及标准形}
\section*{一、 $\lambda$-矩阵的基本概念}
定义 2.1.1 设 $a_{i j}(\lambda) \quad(i=1,2, \cdots, m ; j=1,2, \cdots, n)$ 为数域 $F$ 上的多项式,则称以 $a_{i j}(\lambda)$ 为元素的 $m \times n$ 矩阵

$$
A(\lambda)=\left[\begin{array}{cccc}
a_{11}(\lambda) & a_{12}(\lambda) & \cdots & a_{1 n}(\lambda) \\
a_{21}(\lambda) & a_{22}(\lambda) & \cdots & a_{2 n}(\lambda) \\
\vdots & \vdots & & \vdots \\
a_{m 1}(\lambda) & a_{m 2}(\lambda) & \cdots & a_{m n}(\lambda)
\end{array}\right]
$$

为多项式矩阵或 $\lambda$-矩阵。称多项式 $a_{i j}(\lambda)$ 中 $(i=1,2, \cdots, m ; j=1,2, \cdots, n)$ 最高的次数为 $\boldsymbol{A}(\lambda)$ 的次数。

数字矩阵和特征矩阵 $\lambda \boldsymbol{E}-\boldsymbol{A}$ 都是 $\lambda$-矩阵的特例。\\
$\lambda$-矩阵的加法,数乘和乘法运算,矩阵的转置与数字矩阵相同,而且有相同的运算规律。\\
$\boldsymbol{n}$ 阶 $\lambda$-矩阵 $\boldsymbol{A}(\lambda)$ 的行列式:

$$
\operatorname{det} \boldsymbol{A}(\lambda)=|\boldsymbol{A}(\lambda)|=\sum_{j_{1} \cdots j_{n}} \sum_{\text {全排列 }}(-1)^{\tau\left(j_{1} \cdots j_{n}\right)} a_{1 j_{1}}(\lambda) a_{2 j_{2}}(\lambda) \cdots a_{n j_{n}}(\lambda)
$$

$\lambda$-矩阵行列式的性质与数字矩阵相同。一般情况下,$\lambda$-矩阵的行列式 $|\boldsymbol{A}(\lambda)|$ 是 $\lambda$的一个多项式。

定义2.1.2 如果 $\lambda$-矩阵 $\boldsymbol{A}(\lambda)$ 中有一个 $r(r \geqslant 1)$ 阶子式不为零,而所有 $r+1$ 阶子式(如果有的话)全为零,则称 $\boldsymbol{A}(\lambda)$ 的秩为 $r$ ,记为 $\operatorname{rank} \boldsymbol{A}(\lambda)=r$ 。零矩阵的秩为 0 。

由于 $\boldsymbol{A}(\lambda)$ 的行列式及一切子式都是 $\lambda$ 的多项式,所以定义2.1.2中的 $r+1$阶子式为零的含义是 $r+1$ 阶子式恒为零。

定义2.1.3 一个 $n$ 阶 $\lambda$-矩阵称为可逆的,如果有一个 $n$ 阶 $\lambda$-矩阵 $\boldsymbol{B}(\lambda)$ ,满足


\begin{equation*}
\boldsymbol{A}(\lambda) \boldsymbol{B}(\lambda)=\boldsymbol{B}(\lambda) \boldsymbol{A}(\lambda)=\boldsymbol{E} \tag{2.1.1}
\end{equation*}


这里 $\boldsymbol{E}$ 是 $n$ 阶单位矩阵。 $\boldsymbol{B}(\lambda)$ 称为 $\boldsymbol{A}(\lambda)$ 的逆矩阵,记为 $\boldsymbol{A}^{-1}(\lambda)$ 。\\
定理2.1.1 一个 $n$ 阶 $\lambda$-矩阵 $A(\lambda)$ 可逆的充要条件是 $\operatorname{det} A(\lambda)$ 是一个非零的常数。

证明 必要性:设 $\boldsymbol{A}(\lambda)$ 可逆,在式(2.1.1)的两边求行列式得


\begin{equation*}
|\boldsymbol{A}(\lambda)||\boldsymbol{B}(\lambda)|=1 \tag{2.1.2}
\end{equation*}


因为 $|\boldsymbol{A}(\lambda)|$ 与 $|\boldsymbol{B}(\lambda)|$ 都是 $\lambda$ 的多项式,所以根据式(2.1.2)推知,$|\boldsymbol{A}(\lambda)|$ 与 $\mid \boldsymbol{B} (\lambda)$|都是零次多项式,此即 $|\boldsymbol{A}(\lambda)|$ 是非零的常数。

充分性:设 $d=|A(\lambda)|$ 是一个非零的数。矩阵

$$
\frac{1}{d} \widetilde{A}(\lambda)
$$

是一个 $\lambda$ 一矩阵,其中 $\tilde{\boldsymbol{A}}(\lambda)$ 是 $\boldsymbol{A}(\lambda)$ 的伴随矩阵,所以

$$
A(\lambda) \frac{1}{d} \widetilde{A}(\lambda)=\frac{1}{d} A(\lambda) \widetilde{A}(\lambda)=E
$$

因此 $\boldsymbol{A}(\lambda)$ 可逆,且它的逆矩阵是 $\frac{1}{d} \widetilde{\boldsymbol{A}}(\lambda)$ .\\
例 2.1 .1 已知 $A(\lambda)=\left[\begin{array}{cc}\lambda+1 & \lambda \\ \lambda & \lambda-1\end{array}\right]$ ,求 $A^{-1}(\lambda)$ .\\
解 因为

故

$$
\begin{gathered}
|\boldsymbol{A}(\lambda)|=-1 \neq 0 \\
\widetilde{\boldsymbol{A}}(\lambda)=\left[\begin{array}{cc}
\lambda-1 & -\lambda \\
-\lambda & \lambda+1
\end{array}\right] \\
\boldsymbol{A}^{-1}(\lambda)=\left[\begin{array}{cc}
-\lambda+1 & \lambda \\
\lambda & -\lambda-1
\end{array}\right]
\end{gathered}
$$

注:$n$ 阶 $\lambda$-矩阵 $\boldsymbol{A}(\lambda)$ 的秩为 $n$ ,不等价于 $\boldsymbol{A}(\lambda)$ 可逆,这是与数字矩阵不相同之处。例如 $A(\lambda)=\left[\begin{array}{ll}\lambda & 1 \\ 1 & \lambda\end{array}\right]$ 的秩为 2 ,但是它不可逆。

定义2.1.4 下列各种类型的变换,叫做 $\lambda$-矩阵的初等变换:\\
(1)矩阵的任意二行(列)互换位置;\\
(2)非零常数 $c$ 乘矩阵的某一行(列);\\
(3)矩阵的某一行(列)的 $\varphi(\lambda)$ 倍加到另一行(列)上去,其中 $\varphi(\lambda)$ 是 $\lambda$ 的一个多项式。

对单位矩阵施行一次上述三种类型的初等变换,便得相应的三种 $\lambda$-矩阵的初

等矩阵 $\boldsymbol{P}(i, j), \boldsymbol{P}(i(c)), \boldsymbol{P}(i, j(\varphi))$ 即

$$
\begin{gathered}
\boldsymbol{P}(i, j)=\left[\begin{array}{lllllll}
1 & & & & & \\
& \ddots & & & & & \\
& & 0 & & 1 & & \\
& & & \ddots & & & \\
& & 1 & & 0 & & \\
& & & & \ddots & \\
& & & & & 1
\end{array}\right]-i \text { 行 } \\
\boldsymbol{P}(i(c))=\left[\begin{array}{lllll}
1 & & & & \\
& \ddots & & & \\
& & c & & \\
& & & \ddots & \\
& & & & 1
\end{array}\right]-i \text { 行 } \\
\boldsymbol{P}(i, j(\varphi))=\left[\begin{array}{cccccc}
1 & & & & & \\
& \ddots & & & & \\
& & 1 & & & \\
& & & \ddots & & \\
& & & & & \ddots \\
& & & & & \\
1 & & & 1
\end{array}\right]-i \text { 行 } \\
\begin{array}{c}
i \text { 列 } \\
\end{array}
\end{gathered}
$$

定理2.1.2 对一个 $m \times n \lambda$-矩阵 $\boldsymbol{A}(\lambda)$ 的行作初等行变换,相当于用相应的 $\boldsymbol{m}$ 阶初等矩阵左乘 $\boldsymbol{A}(\lambda)$ 。对 $\boldsymbol{A}(\lambda)$ 的列作初等列变换,相当于用相应的 $n$ 阶初等矩阵右乘 $\boldsymbol{A}(\lambda)$ 。

容易验证,初等矩阵都是可逆的,并且

$$
\begin{gathered}
\boldsymbol{P}(i, j)^{-1}=\boldsymbol{P}(i, j), \quad \boldsymbol{P}(i(c))^{-1}=\boldsymbol{P}\left(i\left(c^{-1}\right)\right) \\
\boldsymbol{P}(i, j(\varphi))^{-1}=\boldsymbol{P}(i, j(-\varphi))
\end{gathered}
$$

定义2.1.5 如果 $\boldsymbol{A}(\lambda)$ 经过有限次的初等变换后变成 $\boldsymbol{B}(\lambda)$ ,则称 $\boldsymbol{A}(\lambda)$ 与 $B(\lambda)$ 等价,记之为 $A(\lambda) \simeq B(\lambda)$ 。

定理 2.1.3 $\boldsymbol{A}(\lambda)$ 与 $\boldsymbol{B}(\lambda)$ 等价的充要条件是存在两个可逆矩阵 $\boldsymbol{P}(\lambda)$ 与 $\boldsymbol{Q}$ ( $\lambda$ ),使得

$$
\boldsymbol{B}(\lambda)=\boldsymbol{P}(\lambda) \boldsymbol{A}(\lambda) \boldsymbol{Q}(\lambda) .
$$

证明请读者完成。\\
$\lambda$-矩阵的等价关系满足:\\
(1)自反性:每一个 $\lambda$-矩阵与自己等价;\\
(2)对称性:若 $\boldsymbol{A}(\lambda) \simeq \boldsymbol{B}(\lambda)$ ,则 $\boldsymbol{B}(\lambda) \simeq \boldsymbol{A}(\lambda)$ ;\\
(3)传递性:若 $\boldsymbol{A}(\lambda) \simeq \boldsymbol{B}(\lambda), \boldsymbol{B}(\lambda) \simeq \boldsymbol{C}(\lambda)$ ,则 $\boldsymbol{A}(\lambda) \simeq \boldsymbol{C}(\lambda)$ 。

\section*{二、 $\lambda$-矩阵的 Smith 标准形}
这一节主要证明任何一个 $\lambda$-矩阵等价于对角矩阵。\\
引理2.1.1 设 $\lambda$-矩阵 $\boldsymbol{A}(\lambda)$ 的左上角元素 $a_{11}(\lambda) \neq 0$ ,并且 $\boldsymbol{A}(\lambda)$ 中至少有一个元素不能被它整除,那么一定可以找到一个与 $\boldsymbol{A}(\lambda)$ 等价的矩阵 $\boldsymbol{B}(\lambda)$ ,它的左上角元素也不为零,但是次数比 $a_{11}(\lambda)$ 的次数低。

证明 根据 $\boldsymbol{A}(\lambda)$ 中不能被 $a_{11}(\lambda)$ 整除的元素所在的位置,分三种情况讨论。\\
(1)若在 $\boldsymbol{A}(\lambda)$ 的第一列中有一个元素 $a_{i 1}(\lambda)$ 不能被 $a_{11}(\lambda)$ 整除,即有

$$
a_{i 1}(\lambda)=a_{11}(\lambda) g(\lambda)+r(\lambda),
$$

其中余式 $r(\lambda) \neq 0$ ,且次数比 $a_{11}(\lambda)$ 的次数低。\\
对 $\boldsymbol{A}(\lambda)$ 作两次初等行变换,首先第一行乘以 $g(\lambda)$ 加到第 $i$ 行,第 $i$ 行第一列的元素成为 $r(\lambda)$ ,然后把第一行和第 $i$ 行互换得到新的 $\lambda$ 一矩阵 $\boldsymbol{B}(\lambda), \boldsymbol{B}(\lambda)$ 左上角元素为 $r(\lambda)$ ,它为引理所需之矩阵。\\
(2)在 $\boldsymbol{A}(\lambda)$ 的第一行中有一个元素 $a_{1 i}(\lambda)$ 不能被 $a_{11}(\lambda)$ 整除,这种情况的证明与情况(1)类似。\\
(3) $\boldsymbol{A}(\lambda)$ 的第一行与第一列中的元素都可以被 $a_{11}(\lambda)$ 整除,但 $\boldsymbol{A}(\lambda)$ 中有一个元素 $a_{i j}(\lambda)(i>1, j>1)$ 不能被 $a_{11}(\lambda)$ 整除。我们设 $a_{i 1}(\lambda)=a_{11}(\lambda) \varphi(\lambda)$ ,对 $\boldsymbol{A}(\lambda)$ 作两次初等行变换,首先第一行乘以 $\varphi(\lambda)$ 加到第 $i$ 行,第 $i$ 行第一列的元素变为 0 ,第 $i$ 行第 $j$ 列的元素变为 $a_{i j}(\lambda)-a_{1 j}(\lambda) \varphi(\lambda)$ ;其次把第 $i$ 行的元素乘以 -1 加到第一行,第一行第一列的元素仍为 $a_{11}(\lambda)$ ,第一行第 $j$ 列的元素变为 $a_{i j}(\lambda)+[1-\varphi(\lambda)] a_{1 j}(\lambda)$ ,它不能被 $a_{11}(\lambda)$ 所整除,这就化为已经证明了的情况 2.

定理 2.1.4 任意一个非零的 $m \times n$ 阶 $\lambda$-矩阵 $\boldsymbol{A}(\lambda)$ 都等价于一个"对角形"矩阵,即

\[
A(\lambda) \simeq\left[\begin{array}{llllll}
d_{1}(\lambda) & & & & &  \tag{2.1.3}\\
& d_{2}(\lambda) & & & & \\
& & \ddots & & & \\
& & & d_{r}(\lambda) & & \\
& & & & 0 & \\
& & & & & \ddots \\
& & & & & 0
\end{array}\right]_{m \times n}
\]

其中 $r \geqslant 1, d_{i}(\lambda)$ 是首项系数为 1 的多项式,且

$$
d_{i}(\lambda) \mid d_{i+1}(\lambda) \quad(i=1,2, \cdots, r-1)
$$

记号 $d_{i}(\lambda) \mid d_{i+1}(\lambda)$ 表示 $d_{i+1}(\lambda)$ 能被 $d_{i}(\lambda)$ 整除。式(2.1.3)右端矩阵其余元素全为 0 。

证明 设 $a_{11}(\lambda) \neq 0$ ,否则,总可以经过行列调整,使得 $A(\lambda)$ 的左上角元素不为零。若 $a_{11}(\lambda)$ 不能整除 $\boldsymbol{A}(\lambda)$ 的所有元素,由引理2.1.1,可以找到与 $\boldsymbol{A}(\lambda)$ 等价的 $\boldsymbol{B}_{1}(\lambda)$ ,它的左上角元素 $b_{1}(\lambda) \neq 0$ ,并且次数比 $a_{11}(\lambda)$ 低。如果 $b_{1}(\lambda)$ 还不能整除 $\boldsymbol{B}_{1}(\lambda)$ 的所有元素,由引理2.1.1 又可以找到与 $\boldsymbol{B}_{1}(\lambda)$ 等价的 $\boldsymbol{B}_{2}(\lambda)$ ,它的左上角元素 $b_{2}(\lambda) \neq 0$ ,并且次数比 $b_{1}(\lambda)$ 低,如果 $b_{2}(\lambda)$ 还不能整除 $\boldsymbol{B}_{2}(\lambda)$ 的所有元素,继续上述步骤,得到一系列彼此等价的 $\lambda$-矩阵 $\boldsymbol{A}(\lambda), \boldsymbol{B}_{1}(\lambda), \cdots$ 它们的左上角元素皆不为零,而且次数越来越低。但是多项式的次数是非负整数,不可能无止境地降低,因此在有限步之后,就会得到一个 $\lambda$-矩阵 $\boldsymbol{B}(\lambda)$ ,它的左上角元素 $b_{s}(\lambda) \neq 0$ ,而且可以整除 $\boldsymbol{B}_{s}(\lambda)$ 的全部元素 $b_{i j}(\lambda)$ ,即

$$
b_{i j}(\lambda)=b_{s}(\lambda) q_{i j}(\lambda) .
$$

显然,可对 $\boldsymbol{B}_{s}(\lambda)$ 分别作一系列的初等行变换与初等列变换,使得第一行除左上角元素 $b_{s}(\lambda)$ 外全为零,第一列除左上角元素 $b_{s}(\lambda)$ 外全为零。即

$$
\boldsymbol{B}_{s}(\lambda) \simeq\left[\begin{array}{cccc}
b_{s}(\lambda) & 0 & \cdots & 0 \\
0 & & & \\
\vdots & & \boldsymbol{A}_{1}(\lambda) & \\
0 & & &
\end{array}\right]
$$

因为 $\boldsymbol{A}_{1}(\lambda)$ 的元素是 $\boldsymbol{B}_{s}(\lambda)$ 元素的组合,而 $\boldsymbol{B}_{s}(\lambda)$ 的 $b_{s}(\lambda)$ 可以整除 $\boldsymbol{B}_{s}(\lambda)$ 的所有元素,所以 $\boldsymbol{b}_{\mathbf{s}}(\lambda)$ 可以整除 $\boldsymbol{A}_{1}(\lambda)$ 的所有元素。如果 $\boldsymbol{A}_{1}(\lambda) \neq 0$ ,则对于 $\boldsymbol{A}_{1}(\lambda)$可以重复上述过程,进而把矩阵化成

$$
\left[\begin{array}{ccccc}
d_{1}(\lambda) & 0 & 0 & \cdots & 0 \\
0 & d_{2}(\lambda) & 0 & \cdots & 0 \\
0 & 0 & & & \\
\vdots & \vdots & & A_{2}(\lambda) & \\
0 & 0 & & &
\end{array}\right]
$$

其中 $d_{1}(\lambda)$ 与 $d_{2}(\lambda)$ 都是首项系数为 1 的多项式 $\left(d_{1}(\lambda)\right.$ 与 $b_{s}(\lambda)$ 只差一个常数倍数),而且 $d_{2}(\lambda)$ 能整除 $A_{2}(\lambda)$ 的所有元素。继续上述步骤,最后把 $A(\lambda)$ 化成所要求的形式。

定义2.1.6 与 $\boldsymbol{A}(\lambda)$ 等价的式(2.1.3)的右端矩阵称为 $\boldsymbol{A}(\boldsymbol{\lambda})$ 的 Smith 标准形.$d_{1}(\lambda), d_{2}(\lambda), \cdots d_{r}(\lambda)$ 称为 $A(\lambda)$ 的不变因子。

运用初等变换求 $\lambda$-矩阵的 Smith 标准形时,为了清楚表明所用的初等变换,用 $r_{i}$ 表示矩阵的第 $i$ 行,用 $c_{i}$ 表示矩阵的第 $i$ 列。且用记号 $r_{i} \longleftrightarrow r_{j}$ 表示互换矩阵的第 $i$ 行与第 $j$ 行;用记号 $c r_{i}$ 表示矩阵的第 $i$ 行乘常数 $c$ ;用 $\varphi(\lambda) r_{i}+r_{j}$ 表示矩阵的第 $i$ 行乘 $\varphi(\lambda)$ 加到第 $j$ 行上去。

不变因子 $d_{1}(\lambda), d_{2}(\lambda), \cdots, d_{r}(\lambda)$ 中前几个也可能是 1 。例如 $A(\lambda)$ 中的所有

元素没有公因子时,$d_{1}(\lambda) \equiv 1$ .在对 $A(\lambda)$ 用初等变换化成 Smith 标准形时,$A(\lambda)$的特点不外乎三种:(1) $\boldsymbol{A}(\lambda)$ 的元素中至少有一个非零常数(属 $\boldsymbol{A}(\lambda)$ 无公因子情况);(2)$A(\lambda)$ 的元素有公因子;(3)$A(\lambda)$ 的所有元素既无非零常数又无公因子。例 2.1.2、例 2.1.3、例 2.1.4 依次属于这三种。

例2.1.2 用初等变换把 $\lambda$-矩阵

$$
A(\lambda)=\left[\begin{array}{ccc}
\lambda & \lambda & 2 \\
\frac{1}{2} \lambda^{2}+2 & \lambda-\frac{1}{2} & \lambda \\
0 & \frac{1}{2} \lambda & \lambda+1
\end{array}\right]
$$

优成 Smith 标准形。\\
解 $\boldsymbol{A}(\lambda)$ 的元素中有非零常数 2

$$
\begin{aligned}
& \boldsymbol{A}(\lambda)=\left[\begin{array}{ccc}
\lambda & \lambda & 2 \\
\frac{1}{2} \lambda^{2}+2 & \lambda-\frac{1}{2} & \lambda \\
0 & \frac{1}{2} \lambda & \lambda+1
\end{array}\right] \stackrel{c_{1} \longleftrightarrow c_{3}\left[\begin{array}{ccc}
2 & \lambda & \lambda \\
\lambda & \lambda-\frac{1}{2} & \frac{1}{2} \lambda^{2}+2 \\
\lambda+1 & \frac{1}{2} \lambda & 0
\end{array}\right]}{=} \\
& \stackrel{-\frac{\lambda}{3} c_{1}+c_{3}}{-\frac{\lambda}{2} c_{1}+c_{2}}\left[\begin{array}{ccc}
2 & 0 & 0 \\
\lambda & -\frac{1}{2} \lambda^{2}+\lambda-\frac{1}{2} & 2 \\
\lambda+1 & -\frac{1}{2} \lambda^{2} & -\frac{1}{2} \lambda^{2}-\frac{1}{2} \lambda
\end{array}\right] \\
& -\frac{\lambda}{2} r_{1}+r_{2} \\
& \underset{2 c_{2}, 2 c_{3}}{\stackrel{\lambda+1}{2} r_{1}+r_{3}}\left[\begin{array}{ccc}
2 & 0 & 0 \\
0 & -\lambda^{2}+2 \lambda-1 & 4 \\
0 & -\lambda^{2} & -\lambda^{2}-\lambda
\end{array}\right] \\
& \underset{\frac{1}{2} c_{1}}{\stackrel{c_{2} \longleftrightarrow c_{3}}{\simeq}}\left[\begin{array}{ccc}
1 & 0 & 0 \\
0 & 4 & -\lambda^{2}+2 \lambda-1 \\
0 & -\lambda^{2}-\lambda & -\lambda^{2}
\end{array}\right] \\
& \stackrel{r_{2}}{4}\left(\lambda^{2}+\lambda\right)\left[\begin{array}{ccc}
1 & 0 & 0 \\
0 & 4 & -\lambda^{2}+2 \lambda-1 \\
0 & 0 & \frac{1}{4}\left(-\lambda^{4}+\lambda^{3}-3 \lambda^{2}-\lambda\right)
\end{array}\right] \\
& \underset{4 r^{3}}{\stackrel{c_{2}}{4}\left(\lambda^{2}-2 \lambda+1\right)}\left[\begin{array}{ccc}
1 & 0 & 0 \\
0 & 4 & 0 \\
0 & 0 & -\lambda^{4}+\lambda^{3}-3 \lambda^{2}-\lambda
\end{array}\right]
\end{aligned}
$$

$$
\begin{aligned}
& \frac{1}{4} c_{2} \\
& \simeq\left[\begin{array}{ccc}
1 & 0 & 0 \\
0 & 1 & 0 \\
-c_{3} & 0 & \lambda^{4}-\lambda^{3}+3 \lambda^{2}+\lambda
\end{array}\right]
\end{aligned}
$$

例2.1.3 用初等变换把 $\lambda$-矩阵

$$
A(\lambda)=\left[\begin{array}{cc}
\lambda^{3}-\lambda & 2 \lambda^{2} \\
\lambda^{2}+5 \lambda & 3 \lambda
\end{array}\right]
$$

化成 Smith 标准形。\\
解 $\boldsymbol{A}(\lambda)$ 的元素有公因子 $\lambda$ ,所以可以用初等变换把左上角元素变成 $\lambda$

$$
\begin{aligned}
\boldsymbol{A}(\lambda) & =\left[\begin{array}{cc}
\lambda^{3}-\lambda & 2 \lambda^{2} \\
\lambda^{2}+5 \lambda & 3 \lambda
\end{array}\right] c_{1} \leftrightarrows c_{2}\left[\begin{array}{cc}
2 \lambda^{2} & \lambda^{3}-\lambda \\
3 \lambda & \lambda^{2}+5 \lambda
\end{array}\right] \\
r_{1} & \Longleftrightarrow r_{2}\left[\begin{array}{cc}
3 \lambda & \lambda^{2}+5 \lambda \\
2 \lambda^{2} & \lambda^{3}-\lambda
\end{array}\right] \stackrel{\frac{1}{3} c_{1}}{\simeq}\left[\begin{array}{cc}
\lambda & \lambda^{2}+5 \lambda \\
\frac{2 \lambda^{2}}{3} & \lambda^{3}-\lambda
\end{array}\right]
\end{aligned}
$$

然后用初等变换把公因子 $\lambda$ 所在的行、列的其余元素均化为零。

$$
\begin{gathered}
\boldsymbol{A}(\lambda) \simeq\left[\begin{array}{cc}
\lambda & \lambda^{2}+5 \lambda \\
\frac{2 \lambda^{2}}{3} & \lambda^{3}-\lambda
\end{array}\right] \stackrel{\frac{2}{3} \lambda \times r_{1}+r_{2}}{\simeq}\left[\begin{array}{cc}
\lambda & \lambda^{2}+5 \lambda \\
0 & \frac{\lambda}{3}\left(\lambda^{2}-10 \lambda-3\right)
\end{array}\right] \\
\stackrel{-(\lambda+5) c_{1}+c_{2}}{\simeq}\left[\begin{array}{cc}
\lambda & 0 \\
0 & \lambda\left(\lambda^{2}-10 \lambda-3\right)
\end{array}\right]
\end{gathered}
$$

例2.1.4 用初等变换把 $\lambda$-矩阵

$$
\boldsymbol{A}(\lambda)=\left[\begin{array}{ccr}
1-\lambda & \lambda^{2} & \lambda \\
\lambda & \lambda & -\lambda \\
1+\lambda^{2} & \lambda^{2} & -\lambda^{2}
\end{array}\right]
$$

化成 Smith 标准形。\\
解 $\boldsymbol{A}(\lambda)$ 的元素无公因子,也无常数元素。用初等变换把矩阵中某一个元素变成常数

$$
\begin{gathered}
\boldsymbol{A}(\lambda)=\left[\begin{array}{ccc}
1-\lambda & \lambda^{2} & \lambda \\
\lambda & \lambda & -\lambda \\
1+\lambda^{2} & \lambda^{2} & -\lambda^{2}
\end{array}\right] \stackrel{r_{2}+r_{1}}{\simeq}\left[\begin{array}{ccc}
1 & \lambda^{2}+\lambda & 0 \\
\lambda & \lambda & -\lambda \\
1+\lambda^{2} & \lambda^{2} & -\lambda^{2}
\end{array}\right] \\
-\left(1+\lambda^{2}\right) r_{1}+r_{3}\left[\begin{array}{ccc}
1 & \lambda^{2}+\lambda & 0 \\
0 & -\lambda^{3}-\lambda^{2}+\lambda & -\lambda \\
0 & -\lambda^{4}-\lambda^{3}-\lambda & -\lambda^{2}
\end{array}\right] \\
\simeq\left(\lambda^{2}+\lambda\right) c_{1}+c_{2}\left[\begin{array}{ccc}
1 & 0 & 0 \\
0 & -\lambda^{3}-\lambda^{2}+\lambda & -\lambda \\
0 & -\lambda^{4}-\lambda^{3}-\lambda & -\lambda^{2}
\end{array}\right]
\end{gathered}
$$

剩下的右下角的二阶矩阵有公因子 $\lambda$ ,参照例2.1.3用的方法。有

$$
\begin{aligned}
\boldsymbol{A}(\lambda) & \simeq\left[\begin{array}{ccc}
1 & 0 & 0 \\
0 & -\lambda^{3}-\lambda^{2}+\lambda & -\lambda \\
0 & -\lambda^{4}-\lambda^{3}-\lambda & -\lambda^{2}
\end{array}\right] \\
c_{2} & \stackrel{\sim}{\simeq} c_{3}\left[\begin{array}{ccc}
1 & 0 & 0 \\
0 & -\lambda & -\lambda^{3}-\lambda^{2}+\lambda \\
0 & -\lambda^{2} & -\lambda^{4}-\lambda^{3}-\lambda
\end{array}\right] \\
& \stackrel{-\lambda r_{2}+r_{3}}{\simeq}\left[\begin{array}{ccc}
1 & 0 & 0 \\
0 & -\lambda & -\lambda^{3}-\lambda^{2}+\lambda \\
0 & 0 & -\lambda^{2}-\lambda
\end{array}\right] \\
& -\left(\lambda^{2}+\lambda-1\right) c_{2}+c_{3}\left[\begin{array}{ccc}
1 & 0 & 0 \\
0 & -\lambda & 0 \\
\simeq & 0 & -\lambda^{2}-\lambda
\end{array}\right] \\
& \stackrel{-1 \times r_{2}}{-1 \times r_{3}}\left[\begin{array}{ccc}
1 & 0 & 0 \\
0 & \lambda & 0 \\
0 & 0 & \lambda(\lambda+1)
\end{array}\right]
\end{aligned}
$$

例2.1.5 用初等变换化 $\lambda$-矩阵

$$
\boldsymbol{A}(\lambda)=\left[\begin{array}{ccc}
\lambda(\lambda+1) & 1 & \lambda^{2} \\
\lambda & \lambda+1 & \lambda-1 \\
\lambda^{3} & \lambda(\lambda-1) & \lambda^{2}
\end{array}\right]
$$

为 Smith 标准形。\\
解 $\boldsymbol{A}(\lambda)$ 的元素中有常数。

$$
\begin{aligned}
\boldsymbol{A}(\lambda) & =\left[\begin{array}{ccc}
\lambda(\lambda+1) & 1 & \lambda^{2} \\
\lambda & \lambda+1 & \lambda-1 \\
\lambda^{3} & \lambda(\lambda-1) & \lambda^{2}
\end{array}\right] \\
c_{1} & \simeq\left[\begin{array}{ccc}
1 & \lambda(\lambda+1) & \lambda^{2} \\
\lambda+1 & \lambda & \lambda-1 \\
\lambda(\lambda-1) & \lambda^{3} & \lambda^{2}
\end{array}\right] \\
& -(\lambda+1) r_{1}+r_{2}\left[\begin{array}{ccc}
1 & \lambda(\lambda+1) & \lambda^{2} \\
0 & -\lambda^{3}-2 \lambda^{2} & -\lambda^{3}-\lambda^{2}+\lambda-1 \\
0 & -\lambda^{4}+\lambda^{3}+\lambda^{2} & -\lambda^{4}+\lambda^{3}+\lambda^{2}
\end{array}\right] \\
& \stackrel{-\lambda(\lambda-1) r_{1}+r_{3}}{\simeq} \\
& \stackrel{-\lambda^{2} c_{1}+c_{3}}{\simeq}\left[\begin{array}{ccc}
1 & 0 & 0 \\
0 & -\lambda^{3}-2 \lambda^{2} & -\lambda^{3}-\lambda^{2}+\lambda-1 \\
0 & -\lambda^{4}+\lambda^{3}+\lambda^{2} & -\lambda^{4}+\lambda^{3}+\lambda^{2}
\end{array}\right]
\end{aligned}
$$

剩下的二阶矩阵是元素既无公因子又无常数的矩阵,参照例2.1.4的方法可把二阶矩阵用初等变换化某一个元素成常数。

$$
\begin{aligned}
\boldsymbol{A}(\lambda) & \simeq\left[\begin{array}{ccc}
1 & 0 & 0 \\
0 & -\lambda^{3}-2 \lambda^{2} & -\lambda^{3}-\lambda^{2}+\lambda-1 \\
0 & -\lambda^{4}+\lambda^{3}+\lambda^{2} & -\lambda^{4}+\lambda^{3}+\lambda^{2}
\end{array}\right] \\
& -c_{2}+c_{3}\left[\begin{array}{ccc}
1 & 0 & 0 \\
0 & -\lambda^{3}-2 \lambda^{2} & \lambda^{2}+\lambda-1 \\
0 & -\lambda^{4}+\lambda^{3}+\lambda^{2} & 0
\end{array}\right] \\
& +\lambda c_{3}+c_{2}\left[\begin{array}{ccc}
1 & 0 & 0 \\
0 & -\lambda^{2}-\lambda & \lambda^{2}+\lambda-1 \\
0 & -\lambda^{4}+\lambda^{3}+\lambda^{2} & 0
\end{array}\right] \\
& \stackrel{c_{2}+c_{3}}{\simeq}\left[\begin{array}{ccc}
1 & 0 & 0 \\
0 & -\lambda^{2}-\lambda & -1 \\
0 & -\lambda^{4}+\lambda^{3}+\lambda^{2} & -\lambda^{4}+\lambda^{3}+\lambda^{2}
\end{array}\right] \\
& \stackrel{c_{2}}{\simeq}\left[\begin{array}{ccc}
1 & 0 & 0 \\
0 & -1 & -\lambda^{2}-\lambda \\
0 & -\lambda^{4}+\lambda^{3}+\lambda^{2} & -\lambda^{4}+\lambda^{3}+\lambda^{2}
\end{array}\right] \\
& \left(-\lambda^{2}-\lambda\right) c_{2}+c_{3}\left[\begin{array}{ccc}
1 & 0 & 0 \\
0 & -1 & 0 \\
0 & -\lambda^{4}+\lambda^{3}+\lambda^{2} & \left(-\lambda^{2}-\lambda+1\right)\left(-\lambda^{4}+\lambda^{3}+\lambda^{2}\right)
\end{array}\right] \\
& \left(-\lambda^{4}+\lambda^{3}+\lambda^{2}\right) r_{2}+r_{3}\left[\begin{array}{ccc}
1 & 0 & 0 \\
0 & 1 & 0 \\
0 & 0 & \left(-\lambda^{2}-\lambda+1\right)\left(-\lambda^{4}+\lambda^{3}+\lambda^{2}\right)
\end{array}\right]
\end{aligned}
$$

例 2.1.6 用初等变换把 $\lambda$-矩阵

$$
A(\lambda)=\left[\begin{array}{lll}
\lambda(\lambda+1) & & \\
& \lambda & \\
& & (\lambda+1)^{2}
\end{array}\right]
$$

化为 Smith 标准形。\\
解 $\boldsymbol{A}(\lambda)$ 虽然是对角形,但不是 Smith 标准形。

$$
\begin{aligned}
A(\lambda)= & {\left[\begin{array}{lll}
\lambda(\lambda+1) & & \\
& \lambda & \\
& & (\lambda+1)^{2}
\end{array}\right] } \\
& \stackrel{c_{2}+c_{3}}{\simeq}\left[\begin{array}{llc}
\lambda(\lambda+1) & & \\
& \lambda & \lambda \\
& & (\lambda+1)^{2}
\end{array}\right]
\end{aligned}
$$

$$
\begin{aligned}
& -(\lambda+2) r_{2}+r_{3}\left[\begin{array}{ccc}
\lambda(\lambda+1) & 0 & 0 \\
0 & \lambda & \lambda \\
0 & -\lambda(\lambda+2) & 1
\end{array}\right] \\
& -\lambda r_{3}+r_{2}\left[\begin{array}{ccc}
\lambda(\lambda+1) & 0 & 0 \\
0 & \lambda^{3}+2 \lambda^{2}+\lambda & 0 \\
0 & -\lambda^{2}-2 \lambda & 1
\end{array}\right] \\
& \left(\lambda^{2}+2 \lambda\right) c_{3}+c_{2}\left[\begin{array}{ccc}
\lambda(\lambda+1) & 0 & 0 \\
0 & \lambda(\lambda+1)^{2} & 0 \\
0 & 0 & 1
\end{array}\right] \\
& \simeq\left[\begin{array}{cc}
1 & \lambda(\lambda+1) \\
& \lambda(\lambda+1)^{2}
\end{array}\right]
\end{aligned}
$$

\section*{三、Smith 标准形的唯一性}
为了研究标准形的唯一性,需要引进行列式因子。\\
定义2.1.7 设 $\lambda$-矩阵 $\boldsymbol{A}(\lambda)$ 的秩为 $r$ ,对于正整数 $k, 1 \leqslant k \leqslant r, \boldsymbol{A}(\lambda)$ 必有非零的 $k$ 阶子式, $\boldsymbol{A}(\lambda)$ 的全部 $k$ 阶子式的首项系数为 1 的最大公因式 $\boldsymbol{D}_{k}(\lambda)$ 称为 $\boldsymbol{A}(\lambda)$ 的 $k$ 阶行列式因子。

根据定义可知,对于秩为 $r$ 的 $\lambda$-矩阵 $\boldsymbol{A}(\lambda)$ ,行列式因子一共有 $r$ 个。\\
定理2.1.5 等价矩阵有相同的各阶行列式因子,从而有相同的秩。\\
证明 我们只要证明,$\lambda$-矩阵经过一次初等变换,秩与行列式因子是不变的。\\
设 $\lambda$-矩阵 $\boldsymbol{A}(\lambda)$ 经过一次初等行变换变成 $\boldsymbol{B}(\lambda), D_{k}(\lambda)$ 与 $\widetilde{D}_{k}(\lambda)$ 分别是 $\boldsymbol{A}(\lambda)$ 与 $\boldsymbol{B}(\lambda)$ 的 $k$ 阶行列式因子。我们来证明 $\boldsymbol{D}_{k}(\lambda)=\widetilde{\boldsymbol{D}}_{k}(\lambda)$ ,分三种情况讨论:\\
(1) $\boldsymbol{B}(\lambda)$ 是由 $\boldsymbol{A}(\lambda)$ 交换某两行而得到。这时 $\boldsymbol{B}(\lambda)$ 的每个 $k$ 阶子式或者等于 $\boldsymbol{A}(\lambda)$ 的某个 $k$ 阶子式,或者与 $\boldsymbol{A}(\lambda)$ 的某一个 $k$ 阶子式相差一个符号,因此 $\boldsymbol{D}_{k} (\lambda)$ 是 $\boldsymbol{B}(\lambda)$ 的 $k$ 阶子式的公因式,从而 $\boldsymbol{D}_{k}(\lambda) \mid \widetilde{\boldsymbol{D}}_{k}(\lambda)$ 。\\
(2) $\boldsymbol{B}(\lambda)$ 是由 $\boldsymbol{A}(\lambda)$ 的某一行乘以非零数 $c$ 而得。这时 $\boldsymbol{B}(\lambda)$ 的每个 $k$ 阶子式或者等于 $\boldsymbol{A}(\lambda)$ 的某个 $k$ 阶子式,或者等于 $\boldsymbol{A}(\lambda)$ 的某一个 $k$ 阶子式的 $c$ 倍。因此 $\boldsymbol{D}_{k}(\lambda)$ 是 $\boldsymbol{B}(\lambda)$ 的 $k$ 阶子式的公因式,从而 $\boldsymbol{D}_{k}(\lambda) \mid \widetilde{\boldsymbol{D}}_{k}(\lambda)$ 。\\
(3) $\boldsymbol{B}(\lambda)$ 是由 $\boldsymbol{A}(\lambda)$ 第 $j$ 行的 $\varphi(\lambda)$ 倍加到第 $i$ 行上得到,其中 $\varphi(\lambda)$ 为 $\lambda$ 的某个多项式。这时 $\boldsymbol{B}(\lambda)$ 中那些包含 $i$ 行与 $j$ 行的 $k$ 阶子式和那些不包含 $i$ 行的 $k$阶子式都等于 $\boldsymbol{A}(\lambda)$ 中对应的 $k$ 阶子式; $\boldsymbol{B}(\lambda)$ 中那些包含 $i$ 行但不包含 $j$ 行的 $k$ 阶子式,按 $i$ 行分成两部分之和,一部分恰等于 $\boldsymbol{A}(\lambda)$ 的一个 $k$ 阶子式;另一部分是 $\boldsymbol{A}(\lambda)$ 的另一个 $k$ 阶子式的 $\pm \varphi(\lambda)$ 倍,也就是 $\boldsymbol{A}(\lambda)$ 的两个 $k$ 阶子式的组合。因此\\
$\boldsymbol{D}_{k}(\lambda)$ 是 $\boldsymbol{B}(\lambda)$ 的 $k$ 阶子式的公因式,从而 $\boldsymbol{D}_{k}(\lambda) \mid \widetilde{\boldsymbol{D}}_{k}(\lambda)$ 。\\
对于列变换,可以完全一样地讨论。总之,如果 $\boldsymbol{A}(\lambda)$ 经过一次初等变换变成 $\boldsymbol{B}(\lambda)$ ,那么 $\boldsymbol{D}_{k}(\lambda) \mid \widetilde{\boldsymbol{D}}_{k}(\lambda)$ 。又由初等变换的可逆性, $\boldsymbol{B}(\lambda)$ 也可以经过一次初等变换变成 $\boldsymbol{A}(\lambda)$ 。由上面的讨论,同样应有 $\widetilde{\boldsymbol{D}}_{k}(\lambda) \mid \boldsymbol{D}_{k}(\lambda)$ ,于是 $\boldsymbol{D}_{k}(\lambda)=\widetilde{\boldsymbol{D}}_{k}(\lambda)$ 。

当 $\boldsymbol{A}(\lambda)$ 的全部 $k$ 阶子式为零时, $\boldsymbol{D}_{k}(\lambda)=0$ ,于是 $\widetilde{\boldsymbol{D}}_{k}(\lambda)=0, \boldsymbol{B}(\lambda)$ 的全部 $k$阶子式也就等于零;反之亦然。因此 $\boldsymbol{A}(\lambda)$ 与 $\boldsymbol{B}(\lambda)$ 既有相同的各阶行列式因子,又有相同的秩。

设 $\lambda$-矩阵的 Smith 标准形为

$$
\left[\begin{array}{ccccccc}
d_{1}(\lambda) & & & & & & \\
& d_{2}(\lambda) & & & & & \\
& & \ddots & & & & \\
& & & d_{r}(\lambda) & & & \\
& & & & 0 & & \\
& & & & & \ddots & \\
& & & & & & 0
\end{array}\right]
$$

其中 $d_{1}(\lambda), d_{2}(\lambda), \cdots, d_{r}(\lambda)$ 是首项系数为 1 的多项式,且 $d_{i}(\lambda) \mid d_{i+1}(\lambda) \quad(i= 1,2, \cdots, r-1)$ 。不难知道,$k$ 阶行列式因子为


\begin{equation*}
D_{k}(\lambda)=d_{1}(\lambda) d_{2}(\lambda) \cdots d_{k}(\lambda) \tag{2.1.4}
\end{equation*}


定理2.1.6 $\lambda$-矩阵的 Smith 标准形是唯一的。\\
证明 根据式(2.1.4)知, $\boldsymbol{A}(\boldsymbol{\lambda})$ 的不变因子为


\begin{gather*}
d_{1}(\lambda)=D_{1}(\lambda), \quad d_{2}(\lambda)=\frac{D_{2}(\lambda)}{D_{1}(\lambda)}, \cdots \\
d_{r}(\lambda)=\frac{D_{r}(\lambda)}{D_{r-1}(\lambda)} \tag{2.1.5}
\end{gather*}


这说明 $\boldsymbol{A}(\lambda)$ 的不变因子由 $\boldsymbol{A}(\lambda)$ 的行列式因子唯一确定,因此 $\boldsymbol{A}(\lambda)$ 的 Smith 标准形是唯一的。

应用 $\lambda$-矩阵的 Smith 标准形,容易证明下述几个定理(证明留给读者)。\\
定理 2.1.7 $\lambda$-矩阵 $\boldsymbol{A}(\lambda)$ 与 $\boldsymbol{B}(\lambda)$ 等价的充要条件是对于任何的 $k$ ,它们的 $k$阶行列式因子相同。

定理 $2.1 .8 \lambda$-矩阵 $\boldsymbol{A}(\lambda)$ 与 $\boldsymbol{B}(\lambda)$ 等价的充要条件是 $\boldsymbol{A}(\lambda)$ 与 $\boldsymbol{B}(\lambda)$ 有相同的不变因子。

根据定理 2.1.8,可以应用初等变换求 $\lambda$-矩阵的 Smith 标准形,也可以应用行列式因子求标准形。

定理2.1.7和定理2.1.8都已蕴涵了秩相等的条件。特别地,当 $n$ 阶 $\lambda$-矩阵 $\boldsymbol{A}(\lambda)$ 为满秩时,由初等变换的定义知, $\operatorname{det} \boldsymbol{A}(\lambda)=c d_{1}(\lambda) \cdots d_{n}(\lambda)$ ,其中 $c$ 为一个不等于零的常数。这表明每个不变因子 $d_{i}(\lambda)$ 是行列式 $\operatorname{det} \boldsymbol{A}(\lambda)$ 的因子,又不变

因子 $d_{i}(\lambda)$ 是由矩阵 $\boldsymbol{A}(\lambda)$ 唯一确定,故它们是 $\boldsymbol{A}(\lambda)$ 的不变量,这也正是称 $d_{i}(\lambda)$为不变因子的由来。

推论2.1.1 $\lambda$-矩阵 $\boldsymbol{A}(\lambda)$ 可逆的充要条件是 $\boldsymbol{A}(\lambda)$ 与单位矩阵等价。\\
证明 必要性 设 $A(\lambda)$ 为一个 $n$ 阶可逆矩阵,则由定理2.1.1知

$$
|A(\lambda)|=d \neq 0
$$

即 $\boldsymbol{A}(\lambda)$ 的 $n$ 阶行列式因子

$$
D_{n}(\lambda)=1
$$

由式(2.1.4)知,有关系

$$
D_{k}(\lambda) \mid D_{k+1}(\lambda) \quad(k=1,2, \cdots, n-1)
$$

故得

$$
D_{k}(\lambda)=1 \quad(k=1,2, \cdots, n)
$$

于是

$$
d_{k}(\lambda)=1 \quad(k=1,2, \cdots, n)
$$

这说明 $\boldsymbol{A}(\lambda)$ 的标准形为单位矩阵。\\
充分性 设

$$
A(\lambda) \simeq E_{n},
$$

所以 $\boldsymbol{A}(\lambda)$ 的行列式是一个非零的常数,由定理2.1.1知 $\boldsymbol{A}(\lambda)$ 可逆。\\
推论2.1.2 $\lambda$-矩阵 $\boldsymbol{A}(\lambda)$ 可逆的充要条件是 $\boldsymbol{A}(\lambda)$ 可以表示成一系列初等矩阵的乘积。

证明 由推论2.1.1知 $A(\lambda)$ 可逆的充要条件是

$$
\boldsymbol{A}(\lambda) \simeq \boldsymbol{E}_{n}
$$

而 $\boldsymbol{A}(\lambda)$ 与 $\boldsymbol{E}_{n}$ 等价的充要条件是有一系列初等矩阵 $\boldsymbol{P}_{1}, \boldsymbol{P}_{2}, \cdots, \boldsymbol{P}_{l}, \boldsymbol{Q}_{1}, \boldsymbol{Q}_{2}, \cdots, \boldsymbol{Q}_{m}$使得

$$
A(\lambda)=P_{1} P_{2} \cdots P_{l} E_{n} Q_{1} \cdots Q_{m}=P_{1} \cdots P_{l} Q_{1} \cdots Q_{m}
$$

应用行列式因子求不变因子一般都较复杂,但对于例2.1.4类型的矩阵就较简单。例2.1.4中矩阵的各阶行列式因子为:$D_{1}(\lambda)=1, D_{2}(\lambda)=\lambda(\lambda+1)$ , $D_{3}(\lambda)=\lambda^{2}(\lambda+1)^{3}$ ,故由式(2.1.5)可得 $d_{1}(\lambda)=1, d_{2}(\lambda)=\lambda(\lambda+1), d_{3}(\lambda)=\lambda (\lambda+1)^{2}$ .

\section*{§ 2.2 初等因子与相似条件}
\section*{一、初等因子}
定义2.2.1 $\lambda$-矩阵的行列式因子 $D_{k}(\lambda)$ 与不变因子 $d_{k}(\lambda)$ 都是 $\lambda$ 的多项式,它们都是由 $\boldsymbol{A}(\lambda)$ 的元素 $a_{i j}(\lambda)$ 经过"加、减、乘"而得到。在复数域 $\mathbf{C}$ 内,作为多项式的不变因子 $d_{k}(\lambda)$ 总可以分解为互不相同的一次因式方㔍的乘积,令

因为

$$
\begin{array}{cc}
d_{1}(\lambda)=\left(\lambda-\lambda_{1}\right)^{k_{11}}\left(\lambda-\lambda_{2}\right)^{k_{12}} \cdots\left(\lambda-\lambda_{t}\right)^{k_{1 t}} \\
d_{2}(\lambda)=\left(\lambda-\lambda_{1}\right)^{k_{21}}\left(\lambda-\lambda_{2}\right)^{k_{22}} \cdots\left(\lambda-\lambda_{t}\right)^{k_{2 t}} \\
\vdots & \vdots \\
d_{r}(\lambda)= & \left(\lambda-\lambda_{1}\right)^{k_{r 1}}\left(\lambda-\lambda_{2}\right)^{k_{r 2}} \cdots\left(\lambda-\lambda_{t}\right)^{k_{r t}}
\end{array}
$$

所以

$$
d_{k-1}(\lambda) \mid d_{k}(\lambda) \quad(k=2, \cdots, r)
$$

这里的 $\lambda_{1}, \lambda_{2}, \cdots, \lambda_{t}$ 是 $d_{r}(\lambda)$ 的全部相异零点,所以 $k_{r 1}, k_{r 2}, \cdots, k_{r t}$ 无一为零。但 $k_{1 j}, k_{2 j}, \cdots, k_{r-1, j}$ 中 $(j=1,2, \cdots, t)$ 可能出现零,而且若有 $k_{i j}=0(j=1,2, \cdots, t ; i=1$, $2, \cdots, r-1$ ),那么也必有 $k_{1 j}=k_{2 j}=\cdots=k_{i-1, j}=0$ .我们将

\[
\left\{\begin{array}{cccc}
\left(\lambda-\lambda_{1}\right)^{k_{11}}, & \left(\lambda-\lambda_{2}\right)^{k_{12}}, & \cdots, & \left(\lambda-\lambda_{t}\right)^{k_{1 t}}  \tag{2.2.1}\\
\left(\lambda-\lambda_{1}\right)^{k_{21}}, & \left(\lambda-\lambda_{2}\right)^{k_{22}}, & \cdots, & \left(\lambda-\lambda_{t}\right)^{k_{2 t}} \\
\vdots & \vdots & \vdots \\
\left(\lambda-\lambda_{1}\right)^{k_{r 1}}, & \left(\lambda-\lambda_{2}\right)^{k_{r 2}}, & \cdots, & \left(\lambda-\lambda_{t}\right)^{k_{n}}
\end{array}\right.
\]

中不是常数的因子全体叫做 $A(\lambda)$ 的初等因子。\\
例如,若 $\lambda$-矩阵的不变因子为

$$
1,1,(\lambda-2)^{5}(\lambda-3)^{3},(\lambda-2)^{5}(\lambda-3)^{4}(\lambda+2)
$$

则它的初等因子为

$$
(\lambda-2)^{5},(\lambda-3)^{3},(\lambda-2)^{5},(\lambda-3)^{4},(\lambda+2)
$$

若两个 $\lambda$-矩阵 $\boldsymbol{A}(\lambda)$ 与 $\boldsymbol{B}(\lambda)$ 等价,根据定理 2.1.8 它们有相同的不变因子,因此它们的初等因子也相同。两个 $\lambda$-矩阵的初等因子相同时,它们可能不等价,例如

$$
\begin{aligned}
& A(\lambda)=\left[\begin{array}{cccc}
1 & 0 & 0 & 0 \\
0 & \lambda-4 & 0 & 0 \\
0 & 0 & (\lambda-4)^{2} & 0
\end{array}\right] \\
& B(\lambda)=\left[\begin{array}{cccc}
\lambda-4 & 0 & 0 & 0 \\
0 & (\lambda-4)^{2} & 0 & 0 \\
0 & 0 & 0 & 0
\end{array}\right]
\end{aligned}
$$

都是 $\lambda$-矩阵,初等因子都是 $\lambda-4,(\lambda-4)^{2}$ ,但它们的秩不相同, $\boldsymbol{A}(\lambda)$ 与 $\boldsymbol{B}(\lambda)$ 不等价。

定理2.2.1 $n$ 阶 $\lambda$-矩阵 $\boldsymbol{A}(\lambda)$ 与 $\boldsymbol{B}(\lambda)$ 等价的充要条件是它们的秩相等和有相同的初等因子。

证明 必要性显然。现证充分性。设 $\boldsymbol{A}(\lambda)$ 与 $\boldsymbol{B}(\lambda)$ 的秩都为 $r$ ,并都有式 (2.2.1)的初等因子,其中 $k_{1 j} \leqslant k_{2 j} \leqslant \cdots \leqslant k_{r j}(j=1,2, \cdots, t)$ 。由初等因子定义知, $\boldsymbol{A}(\lambda)$ 与 $\boldsymbol{B}(\lambda)$ 的 $r$ 阶不变因子 $d_{r}(\lambda)$ 与 $\tilde{d}_{r}(\lambda)$ 相等,即

$$
d_{r}(\lambda)=\left(\lambda-\lambda_{1}\right)^{k_{r 1}}\left(\lambda-\lambda_{2}\right)^{k_{r 2}} \cdots\left(\lambda-\lambda_{t}\right)^{k_{r}}=\tilde{d}_{r}(\lambda)
$$

同样,对于任意的 $k(1 \leqslant k \leqslant r)$ 阶不变因子有

$$
d_{k}(\lambda)=\tilde{d}_{k}(\lambda)
$$

因此,根据定理 2.1.8 有 $\boldsymbol{A}(\lambda) \simeq \boldsymbol{B}(\lambda)$ 。\\
例 2.2.1 已知 $5 \times 6 \lambda$-矩阵 $\boldsymbol{A}(\lambda)$ 的秩为 4 ,其初等因子为

$$
\lambda, \lambda, \lambda^{2}, \lambda-1,(\lambda-1)^{2},(\lambda-1)^{3},(\lambda+2 i)^{3},(\lambda-2 i)^{3}
$$

试求 $\boldsymbol{A}(\lambda)$ 的 Smith 标准形。\\
解 首先容易求出 $\boldsymbol{A}(\lambda)$ 的不变因子

$$
\begin{aligned}
& d_{4}(\lambda)=\lambda^{2}(\lambda-1)^{3}(\lambda+2 \mathrm{i})^{3}(\lambda-2 \mathrm{i})^{3} \\
& d_{3}(\lambda)=\lambda(\lambda-1)^{2} \\
& d_{2}(\lambda)=\lambda(\lambda-1) \\
& d_{1}(\lambda)=1
\end{aligned}
$$

于是 $\boldsymbol{A}(\lambda)$ 的 Smith 标准形为

$$
\boldsymbol{A}(\lambda)=\left[\begin{array}{cccccc}
1 & 0 & 0 & 0 & 0 & 0 \\
0 & \lambda(\lambda-1) & 0 & 0 & 0 & 0 \\
0 & 0 & \lambda(\lambda-1)^{2} & 0 & 0 & 0 \\
0 & 0 & 0 & \lambda^{2}(\lambda-1)^{3}\left(\lambda^{2}+4\right)^{3} & 0 & 0 \\
0 & 0 & 0 & 0 & 0 & 0
\end{array}\right]
$$

对于准对角形矩阵

$$
A(\lambda)=\left[\begin{array}{cc}
B(\lambda) & 0 \\
0 & C(\lambda)
\end{array}\right]
$$

不能从 $\boldsymbol{B}(\lambda)$ 与 $\boldsymbol{C}(\lambda)$ 的不变因子求得 $\boldsymbol{A}(\lambda)$ 的不变因子,但是能从 $\boldsymbol{B}(\lambda)$ 与 $\boldsymbol{C}(\lambda)$的初等因子立即得到 $\boldsymbol{A}(\lambda)$ 的初等因子。此即

定理 2.2.2 设 $\lambda$-矩阵

$$
\boldsymbol{A}(\lambda)=\left[\begin{array}{cc}
\boldsymbol{B}(\lambda) & 0 \\
0 & \boldsymbol{C}(\lambda)
\end{array}\right]
$$

为准对角形矩阵,则 $\boldsymbol{B}(\lambda) 、 \boldsymbol{C}(\lambda)$ 的各个初等因子之全体是 $\boldsymbol{A}(\lambda)$ 的全部初等因子。

证明 先将 $\boldsymbol{B}(\lambda)$ 和 $\boldsymbol{C}(\lambda)$ 分别化成标准形

$$
\boldsymbol{B}(\lambda) \simeq\left[\begin{array}{lllllll}
d_{1}(\lambda) & & & & & & \\
& d_{2}(\lambda) & & & & & \\
& & \ddots & & & & \\
& & & d_{r_{1}}(\lambda) & & & \\
& & & & 0 & & \\
& & & & & \ddots & \\
& & & & & & 0
\end{array}\right]
$$

$$
C(\lambda) \simeq\left[\begin{array}{ccccccc}
\tilde{d}_{1}(\lambda) & & & & & & \\
& \tilde{d}_{2}(\lambda) & & & & & \\
& & \ddots & & & & \\
& & & \tilde{d}_{r_{2}}(\lambda) & & & \\
& & & & 0 & & \\
& & & & & \ddots & \\
& & & & & & 0
\end{array}\right]
$$

$\boldsymbol{A}(\lambda)$ 的秩 $r=r_{1}+r_{2}$ .把 $d_{i}(\lambda)$ 和 $\tilde{d}_{i}(\lambda)$ 分解为不同的一次因式的幂积,即

$$
\begin{aligned}
d_{i}(\lambda) & =\left(\lambda-\lambda_{1}\right)^{e_{i 1}}\left(\lambda-\lambda_{2}\right)^{e_{i 2}} \cdots\left(\lambda-\lambda_{t}\right)^{e_{i 1}} \\
\tilde{d}_{j}(\lambda) & =\left(\lambda-\lambda_{1}\right)^{h_{j 1}}\left(\lambda-\lambda_{2}\right)^{h_{j 2}} \cdots\left(\lambda-\lambda_{t}\right)^{h_{j i}} \\
& \left(i=1,2, \cdots, r_{1} ; j=1,2, \cdots, r_{2}\right)
\end{aligned}
$$

因此, $\boldsymbol{B}(\lambda)$ 和 $\boldsymbol{C}(\lambda)$ 的初等因子分别是

$$
\left(\lambda-\lambda_{1}\right)^{e_{i 1}},\left(\lambda-\lambda_{2}\right)^{e_{i 2}}, \cdots,\left(\lambda-\lambda_{t}\right)^{e_{i 1}}
$$

和

$$
\left(\lambda-\lambda_{1}\right)^{h_{\mathrm{fl}}},\left(\lambda-\lambda_{2}\right)^{h_{2}}, \cdots,\left(\lambda-\lambda_{t}\right)^{h_{j t}}
$$

中不为常数的幂。\\
现在证明 $\boldsymbol{B}(\lambda), \boldsymbol{C}(\lambda)$ 的初等因子就是 $\boldsymbol{A}(\lambda)$ 的全部初等因子。将 $\left(\lambda-\lambda_{1}\right)$的指数

$$
e_{11}, e_{21}, \cdots, e_{r_{1} 1}, h_{11}, h_{21}, \cdots, h_{r_{2} 1}
$$

按大小的顺序排列,设为

$$
0 \leqslant c_{1} \leqslant c_{2} \leqslant \cdots \leqslant c_{r}
$$

因为 $\boldsymbol{A}(\lambda)$ 是 $\boldsymbol{B}(\lambda)$ 与 $\boldsymbol{C}(\lambda)$ 所构成的准对角形矩阵,所以在 $\boldsymbol{B}(\lambda)$ 与 $\boldsymbol{C}(\lambda)$ 上施行初等变换,实际上是在 $\boldsymbol{A}(\lambda)$ 上施行初等变换,于是

$$
\boldsymbol{A}(\lambda) \simeq\left[\begin{array}{lllllll}
d_{1}(\lambda) & & & & & & \\
& \ddots & & & & & \\
& & d_{r_{1}}(\lambda) & & & & \\
& & & \tilde{d}_{1}(\lambda) & & & \\
& & & & \ddots & & \\
& & & & \tilde{d}_{r_{2}}(\lambda) & & \\
& & & & & & 0 \\
& & & & & & \ddots
\end{array}\right]
$$

$$
\simeq\left[\begin{array}{llllll}
\left(\lambda-\lambda_{1}\right)^{c_{1}} \varphi_{1}(\lambda) & & & & & \\
& \left(\lambda-\lambda_{1}\right)^{c_{2}} \varphi_{2}(\lambda) & & & & \\
& & \ddots & & & \\
& & & \left(\lambda-\lambda_{1}\right)^{c_{r}} \varphi_{r}(\lambda) & & \\
& & & & 0 & \\
& & & & & 0
\end{array}\right]
$$

式中 $r$ 个多项式 $\varphi_{1}(\lambda), \varphi_{2}(\lambda), \cdots, \varphi_{r}(\lambda)$ 都不含因式 $\left(\lambda-\lambda_{1}\right)$ 。设 $A(\lambda)$ 的行列式因子为

$$
D_{1}^{*}(\lambda), D_{2}^{*}(\lambda), \cdots, D_{r}^{*}(\lambda)
$$

所以在这些行列式因子中因式 $\left(\lambda-\lambda_{1}\right)$ 的最高幂指数分别等于

$$
c_{1}, \sum_{i=1}^{2} c_{i}, \cdots, \sum_{i=1}^{r-1} c_{i}, \sum_{i=1}^{r} c_{i}
$$

根据行列式因子与不变因子的关系式(2.1.5),则知含在不变因子 $d_{1}^{*}(\lambda), d_{2}^{*} (\lambda), \cdots, d_{r}^{*}(\lambda)$ 中因式 $\left(\lambda-\lambda_{1}\right)$ 的最高幂指数分别等于 $c_{1}, c_{2}, \cdots, c_{r}$ ,这就是说, $\boldsymbol{A}(\lambda)$ 中与 $\left(\lambda-\lambda_{1}\right)$ 相应的初等因子是

$$
\left(\lambda-\lambda_{1}\right)^{c_{1}},\left(\lambda-\lambda_{1}\right)^{c_{2}}, \cdots,\left(\lambda-\lambda_{1}\right)^{c_{r}}
$$

中不为零指数的幂 $\left(\lambda-\lambda_{1}\right)^{c_{j}}$(即 $c_{j} \neq 0$ ),因而就是 $\boldsymbol{B}(\lambda) 、 \boldsymbol{C}(\lambda)$ 中与 $\left(\lambda-\lambda_{1}\right)$ 相应的全部初等因子。同理,对 $\left(\lambda-\lambda_{2}\right),\left(\lambda-\lambda_{3}\right), \cdots,\left(\lambda-\lambda_{t}\right)$ 也可得相同结论。于是我们证明了 $\boldsymbol{B}(\lambda), \boldsymbol{C}(\lambda)$ 的全部初等因子都是 $\boldsymbol{A}(\lambda)$ 的初等因子。

剩下要证明,除此之外,$A(\lambda)$ 再没有别的初等因子。\\
设 $(\lambda-a)^{k}$ 是 $A(\lambda)$ 的一个初等因子,于是 $(\lambda-a)^{k}$ 一定是包含在某一个不变因子 $d_{i}^{*}(\lambda)$ 中 $(\lambda-a)$ 的最高次幂,因此,$(\lambda-a)^{k} \mid d_{r}^{*}(\lambda)$ ,故 $(\lambda-a)^{k} \mid D_{r}^{*} (\lambda)$ ,此即 $\lambda=a$ 是 $D_{r}^{*}(\lambda)$ 的一个零点,即 $D_{r}^{*}(a)=0$ 。另一方面,由于

$$
A(\lambda) \simeq\left[\begin{array}{lllllll}
d_{1}(\lambda) & & & & & & \\
& \ddots & & & & & \\
& & d_{r_{1}}(\lambda) & & & & \\
& & & \tilde{d}_{1}(\lambda) & & & \\
& & & & \ddots & & \\
& & & & \tilde{d}_{r_{2}}(\lambda) & & \\
& & & & & 0 & \\
& & & & & & \ddots \\
& & & & 0
\end{array}\right]
$$

故

$$
D_{r}^{*}(\lambda)=d_{1}(\lambda) d_{2}(\lambda) \cdots d_{r_{1}}(\lambda) \tilde{d}_{1}(\lambda) \tilde{d}_{2}(\lambda) \cdots \tilde{d}_{r_{2}}(\lambda)
$$

因为

$$
d_{i}(\lambda)\left|d_{r_{1}}(\lambda), \tilde{d}_{j}(\lambda)\right| \tilde{d}_{r_{2}}(\lambda) \quad\left(i=1,2, \cdots, r_{1} ; j=1,2, \cdots, r_{2}\right)
$$

所以

$$
d_{r_{1}}(a) \tilde{d}_{r_{2}}(a)=0
$$

这表明 $a$ 必是 $\lambda_{1}, \lambda_{2}, \cdots, \lambda_{t}$ 中的某一个,所以 $(\lambda-a)^{k}$ 是与某个 $\left(\lambda-\lambda_{i}\right)(i=1$ , $2, \cdots, t)$ 相应的一个初等因子。由上面证明可知,$(\lambda-a)^{k}$ 一定是某个 $(\lambda-a)^{h_{i i}}$ 或 $(\lambda-a)^{e_{j}}\left(i=1,2, \cdots, r_{1} ; j=1,2, \cdots, r_{2} ; l=1,2, \cdots, t\right)$ ,此即证明了除 $B(\lambda)$ 与 $C(\lambda)$的全部初等因子外, $\boldsymbol{A}(\lambda)$ 再没有别的初等因子。

应用归纳法,可把定理2.2.2推广为\\
定理2.2.3 若 $\lambda$-矩阵

$$
\boldsymbol{A}(\lambda)=\left[\begin{array}{llll}
\boldsymbol{B}_{1}(\lambda) & & & \\
& \boldsymbol{B}_{2}(\lambda) & & \\
& & \ddots & \\
& & & \boldsymbol{B}_{t}(\lambda)
\end{array}\right]
$$

则 $\boldsymbol{B}_{1}(\lambda), \boldsymbol{B}_{2}(\lambda), \cdots, \boldsymbol{B}_{t}(\lambda)$ 各个初等因子的全体构成 $\boldsymbol{A}(\lambda)$ 的全部初等因子。\\
证略。\\
根据定理2.2.3立即得到下述结论。\\
定理 2.2.4 若 $\lambda$-矩阵

$$
\boldsymbol{A}(\lambda)=\left[\begin{array}{lllllll}
f_{1}(\lambda) & & & & & & \\
& f_{2}(\lambda) & & & & & \\
& & \ddots & & & & \\
& & & f_{r}(\lambda) & & & \\
& & & & 0 & & \\
& & & & & \ddots & \\
& & & & & & 0
\end{array}\right]
$$

则 $f_{1}(\lambda), f_{2}(\lambda), \cdots, f_{r}(\lambda)$ 的所有一次因式的幂积构成 $\boldsymbol{A}(\lambda)$ 的全部初等因子。\\
注 若 $\boldsymbol{A}(\lambda)$ 是对角形,根据定理2.2.4不必把 $\boldsymbol{A}(\lambda)$ 变换成 Smith 标准就立刻可以得到 $\boldsymbol{A}(\lambda)$ 的初等因子。例2.1.4就可以简单得多。

例 2.2.2 求 $\lambda$-矩阵

$$
A(\lambda)=\left[\begin{array}{ccccc}
\lambda-a & b_{1} & & & \\
& \lambda-a & b_{2} & & \\
& & \ddots & \ddots & \\
& & & \ddots & b_{n-1} \\
& & & & \lambda-a
\end{array}\right]
$$

的不变因子和初等因子,其中 $b_{1}, b_{2}, \cdots, b_{n-1}$ 都是常数,且 $b_{1}, b_{2}, \cdots, b_{n-1} \neq 0$ .\\
解 $1 \boldsymbol{A}(\lambda)$ 行列式因子易得为

$$
D_{1}(\lambda)=D_{2}(\lambda)=\cdots=D_{n-1}(\lambda)=1, D_{n}(\lambda)=(\lambda-a)^{n}
$$

于是 $\boldsymbol{A}(\lambda)$ 的不变因子为

$$
d_{1}(\lambda)=d_{2}(\lambda)=\cdots=d_{n-1}(\lambda)=1, d_{n}(\lambda)=(\lambda-a)^{n}
$$

因而初等因子只有一个

$$
(\lambda-a)^{n}
$$

解 2 对 $A(\lambda)$ 用初等变换求得不变因子为

$$
\underbrace{1,1, \cdots, 1}_{n-1 \uparrow},(\lambda-a)^{n}
$$

故初等因子为

$$
(\lambda-a)^{n}
$$

例 2.2.3 求 $\lambda$-矩阵

$$
A(\lambda)=\left[\begin{array}{ccccc}
\lambda & 0 & \cdots & 0 & a_{n} \\
-1 & \lambda & \cdots & 0 & a_{n-1} \\
& \ddots & \ddots & & \vdots \\
0 & & -1 & \lambda & a_{2} \\
& & & -1 & \lambda+a_{1}
\end{array}\right]
$$

的初等因子和 Smith 标准形。\\
解 将 $\boldsymbol{A}(\lambda)$ 之第二行,第三行,$\cdots$ ,第 $n$ 行分别乘以 $\lambda, \lambda^{2}, \cdots, \lambda^{n-1}$ 都加到第一行上去,得到

$$
\boldsymbol{A}(\lambda) \simeq\left[\begin{array}{ccccc}
0 & 0 & \cdots & 0 & f(\lambda) \\
-1 & \lambda & \cdots & 0 & a_{n-1} \\
0 & -1 & \cdots & 0 & a_{n-2} \\
\vdots & \vdots & & \vdots & \vdots \\
0 & 0 & \cdots & \lambda & a_{2} \\
0 & 0 & \cdots & -1 & \lambda+a_{1}
\end{array}\right]
$$

其中

$$
f(\lambda)=\lambda^{n}+a_{1} \lambda^{n-1}+\cdots+a_{n-2} \lambda^{2}+a_{n-1} \lambda+a_{n}
$$

易得

$$
\operatorname{det} A(\lambda)=f(\lambda)
$$

故

$$
D_{n}(\lambda)=f(\lambda)
$$

又

$$
D_{n-1}(\lambda)=1
$$

于是

$$
D_{1}(\lambda)=D_{2}(\lambda)=\cdots=D_{n-2}(\lambda)=1
$$

所以 $d_{1}(\lambda)=d_{2}(\lambda)=\cdots=d_{n-1}(\lambda)=1, d_{n}(\lambda)=f(\lambda)$\\
因此 $\boldsymbol{A}(\lambda)$ 之 Smith 标准形为

$$
A(\lambda) \simeq\left[\begin{array}{llll}
1 & & & \\
& \ddots & & \\
& & 1 & \\
& & & f(\lambda)
\end{array}\right]
$$

\section*{二、矩阵相似条件}
数字矩阵的特征矩阵是 $\lambda$-矩阵,它是研究数字矩阵的重要工具。我们将把数字矩阵的相似归结为它们的特征矩阵等价。

引理 设 $\boldsymbol{A} 、 \boldsymbol{B}$ 是两个 $n$ 阶数字矩阵,则 $\boldsymbol{A} \sim \boldsymbol{B}$ 的充要条件是 $(\lambda \boldsymbol{E}-\boldsymbol{A}) \sim(\lambda \boldsymbol{E}-$ B).

证明 必要性 设 $\boldsymbol{A} \sim \boldsymbol{B}$ ,则存在 $\boldsymbol{P} \in C_{n}^{n \times n}$ ,满足

$$
\begin{gathered}
\boldsymbol{P}^{-1} \boldsymbol{A P}=\boldsymbol{B} \\
\lambda \boldsymbol{E}-\boldsymbol{B}=\lambda \boldsymbol{E}-\boldsymbol{P}^{-1} \boldsymbol{A P}=\boldsymbol{P}^{-1}(\lambda \boldsymbol{E}-\boldsymbol{A}) \boldsymbol{P}
\end{gathered}
$$

故

$$
\begin{gathered}
\boldsymbol{P}^{-1}(\lambda \boldsymbol{E}-\boldsymbol{A}) \boldsymbol{P}=\lambda \boldsymbol{E}-\boldsymbol{B} \\
\lambda \boldsymbol{E}-\boldsymbol{P}^{-1} \boldsymbol{A P}=\lambda \boldsymbol{E}-\boldsymbol{B} \\
\boldsymbol{B}=\boldsymbol{P}^{-1} \boldsymbol{A P}
\end{gathered}
$$

故\\
即\\
充分性 设

定理 2.2.5 $\boldsymbol{A} \sim \boldsymbol{B}$ 的充要条件是 $(\lambda \boldsymbol{E}-\boldsymbol{A}) \simeq(\lambda \boldsymbol{E}-\boldsymbol{B})$ .\\
证明 必要性显然。现证充分性。设 $(\lambda \boldsymbol{E}-\boldsymbol{A})$ 与 $(\lambda \boldsymbol{E}-\boldsymbol{B})$ 等价,则存在可逆 $\lambda$-矩阵 $\boldsymbol{U}(\lambda), \boldsymbol{V}(\lambda)$ ,便得

或

$$
\lambda \boldsymbol{E}-\boldsymbol{A}=\boldsymbol{U}(\lambda)(\lambda \boldsymbol{E}-\boldsymbol{B}) \boldsymbol{V}(\lambda)
$$

其中


\begin{align*}
& (\lambda)(\lambda E-A)=(\lambda E-B) V(  \tag{1}\\
& U(\lambda)=(\lambda E-A) Q(\lambda)+U_{0}  \tag{2}\\
& V(\lambda)=R(\lambda)(\lambda E-A)+V_{0} \tag{3}
\end{align*}


$\boldsymbol{U}_{0}, \boldsymbol{V}_{0}$ 为数字矩阵。把式(2)与式(3)代入式(1)得


\begin{equation*}
\left[\boldsymbol{U}^{-1}(\lambda)-(\lambda \boldsymbol{E}-\boldsymbol{B}) \boldsymbol{R}(\lambda)\right](\lambda \boldsymbol{E}-\boldsymbol{A})=(\lambda \boldsymbol{E}-\boldsymbol{B}) \boldsymbol{V}_{0} \tag{4}
\end{equation*}


式(4)右端是一个次数为 1 的 $\lambda$-矩阵或 $V_{0}=0$ ,而左端 $(\lambda \boldsymbol{E}-\boldsymbol{A})$ 也是一个次数为 1的 $\lambda$-矩阵,所以


\begin{equation*}
\boldsymbol{U}^{-1}(\lambda)-(\lambda \boldsymbol{E}-\boldsymbol{B}) \boldsymbol{R}(\lambda)=\boldsymbol{P} \tag{5}
\end{equation*}


$\boldsymbol{P}$ 是一个数字矩阵。因此式(4)可以写成


\begin{equation*}
P(\lambda E-A)=(\lambda E-B) V_{0} \tag{6}
\end{equation*}


现在证明 $\boldsymbol{P}$ 可逆,且 $\boldsymbol{P}=U_{0}^{-1}$\\
由式(5)可得

即


\begin{align*}
& \boldsymbol{U}(\lambda) \boldsymbol{P}=\boldsymbol{E}-\boldsymbol{U}(\lambda)(\lambda \boldsymbol{E}-\boldsymbol{B}) \boldsymbol{R}(\lambda) \\
& \boldsymbol{E}=\boldsymbol{U}(\lambda) \boldsymbol{P}+\boldsymbol{U}(\lambda)(\lambda \boldsymbol{E}-\boldsymbol{B}) \boldsymbol{R}(\lambda) \tag{7}
\end{align*}


由式(1)可把式(7)改为


\begin{equation*}
\boldsymbol{E}=\boldsymbol{U}(\lambda) \boldsymbol{P}+(\lambda \boldsymbol{E}-\boldsymbol{A}) \boldsymbol{V}^{-1}(\lambda) \boldsymbol{R}(\lambda) \tag{8}
\end{equation*}


把式(2)代入式(8)便得

$$
\boldsymbol{E}=\left[(\lambda \boldsymbol{E}-\boldsymbol{A}) \boldsymbol{Q}(\lambda)+\boldsymbol{U}_{0}\right] \boldsymbol{P}+(\lambda \boldsymbol{E}-\boldsymbol{A}) \boldsymbol{V}^{-1}(\lambda) \boldsymbol{R}(\lambda)
$$


\begin{equation*}
=\boldsymbol{U}_{0} \boldsymbol{P}+(\lambda \boldsymbol{E}-\boldsymbol{A})\left[\boldsymbol{Q}(\lambda) \boldsymbol{P}+\boldsymbol{V}^{-1}(\lambda) \boldsymbol{R}(\lambda)\right] \tag{9}
\end{equation*}


比较式(9)两边 $\lambda$-矩阵的次数知,式(9)右边第二项必为零,故

即


\begin{align*}
& \boldsymbol{E}=\boldsymbol{U}_{0} \boldsymbol{P} \\
& \boldsymbol{P}=\boldsymbol{U}_{0}^{-1} \tag{10}
\end{align*}


把式(10)代入式(6)得

$$
\boldsymbol{U}_{0}^{-1}(\lambda \boldsymbol{E}-\boldsymbol{A})=(\lambda \boldsymbol{E}-\boldsymbol{B}) \boldsymbol{V}_{0}
$$

或


\begin{equation*}
\lambda E-A=U_{0}(\lambda E-B) V_{0}=\lambda U_{0} V_{0}-U_{0} B V_{0} \tag{11}
\end{equation*}


比较式(11)两端得

故得

$$
\begin{array}{ll}
\boldsymbol{U}_{0} \boldsymbol{V}_{0}=\boldsymbol{E}, & \boldsymbol{A}=\boldsymbol{U}_{0} \boldsymbol{B} \boldsymbol{V}_{0} \\
\boldsymbol{U}_{0}=\boldsymbol{V}_{0}^{-1}, & \boldsymbol{A}=\boldsymbol{V}_{0}^{-1} \boldsymbol{B} \boldsymbol{V}_{0}
\end{array}
$$

证毕\\
注 今后为了叙述简练,约定对于一个数字矩阵 $\boldsymbol{A}$ ,称 $(\lambda \boldsymbol{E}-\boldsymbol{A})$ 的不变因子是 $\boldsymbol{A}$ 的不变因子,称 $(\lambda \boldsymbol{E}-\boldsymbol{A})$ 的初等因子是 $\boldsymbol{A}$ 的初等因子。

对于任何一个数字矩阵 $\boldsymbol{A},|\lambda \boldsymbol{E}-\boldsymbol{A}| \neq 0$ ,故 $\operatorname{rank}(\lambda \boldsymbol{E}-\boldsymbol{A})=n$ .于是由定理 2.2.1与定理2.2.5得到:

定理 2.2.6 $\boldsymbol{A} \sim \boldsymbol{B}$ 的充要条件是 $\boldsymbol{A} 、 \boldsymbol{B}$ 有相同的初等因子。\\
由定理2.1.8与定理2.2.5得到:\\
定理2.2.7 $\boldsymbol{A} \sim \boldsymbol{B}$ 的充要条件是 $\boldsymbol{A} 、 \boldsymbol{B}$ 有相同的不变因子。\\
例 2.2.4 已知数字矩阵 $\boldsymbol{A}$ 的初等因子为 $\lambda, \lambda, \lambda^{2},(\lambda-1)^{2}, \lambda+1$ ,求 $\boldsymbol{A}$ 的阶数与 $\boldsymbol{A}$ 的不变因子及 Smith 标准形。

解 因为 $\boldsymbol{A}$ 的初等因子乘积 $\lambda \cdot \lambda \cdot \lambda^{2} \cdot(\lambda-1)^{2} \cdot(\lambda+1)$ 是 7 次多项式,故 $\boldsymbol{A}$ 是 7 阶的。\\
$A$ 的不变因子 $\underbrace{1,1,1,1}_{4 \text { 个 }}, \lambda, \lambda, \lambda^{2}(\lambda-1)^{2}(\lambda+1)$\\
$\boldsymbol{A}$ 的 Smith 标准形

$$
\left[\begin{array}{ccccc}
\boldsymbol{E}_{4} & & & & \\
& \lambda & & & \\
& & \lambda & & \\
& & & \lambda & \\
& & & & \lambda^{2}(\lambda-1)^{2}(x+1)
\end{array}\right] \quad \boldsymbol{E}_{4} \text { 是 } 4 \text { 阶单位矩阵. }
$$

例 2.2.5 数字矩阵 $\boldsymbol{A}$ 的特征多项式 $|\lambda \boldsymbol{E}-\boldsymbol{A}|=(\lambda+1)^{2}(\lambda-2)^{3}$ ,求 $\boldsymbol{A}$ 的阶数、 $\boldsymbol{A}$ 的初等因子、不变因子。

解 $\boldsymbol{A}$ 是 5 阶矩阵,答案有如下几种情况:\\
(1) $\boldsymbol{A}$ 的初等因子 $\lambda+1, \lambda+1, \lambda-2, \lambda-2, \lambda-2$\\
$A$ 的不变因子 $1,1, \lambda-2,(\lambda+1)(\lambda-2),(\lambda+1)(\lambda-2)$\\
(2) $\boldsymbol{A}$ 的初等因子 $\lambda+1, \lambda+1, \lambda-2,(\lambda-2)^{2}$\\
$\boldsymbol{A}$ 的不变因子 $1,1,1,(\lambda+1)(\lambda-2),(\lambda-1)(\lambda-2)^{2}$\\
(3) $\boldsymbol{A}$ 的初等因子 $\lambda+1, \lambda+1,(\lambda-2)^{3}$\\
$\boldsymbol{A}$ 的不变因子 $1,1,1, \lambda+1,(\lambda+1)(\lambda-2)^{3}$\\
(4) $\boldsymbol{A}$ 的初等因子 $(\lambda+1)^{2}, \lambda-2, \lambda-2, \lambda-2$\\
$\boldsymbol{A}$ 的不变因子 $1,1, \lambda-2, \lambda-2,(\lambda-2)(\lambda+1)^{2}$\\
(5) $\boldsymbol{A}$ 的初等因子 $(\lambda+1)^{2}, \lambda-2,(\lambda-2)^{2}$\\
$\boldsymbol{A}$ 的不变因子 $\quad 1,1,1, \lambda-2,(\lambda-2)^{2}(\lambda+1)^{2}$\\
(6) $\boldsymbol{A}$ 的初等因子 $(\lambda+1)^{2},(\lambda-2)^{3}$\\
$\boldsymbol{A}$ 的不变因子 $1,1,1,1,(\lambda+1)^{2}(\lambda-2)^{3}$

\section*{§2.3 矩阵的 Jordan 标准形}
\section*{一、Jordan 标准形}
定义2.3.1 称 $n_{i}$ 阶矩阵

$$
J_{i}=\left[\begin{array}{cccc}
\lambda_{i} & 1 & & \\
& \lambda_{i} & 1 & \\
& & \ddots & \ddots .1 \\
& & & \lambda_{i}
\end{array}\right]_{n_{i} \times n_{i}}
$$

为 Jordan 块。设 $J_{1}, J_{2}, \cdots, J_{s}$ ,为 Jordan 块,称准对角矩阵

$$
\boldsymbol{J}=\left[\begin{array}{llll}
\boldsymbol{J}_{\mathbf{1}} & & & \\
& \boldsymbol{J}_{2} & & \\
& & \ddots & \\
& & & \boldsymbol{J}_{s}
\end{array}\right]
$$

为 Jordan 标准形。\\
在例2.2.2中已得到 Jordan 块的初等因子为 $\left(\lambda-\lambda_{i}\right)^{n_{i}}$ ,根据定理 2.2.3 知, Jordan 标准形的初等因子为 $\left(\lambda-\lambda_{1}\right)^{n_{1}},\left(\lambda-\lambda_{2}\right)^{n_{2}}, \cdots,\left(\lambda-\lambda_{3}\right)^{n_{4}}$ 。因此,结合定理2.2.6可以得到:

定理 2.3.1 设 $A \in C^{n \times n}, A$ 的初等因子为

$$
\left(\lambda-\lambda_{1}\right)^{n_{1}},\left(\lambda-\lambda_{2}\right)^{n_{2}}, \cdots,\left(\lambda-\lambda_{s}\right)^{n_{s}}
$$

则

$$
A \sim J
$$

这里

$$
\boldsymbol{J}=\left[\begin{array}{llll}
\boldsymbol{J}_{1} & & & \\
& \boldsymbol{J}_{2} & & \\
& & \ddots & \\
& & & \boldsymbol{J}_{s}
\end{array}\right]
$$

其中

$$
J_{i}=\left[\begin{array}{ccccc}
\lambda_{i} & 1 & & & \\
& \lambda_{i} & 1 & & \\
& & \ddots & \ddots & \\
& & & \ddots & 1 \\
& & & & \lambda_{i}
\end{array}\right]_{n_{i} \times n_{i}} \quad(i=1,2, \cdots, s)
$$

称 $\boldsymbol{J}$ 是矩阵 $\boldsymbol{A}$ 的 Jordan 标准形。\\
若 $n_{i}=1, J_{i}$ 是一阶 Jordan 块,当矩阵 $\boldsymbol{A}$ 的 Jordan 标准形中的 Jordan 块全是一阶时,$J$ 便是对角矩阵,因此可得:

定理 2.3.2 $\boldsymbol{A}$ 可对角化的充要条件是 $\boldsymbol{A}$ 的初等因子都是一次因式。\\
例2. 3.1 求矩阵

$$
A=\left[\begin{array}{rrr}
-1 & -2 & 6 \\
-1 & 0 & 3 \\
-1 & -1 & 4
\end{array}\right]
$$

的 Jordan 标准形。\\
解 先求 $\boldsymbol{A}$ 的初等因子。对( $\lambda \boldsymbol{E}-\boldsymbol{A}$ )运用初等变换可得

$$
\lambda \boldsymbol{E}-\boldsymbol{A}=\left[\begin{array}{ccc}
\lambda+1 & 2 & -6 \\
1 & \lambda & -3 \\
1 & 1 & \lambda-4
\end{array}\right] \simeq\left[\begin{array}{lll}
1 & & \\
& \lambda-1 & \\
& & (\lambda-1)^{2}
\end{array}\right]
$$

$\boldsymbol{A}$ 的初等因子是

$$
\lambda-1,(\lambda-1)^{2}
$$

故 $\boldsymbol{A}$ 的 Jordan 标准形是

$$
\boldsymbol{J}=\left[\begin{array}{lll}
1 & 0 & 0 \\
0 & 1 & 1 \\
0 & 0 & 1
\end{array}\right]
$$

\section*{二、变换矩阵 $P$}
根据定理 2.3.1 知,对于任何一个矩阵 $\boldsymbol{A}$ ,存在 $\boldsymbol{P} \in C_{n}^{n \times n}$ ,满足 $\boldsymbol{P}^{-1} \boldsymbol{A} \boldsymbol{P}=\boldsymbol{J}$ .现在介绍求变换矩阵 $\boldsymbol{P}$ 的方法。先看几个例子。

例2. 3.2 求矩阵

$$
A=\left[\begin{array}{rrr}
17 & 0 & -25 \\
0 & 1 & 0 \\
9 & 0 & -13
\end{array}\right]
$$

的 Jordan 标准形,并求变换矩阵 $\boldsymbol{P}$ .\\
解

因此

$$
\lambda \boldsymbol{E}-\boldsymbol{A} \simeq\left[\begin{array}{lll}
1 & & \\
& 1 & \\
& & (\lambda-1)(\lambda-2)^{2}
\end{array}\right]
$$

此即

$$
\boldsymbol{P}^{-1} A \boldsymbol{P}=\boldsymbol{J}=\left[\begin{array}{lll}
1 & 0 & 0 \\
0 & 2 & 1 \\
0 & 0 & 2
\end{array}\right]
$$


\begin{equation*}
A P=P J \tag{1}
\end{equation*}


命


\begin{equation*}
P=\left(X_{1}, X_{2}, X_{3}\right) \tag{2}
\end{equation*}


把 $\boldsymbol{P}$ 代人式(1)得

\[
A\left(X_{1}, X_{2}, X_{3}\right)=\left(X_{1}, X_{2}, X_{3}\right)\left[\begin{array}{lll}
1 & 0 & 0  \tag{3}\\
0 & 2 & 1 \\
0 & 0 & 2
\end{array}\right]
\]

比较式(3)两端得

$$
\begin{gathered}
A X_{1}=X_{1}, A X_{2}=2 X_{2}, A X_{3}=X_{2}+2 X_{3} \\
(E-A) X_{1}=0,(2 E-A) X_{2}=0,(2 E-A) X_{3}=-X_{2}
\end{gathered}
$$

由齐次线性方程组 $(\boldsymbol{E}-\boldsymbol{A}) \boldsymbol{X}=0$ 可求得

$$
X_{1}=(0,1,0)^{\mathrm{T}}
$$

由齐次线性方程组 $(2 \boldsymbol{E}-\boldsymbol{A}) \boldsymbol{X}=0$ 可求得

$$
X_{2}=(5,0,3)^{\mathrm{T}}
$$

把 $\boldsymbol{X}_{2}$ 代人 $(2 \boldsymbol{E}-\boldsymbol{A}) \boldsymbol{X}=-\boldsymbol{X}_{2}$ 可求得

$$
\boldsymbol{X}_{3}=(2,0,1)^{\mathrm{T}}
$$

所以

$$
P=\left(X_{1} X_{2} X_{3}\right)=\left[\begin{array}{lll}
0 & 5 & 2 \\
1 & 0 & 0 \\
0 & 3 & 1
\end{array}\right]
$$

例2.3.3 求化矩阵

$$
A=\left[\begin{array}{rrr}
-1 & -2 & 6 \\
-1 & 0 & 3 \\
-1 & -1 & 4
\end{array}\right]
$$

为 Jordan 标准形的变换矩阵 $\boldsymbol{P}$ .\\
解 由例2.3.1 知

$$
A \sim J=\left[\begin{array}{lll}
1 & 0 & 0 \\
0 & 1 & 1 \\
0 & 0 & 1
\end{array}\right]
$$

故存在 $P \in C_{3}^{3 \times 3}$ ,满足


\begin{gather*}
\boldsymbol{A P}=\boldsymbol{P J}  \tag{1}\\
\boldsymbol{P}=\left(\boldsymbol{X}_{1}, \boldsymbol{X}_{2}, \boldsymbol{X}_{3}\right)
\end{gather*}


命\\
把 $\boldsymbol{P}$ 代人式(1)得

\[
\left(A X_{1}, A X_{2}, A X_{3}\right)=\left(X_{1}, X_{2}, X_{3}\right)\left[\begin{array}{lll}
1 & 0 & 0  \tag{2}\\
0 & 1 & 1 \\
0 & 0 & 1
\end{array}\right]
\]

比较式(2)两边得

即

$$
\begin{gathered}
A X_{1}=X_{1}, A X_{2}=X_{2}, A X_{3}=X_{2}+X_{3} \\
(E-A) X_{1}=0,(E-A) X_{2}=0,(E-A) X_{3}=-X_{2}
\end{gathered}
$$

在上述每个方程组中只要依次各取一个解分别为 $\boldsymbol{X}_{1}, \boldsymbol{X}_{2}, \boldsymbol{X}_{3}$ ,组成 $\boldsymbol{P}=\left(\boldsymbol{X}_{1}\right.$ , $X_{2}, X_{3}$ )即可。

易见 $\boldsymbol{X}_{1}, \boldsymbol{X}_{2}$ 是 $\boldsymbol{A}$ 的特征值为 1 的两个线性无关的特征向量。解方程组

$$
(\boldsymbol{E}-\boldsymbol{A}) \boldsymbol{X}=0
$$

可求得两个线性无关的特征向量

$$
\boldsymbol{\xi}=(-1,1,0)^{\mathrm{T}}, \boldsymbol{\eta}=(3,0,1)^{\mathrm{T}}
$$

若取 $\boldsymbol{X}_{1}=\boldsymbol{\xi}, \boldsymbol{X}_{2}=\boldsymbol{\eta}$ ,代入 $(\boldsymbol{E}-\boldsymbol{A}) \boldsymbol{X}_{3}=-\boldsymbol{X}_{2}$ ,该方程组无解,这时不能认为 $\boldsymbol{P}$ 不存在。因为 $\boldsymbol{A}$ 的特征子空间是二维的,即 $\boldsymbol{A}$ 的线性无关特征向量不仅是 $\boldsymbol{\xi}, \boldsymbol{\eta}$ 。例如,只要 $\boldsymbol{S}, \boldsymbol{t}$ 满足 $\boldsymbol{S} \boldsymbol{t} \neq 1$ 的任意数, $\boldsymbol{\xi}+\boldsymbol{S}_{\boldsymbol{\eta}}, \boldsymbol{t \xi}+\boldsymbol{\eta}$ 也是 $\boldsymbol{A}$ 的线性无关的特征向量。因此,若取 $\boldsymbol{X}_{1}=\boldsymbol{\xi}, \boldsymbol{X}_{2}=\boldsymbol{\xi}+k \boldsymbol{\eta}(k \neq 0), k$ 只要使得方程组 $(\boldsymbol{E}-\boldsymbol{A}) \boldsymbol{X}_{3}=-\boldsymbol{X}_{2}$ 有解。不难知道当 $k=1$ 时,取 $\boldsymbol{X}_{2}=\boldsymbol{\xi}+\boldsymbol{\eta}=(2,1,1)^{\mathrm{T}}$ 代入 $(\boldsymbol{E}-\boldsymbol{A}) \boldsymbol{X}_{3}=-\boldsymbol{X}_{2}$ 方程组有解为

$$
x_{1}=-x_{2}+3 x_{3}-1\left(x_{2}, x_{3} \text { 为任意数 }\right)
$$

取它的一个解 $\boldsymbol{X}_{3}=(2,0,1)^{\mathrm{T}}$ ,就可.于是

$$
P=\left[\begin{array}{rrr}
-1 & 2 & 2 \\
1 & 1 & 0 \\
0 & 1 & 1
\end{array}\right]
$$

容易验证有

$$
\boldsymbol{P}^{-1} \boldsymbol{A} \boldsymbol{P}=\left[\begin{array}{lll}
1 & 0 & 0 \\
0 & 1 & 1 \\
0 & 0 & 1
\end{array}\right]
$$

从以上两例可以概括出求 Jordan 标准形变换矩阵 $\boldsymbol{P}$ 的过程。\\
设 $\boldsymbol{A}$ 的 Jordan 标准形为 $\boldsymbol{J}$ ,则

\[
A \boldsymbol{P}=\boldsymbol{P} \boldsymbol{J}=\boldsymbol{P}\left[\begin{array}{llll}
\boldsymbol{J}_{1} & & &  \tag{2.3.1}\\
& \boldsymbol{J}_{2} & & \\
& & \ddots & \\
& & & \boldsymbol{J}_{s}
\end{array}\right]
\]

\[
\boldsymbol{J}_{i}=\left[\begin{array}{ccccc}
\lambda_{i} & 1 & & &  \tag{2.3.2}\\
& \lambda_{i} & 1 & & \\
& & \ddots & \ddots & \\
& & & \ddots & 1 \\
& & & & \lambda_{i}
\end{array}\right]_{n_{i} \times n_{i}}
\]

其中

把变换矩阵 $\boldsymbol{P}$ 按 Jordan 块 $\boldsymbol{J}_{i}$ 的阶数 $n_{i}$ 进行相应的分块,即设


\begin{equation*}
\boldsymbol{P}=\left(\boldsymbol{P}_{1}, \boldsymbol{P}_{2}, \cdots, \boldsymbol{P}_{s}\right) \tag{2.3.3}
\end{equation*}


其中 $\boldsymbol{P}_{i} \in C^{n \times n_{i}}$ ,因此

$$
\begin{aligned}
\boldsymbol{A}\left(\boldsymbol{P}_{1}, \boldsymbol{P}_{2}, \cdots, \boldsymbol{P}_{s}\right) & =\left(\boldsymbol{P}_{1}, \boldsymbol{P}_{2}, \cdots, \boldsymbol{P}_{s}\right) \boldsymbol{J} \\
& =\left(\boldsymbol{P}_{1}, \boldsymbol{P}_{2}, \cdots, \boldsymbol{P}_{s}\right)\left[\begin{array}{llll}
\boldsymbol{J}_{1} & & & \\
& \boldsymbol{J}_{2} & & \\
& & \ddots & \\
& & & \boldsymbol{J}_{s}
\end{array}\right]
\end{aligned}
$$

故

$$
\left(A P_{1}, A P_{2}, \cdots, A P_{s}\right)=\left(P_{1} J_{1}, P_{2} J_{2}, \cdots, P_{s} J_{s}\right)
$$

比较上式两端得


\begin{equation*}
\boldsymbol{A} \boldsymbol{P}_{i}=\boldsymbol{P}_{i} \boldsymbol{J}_{i} \quad(i=1,2, \cdots, s) \tag{2.3.4}
\end{equation*}


对 $P_{i}$ 再按列分块


\begin{equation*}
\boldsymbol{P}_{i}=\left(\boldsymbol{X}_{i 1}, \boldsymbol{X}_{i 2}, \cdots, \boldsymbol{X}_{i n_{i}}\right) \in C^{n \times n_{i}} \tag{2.3.5}
\end{equation*}


其中 $\boldsymbol{X}_{i 1}, \boldsymbol{X}_{i 2}, \cdots, \boldsymbol{X}_{i n_{i}}$ 是 $n_{i}$ 个线性无关的 $n$ 维列向量,代入式(2.3.4)可得

\[
\left\{\begin{array}{l}
\boldsymbol{A} \boldsymbol{X}_{i 1}=\lambda_{i} \boldsymbol{X}_{i 1}  \tag{$i=1,2,\cdots,s$}\\
\boldsymbol{A} \boldsymbol{X}_{i 2}=\boldsymbol{X}_{i 1}+\lambda_{i} \boldsymbol{X}_{i 2} \\
\vdots \\
\boldsymbol{A} \boldsymbol{X}_{i n_{i}}=\boldsymbol{X}_{i n_{i}-1}+\lambda_{i} \boldsymbol{X}_{i n_{i}}
\end{array}\right.
\]

由第一个方程看到,列向量 $\boldsymbol{X}_{i 1}$ 是矩阵 $A$ 的特征值为 $\lambda_{i}$ 所对应的特征向量。且由 $\boldsymbol{X}_{i 1}$ 继而可以求得 $\boldsymbol{X}_{i 2}, \boldsymbol{X}_{i 3}, \cdots, \boldsymbol{X}_{i n_{i}}$ 。因此,长方形矩阵 $\boldsymbol{P}_{i}$ 以至 $\boldsymbol{P}$ 都可求得。由前面例子中可以看到,特征向量 $\boldsymbol{X}_{i 1}$ 的选取要保证 $\boldsymbol{X}_{i 2}$ 可以求出,类似地 $\boldsymbol{X}_{i 2}$ 的选取(因为 $\boldsymbol{X}_{i 2}$ 的选定并不唯一,只要适当选取一个就可)也要保证 $\boldsymbol{X}_{i 3}$ 可以求出,如此等等。

注 1 :在上述分析中还可以得到如下三点结论:\\
(1)每一个 Jordan 块 $J_{i}$ 对应着属于 $\lambda_{i}$ 的一个特征向量;\\
(2)对于给定特征值 $\lambda_{i}$ ,其对应 Jordan 块的个数等于 $\lambda_{i}$ 的几何重复度;\\
(3)对于给定特征值 $\lambda_{i}$ 所对应全体 Jordan 块的阶数之和等于 $\lambda_{i}$ 的代数重复度.

在例2.3.1中 $\lambda=1$ 的几何重复度为 2 ,所以它对应的 Jordan 块有 2 个,其中

一个是1阶,一个是2阶,阶数之和为 3 ,正是 $\lambda=1$ 的代数重复度。\\
在例2.3.2中,$\lambda=2$ 的几何重复度为 1 ,所以它对应的 Jordan 块有 1 个,其阶数恰是它的代数重复度.$\lambda=1$ 的几何重复度为 1 ,代数重复度为 1 ,它所对应的是 1 个 1 阶 Jordan 块。

注2:在求 $\boldsymbol{P}$ 的过程中例2.3.2题不会发生困难。例2.3.3题就可能产生困难,其主要原因是后者特征子空间是二维,于是就有可能非齐次方程组无解。

例 2.3.4 已知数字矩阵 $\boldsymbol{A}$ 的特征多项式 $|\lambda \boldsymbol{E}-\boldsymbol{A}|=(\lambda+1)^{4}(\lambda-2)^{3}$ ,且 $\lambda=-1$ 的几何重复度为 $2, \lambda=2$ 的几何重复度为 2 ,求 $\boldsymbol{A}$ 的初等因子及 Jordan 标准形.

解 两种可能性.(1)初等因子 $\lambda+1,(\lambda+1)^{3}, \lambda-2,(\lambda-2)^{2}$ ,(2)初等因子 $(\lambda+1)^{2},(\lambda+1)^{2}, \lambda-2,(\lambda-2)^{2}$(Jordan 标准形略).

\section*{三、求 Jordan 标准形的另一方法}
前面介绍了应用求 $\boldsymbol{A}$ 的初等因子的方法可以写出 Jordan 标准形。现在介绍应用计算 $\operatorname{rank}\left(\lambda_{i} \boldsymbol{E}-\boldsymbol{A}\right)^{k}$ ,得出对应于 $\lambda_{i}$ 的 Jordan 块的阶数、个数,然后可以写出 Jordan标准形。先看例 2.3. 4.

例2. 3.5 已知

$$
\begin{aligned}
& \boldsymbol{J}_{i}=\left[\begin{array}{ccccc}
\lambda_{i} & 1 & & & \\
& \lambda_{i} & 1 & & \\
& & \ddots & \ddots & \\
& & & \ddots & 1 \\
& & & & \lambda_{i}
\end{array}\right]_{n_{i} \times n_{i}} \\
& \boldsymbol{J}_{j}=\left[\begin{array}{ccccc}
\lambda_{j} & 1 & & & \\
& \lambda_{j} & 1 & & \\
& & \ddots & \ddots & \\
& & & \ddots & 1 \\
& & & & \lambda_{j}
\end{array}\right]_{n_{j} \times n_{i}}
\end{aligned}
$$

用 $\boldsymbol{E}_{k}$ 表示 $k$ 阶单位矩阵。\\
A.若 $\lambda_{i} \neq \lambda_{j}$ ,则\\
(a) $\operatorname{rank}\left(\lambda_{i} \boldsymbol{E}_{i}-\boldsymbol{J}_{i}\right)=n_{i}-1$\\
$\operatorname{rank}\left(\lambda_{i} E_{i}-J_{i}\right)^{2}=n_{i}-2$\\
$\vdots$\\
$\operatorname{rank}\left(\lambda_{i} E_{i}-J_{i}\right)^{n_{i}-1}=n_{i}-\left(n_{i}-1\right)=1$\\
$\operatorname{rank}\left(\lambda_{i} \boldsymbol{E}_{i}-\boldsymbol{J}_{i}\right)^{h}=0 \quad\left(h \geqslant n_{i}\right)$\\
(b) $\operatorname{rank}\left(\lambda_{i} E_{j}-J_{j}\right)^{l}=n_{j} \quad(l=1,2, \cdots)$\\
(c) $\operatorname{rank}\left[\lambda_{i} E_{i+j}-\left(\begin{array}{cc}J_{i} & \\ & J_{j}\end{array}\right)\right]=\left(n_{i}-1\right)+n_{j}=n_{i}+n_{j}-1$

$$
\begin{aligned}
& \operatorname{rank}\left[\lambda_{i} E_{i+j}-\left(\begin{array}{cc}
J_{i} & \\
& J_{j}
\end{array}\right)\right]^{2}=\left(n_{i}-2\right)+n_{j}=n_{i}+n_{j}-2 \\
& \vdots \\
& \operatorname{rank}\left[\lambda_{i} E_{i+j}-\left(\begin{array}{cc}
J_{i} & \\
& J_{j}
\end{array}\right)\right]^{h}=n_{j} \quad\left(h \geqslant n_{i}\right)
\end{aligned}
$$

B.若 $\lambda_{i}=\lambda_{j}$ ,不妨设 $n_{i}>n_{j}$ ,则\\
( a) $\operatorname{rank}\left(\lambda_{i} \boldsymbol{E}_{j}-\boldsymbol{J}_{j}\right)=n_{j}-1$

$$
\begin{aligned}
& \operatorname{rank}\left(\lambda_{i} E_{j}-J_{j}\right)^{2}=n_{j}-2 \\
& \vdots \\
& \operatorname{rank}\left(\lambda_{i} E_{j}-J_{j}\right)^{h}=0 \quad\left(h \geqslant n_{j}\right)
\end{aligned}
$$

(b) $\operatorname{rank}\left[\lambda_{i} E_{i+j}-\binom{J_{i}}{J_{j}}\right]=\left(n_{i}-1\right)+\left(n_{j}-1\right)=n_{i}+n_{j}-2$

$$
\begin{aligned}
& \operatorname{rank}\left[\lambda_{i} E_{i+j}-\left(\begin{array}{cc}
J_{i} & \\
& J_{j}
\end{array}\right)\right]^{2}=\left(n_{i}-2\right)+\left(n_{j}-2\right)=n_{i}+n_{j}-4 \\
& \vdots \\
& \operatorname{rank}\left[\lambda_{i} E_{i+j}-\left(\begin{array}{cc}
J_{i} & \\
& J_{j}
\end{array}\right)\right]^{n_{j}-1}=\left(n_{i}-\left(n_{j}-1\right)\right)+\left(n_{j}-\left(n_{j}-1\right)\right)=n_{i}-n_{j}+2 \\
& \operatorname{rank}\left[\lambda_{i} E_{i+j}-\left(\begin{array}{cc}
J_{i} & \\
& J_{j}
\end{array}\right)\right]^{n_{j}}=\left(n_{i}-n_{j}\right)+\left(n_{j}-n_{j}\right)=n_{i}-n_{j} \\
& \operatorname{rank}\left[\lambda_{i} E_{i+j}-\left(\begin{array}{cc}
J_{i} & \\
& J_{j}
\end{array}\right)\right]^{n_{j}+1}=\left[n_{i}-\left(n_{j}+1\right)\right]+0 \\
& =n_{i}-\left(n_{j}+1\right) \\
& \operatorname{rank}\left[\lambda_{i} E_{i+j}-\left(\begin{array}{cc}
J_{i} & \\
& J_{j}
\end{array}\right)\right]^{n_{j}+2}=n_{i}-\left(n_{j}+2\right) \\
& \vdots \\
& \operatorname{rank}\left[\lambda_{i} E_{i+j}-\left(\begin{array}{cc}
J_{i} & \\
& J_{j}
\end{array}\right)\right]^{n_{i}}=0
\end{aligned}
$$

反过来,可以借助 $\operatorname{rank}\left(\lambda_{i} \boldsymbol{E}_{i}-\boldsymbol{J}_{i}\right)^{k}, \operatorname{rank}\left(\lambda_{i} \boldsymbol{E}_{j}-\boldsymbol{J}_{j}\right)^{k}$ , $\operatorname{rank}\left[\lambda_{i} E_{i+j}-\left(\begin{array}{cc}J_{i} & \\ & J_{j}\end{array}\right)\right]^{k}$ 得出 $J_{i}, J_{j}$ 的阶数 $n_{i}, n_{j}$ .

由于 $\operatorname{rank}\left(\lambda_{i} \boldsymbol{E}-\boldsymbol{J}\right)^{k}=\operatorname{rank}\left(\lambda_{i} \boldsymbol{E}-\boldsymbol{A}\right)^{k}$ ,因此可以借助计算 $\operatorname{rank}\left(\lambda_{i} \boldsymbol{E}-\boldsymbol{A}\right)^{k}$ 得到 Jordan 块的个数,阶数分析,继而可得 $\boldsymbol{J}$ 的形状。

例2.3.6 已知 10 阶矩阵 $\boldsymbol{A}$ 的特征多项式

$$
|\lambda \boldsymbol{E}-\boldsymbol{A}|=(\lambda-2)^{7}(\lambda-3)^{3}
$$

经过计算得到下述数据:\\
当 $\lambda=2$ 时:(1) $\operatorname{rank}(2 \boldsymbol{E}-\boldsymbol{A})=7$ ;(2) $\operatorname{rank}(2 \boldsymbol{E}-\boldsymbol{A})^{2}=4$ ;(3) $\operatorname{rank}(2 \boldsymbol{E}-\boldsymbol{A})^{3}=3$ ; (4) $\operatorname{rank}(2 \boldsymbol{E}-\boldsymbol{A})^{4}=3$ .

当 $\lambda=3$ 时:(5) $\mathrm{rank}(3 \boldsymbol{E}-\boldsymbol{A})=8$ ;(6) $\mathrm{rank}(3 \boldsymbol{E}-\boldsymbol{A})^{2}=7$ ;(7) $\mathrm{rank}(3 \boldsymbol{E}-\boldsymbol{A})^{3}=7$ .\\
根据所得 7 个数据得到 $\boldsymbol{A}$ 的 Jordan 标准形。\\
由数据(1)知,对于 $\lambda=2$ 的 Jordan 块共有 $10-7=3$ 块;由数据(2)知,$\lambda=2$ 的 Jordan 块,阶数 $\geqslant 2$ 的有 $7-4=3$ 块;由数据(3)知,$\lambda=2$ 的 Jordan 块阶数 $\geqslant 3$ 的有 $4-3=1$ 块;由数据(4)知,$\lambda=2$ 的 Jordan 块最高阶数为 3 阶。因此 $\lambda=2$ 的 Jordan块分别为 3 阶 1 块,2阶 2 块共 3 块。

由数据(5)知,$\lambda=3$ 的 Jordan 块共有 $10-8=2$ 块;由数据(6)知,$\lambda=3$ 的 Jordan块阶数 $\geqslant 2$ 的有 $8-7=1$ 块;由数据(7)知,$\lambda=3$ 的 Jordan 块最高阶数为 2 阶。因此 $\lambda=3$ 的 Jordan 块分别为 2 阶 1 块, 1 阶 1 块。

一般地可作出如下的分析:\\
对 $n$ 阶矩阵 $\boldsymbol{A}$ 若得到: $\operatorname{rank}\left(\lambda_{i} \boldsymbol{E}-\boldsymbol{A}\right)=S_{1} ; \operatorname{rank}\left(\lambda_{i} \boldsymbol{E}-\boldsymbol{A}\right)^{2}=S_{2} ; \operatorname{rank}\left(\lambda_{i} \boldsymbol{E}-\boldsymbol{A}\right)^{3}= S_{3} ; \cdots ; \operatorname{rank}\left(\lambda_{i} \boldsymbol{E}-\boldsymbol{A}\right)^{l}=S_{l} ; \operatorname{rank}\left(\lambda_{i} \boldsymbol{E}-\boldsymbol{A}\right)^{l+1}=S_{l}$ 。则得到对于 $\lambda=\lambda_{i}$ 的 Jordan 块阶数,块数情况分析如下:共有 $n-S_{1}$ 个 Jordan 块,其中阶数最高的为 $l$ 阶。阶数 $\geqslant 2$ 的 Jordan 块有 $S_{1}-S_{2}$ 个,阶数 $\geqslant 3$ 的有 $S_{2}-S_{3}$ 个,阶数 $\geqslant 4$ 的有 $S_{3}-S_{4}$ 个,$\cdots, l$ 阶的有 $S_{l-1}-S_{l}$ 块。

例2. 3.7 已知

$$
A=\left[\begin{array}{llll}
1 & 2 & 3 & 4 \\
0 & 1 & 2 & 3 \\
0 & 0 & 1 & 2 \\
0 & 0 & 0 & 1
\end{array}\right]
$$

试求 $\boldsymbol{A}$ 的 Jordan 标准形.\\
解

$$
\begin{gathered}
|\lambda \boldsymbol{E}-\boldsymbol{A}|=(\lambda-1)^{4} \\
\operatorname{rank}(\boldsymbol{E}-\boldsymbol{A})=3, \operatorname{rank}(\boldsymbol{E}-\boldsymbol{A})^{2}=2 \\
\operatorname{rank}(\boldsymbol{E}-\boldsymbol{A})^{3}=1, \operatorname{rank}(\boldsymbol{E}-\boldsymbol{A})^{4}=0
\end{gathered}
$$

因此,Jordan 块是 4 阶 1 块,即

$$
A \sim\left[\begin{array}{llll}
1 & 1 & & \\
& 1 & 1 & \\
& & 1 & 1 \\
& & & 1
\end{array}\right]
$$

\section*{四、Jordan 标准形的应用}
已知常系数线性微分方程组

\[
\left\{\begin{array}{l}
\frac{\mathrm{d} x_{1}}{\mathrm{~d} t}=a_{11} x_{1}+a_{12} x_{2}+\cdots+a_{1 n} x_{n}  \tag{2.3.7}\\
\frac{\mathrm{~d} x_{2}}{\mathrm{~d} t}=a_{21} x_{1}+a_{22} x_{2}+\cdots+a_{2 n} x_{n} \\
\vdots \\
\vdots \\
\frac{\mathrm{~d} x_{n}}{\mathrm{~d} t}=a_{n 1} x_{1}+a_{n 2} x_{2}+\cdots+a_{n n} x_{n}
\end{array}\right.
\]

其中 $a_{i j}(i, j=1,2, \cdots, n)$ 均为常数。将此方程写成矩阵形式


\begin{equation*}
\frac{\mathrm{d} \boldsymbol{X}}{\mathrm{~d} t}=\boldsymbol{A} \boldsymbol{X} \tag{2.3.8}
\end{equation*}


这里

$$
\begin{gathered}
\boldsymbol{A}=\left(a_{i j}\right) \in C^{n \times n}, \boldsymbol{X}=\left(x_{1}(t), x_{2}(t), \cdots, x_{n}(t)\right)^{\mathrm{T}} \\
\frac{\mathrm{~d} \boldsymbol{X}}{\mathrm{~d} t}=\left(\frac{\mathrm{d} x_{1}}{\mathrm{~d} t}, \frac{\mathrm{~d} x_{2}}{\mathrm{~d} t}, \cdots, \frac{\mathrm{~d} x_{n}}{\mathrm{~d} t}\right)^{\mathrm{T}}
\end{gathered}
$$

设 $\boldsymbol{J}$ 是 $\boldsymbol{A}$ 的 Jordan 标准形,则


\begin{gather*}
\boldsymbol{P}^{-1} \boldsymbol{A} \boldsymbol{P}=\boldsymbol{J}  \tag{2.3.9}\\
\boldsymbol{X}=\boldsymbol{P} \boldsymbol{Y} \quad \boldsymbol{Y}=\left(y_{1}(t), y_{2}(t), \cdots, y_{n}(t)\right)^{\mathrm{T}} \tag{2.3.10}
\end{gather*}


令

把式(2.3.10)代人式(2.3.8)得


\begin{equation*}
\boldsymbol{P} \frac{\mathrm{d} \boldsymbol{Y}}{\mathrm{~d} t}=\boldsymbol{A P Y} \tag{2.3.11}
\end{equation*}


将 $\boldsymbol{P}^{-1}$ 左乘上式两端得


\begin{equation*}
\frac{\mathrm{d} \boldsymbol{Y}}{\mathrm{~d} t}=\boldsymbol{P}^{-1} \boldsymbol{A} \boldsymbol{P} \boldsymbol{Y}=\boldsymbol{J} \boldsymbol{Y} \tag{2.3.12}
\end{equation*}


若由式(2.3.12)可求得 $\boldsymbol{Y}$ ,然后通过式(2.3.10)可求得原来方程的解 $\boldsymbol{X}$ 。\\
例 2.3 .8 求微分方程组

$$
\left\{\begin{array}{l}
\frac{\mathrm{d} x_{1}}{\mathrm{~d} t}=-x_{1}-2 x_{2}+6 x_{3} \\
\frac{\mathrm{~d} x_{2}}{\mathrm{~d} t}=-x_{1}+3 x_{3} \\
\frac{\mathrm{~d} x_{3}}{\mathrm{~d} t}=-x_{1}-x_{2}+4 x_{3}
\end{array}\right.
$$

的解.\\
解 命 $\mathrm{X}=\left(x_{1}, x_{2}, x_{3}\right)^{\mathrm{T}}$ ,则方程组可写为

$$
\frac{\mathrm{d} \boldsymbol{X}}{\mathrm{~d} t}=\left[\begin{array}{rrr}
-1 & -2 & 6 \\
-1 & 0 & 3 \\
-1 & -1 & 4
\end{array}\right] \boldsymbol{X}=\boldsymbol{A} \boldsymbol{X}
$$

其中

$$
A=\left[\begin{array}{rrr}
-1 & -2 & 6 \\
-1 & 0 & 3 \\
-1 & -1 & 4
\end{array}\right]
$$

根据例2.3.3知

$$
\boldsymbol{P}^{-1} \boldsymbol{A} \boldsymbol{P}=\left[\begin{array}{lll}
1 & 0 & 0 \\
0 & 1 & 1 \\
0 & 0 & 1
\end{array}\right]
$$

其中

$$
\boldsymbol{P}=\left[\begin{array}{rrr}
-1 & 2 & 2 \\
1 & 1 & 0 \\
0 & 1 & 1
\end{array}\right] \quad \boldsymbol{P}^{-1}=\left[\begin{array}{rrr}
-1 & 0 & 2 \\
1 & 1 & -2 \\
-1 & -1 & 3
\end{array}\right]
$$

令 $\boldsymbol{X}=\boldsymbol{P Y}$ ,则由式(2.3.12)得

$$
\frac{\mathrm{d} y_{1}}{\mathrm{~d} t}=y_{1}, \quad \frac{\mathrm{~d} y_{2}}{\mathrm{~d} t}=y_{2}+y_{3}, \quad \frac{\mathrm{~d} y_{3}}{\mathrm{~d} t}=y_{3}
$$

不难求得

$$
y_{1}=k_{1} \mathrm{e}^{t}, \quad y_{3}=k_{3} \mathrm{e}^{t}, \quad y_{2}=\left(k_{3} t+k_{2}\right) \mathrm{e}^{t}
$$

代人 $\boldsymbol{X}=\boldsymbol{P Y}$ 得

$$
\begin{aligned}
& x_{1}=-k_{1} \mathrm{e}^{t}+2 k_{3} \mathrm{e}^{t}+2\left(k_{3} t+k_{2}\right) \mathrm{e}^{t} \\
& x_{2}=k_{1} \mathrm{e}^{t}+\left(k_{3} t+k_{2}\right) \mathrm{e}^{t} \\
& x_{3}=k_{3} \mathrm{e}^{t}+\left(k_{3} t+k_{2}\right) \mathrm{e}^{t}
\end{aligned}
$$

其中 $k_{1}, k_{2}, k_{3}$ 为任意常数。\\
例2.3.9 已知

$$
A=\left[\begin{array}{rrr}
3 & 0 & 1 \\
-1 & 2 & 1 \\
1 & 0 & 3
\end{array}\right]
$$

求 $A^{100}$ 。\\
解 先求 $\boldsymbol{A}$ 的初等因子,然后求得 $\boldsymbol{A}$ 的 Jordan 标准形

$$
J=\left[\begin{array}{lll}
2 & 1 & \\
& 2 & \\
& & 4
\end{array}\right]
$$

设 $\boldsymbol{P}=\left(\boldsymbol{\alpha}_{1}, \boldsymbol{\alpha}_{2}, \boldsymbol{\alpha}_{3}\right)$ ,且 $\boldsymbol{P}^{-1} \boldsymbol{A P}=\boldsymbol{J}$ ,即 $\boldsymbol{A P}=\boldsymbol{P J}$ .\\
于是

$$
\begin{array}{ll}
\boldsymbol{A} \boldsymbol{\alpha}_{1}=2 \boldsymbol{\alpha}_{1} & (2 \boldsymbol{E}-\boldsymbol{A}) \boldsymbol{\alpha}_{1}=0 \\
\boldsymbol{A} \boldsymbol{\alpha}_{2}=\boldsymbol{\alpha}_{1}+2 \boldsymbol{\alpha}_{2} & (2 \boldsymbol{E}-\boldsymbol{A}) \boldsymbol{\alpha}_{2}=-\boldsymbol{\alpha}_{1} \\
\boldsymbol{A} \boldsymbol{\alpha}_{3}=4 \boldsymbol{\alpha}_{3} & (4 \boldsymbol{E}-\boldsymbol{A}) \boldsymbol{\alpha}_{3}=0
\end{array}
$$

不难求得

$$
\boldsymbol{\alpha}_{1}=(0,1,0)^{\mathrm{T}}
$$

$$
\begin{gathered}
\boldsymbol{\alpha}_{2}=\left(-\frac{1}{2}, 0, \frac{1}{2}\right)^{\mathrm{T}} \\
\boldsymbol{\alpha}_{3}=(1,0,1)^{\mathrm{T}}
\end{gathered}
$$

故

$$
\boldsymbol{P}=\left[\begin{array}{ccc}
0 & -\frac{1}{2} & 1 \\
1 & 0 & 0 \\
0 & \frac{1}{2} & 1
\end{array}\right], \quad \boldsymbol{P}^{-1}=\left[\begin{array}{rrr}
0 & 1 & 0 \\
-1 & 0 & 1 \\
\frac{1}{2} & 0 & \frac{1}{2}
\end{array}\right]
$$

于是

$$
\begin{aligned}
A & =\left[\begin{array}{ccc}
0 & -\frac{1}{2} & 1 \\
1 & 0 & 0 \\
0 & \frac{1}{2} & 1
\end{array}\right]\left[\begin{array}{lll}
2 & 1 & \\
& 2 & \\
& & 4
\end{array}\right]\left[\begin{array}{rrr}
0 & 1 & 0 \\
-1 & 0 & 1 \\
\frac{1}{2} & 0 & \frac{1}{2}
\end{array}\right] \\
A^{100} & =\left[\begin{array}{ccc}
0 & -\frac{1}{2} & 1 \\
1 & 0 & 0 \\
0 & \frac{1}{2} & 1
\end{array}\right]\left[\begin{array}{ccc}
2^{100} & 100 \cdot 2^{99} & \\
& 2^{100} & \\
& & \\
100
\end{array}\right]\left[\begin{array}{rrr}
0 & 1 & 0 \\
-1 & 0 & 1 \\
\frac{1}{2} & 0 & \frac{1}{2}
\end{array}\right] \\
& =\left[\begin{array}{ccc}
-2^{99}+2^{199} & 0 & -2^{99}+2^{199} \\
-100 \cdot 2^{99} & 2^{100} & 100 \cdot 2^{99} \\
-2^{99}+2^{199} & 0 & 2^{99}+2^{199}
\end{array}\right]
\end{aligned}
$$

例 2.3.10 已知 $\boldsymbol{A}$ 是 $n$ 阶实数矩阵,且 $\boldsymbol{A}^{2}=k \boldsymbol{A}$( $k$ 为非零实数)。试证: $\boldsymbol{A}$ 相似于对角阵。

证明 由 $\boldsymbol{A}^{2}=k \boldsymbol{A}$ 可知 $\boldsymbol{A}$ 的特征值只可能是 $0, k$ .\\
方法一 由 $A^{2}=k A$ 得 $A(k E-A)=0$\\
故

$$
\text { 秩 }(\boldsymbol{A})+\text { 秩 }(k \boldsymbol{E}-\boldsymbol{A}) \leqslant n
$$

又

$$
\text { 秩 }(\boldsymbol{A})+\text { 秩 }(k \boldsymbol{E}-\boldsymbol{A}) \geqslant \text { 秩 }(\boldsymbol{A}+k \boldsymbol{E}-\boldsymbol{A})=\text { 秩 }(k \boldsymbol{E})=n
$$

因此

$$
\text { 秩 }(\boldsymbol{A})+\text { 秩 }(k \boldsymbol{E}-\boldsymbol{A})=n
$$

若秩 $(\boldsymbol{A})=r$ ,则 $\boldsymbol{A}$ 的属于特征值为 0 的线性无关特征向量有 $\boldsymbol{n}-r$ 个, $\boldsymbol{A}$ 的属于特征值为 $k$ 的线性无关特征向量有 $n-$ 秩 $(k \boldsymbol{E}-\boldsymbol{A})=n-[n-$ 秩 $(\boldsymbol{A})]=r$ 。所以 $\boldsymbol{A}$ 共有 $(n-r)+r=n$ 个线性无关特征向量.于是 $A$ 可对角化.

方法二 设 $\boldsymbol{A}$ 的 Jordan 标准形为 $\boldsymbol{J}=\operatorname{diag}\left(\boldsymbol{J}_{1}, \boldsymbol{J}_{2}, \cdots, \boldsymbol{J}_{r}\right)$ 。于是存在可逆矩阵\\
$\boldsymbol{P}$ ,满足

$$
P^{-1} A P=J, \quad A=P J P^{-1}
$$

代人 $\boldsymbol{A}^{2}=k \boldsymbol{A}$ 可得 $\boldsymbol{J}^{2}=k \boldsymbol{J}, \boldsymbol{J}_{i}^{2}=k \boldsymbol{J}_{i}(i=1,2, \cdots, r)$ .不难验算可知,若 $\boldsymbol{J}_{i}^{2}=k \boldsymbol{J}_{i} . \boldsymbol{J}_{i}$必须是一阶 Jordan 块。因此 $\boldsymbol{A}$ 的 Jordan 块 $\boldsymbol{J}_{i}(i=1,2, \cdots, r)$ 全是一阶的。因此 $\boldsymbol{A}$与对角形矩阵相似。

例2.3.11 试证任一方阵 $\boldsymbol{A}$ 与其转置矩阵 $\boldsymbol{A}^{\mathrm{T}}$ 相似。\\
证明 设 $\boldsymbol{A}$ 的 Jordan 标准形 $\boldsymbol{J}=\operatorname{diag}\left(\boldsymbol{J}_{1}, \boldsymbol{J}_{2}, \cdots, \boldsymbol{J}_{r}\right)$ ,即存在可逆矩阵 $\boldsymbol{P}$ ,满足

$$
\boldsymbol{J}=\operatorname{diag}\left(\boldsymbol{J}_{1}, \boldsymbol{J}_{2}, \cdots, \boldsymbol{J}_{r}\right)=\boldsymbol{P} \boldsymbol{A} \boldsymbol{P}^{-1}
$$

于是

$$
\boldsymbol{J}^{\mathrm{T}}=\operatorname{diag}\left(\boldsymbol{J}_{1}^{\mathrm{T}}, \boldsymbol{J}_{2}^{\mathrm{T}}, \cdots, \boldsymbol{J}_{r}^{\mathrm{T}}\right)=\left(\boldsymbol{P}^{\mathrm{T}}\right)^{-1} \boldsymbol{A}^{\mathrm{T}} \boldsymbol{P}^{\mathrm{T}}
$$

这表明 $\boldsymbol{A}^{\mathrm{T}} \sim \boldsymbol{J}^{\mathrm{T}}$ ,所以如果能证明对于每一个 $i(i=1,2, \cdots, r)$ 都有 $\boldsymbol{J}_{i}^{\mathrm{T}} \sim \boldsymbol{J}_{i}$ 。则根据相似的传递性便知 $\boldsymbol{A} \sim \boldsymbol{A}^{\mathrm{T}}$ 。事实上,若令

$$
\boldsymbol{P}_{i}=\left[\begin{array}{cccc}
0 & \cdots & 0 & 1 \\
0 & \cdots & 1 & 0 \\
\vdots & & \vdots & \vdots \\
1 & \cdots & \cdots & 0
\end{array}\right] \quad\left(\boldsymbol{P}_{i} \text { 的阶数 }=J_{i} \text { 的阶数 }\right)
$$

则不难验证 $\boldsymbol{P}_{i}^{-1}=\boldsymbol{P}_{i}, \boldsymbol{P}_{i} \boldsymbol{J}_{i} \boldsymbol{P}_{i}=\boldsymbol{J}_{i}^{\mathrm{T}}$(证毕)

\section*{§2.4 矩阵的有理标准形}
由定理 2. 2.7 知,$n$ 阶数字矩阵 $\boldsymbol{A} \sim \boldsymbol{B}$ 的重要条件是 $\boldsymbol{A} 、 \boldsymbol{B}$ 有相同的不变因子.\\
例2.4.1 求 $n$ 阶数字矩阵

$$
C\left[\begin{array}{cccccc}
0 & 0 & 0 & \cdots & 0 & -a_{n} \\
1 & 0 & 0 & \cdots & 0 & -a_{n-1} \\
& 1 & 0 & \cdots & 0 & -a_{n-2} \\
& & & & & \vdots \\
& & & & & -a_{2} \\
& & & & 1 & -a_{1}
\end{array}\right]
$$

的不变因子.\\
解 $\lambda \boldsymbol{E}-\boldsymbol{C}=\left[\begin{array}{ccccc}\lambda & 0 & \cdots & 0 & a_{n} \\ -1 & \lambda & \cdots & 0 & a_{n-1} \\ & -1 & \ddots & & \vdots \\ & & & & a_{2} \\ & & & -1 & \lambda+a_{1}\end{array}\right]$\\
由例 2.1.5 知它的不变因子为

$$
\underbrace{1,1 \cdots, 1}_{(n-1)}, \lambda^{n}+a_{1} \lambda^{n-1}+\cdots+a_{n-1} \lambda+a_{n}
$$

定理2.4.1 设 $n$ 阶数字矩阵 $A$ 的非常数的不变因子为:$\varphi_{1}(\lambda), \varphi_{2}(\lambda), \cdots$ , $\varphi_{k}(\lambda)$ ,其中

$$
\begin{gathered}
\varphi_{i}(\lambda)=\lambda^{n_{i}}+a_{i 1} \lambda^{n_{i}-1}+a_{i 2} \lambda^{n_{i}-2}+\cdots+a_{i} n_{i-1} \lambda+a_{i} n_{i} \\
(i=1,2, \cdots, k), n_{1}+n_{2}+\cdots+n_{k}=n
\end{gathered}
$$

则

其中

$$
\begin{gathered}
\boldsymbol{A} \sim \boldsymbol{F} \\
\boldsymbol{F}=\left[\begin{array}{llll}
\boldsymbol{C}_{1} & & & \\
& \boldsymbol{C}_{2} & & \\
& & \ddots & \\
& & & \boldsymbol{C}_{k}
\end{array}\right] \\
C_{i}=\left[\begin{array}{ccccc}
0 & 0 & \cdots & 0 & -a_{i} n_{i} \\
1 & 0 & \cdots & 0 & -a_{i} n_{i-1} \\
& & \ddots & & \ddots \\
& \ddots & & 0 & -a_{i 2} \\
& & & 1 & -a_{i 1}
\end{array}\right]
\end{gathered}
$$

称 $\boldsymbol{F}$ 为 $\boldsymbol{A}$ 的有理标准形.\\
设 $\boldsymbol{P}^{-1} \boldsymbol{A P}=\boldsymbol{J}, \boldsymbol{Q}^{-1} \boldsymbol{A Q}=\boldsymbol{F}$ ,其中 $\boldsymbol{J}$ 是 $\boldsymbol{A}$ 的 Jordan 标准形, $\boldsymbol{F}$ 是 $\boldsymbol{A}$ 的有理标准形,显然 $J \sim F$ .变换矩阵 $P$ 在 $\S 2.3$ 节中已介绍,下边介绍如何求 $Q$ 。

由于 $J \sim F$ ,设 $M^{-1} F M=J$ ,于是

$$
\begin{aligned}
\boldsymbol{F} & =\boldsymbol{M} \boldsymbol{M}^{-1}=\boldsymbol{M}\left(\boldsymbol{P}^{-1} \boldsymbol{A} \boldsymbol{P}\right) \boldsymbol{M}^{-1}=\boldsymbol{M} \boldsymbol{P}^{-1} \boldsymbol{A} \boldsymbol{P} \boldsymbol{M}^{-1} \\
& =\left(\boldsymbol{M} \boldsymbol{P}^{-1}\right) \boldsymbol{A}\left(\boldsymbol{P} \boldsymbol{M}^{-1}\right) \\
& =\left(\boldsymbol{P} \boldsymbol{M}^{-1} \boldsymbol{A}\left(\boldsymbol{P} \boldsymbol{M}^{-1}\right)\right.
\end{aligned}
$$

所以, $\boldsymbol{Q}=\boldsymbol{P} \boldsymbol{M}^{-1}$ ,其中 $\boldsymbol{P}$ 是 $\boldsymbol{A} \sim \boldsymbol{J}$ 的过渡矩阵, $\boldsymbol{M}$ 是 $\boldsymbol{F} \sim \boldsymbol{J}$ 的过渡矩阵.\\
例2.4.2 求矩阵

$$
A=\left[\begin{array}{ccc}
17 & 0 & -25 \\
0 & 1 & 0 \\
9 & 0 & -13
\end{array}\right]
$$

的有理标准形 $\boldsymbol{F}$ 及变换矩阵 $\boldsymbol{Q}$ 。\\
解 由例2.3.2知

$$
\lambda E-A \simeq\left[\begin{array}{lll}
1 & & \\
& 1 & \\
& & (\lambda-1)(\lambda-2)^{2}
\end{array}\right]=\left[\begin{array}{lll}
1 & & \\
& 1 & \\
& & \lambda^{3}-5 \lambda^{2}+8 \lambda-4
\end{array}\right]
$$

因此

$$
A \sim J=\left[\begin{array}{lll}
1 & 0 & 0 \\
0 & 2 & 1 \\
0 & 0 & 2
\end{array}\right] \sim F=\left[\begin{array}{ccc}
0 & 0 & 4 \\
1 & 0 & -8 \\
0 & 1 & 5
\end{array}\right]
$$

由例2.3.1知

$$
\boldsymbol{P}=\left[\begin{array}{lll}
0 & 5 & 2 \\
1 & 0 & 0 \\
0 & 3 & 1
\end{array}\right]
$$

为了求 $\boldsymbol{Q}$ 需先求 $\boldsymbol{F} \sim \boldsymbol{J}$ 的变换矩阵 $\boldsymbol{M}$ ,即 $\boldsymbol{F} \boldsymbol{M}=\boldsymbol{M J}$\\
设 $M=\left(\beta_{1}, \beta_{2}, \beta_{3}\right)$ ,代人 $F M=M J$ 得

$$
\boldsymbol{F}=\left(\boldsymbol{\beta}_{1}, \boldsymbol{\beta}_{2}, \boldsymbol{\beta}_{3}\right)=\left(\boldsymbol{\beta}_{1}, \boldsymbol{\beta}_{2}, \boldsymbol{\beta}_{3}\right)\left[\begin{array}{lll}
1 & 0 & 0 \\
0 & 2 & 1 \\
0 & 0 & 2
\end{array}\right]
$$

比较两边得

$$
F \beta_{1}=\beta_{1}, F \beta_{2}=2 \beta_{2}, F \beta_{3}=\beta_{2}+2 \beta_{3}
$$

解之得

$$
\beta_{1}=(4,-4,1)^{\mathrm{T}}, \beta_{2}=(2,-3,1)^{\mathrm{T}}, \beta_{3}=(-1,1,0)^{\mathrm{T}}
$$

于是

$$
\begin{gathered}
M=\left(\beta_{1}, \beta_{2}, \beta_{3}\right)=\left[\begin{array}{ccc}
4 & 2 & -1 \\
-4 & -3 & 1 \\
1 & 1 & 0
\end{array}\right] \\
M^{-1}=\left[\begin{array}{ccc}
1 & 1 & 1 \\
-1 & -1 & 0 \\
1 & 2 & 4
\end{array}\right]
\end{gathered}
$$

故

$$
Q=P M^{-1}=\left[\begin{array}{lll}
0 & 5 & 2 \\
1 & 0 & 0 \\
0 & 3 & 1
\end{array}\right]\left[\begin{array}{ccc}
1 & 1 & 1 \\
-1 & -1 & 0 \\
1 & 2 & 4
\end{array}\right]=\left[\begin{array}{ccc}
-3 & -1 & 8 \\
1 & 1 & 1 \\
-2 & -1 & 4
\end{array}\right]
$$

例 2.4 .3 求矩阵

$$
A=\left[\begin{array}{ccc}
-1 & -2 & 6 \\
-1 & 0 & 3 \\
-1 & -1 & 4
\end{array}\right]
$$

的有理标准形及变换矩阵 $\boldsymbol{Q}$ .

解 由例2.3.1知

$$
\lambda \boldsymbol{E}-\boldsymbol{A} \simeq\left[\begin{array}{lll}
1 & & \\
& \lambda-1 & \\
& & (\lambda-1)^{2}
\end{array}\right]=\left[\begin{array}{lll}
1 & & \\
& \lambda-1 & \\
& & \lambda^{2}-2 \lambda+1
\end{array}\right]
$$

因此

$$
A \sim J=\left[\begin{array}{lll}
1 & 0 & 0 \\
0 & 1 & 1 \\
0 & 0 & 1
\end{array}\right] \sim F=\left[\begin{array}{ccc}
1 & 0 & 0 \\
0 & 0 & -1 \\
0 & 0 & 2
\end{array}\right]
$$

由例2.3.3知

$$
P=\left[\begin{array}{ccc}
-1 & 2 & 2 \\
1 & 1 & 0 \\
0 & 1 & 1
\end{array}\right]
$$

为了求 $\boldsymbol{Q}$ 需先求 $\boldsymbol{F} \sim \boldsymbol{J}$ 的变换矩阵 $\boldsymbol{M}$ ,即 $\boldsymbol{F} \boldsymbol{M}=\boldsymbol{M} \boldsymbol{J}$ 。\\
设 $M=\left(\beta_{1}, \beta_{2}, \beta_{3}\right)$ ,代人 $F M=M J$ 得

$$
F\left(\beta_{1}, \beta_{2}, \beta_{3}\right)=\left(\beta_{1}, \beta_{2}, \beta_{3}\right)\left[\begin{array}{lll}
1 & 0 & 0 \\
0 & 1 & 1 \\
0 & 0 & 1
\end{array}\right]
$$

比较上式两边得

$$
\boldsymbol{F} \boldsymbol{\beta}_{1}=\boldsymbol{\beta}_{1}, \boldsymbol{F} \boldsymbol{\beta}_{2}=\boldsymbol{\beta}_{2}, \boldsymbol{F} \boldsymbol{\beta}_{3}=\boldsymbol{\beta}_{2}+\boldsymbol{\beta}_{3}
$$

解之得

$$
\beta_{1}=(1,0,0)^{\mathrm{T}}, \beta_{2}=(0,-1,1)^{\mathrm{T}}, \beta_{3}=(0,1,0)^{\mathrm{T}}
$$

故

$$
\begin{gathered}
M=\left(\beta_{1}, \beta_{2}, \beta_{3}\right)=\left[\begin{array}{ccc}
1 & 0 & 0 \\
0 & -1 & 1 \\
0 & 1 & 0
\end{array}\right] \\
M^{-1}=\left[\begin{array}{lll}
1 & 0 & 0 \\
0 & 0 & 1 \\
0 & 1 & 1
\end{array}\right]
\end{gathered}
$$

于是

$$
\begin{aligned}
\boldsymbol{Q}=\boldsymbol{P M}^{-1} & =\left[\begin{array}{ccc}
-1 & 2 & 2 \\
1 & 1 & 0 \\
0 & 1 & 1
\end{array}\right]\left[\begin{array}{lll}
1 & 0 & 0 \\
0 & 0 & 1 \\
0 & 1 & 1
\end{array}\right] \\
& =\left[\begin{array}{ccc}
-1 & 2 & 4 \\
1 & 0 & 1 \\
0 & 1 & 2
\end{array}\right]
\end{aligned}
$$

例2.4.4 已知 $\boldsymbol{A}$ 的 Jordan 标准形,求 $\boldsymbol{A}$ 的有理标准形.\\
(1)$J=\left[\begin{array}{ccc}-1 & & \\ & 2 & 1 \\ & & 2\end{array}\right]$\\
(2) $\boldsymbol{J}=\left[\begin{array}{lll}1 & & \\ & 1 & 1 \\ & & 1\end{array}\right]$

解(1) $\boldsymbol{A}$ 的初等因子 $\lambda+1,(\lambda-2)^{2}$\\
故 $\boldsymbol{A}$ 的不变因子为 $1,1, \lambda^{3}-3 \lambda^{2}+4$\\
于是 $\boldsymbol{A}$ 的有理标准形为

$$
\boldsymbol{F}=\left[\begin{array}{ccc}
0 & 0 & -4 \\
1 & 0 & 0 \\
0 & 1 & 3
\end{array}\right]
$$

(2) $\boldsymbol{A}$ 的初等因子 $\lambda-1,(\lambda-1)^{2}$\\
故 $\boldsymbol{A}$ 的不变因子为 $1, \lambda-1,(\lambda-1)^{2}$\\
于是 $\boldsymbol{A}$ 的有理标准形为

$$
F=\left[\begin{array}{ccc}
1 & 0 & 0 \\
0 & 0 & -1 \\
0 & 1 & 2
\end{array}\right]
$$

\section*{习 题}
2-1 用初等变换把下列 $\lambda$-矩阵化为标准形\\
(1)$\left[\begin{array}{cc}\lambda^{3}-\lambda & 2 \lambda^{2} \\ \lambda^{2}+5 \lambda & 3 \lambda\end{array}\right]$\\
(2)$\left[\begin{array}{cc}\lambda^{2}-1 & 0 \\ 0 & (\lambda-1)^{3}\end{array}\right]$\\
(3)$\left[\begin{array}{ccc}1-\lambda & \lambda^{2} & \lambda \\ \lambda & \lambda & -\lambda \\ 1+\lambda^{2} & \lambda^{2} & \lambda\end{array}\right]$\\
(4)$\left[\begin{array}{lll}\lambda(\lambda+1) & & \\ & \lambda & \\ & & (\lambda+1)^{2}\end{array}\right]$

2-2 试证:Jordan 块

$$
\boldsymbol{J}(a)=\left[\begin{array}{lll}
a & 1 & 0 \\
0 & a & 1 \\
0 & 0 & a
\end{array}\right]
$$

相似于矩阵

$$
\left[\begin{array}{lll}
a & \varepsilon & 0 \\
0 & a & \varepsilon \\
0 & 0 & a
\end{array}\right]
$$

这里 $\varepsilon \neq 0$ 为任意实数.

2-3 已知 10 阶矩阵

$$
\begin{aligned}
\boldsymbol{A} & =\left[\begin{array}{lllll}
a & 1 & & & \\
& a & 1 & & \\
& & \ddots & \ddots & \\
& & & \ddots & 1 \\
& & & & a
\end{array}\right]_{10 \times 10} \\
\boldsymbol{B} & =\left[\begin{array}{lllll}
a & 1 & & & \\
& a & 1 & & \\
& & \ddots & \ddots & \\
& & & \ddots & 1 \\
\varepsilon & & & & a
\end{array}\right]_{10 \times 10}
\end{aligned}
$$

其中 $\varepsilon=10^{-10}$ ,证明 $A$ 不相似于 $B$ .\\
2-4 设 $A \neq 0, A^{k}=0 \quad(k \geqslant 2)$ 。证明:$A$ 不能与对角矩阵相似。\\
2-5 已知 $\boldsymbol{A}^{k}=\boldsymbol{E}$( $k$ 为正整数).证明: $\boldsymbol{A}$ 与对角矩阵相似.\\
2-6 已知 $\boldsymbol{A}^{2}=\boldsymbol{A}$ .证明: $\boldsymbol{A}$ 相似于矩阵

$$
\left[\begin{array}{llllll}
1 & & & & & \\
& \ddots & & & & \\
& & 1 & & & \\
& & & 0 & & \\
& & & & \ddots & \\
& & & & & 0
\end{array}\right]
$$

2-7 求下列矩阵的 Jordan 标准形及其变换矩阵 $\boldsymbol{P}$ .\\
(1)$\left[\begin{array}{rrr}1 & 2 & 0 \\ 0 & 2 & 0 \\ -2 & -2 & 1\end{array}\right]$\\
(2)$\left[\begin{array}{rrr}-1 & 1 & 1 \\ -5 & 21 & 17 \\ 6 & -26 & -21\end{array}\right]$\\
(3)$\left[\begin{array}{rrr}4 & 5 & -2 \\ -2 & -2 & 1 \\ -1 & -1 & 1\end{array}\right]$\\
(4)$\left[\begin{array}{rrr}3 & 0 & 8 \\ 3 & -1 & 6 \\ -2 & 0 & -5\end{array}\right]$

2-8 用求矩阵秩的方法求矩阵的 Jordan 标准形。\\
(1)$\left[\begin{array}{rrr}13 & 16 & 16 \\ -5 & -7 & -6 \\ -6 & -8 & -7\end{array}\right]$\\
(2)$\left[\begin{array}{rrr}8 & -3 & 6 \\ 3 & -2 & 0 \\ -4 & 2 & -2\end{array}\right]$\\
(3)$\left[\begin{array}{rrrr}3 & -1 & -3 & 1 \\ -1 & 3 & 1 & -3 \\ 3 & -1 & -3 & 1 \\ -1 & 3 & 1 & -3\end{array}\right]$\\
(4)$\left[\begin{array}{rrrr}4 & -1 & -1 & 0 \\ -1 & 4 & 0 & -1 \\ 1 & 0 & 2 & -1 \\ 0 & 1 & -1 & 2\end{array}\right]$

2-9 试写出 Jordan 标准形均为

$$
J=\left[\begin{array}{lll}
1 & 0 & 0 \\
0 & 2 & 1 \\
0 & 0 & 2
\end{array}\right]
$$

的两个矩阵 $\boldsymbol{A}, \boldsymbol{B}$ .\\
2-10 已知 $A$ ,求 $A^{100}$\\
(1)$A=\left[\begin{array}{lll}2 & 1 & 0 \\ 0 & 0 & 1 \\ 0 & 1 & 0\end{array}\right]$\\
(2)$A=\left[\begin{array}{ccc}-1 & 0 & 1 \\ 1 & 2 & 0 \\ -4 & 0 & 3\end{array}\right]$

2-11 已知 $n$ 阶数字矩阵 $\boldsymbol{A}$ 的初等因子为 $\lambda, \lambda^{3},(\lambda+1)^{3}$ .求 $\boldsymbol{A}$ 的不变因子, $\boldsymbol{A}$ 的 Jordan 标准形及有理标准形。

2-12 已知 $n$ 阶数字矩阵 $\boldsymbol{A}$ 的特征多项式 $|\lambda \boldsymbol{E}-\boldsymbol{A}|=\lambda^{2}(\lambda-1)(\lambda+1)^{2}$ .求 $\boldsymbol{A}$ 的不变因子、初等因子, $\boldsymbol{A}$ 的 Jordan 标准形及有理标准形。

2-13 求题 2-7矩阵的有理标准形及变换矩阵 $\boldsymbol{Q}$ 。

\section*{内积空间、正规矩阵、 Hermite 矩阵}
本章由十一节组成,分两大部分.前五节介绍内积空间及内积空间中的几个重要线性变换。后六节介绍正规矩阵与 Hermite 矩阵。

\section*{§3.1 欧氏空间、酉空间}
在线性空间中,向量之间的基本运算只有加法和数乘向量两种运算。向量的度量性质:向量长度、夹角、正交等概念在线性空间理论中没有反映,局限了线性空间理论的应用。在这一节中,借助于内积把度量概念引入到线性空间当中。

解析几何中介绍空间向量的内积(也称点积或数量积)是由向量的长度与夹角引入的,然后推出内积在直角坐标系下的计算公式。而在矩阵理论中引人向量内积是从代数观点出发的,然后引入向量的长度、夹角等几何概念。

\section*{一、欧氏空间、酉空间}
定义3.1.1 设 $V$ 是实数域 $\mathbf{R}$ 上的 $n$ 维线性空间,对 $V$ 中的任意两个向量 $\boldsymbol{\alpha}$ 、 $\boldsymbol{\beta}$ 依一确定法则对应着一个实数,这个实数称为内积,记做 $(\boldsymbol{\alpha}, \boldsymbol{\beta})$ 。并且要求内积 $(\boldsymbol{\alpha}, \boldsymbol{\beta})$ 运算满足下列四个条件:\\
(1)$(\boldsymbol{\alpha}, \boldsymbol{\beta})=(\boldsymbol{\beta}, \boldsymbol{\alpha}) ;$\\
(2)$(k \boldsymbol{\alpha}, \boldsymbol{\beta})=k(\boldsymbol{\alpha}, \boldsymbol{\beta}), k$ 为任意实数;\\
(3)$(\boldsymbol{\alpha}+\boldsymbol{\beta}, \boldsymbol{\nu})=(\boldsymbol{\alpha}, \boldsymbol{\nu})+(\boldsymbol{\beta}, \boldsymbol{\nu})$ ;\\
(4)$(\boldsymbol{\alpha}, \boldsymbol{\alpha}) \geqslant 0$ ,当且仅当 $\boldsymbol{\alpha}=0$ 时 $(\boldsymbol{\alpha}, \boldsymbol{\alpha})=0$ .\\
这里 $\boldsymbol{\alpha}, \boldsymbol{\beta}$ 和 $\boldsymbol{\nu}$ 是 $V$ 中任意向量.称定义有这样内积的 $n$ 维线性空间为 $n$ 维欧几里得空间,简称 $n$ 维欧氏空间。

例3.1.1 设 $R^{n}$ 是 $n$ 维实(列)向量空间,若

令

$$
\begin{aligned}
& \boldsymbol{\alpha}=\left(a_{1}, a_{2}, \cdots, a_{n}\right)^{\mathrm{T}}, \boldsymbol{\beta}=\left(b_{1}, b_{2}, \cdots, b_{n}\right)^{\mathrm{T}} \\
&(\boldsymbol{\alpha}, \boldsymbol{\beta})=\boldsymbol{\alpha}^{\mathrm{T}} \boldsymbol{\beta}=\boldsymbol{\beta}^{\mathrm{T}} \boldsymbol{\alpha} \\
&=a_{1} b_{1}+a_{2} b_{2}+\cdots+a_{n} b_{n}
\end{aligned}
$$

容易验证,所规定的 $(\boldsymbol{\alpha}, \boldsymbol{\beta})$ 满足定义3.1.1中的四个条件.因此在这样定义内积后 $R^{n}$ 成为欧氏空间。

注:今后在讨论 $R^{n}$ 时都用例3.1.1中引进的内积。\\
对于同一个线性空间,可以给予不同的内积,因而可得到不同构造的欧氏空间。现以 $R^{2}$ 为例。

例 3.1.2 设在 $R^{2}$ 中对向量 $\boldsymbol{\alpha}=\left(a_{1}, a_{2}\right)^{\mathrm{T}}$ 和 $\boldsymbol{\beta}=\left(b_{1}, b_{2}\right)^{\mathrm{T}}$ 规定内积为

$$
(\boldsymbol{\alpha}, \boldsymbol{\beta})=2 a_{1} b_{1}+a_{1} b_{2}+a_{2} b_{1}+a_{2} b_{2}
$$

试证:$R^{2}$ 是欧氏空间.\\
解 这只需验证 $(\boldsymbol{\alpha}, \boldsymbol{\beta})$ 满足内积的四个条件即可。

$$
\begin{aligned}
(\boldsymbol{\beta}, \boldsymbol{\alpha}) & =2 b_{1} a_{1}+b_{1} a_{2}+b_{2} a_{1}+b_{2} a_{2} \\
& =2 a_{1} b_{1}+a_{2} b_{1}+a_{1} b_{2}+a_{2} b_{2}=(\boldsymbol{\alpha}, \boldsymbol{\beta}) \\
(k \boldsymbol{\alpha}, \boldsymbol{\beta}) & =2 k a_{1} b_{1}+k a_{1} b_{2}+k a_{2} b_{1}+k a_{2} b_{2} \\
& =k\left(2 a_{1} b_{1}+a_{1} b_{2}+a_{2} b_{1}+a_{2} b_{2}\right)=k(\boldsymbol{\alpha}, \boldsymbol{\beta})
\end{aligned}
$$

设 $\boldsymbol{\nu}=\left(c_{1}, c_{2}\right)^{\mathrm{T}}$ ,则

$$
\begin{aligned}
(\boldsymbol{\alpha}+\boldsymbol{\beta}, \boldsymbol{\nu})= & 2\left(a_{1}+b_{1}\right) c_{1}+\left(a_{1}+b_{1}\right) c_{2}+\left(a_{2}+b_{2}\right) c_{1}+ \\
& \left(a_{2}+b_{2}\right) c_{2} \\
= & 2 a_{1} c_{1}+a_{1} c_{2}+a_{2} c_{1}+a_{2} c_{2}+2 b_{1} c_{1}+b_{1} c_{2}+ \\
& b_{2} c_{1}+b_{2} c_{2} \\
= & (\boldsymbol{\alpha}, \boldsymbol{\nu})+(\boldsymbol{\beta}, \boldsymbol{\nu}) \\
(\boldsymbol{\alpha}, \boldsymbol{\alpha})= & 2 a_{1}^{2}+a_{1} a_{2}+a_{2} a_{1}+a_{2}^{2} \\
= & \left(a_{1}+a_{2}\right)^{2}+a_{1}^{2} \geqslant 0
\end{aligned}
$$

等式成立的充要条件是 $a_{1}=a_{2}=0$ ,即 $\alpha=0$ .\\
例3.1.3 设在 $n^{2}$ 维空间 $R^{n \times n}$ 中对向量( $n$ 阶矩阵) $\boldsymbol{A}, \boldsymbol{B}$ 规定内积为

$$
(A, B)=\operatorname{tr}\left(A^{\mathrm{T}} B\right) \quad A, B \in R^{n \times n}
$$

试证:$R^{n \times n}$ 是欧氏空间。\\
解 设 $\boldsymbol{A}=\left(a_{i j}\right)_{n \times n}, \boldsymbol{B}=\left(b_{i j}\right)_{n \times n}$ ,不难验证

$$
\operatorname{tr}\left(\boldsymbol{A}^{\mathrm{T}} \boldsymbol{B}\right)=\sum_{i, j=1}^{n} a_{i j} b_{i j}
$$

显然,$(\boldsymbol{A}, \boldsymbol{B})=(\boldsymbol{B}, \boldsymbol{A})$ ;

$$
\begin{aligned}
(k \boldsymbol{A}, \boldsymbol{B}) & =\operatorname{tr}\left[(k \boldsymbol{A})^{\mathrm{T}} \boldsymbol{B}\right]=\operatorname{tr}\left[k \boldsymbol{A}^{\mathrm{T}} \boldsymbol{B}\right] \\
& =k \operatorname{tr}\left[\boldsymbol{A}^{\mathrm{T}} \boldsymbol{B}\right]=k(\boldsymbol{A}, \boldsymbol{B}) \\
(\boldsymbol{A}+\boldsymbol{B}, \boldsymbol{C}) & =\operatorname{tr}\left[(\boldsymbol{A}+\boldsymbol{B})^{\mathrm{T}} \boldsymbol{C}\right]=\operatorname{tr}\left[\left(\boldsymbol{A}^{\mathrm{T}}+\boldsymbol{B}^{\mathrm{T}}\right) \boldsymbol{C}\right] \\
& =\operatorname{tr}\left(\boldsymbol{A}^{\mathrm{T}} \boldsymbol{C}+\boldsymbol{B}^{\mathrm{T}} \boldsymbol{C}\right)=\operatorname{tr}\left(\boldsymbol{A}^{\mathrm{T}} \boldsymbol{C}\right)+\operatorname{tr}\left(\boldsymbol{B}^{\mathrm{T}} \boldsymbol{C}\right) \\
& =(\boldsymbol{A}, \boldsymbol{C})+(\boldsymbol{B}, \boldsymbol{C})
\end{aligned}
$$

$(\boldsymbol{A}, \boldsymbol{A})=\operatorname{tr}\left(\boldsymbol{A}^{\mathrm{T}} \boldsymbol{A}\right)=\sum_{i, j=1}^{n} a_{i j}^{2} \geqslant 0$ ,等号成立当且仅当 $a_{i j}=0(\forall i, j)$ ,即 $A=0$ .\\
所以 $R^{n \times n}$ 是欧氏空间。\\
例3.1.4 用 $C[a, b]$ 表示闭区间 $[a, b]$ 上的所有实值连续函数构成的实线性空间,对任意 $f(x) 、 g(x) \in C[a, b]$ ,规定

$$
(f, g)=\int_{b}^{a} f(x) g(x) \mathrm{d} x
$$

容易验证,这样规定的 $(f, g)$ 是 $C[a, b]$ 上的一个内积,从而 $C[a, b]$ 成为一个欧氏空间。

例3.1.5 设 $A$ 为 $n$ 阶正定矩阵,对于 $R^{n}$ 中任意两个列向量 $\boldsymbol{X}, \boldsymbol{Y}$ .规定

$$
(\boldsymbol{X}, \boldsymbol{Y})=\boldsymbol{X}^{\mathrm{T}} \boldsymbol{A} \boldsymbol{Y}
$$

容易验证 $(\boldsymbol{X}, \boldsymbol{Y})$ 是 $R^{n}$ 上的一个内积,于是 $R^{n}$ 成为一个欧氏空间。\\
定义3.1.2 设 $V$ 是复数域 $\mathbf{C}$ 上的 $n$ 维线性空间,对 $V$ 中的任意两个向量 $\boldsymbol{\alpha}$ , $\boldsymbol{\beta}$ 依一确定法则对应着一个复数,这个复数称为内积,记做 $(\boldsymbol{\alpha}, \boldsymbol{\beta})$ 。并且要求内积 $(\boldsymbol{\alpha}, \boldsymbol{\beta})$ 运算满足下列四个条件:\\
(1)$(\boldsymbol{\alpha}, \boldsymbol{\beta})=(\boldsymbol{\beta}, \boldsymbol{\alpha})$ 其中 $(\boldsymbol{\beta}, \boldsymbol{\alpha})$ 是 $(\boldsymbol{\beta}, \boldsymbol{\alpha})$ 的共轭复数;\\
(2)$(k \boldsymbol{\alpha}, \boldsymbol{\beta})=k(\boldsymbol{\alpha}, \boldsymbol{\beta}), k$ 为任意复数;\\
(3)$(\boldsymbol{\alpha}+\boldsymbol{\beta}, \boldsymbol{\nu})=(\boldsymbol{\alpha}, \boldsymbol{\nu})+(\boldsymbol{\beta}, \boldsymbol{\nu})$ ;\\
(4)$(\boldsymbol{\alpha}, \boldsymbol{\alpha})$ 为非负实数,当且仅当 $\boldsymbol{\alpha}=0$ 时 $(\boldsymbol{\alpha}, \boldsymbol{\alpha})=0$ .\\
这里 $\boldsymbol{\alpha}, \boldsymbol{\beta}$ 和 $\boldsymbol{\nu}$ 是 $V$ 中任意向量.称定义有这样内积的 $n$ 维线性空间 $V$ 为 $n$ 维复欧氏空间,或简称 $V$ 为 $n$ 维酉空间。欧氏空间与酉空间通称为内积空间。

注:当在复数域 $\mathbf{C}$ 上的线性空间定义内积时,不能采用实数域 $\mathbf{R}$ 上线性空间中内积的定义方式,否则会出现矛盾。例如,$(\boldsymbol{\alpha}, \boldsymbol{\alpha})>0$ ,若在复数域 $\mathbf{C}$ 上仍采用实数域上内积则有 $(\mathrm{i} \boldsymbol{\alpha}, \mathrm{i} \boldsymbol{\alpha})=\mathrm{i}^{2}(\boldsymbol{\alpha}, \boldsymbol{\alpha})=-(\boldsymbol{\alpha}, \boldsymbol{\alpha})$ 。这样 $(\boldsymbol{\alpha}, \boldsymbol{\alpha})<0$ ,矛盾!为了确保内积的非负性,复内积的四个条件必须作定义3.1.2那样的修改。

例3.1.6 设 $C^{n}$ 是 $n$ 维复(列)向量空间,若

命

$$
\begin{gathered}
\boldsymbol{\alpha}=\left(a_{1}, a_{2}, \cdots, a_{n}\right)^{\mathrm{T}}, \quad \boldsymbol{\beta}=\left(b_{1}, b_{2}, \cdots, b_{n}\right)^{\mathrm{T}} \\
(\boldsymbol{\alpha}, \boldsymbol{\beta})=(\boldsymbol{\beta})^{\mathrm{T}} \boldsymbol{\alpha}=a_{1} \bar{b}_{1}+a_{2} \bar{b}_{2}+\cdots+a_{n} \bar{b}_{n}=\boldsymbol{\beta}^{\mathrm{H}} \boldsymbol{\alpha}
\end{gathered}
$$

容易验证,所规定的 $(\boldsymbol{\alpha}, \boldsymbol{\beta})$ 满足定义3.1.2中的四个条件,因此 $C^{n}$ 成为一个酉空间

例3.1.7 在 $C^{n \times n}$ 中,对任意 $A, B \in C^{n \times n}$ 定义

$$
(A, B)=\operatorname{tr}\left(A \bar{B}^{\mathrm{T}}\right)
$$

容易验证 $(\boldsymbol{A}, \boldsymbol{B})$ 是 $C^{n \times n}$ 的一个内积,从而 $C^{n \times n}$ 成为一个酉空间。 $\operatorname{tr}(\boldsymbol{A})$ 表示 $\boldsymbol{A}$ 的迹,即 $\operatorname{tr}(\boldsymbol{A})$ 是 $\boldsymbol{A}$ 的主对角元素之和。

\section*{二、酉(欧氏)空间的性质}
由于欧氏空间可以作为酉空间的特例,因此下面着重对酉空间讨论,有时更泛

指内积空间,它包括酉空间与欧氏空间。\\
根据定义3.1.1,可以得到欧氏空间中内积的性质:\\
(1)$(\boldsymbol{\alpha}, k \boldsymbol{\beta})=k(\boldsymbol{\alpha}, \boldsymbol{\beta})$\\
(2)$(\boldsymbol{\alpha}, \boldsymbol{\beta}+\boldsymbol{\nu})=(\boldsymbol{\alpha}, \boldsymbol{\beta})+(\boldsymbol{\alpha}, \boldsymbol{\nu})$\\
(3)$\left(\sum_{i=1}^{s} k_{i} \boldsymbol{\alpha}_{i}, \boldsymbol{\beta}\right)=\sum_{i=1}^{s} k_{i}\left(\boldsymbol{\alpha}_{i}, \boldsymbol{\beta}\right)$\\
(4)$\left(\boldsymbol{\alpha}, \sum_{i=1}^{s} k_{i} \boldsymbol{\beta}_{i}\right)=\sum_{i=1}^{s} k_{i}\left(\boldsymbol{\alpha}, \boldsymbol{\beta}_{i}\right)$\\
根据定义3.1.2可以得到西空间中内积的性质:\\
(1)$(\boldsymbol{\alpha}, k \boldsymbol{\beta})=\bar{k}(\boldsymbol{\alpha}, \boldsymbol{\beta})$\\
(2)$(\boldsymbol{\alpha}, \boldsymbol{\beta}+\boldsymbol{\nu})=(\boldsymbol{\alpha}, \boldsymbol{\beta})+(\boldsymbol{\alpha}, \boldsymbol{\nu})$\\
(3)$\left(\sum_{i=1}^{s} k_{i} \boldsymbol{\alpha}_{i}, \boldsymbol{\beta}\right)=\sum_{i=1}^{s} k_{i}\left(\boldsymbol{\alpha}_{i}, \boldsymbol{\beta}\right)$\\
(4)$\left(\boldsymbol{\alpha}, \sum_{i=1}^{s} k_{i} \boldsymbol{\beta}_{i}\right)=\sum_{i=1}^{s} \bar{k}_{i}\left(\boldsymbol{\alpha}, \boldsymbol{\beta}_{i}\right)$\\
设 $V$ 是 $n$ 维西空间,$\left\{\alpha_{i}\right\}$ 为其一组基,对于 $V$ 中任何两个向量

$$
\boldsymbol{\alpha}=\sum_{i=1}^{n} x_{i} \boldsymbol{\alpha}_{i}, \quad \boldsymbol{\beta}=\sum_{j=1}^{n} y_{j} \boldsymbol{\alpha}_{j}
$$

则 $\boldsymbol{\alpha}$ 与 $\boldsymbol{\beta}$ 的内积

令

$$
\begin{aligned}
(\boldsymbol{\alpha}, \boldsymbol{\beta}) & =\left(\sum_{i=1}^{n} x_{i} \boldsymbol{\alpha}_{i}, \sum_{j=1}^{n} y_{j} \boldsymbol{\alpha}_{j}\right)=\sum_{i, j=1}^{n} x_{i} \bar{y}_{j}\left(\boldsymbol{\alpha}_{i}, \boldsymbol{\alpha}_{j}\right) \\
g_{i j} & =\left(\boldsymbol{\alpha}_{i}, \boldsymbol{\alpha}_{j}\right) \quad(i, j=1,2, \cdots, n)
\end{aligned}
$$

$$
\boldsymbol{G}=\left[\begin{array}{cccc}
g_{11} & g_{12} & \cdots & g_{1 n} \\
g_{21} & g_{22} & \cdots & g_{2 n} \\
\vdots & \vdots & & \vdots \\
g_{n 1} & g_{n 2} & \cdots & g_{n n}
\end{array}\right]
$$

称 $\boldsymbol{G}$ 为基 $\left\{\boldsymbol{\alpha}_{i}\right\}$ 的度量矩阵。显然 $g_{j i}=\bar{g}_{i j}$ ,故

$$
\overline{\left(\boldsymbol{G}^{\mathrm{T}}\right)}=\boldsymbol{G} \quad(\boldsymbol{\alpha}, \boldsymbol{\beta})=\boldsymbol{X}^{\mathrm{T}} \boldsymbol{G} \overline{\boldsymbol{Y}}
$$

定义 3.1.3 设 $\boldsymbol{A} \in C^{m \times n}$ ,用 $\overline{\boldsymbol{A}}$ 表示以 $\boldsymbol{A}$ 的元素的共轭复数为元素组成的矩阵。命

$$
\boldsymbol{A}^{\mathrm{H}}=(\overline{\boldsymbol{A}})^{\mathrm{T}}
$$

则称 $\boldsymbol{A}^{\mathbf{H}}$ 为 $\boldsymbol{A}$ 的复共轭转置矩阵。\\
不难验证复共轭转置有下列性质:\\
(1) $\boldsymbol{A}^{\mathrm{H}}=(\overline{\boldsymbol{A}})^{\mathrm{T}}$\\
(2)$(A+B)^{\mathrm{H}}=A^{\mathrm{H}}+B^{\mathrm{H}}$\\
(3)$(k A)^{\mathrm{H}}=\bar{k} A^{\mathrm{H}}$\\
(4)$(A B)^{\mathrm{H}}=B^{\mathrm{H}} A^{\mathrm{H}}$\\
(5)$\left(A^{\mathrm{H}}\right)^{\mathrm{H}}=A$\\
(6)$\left(A^{\mathrm{H}}\right)^{-1}=\left(A^{-1}\right)^{\mathrm{H}}$(当 $A$ 可逆时)\\
定义3.1.4 设 $\boldsymbol{A} \in C^{n \times n}$ ,若 $\boldsymbol{A}^{\mathrm{H}}=\boldsymbol{A}$ ,则称 $\boldsymbol{A}$ 为Hermite 矩阵。若 $\boldsymbol{A}^{\mathrm{H}}=-\boldsymbol{A}$ ,则称 $\boldsymbol{A}$ 为反 Hermite 矩阵。

例如

$$
A=\left[\begin{array}{ccc}
3 & 1+\mathrm{i} & 3+\mathrm{i} \\
1-\mathrm{i} & 0 & 5-2 \mathrm{i} \\
3-\mathrm{i} & 5+2 \mathrm{i} & 9
\end{array}\right], B=\left[\begin{array}{ccc}
0 & 1-\mathrm{i} & 2+3 \mathrm{i} \\
1+\mathrm{i} & 0 & 5+\mathrm{i} \\
2-3 \mathrm{i} & 5-\mathrm{i} & 0
\end{array}\right]
$$

是 Hermite 矩阵。\\
例如

$$
\begin{aligned}
& A=\left[\begin{array}{ccc}
5 \mathrm{i} & -1-\mathrm{i} & 4+3 \mathrm{i} \\
1-\mathrm{i} & 3 \mathrm{i} & -2 \\
-4+3 \mathrm{i} & 2 & 5 \mathrm{i}
\end{array}\right] \\
& B=\left[\begin{array}{ccc}
-3 \mathrm{i} & -1+\mathrm{i} & 2-3 \mathrm{i} \\
1+\mathrm{i} & 2 \mathrm{i} & -1-\mathrm{i} \\
-2-3 \mathrm{i} & 1-\mathrm{i} & 0
\end{array}\right]
\end{aligned}
$$

是反 Hermite 矩阵。\\
不难证明:(1)$A^{H}=A \Leftrightarrow a_{i j}=\bar{a}_{j i} \Leftrightarrow \operatorname{Re}\left(a_{i j}\right)=\operatorname{Re}\left(a_{j i}\right), \operatorname{Im}\left(a_{i j}\right)=-\operatorname{Im}\left(a_{j i}\right)$ . $(i, j=1,2, \cdots, n)$\\
(2)$A^{H}=-A \Leftrightarrow a_{i j}=-\bar{a}_{j i} \Leftrightarrow \operatorname{Re}\left(a_{i j}\right)=-\operatorname{Re}\left(a_{j i}\right), \operatorname{Im}\left(a_{i j}\right)= \operatorname{Im}\left(a_{j i}\right)(i, j=1,2, \cdots, n), \operatorname{Re} a$ 表示复数 $a$ 的实部, $\operatorname{Im} a$ 表示复数 $a$ 的虚部。

显然,实对称矩阵是实 Hermite 矩阵。酉空间的度量矩阵是 Hermite 矩阵,欧氏空间的度量矩阵是实对称矩阵。

对于线性空间不同的基,它们的度量矩阵是不同的,它们之间的关系由下述定理给出:

定理3.1.1 设 $\boldsymbol{a}_{1}, \boldsymbol{a}_{2}, \cdots, \boldsymbol{\alpha}_{n}$ 和 $\boldsymbol{a}_{1}^{\prime}, \boldsymbol{a}_{2}^{\prime}, \cdots, \boldsymbol{a}_{n}^{\prime}$ 为线性空间 $V$ 的两个基,$A$ 、 $\boldsymbol{B}$ 分别为其度量矩阵,基的过渡矩阵为 $\boldsymbol{P}$ ,即

$$
\left(\boldsymbol{\alpha}_{1}^{\prime}, \boldsymbol{\alpha}_{2}^{\prime}, \cdots, \boldsymbol{\alpha}_{n}^{\prime}\right)=\left(\boldsymbol{\alpha}_{1}, \boldsymbol{\alpha}_{2}, \cdots, \boldsymbol{\alpha}_{n}\right) \boldsymbol{P}
$$

则两个度量矩阵 $\boldsymbol{A}$ 与 $\boldsymbol{B}$ 满足

$$
\boldsymbol{B}^{\mathrm{T}}=\boldsymbol{P}^{\mathrm{H}} A^{\mathrm{T}} \boldsymbol{P}
$$

证明:若 $\alpha 、 \beta \in V$ ,且设

$$
\begin{aligned}
& \boldsymbol{\alpha}=\left(\boldsymbol{\alpha}_{1}, \boldsymbol{\alpha}_{2}, \cdots, \boldsymbol{\alpha}_{n}\right) \boldsymbol{X}=\left(\boldsymbol{\alpha}_{1}^{\prime}, \boldsymbol{\alpha}_{2}^{\prime}, \cdots, \boldsymbol{\alpha}_{n}^{\prime}\right) \boldsymbol{X}^{\prime} \\
& \boldsymbol{\beta}=\left(\boldsymbol{\alpha}_{1}, \boldsymbol{\alpha}_{2}, \cdots, \boldsymbol{\alpha}_{n}\right) \boldsymbol{Y}=\left(\boldsymbol{\alpha}_{1}^{\prime}, \boldsymbol{\alpha}_{2}^{\prime}, \cdots, \boldsymbol{\alpha}_{n}^{\prime}\right) \boldsymbol{Y}^{\prime}
\end{aligned}
$$

由坐标变换公式知

$$
X=P X^{\prime}, \quad Y=P Y^{\prime}
$$

于是有

$$
\begin{aligned}
(\boldsymbol{\alpha}, \boldsymbol{\beta}) & =\boldsymbol{X}^{\mathrm{T}} \boldsymbol{A} \overline{\boldsymbol{Y}}=\left(\boldsymbol{P} \boldsymbol{X}^{\prime}\right)^{\mathrm{T}} \boldsymbol{A}\left(\overline{\boldsymbol{P} \boldsymbol{Y}^{\prime}}\right) \\
& =\boldsymbol{X}^{\prime \mathrm{T}} \boldsymbol{P}^{\mathrm{T}} \boldsymbol{A} \overline{\boldsymbol{P}} \overline{\boldsymbol{Y}}^{\prime}
\end{aligned}
$$

又

$$
(\boldsymbol{\alpha}, \boldsymbol{\beta})=\boldsymbol{X}^{{ }^{\mathrm{T}} \boldsymbol{B}} \overline{\boldsymbol{Y}^{\prime}}
$$

故

$$
\boldsymbol{B}=\boldsymbol{P}^{\mathrm{T}} \boldsymbol{A} \overline{\boldsymbol{P}} \quad \text { 即 } \quad \boldsymbol{B}^{\mathrm{T}}=\boldsymbol{P}^{\mathrm{H}} \boldsymbol{A}^{\mathrm{T}} \boldsymbol{P}
$$

\section*{三、酉(欧氏)空间的度量}
现在把几何上的向量长度、夹角、垂直等概念推广到酉空间上。\\
定义3.1.5 设 $V$ 为西(欧氏)空间,向量 $\boldsymbol{\alpha} \in V$ 的长度(模)定义为

$$
\|\boldsymbol{\alpha}\|=\sqrt{(\boldsymbol{\alpha}, \boldsymbol{\alpha})}
$$

例如,$C^{n}$(或 $R^{n}$ )上向量 $\boldsymbol{\alpha}=\left(a_{1}, a_{2}, \cdots, a_{n}\right)^{\mathrm{T}}$ 的长度为

$$
\|\boldsymbol{\alpha}\|=\sqrt{\sum_{i=1}^{n}\left|a_{i}\right|^{2}}
$$

定理3.1.2 设 $V$ 是酉(欧氏)空间,则向量长度 $\|\boldsymbol{\alpha}\|$ 具有以下性质:\\
(1)$\|\boldsymbol{\alpha}\| \geqslant 0$ ,当且仅当 $\alpha=0$ 时,$\|\boldsymbol{\alpha}\|=0$ (非负性)\\
(2)$\|k \boldsymbol{\alpha}\|=|k|\|\boldsymbol{\alpha}\|, k$ 为任意数\\
(3)$\|\boldsymbol{\alpha}+\boldsymbol{\beta}\| \leqslant\|\boldsymbol{\alpha}\|+\|\boldsymbol{\beta}\|$ (三角不等式)\\
(4)$|(\boldsymbol{\alpha}, \boldsymbol{\beta})| \leqslant\|\boldsymbol{\alpha}\|\|\boldsymbol{\beta}\|$ (Cauchy-Schwarz 不等式)\\
性质(1)和(2)请读者自己证明。下面证明两个不等式。\\
先证 Cauchy-Schwarz 不等式。\\
若 $\beta=0$ ,不等式显然成立。设 $\beta \neq 0$ ,则

$$
\begin{aligned}
0 & \leqslant\|\boldsymbol{\alpha}-k \boldsymbol{\beta}\|^{2}=(\boldsymbol{\alpha}-k \boldsymbol{\beta}, \boldsymbol{\alpha}-k \boldsymbol{\beta}) \\
& =(\boldsymbol{\alpha}, \boldsymbol{\alpha})-\bar{k}(\boldsymbol{\alpha}, \boldsymbol{\beta})-k(\boldsymbol{\beta}, \boldsymbol{\alpha})+k \bar{k}(\boldsymbol{\beta}, \boldsymbol{\beta})
\end{aligned}
$$

在上式中令

$$
k=\frac{(\boldsymbol{\alpha}, \boldsymbol{\beta})}{(\boldsymbol{\beta}, \boldsymbol{\beta})}
$$

则

$$
0 \leqslant(\boldsymbol{\alpha}, \boldsymbol{\alpha})-\frac{(\boldsymbol{\alpha}, \boldsymbol{\beta})(\boldsymbol{\beta}, \boldsymbol{\alpha})}{(\boldsymbol{\beta}, \boldsymbol{\beta})}=\|\boldsymbol{\alpha}\|^{2}-\frac{1(\boldsymbol{\alpha}, \boldsymbol{\beta}) 1^{2}}{\|\boldsymbol{\beta}\|^{2}}
$$

即

$$
|(\boldsymbol{\alpha}, \boldsymbol{\beta})| \leqslant\|\boldsymbol{\alpha}\|\|\boldsymbol{\beta}\|
$$

现证三角不等式

$$
\begin{aligned}
\|\boldsymbol{\alpha}+\boldsymbol{\beta}\|^{2} & =(\boldsymbol{\alpha}+\boldsymbol{\beta}, \boldsymbol{\alpha}+\boldsymbol{\beta}) \\
& =(\boldsymbol{\alpha}, \boldsymbol{\alpha})+(\boldsymbol{\alpha}, \boldsymbol{\beta})+(\boldsymbol{\beta}, \boldsymbol{\alpha})+(\boldsymbol{\beta}, \boldsymbol{\beta}) \\
& =\|\boldsymbol{\alpha}\|^{2}+2 \operatorname{Re}(\boldsymbol{\alpha}, \boldsymbol{\beta})+\|\boldsymbol{\beta}\|^{2} \\
& \leqslant\|\boldsymbol{\alpha}\|^{2}+21(\boldsymbol{\alpha}, \boldsymbol{\beta}) I+\|\boldsymbol{\beta}\|^{2}
\end{aligned}
$$

$$
\begin{aligned}
& \leqslant\|\boldsymbol{\alpha}\|^{2}+2\|\boldsymbol{\alpha}\|\|\boldsymbol{\beta}\|+\|\boldsymbol{\beta}\|^{2} \\
& =(\|\boldsymbol{\alpha}\|+\|\boldsymbol{\beta}\|)^{2}
\end{aligned}
$$

其中 $\operatorname{Re}(\boldsymbol{\alpha}, \boldsymbol{\beta})$ 表示( $\boldsymbol{\alpha}, \boldsymbol{\beta}$ )的实部。\\
Cauchy-Schwarz 不等式有十分重要的应用。例如它在 $R^{n}$ 中的形式即为

$$
\begin{aligned}
& \left|a_{1} b_{1}+a_{2} b_{2}+\cdots+a_{n} b_{n}\right| \\
& \leqslant \sqrt{a_{1}^{2}+a_{2}^{2}+\cdots+a_{n}^{2}} \cdot \sqrt{b_{1}^{2}+b_{2}^{2}+\cdots+b_{n}^{2}}
\end{aligned}
$$

在欧氏空间中内积总是实数,因此 Cauchy-Schwarz 不等式可以写成

$$
-1 \leqslant \frac{(\boldsymbol{\alpha}, \boldsymbol{\beta})}{\|\boldsymbol{\alpha}\|\|\boldsymbol{\beta}\|} \leqslant 1
$$

因此在欧氏空间中向量 $\boldsymbol{\alpha}$ 和 $\boldsymbol{\beta}$ 的夹角自然可定义为

$$
\cos (\boldsymbol{\alpha}, \boldsymbol{\beta})=\frac{1(\boldsymbol{\alpha}, \boldsymbol{\beta}) 1}{\|\boldsymbol{\alpha}\|\|\boldsymbol{\beta}\|}
$$

向量 $\boldsymbol{\alpha}$ 和 $\boldsymbol{\beta}$ 之间的距离定义为

$$
d(\boldsymbol{\alpha}, \boldsymbol{\beta})=\|\boldsymbol{\alpha}-\boldsymbol{\beta}\|
$$

若向量 $\boldsymbol{\alpha}$ 的长度 $\|\boldsymbol{\alpha}\|=1$ ,便说 $\boldsymbol{\alpha}$ 是单位向量,对于任何一个向量(非零) $\boldsymbol{\alpha}$ ,向量 $\frac{\boldsymbol{\alpha}}{\|\boldsymbol{\alpha}\|}$ 是单位向量,称由 $\boldsymbol{\alpha}$ 得到 $\frac{\boldsymbol{\alpha}}{\|\boldsymbol{\alpha}\|}$ 的过程为单位化.

\section*{§3.2 标准正交基、Schmidt 方法}
解析几何中当两个向量垂直时,它们的内积为零。于是在内积空间中有如下定义。

定义3.2.1 若向量 $\boldsymbol{\alpha}$ 和 $\boldsymbol{\beta}$ 的内积 $(\boldsymbol{\alpha}, \boldsymbol{\beta})=0$ ,则说 $\boldsymbol{\alpha}$ 与 $\boldsymbol{\beta}$ 正交,记之为 $\boldsymbol{\alpha} \perp \boldsymbol{\beta}$ .\\
若不含零向量的向量组 $\left\{\boldsymbol{\alpha}_{i}\right\}$ 内的向量两两正交,则说向量组 $\left\{\boldsymbol{\alpha}_{i}\right\}$ 是正交向量组

若一个正交向量组内的任一个向量是单位向量,则说向量组是标准正交向量组

根据定义不难证明:\\
(1)向量组 $\left\{\boldsymbol{\alpha}_{i}\right\}$ 是正交向量组的充要条件是:

$$
\left(\boldsymbol{\alpha}_{i}, \boldsymbol{\alpha}_{j}\right)=0 \quad(i \neq j)
$$

(2)向量组 $\left\{\boldsymbol{\alpha}_{i}\right\}$ 是标准正交向量组的充要条件是:

$$
\left(\alpha_{i}, \alpha_{j}\right)=\delta_{i j}= \begin{cases}1 & \text { 当 } i=j \text { 时 } \\ 0 & \text { 当 } i \neq j \text { 时 }\end{cases}
$$

(3)零向量和每个向量都正交;反之,与空间每个向量都正交的向量必是零向量。

定理3.2.1 正交向量组(不含零向量)是线性无关向量组.

证明 设 $\alpha_{1}, \alpha_{2}, \cdots, \alpha_{s}$ 是正交向量组,若

$$
k_{1} \boldsymbol{\alpha}_{1}+k_{2} \boldsymbol{\alpha}_{2}+\cdots+k_{s} \boldsymbol{\alpha}_{s}=0
$$

则对 $\forall \alpha_{j}(j=1,2, \cdots, s)$ 都有

$$
\left(k_{1} \boldsymbol{\alpha}_{1}+k_{2} \boldsymbol{\alpha}_{2}+\cdots+k_{s} \boldsymbol{\alpha}_{s}, \boldsymbol{\alpha}_{j}\right)=\sum_{i=1}^{s} k_{i}\left(\boldsymbol{\alpha}_{i}, \boldsymbol{\alpha}_{j}\right)=0
$$

根据 $\left(\alpha_{i}, \alpha_{j}\right)=0$( $i \neq j$ 时),简化上式得

$$
k_{j}\left(\alpha_{j}, \alpha_{j}\right)=0
$$

由于 $\left(\boldsymbol{\alpha}_{j}, \boldsymbol{\alpha}_{j}\right) \neq 0$ ,故 $k_{j}=0(j=1,2, \cdots, s)$ 。于是 $\boldsymbol{\alpha}_{1}, \boldsymbol{\alpha}_{2}, \cdots, \boldsymbol{\alpha}_{s}$ 线性无关。\\
这个定理说明,在 $n$ 维西空间(或欧氏空间)中,两两正交的非零向量不能超过 $n$ 个,它的几何意义是清楚的。例如,在平面上不存在三个两两垂直的非零向量,在普通空间中不存在四个两两垂直的非零向量。

定义3.2.2 在 $n$ 维内积空间中,由 $n$ 个正交向量组成的基称为正交基。由 $n$个标准正交向量组成的基称为标准正交基。

显然, $\boldsymbol{\alpha}_{1}, \boldsymbol{\alpha}_{2}, \cdots, \boldsymbol{\alpha}_{n}$ 是标准正交基的充要条件是 $\left(\boldsymbol{\alpha}_{i}, \boldsymbol{\alpha}_{j}\right)=\boldsymbol{\delta}_{i j}(i, j=1,2, \cdots, n)$ 。即它的度量矩阵 $\boldsymbol{G}$ 是单位矩阵。

例如,$e_{1}=(1,0, \cdots, 0)^{\mathrm{T}}, e_{2}=(0,1,0, \cdots, 0)^{\mathrm{T}}, \cdots, e_{\mathrm{n}}=(0, \cdots, 0,1)^{\mathrm{T}}$ 是标准正交基。但标准正交基不是唯一的。

每一个 $n$ 维线性空间的基由 $n$ 个线性无关的向量组成。如果这个线性空间是酉空间(或欧氏空间),我们总能构造一个标准正交基,这是西空间(或欧氏空间)的基本定理。Schmidt 方法就是从一组线性无关的向量出发构造一组标准正交向量的一种方法.介绍如下:

设 $\boldsymbol{\alpha}_{1}, \boldsymbol{\alpha}_{2}, \cdots, \boldsymbol{\alpha}_{r}$ 是 $n$ 维西空间(或欧氏空间)中 $r$ 个线性无关的列向量,现求由这 $r$ 个列向量生成的 $r$ 维线性子空间 $\operatorname{span}\left\{\boldsymbol{\alpha}_{1}, \boldsymbol{\alpha}_{2}, \cdots, \boldsymbol{\alpha}_{r}\right\}$ 中的一个标准正交基。分两步进行:\\
(1)正交化\\
命

$$
\begin{aligned}
\boldsymbol{\beta}_{1} & =\boldsymbol{\alpha}_{1} \\
\boldsymbol{\beta}_{2} & =\boldsymbol{\alpha}_{2}-\frac{\left(\boldsymbol{\alpha}_{2}, \boldsymbol{\beta}_{1}\right)}{\left(\boldsymbol{\beta}_{1}, \boldsymbol{\beta}_{1}\right)} \boldsymbol{\beta}_{1} \\
\boldsymbol{\beta}_{3} & =\boldsymbol{\alpha}_{3}-\frac{\left(\boldsymbol{\alpha}_{3}, \boldsymbol{\beta}_{1}\right)}{\left(\boldsymbol{\beta}_{1}, \boldsymbol{\beta}_{1}\right)} \boldsymbol{\beta}_{1}-\frac{\left(\boldsymbol{\alpha}_{3}, \boldsymbol{\beta}_{2}\right)}{\left(\boldsymbol{\beta}_{2}, \boldsymbol{\beta}_{2}\right)} \boldsymbol{\beta}_{2} \\
\vdots & \vdots \quad \vdots \quad \vdots \quad \vdots \quad \\
\boldsymbol{\beta}_{r} & =\boldsymbol{\alpha}_{r}-\frac{\left(\boldsymbol{\alpha}_{r}, \boldsymbol{\beta}_{1}\right)}{\left(\boldsymbol{\beta}_{1}, \boldsymbol{\beta}_{1}\right)} \boldsymbol{\beta}_{1}-\frac{\left(\boldsymbol{\alpha}_{r}, \boldsymbol{\beta}_{2}\right)}{\left(\boldsymbol{\beta}_{2}, \boldsymbol{\beta}_{2}\right)} \boldsymbol{\beta}_{2}-\cdots-\frac{\left(\boldsymbol{\alpha}_{r}, \boldsymbol{\beta}_{r-1}\right)}{\left(\boldsymbol{\beta}_{r-1}, \boldsymbol{\beta}_{r-1}\right)} \boldsymbol{\beta}_{r-1}
\end{aligned}
$$

容易验证 $\boldsymbol{\beta}_{1}, \boldsymbol{\beta}_{2}, \cdots, \boldsymbol{\beta}_{r}$ 是正交向量组.\\
(2)单位化

命

$$
\nu_{1}=\frac{\boldsymbol{\beta}_{1}}{\left\|\boldsymbol{\beta}_{1}\right\|}, \nu_{2}=\frac{\boldsymbol{\beta}_{2}}{\left\|\boldsymbol{\beta}_{2}\right\|}, \cdots, \nu_{r}=\frac{\boldsymbol{\beta}_{r}}{\left\|\boldsymbol{\beta}_{r}\right\|}
$$

显然,$\nu_{1}, \nu_{2}, \cdots, \nu_{r}$ 是标准正交向量组,它是子空间 $\operatorname{span}\left\{\boldsymbol{\alpha}_{1}, \boldsymbol{\alpha}_{2}, \cdots, \boldsymbol{\alpha}_{r}\right\}$ 的一个标准正交基。

定理3.2.2 从 $r$ 维内积空间的任一组基 $\boldsymbol{\alpha}_{1}, \boldsymbol{\alpha}_{2}, \cdots, \boldsymbol{\alpha}_{r}$ 出发,可通过 Schmidt正交化方法构造出一个标准正交基。

例3.2.1 在空间 $R^{4}$ 中,设

$$
\begin{aligned}
& \boldsymbol{\alpha}_{1}=(1,-1,1,-1)^{\mathrm{T}}, \boldsymbol{\alpha}_{2}=(5,1,1,1)^{\mathrm{T}}, \\
& \boldsymbol{\alpha}_{3}=(-3,-3,1,-3)^{\mathrm{T}}
\end{aligned}
$$

求 $\operatorname{span}\left\{\alpha_{1}, \alpha_{2}, \alpha_{3}\right\}$ 的一个标准正交基。\\
解 应用 Schmidt 正交化方法得到

$$
\begin{aligned}
\boldsymbol{\beta}_{1} & =\boldsymbol{\alpha}_{1}=(1,-1,1,-1)^{\mathrm{T}} \\
\boldsymbol{\beta}_{2} & =\boldsymbol{\alpha}_{2}-\frac{\left(\boldsymbol{\alpha}_{2}, \boldsymbol{\beta}_{1}\right)}{\left(\boldsymbol{\beta}_{1}, \boldsymbol{\beta}_{1}\right)} \boldsymbol{\beta}_{1}=\boldsymbol{\alpha}_{2}-\boldsymbol{\beta}_{1}=(4,2,0,2)^{\mathrm{T}} \\
\boldsymbol{\beta}_{3} & =\boldsymbol{\alpha}_{3}-\frac{\left(\boldsymbol{\alpha}_{3}, \boldsymbol{\beta}_{1}\right)}{\left(\boldsymbol{\beta}_{1}, \boldsymbol{\beta}_{1}\right)} \boldsymbol{\beta}_{1}-\frac{\left(\boldsymbol{\alpha}_{3}, \boldsymbol{\beta}_{2}\right)}{\left(\boldsymbol{\beta}_{2}, \boldsymbol{\beta}_{2}\right)} \boldsymbol{\beta}_{2}=\boldsymbol{\alpha}_{3}-\boldsymbol{\beta}_{1}+\boldsymbol{\beta}_{2} \\
& =(0,0,0,0)^{\mathrm{T}}
\end{aligned}
$$

因为 $\boldsymbol{\beta}_{3}=0$ ,故 $\boldsymbol{\alpha}_{1}, \boldsymbol{\alpha}_{2}, \boldsymbol{\alpha}_{3}$ 线性相关,容易验证 $\boldsymbol{\alpha}_{1}, \boldsymbol{\alpha}_{2}$ 线性无关,因此 $\operatorname{span}\left\{\boldsymbol{\alpha}_{1}, \boldsymbol{\alpha}_{2}\right.$ , $\left.\boldsymbol{\alpha}_{3}\right\}=\operatorname{span}\left\{\boldsymbol{\alpha}_{1}, \boldsymbol{\alpha}_{2}\right\}$ ,把 $\boldsymbol{\beta}_{1}, \boldsymbol{\beta}_{2}$ 单位化后,便得 $\operatorname{span}\left\{\boldsymbol{\alpha}_{1}, \boldsymbol{\alpha}_{2}\right\}$ 的一个标准正交基

$$
\begin{gathered}
\nu_{1}=\frac{\beta_{1}}{\left\|\beta_{1}\right\|}=\left(\frac{1}{2},-\frac{1}{2}, \frac{1}{2},-\frac{1}{2}\right)^{\mathrm{T}} \\
\nu_{2}=\frac{\beta_{2}}{\left\|\beta_{2}\right\|}=\left(\frac{2}{\sqrt{6}}, \frac{1}{\sqrt{6}}, 0, \frac{1}{\sqrt{6}}\right)^{\mathrm{T}}
\end{gathered}
$$

例3.2.2 已知

$$
\begin{aligned}
& \boldsymbol{\alpha}_{1}=(1,-1, \mathrm{i}, \mathrm{i})^{\mathrm{T}}, \quad \boldsymbol{\alpha}_{2}=(-1,1, \mathrm{i}, \mathrm{i})^{\mathrm{T}}, \\
& \boldsymbol{\alpha}_{3}=(1,1, \mathrm{i}, \mathrm{i})^{\mathrm{T}}
\end{aligned}
$$

求 $\operatorname{span}\left\{\boldsymbol{\alpha}_{1}, \boldsymbol{\alpha}_{2}, \boldsymbol{\alpha}_{3}\right\}$ 的一个标准正交基。\\
解 命

$$
\begin{aligned}
\boldsymbol{\beta}_{1} & =\boldsymbol{\alpha}_{1}=(1,-1, \mathrm{i}, \mathrm{i})^{\mathrm{T}} \\
\boldsymbol{\beta}_{2} & =\boldsymbol{\alpha}_{2}-\frac{\left(\boldsymbol{\alpha}_{2}, \boldsymbol{\beta}_{1}\right)}{\left(\boldsymbol{\beta}_{1}, \boldsymbol{\beta}_{1}\right)} \boldsymbol{\beta}_{1} \\
& =\boldsymbol{\alpha}_{2}=(-1,1, \mathrm{i}, \mathrm{i})^{\mathrm{T}} \\
\boldsymbol{\beta}_{3} & =\boldsymbol{\alpha}_{3}-\frac{\left(\boldsymbol{\alpha}_{3}, \boldsymbol{\beta}_{1}\right)}{\left(\boldsymbol{\beta}_{1}, \boldsymbol{\beta}_{1}\right)} \boldsymbol{\beta}_{1}-\frac{\left(\boldsymbol{\alpha}_{3}, \boldsymbol{\beta}_{2}\right)}{\left(\boldsymbol{\beta}_{2}, \boldsymbol{\beta}_{2}\right)} \boldsymbol{\beta}_{2} \\
& =(1,1,0,0)^{\mathrm{T}}
\end{aligned}
$$

把 $\boldsymbol{\beta}_{1}, \boldsymbol{\beta}_{2}, \boldsymbol{\beta}_{3}$ 单位化得

$$
\begin{aligned}
& \nu_{1}=\frac{\boldsymbol{\beta}_{1}}{\left|\boldsymbol{\beta}_{1}\right|}=\left(\frac{1}{2},-\frac{1}{2}, \frac{\mathrm{i}}{2}, \frac{\mathrm{i}}{2}\right)^{\mathrm{T}} \\
& \nu_{2}=\frac{\boldsymbol{\beta}_{2}}{\left|\boldsymbol{\beta}_{2}\right|}=\left(-\frac{1}{2}, \frac{1}{2}, \frac{\mathrm{i}}{2}, \frac{\mathrm{i}}{2}\right)^{\mathrm{T}} \\
& \nu_{3}=\frac{\boldsymbol{\beta}_{3}}{\left|\boldsymbol{\beta}_{3}\right|}=\left(\frac{\sqrt{2}}{2}, \frac{\sqrt{2}}{2}, 0,0\right)^{\mathrm{T}}
\end{aligned}
$$

则 $\nu_{1}, \nu_{2}, \nu_{3}$ 为所求之基。

\section*{§3.3 酉变换、正交变换}
\section*{一、酉矩阵、正交矩阵}
定义3.3.1 若 $n$ 阶复矩阵 $\boldsymbol{A}$ 满足

$$
\boldsymbol{A}^{\mathrm{H}} \boldsymbol{A}=\boldsymbol{A} \boldsymbol{A}^{\mathrm{H}}=\boldsymbol{E}
$$

则称 $\boldsymbol{A}$ 是酉矩阵,记之为 $\boldsymbol{A} \in U^{n \times n}$ 。\\
根据定义容易验证:若 $\boldsymbol{A}, \boldsymbol{B} \in U^{n \times n}$ ,则\\
(1) $\boldsymbol{A}^{-1}=\boldsymbol{A}^{\mathrm{H}} \in U^{n \times n}$\\
(2)$|\operatorname{det} A|=1$\\
(3) $\boldsymbol{A}^{\mathrm{T}} \in U^{n \times n}$\\
(4) $\boldsymbol{A} \boldsymbol{B}, \boldsymbol{B} \boldsymbol{A} \in U^{n \times n}$\\
例如矩阵

$$
\begin{aligned}
& \boldsymbol{A}=\left[\begin{array}{cc}
\frac{-1-\mathrm{i}}{2} & \frac{-1-\mathrm{i}}{2} \\
\frac{1+\mathrm{i}}{2} & \frac{-1-\mathrm{i}}{2}
\end{array}\right], \boldsymbol{B}=\left[\begin{array}{cc}
\frac{4}{5}-\frac{1}{5} \mathrm{i} & \frac{2}{5}+\frac{2}{5} \mathrm{i} \\
\frac{2}{5}+\frac{2}{5} \mathrm{i} & \frac{1}{5}-\frac{4}{5} \mathrm{i}
\end{array}\right] \\
& \boldsymbol{C}=\left[\begin{array}{ccc}
\frac{2-2 \mathrm{i}}{4} & \frac{-\sqrt{2}-\sqrt{2} \mathrm{i}}{4} & \frac{\sqrt{2}+\sqrt{2} \mathrm{i}}{4} \\
\frac{-\sqrt{2}-\sqrt{2} \mathrm{i}}{4} & \frac{1-3 \mathrm{i}}{4} & \frac{-1-\mathrm{i}}{4} \\
\frac{\sqrt{2}+\sqrt{2} \mathrm{i}}{4} & \frac{-1-\mathrm{i}}{4} & \frac{1-3 \mathrm{i}}{4}
\end{array}\right] \\
& \boldsymbol{D}=\left[\begin{array}{cc}
\cos \theta & \operatorname{isin} \theta \\
\operatorname{isin} \theta & \cos \theta
\end{array}\right]
\end{aligned}
$$

都是酉矩阵。\\
若 $n$ 阶实矩阵 $\boldsymbol{A}$ 满足

$$
\boldsymbol{A}^{\mathrm{T}} \boldsymbol{A}=\boldsymbol{A} \boldsymbol{A}^{\mathrm{T}}=\boldsymbol{E}
$$

则称 $\boldsymbol{A}$ 是正交矩阵,记之为 $\boldsymbol{A} \in \boldsymbol{E}^{n \times n}$\\
根据定义容易验证:若 $\boldsymbol{A}, \boldsymbol{B} \in \boldsymbol{E}^{n \times n}$ ,则\\
(1) $\boldsymbol{A}^{-1}=\boldsymbol{A}^{\mathrm{T}} \in \boldsymbol{E}^{n \times n}$\\
(2) $\operatorname{det} A= \pm 1$\\
(3) $\boldsymbol{A} \boldsymbol{B}, \boldsymbol{B} \boldsymbol{A} \in \boldsymbol{E}^{n \times n}$\\
例如矩阵

$$
\begin{gathered}
A=\left[\begin{array}{ccc}
1 & 0 & 0 \\
0 & \sin \theta & -\cos \theta \\
0 & \cos \theta & \sin \theta
\end{array}\right], B=\left[\begin{array}{ccc}
2 / 3 & 2 / 3 & -1 / 3 \\
2 / 3 & -1 / 3 & 2 / 3 \\
-1 / 3 & 2 / 3 & 2 / 3
\end{array}\right] \\
C=\left[\begin{array}{ccc}
\frac{1}{\sqrt{3}} & \frac{1}{\sqrt{2}} & \frac{1}{\sqrt{6}} \\
\frac{1}{\sqrt{3}} & -\frac{1}{\sqrt{2}} & \frac{1}{\sqrt{6}} \\
\frac{1}{\sqrt{3}} & 0 & -\frac{2}{\sqrt{6}}
\end{array}\right]
\end{gathered}
$$

都是正交矩阵。\\
定理 3.3.1 设 $\boldsymbol{A} \in C^{n \times n}$ ,则 $\boldsymbol{A}$ 是酉矩阵(正交矩阵)的充要条件是 $\boldsymbol{A}$ 的 $n$ 个列(或行)向量是标准正交向量组。

证明 设 $A=\left[\alpha_{1}, \alpha_{2}, \cdots, \alpha_{n}\right]$ ,则

$$
\boldsymbol{A}^{\mathrm{H}}=\left[\begin{array}{c}
\alpha_{1}^{\mathrm{H}} \\
\alpha_{2}^{\mathrm{H}} \\
\vdots \\
\alpha_{n}^{\mathrm{H}}
\end{array}\right] .
$$

若 $\boldsymbol{A}$ 是西矩阵,则 $\boldsymbol{A}^{\mathbf{H}} \boldsymbol{A}=\boldsymbol{E}$ ,于是

$$
\left[\begin{array}{c}
\boldsymbol{\alpha}_{1}^{\mathbf{H}} \\
\boldsymbol{\alpha}_{2}^{\mathbf{H}} \\
\vdots \\
\boldsymbol{\alpha}_{n}^{\mathbf{H}}
\end{array}\right]\left[\boldsymbol{\alpha}_{1} \cdots \boldsymbol{\alpha}_{n}\right]=\boldsymbol{E}
$$

此即

$$
\left[\begin{array}{cccc}
\alpha_{1}^{\mathrm{H}} \alpha_{1} & \alpha_{1}^{\mathrm{H}} \alpha_{2} & \cdots & \alpha_{1}^{\mathrm{H}} \alpha_{n} \\
\alpha_{2}^{\mathrm{H}} \alpha_{1} & \alpha_{2}^{\mathrm{H}} \alpha_{2} & \cdots & \alpha_{2}^{\mathrm{H}} \alpha_{n} \\
\vdots & \vdots & \vdots & \vdots \\
\alpha_{n}^{\mathrm{H}} \alpha_{1} & \alpha_{n}^{\mathrm{H}} \alpha_{2} & \cdots & \alpha_{n}^{\mathrm{H}} \alpha_{n}
\end{array}\right]=\left[\begin{array}{llll}
1 & & & \\
& 1 & & \\
& & \ddots & \\
& & & 1
\end{array}\right]
$$

比较上式两端得

$$
\alpha_{i}^{\mathrm{H}} \alpha_{j}=\delta_{i j} \quad(i, j=1,2, \cdots, n)
$$

所以列向量组 $\alpha_{1}, \alpha_{2}, \cdots, \alpha_{n}$ 是标准正交向量组。\\
反之,若列向量组 $\alpha_{1}, \cdots, \alpha_{n}$ 是标准正交向量组,故 $\alpha_{i}^{\mathrm{H}} \alpha_{j}=\delta_{i j}(i, j=1,2, \cdots$ , $n$ )于是

$$
\left[\begin{array}{c}
\alpha_{1}^{\mathrm{H}} \\
\alpha_{2}^{\mathrm{H}} \\
\vdots \\
\alpha_{n}^{\mathrm{H}}
\end{array}\right]\left[\alpha_{1} \cdots \alpha_{n}\right]=E
$$

此即

$$
\boldsymbol{A}^{\mathrm{H}} \boldsymbol{A}=\boldsymbol{E}
$$

$\boldsymbol{A}$ 是酉矩阵。\\
类似的方法可证 $\boldsymbol{A}$ 的行向量组是标准正交向量组。\\
定理3.3.2 标准正交基到标准正交基的过渡矩阵是酉矩阵(证明请读者完成)。

\section*{二、西变换、正交变换}
在物理学及力学中常常用到酉变换(也称正交变换)。例如在通常的三维空间中绕原点转动或对称变换等,都是重要的西变换。

定义3.3.2 设 $V$ 是 $n$ 维酉空间,$\sigma$ 是 $V$ 的线性变换,若 $\forall \alpha, \beta \in V$ 都有

$$
(\sigma(\alpha), \sigma(\beta))=(\alpha, \beta)
$$

则称 $\sigma$ 是 $V$ 的酉变换.\\
设 $V$ 是 $n$ 维欧氏空间,若线性变换 $\sigma$ 满足 $\forall \alpha, \beta \in V$ ,都有

$$
(\sigma(\alpha), \sigma(\beta))=(\alpha, \beta)
$$

则称 $\sigma$ 是 $V$ 的正交变换.\\
定理3.3.3 设 $\sigma$ 是酉空间(或欧氏空间)$V$ 的线性变换,则下列命题等价:\\
(1)$\sigma$ 是酉变换(或正交变换);\\
(2)$\|\boldsymbol{\sigma}(\boldsymbol{\alpha})\|=\|\boldsymbol{\alpha}\| \quad \forall \boldsymbol{\alpha} \in V$ ;\\
(3)$\sigma$ 将 $V$ 的标准正交基变到标准正交基;\\
(4)酉变换(或正交变换)在标准正交基下的矩阵表示是酉矩阵(或正交矩阵)。

命题(2)说明酉变换也可称为等距变换。因为

$$
\begin{aligned}
d(\alpha, \beta) & =\|\alpha-\beta\|=\|\sigma(\alpha-\beta)\|=\|\sigma(\alpha)-\sigma(\beta)\| \\
& =d(\sigma(\alpha), \sigma(\beta))
\end{aligned}
$$

即向量 $\boldsymbol{\alpha}, \boldsymbol{\beta}$ 之间的距离在变换 $\boldsymbol{\sigma}$ 下保持不变.\\
证明 $(1) \Rightarrow(2) \quad$ 显然。\\
(2)$\Rightarrow$(1)由(2)有

$$
\begin{gathered}
(\sigma(\alpha+\beta), \sigma(\alpha+\beta))=(\alpha+\beta, \alpha+\beta) \\
(\sigma(\alpha+i \beta), \sigma(\alpha+i \beta))=(\alpha+i \beta, \alpha+i \beta) \quad(i=\sqrt{-1})
\end{gathered}
$$

根据 $\sigma$ 是线性变换与内积性质展开上两式得

$$
\begin{aligned}
& (\sigma(\boldsymbol{\alpha}), \sigma(\boldsymbol{\beta}))+(\sigma(\boldsymbol{\beta}), \sigma(\boldsymbol{\alpha}))=(\boldsymbol{\alpha}, \boldsymbol{\beta})+(\boldsymbol{\beta}, \boldsymbol{\alpha}) \\
& (\sigma(\boldsymbol{\alpha}), \sigma(\boldsymbol{\beta}))-(\boldsymbol{\sigma}(\boldsymbol{\beta}), \sigma(\boldsymbol{\alpha}))=(\boldsymbol{\alpha}, \boldsymbol{\beta})-(\boldsymbol{\beta}, \boldsymbol{\alpha})
\end{aligned}
$$

相加该两式得

$$
(\boldsymbol{\sigma}(\boldsymbol{\alpha}), \boldsymbol{\sigma}(\boldsymbol{\beta}))=(\boldsymbol{\alpha}, \boldsymbol{\beta})
$$

(1)$\Rightarrow$(3)设 $\boldsymbol{\alpha}_{1}, \boldsymbol{\alpha}_{2}, \cdots, \boldsymbol{\alpha}_{n}$ 是 $V$ 的标准正交基,故

$$
\left(\alpha_{i}, \alpha_{j}\right)=\delta_{i j}
$$

若 $\sigma$ 是酉变换,则

$$
\left(\sigma\left(\alpha_{i}\right), \sigma\left(\alpha_{j}\right)\right)=\left(\alpha_{i}, \alpha_{j}\right)=\delta_{i j}
$$

故 $\sigma\left(\alpha_{1}\right), \sigma\left(\alpha_{2}\right), \cdots, \sigma\left(\alpha_{n}\right)$ 是 $V$ 的标准正交基。\\
(3)$\Rightarrow$(1)设 $\boldsymbol{\alpha}_{1}, \boldsymbol{\alpha}_{2}, \cdots, \boldsymbol{\alpha}_{n}$ 与 $\boldsymbol{\sigma}\left(\boldsymbol{\alpha}_{1}\right), \boldsymbol{\sigma}\left(\boldsymbol{\alpha}_{2}\right), \cdots, \boldsymbol{\sigma}\left(\boldsymbol{\alpha}_{n}\right)$ 都是 $V$ 的标准正交基,$\forall \boldsymbol{\alpha}, \boldsymbol{\beta} \in V$ 且

则

$$
\begin{gathered}
\boldsymbol{\alpha}=x_{1} \boldsymbol{\alpha}_{1}+x_{2} \boldsymbol{\alpha}_{2}+\cdots+x_{n} \boldsymbol{\alpha}_{n} \\
\boldsymbol{\beta}=y_{1} \boldsymbol{\alpha}_{1}+y_{2} \boldsymbol{\alpha}_{2}+\cdots+y_{n} \boldsymbol{\alpha}_{n} \\
(\boldsymbol{\sigma}(\boldsymbol{\alpha}), \boldsymbol{\sigma}(\boldsymbol{\beta}))=x_{1} \bar{y}_{1}+x_{2} \bar{y}_{2}+\cdots+x_{n} \bar{y}_{n}=(\boldsymbol{\alpha}, \boldsymbol{\beta})
\end{gathered}
$$

(3)$\Rightarrow$(4)设 $\boldsymbol{\alpha}_{1}, \boldsymbol{\alpha}_{2}, \cdots, \boldsymbol{\alpha}_{n}$ 与 $\boldsymbol{\sigma}\left(\boldsymbol{\alpha}_{1}\right), \sigma\left(\boldsymbol{\alpha}_{2}\right), \cdots, \sigma\left(\boldsymbol{\alpha}_{n}\right)$ 是 $V$ 的标准正交基,$n$ 阶矩阵 $\boldsymbol{A}=\left(a_{i j}\right)$ 是 $\sigma$ 在基 $\boldsymbol{\alpha}_{1}, \boldsymbol{\alpha}_{2}, \cdots, \boldsymbol{\alpha}_{n}$ 下的矩阵表示,则

$$
\left(\boldsymbol{\sigma}\left(\boldsymbol{\alpha}_{1}\right), \boldsymbol{\sigma}\left(\boldsymbol{\alpha}_{2}\right), \cdots, \boldsymbol{\sigma}\left(\boldsymbol{\alpha}_{n}\right)\right)=\left(\boldsymbol{\alpha}_{1}, \boldsymbol{\alpha}_{2}, \cdots, \boldsymbol{\alpha}_{n}\right) A
$$

且 $\forall i, j(i, j=1,2, \cdots, n)$

$$
\begin{aligned}
& \sigma\left(\boldsymbol{\alpha}_{i}\right)=a_{1 i} \boldsymbol{\alpha}_{1}+a_{2 i} \boldsymbol{\alpha}_{2}+\cdots+a_{n i} \boldsymbol{\alpha}_{n} \\
& \sigma\left(\boldsymbol{\alpha}_{j}\right)=a_{1 j} \boldsymbol{\alpha}_{1}+a_{2 j} \boldsymbol{\alpha}_{2}+\cdots+a_{n j} \boldsymbol{\alpha}_{n}
\end{aligned}
$$

于是

$$
\begin{aligned}
\delta_{i j} & =\left(\sigma\left(\boldsymbol{\alpha}_{i}\right), \sigma\left(\boldsymbol{\alpha}_{j}\right)\right)=\left(\sum_{k=1}^{n} a_{k i} \boldsymbol{\alpha}_{k}, \sum_{h=1}^{n} a_{h j} \boldsymbol{\alpha}_{h}\right) \\
& =\sum_{k=1}^{n} \sum_{h=1}^{n} a_{k i} \bar{a}_{h j}\left(\boldsymbol{\alpha}_{k}, \boldsymbol{\alpha}_{h}\right) \\
& =\sum_{k=1}^{n} \sum_{h=1}^{n} a_{k i} \bar{a}_{h j} \boldsymbol{\delta}_{k h} \\
& =\sum_{k=1}^{n} a_{k i} \bar{a}_{k j}
\end{aligned}
$$

此即 $\boldsymbol{A}$ 的列向量是标准正交向量组, $\boldsymbol{A}$ 为西矩阵(或正交矩阵)。\\
(4)$\Rightarrow$(3)显然.\\
例 3.3.1 设 $\alpha \in C^{n}$ ,且 $\alpha^{\mathrm{H}} \alpha=1$ ,若

$$
H=E_{n}-2 \alpha \alpha^{\mathrm{H}} \in C^{n \times n}
$$

则 $\boldsymbol{H}$ 是酉矩阵.\\
解

$$
\begin{aligned}
\boldsymbol{H}^{\mathrm{H}} \boldsymbol{H} & =\left(\boldsymbol{E}_{n}-2 \boldsymbol{\alpha} \boldsymbol{\alpha}^{\mathrm{H}}\right)^{\mathrm{H}}\left(\boldsymbol{E}_{n}-2 \boldsymbol{\alpha} \boldsymbol{\alpha}^{\mathrm{H}}\right) \\
& =\left(\boldsymbol{E}_{n}-2 \boldsymbol{\alpha} \boldsymbol{\alpha}^{\mathrm{H}}\right)\left(\boldsymbol{E}_{n}-2 \boldsymbol{\alpha} \boldsymbol{\alpha}^{\mathrm{H}}\right) \\
& =\boldsymbol{E}_{n}-4 \boldsymbol{\alpha} \boldsymbol{\alpha}^{\mathrm{H}}+4 \boldsymbol{\alpha} \boldsymbol{\alpha}^{\mathrm{H}} \boldsymbol{\alpha} \boldsymbol{\alpha}^{\mathrm{H}}=\boldsymbol{E}_{n}
\end{aligned}
$$

故 $\boldsymbol{H}$ 是酉矩阵。因此 $\boldsymbol{H}$ 是酉空间中西变换在标准正交基下的矩阵表示。它所代表的西变换称为豪斯何尔德(Householder)镜像变换。

例3.3.2 试证

$$
A=\left[\begin{array}{cccccccc}
1 & & & & & & & \\
& \ddots & & & & & & \\
& & 1 & & & & & \\
& & \cos \theta & & & \sin \theta & & \\
& & & 1 & & & & \\
& & & \ddots & & & & \\
& & & & 1 & & & \\
& & -\sin \theta & & & \cos \theta & & \\
& & & & & & 1 & \\
& & & & & & & \ddots
\end{array}\right]_{n \times n}
$$

是正交矩阵。\\
解 易知 $\boldsymbol{A}^{\mathrm{T}} \boldsymbol{A}=\boldsymbol{E}_{n}$ ,故 $\boldsymbol{A}$ 是正交矩阵。该矩阵所代表的正交变换称为吉文斯 (Givens)变换。

例3.3.3 2 阶矩阵

$$
A=\left[\begin{array}{rr}
\cos \theta & -\sin \theta \\
\sin \theta & \cos \theta
\end{array}\right]
$$

是正交矩阵,它表示平面上的绕坐标原点的旋转变换\\
3 阶矩阵

$$
A=\left[\begin{array}{ccr}
1 & 0 & 0 \\
0 & \cos \theta & -\sin \theta \\
0 & \sin \theta & \cos \theta
\end{array}\right]
$$

是正交矩阵,它表示三维空间绕 $x$ 轴的旋转变换。

\section*{§3.4 幂等矩阵、正交投影}
\section*{一、基等矩阵、投影变换}
定义3.4.1 设 $A \in C^{n \times n}$ ,若


\begin{equation*}
\boldsymbol{A}^{2}=\boldsymbol{A} \tag{3.4.1}
\end{equation*}


则称 $A$ 是軍等矩阵。\\
例如,形如

$$
A=\left[\begin{array}{cc}
E_{r} & M \\
0 & 0
\end{array}\right] \in C^{n \times n}, \quad M \in C^{(n-r) \times r}
$$

是幂等矩阵。\\
定理3.4.1 秩为 $r$ 的 $n$ 阶矩阵 $\boldsymbol{A}$ 是幂等矩阵的充要条件是存在 $\boldsymbol{P} \in C_{n}^{n \times n}$ ,使得

\[
\boldsymbol{P}^{-1} \boldsymbol{A} \boldsymbol{P}=\left[\begin{array}{ll}
\boldsymbol{E}_{r} &  \tag{3.4.2}\\
& 0
\end{array}\right]
\]

作为练习证明留给读者.且 $\boldsymbol{A}$ 的特征值非零即 1 .\\
推论: $\operatorname{rank} \boldsymbol{A}=\operatorname{tr} \boldsymbol{A}$ ,且 $\boldsymbol{A}$ 的特征值非零即 1 。\\
证明 由式(3.4.2)可证。\\
定理3.4.2 设 $A^{2}=A$ ,则\\
(1) $\boldsymbol{A}^{\mathrm{T}}, \boldsymbol{A}^{\mathrm{H}}, \boldsymbol{E}-\boldsymbol{A}, \boldsymbol{E}-\boldsymbol{A}^{\mathrm{T}}, \boldsymbol{E}-\boldsymbol{A}^{\mathrm{H}}$ 是幂等矩阵。\\
(2) $\boldsymbol{A}(\boldsymbol{E}-\boldsymbol{A})=(\boldsymbol{E}-\boldsymbol{A}) \boldsymbol{A}=0$\\
(3)$N(\boldsymbol{A})=R(\boldsymbol{E}-\boldsymbol{A}), N(\boldsymbol{E}-\boldsymbol{A})=R(\boldsymbol{A})$\\
(4) $\boldsymbol{A} x=0 \Leftrightarrow x \in R(\boldsymbol{E}-\boldsymbol{A}), A x=x \Leftrightarrow x \in R(\boldsymbol{A})$\\
(5) $\boldsymbol{C}^{n}=R(\boldsymbol{A}) \oplus N(\boldsymbol{A})$\\
证明:(1),(2)由定义 $\boldsymbol{A}^{2}=\boldsymbol{A}$ 立即可知.\\
(3)设 $x \in N(A)$ ,则 $A x=0$\\
于是

$$
(\boldsymbol{E}-\boldsymbol{A}) x=x-\boldsymbol{A} x=x
$$

这说明 $x \in R(\boldsymbol{E}-\boldsymbol{A})$ ,故 $N(\boldsymbol{A}) \subseteq R(\boldsymbol{E}-\boldsymbol{A})$\\
另一方面,若 $x \in R(\boldsymbol{E}-\boldsymbol{A})$ ,即存在 $y \in \boldsymbol{C}^{n}$ ,使得 $(\boldsymbol{E}-\boldsymbol{A}) y=x$ ,故 $\boldsymbol{A}(\boldsymbol{E}-\boldsymbol{A}) y= \boldsymbol{A} x$ ,由(2)知,$A x=0$ ,于是 $x \in N(\boldsymbol{A})$ ,故 $R(\boldsymbol{E}-\boldsymbol{A}) \subseteq N(\boldsymbol{A})$ ,所以 $N(\boldsymbol{A})=R(\boldsymbol{E}-\boldsymbol{A})$ 。

由于 $\boldsymbol{E}-\boldsymbol{A}$ 也是幂等矩阵,所以由刚才得到的等式得

$$
N(\boldsymbol{E}-\boldsymbol{A})=R(\boldsymbol{E}-(\boldsymbol{E}-\boldsymbol{A}))=R(\boldsymbol{A})
$$

(4)由(3)的第一式知 $x \in N(\boldsymbol{A}) \Leftrightarrow x \in R(\boldsymbol{E}-\boldsymbol{A})$ ,即 $\boldsymbol{A} x=0 \Leftrightarrow x \in R(\boldsymbol{E}-\boldsymbol{A})$\\
由(3)的第二式知 $x \in N(\boldsymbol{E}-\boldsymbol{A}) \Leftrightarrow x \in R(\boldsymbol{A})$ ,即 $(\boldsymbol{E}-\boldsymbol{A}) x=0$ 即 $\boldsymbol{A} x=x \Leftrightarrow x \in R(\boldsymbol{A})$ .\\
(5)先证:$C^{n}=R(A)+N(A)$ .\\
对于 $\boldsymbol{C}^{n}$ 中任何向量 $x$ 都有 $\boldsymbol{A} x \in R(\boldsymbol{A}) \subseteq \boldsymbol{C}^{n}$ ,\\
命 $x=A x+\delta$ ,其中 $\delta \in C^{n}$ 则 $A x=A^{2} X+A \delta=A x+A \delta$ ,故 $A \delta=0$ .于是 $\delta \in N(\boldsymbol{A})$ ,所以 $\boldsymbol{C}^{n}=R(\boldsymbol{A})+N(\boldsymbol{A})$ ,

取 $x \in R(\boldsymbol{A}) \cap N(\boldsymbol{A})$ .由 $x \in R(\boldsymbol{A})$ ,故由(4)知 $\boldsymbol{A} x=x$ 由 $x \in N(\boldsymbol{A})$ ,故 $\boldsymbol{A} x=0$综合两式知 $x=0$

于是

$$
C^{n}=R(A) \oplus N(A)
$$

在平面解析几何中,若向量 $\boldsymbol{\alpha}$ 沿 $\boldsymbol{v}$ 到 $\boldsymbol{u}$ 上的投影为 $x, \alpha$ 沿 $u$ 到 $v$ 上的投影为 $y$ ,则

$$
\alpha=x+y
$$

若命变换 $\sigma: \quad \sigma(\alpha)=\sigma(x+y)=x$\\
显然,这样的变换是线性变换,且 $\sigma^{2}=\sigma$ ,若把 $\sigma$ 限制\\
\includegraphics[max width=\textwidth, center]{2025_10_21_263f47d9e71ddd3e429ag-115}\\
在 $\boldsymbol{\mu}$ 上,则是恒等变换。

可以把几何上的投影变换推广到酉空间(或欧氏空间)。\\
定义3.4.2 设 $S, T$ 是 $n$ 维酉空间 $V$ 的子空间,且 $n$ 维西空间 $V=S \oplus T$ ,则对于 $n$ 维酉空间 $V$ 中任一向量 $\boldsymbol{\alpha}$ 均可唯一地表示为


\begin{equation*}
\alpha=x+y, \quad x \in S, \quad y \in T \tag{3.4.3}
\end{equation*}


则称 $\boldsymbol{x}$ 是 $\boldsymbol{\alpha}$ 沿 $T$ 至 $S$ 的投影, $\boldsymbol{y}$ 是 $\boldsymbol{\alpha}$ 沿 $S$ 至 $T$ 的投影。\\
由式(3.4.3)确定的线性映射\\
$\tau_{S, T}: n$ 维酉空间 $V \rightarrow S$

$$
\tau_{S, T}(\alpha)=x
$$

称为 $n$ 维西空间 $V$ 沿 $T$ 至 $S$ 的投影映射。\\
由式(3.4.3)确定的线性变换\\
$\tau_{S, T}: n$ 维酉空间 $V \rightarrow n$ 维酉空间 $V$\\
\includegraphics[max width=\textwidth, center]{2025_10_21_263f47d9e71ddd3e429ag-115(1)}

$$
\tau_{S, T}(\boldsymbol{\alpha})=x
$$

称为 $C^{n}$ 沿 $T$ 至 $S$ 的投影变换。\\
显然,$\tau_{S, T}$ 限制在 $S$ 上是恒等变换。\\
定理3.4.3 设 $\tau$ 是 $n$ 维西(欧氏)空间 $V$ 的线性变换,则下列命题是等价的。\\
(1)$\tau$ 是 $V$ 上的投影变换。\\
(2) $\operatorname{dim}(R(\tau) \cap N(\tau))=0$\\
(3)$V=R(\tau) \oplus N(\tau)$\\
证明:(1)$\Rightarrow$(2)$\forall \alpha \in R(\tau) \cap N(\tau)$ ,由于 $\alpha \in R(\tau)$ ,故 $\tau(\alpha)=\alpha$ ,又由于 $\alpha \in N(\tau)$ ,故 $\tau(\alpha)=0$ 综合两式得 $\alpha=0$ ,此即

$$
\operatorname{dim}(R(\tau) \cap N(\tau))=0
$$

(2)$\Rightarrow$(3)由定理1.3.5知

$$
\begin{aligned}
& \operatorname{dim}(R(\tau)+N(\tau))=\operatorname{dim} R(\tau)+\operatorname{dim} N(\tau)-\operatorname{dim}(R(\tau) \cap N(\tau)) \\
= & \operatorname{dim} R(\tau)+\operatorname{dim} N(\tau)
\end{aligned}
$$

又由定理1.5.2知

$$
\operatorname{dim}(R(\tau)+N(\tau))=n
$$

此即

$$
V=R(\tau)+N(\tau)
$$

于是

$$
V=R(\tau) \oplus N(\tau)
$$

(3)$\Rightarrow$(1),因为

$$
V=R(\tau) \oplus N(\tau)
$$

故 $\forall \alpha \in V$ 有 $\alpha_{2}=\alpha-\tau(\alpha), \alpha=\tau(\alpha)+\alpha_{2}$ ,这里 $\tau(\alpha) \in R(\tau)$ ,故 $\alpha_{2} \in N(\tau)$ , $\tau\left(\alpha_{2}\right)=0$ 。所以 $\tau$ 是 $V$ 列 $R(\tau)$ 的投影变换。

定理3.4.4 设 $\tau$ 是 $n$ 维酉(欧氏)空间 $V$ 的线性变换,则下列命题是等价的。\\
${ }^{(1)} \tau$ 是 $V$ 上的投影变换。\\
(2)$\tau^{2}=\tau$\\
(3)$\tau$ 的矩阵表示 $\boldsymbol{A}$ 满足 $\boldsymbol{A}^{2}=\boldsymbol{A}$ .\\
证明:(1)$\Rightarrow$(2),$\forall \alpha \in V$ ,

$$
\alpha=\alpha_{1}+\alpha_{2}
$$

其中 $\alpha_{1} \in R(\tau), \alpha_{2} \in N(\tau)$ ,于是

$$
\tau\left(\alpha_{1}\right)=\alpha_{1}, \tau\left(\alpha_{2}\right)=0
$$

因此

$$
\begin{gathered}
\tau(\alpha)=\tau\left(\alpha_{1}+\alpha_{2}\right)=\tau\left(\alpha_{1}\right)+\tau\left(\alpha_{2}\right) \\
=\tau\left(\alpha_{1}\right)=\alpha_{1} \\
\tau^{2}(\alpha)=\tau\left(\alpha_{1}\right)=\alpha_{1}=\tau(\alpha)
\end{gathered}
$$

根据 $\alpha$ 的任意性得

$$
\tau^{2}=\tau
$$

(2)$\Rightarrow$(1),$\forall \alpha \in V, \tau(\alpha) \in R(\tau)$\\
设

$$
\begin{gathered}
\alpha=\tau(\alpha)+\alpha_{2} . \\
\tau(\alpha)=\tau^{2}(\alpha)+\tau\left(\alpha_{2}\right)
\end{gathered}
$$

根据假设 $\tau=\tau^{2}$ 代人上式得 $\tau\left(\alpha_{2}\right)=0$ ,于是 $\alpha_{2} \in N(\tau)$ 。所以 $\tau$ 是 $V$ 到 $R(\tau)$ 的投影变换。\\
(2)$\Rightarrow$(3),设 $\alpha_{1}, \cdots, \alpha_{n}$ 是 $V$ 的一组基,$A$ 是 $\tau$ 在该基下的矩阵表示,于是

$$
\begin{gathered}
\tau\left(\alpha_{1}, \cdots, \alpha_{n}\right)=\left(\alpha_{1}, \cdots, \alpha_{n}\right) A \\
\tau^{2}\left(\alpha_{1}, \cdots, \alpha_{n}\right)=\tau\left(\alpha_{1}, \cdots, \alpha_{n}\right) A=\left(\alpha_{1}, \cdots, \alpha_{n}\right) A^{2}
\end{gathered}
$$

根据 $\tau=\tau^{2}$ 得 $\quad\left(\alpha_{1}, \cdots, \alpha_{n}\right) A=\left(\alpha_{1}, \cdots, \alpha_{n}\right) A^{2}$ 。\\
由于 $\alpha_{1}, \cdots, \alpha_{n}$ 线性无关,得 $\boldsymbol{A}=\boldsymbol{A}^{2}$\\
(3)$\Rightarrow$(2),若

因为 $A=A^{2}$ ,故

$$
\begin{aligned}
& \tau\left(\alpha_{1}, \cdots, \alpha_{n}\right)=\left(\alpha_{1}, \cdots, \alpha_{n}\right) A \\
& \tau^{2}\left(\alpha_{1}, \cdots, \alpha_{n}\right)=\left(\alpha_{1}, \cdots, \alpha_{n}\right) A^{2} \\
& \tau\left(\alpha_{1}, \cdots, \alpha_{n}\right)=\tau^{2}\left(\alpha_{1}, \cdots, \alpha_{n}\right)
\end{aligned}
$$

由于 $\alpha_{1}, \cdots, \alpha_{n}$ 线性无关,得 $\tau=\tau^{2}$

\section*{二、正交补、正交投影}
为了把几何学中垂直投影的概念推广到西空间(或欧氏空间),首先要引进子空间的正交与正交补的概念。

定义3.4.3 设 $S, T$ 是 $n$ 维西空间 $V$ 的子空间,若对于任意 $x \in S, y \in T$ ,都有 $(x, y)=0$ ,则称 $S$ 与 $T$ 是正交的,并记为 $S \perp T$ 。

例3.4.1 设 $\alpha_{1}, \alpha_{2}, \alpha_{3}, \alpha_{4}$ 是 $V$ 的标准正交基,则 $S=\operatorname{span}\left\{\alpha_{1}, \alpha_{2}\right\}$ 与 $T=\operatorname{span}$\\
$\left\{\alpha_{3}, \alpha_{4}\right\}$ 是正交的.\\
定理3.4.5 设 $S, T$ 是 $n$ 维酉空间 $V$ 的两个正交子空间,则\\
(1)$S \cap T=\{0\}$\\
(2) $\operatorname{dim}(S+T)=\operatorname{dim} S+\operatorname{dim} T$\\
证明(1)设 $x \in S \cap T$ ,则 $x \in S$ ,故对任何 $y \in T$ ,都有 $(x, y)=0$ ,又因 $x \in T$ ,因此取 $y=x$ ,则 $(x, x)=0$ ,于是 $x=0$ .根据 $x$ 的任意性可得 $S \cap T=\{0\}$ .\\
(2)根据维数公式,并由(1)立刻可得。\\
定义3.4.4 若子空间 $S, T$ 是正交的,则 $S+T$ 称为 $S$ 与 $T$ 的正交和,并记为 $S$ (1)$T$ .

定理 3.4.6 设 $A \in C^{m \times n}$(或 $R^{m \times n}$ ),则\\
(1)$N(\boldsymbol{A})$(1)$R\left(\boldsymbol{A}^{\mathrm{H}}\right)=C^{n}$(或 $R^{n}$ )\\
(2)$R(A)(1) N\left(A^{\mathrm{H}}\right)=C^{m}\left(\right.$ 或 $\left.R^{m}\right)$\\
证明(1)设 $x \in N(A), y \in R\left(A^{\mathrm{H}}\right)$ ,则 $A x=0, y=A^{\mathrm{H}} z, z \in C^{n}$(或 $R^{n}$ ),于是

$$
(x, y)=y^{\mathrm{H}} x=z^{\mathrm{H}} A x=0,
$$

故

$$
N(\boldsymbol{A}) \perp R\left(\boldsymbol{A}^{\mathbf{H}}\right)
$$

又

$$
\operatorname{dim} N(\boldsymbol{A})+\operatorname{dim} R\left(\boldsymbol{A}^{\mathrm{H}}\right)=n-\operatorname{rank} \boldsymbol{A}+\operatorname{rank} \boldsymbol{A}=n
$$

因此

$$
N(\boldsymbol{A}) \oplus R\left(\boldsymbol{A}^{\mathrm{H}}\right)=C^{n}
$$

(2)类似于(1)可得。\\
定义3.4.5 设 $n$ 维西空间 $V$ 子空间 $S 、 T$ 满足

$$
S \oplus T=V
$$

则称 $S$ 为 $T$ 的正交补,记为 $T_{\perp}$ ,或

$$
S=T_{\perp}=\{\alpha \mid(\alpha, \beta)=0, \forall \beta \in T\} .
$$

显然,若 $S$ 为 $T$ 的正交补,则 $T$ 亦为 $S$ 的正交补。\\
定理3.4.7 设 $T$ 是 $n$ 维西空间 $V$ 的子空间,则存在唯一的子空间 $S$ ,使得

$$
S \text { (1) } T=V
$$

证略。\\
例3.4.2 已知 $\boldsymbol{\alpha}_{1}=(1,0,1,1)^{\mathrm{T}}, \boldsymbol{\alpha}_{2}=(0,1,1,2)^{\mathrm{T}}, T=\operatorname{span}\left\{\boldsymbol{\alpha}_{1}, \boldsymbol{\alpha}_{2}\right\}$ ,求 $T$ 的正交补。

解 取 $A=\left(\alpha_{1}, \alpha_{2}\right)$ ,则

$$
\boldsymbol{A}^{\mathrm{H}}=\left[\begin{array}{c}
\boldsymbol{\alpha}_{1}^{\mathrm{H}} \\
\boldsymbol{\alpha}_{2}^{\mathrm{H}}
\end{array}\right]=\left[\begin{array}{llll}
1 & 0 & 1 & 1 \\
0 & 1 & 1 & 2
\end{array}\right]
$$

不难知线性方程组 $\boldsymbol{A}^{\mathrm{H}} \boldsymbol{X}=0$ 的基础解系为 $\boldsymbol{\xi}_{1}=(-1,-1,1,0)^{\mathrm{T}}, \boldsymbol{\xi}_{2}=(-1,-2,0,1)^{\mathrm{T}}$ ,则 $S=\operatorname{span}\left\{\xi_{1}, \xi_{2}\right\}$ ,便是 $T$ 的正交补(什么道理请读者分析)

定义 3.4.6 设 $S$(1)$T=V$ ,若对于 $V$ 中任何向量 $\alpha=x+y$ ,其中 $x \in S, y \in T$ ,线性变换 $\sigma: V \rightarrow V$ 由下式确定

$$
\sigma(\alpha)=x
$$

则称 $\sigma$ 是由 $V$ 到 $S$ 的正交投影。\\
显然 $\sigma$ 是 $V$ 沿 $T$ 到 $S$ 的投影,正交投影是特殊的投影变换,由于 $S$ 的正交补 $T$是唯一的,所以 $V$ 到 $S$ 的正交投影就不必指出沿 $T$ 的正交投影。

定理3.4.8 设 $\tau$ 是 $n$ 维西(欧氏)空间 $V$ 到 $r$ 维子空间 $S$ 的正交投影,则 $\tau$ 在 $\boldsymbol{V}$ 的标准正交基下的矩阵表示 $\boldsymbol{P}_{S}$ 满足

$$
\boldsymbol{P}_{S}=\boldsymbol{U}_{1} \boldsymbol{U}_{1}^{\mathrm{H}}
$$

其中 $\boldsymbol{U}_{1} \in \boldsymbol{U}_{r}^{n \times r}$ 。\\
证明:证明分二步.\\
(1)设 $\alpha_{1}, \cdots, \alpha_{r}$ 是 $S$ 的一个标准正交基,且 $\alpha_{1}, \cdots, \alpha_{r}, \cdots, \alpha_{n}$ 是 $V$ 的标准正交基,则 $\tau$ 在 $\alpha_{1}, \cdots, \alpha_{r}, \cdots, \alpha_{n}$ 下的矩阵表示

$$
\begin{aligned}
\boldsymbol{P}_{1} & =\left[\begin{array}{llllllc}
1 & 0 & \cdots & 0 & 0 & \cdots & 0 \\
0 & 1 & \cdots & 0 & 0 & \cdots & 0 \\
& & & \vdots & \vdots & & \vdots \\
0 & 0 & \cdots & 1 & 0 & \cdots & 0 \\
0 & & \cdots & & & \cdots & 0 \\
\vdots & & & & & & \\
0 & & \cdots & & \cdots & 0
\end{array}\right] \zeta \text { 行 }\left[\begin{array}{ccc}
1 & \cdots & 0 \\
\vdots & \ddots & \vdots \\
0 & \cdots & 1 \\
\vdots & & \vdots \\
\vdots & & \vdots \\
0 & & 0
\end{array}\right]\left[\begin{array}{ccc}
1 & \cdots & 0 \\
\vdots & & \vdots \\
0 & \cdots & 1 \\
\vdots & & \\
\vdots & & \\
0 & \cdots & 0
\end{array}\right]^{\mathrm{H}} \\
& =\left[\begin{array}{l}
E_{r} \\
0
\end{array}\right]\left[\begin{array}{l}
E_{r} \\
0
\end{array}\right]^{\mathrm{H}}
\end{aligned}
$$

(2)设 $u_{1}, \cdots, u_{n}$ 是 $V$ 的任何一个标准正交基,且由 $u_{i}, \cdots, u_{n}$ 到前述的基 $\alpha_{1}, \cdots, \alpha_{n}$ 之过渡矩阵为 $\boldsymbol{P}$ ,即

$$
\left(u_{1}, \cdots, u_{n}\right) \boldsymbol{P}=\left(\boldsymbol{\alpha}_{1}, \cdots, \boldsymbol{\alpha}_{n}\right)
$$

则 $\boldsymbol{\alpha}$ 在 $u_{1}, \cdots, u_{n}$ 下的矩阵表示 $\boldsymbol{P}_{S}$ 满足

$$
\boldsymbol{P}^{-1} \boldsymbol{P}_{S} \boldsymbol{P}=\boldsymbol{P}_{1} \text { 或 } \boldsymbol{P}_{S}=\boldsymbol{P} \boldsymbol{P}_{1} \boldsymbol{P}^{-1}
$$

由定理 3.3.2 知 $\boldsymbol{P}$ 是西矩阵, $\boldsymbol{P}^{-1}=\boldsymbol{P}^{\mathrm{H}}$ ,故

$$
\boldsymbol{P}_{S}=\boldsymbol{P} \boldsymbol{P}_{1} \boldsymbol{P}^{\mathrm{H}}=\boldsymbol{P}\left[\begin{array}{l}
E_{r} \\
0
\end{array}\right] \quad\left[\begin{array}{l}
E_{r} \\
0
\end{array}\right]^{\mathrm{H}} \boldsymbol{P}^{\mathrm{H}}=\left\{\boldsymbol{P}\left[\begin{array}{l}
E_{r} \\
0
\end{array}\right]\right\} \quad\left\{\boldsymbol{P}\left[\begin{array}{l}
E_{r} \\
0
\end{array}\right]\right\}^{\mathrm{H}}=U_{1} U_{1}^{\mathrm{H}}
$$

(证毕)\\
注:若

$$
\left(\alpha_{1}, \cdots, \alpha_{n}\right)=\left(u_{1}, \cdots, u_{n}\right) \boldsymbol{P}=\left(u_{1}, \cdots, u_{n}\right)\left[\begin{array}{cccccc}
p_{11} & p_{12} & \cdots & p_{1 r} & \cdots & p_{1 n} \\
p_{21} & p_{22} & \cdots & p_{2 r} & \cdots & p_{2 n} \\
\cdots & \cdots & \cdots & \cdots & \cdots & \cdots \\
p_{n 1} & p_{n 2} & \cdots & p_{n r} & \cdots & p_{n n}
\end{array}\right]
$$

则

$$
\boldsymbol{U}_{1}=\boldsymbol{P}\left[\begin{array}{l}
E_{r} \\
0
\end{array}\right]=\left[\begin{array}{ccc}
p_{11} & \cdots & p_{1 r} \\
p_{21} & \cdots & p_{2 r} \\
\cdots & \cdots & \cdots \\
p_{n 1} & \cdots & p_{n r}
\end{array}\right]
$$

所以 $\boldsymbol{U}_{1} \in \boldsymbol{U}_{r}^{n \times r}$ 的 $r$ 个列向量是子空间 $S$ 的标准正交基 $\alpha_{1}, \cdots, \alpha_{r}$ 在 $V$ 的基 $u_{1}, \cdots$ , $u_{n}$ 下的坐标向量。

推论: $\boldsymbol{P}_{S}^{\mathrm{H}}=\boldsymbol{P}_{S}=\boldsymbol{P}_{S}^{2}$\\
定义3.4.7 若 $\boldsymbol{\alpha}_{1}, \boldsymbol{\alpha}_{2}, \cdots, \boldsymbol{\alpha}_{r}$ 为 $n$ 维标准正交列向量组,则称 $n \times r$ 矩阵 $\boldsymbol{U}_{1}= \left(\boldsymbol{\alpha}_{1}, \boldsymbol{\alpha}_{2}, \cdots, \boldsymbol{\alpha}_{\boldsymbol{r}}\right)$ 为次酉矩阵,记之为 $\boldsymbol{U}_{1} \in U_{r}^{n \times r}$ 。

定理 3.4.9 $\boldsymbol{U}_{1} \in U_{r}^{n \times r}$ 的充要条件为 $\boldsymbol{U}_{1}^{\mathrm{H}} \boldsymbol{U}_{1}=\boldsymbol{E}_{r}$ .\\
证略。\\
定理 3.4.10 设 $\boldsymbol{A}$ 为 $n$ 阶矩阵,则 $\boldsymbol{A}=\boldsymbol{A}^{\mathrm{H}}=\boldsymbol{A}^{2}$ 的充要条件是存在 $n \times r$ 阶矩阵 $\boldsymbol{U} \in U_{r}^{n \times r}$ ,满足

$$
\boldsymbol{A}=\boldsymbol{U} \boldsymbol{U}^{\mathrm{H}}
$$

其中 $r=\operatorname{rank} A$ .\\
证明 必要性 因 $r=\operatorname{rank} A$ ,故 $A$ 有 $r$ 个线性无关的列向量,将这 $r$ 个列向量用 Schmidt 正交化与单位化方法得出 $r$ 个两两正交的单位向量,以这 $r$ 个向量为列构成一个 $n \times r$ 矩阵 $\boldsymbol{U} \in U_{r}^{n \times r} . \boldsymbol{A}$ 的 $n$ 个列向量都可以由 $\boldsymbol{U}$ 的 $r$ 个列向量线性表出。即若 $U=\left(\boldsymbol{e}_{1}, \boldsymbol{e}_{2}, \cdots, \boldsymbol{e}_{r}\right) \in U_{r}^{n \times r}, \boldsymbol{A}=\left\{\boldsymbol{\alpha}_{1}, \boldsymbol{\alpha}_{2}, \cdots, \boldsymbol{\alpha}_{n}\right\}$ ,则

$$
\begin{aligned}
\boldsymbol{A} & =\left(\boldsymbol{\alpha}_{1}, \boldsymbol{\alpha}_{2}, \cdots, \boldsymbol{\alpha}_{n}\right)=\left(\boldsymbol{e}_{1}, \boldsymbol{e}_{2}, \cdots, \boldsymbol{e}_{r}\right)\left(\begin{array}{cccc}
c_{11} & c_{21} & \cdots & c_{n 1} \\
c_{12} & c_{22} & \cdots & c_{n 2} \\
\vdots & \vdots & & \vdots \\
c_{1 r} & c_{2 r} & \cdots & c_{n r}
\end{array}\right)_{r \times n} \\
& =U V^{\mathrm{H}}
\end{aligned}
$$

其中

$$
\boldsymbol{V}=\left(\begin{array}{cccc}
\bar{c}_{11} & \bar{c}_{12} & \cdots & \bar{c}_{1 r} \\
\bar{c}_{21} & \bar{c}_{22} & \cdots & \bar{c}_{2 r} \\
\vdots & \vdots & & \vdots \\
\bar{c}_{n 1} & \bar{c}_{n 2} & \cdots & \bar{c}_{n r}
\end{array}\right) \in C^{n \times r}
$$

由于向量组 $\boldsymbol{\alpha}_{1}, \boldsymbol{\alpha}_{2}, \cdots, \boldsymbol{\alpha}_{n}$ 的秩为 $r$ ,所以 $\boldsymbol{V}^{\mathrm{H}}$ 的秩为 $r$ .\\
下面证明 $\boldsymbol{V}=\boldsymbol{U}$ 。\\
事实上,由 $A=A^{\mathrm{H}}=A^{2}$ 得 $A=A^{\mathrm{H}} A$ ,即

$$
\boldsymbol{U} \boldsymbol{V}^{\mathrm{H}}=\boldsymbol{V} \boldsymbol{U}^{\mathrm{H}} \boldsymbol{U} \boldsymbol{V}^{\mathrm{H}}
$$

注意到 $\boldsymbol{U}^{\mathrm{H}} \boldsymbol{U}=\boldsymbol{E}_{r}$ ,所以

第三章 内积空间、正规矩阵、Hermite 矩阵

$$
\boldsymbol{U} \boldsymbol{V}^{\mathrm{H}}=\boldsymbol{V} \boldsymbol{V}^{\mathrm{H}}
$$

或

$$
(U-V) V^{\mathrm{H}}=0
$$

因为 $\operatorname{rank} \boldsymbol{V}^{\mathrm{H}}=r$ ,所以

于是

$$
\begin{gathered}
\boldsymbol{U}=\boldsymbol{V} \\
\boldsymbol{A}=\boldsymbol{U} \boldsymbol{U}^{\mathrm{H}}
\end{gathered}
$$

充分性 若 $A=U U^{\mathrm{H}}$ ,则 $A=A^{\mathrm{H}}=A^{2}$ .\\
显然,定理 3.4.9 中 $\boldsymbol{P}_{S}=\boldsymbol{U}_{1} \boldsymbol{U}_{1}{ }^{\mathrm{H}}$ 满足 $\boldsymbol{P}_{S}=\boldsymbol{P}_{S}^{\mathrm{H}}=\boldsymbol{P}_{S}^{2}$ .

\section*{§3.5 对称与反对称变换}
定义3.5.1 设 $\mathscr{B}$ 是欧氏空间 $V$ 的一个线性变换,如果对任意的 $\boldsymbol{\alpha}, \boldsymbol{\beta} \in V$都有

$$
(\mathscr{A}(\boldsymbol{\alpha}), \boldsymbol{\beta})=(\boldsymbol{\alpha}, \mathscr{A}(\boldsymbol{\beta}))
$$

那么称 $\mathscr{A}$ 为 $V$ 的一个对称变换.\\
例3.5.1 设 $W$ 是欧氏空间 $V$ 的一个子空间,那么 $V$ 在 $W$ 上的正交投影变换 $\mathscr{P}$ 就是一个对称变换。

证明 任取 $\alpha, \beta \in V$ ,设

$$
\begin{aligned}
& \boldsymbol{\alpha}=\boldsymbol{\alpha}_{1}+\boldsymbol{\alpha}_{2}, \boldsymbol{\alpha}_{1} \in W, \boldsymbol{\alpha}_{2} \in W_{\perp} \\
& \boldsymbol{\beta}=\boldsymbol{\beta}_{1}+\boldsymbol{\beta}_{2}, \boldsymbol{\beta}_{1} \in W, \boldsymbol{\beta}_{2} \in W_{\perp}
\end{aligned}
$$

由正交投影的定义可知 $\mathscr{P}(\boldsymbol{\alpha})=\boldsymbol{\alpha}_{1}, \mathscr{P}(\boldsymbol{\beta})=\boldsymbol{\beta}_{1}$ ,那么

$$
\begin{aligned}
& (\mathscr{P}(\boldsymbol{\alpha}), \boldsymbol{\beta})=\left(\boldsymbol{\alpha}_{1}, \boldsymbol{\beta}_{1}+\boldsymbol{\beta}_{2}\right)=\left(\boldsymbol{\alpha}_{1}, \boldsymbol{\beta}_{1}\right)+\left(\boldsymbol{\alpha}_{1}, \boldsymbol{\beta}_{2}\right)=\left(\boldsymbol{\alpha}_{1}, \boldsymbol{\beta}_{1}\right) \\
& (\boldsymbol{\alpha}, \mathscr{P}(\boldsymbol{\beta}))=\left(\boldsymbol{\alpha}_{1}+\boldsymbol{\alpha}_{2}, \boldsymbol{\beta}_{1}\right)=\left(\boldsymbol{\alpha}_{1}, \boldsymbol{\beta}_{1}\right)+\left(\boldsymbol{\alpha}_{2}, \boldsymbol{\beta}_{1}\right)=\left(\boldsymbol{\alpha}_{1}, \boldsymbol{\beta}_{1}\right)
\end{aligned}
$$

于是有 $(\mathscr{P}(\boldsymbol{\alpha}),(\boldsymbol{\beta})=(\boldsymbol{\alpha}, \mathscr{P}(\boldsymbol{\beta}))$ 。\\
定理3.5.1 设 $\mathscr{A}$ 是欧氏空间 $V$ 上的一个对称变换,如果 $W$ 是 $A$ 的不变子空间,那么 $W_{\perp}$ 也是 $\mathscr{A}$ 的不变子空间

证明 任取 $\boldsymbol{\alpha} \in W_{\perp}$ ,证明 $\mathcal{A}(\boldsymbol{\alpha}) \in W_{\perp}$ ,对任意的 $\boldsymbol{\beta} \in W$ 有 $\mathscr{A}(\boldsymbol{\beta}) \in W$ ,那么由 $(\mathscr{A}(\boldsymbol{\beta}), \boldsymbol{\alpha})=0$ 可得

$$
(\boldsymbol{\beta}, \mathscr{A}(\boldsymbol{\alpha}))=(\mathscr{A}(\boldsymbol{\beta}), \boldsymbol{\alpha})=0
$$

这表明 $\mathscr{A}(\boldsymbol{\alpha}) \in W_{\perp}$ .\\
定理3.5.2 欧氏空间 $V$ 上的线性变换 $\mathscr{A}$ 是对称变换的充要条件是 $\mathscr{A}$ 在 $V$的任意一个标准正交基下的矩阵是对称矩阵。

证明 必要性 任取 $V$ 的一个标准正交基 $\varepsilon_{1}, \cdots, \varepsilon_{n}, \mathscr{A}$ 在该基下对应的矩阵为 $\boldsymbol{A}=\left(a_{i j}\right)_{n \times n}$ ,那么有

$$
\begin{aligned}
& \mathscr{A}\left(\varepsilon_{i}\right)=a_{1 i} \varepsilon_{1}+a_{2 i} \varepsilon_{2}+\cdots+a_{n i} \varepsilon_{n},\left(\mathscr{A}\left(\varepsilon_{i}\right), \varepsilon_{j}\right)=a_{j i} \\
& \mathscr{A}\left(\varepsilon_{j}\right)=a_{1 j} \varepsilon_{1}+a_{2 j} \varepsilon_{2}+\cdots+a_{n j} \varepsilon_{n},\left(\mathscr{A}\left(\varepsilon_{j}\right), \varepsilon_{i}\right)=a_{i j}
\end{aligned}
$$

由于 $\mathscr{B}$ 为对称变换,所以

$$
a_{j i}=\left(\mathscr{A}\left(\varepsilon_{i}\right), \varepsilon_{j}\right)=\left(\varepsilon_{j}, \mathscr{B}\left(\varepsilon_{i}\right)\right)=\left(\mathscr{B}\left(\varepsilon_{j}\right), \varepsilon_{i}\right)=a_{i j}
$$

这表明 $\boldsymbol{A}$ 为对称矩阵.\\
充分性 设线性变换 $\mathscr{A}$ 在 $V$ 的一个标准正交基 $\varepsilon_{1}, \cdots, \varepsilon_{n}$ 下的矩阵 $\boldsymbol{A}$ 为对称矩阵。 $\boldsymbol{\alpha}, \boldsymbol{\beta}$ 为 $V$ 中任意两个向量,且在 $\boldsymbol{\varepsilon}_{1}, \cdots, \boldsymbol{\varepsilon}_{n}$ 下的坐标分别为 $\boldsymbol{X}, \boldsymbol{Y}$ ,即

$$
\boldsymbol{\alpha}=\left(\varepsilon_{1}, \cdots, \varepsilon_{n}\right) \boldsymbol{X}, \boldsymbol{\beta}=\left(\varepsilon_{1}, \cdots, \varepsilon_{n}\right) \boldsymbol{Y}
$$

于是有

$$
\mathscr{A}(\boldsymbol{\alpha})=\left(\varepsilon_{1}, \cdots, \varepsilon_{n}\right) \boldsymbol{A} \boldsymbol{X}, \mathscr{A}(\boldsymbol{\beta})=\left(\varepsilon_{1}, \cdots, \varepsilon_{n}\right) \boldsymbol{A} \boldsymbol{Y}
$$

那么

$$
(\mathscr{A}(\boldsymbol{\alpha}), \boldsymbol{\beta})=(\boldsymbol{A} \boldsymbol{X})^{\mathrm{T}} \boldsymbol{Y}=\boldsymbol{X}^{\mathrm{T}} \boldsymbol{A}^{\mathrm{T}} \boldsymbol{Y}=\boldsymbol{X}^{\mathrm{T}} \boldsymbol{A} \boldsymbol{Y}=(\boldsymbol{\alpha}, \mathscr{A}(\boldsymbol{\beta}))
$$

这表明 $\mathcal{A}$ 为 $V$ 的对称变换.\\
根据线性代数知识,实对称矩阵一定正交相似于一个对角矩阵。于是有\\
定理3.5.3 欧氏空间的对称变换是可对角化的线性变换。\\
定义3.5.2 设 $\mathscr{A}$ 是欧氏空间 $V$ 上的一个线性变换,如果对任意的 $\boldsymbol{\alpha}, \boldsymbol{\beta} \in V$都有

$$
(\mathscr{A}(\boldsymbol{\alpha}), \boldsymbol{\beta})=-(\boldsymbol{\alpha}, \mathscr{A}(\boldsymbol{\beta}))
$$

那么称 $\mathscr{A}$ 为 $V$ 的一个反对称变换。\\
与对称变换一样,反对称变换也有类似的结论。\\
定理3.5.4 设 $\mathscr{A}$ 是欧氏空间 $V$ 上的一个反对称变换,如果 $W$ 是 $\mathscr{A}$ 的不变子空间,那么 $W_{\perp}$ 也是 $\mathscr{A}$ 的不变子空间。

定理3.5.5 欧氏空间 $V$ 上的线性变换 $\mathscr{A}$ 是反对称变换当且仅当 $\mathscr{A}$ 在 $V$ 的任意一个标准正交基下的矩阵是反对称矩阵。

作为练习,请读者自己独立完成这两个定理的证明。\\
例3.5.2 在 $R^{3}$ 中,设 $u$ 为过直角坐标系原点的平面 $\pi$ 的单位法矢量。变换 $\mathscr{A}$ 是

$$
\mathscr{A}(\alpha)=\alpha-2(u, \alpha) u, \quad \alpha \in R^{3}
$$

容易验证:对于任意 $\boldsymbol{\alpha}, \boldsymbol{\beta} \in R^{3}$ ,任意实数 $k, l$ 都有

$$
\begin{gathered}
\mathscr{A}(k \boldsymbol{\alpha}+l \boldsymbol{\beta})=k \mathscr{A}(\boldsymbol{\alpha})+l, \mathscr{A}(\boldsymbol{\beta}) \\
(\mathscr{A}(\boldsymbol{\alpha}), \mathscr{A}(\boldsymbol{\beta}))=(\boldsymbol{\alpha}, \boldsymbol{\beta}) \\
(\mathscr{A}(\boldsymbol{\alpha}), \boldsymbol{\beta})=(\boldsymbol{\alpha}, \mathscr{A}(\boldsymbol{\beta}))
\end{gathered}
$$

因此 $\mathscr{B}$ 既是正交变换,又是对称变换,称其为镜面反射。\\
更进一步,将 $u$ 扩充为 $R^{3}$ 的一个标准基,$u, \eta, \nu$ 。\\
于是

$$
\mathscr{B}(u)=-u, \quad \mathscr{B}(\eta)=\eta, \quad \mathscr{A}(\nu)=\nu
$$

$\mathscr{A}$ 在 $\boldsymbol{u}, \boldsymbol{\eta}, \boldsymbol{\nu}$ 这个标准基下对应的矩阵为

$$
A=\left[\begin{array}{ccc}
-1 & 0 & 0 \\
0 & 1 & 0 \\
0 & 0 & 1
\end{array}\right]
$$

\section*{§3.6 Schur 引理、正规矩阵}
定义 3.6.1 设 $\boldsymbol{A}, \boldsymbol{B} \in C^{n \times n}$(或 $R^{n \times n}$ ),若存在 $\boldsymbol{U} \in U^{n \times n}$(或 $\boldsymbol{E}^{n \times n}$ ),使得

$$
\boldsymbol{U}^{\mathrm{H}} \boldsymbol{A U}=\boldsymbol{U}^{-1} \boldsymbol{A U}=\boldsymbol{B}\left(\text { 或 } \boldsymbol{U}^{\mathrm{T}} \boldsymbol{A U}=\boldsymbol{U}^{-1} \boldsymbol{A U}=\boldsymbol{B}\right)
$$

则说 $A$ 酉相似(或正交相似)于 $B$ .\\
定理 3.6.1(Schur 引理)任何一个 $n$ 阶复矩阵 $\boldsymbol{A}$ 酉相似于一个上(下)三角矩阵。

证明 用数学归纳法. $\boldsymbol{A}$ 的阶数为 1 时定理显然成立.现设 $\boldsymbol{A}$ 的阶数为 $k-1$时定理成立,考虑 $\boldsymbol{A}$ 的阶数为 $k$ 时的情况。

取 $k$ 阶矩阵 $\boldsymbol{A}$ 的一个特征值 $\lambda_{1}$ ,对应的单位特征向量为 $\boldsymbol{\alpha}_{1}$ ,构造以 $\boldsymbol{\alpha}_{1}$ 为第一列的 $k$ 阶酉矩阵 $\boldsymbol{U}_{1}=\left(\boldsymbol{\alpha}_{1}, \boldsymbol{\alpha}_{2}, \cdots, \boldsymbol{\alpha}_{k}\right)$ ,则

$$
\begin{aligned}
\boldsymbol{A} \boldsymbol{U}_{1} & =\left(\boldsymbol{A} \boldsymbol{\alpha}_{1}, \boldsymbol{A} \boldsymbol{\alpha}_{2}, \cdots, \boldsymbol{A} \boldsymbol{\alpha}_{k}\right) \\
& =\left(\lambda_{1} \boldsymbol{\alpha}_{1}, \boldsymbol{A} \boldsymbol{\alpha}_{2}, \cdots, \boldsymbol{A} \boldsymbol{\alpha}_{k}\right)
\end{aligned}
$$

因为 $\boldsymbol{\alpha}_{1}, \boldsymbol{\alpha}_{2}, \cdots, \boldsymbol{\alpha}_{k}$ 构成 $C^{k}$ 的一个标准正交基,故 $\boldsymbol{A} \boldsymbol{\alpha}_{i}=\sum_{j=1}^{k} a_{i j} \boldsymbol{\alpha}_{j}(i=2,3,4, \cdots, k)$ ,因此

$$
\boldsymbol{A} \boldsymbol{U}_{1}=\left(\boldsymbol{\alpha}_{1}, \boldsymbol{\alpha}_{2}, \cdots, \boldsymbol{\alpha}_{k}\right)\left(\begin{array}{ccccc}
\lambda_{1} & a_{21} & a_{31} & \cdots & a_{k 1} \\
0 & & & & \\
\vdots & & \boldsymbol{A}_{1} & & \\
0 & & & &
\end{array}\right)
$$

其中 $\boldsymbol{A}_{1}$ 是 $k-1$ 阶矩阵,根据归纳假设,存在 $k-1$ 阶酉矩阵 $\boldsymbol{W}$ 满足

$$
\boldsymbol{W}^{\mathrm{H}} \boldsymbol{A}_{1} \boldsymbol{W}=\boldsymbol{R}_{1}-\text { 上三角矩阵 }
$$

命

$$
U_{2}=\left[\begin{array}{ll}
1 & \\
& W
\end{array}\right] \in U^{k \times k}
$$

则

$$
\boldsymbol{U}_{2}^{\mathrm{H}} \boldsymbol{U}_{1}^{\mathrm{H}} \boldsymbol{A} \boldsymbol{U}_{1} \boldsymbol{U}_{2}=\left(\begin{array}{cccc}
\lambda_{1} & b_{21} & \cdots & b_{k 1} \\
0 & & & \\
\vdots & & R_{1} & \\
0 & & &
\end{array}\right)=\text { 上三角矩阵 }
$$

定理 3.6.2(Schur 不等式)设 $A=\left(a_{i j}\right) \in C^{n \times n}, \lambda_{1}, \lambda_{2}, \cdots, \lambda_{n}$ 为 $\boldsymbol{A}$ 的特征值,则

$$
\sum_{i=1}^{n}\left|\lambda_{i}\right|^{2} \leqslant \sum_{i, j}\left|a_{i j}\right|^{2}
$$

其中等号成立的充要条件是 $\boldsymbol{A}$ 西相似于对角矩阵。\\
证明 由 Schur 引理知,存在 $U \in U^{n \times n}$ ,使得

矩阵分析(第3版)

$$
\begin{aligned}
\boldsymbol{U}^{\mathrm{H}} \boldsymbol{A} \boldsymbol{U} & =\boldsymbol{R} \quad \text { 上三角矩阵, } \\
\boldsymbol{U}^{\mathrm{H}} \boldsymbol{A}^{\mathrm{H}} \boldsymbol{U} & =\boldsymbol{R}^{\mathrm{H}} \quad \text { 下三角矩阵, }
\end{aligned}
$$

其中

$$
\boldsymbol{R}=\left(r_{i j}\right) \cdot \in C^{n \times n}, r_{i j}=0(i>j)
$$

故

$$
\boldsymbol{U}^{\mathrm{H}} \boldsymbol{A} \boldsymbol{A}^{\mathrm{H}} U=\boldsymbol{R} \boldsymbol{R}^{\mathrm{H}}
$$

于是

$$
\operatorname{tr}\left(A A^{\mathrm{H}}\right)=\operatorname{tr}\left(R R^{\mathrm{H}}\right)
$$

即

$$
\sum_{i, j=1}^{n}\left|a_{i j}\right|^{2}=\sum_{i, j=1}^{n}\left|r_{i j}\right|^{2}
$$

而

$$
\sum_{i, j=1}^{n}\left|r_{i j}\right|^{2}=\sum_{i=1}^{n}\left|r_{i i}\right|^{2}+\sum_{i \neq j}\left|r_{i j}\right|^{2}=\sum_{i=1}^{n}\left|\lambda_{i}\right|^{2}+\sum_{i \neq j}\left|r_{i j}\right|^{2}
$$

故

$$
\sum_{i=1}^{n}\left|\lambda_{i}\right|^{2} \leqslant \sum_{i, j=1}^{n}\left|r_{i j}\right|^{2}=\sum_{i, j=1}^{n}\left|a_{i j}\right|^{2} .
$$

等号成立的充要条件是当 $i \neq j$ 时,$r_{i j}=0$ .此即 $\boldsymbol{R}$ 是对角矩阵.\\
例3.6.1 已知

$$
A=\left[\begin{array}{rrr}
0 & 3 & 3 \\
-1 & 8 & 6 \\
2 & -14 & -10
\end{array}\right]
$$

试求酉矩阵 $\boldsymbol{W}$ ,使得

$$
\boldsymbol{W}^{\mathrm{H}} \boldsymbol{A} \boldsymbol{W}=\text { 上三角矩阵 }
$$

解 $|\lambda E-A|=\lambda(\lambda+1)^{2}$\\
当 $\lambda=0$ 时,$A$ 有单位特征向量

$$
\varepsilon_{1}=\left(\frac{2}{\sqrt{6}}, \frac{1}{\sqrt{6}},-\frac{1}{\sqrt{6}}\right)^{\mathrm{T}}
$$

由与其内积为零的方程 $\left(\varepsilon_{1}, \varepsilon_{2}\right)=0$

$$
2 x_{1}+x_{2}-x_{3}=0
$$

可取解(单位向量)

$$
\varepsilon_{2}=\left(0, \frac{1}{\sqrt{2}}, \frac{1}{\sqrt{2}}\right)^{\mathrm{T}}
$$

又由内积为零的方程组 $\left(\varepsilon_{1}, \varepsilon_{3}\right)=0$ 与 $\left(\varepsilon_{2}, \varepsilon_{3}\right)=0$ .

$$
\begin{array}{r}
2 x_{1}+x_{2}-x_{3}=0 \\
x_{2}+x_{3}=0
\end{array}
$$

可取解(单位向量)

$$
\varepsilon_{3}=\left(-\frac{1}{\sqrt{3}}, \frac{1}{\sqrt{3}},-\frac{1}{\sqrt{3}}\right)^{\mathrm{T}}
$$

这样,可令

$$
\begin{gathered}
U_{1}=\left[\begin{array}{ccc}
\frac{2}{\sqrt{6}} & 0 & -\frac{1}{\sqrt{3}} \\
\frac{1}{\sqrt{6}} & \frac{1}{\sqrt{2}} & \frac{1}{\sqrt{3}} \\
-\frac{1}{\sqrt{6}} & \frac{1}{\sqrt{2}} & -\frac{1}{\sqrt{3}}
\end{array}\right] \\
U_{1}^{\mathrm{H}} A U_{1}=\left[\begin{array}{ccc}
0 & \frac{50}{\sqrt{12}} & \frac{9}{\sqrt{18}} \\
0 & -5 & -\frac{3}{\sqrt{6}} \\
0 & \frac{32}{\sqrt{6}} & 3
\end{array}\right]=\left[\begin{array}{lll}
0 & \frac{50}{\sqrt{12}} & \frac{9}{\sqrt{18}} \\
0 & A_{1}
\end{array}\right] \\
A_{1}=\left[\begin{array}{cc}
-5 & -\frac{3}{\sqrt{6}} \\
\frac{32}{\sqrt{6}} & 3
\end{array}\right] \\
\left|\lambda E-A_{1}\right|=(\lambda+1)^{2}
\end{gathered}
$$

其中

$$
\eta_{1}=\left(-\frac{3}{\sqrt{105}}, \frac{4 \sqrt{6}}{\sqrt{105}}\right)^{\mathrm{T}}
$$

与 $\boldsymbol{\eta}_{1}$ 正交的单位向量

命

$$
\begin{aligned}
\eta_{2} & =\left(\frac{4 \sqrt{6}}{\sqrt{105}}, \frac{3}{\sqrt{105}}\right)^{\mathrm{T}} \\
V_{1} & =\left[\begin{array}{cc}
-\frac{3}{\sqrt{105}} & \frac{4 \sqrt{6}}{\sqrt{105}} \\
\frac{4 \sqrt{6}}{\sqrt{105}} & \frac{3}{\sqrt{105}}
\end{array}\right]
\end{aligned}
$$

则

$$
V_{1}^{\mathrm{H}} A_{1} V_{1}=\left[\begin{array}{cc}
-1 & \frac{1225}{35 \sqrt{6}} \\
0 & -1
\end{array}\right]
$$

命

$$
U_{2}=\left[\begin{array}{ccc}
1 & 0 & 0 \\
0 & -\frac{3}{\sqrt{105}} & \frac{4 \sqrt{6}}{\sqrt{105}} \\
0 & \frac{4 \sqrt{6}}{\sqrt{105}} & \frac{3}{\sqrt{105}}
\end{array}\right]
$$

则

$$
\begin{gathered}
\boldsymbol{W}=\boldsymbol{U}_{1} \boldsymbol{U}_{2}=\left[\begin{array}{ccc}
\frac{2}{\sqrt{6}} & -\frac{4 \sqrt{2}}{\sqrt{105}} & \frac{1}{\sqrt{35}} \\
\frac{1}{\sqrt{6}} & \frac{5}{\sqrt{210}} & \frac{5}{\sqrt{35}} \\
-\frac{1}{\sqrt{6}} & -\frac{11}{\sqrt{210}} & \frac{3}{\sqrt{35}}
\end{array}\right] \\
\boldsymbol{W}^{\mathrm{H}} \boldsymbol{A} \boldsymbol{W}=\left[\begin{array}{ccc}
0 & \frac{50}{\sqrt{12}} & \frac{9}{\sqrt{18}} \\
0 & -1 & \frac{1225}{35 \sqrt{6}} \\
0 & 0 & -1
\end{array}\right]
\end{gathered}
$$

定义 3.6.2 设 $A \in C^{n \times n}$ ,若


\begin{equation*}
\boldsymbol{A} \boldsymbol{A}^{\mathrm{H}}=\boldsymbol{A}^{\mathrm{H}} \boldsymbol{A} \tag{3.6.1}
\end{equation*}


则称 $\boldsymbol{A}$ 为正规矩阵。\\
若 $\boldsymbol{A} \in R^{n \times n}$ ,显然 $\boldsymbol{A}^{\mathrm{H}}=\boldsymbol{A}^{\mathrm{T}}$ ,于是式(3.6.1)成为


\begin{equation*}
\boldsymbol{A} \boldsymbol{A}^{\mathrm{T}}=\boldsymbol{A}^{\mathrm{T}} \boldsymbol{A} \tag{3.6.2}
\end{equation*}


则称 $\boldsymbol{A}$ 为实正规矩阵.\\
例如

$$
\boldsymbol{A}=\left[\begin{array}{cc}
2+12 \mathrm{i} & -6-4 \mathrm{i} \\
6+4 \mathrm{i} & 8+3 \mathrm{i}
\end{array}\right], \boldsymbol{B}=\left[\begin{array}{cc}
2+\mathrm{i} & 2-\mathrm{i} \\
2-\mathrm{i} & 2+\mathrm{i}
\end{array}\right]
$$

是正规矩阵.

$$
C=\left[\begin{array}{rr}
1 & -1 \\
1 & 1
\end{array}\right], D=\left[\begin{array}{rr}
3 & -2 \\
-2 & 0
\end{array}\right]
$$

是实正规矩阵。\\
显然,对角阵、Hermite 矩阵、反 Hermite 矩阵与酉矩阵都是正规矩阵。实对称矩阵、实反对称矩阵、正交矩阵也是实正规矩阵。

引理3.6.1 设 $\boldsymbol{A}$ 是正规矩阵,则与 $\boldsymbol{A}$ 酉相似的矩阵都是正规矩阵。\\
证略。\\
引理3.6.2 设 $\boldsymbol{A}$ 是正规矩阵,且 $\boldsymbol{A}$ 是三角矩阵,则 $\boldsymbol{A}$ 是对角阵.\\
证明 设

$$
\boldsymbol{A}=\left[\begin{array}{cccc}
a_{11} & a_{12} & \cdots & a_{1 n} \\
& a_{22} & \cdots & a_{2 n} \\
& & \ddots & \vdots \\
& & & a_{n n}
\end{array}\right], \quad \boldsymbol{A}^{\mathbf{H}}=\left[\begin{array}{cccc}
\bar{a}_{11} & & & \\
\bar{a}_{12} & \bar{a}_{22} & & \\
\vdots & & \ddots & \\
\bar{a}_{1 n} & \bar{a}_{2 n} & \cdots & \bar{a}_{n n}
\end{array}\right]
$$

代人 $\boldsymbol{A} \boldsymbol{A}^{\mathrm{H}}=\boldsymbol{A}^{\mathrm{H}} \boldsymbol{A}$ 后比较等式两端矩阵第一行第一列元素,第二行第二列元素,$\cdots$ ,

第 $n$ 行第 $n$ 列元素得 $n$ 个等式

$$
\begin{aligned}
a_{11} \bar{a}_{11}+a_{12} \bar{a}_{12}+\cdots+a_{1 n} \bar{a}_{1 n} & =\bar{a}_{11} a_{11} \\
a_{22} \bar{a}_{22}+\cdots+a_{2 n} \bar{a}_{2 n} & =\bar{a}_{22} a_{22} \\
\ddots \quad \vdots \quad \vdots \quad \vdots \quad \bar{a}_{n n} & =\bar{a}_{n n} a_{n n}
\end{aligned}
$$

根据 $a_{i j} \bar{a}_{i j} \geqslant 0(i \neq j)$ 得 $a_{i j}=0(i \neq j)$ 。因此 $\boldsymbol{A}$ 是对角阵。\\
定理 3.6.3 设 $\boldsymbol{A} \in C^{n \times n}$ ,则 $\boldsymbol{A}$ 是正规矩阵的充要条件是存在 $\boldsymbol{U} \in U^{n \times n}$ ,使得

$$
\boldsymbol{U}^{\mathrm{H}} \boldsymbol{A} \boldsymbol{U}=\operatorname{diag}\left(\lambda_{1}, \lambda_{2}, \cdots, \lambda_{n}\right)
$$

其中 $\lambda_{1}, \lambda_{2}, \cdots, \lambda_{n}$ 是 $\boldsymbol{A}$ 的特征值。\\
证明 必要性 根据 Schur 引理知,存在 $U \in U^{n \times n}$ ,使得

$$
\boldsymbol{U}^{\mathrm{H}} \boldsymbol{A} \boldsymbol{U}=\boldsymbol{B} \text { 上三角阵. }
$$

根据引理 3.6.1 知, $\boldsymbol{B}$ 是正规矩阵,又根据引理 3.6.2 知, $\boldsymbol{B}$ 是对角阵。\\
充分性 因为对角阵 $\operatorname{diag}\left(\lambda_{1}, \lambda_{2}, \cdots, \lambda_{n}\right)$ 是正规矩阵,根据引理3.6.1知,$A$是正规矩阵。

一般地,我们称此定理为正规矩阵的结构定理。\\
推论 3.6. 1 设 $\boldsymbol{A}$ 是正规矩阵,$\lambda_{i}$ 是 $\boldsymbol{A}$ 的特征值,对应的特征向量是 $\boldsymbol{x}$ ,则 $\overline{\lambda_{i}}$是 $A^{\mathrm{H}}$ 的特征值,其对应的特征向量是 $x$ 。

推论 3.6.2 $n$ 阶正规矩阵 $\boldsymbol{A}$ 有 $n$ 个线性无关的特征向量\\
推论 3.6.3 正规矩阵属于不同特征值的特征子空间是互相正交的\\
证明 设 $\lambda_{i} \neq \lambda_{j}, A x_{i}=\lambda_{i} x_{i}, A x_{j}=\lambda_{j} x_{j}$ ,则

$$
\begin{aligned}
\lambda_{i}\left(x_{i}, x_{j}\right) & =\left(\lambda_{i} x_{i}, x_{j}\right)=\left(A x_{i}, x_{j}\right) \\
& =x_{j}^{\mathrm{H}} A x_{i}=\left(x_{j}^{\mathrm{H}} A x_{i}\right)^{\mathrm{T}} \\
& =\overline{x_{i}^{\mathrm{T}} A^{\mathrm{T}} \bar{x}_{j}}=\overline{\left(x_{i}^{\mathrm{H}} A^{\mathrm{H}} x_{j}\right)} \\
& =\overline{\left(x_{i}^{\mathrm{H}} \bar{\lambda}_{j} x_{j}\right)}=\lambda_{j} \overline{\left(x_{i}^{\mathrm{H}} x_{j}\right)} \\
& =\lambda_{j} \overline{\left(x_{i}^{\mathrm{H}} x_{j}\right)^{\mathrm{T}}}=\lambda_{j}\left(x_{i}^{\mathrm{H}} x_{j}\right)^{\mathrm{H}} \\
& =\lambda_{j}\left(x_{j}^{\mathrm{H}} x_{i}\right)=\lambda_{j}\left(x_{i}, x_{j}\right)
\end{aligned}
$$

由于 $\lambda_{i} \neq \lambda_{j}$ ,故

$$
\left(x_{i}, x_{j}\right)=0
$$

在线性代数中,曾讨论过,对已知实对称矩阵 $\boldsymbol{A}$ ,求正交矩阵 $\boldsymbol{Q}$ ,使得 $\boldsymbol{Q}^{-1} \boldsymbol{A Q}=$ 对角矩阵。类似地,现在要讨论,当 $\boldsymbol{A}$ 是正规矩阵,求酉矩阵 $\boldsymbol{U}$ ,使得

$$
\boldsymbol{U}^{\mathrm{H}} \boldsymbol{A} \boldsymbol{U}=\text { 对角矩阵 }
$$

具体求 $\boldsymbol{U}$ 的步骤如下:\\
(1)求 $|\lambda \boldsymbol{E}-\boldsymbol{A}|=0$ 的根 $\lambda_{1}, \lambda_{2}, \cdots, \lambda_{n}$\\
(2)对每一个相异特征值 $\lambda_{i}$ ,求 $\lambda_{i}$ 的特征子空间 $V_{\lambda_{i}}$\\
(3)用 Schmidt 正交化与单位化方法,求 $V_{\lambda_{i}}$ 的标准正交基 $\varepsilon_{i 1}, \varepsilon_{i 2}, \cdots, \varepsilon_{i n_{i}}$\\
(4)命

$$
U=\left(\varepsilon_{11}, \varepsilon_{12}, \cdots, \varepsilon_{1 n_{1}}, \varepsilon_{21}, \cdots, \varepsilon_{2 n_{2}} \cdots, \varepsilon_{s 1}, \cdots, \varepsilon_{s n_{s}}\right)
$$

则西矩阵 $\boldsymbol{U}$ 满足

$$
\boldsymbol{U}^{\mathrm{H}} \boldsymbol{A} \boldsymbol{U}=\operatorname{diag}\left(\lambda_{1}, \cdots, \lambda_{n}\right) .
$$

例3.6.2 已知

$$
A=\left[\begin{array}{rr}
1 & -1 \\
1 & 1
\end{array}\right]
$$

验证 $\boldsymbol{A}$ 是正规矩阵,且求西矩阵 $\boldsymbol{U}$ ,使 $\boldsymbol{U}^{\mathrm{H}} \boldsymbol{A} \boldsymbol{U}$ 为对角矩阵.\\
解 由于 $\boldsymbol{A}^{\mathbf{H}}=\left[\begin{array}{rr}1 & 1 \\ -1 & 1\end{array}\right]$\\
经计算得

$$
A A^{\mathrm{H}}=\left[\begin{array}{ll}
2 & 0 \\
0 & 2
\end{array}\right]=A^{\mathrm{H}} A
$$

所以 $\boldsymbol{A}$ 是正规矩阵。\\
$\boldsymbol{A}$ 的特征多项式

$$
|\lambda E-A|=\left|\begin{array}{cc}
\lambda-1 & 1 \\
-1 & \lambda-1
\end{array}\right|=\lambda^{2}-2 \lambda+2
$$

于是 $\boldsymbol{A}$ 的特征值: $\boldsymbol{\lambda}_{1}=\mathbf{1}+\mathrm{i}, \boldsymbol{\lambda}_{2}=\mathbf{1}-\mathrm{i}$\\
当 $\lambda_{1}=1+\mathrm{i}$ 时,特征矩阵

$$
\lambda_{1} E-A=\left[\begin{array}{rr}
\mathrm{i} & 1 \\
-1 & \mathrm{i}
\end{array}\right] \rightarrow\left[\begin{array}{rr}
-1 & \mathrm{i} \\
0 & 0
\end{array}\right]
$$

故

$$
x_{1}=\mathrm{i} x_{2}
$$

所以属于 $\lambda_{1}=1+\mathrm{i}$ 的单位特征向量 $\alpha_{1}=\left(\frac{\mathrm{i}}{\sqrt{2}}, \frac{1}{\sqrt{2}}\right)^{\mathrm{T}}$\\
当 $\lambda_{2}=1-\mathrm{i}$ 时,特征矩阵

故

$$
\lambda_{2} E-A=\left[\begin{array}{rr}
-\mathrm{i} & 1 \\
-1 & -\mathrm{i}
\end{array}\right] \rightarrow\left[\begin{array}{ll}
1 & \mathrm{i} \\
0 & 0
\end{array}\right]
$$

故

$$
x_{1}=-\mathrm{i} x_{2}
$$

所以属于 $\lambda_{2}=1-\mathrm{i}$ 的单位特征向量 $\boldsymbol{\alpha}_{2}=\left(\frac{-\mathrm{i}}{\sqrt{2}}, \frac{1}{\sqrt{2}}\right)^{\mathrm{T}}$\\
命

$$
U=\left(\alpha_{1}, \alpha_{2}\right)=\left[\begin{array}{cc}
\frac{\mathrm{i}}{\sqrt{2}} & -\frac{\mathrm{i}}{\sqrt{2}} \\
\frac{1}{\sqrt{2}} & \frac{1}{\sqrt{2}}
\end{array}\right]
$$

$\boldsymbol{U}$ 是西矩阵,且满足

$$
\boldsymbol{U}^{\mathrm{H}} \boldsymbol{A} \boldsymbol{U}=\left[\begin{array}{cc}
1+\mathrm{i} & 0 \\
0 & 1-\mathrm{i}
\end{array}\right]
$$

例3.6.3 已知

$$
A=\left[\begin{array}{rrr}
0 & \mathrm{i} & -1 \\
-\mathrm{i} & 0 & \mathrm{i} \\
-1 & -\mathrm{i} & 0
\end{array}\right]
$$

验证 $\boldsymbol{A}$ 是正规矩阵,且求西矩阵 $\boldsymbol{U}$ ,使得 $\boldsymbol{U}^{\mathrm{H}} \boldsymbol{A U}$ 为对角矩阵\\
解 $\boldsymbol{A}^{\mathrm{H}}=\left[\begin{array}{crr}0 & \mathrm{i} & -1 \\ -\mathrm{i} & 0 & \mathrm{i} \\ -1 & -\mathrm{i} & 0\end{array}\right]=\boldsymbol{A}$\\
$\boldsymbol{A}$ 是 Hermite 矩阵

$$
\begin{aligned}
|\lambda E-A| & =\left|\begin{array}{rrr}
\lambda & -\mathrm{i} & 1 \\
\mathrm{i} & \lambda & -\mathrm{i} \\
1 & \mathrm{i} & \lambda
\end{array}\right| \\
& =\lambda^{3}-3 \lambda-2=(\lambda+1)^{2}(\lambda-2)
\end{aligned}
$$

$A$ 的特征值 $\lambda_{1}=\lambda_{2}=-1, \lambda_{3}=2$\\
对 $\lambda_{1}=-1$ 的特征矩阵作初等行变换得

$$
\begin{aligned}
& \lambda_{1} E-A= {\left[\begin{array}{rrr}
-1 & -\mathrm{i} & 1 \\
\mathrm{i} & -1 & -\mathrm{i} \\
1 & \mathrm{i} & -1
\end{array}\right] \rightarrow\left[\begin{array}{rrr}
1 & \mathrm{i} & -1 \\
0 & 0 & 0 \\
0 & 0 & 0
\end{array}\right] } \\
& x_{1}+\mathrm{i} x_{2}-x_{3}=0
\end{aligned}
$$

解得属于特征值 -1 的特征向量为

$$
\alpha_{1}=(1,0,1)^{\mathrm{T}}, \alpha_{2}=(0,1, \mathrm{i})^{\mathrm{T}}
$$

用 Schmidt 方法把 $\boldsymbol{\alpha}_{1}, \boldsymbol{\alpha}_{2}$ 单位化并正交化得

$$
\nu_{1}=\left(\frac{1}{\sqrt{2}}, 0, \frac{1}{\sqrt{2}}\right)^{\mathrm{T}}, \nu_{2}=\left(-\frac{\mathrm{i}}{\sqrt{6}}, \frac{2}{\sqrt{6}}, \frac{\mathrm{i}}{\sqrt{6}}\right)^{\mathrm{T}}
$$

对 $\lambda_{3}=2$ 的特征矩阵作初等行变换得

于是

$$
\begin{gathered}
\lambda_{3} \boldsymbol{E}-\boldsymbol{A}=\left[\begin{array}{rrr}
2 & -\mathrm{i} & 1 \\
\mathrm{i} & 2 & -\mathrm{i} \\
1 & \mathrm{i} & 2
\end{array}\right] \rightarrow\left[\begin{array}{rrr}
1 & 0 & 1 \\
0 & 1 & -\mathrm{i} \\
0 & 0 & 0
\end{array}\right] \\
x_{1}+x_{3}=0, x_{2}-\mathrm{i} x_{3}=0
\end{gathered}
$$

故 $\boldsymbol{A}$ 的属于 $\lambda_{3}=2$ 的单位特征向量

$$
\nu_{3}=\left(-\frac{1}{\sqrt{3}}, \frac{\mathrm{i}}{\sqrt{3}}, \frac{1}{\sqrt{3}}\right)^{\mathrm{T}}
$$

命

$$
U=\left(\nu_{1}, \nu_{2}, \nu_{3}\right)=\left[\begin{array}{ccc}
\frac{1}{\sqrt{2}} & -\frac{\mathrm{i}}{\sqrt{6}} & -\frac{1}{\sqrt{3}} \\
0 & \frac{2}{\sqrt{6}} & \frac{\mathrm{i}}{\sqrt{3}} \\
\frac{1}{\sqrt{2}} & \frac{\mathrm{i}}{\sqrt{6}} & \frac{1}{\sqrt{3}}
\end{array}\right]
$$

$\boldsymbol{U}$ 是酉矩阵,且

$$
\boldsymbol{U}^{\mathrm{H}} \boldsymbol{A} \boldsymbol{U}=\left[\begin{array}{rrr}
-1 & 0 & 0 \\
0 & -1 & 0 \\
0 & 0 & 2
\end{array}\right]
$$

定理3.6.4 设 $\boldsymbol{A}$ 是正规矩阵,则\\
(1) $\boldsymbol{A}$ 是 Hermite 矩阵的充要条件是 $A$ 的特征值是实数。\\
(2) $\boldsymbol{A}$ 是反 Hermite 矩阵的充要条件是 $A$ 的特征值的实部为零。\\
(3) $\boldsymbol{A}$ 是西矩阵的充要条件是 $\boldsymbol{A}$ 的特征值的模长等于 1 .\\
证明 根据定理3.6.3知,存在 $U \in U^{n \times n}$ ,使得

$$
U^{H} A U=\operatorname{diag}\left(\lambda_{1}, \lambda_{2}, \cdots, \lambda_{n}\right)
$$

故

$$
\begin{aligned}
\boldsymbol{A} & =\boldsymbol{U} \operatorname{diag}\left(\lambda_{1}, \lambda_{2}, \cdots, \lambda_{n}\right) \boldsymbol{U}^{\mathrm{H}} \\
\boldsymbol{A}^{\mathrm{H}} & =\boldsymbol{U} \operatorname{diag}\left(\bar{\lambda}_{1}, \bar{\lambda}_{2}, \cdots, \bar{\lambda}_{n}\right) \boldsymbol{U}^{\mathrm{H}}
\end{aligned}
$$

(1)若 $\boldsymbol{A}^{\mathrm{H}}=\boldsymbol{A}$ ,则 $\bar{\lambda}_{i}=\lambda_{i}$ ,此即 $\lambda_{i}$ 为实数.反之,若 $\bar{\lambda}_{i}=\lambda_{i}$ ,则 $\boldsymbol{A}^{\mathrm{H}}=\boldsymbol{A}$ .\\
(2)若 $\boldsymbol{A}^{\mathrm{H}}=-\boldsymbol{A}$ ,则 $\bar{\lambda}_{i}=-\lambda_{i}$ ,故 $\lambda_{i}$ 的实部为零。反之,若 $\bar{\lambda}_{i}=-\lambda_{i}$ ,则 $\boldsymbol{A}^{\mathrm{H}}=-\boldsymbol{A}$ 。\\
(3)若 $\boldsymbol{A} \boldsymbol{A}^{\mathrm{H}}=\boldsymbol{E}$ ,则 $\lambda_{i} \bar{\lambda}_{i}=1,\left|\lambda_{i}\right|=1$ .反之,若 $\left|\lambda_{i}\right|=1, \lambda_{i} \bar{\lambda}_{i}=1$ ,故 $\boldsymbol{A} \boldsymbol{A}^{\mathrm{H}}=\boldsymbol{E}$ .值得指出的是,决不能根据定理 3.6.3 推出正规矩阵仅此三类。\\
例如,若

$$
A=\left[\begin{array}{lr}
1 & -1 \\
1 & 1
\end{array}\right], \quad A^{H}=\left[\begin{array}{rr}
1 & 1 \\
-1 & 1
\end{array}\right]
$$

则 $\boldsymbol{A}$ 是正规矩阵,但 $|\lambda \boldsymbol{E}-\boldsymbol{A}|=\lambda^{2}-2 \lambda+2, \boldsymbol{A}$ 的特征值为 $1+\mathrm{i}$ 与 $1-\mathrm{i}$ 。所以 $\boldsymbol{A}$ 不属于此三类矩阵。

例 3.6.4 已知 $\boldsymbol{A}^{\mathrm{H}}=\boldsymbol{A}, \boldsymbol{A}^{k}=0$( $k$ 为自然数),则 $\boldsymbol{A}=0$ .\\
解 存在 $U \in U^{n \times n}$ ,满足

$$
\boldsymbol{A}=\boldsymbol{U}\left[\begin{array}{lll}
\lambda_{1} & & \\
& \ddots & \\
& & \lambda_{n}
\end{array}\right] \boldsymbol{U}^{\mathrm{H}}, \quad \lambda_{i} \in \mathbf{R}
$$

故得

$$
\lambda_{i}^{k}=0, \quad \lambda_{i} \in \mathbf{R} \quad(i=1,2, \cdots, n)
$$

于是

$$
\lambda_{i}=0
$$

$$
(i=1,2, \cdots, n)
$$

这表明 $\boldsymbol{A}=0$ .\\
注 1:若 $\boldsymbol{A}^{k}=0$ ,就有 $\boldsymbol{A}=0$ 吗?请读者举例: $\boldsymbol{A}^{k}=0$ 但 $\boldsymbol{A} \neq 0$ .\\
注 2:若 $n$ 阶矩阵 $\boldsymbol{A}$ 的 $n$ 个特征值全为 0 就有 $\boldsymbol{A}=0$ 吗?可以举例,不是零矩阵的特征值全为 0 。

例3.6.5 已知 $\boldsymbol{U}$ 是 $n$ 阶酉矩阵,且 $\boldsymbol{U}-\boldsymbol{E}$ 可逆,试证

$$
A=(U-E)^{-1}(U+E)
$$

是反 Hermite 矩阵。\\
解

$$
\begin{aligned}
\boldsymbol{A}^{\mathrm{H}} & =(\boldsymbol{U}+\boldsymbol{E})^{\mathrm{H}}(\boldsymbol{U}-\boldsymbol{E})^{-\mathrm{H}} \\
& =\left(\boldsymbol{U}^{\mathrm{H}}+\boldsymbol{E}\right)\left(\boldsymbol{U}^{\mathrm{H}}-\boldsymbol{E}\right)^{-1} \\
& =\left[(\boldsymbol{E}+\boldsymbol{U}) \boldsymbol{U}^{\mathrm{H}}\right]\left[(\boldsymbol{E}-\boldsymbol{U}) \boldsymbol{U}^{\mathrm{H}}\right]^{-1} \\
& =(\boldsymbol{E}+\boldsymbol{U}) \boldsymbol{U}^{\mathrm{H}}\left(\boldsymbol{U}^{\mathrm{H}}\right)^{-1}(\boldsymbol{E}-\boldsymbol{U})^{-1} \\
& =(\boldsymbol{U}+\boldsymbol{E})(\boldsymbol{E}-\boldsymbol{U})^{-1} \\
& =-(\boldsymbol{U}+\boldsymbol{E})(\boldsymbol{U}-\boldsymbol{E})^{-1}
\end{aligned}
$$

由于

$$
(U+E)(U-E)=(U-E)(U+E)=\left(U^{2}-E^{2}\right)
$$

用 $(\boldsymbol{U}-\boldsymbol{E})^{-1}$ 左乘上式,再右乘上式两端得

$$
(U-E)^{-1}(U+E)=(U+E)(U-E)^{-1}
$$

因此

$$
A^{\mathrm{H}}=-(U+E)(U-E)^{-1}=-(U-E)^{-1}(U+E)=-A
$$

下面给出两个正规矩阵同时酉对角化的定理。\\
定理3.6.5 设 $\boldsymbol{A}, \boldsymbol{B}$ 都是正规矩阵,则 $\boldsymbol{A}, \boldsymbol{B}$ 可以同时西对角化的充要条件是 $\boldsymbol{A B}=\boldsymbol{B A} . \boldsymbol{A}, \boldsymbol{B}$ 可以同时酉对角化的含义是存在一个 $n$ 阶西矩阵 $\boldsymbol{U}$ ,使得

$$
\boldsymbol{U}^{\mathrm{H}} \boldsymbol{A} \boldsymbol{U}=\left[\begin{array}{lll}
\lambda_{1} & & \\
& \ddots & \\
& & \lambda_{n}
\end{array}\right], \quad \boldsymbol{U}^{\mathrm{H}} \boldsymbol{B} \boldsymbol{U}=\left[\begin{array}{lll}
u_{1} & & \\
& \ddots & \\
& & u_{n}
\end{array}\right]
$$

证明 必要性 由已知条件可得

$$
A=U\left[\begin{array}{lll}
\lambda_{1} & & \\
& \ddots & \\
& & \lambda_{n}
\end{array}\right] U^{\mathrm{H}}, \quad B=U\left[\begin{array}{lll}
u_{1} & & \\
& \ddots & \\
& & u_{n}
\end{array}\right] U^{\mathrm{H}}
$$

显然有

$$
A B=B A
$$

充分性 设 $|\lambda \boldsymbol{I}-\boldsymbol{A}|=\left(\lambda-\lambda_{1}\right)^{k_{1}}\left(\lambda-\lambda_{2}\right)^{k_{2}} \cdots\left(\lambda-\lambda_{t}\right)^{k_{t}}$ ,且 $\sum_{i=1}^{t} k_{i}=n$ .由于 $\boldsymbol{A}$ 为正规矩阵。所以存在西矩阵 $\boldsymbol{W}$ 使得

$$
\boldsymbol{W}^{\mathrm{H}} \boldsymbol{A} \boldsymbol{W}=\left[\begin{array}{cccc}
\lambda_{1} \boldsymbol{E}_{k_{1}} & & & \\
& \lambda_{2} \boldsymbol{E}_{k_{2}} & & \\
& & \ddots & \\
& & & \lambda_{t} \boldsymbol{E}_{k_{t}}
\end{array}\right]=\boldsymbol{C}
$$

记 $\boldsymbol{D}=\boldsymbol{W}^{\mathrm{H}} \boldsymbol{B} \boldsymbol{W}$ ,因为 $\boldsymbol{B}$ 为正规矩阵,易知 $\boldsymbol{D}$ 也是正规矩阵。将 $\boldsymbol{D}$ 分块如下:

$$
\boldsymbol{W}^{\mathrm{H}} \boldsymbol{B} \boldsymbol{W}=\boldsymbol{D}=\left[\begin{array}{cccc}
\boldsymbol{D}_{11} & \boldsymbol{D}_{12} & \cdots & \boldsymbol{D}_{1 t} \\
\boldsymbol{D}_{21} & \boldsymbol{D}_{22} & \cdots & \boldsymbol{D}_{2 t} \\
\vdots & \vdots & & \vdots \\
\boldsymbol{D}_{t 1} & \boldsymbol{D}_{t 2} & \cdots & \boldsymbol{D}_{t t}
\end{array}\right], \boldsymbol{D}_{j j} \text { 为 } k_{j} \times k_{j} \text { 阶矩阵 }
$$

由 $\boldsymbol{A B}=\boldsymbol{B A}$ ,可知

$$
\left(\boldsymbol{W}^{\mathrm{H}} \boldsymbol{A} \boldsymbol{W}\right)\left(\boldsymbol{W}^{\mathrm{H}} \boldsymbol{B} \boldsymbol{W}\right)=\left(\boldsymbol{W}^{\mathrm{H}} \boldsymbol{B} \boldsymbol{W}\right)\left(\boldsymbol{W}^{\mathrm{H}} \boldsymbol{A} \boldsymbol{W}\right)
$$

即 $\boldsymbol{C D}=\boldsymbol{D C}$ ,

$$
\begin{gathered}
{\left[\begin{array}{cccc}
\lambda_{1} \boldsymbol{E}_{k_{1}} & & & \\
& \lambda_{2} \boldsymbol{E}_{k_{2}} & & \\
& & \ddots & \\
& & & \lambda_{t} \boldsymbol{E}_{k_{t}}
\end{array}\right]\left[\begin{array}{ccc}
\boldsymbol{D}_{11} & \cdots & \boldsymbol{D}_{1 t} \\
\boldsymbol{D}_{12} & \cdots & \boldsymbol{D}_{2 t} \\
\vdots & & \vdots \\
\boldsymbol{D}_{t 1} & \cdots & \boldsymbol{D}_{t t}
\end{array}\right]} \\
=\left[\begin{array}{ccc}
\boldsymbol{D}_{11} & \cdots & \boldsymbol{D}_{1 t} \\
\boldsymbol{D}_{12} & \cdots & \boldsymbol{D}_{2 t} \\
\vdots & & \vdots \\
\boldsymbol{D}_{t 1} & \cdots & \boldsymbol{D}_{t t}
\end{array}\right]\left[\begin{array}{llll}
\lambda_{1} \boldsymbol{E}_{k_{1}} & & & \\
& \lambda_{2} \boldsymbol{E}_{k_{2}} & & \\
& & \ddots & \\
& & & \lambda_{t} \boldsymbol{E}_{k_{t}}
\end{array}\right]
\end{gathered}
$$

于是由分块矩阵乘法可知 $\boldsymbol{D}_{i j}=0, i \neq j$ ,从而有

$$
\boldsymbol{D}=\left[\begin{array}{llll}
\boldsymbol{D}_{11} & & & \\
& \boldsymbol{D}_{22} & & \\
& & \ddots & \\
& & & \boldsymbol{D}_{t t}
\end{array}\right]
$$

由 $\boldsymbol{D}$ 为正规矩阵可知 $\boldsymbol{D}_{j j}$ 为 $k_{j}$ 阶正规矩阵,那么必须存在 $k_{j}$ 阶酉矩阵 $\boldsymbol{V}_{j}, \boldsymbol{j}= 1,2, \cdots, t$ ,使得 $\boldsymbol{V}_{j}^{\mathrm{H}} \boldsymbol{D}_{j j} \boldsymbol{V}_{j}=\boldsymbol{A}_{j}, \boldsymbol{\Lambda}_{j}$ 为 $k_{j}$ 阶对角矩阵,记

$$
V=\left[\begin{array}{llll}
V_{1} & & & \\
& V_{2} & & \\
& & \ddots & \\
& & & V_{t}
\end{array}\right]
$$

那么有

$$
\begin{gathered}
{\left[\begin{array}{cccc}
\boldsymbol{V}_{1}^{\mathrm{H}} & & & \\
& \boldsymbol{V}_{2}^{\mathrm{H}} & & \\
& & \ddots & \\
& & & \boldsymbol{V}_{t}^{\mathrm{H}}
\end{array}\right]\left[\begin{array}{llll}
\boldsymbol{D}_{11} & & & \\
& \boldsymbol{D}_{22} & & \\
& & \ddots & \\
& & & \boldsymbol{D}_{t t}
\end{array}\right]\left[\begin{array}{llll}
\boldsymbol{V}_{1} & & & \\
& \boldsymbol{V}_{2} & & \\
& & \ddots & \\
& & &
\end{array}=\right.} \\
{\left[\begin{array}{llll}
u_{1} & & & \\
& u_{2} & & \\
& & \ddots & \\
& & & u_{n}
\end{array}\right]}
\end{gathered}
$$

记 $\boldsymbol{U}=\boldsymbol{W} \boldsymbol{V}$ ,因为 $\boldsymbol{W}, \boldsymbol{V}$ 都是西矩阵,所以 $\boldsymbol{U}$ 为酉矩阵,从而有

$$
\begin{aligned}
& \boldsymbol{U}^{\mathrm{H}} \boldsymbol{A} \boldsymbol{U}=\boldsymbol{V}^{\mathrm{H}}\left(\boldsymbol{U}_{1}^{\mathrm{H}} \boldsymbol{A} \boldsymbol{U}_{1}\right) \boldsymbol{V}=\left[\begin{array}{lll}
\lambda_{1} & & \\
& \ddots & \\
& & \lambda_{n}
\end{array}\right] \\
& \boldsymbol{U}^{\mathrm{H}} \boldsymbol{B} \boldsymbol{U}=\boldsymbol{V}^{\mathrm{H}}\left(\boldsymbol{U}_{1}^{\mathrm{H}} \boldsymbol{B} \boldsymbol{U}_{1}\right) \boldsymbol{V}=\left[\begin{array}{lll}
u_{1} & & \\
& \ddots & \\
& & u_{n}
\end{array}\right]
\end{aligned}
$$

由此定理立即可以得到下面的结论\\
(1)如果 $\boldsymbol{A}, ~ B$ 都是 Hermite 矩阵,且 $\boldsymbol{A B}=\boldsymbol{B A}$ ,那么存在酉矩阵 $\boldsymbol{U}$ 使得

$$
\boldsymbol{U}^{\mathrm{H}} \boldsymbol{A} \boldsymbol{U}=\left[\begin{array}{lll}
\lambda_{1} & & \\
& \ddots & \\
& & \lambda_{n}
\end{array}\right], \boldsymbol{U}^{\mathrm{H}} \boldsymbol{B} \boldsymbol{U}=\left[\begin{array}{lll}
u_{1} & & \\
& \ddots & \\
& & u_{n}
\end{array}\right]
$$

(2)如果 $\boldsymbol{A}, ~ \boldsymbol{B}$ 都是实对称矩阵,且 $\boldsymbol{A B}=\boldsymbol{B A}$ ,那么存在酉矩阵 $\boldsymbol{U}$ 使得

$$
\boldsymbol{U}^{\mathrm{H}} \boldsymbol{A} \boldsymbol{U}=\left[\begin{array}{lll}
\lambda_{1} & & \\
& \ddots & \\
& & \lambda_{n}
\end{array}\right], \boldsymbol{U}^{\mathrm{H}} \boldsymbol{B} \boldsymbol{U}=\left[\begin{array}{lll}
u_{1} & & \\
& \ddots & \\
& & u_{n}
\end{array}\right]
$$

\section*{§.7 Hermite 变换、正规变换}
定义3.7.1 设 $V$ 是一个酉空间, $\mathscr{A}$ 为 $V$ 上的一个线性变换,如果对任意的 $\boldsymbol{\alpha}, \boldsymbol{\beta} \in V$ 都有

$$
(\mathscr{A}(\boldsymbol{\alpha}), \boldsymbol{\beta})=(\boldsymbol{\alpha}, \mathscr{A}(\boldsymbol{\beta})),
$$

那么称 $\mathscr{A}$ 为 $V$ 的一个 Hermite 变换,或者自伴变换。\\
由定义可以看出,酉空间中的 Hermite 变换与欧氏空间中的对称变换很类似。\\
定理3.7.1 酉空间 $V$ 上的线性变换, $\mathscr{A}$ 是 Hermite 变换的充要条件。 $\mathscr{B}$ 在 $V$的任意一个标准正交基下的矩阵 $\boldsymbol{A}$ 满足

$$
\boldsymbol{A}^{\mathrm{H}}=\boldsymbol{A}
$$

即 $\boldsymbol{A}$ 为 Hermite 矩阵。\\
证明 必要性 设 $\varepsilon_{1}, \cdots, \varepsilon_{n}$ 是 $V$ 的一个标准正交基, $\mathscr{A}$ 在该基下对应的矩阵为 $\boldsymbol{A}=\left(a_{i j}\right)_{n \times n}$ ,于是有

$$
\left(\mathscr{A}\left(\varepsilon_{i}\right), \varepsilon_{j}\right)=a_{j i}, \quad\left(\mathscr{A}\left(\varepsilon_{j}\right), \varepsilon_{i}\right)=a_{i j}
$$

从而

$$
a_{j i}=\left(\mathscr{A}\left(\varepsilon_{i}\right), \varepsilon_{j}\right)=\left(\varepsilon_{i}, \mathscr{A}\left(\varepsilon_{j}\right)\right)=\overline{\left(\mathscr{A}\left(\varepsilon_{j}\right), \varepsilon_{i}\right)}=\overline{a_{i j}}
$$

这表明 $\boldsymbol{A}=\boldsymbol{A}^{\mathrm{H}}$ ,即 $\boldsymbol{A}$ 为 Hermite 矩阵。\\
充分性 设 $\boldsymbol{\varepsilon}_{1}, \cdots, \boldsymbol{\varepsilon}_{n}$ 是 $V$ 的一个标准正交基且 $\mathscr{A}$ 在该基下对应的矩阵 $\boldsymbol{A}$ 满足 $\boldsymbol{A}=\boldsymbol{A}^{\mathrm{H}}$ .在 $V$ 中任取两个向量

$$
\boldsymbol{\alpha}=\left(\varepsilon_{1}, \cdots, \varepsilon_{n}\right) \boldsymbol{X}, \quad \boldsymbol{\beta}=\left(\varepsilon_{1}, \cdots, \varepsilon_{n}\right) \boldsymbol{Y}
$$

那么

$$
(\mathscr{A}(\boldsymbol{\alpha}), \boldsymbol{\beta})=\boldsymbol{Y}^{\mathrm{H}} \boldsymbol{A} \boldsymbol{X}=\boldsymbol{Y}^{\mathrm{H}} \boldsymbol{A}^{\mathrm{H}} \boldsymbol{X}=(\boldsymbol{A} \boldsymbol{Y})^{\mathrm{H}} \boldsymbol{X}=(\boldsymbol{\alpha}, \mathscr{A}(\boldsymbol{\beta}))
$$

这说明 $\mathscr{A}$ 为 $V$ 上的一个 Hermite 变换。\\
定理3.7.2 西空间 $V$ 上的 Hermite 变换 $\mathscr{A}$ 的特征值为实数。\\
证明 设 $\lambda$ 是 $\mathscr{A}$ 的任意一个特征值, $\boldsymbol{\alpha}$ 为. $\mathscr{A}$ 的属于特征值 $\lambda$ 的特征向量。\\
由于

$$
\begin{aligned}
& (\mathscr{A}(\boldsymbol{\alpha}), \boldsymbol{\alpha})=(\lambda \boldsymbol{\alpha}, \boldsymbol{\alpha})=\lambda\|\boldsymbol{\alpha}\|^{2} \\
& (\boldsymbol{\alpha}, \mathscr{A}(\boldsymbol{\alpha}))=(\boldsymbol{\alpha}, \lambda \boldsymbol{\alpha})=\bar{\lambda}\|\boldsymbol{\alpha}\|^{2}
\end{aligned}
$$

所以 $\lambda\|\boldsymbol{\alpha}\|^{2}=\bar{\lambda}\|\boldsymbol{\alpha}\|^{2}$ .又 $\|\boldsymbol{\alpha}\|^{2} \neq 0$ ,于是有 $\lambda=\bar{\lambda}$ ,这表明 $\lambda$ 是一个实数.\\
类似地,有下面的定义和定理。\\
定义3.7.2 设 $V$ 是一个西空间, $\mathscr{A}$ 为 $V$ 上的一个线性变换,如果对任意的 $\boldsymbol{\alpha}, \boldsymbol{\beta} \in V$ 都有

$$
(\mathscr{A}(\boldsymbol{\alpha}), \boldsymbol{\beta})=-(\boldsymbol{\alpha}, \mathscr{B}(\boldsymbol{\beta}))
$$

那么,称 $\mathscr{A}$ 为 $V$ 的一个反 Hermite 变换。\\
定理3.7.3 西空间 $V$ 上的线性变换, $\mathscr{A}$ 是反 Hermite 变换当且仅当 $\mathscr{A}$ 在 $V$的任意一个标准正交基下的矩阵 $\mathscr{A}$ 满足

$$
\boldsymbol{A}^{\mathrm{H}}=-\boldsymbol{A}
$$

即 $\boldsymbol{A}$ 为反 Hermite 矩阵。\\
定义3.7.3 设 $V$ 是一个(欧氏)西空间, $\mathscr{A}$ 为 $V$ 上的一个线性变换,如果存在 $V$ 上的一个线性变换 $\mathscr{A}^{\mathrm{H}}$ ,使得

$$
(\mathscr{A}(\boldsymbol{\alpha}), \boldsymbol{\beta})=\left(\boldsymbol{\alpha}, \mathscr{A}^{\mathrm{H}}(\boldsymbol{\beta})\right) \quad \forall \boldsymbol{\alpha}, \boldsymbol{\beta} \in V
$$

那么称 $\mathscr{B}$ 有一个伴随变换 $\mathscr{A}^{\mathrm{H}}$ 。\\
利用高等代数的知识,可以证明,(欧氏)西空间 $V$ 上的每一个线性变换有且有唯一的一个伴随变换,这时,称其为 $\boldsymbol{A}$ 的伴随变换,这方面知识请读者参阅高等代数中的相关知识。

定理3.7.4 设 $V$ 是一个 $n$ 维(欧氏)西空间,$\varepsilon_{1}, \varepsilon_{2}, \cdots, \varepsilon_{n}$ 是 $V$ 的一个标准正交基, $\mathscr{A}$ 是 $V$ 上的线性变换,且 $\mathscr{A}$ 在基 $\varepsilon_{1}, \cdots, \varepsilon_{n}$ 下对应的矩阵为 $\boldsymbol{A}=\left(a_{i j}\right)$ ,那么 $\mathscr{A}$ 的伴随变换 $\mathscr{A}^{\mathrm{H}}$ 在 $\boldsymbol{\varepsilon}_{1}, \cdots, \boldsymbol{\varepsilon}_{n}$ 下的矩阵表示 $\boldsymbol{B}$ 为

$$
\boldsymbol{B}=\boldsymbol{A}^{\mathrm{H}}
$$

另外

$$
\overline{\left(\mathscr{A}^{\mathrm{H}}\left(\varepsilon_{j}\right), \varepsilon_{i}\right)}=\overline{b_{i j}}
$$

证明 由伴随变换 $\mathscr{A}^{\mathrm{H}}$ 定义可知

$$
\left(\mathscr{A}\left(\varepsilon_{i}\right), \varepsilon_{j}\right)=\left(\varepsilon_{i}, \mathscr{A}^{\mathrm{H}}\left(\varepsilon_{j}\right)\right) \quad(i, j=1,2, \cdots, n)
$$

又 $\left(\mathscr{A}\left(\varepsilon_{i}\right), \varepsilon_{j}\right)=a_{j i}$ .设 $\boldsymbol{B}=\left(b_{i j}\right)_{n \times n}$ ,那么有

$$
\left(\mathscr{B}\left(\boldsymbol{\varepsilon}_{i}\right), \boldsymbol{\varepsilon}_{j}\right)=\left(\boldsymbol{\varepsilon}_{i}, \mathscr{A B}^{\mathbf{H}}\left(\boldsymbol{\varepsilon}_{j}\right)\right)=\overline{\left(\mathscr{A}^{\mathbf{H}}\left(\boldsymbol{\varepsilon}_{j}\right), \boldsymbol{\varepsilon}_{i}\right)}=\overline{b_{i j}}
$$

从而 $a_{j i}=\bar{b}_{i j}, b_{i j}=\overline{a_{j i}}$ ,即 $\boldsymbol{B}=\boldsymbol{A}^{\mathbf{H}}$ 。\\
下面看几个关于伴随变换的例子。\\
例3.7.1 设 $\mathscr{A}$ 为欧氏空间 $V$ 上的一个对称变换,那么有 $\mathscr{A}^{\mathrm{H}}=\mathscr{A}$ 。因为根据对称变换的定义有

$$
(\mathscr{B}(\boldsymbol{\alpha}), \boldsymbol{\beta})=(\boldsymbol{\alpha}, \mathscr{A}(\boldsymbol{\beta})) \quad \forall \boldsymbol{\alpha}, \boldsymbol{\beta} \in V
$$

设 $\mathscr{A}$ 为欧氏空间 $V$ 上的一个反对称变换,那么有 $\mathscr{A}^{\mathrm{H}}=-\mathscr{A}$ 。根据反对称变换的定义有

$$
(\mathscr{A}(\boldsymbol{\alpha}), \boldsymbol{\beta})=-(\boldsymbol{\alpha}, \mathscr{A}(\boldsymbol{\beta}))=(\boldsymbol{\alpha},-\mathscr{A}(\boldsymbol{\beta}))
$$

例3.7.2 设 $\mathscr{A}$ 为西空间 $V$ 上的一个 Hermite 变换,那么有 $\mathscr{A}^{\mathrm{H}}=\mathscr{A}$ 。 Hermite变换也经常被称做自伴随变换。

设 $\mathscr{A}$ 为西空间 $V$ 上的一个反 Hermite 变换,那么有 $\mathscr{A}^{\mathrm{H}}=-\mathscr{A}$ 。\\
例3.7.3 设 $\mathscr{A}$ 为欧氏空间 $V$ 上的一个正交变换,那么有 $\mathscr{A}^{\mathrm{H}}=\mathscr{A}^{-1}$ ,由定义有

$$
\begin{aligned}
(\mathscr{A}(\boldsymbol{\alpha}), \boldsymbol{\beta}) & =\left(\mathscr{A}(\boldsymbol{\alpha}), \mathscr{A}\left(\mathscr{B}^{-1}(\boldsymbol{\beta})\right)\right) \\
& =\left(\boldsymbol{\alpha}, \mathscr{A}^{-1}(\boldsymbol{\beta})\right) \quad \forall \boldsymbol{\alpha}, \boldsymbol{\beta} \in V .
\end{aligned}
$$

例3.7.4 设 $\mathscr{A}$ 为西空间 $V$ 上的一个酉变换,那么有 $\mathscr{A}^{\mathrm{H}}=\mathscr{A}^{-1}$ 。\\
关于伴随变换有如下一些重要性质。\\
定理3.7.5 设 $V$ 是 $n$ 维(欧氏)西空间, $\mathcal{A}$ 和 $\mathscr{B}$ 都是 $V$ 上的线性变换,$k$ 为一个(实)复数。那么\\
(1)$(\mathscr{A}+\mathscr{B})^{\mathrm{H}}=\mathscr{A}^{\mathrm{H}}+\mathscr{B}^{\mathrm{H}}$\\
(2)$(k \cdot \mathscr{A})^{\mathrm{H}}=\bar{k} \cdot \mathscr{B}^{\mathrm{H}}$\\
(3)$(\mathscr{A} \mathscr{B})^{\mathrm{H}}=\mathscr{B}^{\mathrm{H}} \mathscr{A}^{\mathrm{H}}$\\
(4)$\left(\mathscr{A}^{\mathrm{H}}\right)^{\mathrm{H}}=\mathscr{A}$\\
证明(1)任取 $\boldsymbol{\alpha}, \boldsymbol{\beta} \in V$ ,有

$$
\begin{aligned}
((\mathscr{A}+\mathscr{B})(\boldsymbol{\alpha}), \boldsymbol{\beta}) & =(\mathscr{A}(\boldsymbol{\alpha})+\mathscr{B}(\boldsymbol{\alpha}), \boldsymbol{\beta}) \\
& =(\mathscr{B}(\boldsymbol{\alpha}), \boldsymbol{\beta})+(\mathscr{B}(\boldsymbol{\alpha}), \boldsymbol{\beta}) \\
& =\left(\boldsymbol{\alpha}, \mathscr{A}^{\mathrm{H}}(\boldsymbol{\beta})\right)+\left(\boldsymbol{\alpha}, \mathscr{B}^{\mathrm{H}}(\boldsymbol{\beta})\right) \\
& =\left(\boldsymbol{\alpha}, \mathscr{A}^{\mathrm{H}}(\boldsymbol{\beta})+\mathscr{B}^{\mathrm{H}}(\boldsymbol{\beta})\right) \\
& =\left(\boldsymbol{\alpha},\left(\mathscr{A}^{\mathrm{H}}+\mathscr{B}^{\mathrm{H}}\right)(\boldsymbol{\beta})\right)
\end{aligned}
$$

由伴随变换的唯一性可得,$(\mathscr{A}+\mathscr{B})^{\mathrm{H}}=\mathscr{A}^{\mathrm{H}}+\mathscr{B}^{\mathrm{H}}$ 。\\
(2)任取 $\boldsymbol{\alpha}, \boldsymbol{\beta} \in V$ ,有

$$
\begin{gathered}
((k \mathscr{A})(\boldsymbol{\alpha}), \boldsymbol{\beta})=(k \mathscr{B}(\boldsymbol{\alpha}), \boldsymbol{\beta})=k(\mathscr{A}(\boldsymbol{\alpha}), \boldsymbol{\beta}) \\
=k\left(\boldsymbol{\alpha}, \mathscr{A}^{\mathrm{H}}(\boldsymbol{\beta})\right)=\left(\boldsymbol{\alpha}, \bar{k}, \mathscr{B}^{\mathrm{H}}(\boldsymbol{\beta})\right)=\left(\boldsymbol{\alpha},\left(\bar{k}, \mathscr{A}^{\mathrm{H}}\right)(\boldsymbol{\beta})\right)
\end{gathered}
$$

于是由伴随变换的唯一性可得,$(k, \mathscr{A})^{\mathrm{H}}=\vec{k}, \mathscr{A}^{\mathrm{H}}$ 。\\
(3)任取 $\boldsymbol{\alpha}, \boldsymbol{\beta} \in V$ ,有

$$
\begin{aligned}
((\mathscr{A} \mathscr{B})(\boldsymbol{\alpha}), \boldsymbol{\beta}) & =\left(\mathscr{B}(\boldsymbol{\alpha}), \mathscr{B}^{\mathrm{H}}(\boldsymbol{\beta})\right) \\
& =\left(\boldsymbol{\alpha}, \mathscr{B}^{\mathrm{H}} \mathscr{B}^{\mathrm{H}}(\boldsymbol{\beta})\right) \\
& =\left(\boldsymbol{\alpha},\left(\mathscr{B}^{\mathrm{H}} \mathscr{A}^{\mathrm{H}}\right)(\boldsymbol{\beta})\right)
\end{aligned}
$$

从而有 $(\mathscr{A} \mathscr{B})^{\mathrm{H}}=\mathscr{B}^{\mathrm{H}} \mathscr{A}^{\mathrm{H}}$ 。\\
(4)任取 $\boldsymbol{\alpha}, \boldsymbol{\beta} \in V$ ,有

$$
\left(\mathscr{A}^{\mathrm{H}}(\boldsymbol{\alpha}), \boldsymbol{\beta}\right)=\overline{\left(\boldsymbol{\beta}, \mathscr{A}^{\mathrm{H}}(\boldsymbol{\alpha})\right)}=\overline{(\mathscr{A}(\boldsymbol{\beta}), \boldsymbol{\alpha})}=(\boldsymbol{\alpha}, \mathscr{A}(\boldsymbol{\beta}))
$$

因此 $\left(\mathscr{A}^{\mathrm{H}}\right)^{\mathrm{H}}=\mathscr{A}$ 。\\
定理3.7.6 设 $V$ 是 $n$ 维(欧氏)酉空间, $\mathscr{B}$ 是 $V$ 上的一个线性变换,如果 $W$是 $A$ 的不变子空间,那么 $W_{\perp}$ 也是 $\mathscr{B}^{\mathrm{H}}$ 的不变子空间。

证明 任取 $\boldsymbol{\alpha} \in W_{\perp}$ .要证明 $\mathscr{A}^{\mathrm{H}}(\boldsymbol{\alpha}) \in W_{\perp}$ .对任意的 $\boldsymbol{\beta} \in W$ ,由已知条件有 $\mathscr{B}(\boldsymbol{\beta}) \in W$ .从而有

$$
\left(\boldsymbol{\beta}, \mathscr{A}^{\mathrm{H}}(\boldsymbol{\alpha})\right)=(\mathscr{A}(\boldsymbol{\beta}), \boldsymbol{\alpha})=0
$$

所以 $\mathscr{A}^{\mathrm{H}}(\boldsymbol{\alpha}) \in W_{\perp}$ 。\\
定义3.7.4 设 $V$ 是(欧氏)酉空间, $\mathscr{A}$ 是 $V$ 上的线性变换,如果 $\mathscr{A}$ 满足

$$
\mathscr{A}^{\mathrm{H}} \mathscr{A}=\mathscr{A}^{\mathrm{H}}
$$

那么称 $\mathscr{A}$ 是正规变换。\\
例如,欧氏空间上的正交变换,(反)对称变换,酉空间上的西变换,(反) Hermite 变换都是正规变换。

定理3.7.7 设 $V$ 是一个 $n$ 维(欧氏)西空间, $\mathscr{A}$ 是 $V$ 上的一个线性变换, $\mathscr{A}$是正规变换,当且仅当 $\mathscr{A}$ 在 $V$ 的任意一个标准正交基下的矩阵是正规矩阵。

证明 任取 $V$ 的一个标准正交基 $\varepsilon_{1}, \cdots, \varepsilon_{n}$ ,设 $\mathscr{A}$ 在这个基下对应的矩阵为 $A$ ,那么 $\mathscr{A}^{\mathrm{H}}$ 在这个基下对应的矩阵为 $A^{\mathrm{H}}$ ,又

$$
\mathscr{A} \mathscr{A}^{\mathrm{H}}=\mathscr{A}^{\mathrm{H}} \mathscr{A} \Leftrightarrow A A^{\mathrm{H}}=A^{\mathrm{H}} A
$$

于是有 $\mathscr{A}$ 是 $V$ 上的正规变换,当且仅当 $\boldsymbol{A}$ 为正规矩阵。\\
在§3.6节中,已经证明了:正规矩阵一定西相似于一个对角矩阵。由此可以得到

定理3.7.8 设 $\mathscr{A}$ 是西空间 $V$ 上的一个正规变换。那么存在 $V$ 的一个标准正交基,使得 $\mathscr{A}$ 在这个基下对应的矩阵为对角矩阵。即西空间上的正规变换是可对角化的线性变换。

此外,设, $\mathscr{A}$ 是西空间 $V$ 上的一个线性变换且存在 $V$ 的一个标准正交基 $\varepsilon_{1}, \cdots$ , $\varepsilon_{n}$ ,使得 $\mathscr{A}$ 在这个基下对应的矩阵 $\boldsymbol{A}$ 为对角矩阵 $\operatorname{diag}\left\{\lambda_{1}, \cdots, \lambda_{n}\right\}$ ,那么 $\mathscr{A}$ 的伴随变换, $\mathscr{A}^{\mathrm{H}}$ 在这个基下对应的矩阵为

$$
\boldsymbol{A}^{\mathrm{H}}=\operatorname{diag}\left\{\bar{\lambda}_{1}, \cdots, \bar{\lambda}_{n}\right\}
$$

于是有 $\boldsymbol{A} \boldsymbol{A}^{\mathrm{H}}=\boldsymbol{A}^{\mathrm{H}} \boldsymbol{A}$ ,这表明 $. \mathscr{A B}^{\mathrm{H}}=\mathscr{A}^{\mathrm{H}} \mathscr{A}$ ,即 $\mathscr{A}$ 为 $V$ 的一个正规变换,由此有\\
定理3.7.9 西空间 $V$ 上的一个线性变换, $\mathscr{A}$ 为正规变换,当且仅当在 $V$ 中存在一个标准正交基,使得 $\mathscr{A}$ 在这个基下对应的矩阵为对角矩阵。

\section*{§3.8 Hermite 矩阵、Hermite 二次齐式}
Hermite 矩阵是特征值全为实数的正规矩阵,是实对称矩阵的推广并有相似之处。它在物理、力学及工程中有广泛应用。

\section*{一、 Hermite 矩阵、实对称矩阵}
例3.8.1 对于给定的 $n$ 阶矩阵 $A$ ,根据定义不难证明:\\
(1) $\boldsymbol{A}+\boldsymbol{A}^{\mathrm{H}}, \boldsymbol{A} \boldsymbol{A}^{\mathrm{H}}, \boldsymbol{A}^{\mathrm{H}} \boldsymbol{A}$ 是 Hermite 矩阵;\\
(2) $\boldsymbol{A}-\boldsymbol{A}^{\mathrm{H}}$ 是反 Hermite 矩阵。\\
定理3.8.1 $A^{H}=A \Leftrightarrow a_{i j}=\bar{a}_{j i} \quad \Leftrightarrow R e a_{i j}=R e a_{j i}$\\
(证略)\\
Hermite 矩阵的简单性质:\\
(1)已知 $\boldsymbol{A}$ 是 Hermite 矩阵,则 $\boldsymbol{A}^{k}$ 也是 Hermite 矩阵( $k$ 为正整数);\\
(2)已知 $\boldsymbol{A}$ 是可逆 Hermite 矩阵,则 $\boldsymbol{A}^{-1}$ 也是 Hermite 矩阵;\\
(3)已知 $\boldsymbol{A}$ 是 Hermite(反 Hermite)矩阵,则 $\mathrm{i} \boldsymbol{A}$ 是反 Hermite(Hermite)矩阵 ( $\mathrm{i}=\sqrt{-1}$ );\\
(4)已知 $\boldsymbol{A} 、 \boldsymbol{B}$ 是 Hermite 矩阵,则 $k \boldsymbol{A}+l \boldsymbol{B}$ 是 Hermite 矩阵( $k 、 l$ 为实数);\\
(5)已知 $\boldsymbol{A} 、 \boldsymbol{B}$ 是 Hermite 矩阵,则 $\boldsymbol{A} \boldsymbol{B}$ 是 Hermite 矩阵的充要条件是\\
$\boldsymbol{A B}=\boldsymbol{B} \boldsymbol{A}$.\\
定理3.8.2 若 $\boldsymbol{A}$ 是 $n$ 阶复矩阵,则\\
(1) $\boldsymbol{A}$ 是 Hermite 矩阵的充要条件是对于任意 $\boldsymbol{X} \in C^{n}, \boldsymbol{X}^{\mathrm{H}} \boldsymbol{A} \boldsymbol{X}$ 是实数;\\
(2) $\boldsymbol{A}$ 是 Hermite 矩阵的充要条件是对于任意 $n$ 阶方阵 $\boldsymbol{S}, \boldsymbol{S}^{\mathrm{H}} \boldsymbol{A} \boldsymbol{S}$ 是 Hermite 矩阵。

证明(1)必要性 因为 $\boldsymbol{X}^{\mathrm{H}} \boldsymbol{A} \boldsymbol{X}$ 是数,故

$$
\overline{\left(\boldsymbol{X}^{\mathrm{H}} \boldsymbol{A} \boldsymbol{X}\right)}=\left(\boldsymbol{X}^{\mathrm{H}} \boldsymbol{A} \boldsymbol{X}\right)^{\mathrm{H}}=\boldsymbol{X}^{\mathrm{H}} \boldsymbol{A}^{\mathrm{H}} \boldsymbol{X}=\boldsymbol{X}^{\mathrm{H}} \boldsymbol{A} \boldsymbol{X}
$$

这表明 $\boldsymbol{X}^{\mathrm{H}} \boldsymbol{A X}$ 是实数。\\
充分性 因为对于任何 $\boldsymbol{X}, \boldsymbol{Y} \in C^{n}$ ,有实数

$$
\begin{gathered}
(\boldsymbol{X}+\boldsymbol{Y})^{\mathrm{H}} \boldsymbol{A}(\boldsymbol{X}+\boldsymbol{Y})=\left(\boldsymbol{X}^{\mathrm{H}}+\boldsymbol{Y}^{\mathrm{H}}\right) \boldsymbol{A}(\boldsymbol{X}+\boldsymbol{Y}) \\
=\boldsymbol{X}^{\mathrm{H}} \boldsymbol{A} \boldsymbol{X}+\boldsymbol{Y}^{\mathrm{H}} \boldsymbol{A} \boldsymbol{Y}+\boldsymbol{X}^{\mathrm{H}} \boldsymbol{A} \boldsymbol{Y}+\boldsymbol{Y}^{\mathrm{H}} \boldsymbol{A} \boldsymbol{X}
\end{gathered}
$$

而由假设知 $\boldsymbol{X}^{\mathrm{H}} \boldsymbol{A} \boldsymbol{X}, \boldsymbol{Y}^{\mathrm{H}} \boldsymbol{A} \boldsymbol{Y}$ 是实数,于是对于任意 $\boldsymbol{X}, \boldsymbol{Y} \in \boldsymbol{C}^{n}$ ,

$$
\boldsymbol{X}^{\mathrm{H}} \boldsymbol{A} \boldsymbol{Y}+\boldsymbol{Y}^{\mathrm{H}} \boldsymbol{A} \boldsymbol{X}=\text { 实数 }
$$

$$
\boldsymbol{X}=\left[\begin{array}{c}
0 \\
\vdots \\
0 \\
1 \\
0 \\
\vdots \\
0
\end{array}\right] \longrightarrow j \text { 位, } \quad \boldsymbol{Y}=\left[\begin{array}{c}
0 \\
\vdots \\
0 \\
1 \\
0 \\
\vdots \\
0
\end{array}\right] \longrightarrow k \text { 位 }
$$

代人上式得

$$
a_{j k}+a_{k j}=\text { 实数 }
$$

此即表明

$$
\operatorname{Im}\left(a_{j k}\right)=-\operatorname{Im}\left(a_{k j}\right)
$$

$\operatorname{Im}\left(a_{j k}\right)$ 表示 $a_{j k}$ 的虚部\\
又取

$$
\boldsymbol{X}=\left[\begin{array}{c}
0 \\
\vdots \\
0 \\
\mathrm{i} \\
0 \\
\vdots \\
0
\end{array}\right] \longrightarrow j \text { 位, } \quad \boldsymbol{Y}=\left[\begin{array}{c}
0 \\
\vdots \\
0 \\
1 \\
0 \\
\vdots \\
0
\end{array}\right] \longrightarrow k \text { 位 } \quad(\mathrm{i}=\sqrt{-1})
$$

可得

$$
a_{j k}-a_{k j}=\text { 纯虚数 }
$$

此即表明

$$
\operatorname{Re}\left(a_{j k}\right)=\operatorname{Re}\left(a_{k j}\right)
$$

因此

$$
a_{j k}=\bar{a}_{k j}, \quad \boldsymbol{A}=A^{\mathbf{H}}
$$

(2)的证明请读者自己完成。\\
Hermite 矩阵是特征值为实数的正规矩阵,因此根据定理3.6.4可得\\
定理 3.8.3 设 $\boldsymbol{A} \in C^{n \times n}$ ,则 $\boldsymbol{A}$ 是 Hermite 矩阵的充要条件是存在 $\boldsymbol{U} \in U^{n \times n}$ ,使得

$$
\boldsymbol{U}^{\mathrm{H}} \boldsymbol{A} \boldsymbol{U}=\operatorname{diag}\left(\lambda_{1}, \lambda_{2}, \cdots, \lambda_{n}\right)
$$

其中 $\lambda_{1}, \lambda_{2}, \cdots, \lambda_{n}$ 是实数。\\
可将定理3.8.3简言为Hermite矩阵西相似于实对角矩阵。\\
实对称矩阵的特征值全是实数,它的特征向量全是实特征向量(这是与 Hermite 矩阵不同之处),所以实对称矩阵正交相似于实对角矩阵,此即

定理 3.8.4 设 $\boldsymbol{A} \in R^{n \times n}$ ,则 $\boldsymbol{A}$ 是实对称矩阵的充要条件是存在 $\boldsymbol{Q} \in \boldsymbol{E}^{n \times n}$ ,使得

$$
\boldsymbol{Q}^{\mathrm{T}} \boldsymbol{A} \boldsymbol{Q}=\operatorname{diag}\left(\lambda_{1}, \lambda_{2}, \cdots, \lambda_{n}\right)
$$

其中 $\lambda_{1}, \lambda_{2}, \cdots, \lambda_{n}$ 是实数。\\
若 Hermite 矩阵 $\boldsymbol{A}$ 的秩为 $r$ ,不妨设 $\lambda_{r+1}=\lambda_{r+2}=\cdots=\lambda_{n}=0$ ,可以证明:\\
定理3.8.5 设 $\boldsymbol{A}$ 是秩为 $r$ 的 $n$ 阶 Hermite 矩阵,则存在 $\boldsymbol{P} \in C_{n}^{n \times n}$ ,满足

$$
\boldsymbol{P}^{\mathrm{H}} \boldsymbol{A} \boldsymbol{P}=\operatorname{diag}\left(b_{1}, b_{2}, \cdots, b_{r}, 0, \cdots, 0\right)
$$

其中 $b_{1}, b_{2}, \cdots, b_{r}$ 为实数。\\
例 3.8 .2 已知正规矩阵

$$
A=\left[\begin{array}{ccc}
1 & 0 & 2 i \\
0 & 3 & 0 \\
-2 i & 0 & 1
\end{array}\right]
$$

试求西矩阵 $\boldsymbol{U}$ ,使得 $\boldsymbol{U}^{\mathrm{H}} \boldsymbol{A} \boldsymbol{U}$ 为对角矩阵。\\
解:$|\lambda E-A|=(\lambda+1)(\lambda-3)^{2}$\\
$\boldsymbol{A}$ 的特征值 $\lambda_{1}=-1, \lambda_{2}=\lambda_{3}=3$\\
当 $\lambda_{1}=-1$ 时,$\lambda_{1} E-\boldsymbol{A}=\left[\begin{array}{ccc}-2 & 0 & -2 i \\ 0 & -4 & 0 \\ -2 i & 0 & -2\end{array}\right] \rightarrow\left[\begin{array}{lll}1 & 0 & i \\ 0 & 1 & 0 \\ 0 & 0 & 0\end{array}\right]$ .

$$
x_{1}=-i x_{3}, x_{2}=0
$$

故 $\alpha_{1}=(1,0, i)^{\mathrm{T}}$ ,单位化得 $v_{1}=\left(\frac{1}{\sqrt{2}}, 0, \frac{i}{\sqrt{2}}\right)^{\mathrm{T}}$\\
当 $\lambda_{2}=\lambda_{3}=3$ 时,$\lambda_{2} E-\boldsymbol{A}=\left[\begin{array}{ccc}2 & 0 & -2 i \\ 0 & 0 & 0 \\ 2 i & 0 & -2\end{array}\right] \rightarrow\left[\begin{array}{ccc}1 & 0 & -i \\ 0 & 0 & 0 \\ 0 & 0 & 0\end{array}\right]$ .

$$
x_{1}=i x_{3}
$$

故 $\quad \alpha_{2}=(i, 0,1)^{\mathrm{T}}, \alpha_{3}=(0,1,0)^{\mathrm{T}}$ .\\
$\alpha_{2}$ 与 $\alpha_{3}$ 已经正交化.单位化后得 $v_{2}=\left(\frac{i}{\sqrt{2}}, 0, \frac{1}{\sqrt{2}}\right)^{\mathrm{T}}, v_{3}=\alpha_{3}=(0,1,0)^{\mathrm{T}}$ .\\
命 $\boldsymbol{U}=\left(v_{1} v_{2} v_{3}\right)=\left[\begin{array}{ccc}\frac{1}{\sqrt{2}} & \frac{i}{\sqrt{2}} & 0 \\ 0 & 0 & 1 \\ \frac{i}{\sqrt{2}} & \frac{1}{\sqrt{2}} & 0\end{array}\right]$

容易验证

$$
\boldsymbol{U}^{\mathrm{H}} \boldsymbol{A} \boldsymbol{U}=\left[\begin{array}{lll}
-1 & & \\
& 3 & \\
& & 3
\end{array}\right]
$$

\section*{二、Hermite 二次齐式、实二次齐式}
实对称矩阵能运用到实二次齐式上,是因为与实对称矩阵相合的是实对称矩阵。根据定理3.8.2,与 Hermite 矩阵复相合的是 Hermite 矩阵,因此,Hermite 矩阵成为讨论与之相应的二次齐式的有用数学工具。

由 $n$ 个复变量 $x_{1}, x_{2}, \cdots, x_{n}$ ,系数为复数的二次齐次复多项式

若命

$$
\begin{gathered}
f\left(x_{1}, x_{2}, \cdots, x_{n}\right)=\sum_{i, j=1}^{n} a_{i j} \bar{x}_{i} x_{j} \quad\left(\text { 规定 } \bar{a}_{i j}=a_{j i}\right) \\
X=\left(x_{1}, x_{2}, \cdots, x_{n}\right)^{\mathrm{T}} \in C^{n} \\
A=\left[\begin{array}{cccc}
a_{11} & a_{12} & \cdots & a_{1 n} \\
a_{21} & a_{22} & \cdots & a_{2 n} \\
\vdots & \vdots & & \vdots \\
a_{n 1} & a_{n 2} & \cdots & a_{n n}
\end{array}\right]
\end{gathered}
$$

则 $\boldsymbol{A}^{\mathrm{H}}=\boldsymbol{A}$ ,于是

$$
f\left(x_{1}, x_{2}, \cdots, x_{n}\right)=\boldsymbol{X}^{\mathrm{H}} \boldsymbol{A} \boldsymbol{X}
$$

称 $\boldsymbol{X}^{\mathrm{H}} \boldsymbol{A} \boldsymbol{X}$ 是 Hermite 二次齐式.若作可逆线性变换 $\boldsymbol{X}=\boldsymbol{C} \boldsymbol{Y}$ ,则

$$
f\left(x_{1}, x_{2}, \cdots, x_{n}\right)=\boldsymbol{X}^{\mathrm{H}} \boldsymbol{A X}=\boldsymbol{Y}^{\mathrm{H}}\left(\mathbf{C}^{\mathrm{H}} \boldsymbol{A C}\right) \boldsymbol{Y}=\boldsymbol{Y}^{\mathrm{H}} \boldsymbol{B} \boldsymbol{Y}
$$

显然, $\boldsymbol{B}=\mathbf{C}^{\mathrm{H}} \boldsymbol{A} \boldsymbol{C}$ ,且有 $\boldsymbol{B}^{\mathrm{H}}=\boldsymbol{B}$ .\\
例如,复二次齐次多项式 $f\left(x_{1}, x_{2}\right)=2 x_{1} \bar{x}_{1}+(1+\mathrm{i}) x_{1} \bar{x}_{2}+(1-\mathrm{i}) \bar{x}_{1} x_{2}+3 x_{2} \bar{x}_{2}$是 Hermite 二次齐式,因为 $f\left(x_{1}, x_{2}\right)=\boldsymbol{X}^{\mathrm{H}} \boldsymbol{A} \boldsymbol{X}$ ,其中 $\boldsymbol{X}=\left(x_{1}, x_{2}\right)^{\mathrm{T}}$ 且 $\boldsymbol{A}= \left[\begin{array}{cc}2 & 1-\mathrm{i} \\ 1+\mathrm{i} & 3\end{array}\right]$ 是 Hermite 矩阵。

注:任何一个实二次齐次多项式都可以写成实二次型。但是一个复二次齐次多项式不一定是一个 Hermite 二次型。例如 $x_{1} \bar{x}_{1}+3 x_{1} \bar{x}_{2}+5 \bar{x}_{1} x_{2}+x_{2} \bar{x}_{2}$ 就不是 Hermite 二次型。根据定理 3.8.3 可得

定理 3.8.6 对于 Hermite 二次齐式 $f\left(x_{1}, \cdots, x_{n}\right)=\boldsymbol{X}^{\mathrm{H}} \boldsymbol{A} \boldsymbol{X}$ ,存在西变换 $\boldsymbol{X}= \boldsymbol{U Y}$ ,使得二次齐式成为标准形

$$
f(\boldsymbol{X})=\lambda_{1} y_{1} \bar{y}_{1}+\lambda_{2} y_{2} \bar{y}_{2}+\cdots+\lambda_{n} y_{n} \bar{y}_{n}
$$

$\lambda_{i} \in \mathbf{R}$ 是 $\boldsymbol{A}$ 的特征值.\\
根据定理 3.8.5 可得\\
定理 3.8.7 秩为 $r$ 的 Hermite 二次齐式 $f\left(x_{1}, x_{2}, \cdots, x_{n}\right)=\boldsymbol{X}^{\mathrm{H}} \boldsymbol{A} \boldsymbol{X}$ ,存在可逆线性变换 $\boldsymbol{X}=\boldsymbol{P Y}$ ,使得二次齐式化为标准形

$$
f(\boldsymbol{X})=b_{1} y_{1} \bar{y}_{1}+b_{2} y_{2} \bar{y}_{2}+\cdots+b_{r} y_{r} \bar{y}_{r}
$$

其中 $b_{1}, b_{2}, \cdots, b_{r}$ 为实数。\\
例 3.8 . 3 已知 Hermite 二次型

$$
f\left(x_{1}, x_{2}, x_{3}\right)=\frac{1}{2} \bar{x}_{1} x_{1}+\frac{3}{2} i \bar{x}_{1} x_{3}+2 \bar{x}_{2} x_{2}-\frac{3}{2} i \bar{x}_{3} x_{1}+\frac{1}{2} \bar{x}_{3} x_{3}
$$

求酉变换 $\boldsymbol{Z}=\boldsymbol{U y}$ 将 $f\left(x_{1}, x_{2}, x_{3}\right)$ 变为 Hermite 标准二次型。\\
解:所给 Hermite 二次型 $f\left(x_{1}, x_{2}, x_{3}\right)$ 对应的 Hermite 矩阵

$$
A=\left[\begin{array}{ccc}
\frac{1}{2} & 0 & \frac{3}{2} i \\
0 & 2 & 0 \\
-\frac{3}{2} i & 0 & \frac{1}{2}
\end{array}\right]
$$

于是

$$
f\left(x_{1}, x_{2}, x_{3}\right)=\boldsymbol{X}^{\mathrm{H}} \boldsymbol{A} \boldsymbol{X}
$$

其中 $\boldsymbol{X}=\left(x_{1}, x_{2}, x_{3}\right)^{\mathrm{T}}$ 。由于 $\boldsymbol{A}$ 为一个 Hermite 矩阵,所以 $\boldsymbol{A}$ 可以酉对角化。

$$
|\lambda E-A|=(\lambda+1)(\lambda-2)^{2}
$$

$A$ 的特征值:$\lambda_{1}=\lambda_{2}=2, \lambda_{3}=-1$\\
$A$ 的特征值 $\lambda_{1}=\lambda_{2}=2$ 的正交单位特征向量:$\alpha_{1}=\left(\frac{i}{\sqrt{2}}, 0, \frac{1}{\sqrt{2}}\right)^{\mathrm{T}}, \alpha_{2}=(0,1$ , $0)^{\mathrm{T}}$ . $\boldsymbol{A}$ 的特征值 $\lambda_{3}=-1$ 的单位特征向量.$\alpha_{3}=\left(-\frac{i}{\sqrt{2}}, 0, \frac{1}{\sqrt{2}}\right)^{\mathrm{T}}$ ,于是

$$
U=\left(\alpha_{1}, \alpha_{2}, \alpha_{3}\right)=\left[\begin{array}{ccc}
\frac{i}{\sqrt{2}} & 0 & -\frac{i}{\sqrt{2}} \\
0 & 1 & 0 \\
\frac{1}{\sqrt{2}} & 0 & \frac{1}{\sqrt{2}}
\end{array}\right]
$$

容易验算

$$
\boldsymbol{U}^{\mathrm{H}} \boldsymbol{A} \boldsymbol{U}=\left[\begin{array}{lll}
2 & & \\
& 2 & \\
& & -1
\end{array}\right]
$$

命 $\boldsymbol{X}=\boldsymbol{U} y$ ,其中 $y=\left(y_{1}, y_{2}, y_{3}\right)$ ,代人二次型得

$$
\begin{aligned}
f\left(x_{1}, x_{2}, x_{3}\right) & =\boldsymbol{X}^{\mathrm{H}} \boldsymbol{A} \boldsymbol{X}=y^{\mathrm{H}}\left(\boldsymbol{U}^{\mathrm{H}} \boldsymbol{A} \boldsymbol{U}\right) \boldsymbol{y} \\
& =2 \bar{y}_{1} y_{1}+2 \bar{y}_{2} y_{2}-\bar{y}_{3} y_{3}
\end{aligned}
$$

\section*{§3.9 正定二次齐式、正定 Hermite 矩阵}
在 Hermite 二次齐式和实二次齐式中,正定二次齐式是一类十分重要的二次齐式。

定义3.9.1 给定 Hermite 二次齐式。

$$
f(\boldsymbol{X})=f\left(x_{1}, x_{2}, \cdots, x_{n}\right)=\sum_{i=1}^{n} \sum_{j=1}^{n} a_{i j} \bar{x}_{i} x_{j}=\boldsymbol{X}^{\mathrm{H}} \boldsymbol{A} \boldsymbol{X}
$$

$\boldsymbol{X}=\left(x_{1}, x_{2}, \cdots, x_{n}\right)^{\mathrm{T}}$ ,如果对任一组不全为零的复数 $x_{1}, x_{2}, \cdots, x_{n}$ ,都有 $f\left(x_{1}\right.$ , $\left.x_{2}, \cdots, x_{n}\right)>0(\geqslant 0)$ ,则称该二次齐式是正定的(半正定的).并称相对应的 Hermite 矩阵 $A$ 是正定的(半正定的).

例如,$f\left(x_{1}, x_{2}, x_{3}\right)=2 \bar{x}_{1} x_{1}+3 \bar{x}_{2} x_{2}+\bar{x}_{3} x_{3}$ 是正定的。 $f\left(x_{1}, x_{2}, x_{3}\right)=3 \bar{x}_{1} x_{1}+ 2 \bar{x}_{2} x_{2}$ 是半正定的.

如果对于任一组不全为零的复数 $x_{1}, x_{2}, \cdots, x_{n}$ 都有 $f\left(x_{1}, x_{2}, \cdots, x_{n}\right)<0(\leqslant$ 0 ),则称该二次齐式是负定的(半负定的).并称相对应的 Hermite 矩阵 $\boldsymbol{A}$ 是负定的(半负定的).

显然,若 Hermite 二次齐式 $f\left(x_{1}, x_{2}, \cdots, x_{n}\right)$ 是正定的(负定的),则 $-f\left(x_{1}\right.$ , $x_{2}, \cdots, x_{n}$ )是负定的(正定的)。

为了证明下述定理唯一性的需要,先介绍引理。\\
引理3.9.1 若 $\boldsymbol{A}$ 是正线上三角阵,又是西矩阵,则 $\boldsymbol{A}$ 是单位阵。\\
证明作为练习留给读者。\\
引理 3.9.2 Hermite 二次型 $f(\boldsymbol{X})=\boldsymbol{X}^{\mathrm{H}} \boldsymbol{A} \boldsymbol{X}$ 的正定性(或负定性、半正定性、半负定性)经满秩线性变换 $\boldsymbol{X}=\boldsymbol{P Y}$ 下保持不变。(正因为有此性质,定义 3.9.1 才合理)。(证略)

定理 3.9.1 对于 Hermite 二次齐式,$f(\boldsymbol{X})=\boldsymbol{X}^{\mathrm{H}} \boldsymbol{A} \boldsymbol{X}, \boldsymbol{X} \in C^{n}$ ,下列命题是等价的:\\
(1)$f(\boldsymbol{X})$ 是正定的;\\
(2)对于任何 $n$ 阶可逆矩阵 $\boldsymbol{P}$ ,都有 $\boldsymbol{P}^{\mathrm{H}} \boldsymbol{A P}$ 为正定矩阵;\\
(3) $\boldsymbol{A}$ 的 $n$ 个特征值全大于零;\\
(4)存在 $n$ 阶可逆矩阵 $\boldsymbol{P}$ ,使得 $\boldsymbol{P}^{\mathrm{H}} \boldsymbol{A P}=\boldsymbol{E}$ ;\\
(5)存在 $n$ 阶可逆矩阵 $\boldsymbol{Q}$ ,使得 $\boldsymbol{A}=\boldsymbol{Q}^{\mathrm{H}} \boldsymbol{Q}$ ;\\
(6)存在正线上三角矩阵 $\boldsymbol{R}$ ,使得 $\boldsymbol{A}=\boldsymbol{R}^{\mathrm{H}} \boldsymbol{R}$ ,且分解是唯一的.\\
证明 遵循 $(1) \Rightarrow(2) \Rightarrow(3) \Rightarrow(4) \Rightarrow(5) \Rightarrow(6) \Rightarrow(1)$ 路线进行证明.\\
$(1) \Rightarrow(2) \quad$ 即引理 3.9.2.\\
(2)$\Rightarrow$(3)对于 Hermite 矩阵 $\boldsymbol{A}$ ,存在西矩阵 $\boldsymbol{U}$ ,满足

$$
\boldsymbol{U}^{-1} \boldsymbol{A} \boldsymbol{U}=\boldsymbol{U}^{\mathrm{H}} \boldsymbol{A} \boldsymbol{U}=\operatorname{diag}\left(\lambda_{1}, \lambda_{2}, \cdots, \lambda_{n}\right)
$$

由(2)知 $\operatorname{diag}\left(\lambda_{1}, \lambda_{2}, \cdots, \lambda_{n}\right)$ 是正定的,所以 $\lambda_{1} \bar{x}_{1} x_{1}+\lambda_{2} \bar{x}_{2} x_{2}+\cdots+\lambda_{n} \bar{x}_{n} x_{n}>0$ ,于是 $\lambda_{1}, \lambda_{2}, \cdots, \lambda_{n}$ 全大于零。\\
(3)$\Rightarrow$(4)因为

$$
\boldsymbol{U}^{-1} \boldsymbol{A} \boldsymbol{U}=\boldsymbol{U}^{\mathrm{H}} \boldsymbol{A} \boldsymbol{U}=\operatorname{diag}\left(\lambda_{1}, \lambda_{2}, \cdots, \lambda_{n}\right)
$$

命 $\boldsymbol{P}_{1}=\operatorname{diag}\left(\frac{1}{\sqrt{\lambda_{1}}}, \frac{1}{\sqrt{\lambda_{2}}}, \cdots, \frac{1}{\sqrt{\lambda_{n}}}\right)$ ,则

$$
\boldsymbol{P}_{1}^{\mathrm{H}} \boldsymbol{U}^{\mathrm{H}} \boldsymbol{A} \boldsymbol{U} \boldsymbol{P}_{1}=\boldsymbol{P}_{1}^{\mathrm{H}} \operatorname{diag}\left(\lambda_{1}, \lambda_{2}, \cdots, \lambda_{n}\right) \boldsymbol{P}_{1}=\boldsymbol{E}
$$

若命 $\boldsymbol{P}=\boldsymbol{U} \boldsymbol{P}_{1}$ ,代人上式得

$$
\boldsymbol{P}^{\mathrm{H}} \boldsymbol{A} \boldsymbol{P}=\boldsymbol{E}
$$

(4)$\Rightarrow$(5)由于 $\boldsymbol{P}^{\mathrm{H}} \boldsymbol{A} \boldsymbol{P}=\boldsymbol{E}$ ,故 $\boldsymbol{A}=\left(\boldsymbol{P}^{\mathrm{H}}\right)^{-1} \boldsymbol{P}^{-1}=\left(\boldsymbol{P}^{-1}\right)^{\mathrm{H}} \boldsymbol{P}^{-1}$ 若命 $\boldsymbol{Q}=\boldsymbol{P}^{-1}$ ,代人上式得

$$
\boldsymbol{A}=\boldsymbol{Q}^{\mathrm{H}} \boldsymbol{Q}
$$

(5)$\Rightarrow$(6)因为 $\boldsymbol{A}=\boldsymbol{Q}^{\mathrm{H}} \boldsymbol{Q}$ ,其中 $\boldsymbol{Q}$ 为可逆矩阵,根据矩阵 $\boldsymbol{U} \boldsymbol{R}$ 分解定理4.2.1得到 $\boldsymbol{Q}=\boldsymbol{U}_{1} \boldsymbol{R}$ ,其中 $\boldsymbol{U}_{1}$ 是酉矩阵, $\boldsymbol{R}$ 是正线上三角阵。因此

$$
A=Q^{\mathrm{H}} Q=R^{\mathrm{H}} U_{1}^{\mathrm{H}} U_{1} R=R^{\mathrm{H}} R
$$

现证分解的唯一性:设 $\boldsymbol{A}$ 有两种正线上三角分解,即

$$
\boldsymbol{A}=\boldsymbol{R}^{\mathrm{H}} \boldsymbol{R}=\boldsymbol{R}_{1}^{\mathrm{H}} \boldsymbol{R}_{1}
$$

故

$$
\begin{aligned}
\boldsymbol{E} & =\left(\boldsymbol{R}^{\mathbf{H}}\right)^{-1} \boldsymbol{R}_{1}^{\mathbf{H}} \boldsymbol{R}_{1} \boldsymbol{R}^{-1} \\
& =\left(\boldsymbol{R}_{1} \boldsymbol{R}^{-1}\right)^{\mathbf{H}}\left(\boldsymbol{R}_{1} \boldsymbol{R}^{-1}\right)
\end{aligned}
$$

容易验证 $\boldsymbol{R}_{1} \boldsymbol{R}^{-1}$ 仍是正线上三角阵,又由上式知 $\boldsymbol{R}_{1} \boldsymbol{R}^{-1}$ 是酉矩阵。根据引理 3.9.1 可得 $\boldsymbol{R}_{1} \boldsymbol{R}^{-1}=\boldsymbol{E}$ ,即 $\boldsymbol{R}_{1}=\boldsymbol{R}$ 。\\
(6)$\Rightarrow$(1)因为 $\boldsymbol{A}=\boldsymbol{R}^{\mathrm{H}} \boldsymbol{R}$ ,所以

$$
f(\boldsymbol{X})=\boldsymbol{X}^{\mathrm{H}} \boldsymbol{A} \boldsymbol{X}=\boldsymbol{X}^{\mathrm{H}} \boldsymbol{R}^{\mathrm{H}} \boldsymbol{R} \boldsymbol{X}=(\boldsymbol{R} \boldsymbol{X})^{\mathrm{H}}(\boldsymbol{R} \boldsymbol{X})
$$

由于 $\boldsymbol{R}$ 为正线上三角阵,故当 $\boldsymbol{X} \neq 0$ 时, $\boldsymbol{R X} \neq 0$ ,于是

$$
f(X)=X^{\mathrm{H}} A X=(R X)^{\mathrm{H}}(R X)>0
$$

此即 $f(\boldsymbol{X})$ 是正定的.\\
线性代数中介绍判别正定实对称矩阵时有行列式法则。类似地,正定 Hermite (实对称)矩阵也有。

定理 3.9.2 $n$ 阶 Hermite(实对称)矩阵 $\boldsymbol{A}=\left(a_{i j}\right)$ 正定的充要条件是 $\boldsymbol{A}$ 的 $n$个顺序主子式全大于零。即

$$
a_{11}>0, \quad\left|\begin{array}{ll}
a_{11} & a_{12} \\
a_{21} & a_{22}
\end{array}\right|>0
$$

$$
\left|\begin{array}{lll}
a_{11} & a_{12} & a_{13} \\
a_{21} & a_{22} & a_{23} \\
a_{31} & a_{32} & a_{33}
\end{array}\right|>0, \cdots,\left|\begin{array}{cccc}
a_{11} & a_{12} & \cdots & a_{1 n} \\
a_{21} & a_{22} & \cdots & a_{2 n} \\
\vdots & \vdots & & \vdots \\
a_{n 1} & a_{n 2} & \cdots & a_{n n}
\end{array}\right|>0
$$

(证略)\\
推论3.9.1 $n$ 阶 Hermite(实对称)矩阵 $\boldsymbol{A}=\left(a_{i j}\right)$ 负定的充要条件是 $\boldsymbol{A}$ 的 $n$个顺序主子式负、正相间,即

$$
\begin{gathered}
a_{11}<0, \quad\left|\begin{array}{ll}
a_{11} & a_{12} \\
a_{21} & a_{22}
\end{array}\right|>0, \\
\left|\begin{array}{lll}
a_{11} & a_{12} & a_{13} \\
a_{21} & a_{22} & a_{23} \\
a_{31} & a_{32} & a_{33}
\end{array}\right|<0, \cdots,(-1)^{n}\left|\begin{array}{cccc}
a_{11} & a_{12} & \cdots & a_{1 n} \\
a_{21} & a_{22} & \cdots & a_{2 n} \\
\vdots & \vdots & & \vdots \\
a_{n 1} & a_{n 2} & \cdots & a_{n n}
\end{array}\right|>0
\end{gathered}
$$

根据定理3.9.2与 $-f(\boldsymbol{X})$ 是正定的即可证明推论。\\
定理3.9.3 设 $A$ 是 Hermite 矩阵,下列命题是等价的:\\
(1) $\boldsymbol{A}$ 是半正定的\\
(2)对于任何 $n$ 阶可逆矩阵 $\boldsymbol{P}$ ,都有 $\boldsymbol{P}^{\mathrm{H}} \boldsymbol{A P}$ 是半正定的\\
(3) $\boldsymbol{A}$ 的 $n$ 个特征值全是非负的\\
(4)存在 $n$ 阶可逆矩阵 $\boldsymbol{P}$ ,使得

$$
\boldsymbol{P}^{\mathbf{H}} \boldsymbol{A} \boldsymbol{P}=\left[\begin{array}{cc}
E_{r} & 0 \\
0 & 0
\end{array}\right]
$$

(5)存在秩为 $r$ 的 $n$ 阶矩阵 $\boldsymbol{Q}$ ,使得

$$
\boldsymbol{A}=\boldsymbol{Q}^{\mathrm{H}} \boldsymbol{Q}
$$

证明(1)$\Rightarrow$(2)$\Rightarrow$(3)请读者自己完成。\\
(3)$\Rightarrow$(4)存在 $\boldsymbol{U} \in U^{n \times n}$ ,满足

$$
\boldsymbol{U}^{\mathrm{H}} \boldsymbol{A} \boldsymbol{U}=\operatorname{diag}\left(\lambda_{1}, \lambda_{2}, \cdots, \lambda_{r}, 0, \cdots, 0\right)
$$

其中 $r=\operatorname{rank} A, \lambda_{1}, \lambda_{2}, \cdots, \lambda_{r}>0$ .命

$$
P_{1}=\operatorname{diag}\left(\frac{1}{\sqrt{\lambda_{1}}}, \frac{1}{\sqrt{\lambda_{2}}}, \cdots, \frac{1}{\sqrt{\lambda_{r}}}, 1, \cdots, 1\right)
$$

则

$$
\boldsymbol{P}_{1}^{\mathrm{H}} \boldsymbol{U}^{\mathrm{H}} \boldsymbol{A} \boldsymbol{U} \boldsymbol{P}_{1}=\operatorname{diag}(1,1, \cdots, 1,0, \cdots, 0)=\left[\begin{array}{cc}
\boldsymbol{E}_{r} & 0 \\
0 & 0
\end{array}\right]
$$

令 $\boldsymbol{P}=\boldsymbol{U} \boldsymbol{P}_{1}$ ,则

$$
\boldsymbol{P}^{\mathrm{H}} \boldsymbol{A} \boldsymbol{P}=\left[\begin{array}{cc}
E_{r} & 0 \\
0 & 0
\end{array}\right]
$$

(4)$\Rightarrow$(5)由(4)可得

$$
\begin{aligned}
A & =\left(P^{\mathrm{H}}\right)^{-1}\left[\begin{array}{cc}
E_{r} & 0 \\
0 & 0
\end{array}\right] P^{-1} \\
& =\left(P^{-1}\right)^{\mathrm{H}}\left[\begin{array}{cc}
E_{r} & 0 \\
0 & 0
\end{array}\right]\left[\begin{array}{cc}
E_{r} & 0 \\
0 & 0
\end{array}\right] P^{-1} \\
& =\left(\left[\begin{array}{cc}
E_{r} & 0 \\
0 & 0
\end{array}\right] P^{-1}\right)^{\mathrm{H}}\left(\left[\begin{array}{cc}
E_{r} & 0 \\
0 & 0
\end{array}\right] P^{-1}\right) \\
& =Q^{\mathrm{H}} \boldsymbol{Q},
\end{aligned}
$$

其中

$$
\boldsymbol{Q}=\left[\begin{array}{cc}
\boldsymbol{E}_{r} & 0 \\
0 & 0
\end{array}\right] \boldsymbol{P}^{-1} \in C_{r}^{n \times n}
$$

(5)$\Rightarrow$(1)由于 $\boldsymbol{A}=\boldsymbol{Q}^{\mathrm{H}} \boldsymbol{Q}$ ,故

$$
\boldsymbol{X}^{\boldsymbol{H}} \boldsymbol{A} \boldsymbol{X}=\boldsymbol{X}^{\boldsymbol{H}} \boldsymbol{Q}^{\mathrm{H}} \boldsymbol{Q} \boldsymbol{X}=(\boldsymbol{Q} \boldsymbol{X})^{\mathrm{H}}(\boldsymbol{Q} \boldsymbol{X})
$$

因为 $Q \in C_{r}^{n \times n}$ ,所以方程组 $Q X=0$ 有非零解。即存在 $X \neq 0$ ,满足 $Q X=0$ 。从而

$$
X^{\mathrm{H}} A X=(Q X)^{\mathrm{H}}(Q X) \geqslant 0
$$

此即 $\boldsymbol{A}$ 是半正定的。\\
定理3.9.4 设 $\boldsymbol{A}$ 是正定(半正定)Hermite 矩阵,则存在唯一的正定(半正定)Hermite 矩阵 $\boldsymbol{H}$ ,满足 $\boldsymbol{A}=\boldsymbol{H}^{2}$ ,且任何一个与 $\boldsymbol{A}$ 可交换的矩阵 $\boldsymbol{B}$ 必和 $\boldsymbol{H}$ 可交换 (即若 $\boldsymbol{A B}=\boldsymbol{B A}$ ,则 $\boldsymbol{H B}=\boldsymbol{B H}$ )。

证明 因为 $\boldsymbol{A}$ 是正定(半正定)Hermite 矩阵,故

$$
A=U \operatorname{diag}\left(\lambda_{1}, \lambda_{2}, \cdots, \lambda_{n}\right) U^{\mathrm{H}}
$$

其中 $\boldsymbol{U}$ 是酉矩阵,$\lambda_{1}, \lambda_{2}, \cdots, \lambda_{n}$ 全大于零(非负)。令

$$
\boldsymbol{H}=\boldsymbol{U} \operatorname{diag}\left(\sqrt{\lambda_{1}}, \sqrt{\lambda_{2}}, \cdots, \sqrt{\lambda_{n}}\right) \boldsymbol{U}^{\mathrm{H}}
$$

显然 $\boldsymbol{H}^{2}=\boldsymbol{A}$ .现证 $\boldsymbol{H}$ 是唯一的.\\
设还有一个正定 Hermite 矩阵 $\boldsymbol{H}_{1}$ ,满足 $\boldsymbol{A}=\boldsymbol{H}_{1}^{2}$ ,故可设

$$
H_{1}=U_{1} \operatorname{diag}\left(\mu_{1}, \mu_{2}, \cdots, \mu_{n}\right) U_{1}^{\mathrm{H}} \quad\left(\mu_{i}>0\right)
$$

根据 $\boldsymbol{A}=\boldsymbol{H}_{1}^{2}$ ,得到 $\mu_{1}^{2}=\lambda_{1}, \mu_{2}^{2}=\lambda_{2}, \cdots, \mu_{n}^{2}=\lambda_{n}$ .于是

$$
H_{1}=U_{1} \operatorname{diag}\left(\sqrt{\lambda_{1}}, \sqrt{\lambda_{2}}, \cdots, \sqrt{\lambda_{n}}\right) U_{1}^{\mathrm{H}}
$$

根据 $\boldsymbol{A}=\boldsymbol{H}^{\mathbf{2}}=\boldsymbol{H}_{1}^{\mathbf{2}}$ ,所以

$$
U \operatorname{diag}\left(\lambda_{1}, \lambda_{2}, \cdots, \lambda_{n}\right) U^{\mathrm{H}}=U_{1} \operatorname{diag}\left(\lambda_{1}, \lambda_{2}, \cdots, \lambda_{n}\right) U_{1}^{\mathrm{H}}
$$

下面进一步证明

$$
\begin{gathered}
U \operatorname{diag}\left(\sqrt{\lambda_{1}}, \sqrt{\lambda_{2}}, \cdots, \sqrt{\lambda_{n}}\right) U^{\mathrm{H}} \\
=U_{1} \operatorname{diag}\left(\sqrt{\lambda_{1}}, \sqrt{\lambda_{2}}, \cdots, \sqrt{\lambda_{n}}\right) U_{1}^{\mathrm{H}}
\end{gathered}
$$

事实上

$$
\operatorname{diag}\left(\lambda_{1}, \lambda_{2}, \cdots, \lambda_{n}\right) U^{\mathrm{H}} U_{1}=U^{\mathrm{H}} U_{1} \operatorname{diag}\left(\lambda_{1}, \lambda_{2}, \cdots, \lambda_{n}\right)
$$

不妨设酉矩阵

$$
\boldsymbol{U}^{\mathrm{H}} \boldsymbol{U}_{1}=\left[\begin{array}{cccc}
p_{11} & p_{12} & \cdots & p_{1 n} \\
p_{21} & p_{22} & \cdots & p_{2 n} \\
\vdots & \vdots & & \vdots \\
p_{n 1} & p_{n 2} & \cdots & p_{n n}
\end{array}\right]
$$

代人上式得

$$
\left[\begin{array}{cccc}
\lambda_{1} p_{11} & \lambda_{1} p_{12} & \cdots & \lambda_{1} p_{1 n} \\
\lambda_{2} p_{21} & \lambda_{2} p_{22} & \cdots & \lambda_{2} p_{2 n} \\
\vdots & \vdots & & \vdots \\
\lambda_{n} p_{n 1} & \lambda_{n} p_{n 2} & \cdots & \lambda_{n} p_{n n}
\end{array}\right]=\left[\begin{array}{cccc}
\lambda_{1} p_{11} & \lambda_{2} p_{12} & \cdots & \lambda_{n} p_{1 n} \\
\lambda_{1} p_{21} & \lambda_{2} p_{22} & \cdots & \lambda_{n} p_{2 n} \\
\vdots & \vdots & & \vdots \\
\lambda_{1} p_{n 1} & \lambda_{2} p_{n 2} & \cdots & \lambda_{n} p_{n n}
\end{array}\right]
$$

比较等号两端得

$$
\lambda_{i} p_{i j}=\lambda_{j} p_{i j} \quad(i, j=1,2, \cdots, n)
$$

当 $\lambda_{i} \neq \lambda_{j}$ 时,$p_{i j}=0$ ,故 $\sqrt{\lambda_{i}} p_{i j}=\sqrt{\lambda_{j}} p_{i j}$ ;当 $\lambda_{i}=\lambda_{j}$ 时,$\sqrt{\lambda_{i}} p_{i j}=\sqrt{\lambda_{j}} p_{i j}$ .于是有

即

$$
\begin{gathered}
\operatorname{diag}\left(\sqrt{\lambda_{1}}, \sqrt{\lambda_{2}}, \cdots, \sqrt{\lambda_{n}}\right) \boldsymbol{U}^{\mathrm{H}} \boldsymbol{U}_{1} \\
=\boldsymbol{U}^{\mathrm{H}} \boldsymbol{U}_{1} \operatorname{diag}\left(\sqrt{\lambda_{1}}, \sqrt{\lambda_{2}}, \cdots, \sqrt{\lambda_{n}}\right) \\
\boldsymbol{H}=\boldsymbol{H}_{1}
\end{gathered}
$$

定理最后部分结论"且任何一个与 $\boldsymbol{A}$ 可交换的矩阵 $\boldsymbol{B}$ 必和 $\boldsymbol{H}$ 可交换",可仿照定理证明中使用的方法证明。

对于实二次齐式有类似定义与结论。下面只作叙述而不再详细论证。\\
定义3.9.2 给定实二次齐式

$$
f(X)=f\left(x_{1}, x_{2}, \cdots, x_{n}\right)=\sum_{i=1}^{n} \sum_{j=1}^{n} a_{i j} x_{i} x_{j}=\boldsymbol{X}^{\mathrm{T}} \boldsymbol{A} \boldsymbol{X}
$$

$\boldsymbol{X}=\left(x_{1}, x_{2}, \cdots, x_{n}\right)^{\mathrm{T}}$ ,如果对任一组不全为零的实数 $x_{1}, x_{2}, \cdots, x_{n}$ 都有 $f\left(x_{1}, x_{2}\right.$ , $\left.\cdots, x_{n}\right)>0(\geqslant 0)$ ,则称该二次齐式是正定的(半正定的).并称相对应的实对称矩阵 $\boldsymbol{A}$ 为正定的(半正定的)。

定理 3.9.5 对于实二次齐式 $f(\boldsymbol{X})=\boldsymbol{X}^{\mathrm{T}} \boldsymbol{A} \boldsymbol{X}, \boldsymbol{X} \in R^{n}$ ,下列命题是等价的:\\
(1)$f(X)$ 是正定的;\\
(2)对于任何 $n$ 阶可逆矩阵 $\boldsymbol{P}$ ,都有 $\boldsymbol{P}^{\mathrm{T}} \boldsymbol{A P}$ 为正定矩阵;\\
(3) $\boldsymbol{A}$ 的 $n$ 个特征值全大于零;\\
(4)存在 $n$ 阶可逆矩阵 $P$ ,使得 $P^{\mathrm{T}} A P=E$ ;\\
(5)存在 $n$ 阶可逆矩阵 $\boldsymbol{Q}$ ,使得 $\boldsymbol{A}=\boldsymbol{Q}^{\mathrm{T}} \boldsymbol{Q}$ ;\\
(6)存在正线上三角矩阵 $\boldsymbol{R}$ ,使得 $\boldsymbol{A}=\boldsymbol{R}^{\mathrm{T}} \boldsymbol{R}$ ,且分解是唯一的。\\
证略。\\
定理3.9.6 设 $\boldsymbol{A}$ 是实对称矩阵,下列命题是等价的:\\
(1) $\boldsymbol{A}$ 是半正定的;\\
(2)对于任何 $n$ 阶可逆矩阵 $\boldsymbol{P}$ ,都有 $\boldsymbol{P}^{\mathrm{T}} \boldsymbol{A P}$ 是半正定的;\\
(3) $\boldsymbol{A}$ 的 $n$ 个特征值全是非负的;\\
(4)存在 $n$ 阶可逆矩阵 $\boldsymbol{P}$ ,使得 $\boldsymbol{P}^{\mathrm{T}} \boldsymbol{A P}=\left[\begin{array}{cc}\boldsymbol{E}_{r} & 0 \\ 0 & 0\end{array}\right]$ ;\\
(5)存在 $n$ 阶秩为 $r$ 的矩阵 $\boldsymbol{Q}$ ,使得 $\boldsymbol{A}=\boldsymbol{Q}^{\mathrm{T}} \boldsymbol{Q}$ .\\
证略。\\
定理 3.9.7 设 $\boldsymbol{A}$ 是正定(半正定)实对称矩阵,则存在唯一的正定(半正定)实对称矩阵 $\boldsymbol{H}$ ,满足 $\boldsymbol{A}=\boldsymbol{H}^{2}$ ,且任何一个与 $\boldsymbol{A}$ 可交换的矩阵必和 $\boldsymbol{H}$ 可交换。

例3.9.1 已知 $\boldsymbol{A} 、 \boldsymbol{B}$ 是 $n$ 阶正定 Hermite 矩阵,则 $|\lambda \boldsymbol{B}-\boldsymbol{A}|=0$ 的根全是正的实数。

证明 因为 $\boldsymbol{B}$ 是正定的,存在 $\boldsymbol{P} \in C_{n}^{n \times n}$ ,满足

$$
\boldsymbol{P}^{\mathrm{H}} \boldsymbol{B} \boldsymbol{P}=\boldsymbol{E}
$$

且 $\boldsymbol{P}^{\mathrm{H}} \boldsymbol{A P}$ 是正定 Hermite 矩阵。因此 $\left|\lambda \boldsymbol{E}-\boldsymbol{P}^{\mathrm{H}} \boldsymbol{A P}\right|=0$ 的根是正的实数。而

$$
\left|\lambda E-P^{\mathrm{H}} A P\right|=\left|\lambda P^{\mathrm{H}} B P-P^{\mathrm{H}} A P\right|=\left|P^{\mathrm{H}}\right||\lambda B-A||P|
$$

故 $|\lambda \boldsymbol{B}-\boldsymbol{A}|=0$ 的根是正的实数。\\
例3.9.2 已知 $\boldsymbol{A} 、 \boldsymbol{B}$ 为 $n$ 阶正交矩阵,并且 $|\boldsymbol{A}|=-|\boldsymbol{B}|$ ,试证: $\boldsymbol{A}+\boldsymbol{B}$ 不可逆.

证明

$$
\begin{aligned}
|A+B| & =\left|B B^{-1} A+B\right|=\left|B\left(B^{-1} A+E\right)\right| \\
& =|B|\left|B^{-1} A+E\right|
\end{aligned}
$$

由于 $B^{-1} A$ 是正交矩阵,它的特征值 $\lambda$ 的模为 1 (即 $\pm 1$ 或 $\cos \theta \pm i \sin \theta$ )且由 $|\boldsymbol{A}|=-|\boldsymbol{B}|$ 可得 $\left|\boldsymbol{B}^{-1} \boldsymbol{A}\right|=\left|\boldsymbol{B}^{-1}\right||\boldsymbol{A}|=-\left|\boldsymbol{B}^{-1}\right||\boldsymbol{B}|=-1$ ,故 $\boldsymbol{B}^{-1} \boldsymbol{A}$ 至少有一个特征值是 -1 .因此 $\left|\boldsymbol{B}^{-1} \boldsymbol{A}+\boldsymbol{E}\right|=0$ ,即 $|\boldsymbol{A}+\boldsymbol{B}|=0, \boldsymbol{A}+\boldsymbol{B}$ 不可逆.

\section*{§ 3. 10 Hermite 矩阵偶在复相合下的标准形}
两个 Hermite 矩阵同时与对角矩阵复相合的问题是现在要研究的内容。应用于 Hermite 二次齐式就是两个 Hermite 齐式同时化简成标准形。它在振动理论、物理及其他工程中有重要的应用。

定理3.10.1 设 $A, B$ 为 $n$ 阶 Hermite 矩阵,且 $B$ 是正定的,则存在 $T \in C_{n}^{n \times n}$ ,使得


\begin{equation*}
\boldsymbol{T}^{\mathrm{H}} \boldsymbol{A} \boldsymbol{T}=\operatorname{diag}\left(\mu_{1}, \mu_{2}, \cdots, \mu_{n}\right)=\boldsymbol{M} \tag{3.10.1}
\end{equation*}


与


\begin{equation*}
T^{\mathrm{H}} B T=E \tag{3.10.2}
\end{equation*}


同时成立,其中 $\mu_{1}, \mu_{2}, \cdots, \mu_{n}$ 是与 $\boldsymbol{T}$ 无关的实数,是 $|\lambda \boldsymbol{B}-\boldsymbol{A}|=0$ 的根。\\
证明 $\boldsymbol{B}$ 是正定 Hermite 矩阵,故存在 $\boldsymbol{T}_{1} \in C_{n}^{n \times n}$ ,使得

$$
T_{1}^{\mathrm{H}} B T_{1}=E
$$

由于 $\boldsymbol{T}_{1}^{\mathrm{H}} \boldsymbol{A} \boldsymbol{T}_{1}$ 是 Hermite 矩阵,存在 $\boldsymbol{T}_{2} \in \boldsymbol{U}_{n}^{n \times n}$ ,使得

$$
\boldsymbol{T}_{2}^{\mathrm{H}} \boldsymbol{T}_{1}^{\mathrm{H}} \boldsymbol{A} \boldsymbol{T}_{1} \boldsymbol{T}_{2}=\operatorname{diag}\left(\mu_{1}, \mu_{2}, \cdots, \mu_{n}\right)=\boldsymbol{M}
$$

其中 $\mu_{i}$ 是 Hermite 矩阵 $\boldsymbol{T}_{1}^{\mathrm{H}} \boldsymbol{A} \boldsymbol{T}_{1}$ 的 $n$ 个实特征值。\\
若命 $\boldsymbol{T}=\boldsymbol{T}_{1} \boldsymbol{T}_{2}$ ,则有

$$
T^{\mathrm{H}} A T=M, T^{\mathrm{H}} B T=E
$$

余下要证明 $\mu_{1}, \mu_{2}, \cdots, \mu_{n}$ 与 $\boldsymbol{T}$ 无关。事实上,命 $\boldsymbol{S}=\boldsymbol{T}_{1}^{\mathrm{H}} \boldsymbol{A} \boldsymbol{T}_{1}$ ,则 $\mu_{i}$ 是特征方程 $|\lambda E-S|=0$ 的根。而 $|\lambda E-S|=\left|\lambda T_{1}^{\mathrm{H}} B T_{1}-T_{1}^{\mathrm{H}} A T_{1}\right|=\left|T_{1}^{\mathrm{H}}\right||\lambda B-A|\left|T_{1}\right|$ ,因此 $\boldsymbol{\mu}_{\boldsymbol{i}}$ 是方程

$$
|\lambda B-A|=0
$$

的根.它由矩阵 $\boldsymbol{A}, \boldsymbol{B}$ 所确定而与 $T$ 无关.\\
例 3.10.1 设

$$
A=\left[\begin{array}{cc}
1 & 1+\mathrm{i} \\
1-\mathrm{i} & 2
\end{array}\right], B=\left[\begin{array}{rr}
2 & \mathrm{i} \\
-\mathrm{i} & 2
\end{array}\right]
$$

验证 $\boldsymbol{A}$ 是 Hermite 矩阵 $\boldsymbol{B}$ 是正定的 Hermite 矩阵,并求满秩矩阵 $\boldsymbol{T}$ ,使得 $\boldsymbol{T}^{\mathrm{H}} \boldsymbol{A T}$ 为对角矩阵, $\boldsymbol{T}^{\mathrm{H}} \boldsymbol{B} \boldsymbol{T}=\boldsymbol{E}$ .

解 易证 $\boldsymbol{B}$ 是正定 Hermite 矩阵。\\
$\boldsymbol{B}$ 的特征多项式

$$
|\lambda E-B|=(\lambda-3)(\lambda-1)
$$

故 $B$ 的特征值 $\lambda_{1}=3, \lambda_{2}=1$\\
当 $\lambda=3$ 时,特征矩阵

$$
\lambda E-B=\left[\begin{array}{rr}
1 & -\mathrm{i} \\
\mathrm{i} & 1
\end{array}\right] \rightarrow\left[\begin{array}{rr}
1 & -\mathrm{i} \\
0 & 0
\end{array}\right]
$$

所以属于 $\lambda=3$ 的单位特征向量 $\alpha_{1}=\left(\frac{\mathrm{i}}{\sqrt{2}}, \frac{1}{\sqrt{2}}\right)^{\mathrm{T}}$\\
当 $\lambda=1$ 时,特征矩阵

$$
\lambda E-B=\left[\begin{array}{rr}
-1 & -\mathrm{i} \\
\mathrm{i} & -1
\end{array}\right] \rightarrow\left[\begin{array}{ll}
1 & \mathrm{i} \\
0 & 0
\end{array}\right]
$$

所以属于 $\lambda=1$ 的单位特征向量 $\alpha_{2}=\left(-\frac{\mathrm{i}}{\sqrt{2}}, \frac{1}{\sqrt{2}}\right)^{\mathrm{T}}$\\
命

$$
U_{1}=\left(a_{1}, a_{2}\right)=\left[\begin{array}{cc}
\frac{\mathrm{i}}{\sqrt{2}} & -\frac{\mathrm{i}}{\sqrt{2}} \\
\frac{1}{\sqrt{2}} & \frac{1}{\sqrt{2}}
\end{array}\right]
$$

则

$$
U_{1}^{\mathrm{H}} B U_{1}=\left[\begin{array}{ll}
3 & \\
& 1
\end{array}\right]
$$

故可命

$$
\boldsymbol{T}_{1}=\left[\begin{array}{cc}
\frac{\mathrm{i}}{\sqrt{2}} & -\frac{\mathrm{i}}{\sqrt{2}} \\
\frac{1}{\sqrt{2}} & \frac{1}{\sqrt{2}}
\end{array}\right]\left[\begin{array}{ll}
\frac{1}{\sqrt{3}} & \\
& 1
\end{array}\right]=\left[\begin{array}{cc}
\frac{\mathrm{i}}{\sqrt{6}} & -\frac{\mathrm{i}}{\sqrt{2}} \\
\frac{1}{\sqrt{6}} & \frac{1}{\sqrt{2}}
\end{array}\right]
$$

则

$$
\begin{gathered}
T_{1}^{\mathrm{H}} B T_{1}=E \\
S=T_{1}^{\mathrm{H}} A T_{1}=\left[\begin{array}{cc}
\frac{5}{6} & \frac{1-2 \mathrm{i}}{\sqrt{12}} \\
\frac{1+2 \mathrm{i}}{\sqrt{12}} & \frac{1}{2}
\end{array}\right]
\end{gathered}
$$

$S$ 的特征多项式 $|\lambda E-S|=\lambda\left(\lambda-\frac{4}{3}\right), S$ 的特征值为 $\lambda_{1}=0, \lambda_{2}=\frac{4}{3}$ 容易求得属于 $\lambda_{1}=0$ 单位特征向量 $\xi_{1}=\left(\sqrt{\frac{3}{8}}, \frac{-1-2 i}{\sqrt{8}}\right)^{\mathrm{T}}$ ,属于 $\lambda_{2}=\frac{4}{3}$ 的单位特征向量 $\xi_{2}=\left(\frac{1-2 \mathrm{i}}{\sqrt{8}}, \sqrt{\frac{3}{8}}\right)^{\mathrm{T}}$

命

$$
T_{2}=\left(\xi_{1}, \xi_{2}\right)=\left[\begin{array}{cc}
\frac{\sqrt{6}}{4} & \frac{(1-2 i) \sqrt{2}}{4} \\
\frac{(-1-2 i) \sqrt{2}}{4} & \frac{\sqrt{6}}{4}
\end{array}\right]
$$

则 $\boldsymbol{T}_{2}^{\mathrm{H}} \boldsymbol{S} \boldsymbol{T}_{2}=\left[\begin{array}{ll}0 & \\ & \frac{4}{3}\end{array}\right], \boldsymbol{T}_{2}^{\mathrm{H}}\left(\boldsymbol{T}_{1}^{\mathrm{H}} \boldsymbol{B} \boldsymbol{T}_{1}\right) \boldsymbol{T}_{2}=\boldsymbol{E}$\\
命

$$
T=T_{1} T_{2}=\left[\begin{array}{cc}
\frac{i-1}{2} & \frac{1-i}{2 \sqrt{3}} \\
-\frac{i}{2} & \frac{2-i}{2 \sqrt{3}}
\end{array}\right]
$$

不难验证得

$$
T^{\mathrm{H}} A T=\left[\begin{array}{ll}
0 & \\
& \frac{4}{3}
\end{array}\right]
$$

$$
T^{\mathrm{H}} B T=E
$$

将定理 3.10.1 的结论应用于 Hermite 二次齐式则有\\
定理3.10.2 已给两个 Hermite 二次齐式

$$
\begin{aligned}
& f_{1}=\boldsymbol{X}^{\mathbf{H}} \boldsymbol{A} \boldsymbol{X}=\sum_{i, j=1}^{n} a_{i j} \bar{x}_{i} x_{j}, \\
& f_{2}=\boldsymbol{X}^{\mathbf{H}} \boldsymbol{B} \boldsymbol{X}=\sum_{i, j=1}^{n} b_{i j} \bar{x}_{i} x_{j},
\end{aligned}
$$

且 $f_{2}$ 是正定的.则存在满秩线性变换

$$
X=T Y
$$

使得 $f_{1}, f_{2}$ 化成标准形

$$
\begin{aligned}
& f_{1}=\mu_{1} y_{1} \bar{y}_{1}+\mu_{2} y_{2} \bar{y}_{2}+\cdots+\mu_{n} y_{n} \bar{y}_{n} \\
& f_{2}=y_{1} \bar{y}_{1}+y_{2} \bar{y}_{2}+\cdots+y_{n} \bar{y}_{n}
\end{aligned}
$$

其中 $\mu_{1}, \mu_{2}, \cdots, \mu_{n}$ 是方程 $|\lambda \boldsymbol{B}-\boldsymbol{A}|=0$ 的根(全是实数).\\
定义 3.10.1 设 $\boldsymbol{A} 、 \boldsymbol{B}$ 都是 $n$ 阶 Hermite 矩阵,且 $\boldsymbol{B}$ 是正定的,求 $\lambda$ 使方程


\begin{equation*}
\boldsymbol{A} \boldsymbol{x}=\lambda \boldsymbol{B} \boldsymbol{x} \tag{3.10.3}
\end{equation*}


有非零解 $\boldsymbol{x}=\left(a_{1}, a_{2}, \cdots, a_{n}\right)^{\mathrm{T}}$ .式(3.10.3)有非零解的充要条件是关于 $\lambda$ 的 $n$ 次代数方程


\begin{equation*}
|\lambda B-A|=0 \tag{3.10.4}
\end{equation*}


成立。称方程(3.10.4)是 $\boldsymbol{A}$ 相对于 $\boldsymbol{B}$ 的特征方程。它的根 $\lambda_{1}, \lambda_{2}, \cdots, \lambda_{n}$ 称为 $\boldsymbol{A}$相对于 $B$ 的广义特征值,把 $\lambda_{i}$ 代人式(3.10.3)所得非零解 $x$ 称为与 $\lambda_{i}$ 相对应的广义特征向量。

定理 3.10.3 形如式(3.10.3)的广义特征值与广义特征向量有如下性质:\\
(1)有 $n$ 个实的广义特征值;\\
(2)有 $n$ 个线性无关的广义特征向量 $x_{1}, x_{2}, \cdots, x_{n}$ ,即


\begin{equation*}
\boldsymbol{A} \boldsymbol{x}_{k}=\lambda_{k} \boldsymbol{B} \boldsymbol{x}_{k} \quad(k=1,2, \cdots, n) \tag{3.10.5}
\end{equation*}


(3)这 $n$ 个广义特征向量可以这样选取,使其满足


\begin{gather*}
\boldsymbol{x}_{i}^{\mathrm{H}} \boldsymbol{B} \boldsymbol{x}_{j}=\delta_{i j}  \tag{3.10.6}\\
\boldsymbol{x}_{i}^{\mathrm{H}} \boldsymbol{A} \boldsymbol{x}_{j}=\lambda_{j} \delta_{i j} \tag{3.10.7}
\end{gather*}


其中 $\boldsymbol{\delta}_{i j}$ 为克氏符号。\\
证明(1)由定理3.10.1的证明过程可知.\\
(2)取 $\boldsymbol{T}_{1}$ 使 $\boldsymbol{T}_{1}^{\mathrm{H}} \boldsymbol{B} \boldsymbol{T}_{1}=\boldsymbol{E}$ ,设 Hermite 矩阵 $\boldsymbol{S}=\boldsymbol{T}_{1}^{\mathrm{H}} \boldsymbol{A} \boldsymbol{T}_{1}$ 的 $n$ 个特征值为 $\lambda_{1}$ , $\lambda_{2}, \cdots, \lambda_{n}$ ,它们所对应的线性无关的特征向量有 $n$ 个,分别记为 $y_{1}, y_{2}, \cdots, y_{n}$ ,故

$$
\boldsymbol{S} \boldsymbol{y}_{k}=\lambda_{k} \boldsymbol{y}_{k} \quad(k=1,2, \cdots, n)
$$

或


\begin{equation*}
\boldsymbol{T}_{1}^{\mathrm{H}} \boldsymbol{A} \boldsymbol{T}_{1} \boldsymbol{y}_{k}=\lambda_{k} \boldsymbol{y}_{k} \tag{3.10.8}
\end{equation*}


令


\begin{equation*}
x_{k}=T_{1} y_{k} \tag{3.10.9}
\end{equation*}


代人式(3.10.8)得

或

$$
\begin{gathered}
\boldsymbol{T}_{1}^{\mathrm{H}} \boldsymbol{A} \boldsymbol{x}_{k}=\lambda_{k} \boldsymbol{T}_{1}^{-1} \boldsymbol{x}_{k} \\
\boldsymbol{A} \boldsymbol{x}_{k}=\lambda_{k}\left(\boldsymbol{T}_{1}^{\mathrm{H}}\right)^{-1} \boldsymbol{T}_{1}^{-1} \boldsymbol{x}_{k}
\end{gathered}
$$

应用 $\boldsymbol{B}=\left(\boldsymbol{T}_{1}^{\mathrm{H}}\right)^{-1} \boldsymbol{T}_{1}^{-1}$ ,便得


\begin{equation*}
\boldsymbol{A} \boldsymbol{x}_{k}=\lambda_{k} \boldsymbol{B} \boldsymbol{x}_{k} \tag{3.10.10}
\end{equation*}


这表明 $\boldsymbol{x}_{k}(k=1,2, \cdots, n)$ 是 $n$ 个广义特征向量.现证明它们是线性无关的.\\
设

$$
k_{1} x_{1}+\cdots+k_{n} x_{n}=0
$$

将式(3.10.9)代人上式得

$$
k_{1} T_{1} y_{1}+\cdots+k_{n} T_{1} y_{n}=0
$$

以 $\boldsymbol{T}_{1}^{-1}$ 左乘上式每一项得

$$
k_{1} y_{1}+\cdots+k_{n} y_{n}=0
$$

由于 $y_{1}, y_{2}, \cdots, y_{n}$ 是线性无关的,所以

$$
k_{1}=k_{2}=\cdots=k_{n}=0
$$

此即 $x_{1}, x_{2}, \cdots, x_{n}$ 是线性无关的。\\
(3)根据 Hermite 矩阵与对角矩阵酉相似,便知可以这样选取 $y_{1}, y_{2}, \cdots, y_{n}$ ,使它们满足

$$
\boldsymbol{y}_{i}^{\mathrm{H}} \boldsymbol{y}_{j}=\delta_{i j} \quad(i, j=1,2, \cdots, n),
$$

其中 $\delta_{i j}$ 为克氏符号,当 $i=j$ 时,$\delta_{i j}=1$ ;当 $i \neq j$ 时,$\delta_{i j}=0$ .由式(3.10.9)得

$$
\left(\boldsymbol{T}_{1}^{-1} x_{i}\right)^{\mathrm{H}}\left(\boldsymbol{T}_{1}^{-1} x_{j}\right)=\delta_{i j}
$$

展开得

$$
\boldsymbol{x}_{i}^{\mathrm{H}}\left(\boldsymbol{T}_{1}^{-1}\right)^{\mathrm{H}} \boldsymbol{T}_{1}^{-1} x_{j}=\delta_{i j}
$$

由式 $\boldsymbol{E}=\boldsymbol{T}_{1}^{\mathrm{H}} \boldsymbol{B} \boldsymbol{T}_{1}$ 得

$$
\boldsymbol{x}_{i}^{\mathrm{H}} \boldsymbol{B} \boldsymbol{x}_{j}=\delta_{i j}
$$

又由式(3.10.10)知

$$
\boldsymbol{x}_{i}^{\mathrm{H}} \boldsymbol{A} \boldsymbol{x}_{j}=\lambda_{j} x_{i}^{\mathrm{H}} \boldsymbol{B} \boldsymbol{x}_{j}=\lambda_{j} \delta_{i j}
$$

在矩阵理论与应用中,称满足式(3.10.6)的广义特征向量 $x_{1}, x_{2}, \cdots, x_{n}$ 为特征主向量。以这 $n$ 个特征主向量为列向量构成的矩阵

$$
\boldsymbol{T}=\left(x_{1}, x_{2}, \cdots, x_{n}\right)
$$

是满秩的,称为 $\boldsymbol{A}$ 相对于 $\boldsymbol{B}$ 的主矩阵.\\
定理3.10.4 设 $\boldsymbol{A} 、 \boldsymbol{B}$ 为 Hermite 矩阵,且 $\boldsymbol{B}$ 为正定的,则存在行列式等于 1的矩阵 $\boldsymbol{P}$ ,使

$$
\boldsymbol{P}^{\mathrm{H}} \boldsymbol{A} \boldsymbol{P}=\operatorname{diag}\left(a_{1}, a_{2}, \cdots, a_{n}\right)
$$

与

$$
\boldsymbol{P}^{\mathrm{H}} \boldsymbol{B} \boldsymbol{P}=\operatorname{diag}\left(b_{1}, b_{2}, \cdots, b_{n}\right)
$$

同时成立。其中 $a_{1}, a_{2}, \cdots, a_{n}$ 均为实数,$b_{1}, b_{2}, \cdots, b_{n}$ 均为正实数。\\
证明 由定理 3.10.1 知,存在满秩矩阵 $T$ ,使得

$$
\boldsymbol{T}^{\mathrm{H}} \boldsymbol{A} \boldsymbol{T}=\operatorname{diag}\left(\mu_{1}, \mu_{2}, \cdots, \mu_{n}\right)
$$

与

$$
T^{\mathrm{H}} B T=E
$$

同时成立。命

$$
P=\left(\frac{1}{\operatorname{det} T}\right)^{1 / n} T
$$

则

$$
\operatorname{det} \boldsymbol{P}=1
$$

且

$$
\begin{aligned}
\boldsymbol{P}^{\mathrm{H}} \boldsymbol{A P} & =\left(\frac{1}{\operatorname{det} T}\right)^{1 / n} \boldsymbol{T}^{\mathrm{H}} \boldsymbol{A}\left(\frac{1}{\operatorname{det} T}\right)^{1 / n} \boldsymbol{T} \\
& =\operatorname{diag}\left(a_{1}, a_{2}, \cdots, a_{n}\right)
\end{aligned}
$$

其中

$$
a_{i}=\left(\frac{1}{\operatorname{det} \bar{T} \operatorname{det} T}\right)^{1 / n} \mu_{i} \quad(i=1,2, \cdots, n)
$$

其中

$$
P^{\mathrm{H}} B P=\left(\frac{1}{\operatorname{det} T \operatorname{det} T}\right)^{1 / n} E=\operatorname{diag}\left(b_{1}, b_{2}, \cdots, b_{n}\right)
$$

$$
b_{1}=b_{2}=\cdots=b_{n}=\left(\frac{1}{\operatorname{det} T \operatorname{det} T}\right)^{1 / n}>0
$$

\section*{§3.11 Rayleigh 商}
本节将对 Hermite 矩阵的特征值运用 Rayleigh 商进行讨论。所有结论对于实对称矩阵完全适用。

定义 3.11.1 设 $\boldsymbol{A}^{\mathbf{H}}=\boldsymbol{A}$ ,称实数

$$
R(X)=\frac{X^{\mathrm{H}} A X}{X^{\mathrm{H}} X} \quad\left(X \in C^{n}, X \neq 0\right)
$$

为 Hermite 矩阵 $\boldsymbol{A}$ 的 Rayleigh 商。\\
由于 Hermite 矩阵 $\boldsymbol{A}$ 的特征值全是实数,不妨设 $\boldsymbol{A}$ 的 $\boldsymbol{n}$ 个特征值如下排列

$$
\lambda_{1} \leqslant \lambda_{2} \leqslant \cdots \leqslant \lambda_{n}
$$

定理 3.11.1 Hermite 矩阵 $\boldsymbol{A}$ 的 Rayleigh 商具有如下性质:\\
(1)$R(k \boldsymbol{X})=R(\boldsymbol{X}) \quad(k \in \mathbf{R})$\\
(2)$\lambda_{1} \leqslant R(X) \leqslant \lambda_{n}$\\
(3) $\min _{X \neq 0} R(X)=\lambda_{1}, \quad \max _{X \neq 0} R(X)=\lambda_{n}$\\
证明(1)由定义3.11.1可得。\\
(2)矩阵 $A$ 可以西对角化,即

$$
\boldsymbol{U}^{\mathrm{H}} \boldsymbol{A} \boldsymbol{U}=\operatorname{diag}\left(\lambda_{1}, \lambda_{2}, \cdots, \lambda_{n}\right)=\boldsymbol{\Lambda}
$$

命 $\boldsymbol{X}=\boldsymbol{U} \boldsymbol{Y}$ ,则

因为

$$
\begin{aligned}
\boldsymbol{R}(\boldsymbol{X})= & \frac{\mathbf{Y}^{\mathrm{H}} \boldsymbol{U}^{\mathrm{H}} \boldsymbol{A} \boldsymbol{U} \boldsymbol{Y}}{\boldsymbol{Y}^{\mathrm{H}} \boldsymbol{Y}}=\frac{\boldsymbol{Y}^{\mathrm{H}} \boldsymbol{A} \boldsymbol{Y}}{\boldsymbol{Y}^{\mathrm{H}} \boldsymbol{Y}} \\
= & \frac{\lambda_{1} y_{1} \bar{y}_{1}+\lambda_{2} y_{2} \bar{y}_{2}+\cdots+\lambda_{n} y_{n} \bar{y}_{n}}{\boldsymbol{Y}^{\mathrm{H}} \boldsymbol{Y}} \\
\lambda_{1}\left(y_{1} \bar{y}_{1}+\cdots+y_{n} \bar{y}_{n}\right) & \leqslant \lambda_{1} \bar{y}_{1} \bar{y}_{1}+\cdots+\lambda_{n} \bar{y}_{n} \bar{y}_{n} \\
& \leqslant \lambda_{n}\left(y_{1} \bar{y}_{1}+\cdots+y_{n} \bar{y}_{n}\right)
\end{aligned}
$$

即

$$
\lambda_{1} \boldsymbol{Y}^{\mathrm{H}} \boldsymbol{Y} \leqslant \boldsymbol{Y}^{\mathrm{H}} \boldsymbol{\Lambda} \boldsymbol{Y} \leqslant \lambda_{n} \boldsymbol{Y}^{\mathrm{H}} \boldsymbol{Y}
$$

于是

$$
\lambda_{1} \leqslant R(X) \leqslant \lambda_{n}
$$

(3)对于(2)中的每一个 $U$ 适当选取 $X$ ,使得 $y_{2}=y_{3}=\cdots=y_{n}=0$ ,便得

$$
R(X)=\lambda_{1}
$$

类似地,适当选取 $\boldsymbol{X}$ ,使得 $y_{1}=y_{2}=\cdots=y_{n-1}=0$ ,便得

$$
R(X)=\lambda_{n}
$$

综合之,便得

$$
\min _{X \neq 0} R(X)=\lambda_{1}, \quad \max _{X \neq 0} R(X)=\lambda_{n}
$$

定理3.11.2 设 $X_{1}, X_{2}, \cdots, X_{k-1}$ 是 Hermite 矩阵 $A$ 的分别属于特征值 $\lambda_{1}$ , $\lambda_{2}, \cdots, \lambda_{k-1}$ 的特征向量,$R_{k}$ 是子空间 $\operatorname{span}\left(X_{1}, X_{2}, \cdots, X_{k-1}\right)$ 的正交补子空间,则

$$
\lambda_{k}=\min _{X \in R_{k}} R(X)
$$

证明 不妨设 $\boldsymbol{X}_{1}, \boldsymbol{X}_{2}, \cdots, \boldsymbol{X}_{k-1}, \boldsymbol{X}_{k}, \cdots, \boldsymbol{X}_{n}$ 为 $\boldsymbol{A}$ 的 $n$ 个标准正交的特征向量组.显然

$$
R_{k}=\operatorname{span}\left(X_{k}, X_{k+1}, \cdots, X_{n}\right)
$$

对于任意 $n$ 维向量 $\boldsymbol{X}$ ,均有

$$
X=C_{1} X_{1}+C_{2} X_{2}+\cdots+C_{n} X_{n}
$$

于是

$$
\begin{aligned}
R(X) & =\frac{X^{\mathrm{H}} A X}{X^{\mathrm{H}} X} \\
& =\frac{\left(C_{1} X_{1}+C_{2} X_{2}+\cdots+C_{n} X_{n}\right)^{\mathrm{H}} A\left(C_{1} X_{1}+C_{2} X_{2}+\cdots+C_{n} X_{n}\right)}{\left(C_{1} X_{1}+C_{2} X_{2}+\cdots+C_{n} X_{n}\right)^{\mathrm{H}}\left(C_{1} X_{1}+C_{2} X_{2}+\cdots+C_{n} X_{n}\right)} \\
& =\frac{\lambda_{1} \bar{C}_{1} C_{1}+\lambda_{2} \bar{C}_{2} C_{2}+\cdots+\lambda_{n} \bar{C}_{n} C_{n}}{C_{1} \bar{C}_{1}+C_{2} \bar{C}_{2}+\cdots+C_{n} \bar{C}_{n}} \\
& =\lambda_{1} a_{1}+\lambda_{2} a_{2}+\cdots+\lambda_{n} a_{n}
\end{aligned}
$$

其中

$$
a_{i}=\frac{\bar{C}_{i} C_{i}}{\bar{C}_{1} C_{1}+\bar{C}_{2} C_{2}+\cdots+\bar{C}_{n} C_{n}} \geqslant 0 \text {, 且 } \sum_{i=1}^{n} a_{i}=1
$$

当 $k=1$ 时,$R_{1}=C^{n}$ .此即定理3.11.1.\\
当 $k=2$ 时,$X \in R_{2}$ ,这时 $C_{1}=0$ ,故

于是

$$
\begin{gathered}
X=C_{2} X_{2}+C_{3} X_{3}+\cdots+C_{n} X_{n} \\
R(X)=\lambda_{2} a_{2}+\lambda_{3} a_{3}+\cdots+\lambda_{n} a_{n} . \\
\lambda_{2}=\min _{X \in R_{2}} R(X)
\end{gathered}
$$

其余类推。\\
类似地还可以证明:\\
定理3.11.3 设 $X \in \operatorname{span}\left(X_{r}, X_{r+1}, \cdots, X_{s}\right), 1 \leqslant r<s \leqslant n$ ,则

$$
\min _{X \neq 0} R(X)=\lambda_{r}, \quad \max _{X \neq 0} R(X)=\lambda_{s}
$$

定理3.11.4 设 $V_{k}$ 是 $n$ 维复向量空间中任意 $k$ 维子空间,则有极小一极大原理

$$
\lambda_{k}=\min _{V_{k}} \max _{X \in V_{k}} R(X)
$$

或极大一极小原理

$$
\lambda_{k}=\max _{V_{n-k+1}} \min _{X \in V_{n-k+1}} R(X)
$$

证明 $k-1$ 维子空间 $\operatorname{span}\left(\boldsymbol{X}_{1}, \boldsymbol{X}_{2}, \cdots, \boldsymbol{X}_{k-1}\right)$ 的正交补子空间 $R_{k}$ 是 $n-k+1$维,因此 $V_{k}$ 与 $R_{k}$ 必有公共的非零向量 $\boldsymbol{Y}_{k}$ ,故

$$
\min _{\boldsymbol{X} \in R_{k}} R(\boldsymbol{X})=\lambda_{k} \leqslant R\left(\boldsymbol{Y}_{k}\right)
$$

又因为 $\boldsymbol{Y}_{k} \in V_{k}$ ,故

因此

$$
\begin{aligned}
& R\left(\boldsymbol{Y}_{k}\right) \leqslant \max _{\boldsymbol{X} \in V_{k}} R(\boldsymbol{X}) \\
& \lambda_{k} \leqslant \min _{V_{k}} \max _{\boldsymbol{X} \in V_{k}} R(\boldsymbol{X})
\end{aligned}
$$

又由前面定理知

$$
\min _{V_{k}} \max _{X \in V_{k}} R(X) \leqslant \max _{X \in L\left(X_{1}, X_{2}, \cdots, X_{k}\right)} R(X)=\lambda_{k}
$$

综合两不等式可得

$$
\lambda_{k}=\min _{V_{k}} \max _{X \in V_{k}} R(X)
$$

令 $\boldsymbol{B}=-\boldsymbol{A}$ ,则 $\boldsymbol{B}$ 的特征值按递减顺序排列

$$
\mu_{1} \geqslant \mu_{2} \geqslant \cdots \geqslant \mu_{n}
$$

其中 $\mu_{k}=-\lambda_{n-k+1}$ ,由刚才所证有

$$
\begin{aligned}
\lambda_{n-k+1} & =-\mu_{k}=-\min _{V_{k}} \max _{\boldsymbol{X} \in V_{k}} \frac{\boldsymbol{X}^{\mathrm{H}} \boldsymbol{B} \boldsymbol{X}}{\boldsymbol{X}^{\mathrm{H}} \boldsymbol{X}} \\
& =-\min _{V_{k}}\left\{\max _{\boldsymbol{X} \in V_{k}} \frac{-\boldsymbol{X}^{\mathrm{H}} \boldsymbol{A} \boldsymbol{X}}{\boldsymbol{X}^{\mathrm{H}} \boldsymbol{X}}\right\} \\
& =-\min _{V_{k}}\left\{-\min _{\boldsymbol{X} \in V_{k}} \frac{\boldsymbol{X}^{\mathrm{H}} \boldsymbol{A} \boldsymbol{X}}{\boldsymbol{X}^{\mathrm{H}} \boldsymbol{X}}\right\}
\end{aligned}
$$

$$
=\max _{V_{k}} \min _{X \in V_{k}} \frac{X^{\mathrm{H}} A X}{X^{\mathrm{H}} X}=\max _{V_{k}} \min _{X \in V_{k}} R(X)
$$

把 $n-k+1$ 用 $i$ 代替上式得

$$
\lambda_{i}=\max _{V_{n-i+1}} \min _{X \in V_{n-i+1}} R(X)
$$

最后应用 Rayleigh 商研究 Hermite 矩阵特征值的摄动定理,即讨论矩阵的元素发生微小变化时对应矩阵特征值的变化范围。

定理3.11.5 设 $\boldsymbol{A}, \boldsymbol{B}$ 是 Hermite 矩阵, $\boldsymbol{\lambda}_{i}(\boldsymbol{A}), \boldsymbol{\lambda}_{i}(\boldsymbol{B})$ 与 $\boldsymbol{\lambda}_{i}(\boldsymbol{A}+\boldsymbol{B})$ 分别表示矩阵 $\boldsymbol{A}, \boldsymbol{B}$ 与 $\boldsymbol{A}+\boldsymbol{B}$ 的特征值,且特征值从小到大按递增顺序排列。则对于每一个 $k$ ,有

$$
\lambda_{k}(\boldsymbol{A})+\lambda_{1}(\boldsymbol{B}) \leqslant \lambda_{k}(\boldsymbol{A}+\boldsymbol{B}) \leqslant \lambda_{k}(\boldsymbol{A})+\lambda_{n}(\boldsymbol{B})
$$

证明 因为

$$
\begin{aligned}
\lambda_{k}(A+B)= & \max _{V_{n-k+1}} \min _{X \in V_{n-k+1}} \frac{X^{\mathrm{H}}(A+B) X}{X^{\mathrm{H}} X} \\
= & \max _{V_{n-k+1}} \min _{X \in V_{n-k+1}}\left[\frac{X^{\mathrm{H}} A X}{X^{\mathrm{H}} X}+\frac{X^{\mathrm{H}} B X}{X^{\mathrm{H}} X}\right] \leqslant \\
& \max _{V_{n-k+1}} \min _{X \in V_{n-k+1}}\left[\frac{X^{\mathrm{H}} A X}{X^{\mathrm{H}} X}+\lambda_{n}(B)\right] \\
= & \lambda_{k}(A)+\lambda_{n}(B) \\
\lambda_{k}(A+B)= & \max _{V_{n-k+1}} \min _{X \in V_{n-k+1}}\left[\frac{X^{\mathrm{H}} A X}{X^{\mathrm{H}} X}+\frac{X^{\mathrm{H}} B X}{X^{\mathrm{H}} X}\right] \geqslant \\
& \max _{V_{n-k+1}} \min _{X \in V_{n-k+1}}\left[\frac{X^{\mathrm{H}} A X}{X^{\mathrm{H}} X}+\lambda_{1}(B)\right] \\
= & \lambda_{k}(A)+\lambda_{1}(B)
\end{aligned}
$$

例3.11.1 设 $\boldsymbol{A}, \boldsymbol{B}$ 是 Hermite 矩阵,且 $\boldsymbol{B}$ 是半正定的,则

$$
\lambda_{k}(\boldsymbol{A}) \leqslant \lambda_{k}(\boldsymbol{A}+\boldsymbol{B})
$$

解 因为

$$
\lambda_{k}(\boldsymbol{A}+\boldsymbol{B}) \geqslant \lambda_{k}(\boldsymbol{A})+\lambda_{1}(\boldsymbol{B})
$$

由于 $\boldsymbol{B}$ 为半正定矩阵,所以 $\lambda_{1}(\boldsymbol{B}) \geqslant 0$ .从而得到所需结论.

\section*{习 题}
3-1 已知 $\boldsymbol{A}=\left(a_{i j}\right)$ 是 $n$ 阶正定 Hermite 矩阵,在 $n$ 维线性空间 $C^{n}$ 中向量

$$
\boldsymbol{\alpha}=\left[x_{1}, x_{2}, \cdots, x_{n}\right], \quad \boldsymbol{\beta}=\left[y_{1}, y_{2}, \cdots, y_{n}\right]
$$

定义内积 $(\boldsymbol{\alpha}, \boldsymbol{\beta})=\boldsymbol{\alpha} \boldsymbol{A} \boldsymbol{\beta}^{\mathbf{H}}$\\
(1)证明在上述定义下,$C^{n}$ 是酉空间;\\
(2)写出 $C^{n}$ 中的 Canchy-Schwarz 不等式。

3-2 已知 $A=\left(\begin{array}{rrrrr}2 & 1 & -1 & 1 & -3 \\ 1 & 1 & -1 & 0 & 1\end{array}\right)$ ,求 $N(A)$ 的标准正交基.\\
3-3 已知\\
(1) $\boldsymbol{A}=\left[\begin{array}{rrr}3 & 0 & 8 \\ 3 & -1 & 6 \\ -2 & 0 & -5\end{array}\right]$\\
(2)$A=\left[\begin{array}{rrr}-1 & -2 & 6 \\ -1 & 0 & 3 \\ -1 & -1 & 4\end{array}\right]$

试求西矩阵 $\boldsymbol{U}$ ,使得 $\boldsymbol{U}^{\mathrm{H}} \boldsymbol{A U}$ 是上三角矩阵。\\
3-4 试证:在 $\boldsymbol{C}^{n}$ 上的任何一个正交投影矩阵 $\boldsymbol{P}$ 是半正定的 Hermite 矩阵。\\
3-5 验证下列矩阵是正规矩阵,并求酉矩阵 $\boldsymbol{U}$ ,使 $\boldsymbol{U}^{\mathbf{H}} \boldsymbol{A} \boldsymbol{U}$ 为对角矩阵,已知\\
(1)$A=\left[\begin{array}{ccc}\frac{1}{3} & -\frac{1}{3 \sqrt{2}} & -\frac{i}{\sqrt{6}} \\ -\frac{1}{3 \sqrt{2}} & \frac{1}{6} & \frac{i}{2 \sqrt{3}} \\ \frac{i}{\sqrt{6}} & -\frac{i}{2 \sqrt{3}} & \frac{1}{2}\end{array}\right]$\\
(2)$A=\left[\begin{array}{rrr}0 & -1 & i \\ 1 & 0 & 0 \\ i & 0 & 0\end{array}\right]$\\
(3)$A=\frac{1}{9}\left[\begin{array}{rrr}4+3 \mathrm{i} & 4 \mathrm{i} & -6-2 \mathrm{i} \\ -4 \mathrm{i} & 4-3 \mathrm{i} & -2-6 \mathrm{i} \\ 6+2 \mathrm{i} & -2-6 \mathrm{i} & 0\end{array}\right]$\\
(4)$A=\left[\begin{array}{rr}1 & -1 \\ 1 & 1\end{array}\right]$\\
3-6 求正交矩阵 $\boldsymbol{Q}$ ,使 $\boldsymbol{Q}^{\mathrm{T}} \boldsymbol{A} \boldsymbol{Q}$ 为对角矩阵,已知\\
(1)$A=\left[\begin{array}{rrr}2 & -2 & 0 \\ -2 & 1 & -2 \\ 0 & -2 & 0\end{array}\right]$\\
(2) $\boldsymbol{A}=\left[\begin{array}{rrrr}1 & 1 & 0 & -1 \\ 1 & 1 & -1 & 0 \\ 0 & -1 & 1 & 1 \\ -1 & 0 & 1 & 1\end{array}\right]$\\
3-7 试求矩阵 $\boldsymbol{P}$ ,使 $\boldsymbol{P}^{\mathrm{H}} \boldsymbol{A P}=\boldsymbol{E}$(或 $\boldsymbol{P}^{\mathrm{T}} \boldsymbol{A P}=\boldsymbol{E}$ ),已知\\
(1)$A=\left[\begin{array}{rcc}1 & \mathrm{i} & 1+\mathrm{i} \\ -\mathrm{i} & 0 & 1 \\ 1-\mathrm{i} & 1 & 2\end{array}\right]$(2)$A=\left[\begin{array}{rrr}2 & 2 & -2 \\ 2 & 5 & -4 \\ -2 & -4 & 5\end{array}\right]$\\
3-8 设 $n$ 阶酉矩阵 $\boldsymbol{U}$ 的特征根不等于 -1 ,试证:矩阵 $\boldsymbol{E}+\boldsymbol{U}$ 满秩, $\boldsymbol{W}=\mathrm{i}$\\
$(\boldsymbol{E}-\boldsymbol{U})(\boldsymbol{E}+\boldsymbol{U})^{-1}$ 是 Hermite 矩阵。反之,若 $\boldsymbol{W}$ 是 Hermite 矩阵,则 $\boldsymbol{E}-\mathrm{i} \boldsymbol{W}$ 满秩,且 $U=(E+\mathrm{i} W)(E-\mathrm{i} W)^{-1}$ 是酉矩阵。

3-9 若 $S, T$ 分别是实对称和反实对称矩阵,且 $\operatorname{det}(E-T-i S) \neq 0$ ,试证: $(\boldsymbol{E}+\boldsymbol{T}+\mathrm{i} \boldsymbol{S})(\boldsymbol{E}-\boldsymbol{T}-\mathrm{i} \boldsymbol{S})^{-1}$ 是西矩阵。

3-10 设 $\boldsymbol{A} 、 \boldsymbol{B}$ 均是实对称矩阵,试证: $\boldsymbol{A}$ 与 $\boldsymbol{B}$ 正交相似的充要条件是 $\boldsymbol{A}$ 与 $\boldsymbol{B}$ 的特征值相同.

3-11 设 $\boldsymbol{A} 、 \boldsymbol{B}$ 均是 Hermite 矩阵,试证: $\boldsymbol{A}$ 与 $\boldsymbol{B}$ 酉相似的充要条件是 $\boldsymbol{A}$ 与 $\boldsymbol{B}$的特征值相同.

3-12 设 $\boldsymbol{A} 、 \boldsymbol{B}$ 均是正规矩阵,试证: $\boldsymbol{A}$ 与 $\boldsymbol{B}$ 酉相似的充要条件是 $\boldsymbol{A}$ 与 $\boldsymbol{B}$ 的特征值相同。

3-13 设 $\boldsymbol{A}$ 是 Hermite 矩阵,且 $\boldsymbol{A}^{2}=\boldsymbol{A}$ ,则存在酉矩阵 $\boldsymbol{U}$ ,使得

$$
\boldsymbol{U}^{\mathrm{H}} \boldsymbol{A} \boldsymbol{U}=\left[\begin{array}{cc}
\boldsymbol{E}_{r} & 0 \\
0 & 0
\end{array}\right]
$$

3-14 设 $\boldsymbol{A}$ 是 Hermite 矩阵,且 $\boldsymbol{A}^{2}=\boldsymbol{E}$ ,则存在酉矩阵 $\boldsymbol{U}$ ,使得

$$
\boldsymbol{U}^{\mathbf{H}} \boldsymbol{A} \boldsymbol{U}=\left[\begin{array}{cc}
\boldsymbol{E}_{r} & 0 \\
0 & -\boldsymbol{E}_{n-r}
\end{array}\right]
$$

3-15 已知 Hermite 二次型\\
$f(z)=f\left(x_{1}, x_{2}, x_{3}\right)=-i x_{1} \bar{x}_{2}-x_{1} \bar{x}_{3}+i x_{2} \bar{x}_{1}-i x_{2} \bar{x}_{3}-x_{3} \bar{x}_{1}+i x_{3} \bar{x}_{3}$\\
求酉变换 $\boldsymbol{Z}=\boldsymbol{U}_{y}$ ,并将 $f(z)$ 化成标准形.\\
3-16 已知 Hermite 二次型\\
$f\left(x_{1}, x_{2}, x_{3}\right)=\frac{1}{2} \bar{x}_{1} x_{1}+\frac{3}{2} i \bar{x}_{1} x_{3}+2 \bar{x}_{2} x_{2}-\frac{3}{2} i \bar{x}_{3} x_{1}+\frac{1}{2} i \bar{x}_{3} x_{3}$\\
求酉变换 $Z=U y$ 并将 $f\left(x_{1}, x_{2}, x_{3}\right)$ 化成标准形。\\
3-17 设 $\boldsymbol{A}$ 为正定 Hermite 矩阵, $\boldsymbol{B}$ 为反 Hermite 矩阵,试证: $\boldsymbol{A B}$ 与 $\boldsymbol{B A}$ 的特征值实部为 0 .

3-18 设 $\boldsymbol{A} 、 \boldsymbol{B}$ 均是 Hermite 矩阵,且 $\boldsymbol{A}$ 正定,试证: $\boldsymbol{A B}$ 与 $\boldsymbol{B A}$ 的特征值都是实数。

3-19 设 $\boldsymbol{A}$ 是半正定 Hermite 矩阵,且 $\boldsymbol{A} \neq 0$ ,试证:$|\boldsymbol{A}+\boldsymbol{E}|>1$ .\\
3-20 设 $\boldsymbol{A}$ 是半正定 Hermite 矩阵, $\boldsymbol{A} \neq 0, \boldsymbol{B}$ 是正定 Hermite 矩阵,试证: $|A+B|>|A|$ .

3-21 设 $\boldsymbol{A}$ 是正定 Hermite 矩阵,且 $\boldsymbol{A} \in U^{n \times n}$ ,则 $\boldsymbol{A}=\boldsymbol{E}$ 。\\
3-22 试证:(1)两个半正定 Hermite 矩阵之和是半正定的;(2)半正定 Hermite 矩阵与正定 Hermite 矩阵之和是正定的。

3-23 设 $\boldsymbol{A}$ 是正定 Hermite 矩阵, $\boldsymbol{B}$ 是反 Hermite 矩阵,试证: $\boldsymbol{A}+\boldsymbol{B}$ 是可逆矩阵。

3-24 设 $\boldsymbol{A} 、 \boldsymbol{B}$ 是 $n$ 阶正规矩阵,试证: $\boldsymbol{A}$ 与 $\boldsymbol{B}$ 相似的充要条件是 $\boldsymbol{A}$ 与 $\boldsymbol{B}$ 酉

相似。\\
3-25 设 $A^{H}=A$ ,试证:总存在 $t>0$ ,使得 $A+t E$ 是正定 Hermite 矩阵,$A- t \boldsymbol{E}$ 是负定 Hermite 矩阵。

3-26 设 $\boldsymbol{A}, \boldsymbol{B}$ 均为正规矩阵。且 $\boldsymbol{A} \boldsymbol{B}=\boldsymbol{B} \boldsymbol{A}$ ,则 $\boldsymbol{A} \boldsymbol{B}$ 与 $\boldsymbol{B} \boldsymbol{A}$ 均为正规矩阵。\\
3-27 设 $\boldsymbol{A}^{\mathbf{H}}=-\boldsymbol{A}$ ,试证: $\boldsymbol{U}=(\boldsymbol{A}+\boldsymbol{E})^{-1}(\boldsymbol{A}-\boldsymbol{E})$ 是酉矩阵。\\
3-28 设 $\boldsymbol{A}$ 为 $n$ 阶正规矩阵, $\boldsymbol{\lambda}_{1}, \boldsymbol{\lambda}_{2}, \cdots, \boldsymbol{\lambda}_{n}$ 为 $\boldsymbol{A}$ 的特征值,试证: $\boldsymbol{A}^{\mathrm{H}} \boldsymbol{A}$ 的特征值为 $\left|\lambda_{1}\right|^{2},\left|\lambda_{2}\right|^{2}, \cdots,\left|\lambda_{n}\right|^{2}$ 。

3-29 设 $\boldsymbol{A} \in C^{m \times n}$ ,试证:(1) $\boldsymbol{A}^{\mathrm{H}} \boldsymbol{A}$ 和 $\boldsymbol{A} \boldsymbol{A}^{\mathrm{H}}$ 都是半正定的 Hermite 矩阵; (2) $\boldsymbol{A}^{\mathrm{H}} \boldsymbol{A}$ 和 $\boldsymbol{A} \boldsymbol{A}^{\mathrm{H}}$ 的非零特征值相同.

3-30 设 $\boldsymbol{A}$ 是正规矩阵。试证:(1)若 $\boldsymbol{A}^{r}=0$( $r$ 是自然数),则 $\boldsymbol{A}=0$ ;\\
(2)若 $A^{2}=A$ ,则 $A^{\mathrm{H}}=A$ ;(3)若 $A^{3}=A^{2}$ ,则 $A^{2}=A$ .\\
3-31 设 $A^{\mathrm{H}}=A, B^{\mathrm{H}}=-B$ ,证明以下三个条件等价:\\
(1) $\boldsymbol{A}+\boldsymbol{B}$ 为正规矩阵;(2) $\boldsymbol{A} \boldsymbol{B}=\boldsymbol{B} \boldsymbol{A}$ ;(3)$(\boldsymbol{A B})^{\mathrm{H}}=-\boldsymbol{A B}$ 。\\
3-32 设 $A \in C^{n \times n}$ ,那么 $A$ 可以唯一的写成 $A=S+\mathrm{i} T$ ,其中 $S, T$ 为 Hermite矩阵,且 $\boldsymbol{A}$ 可以唯一的写成 $\boldsymbol{A}=\boldsymbol{B}+\boldsymbol{C}$ ,其中 $\boldsymbol{B}$ 是 Hermite 矩阵, $\boldsymbol{C}$ 是反 Hermite 矩阵。

\section*{矩 阵 分 解}
在前面讨论正定 Hermite 矩阵的分解后,本章继续介绍矩阵的五种分解:矩阵的满秩分解、正交三角分解、奇异值分解、极分解和谱分解。

\section*{§4.1 矩阵的满秩分解}
由线性代数知道,可以对矩阵 $\boldsymbol{A}$ 只作初等行变换求得矩阵秩。同样,也可只作初等行变换得到矩阵的满秩分解。

定理4.1.1 设 $A \in C_{r}^{m \times n}$ ,则存在 $B \in C_{r}^{m \times r}, C \in C_{r}^{r \times n}$ ,满足


\begin{equation*}
A=B C \tag{4.1.1}
\end{equation*}


证明 首先设 $\boldsymbol{A}$ 的前 $r$ 个列向量是线性无关的.对矩阵 $\boldsymbol{A}$ 只作初等行变换可把 $\boldsymbol{A}$ 变为

$$
\left[\begin{array}{cc}
E_{r} & D \\
0 & 0
\end{array}\right]
$$

此即存在 $\boldsymbol{P} \in \boldsymbol{C}_{m}^{m \times m}$ ,满足

或

$$
\begin{aligned}
P A & =\left[\begin{array}{cc}
E_{r} & D \\
0 & 0
\end{array}\right] \\
A & =P^{-1}\left[\begin{array}{cc}
E_{r} & D \\
0 & 0
\end{array}\right] \\
& =P^{-1}\left[\begin{array}{c}
E_{r} \\
0
\end{array}\right]\left[\begin{array}{ll}
E_{r} & D
\end{array}\right]=B C
\end{aligned}
$$

其中

$$
\boldsymbol{B}=\boldsymbol{P}^{-1}\left[\begin{array}{l}
\boldsymbol{E}_{r} \\
0
\end{array}\right] \in C_{r}^{m \times r}, \quad \boldsymbol{C}=\left[\begin{array}{ll}
\boldsymbol{E}_{r} & \boldsymbol{D}
\end{array}\right] \in C_{r}^{r \times n}
$$

现在设 $\boldsymbol{A}$ 的前 $r$ 个列向量线性相关的情况。这只需把 $\boldsymbol{A}$ 先作列变换,使得前 $r$ 个列向量线性无关,然后用刚才证明的方法即可。此即存在 $\boldsymbol{P} \in \boldsymbol{C}_{m}^{m \times m}, \boldsymbol{Q} \in C_{n}^{n \times n}$ ,满足

$$
P A Q=\left[\begin{array}{cc}
E_{r} & D \\
0 & 0
\end{array}\right]
$$

或

$$
\begin{aligned}
A & =P^{-1}\left[\begin{array}{cc}
E_{r} & D \\
0 & 0
\end{array}\right] Q^{-1} \\
& =P^{-1}\left[\begin{array}{l}
E_{r} \\
0
\end{array}\right]\left[\begin{array}{ll}
E_{r} & D
\end{array}\right] Q^{-1} \\
& =B C
\end{aligned}
$$

其中

$$
\boldsymbol{B}=\boldsymbol{P}^{-1}\left[\begin{array}{l}
\boldsymbol{E}_{r} \\
0
\end{array}\right] \in C_{r}^{m \times r}, \boldsymbol{C}=\left[\begin{array}{ll}
\boldsymbol{E}_{r} & \boldsymbol{D}
\end{array}\right] \boldsymbol{Q}^{-1} \in C_{r}^{r \times n}
$$

定理证毕。\\
对秩为 $r$ 的矩阵 $\boldsymbol{A}$ 作满秩分解时,无论 $\boldsymbol{A}$ 的前 $r$ 个列向量是线性无关还是线性相关,都是对 $\boldsymbol{A}$ 只作初等行变换就可以得到 $\boldsymbol{A}$ 的满秩分解。下面例4.1.1与例 4.1. 2 分别对这两种情况说明。

例4.1.1 求矩阵

$$
A=\left[\begin{array}{rrrrr}
1 & 4 & -1 & 5 & 6 \\
2 & 0 & 0 & 0 & -14 \\
-1 & 2 & -4 & 0 & 1 \\
2 & 6 & -5 & 5 & -7
\end{array}\right]
$$

的满秩分解.\\
解 对矩阵 $\boldsymbol{A}$ 只作初等行变换

$$
A=\left[\begin{array}{rrrrr}
1 & 4 & -1 & 5 & 6 \\
2 & 0 & 0 & 0 & -14 \\
-1 & 2 & -4 & 0 & 1 \\
2 & 6 & -5 & 5 & -7
\end{array}\right] \rightarrow \cdots \rightarrow\left[\begin{array}{ccccc}
1 & 0 & 0 & 0 & -7 \\
0 & 1 & 0 & \frac{10}{7} & \frac{29}{7} \\
0 & 0 & 1 & \frac{5}{7} & \frac{25}{7} \\
0 & 0 & 0 & 0 & 0
\end{array}\right]
$$

$\boldsymbol{A}$ 的秩为 3 ,且前三个列向量线性无关,故

$$
B=\left[\begin{array}{rrr}
1 & 4 & -1 \\
2 & 0 & 0 \\
-1 & 2 & -4 \\
2 & 6 & -5
\end{array}\right], \quad C=\left[\begin{array}{ccccc}
1 & 0 & 0 & 0 & -7 \\
0 & 1 & 0 & \frac{10}{7} & \frac{29}{7} \\
0 & 0 & 1 & \frac{5}{7} & \frac{25}{7}
\end{array}\right]
$$

容易验证

$$
B C=A
$$

例4.1.2 求矩阵

$$
A=\left[\begin{array}{ccccc}
1 & 3 & 2 & 1 & 4 \\
2 & 6 & 1 & 0 & 7 \\
3 & 9 & 3 & 1 & 11
\end{array}\right]
$$

的满秩分解。\\
解 对 $\boldsymbol{A}$ 只作初等行变换

$$
\boldsymbol{A}=\left[\begin{array}{ccccc}
1 & 3 & 2 & 1 & 4 \\
2 & 6 & 1 & 0 & 7 \\
3 & 9 & 3 & 1 & 11
\end{array}\right] \rightarrow \cdots \rightarrow\left[\begin{array}{ccccc}
1 & 3 & 0 & -\frac{1}{3} & \frac{10}{3} \\
0 & 0 & 1 & \frac{2}{3} & \frac{1}{3} \\
0 & 0 & 0 & 0 & 0
\end{array}\right]
$$

$\boldsymbol{A}$ 的秩为 2 ,且第一、第三列向量线性无关,故

$$
B=\left[\begin{array}{ll}
1 & 2 \\
2 & 1 \\
3 & 3
\end{array}\right], \quad C=\left[\begin{array}{rrrrr}
1 & 3 & 0 & -\frac{1}{3} & \frac{10}{3} \\
0 & 0 & 1 & \frac{2}{3} & \frac{1}{3}
\end{array}\right]
$$

容易验证

$$
B C=A
$$

$\boldsymbol{A}$ 的第二、第三列向量线性无关,故

$$
\boldsymbol{B}=\left[\begin{array}{ll}
3 & 2 \\
6 & 1 \\
9 & 3
\end{array}\right], \quad C=\left[\begin{array}{rrrrr}
\frac{1}{3} & 1 & 0 & -\frac{1}{9} & \frac{10}{9} \\
0 & 0 & 1 & \frac{2}{3} & \frac{1}{3}
\end{array}\right]
$$

容易验证

$$
B C=A
$$

矩阵的满秩分解是不唯一的,但是不同的分解之间有以下关系:\\
定理4.1.2 若 $\boldsymbol{A}=\boldsymbol{B} \boldsymbol{C}=\boldsymbol{B}_{1} \boldsymbol{C}_{1}$ 均为 $\boldsymbol{A}$ 的满秩分解,则\\
(1)存在 $\boldsymbol{\theta} \in C_{r}^{r \times r}$ ,满足 $\boldsymbol{B}=\boldsymbol{B}_{1} \boldsymbol{\theta}, \boldsymbol{C}=\boldsymbol{\theta}^{-1} \boldsymbol{C}_{1}$\\
(2) $\boldsymbol{C}^{\mathrm{H}}\left(\boldsymbol{C C}^{\mathrm{H}}\right)^{-1}\left(\boldsymbol{B}^{\mathrm{H}} \boldsymbol{B}\right)^{-1} \boldsymbol{B}^{\mathrm{H}}=\boldsymbol{C}_{1}^{\mathrm{H}}\left(C_{1} \boldsymbol{C}_{1}^{\mathrm{H}}\right)^{-1}\left(\boldsymbol{B}_{1}^{\mathrm{H}} \boldsymbol{B}_{1}\right)^{-1} \boldsymbol{B}_{1}^{\mathrm{H}}$

证明(1)由 $B C=B_{1} C_{1}$ ,故


\begin{equation*}
B C C^{\mathrm{H}}=B_{1} C_{1} C^{\mathrm{H}} \tag{3}
\end{equation*}


因为 $\operatorname{rank} C=\operatorname{rank} C C^{\mathrm{H}}=r$(参阅§4.3节引理4.3.1),$C C^{\mathrm{H}} \in C_{r}^{r \times r}$ ,由式(3)可知


\begin{equation*}
\boldsymbol{B}=\boldsymbol{B}_{1} \boldsymbol{C}_{1} \boldsymbol{C}^{\mathrm{H}}\left(\boldsymbol{C} \boldsymbol{C}^{\mathrm{H}}\right)^{-1}=\boldsymbol{B}_{1} \boldsymbol{\theta}_{1} \tag{4}
\end{equation*}


其中

$$
\boldsymbol{\theta}_{1}=\boldsymbol{C}_{1} \boldsymbol{C}^{\mathrm{H}}\left(\boldsymbol{C} \boldsymbol{C}^{\mathrm{H}}\right)^{-1}
$$

同理可得


\begin{equation*}
\boldsymbol{C}=\left(\boldsymbol{B}^{\mathrm{H}} \boldsymbol{B}\right)^{-1} \boldsymbol{B}^{\mathrm{H}} \boldsymbol{B}_{1} \boldsymbol{C}_{1}=\boldsymbol{\theta}_{2} \boldsymbol{C}_{1} \tag{5}
\end{equation*}


其中

$$
\boldsymbol{\theta}_{2}=\left(\boldsymbol{B}^{\mathrm{H}} \boldsymbol{B}\right)^{-1} \boldsymbol{B}^{\mathrm{H}} \boldsymbol{B}_{1}
$$

将式(4)与式(5)代入 $\boldsymbol{B} \boldsymbol{C}=\boldsymbol{B}_{1} \boldsymbol{C}_{1}$ ,可得

$$
B_{1} C_{1}=B_{1} \theta_{1} \theta_{2} C_{1}
$$

因此

$$
\boldsymbol{B}_{1}^{\mathrm{H}} \boldsymbol{B}_{1} \boldsymbol{C}_{1} \boldsymbol{C}_{1}^{\mathrm{H}}=\boldsymbol{B}_{1}^{\mathrm{H}} \boldsymbol{B}_{1} \boldsymbol{\theta}_{1} \boldsymbol{\theta}_{2} \boldsymbol{C}_{1} \boldsymbol{C}_{1}^{\mathrm{H}}
$$

其中 $\boldsymbol{B}_{1}^{\mathrm{H}} \boldsymbol{B}_{1}, \boldsymbol{C}_{1} \boldsymbol{C}_{1}^{\mathrm{H}}$ 都是可逆矩阵,因此

$$
\boldsymbol{\theta}_{1} \boldsymbol{\theta}_{2}=\boldsymbol{E}
$$

$\boldsymbol{\theta}_{1} \boldsymbol{\theta}_{2}$ 是 $r$ 阶方阵,故式(1)成立。\\
式(1)代人 $\boldsymbol{C}^{\mathrm{H}}\left(\boldsymbol{C} \boldsymbol{C}^{\mathrm{H}}\right)^{-1}\left(\boldsymbol{B}^{\mathrm{H}} \boldsymbol{B}\right)^{-1} \boldsymbol{B}^{\mathrm{H}}$ 可得式(2).

\section*{§4.2 矩阵的正交三角分解( $U R 、 Q R$ 分解)}
作为 Schmidt 方法的应用,本节介绍将其用于矩阵分解上,得出矩阵的正交三角分解,称 $U R$ 分解或 $Q R$ 分解。

定理 4.2.1 设 $\boldsymbol{A} \in C_{n}^{n \times n}$ ,则 $\boldsymbol{A}$ 可以唯一地分解为

或


\begin{gather*}
A=U R  \tag{4.2.1}\\
A=R_{1} U_{1} \tag{4.2.2}
\end{gather*}


其中 $\boldsymbol{U}, \boldsymbol{U}_{1} \in U^{n \times n}, \boldsymbol{R}$ 是正线上三角阵, $\boldsymbol{R}_{1}$ 是正线下三角阵。(即 $\boldsymbol{R}$ 和 $\boldsymbol{R}_{1}$ 的主对角线上元素全是正的).

证明 把 $\boldsymbol{A}$ 按列向量分块

$$
\boldsymbol{A}=\left(\boldsymbol{\alpha}_{1}, \boldsymbol{\alpha}_{2}, \cdots, \boldsymbol{\alpha}_{n}\right)
$$

由于 $\boldsymbol{A} \in C_{n}^{n \times n}$ ,故 $\boldsymbol{\alpha}_{1}, \boldsymbol{\alpha}_{2}, \cdots, \boldsymbol{\alpha}_{n}$ 线性无关,用 Schmidt 方法将 $\boldsymbol{\alpha}_{i}$ 正交化得 $\boldsymbol{\beta}_{1}$ , $\boldsymbol{\beta}_{2}, \cdots, \boldsymbol{\beta}_{n}$ ,再单位化得 $v_{1}, v_{2}, \cdots, v_{n}$ .且可得

$$
\begin{aligned}
\boldsymbol{\alpha}_{1}= & c_{11} \boldsymbol{v}_{1} \\
\boldsymbol{\alpha}_{2}= & c_{21} \boldsymbol{v}_{1}+c_{22} \boldsymbol{v}_{2} \\
\boldsymbol{\alpha}_{3}= & c_{31} \boldsymbol{v}_{1}+c_{32} \boldsymbol{v}_{2}+c_{33} \boldsymbol{v}_{3} \\
& \vdots \\
\boldsymbol{\alpha}_{n}= & c_{n 1} \boldsymbol{v}_{1}+c_{n 2} \boldsymbol{v}_{2}+\cdots+c_{n n} \boldsymbol{v}_{n}
\end{aligned}
$$

其中 $c_{i i}=\left|\boldsymbol{\beta}_{i}\right|>0$ ,于是

$$
\begin{aligned}
\boldsymbol{A} & =\left(\boldsymbol{\alpha}_{1}, \boldsymbol{\alpha}_{2}, \cdots, \boldsymbol{\alpha}_{n}\right) \\
& =\left(c_{11} \boldsymbol{v}_{1}, c_{21} \boldsymbol{v}_{1}+c_{22} \boldsymbol{v}_{2}, \cdots, c_{n 1} \boldsymbol{v}_{1}+c_{n 2} \boldsymbol{v}_{2}+\cdots+c_{n n} \boldsymbol{v}_{n}\right) \\
& =\left(\boldsymbol{v}_{1}, \boldsymbol{v}_{2}, \cdots, \boldsymbol{v}_{n}\right)\left[\begin{array}{cccc}
c_{11} & c_{21} & \cdots & c_{n 1} \\
& c_{22} & \cdots & c_{n 2} \\
& & \ddots & \vdots \\
& 0 & & c_{n n}
\end{array}\right] \\
& =\boldsymbol{U} \boldsymbol{R}
\end{aligned}
$$

其中 $\boldsymbol{U}=\left(\boldsymbol{v}_{1}, \boldsymbol{v}_{2}, \cdots, \boldsymbol{v}_{n}\right) \in U^{n \times n}, \boldsymbol{R}$ 是正线上三角阵。\\
现证分解的唯一性.设 $\boldsymbol{A}$ 有两个分解式

$$
\begin{gathered}
A=U R=\tilde{U} \tilde{R} \\
\tilde{U}^{-1} U=\tilde{R} R^{-1}
\end{gathered}
$$

则\\
由于 $\tilde{\boldsymbol{U}}^{-1} \boldsymbol{U}$ 是西矩阵,$\tilde{\boldsymbol{R}} \boldsymbol{R}^{-1}$ 是正线上三角阵,故由引理 3.9.1 知

$$
\begin{aligned}
\tilde{U}^{-1} U & =E, \tilde{R} R^{-1}=E \\
U & =\tilde{U}, R=\tilde{R}
\end{aligned}
$$

因此\\
又因 $\boldsymbol{A} \in C_{n}^{n \times n}$ ,故 $\boldsymbol{A}^{\mathrm{T}} \in C_{n}^{n \times n}$ .根据刚才已证式(4.2.1)得

$$
\boldsymbol{A}^{\mathrm{T}}=\tilde{U} \tilde{\boldsymbol{R}}
$$

其中 $\tilde{\boldsymbol{U}} \in U^{n \times n}, \tilde{\boldsymbol{R}}$ 是正线上三角阵。于是

$$
\boldsymbol{A}=\tilde{\boldsymbol{R}}^{\mathrm{T}} \tilde{\boldsymbol{U}}^{\mathrm{T}}=\boldsymbol{R}_{1} \boldsymbol{U}_{1}
$$

其中 $\boldsymbol{R}_{1}$ 是正线下三角阵, $\boldsymbol{U}_{1}$ 是西矩阵。\\
用类似方法可以证明下述定理。\\
定理 4.2.2 设 $\boldsymbol{A} \in C_{r}^{m \times r}$(称 $\boldsymbol{A}$ 是列满秩矩阵),则 $\boldsymbol{A}$ 可以唯一地分解为


\begin{equation*}
A=U R \tag{4.2.3}
\end{equation*}


其中 $\boldsymbol{U} \in U_{r}^{m \times r}, \boldsymbol{R}$ 是 $r$ 阶正线上三角阵。\\
证明 设 $\boldsymbol{A}=\left(\boldsymbol{\alpha}_{1}, \boldsymbol{\alpha}_{2}, \cdots, \boldsymbol{\alpha}_{r}\right)$ ,将 $\boldsymbol{\alpha}_{1}, \boldsymbol{\alpha}_{2}, \cdots, \boldsymbol{\alpha}_{r}$ 用 Schmidt 方法标准正交化得 $v_{1}, v_{2}, \cdots, v_{r}$ ,且有 $r$ 个关系式:

$$
\begin{aligned}
\boldsymbol{\alpha}_{1} & =k_{11} v_{1} \\
\boldsymbol{\alpha}_{2} & =k_{21} v_{1}+k_{22} v_{2} \\
\vdots & \\
\boldsymbol{\alpha}_{\mathrm{r}} & =k_{r 1} v_{1}+k_{r 2} v_{2}+\cdots+k_{r r} v_{r}
\end{aligned}
$$

则

$$
\begin{aligned}
\boldsymbol{A} & =\left(\boldsymbol{\alpha}_{1}, \boldsymbol{\alpha}_{2}, \cdots, \boldsymbol{\alpha}_{r}\right)=\left(\boldsymbol{v}_{1}, \boldsymbol{v}_{2}, \cdots, \boldsymbol{v}_{r}\right)\left[\begin{array}{cccc}
k_{11} & k_{21} & \cdots & k_{r 1} \\
& k_{22} & \cdots & k_{r 2} \\
& & \ddots & \vdots \\
& & & k_{r r}
\end{array}\right] \\
& =\boldsymbol{U} \boldsymbol{R}
\end{aligned}
$$

其中 $\boldsymbol{U}=\left(\boldsymbol{v}_{1}, \boldsymbol{v}_{2}, \cdots, \boldsymbol{v}_{r}\right) \in U_{r}^{m \times r}, \boldsymbol{R}=\left(k_{i j}\right)$ 是正线上三角阵。\\
现证分解是唯一的。设

则

$$
\begin{gathered}
\boldsymbol{A}=\boldsymbol{U}_{1} \boldsymbol{R}_{1}=\boldsymbol{U}_{2} \boldsymbol{R}_{2} \\
\boldsymbol{A}^{\mathrm{H}} \boldsymbol{A}=\boldsymbol{R}_{1}^{\mathrm{H}} \boldsymbol{R}_{1}=\boldsymbol{R}_{2}^{\mathrm{H}} \boldsymbol{R}_{2}
\end{gathered}
$$

因为 $\boldsymbol{A}^{\mathrm{H}} \boldsymbol{A}$ 是正定 Hermite 矩阵,它的三角分解是唯一的,故 $\boldsymbol{R}_{1}=\boldsymbol{R}_{2}$ ,于是 $\boldsymbol{U}_{1}=\boldsymbol{U}_{2}$ .\\
推论 4.2.1 若 $\boldsymbol{A} \in C_{\mathrm{r}}^{r \times n}$ ,则 $\boldsymbol{A}$ 可以唯一地分解成


\begin{equation*}
A=L U \tag{4.2.4}
\end{equation*}


其中 $\boldsymbol{L}$ 是 $r$ 阶正线下三角矩阵,$U \in U_{r}^{r \times n}$ 。\\
证明 由定理4.2.2 知 $\boldsymbol{A}^{\mathrm{T}}=\boldsymbol{U}_{1} \boldsymbol{R}$ ,故 $\boldsymbol{A}=\boldsymbol{R}^{\mathrm{T}} \boldsymbol{U}_{1}^{\mathrm{T}}$ .这里 $\boldsymbol{R}^{\mathrm{T}}$ 即为 $r$ 阶正线下三角矩阵, $\boldsymbol{U}_{1}^{\mathrm{T}} \in U_{r}^{r \times n}$ 。

定理4.2.3 若 $\boldsymbol{A} \in C_{r}^{m \times n}$ ,则 $\boldsymbol{A}$ 可以分解为


\begin{equation*}
A=U_{1} R_{1} L_{2} U_{2} \tag{4.2.5}
\end{equation*}


其中 $\boldsymbol{U}_{1} \in U_{r}^{m \times r}, \boldsymbol{U}_{2} \in U_{r}^{r \times n}, \boldsymbol{R}_{1}$ 是 $r$ 阶正线上三角阵, $\boldsymbol{L}_{2}$ 是 $r$ 阶正线下三角阵。\\
证明 根据满秩分解定理知

$$
A=B C
$$

对 $\boldsymbol{B}, \boldsymbol{C}$ 分别用定理4.2.2 及其推论即得式(4.2.5)。\\
下述例4.2.1说明 $\boldsymbol{U} \boldsymbol{R}$ 分解在解方程组上的一个应用,并介绍如何对 $\boldsymbol{A}$ 作 $\boldsymbol{U} \boldsymbol{R}$分解。

例 4.2.1 用 UR 分解方法解方程组

$$
A x=b
$$

其中

$$
\boldsymbol{A}=\left[\begin{array}{rrr}
-3 & 1 & -2 \\
1 & 1 & 1 \\
1 & -1 & 0 \\
1 & -1 & 1
\end{array}\right], \boldsymbol{b}=\left[\begin{array}{r}
1 \\
0 \\
-2 \\
1
\end{array}\right]
$$

解 将 $\boldsymbol{A}=\left(\boldsymbol{\alpha}_{1}, \boldsymbol{\alpha}_{2}, \boldsymbol{\alpha}_{3}\right)$ 的列向量 $\boldsymbol{\alpha}_{1}, \boldsymbol{\alpha}_{2}, \boldsymbol{\alpha}_{3}$ 用 Schmidt 方法标准正交化得

$$
\begin{aligned}
& v_{1}=\left(-\frac{3}{\sqrt{12}}, \frac{1}{\sqrt{12}}, \frac{1}{\sqrt{12}}, \frac{1}{\sqrt{12}}\right)^{\mathrm{T}} \\
& v_{2}=\left(0, \frac{2}{\sqrt{6}},-\frac{1}{\sqrt{6}},-\frac{1}{\sqrt{6}}\right)^{\mathrm{T}} \\
& v_{3}=\left(0,0,-\frac{1}{\sqrt{2}}, \frac{1}{\sqrt{2}}\right)^{\mathrm{T}}
\end{aligned}
$$

命 $U=\left(v_{1}, v_{2}, v_{3}\right)$ ,则

$$
\boldsymbol{R}=\boldsymbol{U}^{\mathrm{H}} \boldsymbol{A}=\left[\begin{array}{ccc}
2 \sqrt{3} & -\frac{2}{\sqrt{3}} & \frac{4}{\sqrt{3}} \\
0 & \frac{4}{\sqrt{6}} & \frac{1}{\sqrt{6}} \\
0 & 0 & \frac{1}{\sqrt{2}}
\end{array}\right]
$$

$$
\begin{gathered}
\boldsymbol{R}^{-1}=\left[\begin{array}{ccc}
\frac{\sqrt{3}}{6} & \frac{\sqrt{6}}{12} & -\frac{3 \sqrt{2}}{4} \\
0 & \frac{\sqrt{6}}{4} & -\frac{\sqrt{2}}{4} \\
0 & 0 & \sqrt{2}
\end{array}\right] \\
\boldsymbol{x}=\boldsymbol{R}^{-1} \boldsymbol{U}^{\mathrm{H}} \boldsymbol{b}=\left[\begin{array}{cccc}
-\frac{1}{4} & \frac{1}{4} & \frac{3}{4} & -\frac{3}{4} \\
0 & \frac{1}{2} & 0 & -\frac{1}{2} \\
0 & 0 & -1 & 1
\end{array}\right]\left[\begin{array}{r}
1 \\
0 \\
-2 \\
1
\end{array}\right]=\left[\begin{array}{r}
-\frac{5}{2} \\
-\frac{1}{2} \\
3
\end{array}\right]
\end{gathered}
$$

读者不难验证: $\boldsymbol{A x}=\boldsymbol{b}$ .

\section*{§4.3 矩阵的奇异值分解}
为了引入矩阵的奇异值,先介绍两个引理。\\
引理4.3.1 对于任何一个矩阵 $\boldsymbol{A}$ 都有

$$
\operatorname{rank}\left(\boldsymbol{A} \boldsymbol{A}^{\mathrm{H}}\right)=\operatorname{rank}\left(\boldsymbol{A}^{\mathrm{H}} \boldsymbol{A}\right)=\operatorname{rank} \boldsymbol{A}
$$

证明 若 $x \in C^{n}$ 是 $\boldsymbol{A}^{\mathrm{H}} \boldsymbol{A x}=0$ 的解,则 $\boldsymbol{x}^{\mathrm{H}} \boldsymbol{A}^{\mathrm{H}} \boldsymbol{A x}=0$ ,即 $(\boldsymbol{A x})^{\mathrm{H}}(\boldsymbol{A x})=0$ ,因此 $A x=0$ ,这表明 $x$ 也是 $A x=0$ 的解.反之,若 $x$ 是 $A x=0$ 的解,也必是 $A^{\mathrm{H}} A x=0$ 的解.所以 $\boldsymbol{A x}=0$ 与 $\boldsymbol{A}^{\mathrm{H}} \boldsymbol{A x}=0$ 是同解方程组.故 $\operatorname{rank}\left(\boldsymbol{A}^{\mathrm{H}} \boldsymbol{A}\right)=\operatorname{rank} \boldsymbol{A}$ .

又因 $\operatorname{rank} \boldsymbol{A}=\operatorname{rank} \boldsymbol{A}^{\mathrm{H}}$ 。证毕。\\
引理4.3.2 对于任何一个矩阵 $\boldsymbol{A}$ 都有 $\boldsymbol{A}^{\mathrm{H}} \boldsymbol{A}$ 与 $\boldsymbol{A} \boldsymbol{A}^{\mathrm{H}}$ 是半正定 Hermite 矩阵。证略。\\
设 $\boldsymbol{A} \in C_{r}^{m \times n}, \lambda_{i}$ 是 $\boldsymbol{A} \boldsymbol{A}^{\mathrm{H}}$ 的特征值,$\mu_{i}$ 是 $\boldsymbol{A}^{\mathrm{H}} \boldsymbol{A}$ 的特征值,它们都是实数。且设

$$
\begin{aligned}
& \lambda_{1} \geqslant \lambda_{2} \geqslant \cdots \geqslant \lambda_{r}>\lambda_{r+1}=\lambda_{r+2}=\cdots=\lambda_{m}=0 \\
& \mu_{1} \geqslant \mu_{2} \geqslant \cdots \geqslant \mu_{r}>\mu_{r+1}=\mu_{r+2}=\cdots=\mu_{n}=0
\end{aligned}
$$

特征值 $\lambda_{i}$ 与 $\mu_{i}$ 之间有下面定理。\\
定理4.3.1 设 $A \in C_{r}^{m \times n}$ ,则

$$
\lambda_{i}=\mu_{i}>0 \quad(i=1,2, \cdots, r)
$$

证明 由 $\boldsymbol{A} \boldsymbol{A}^{\mathrm{H}} \boldsymbol{x}=\lambda_{i} \boldsymbol{x}$ 可得 $\boldsymbol{A}^{\mathrm{H}} \boldsymbol{A} \boldsymbol{A}^{\mathrm{H}} \boldsymbol{x}=\lambda_{i} \boldsymbol{A}^{\mathrm{H}} \boldsymbol{x}$ ,这表明 $\lambda_{i}$ 既是 $\boldsymbol{A} \boldsymbol{A}^{\mathrm{H}}$ 的特征值,又是 $\boldsymbol{A}^{\mathrm{H}} \boldsymbol{A}$ 的特征值。同理可证 $\mu_{i}$ 也是 $\boldsymbol{A} \boldsymbol{A}^{\mathrm{H}}$ 的特征值(但不能认为 $\lambda_{i}=\mu_{i}$ )。

设 $x_{1}, x_{2}, \cdots, x_{p}$ 是 $\boldsymbol{A} \boldsymbol{A}^{\mathrm{H}}$ 对应于 $\lambda_{i} \neq 0$ 的线性无关特征向量,由上述讨论可知 $\boldsymbol{A}^{\mathrm{H}} x_{1}, \boldsymbol{A}^{\mathrm{H}} \boldsymbol{x}_{2}, \cdots, \boldsymbol{A}^{\mathrm{H}} \boldsymbol{x}_{p}$ 是 $\boldsymbol{A}^{\mathrm{H}} \boldsymbol{A}$ 的特征值为 $\lambda_{i}$ 的特征向量,不难证明它们是线性无关的。这说明 $\boldsymbol{A} \boldsymbol{A}^{\mathrm{H}}$ 的 $p$ 重特征值也是 $\boldsymbol{A}^{\mathrm{H}} \boldsymbol{A}$ 的 $p$ 重特征值。因此 $\lambda_{i}=\mu_{i}>0$ 。

定义 4.3.1 设 $\boldsymbol{A} \in C_{r}^{m \times n}, \boldsymbol{A} \boldsymbol{A}^{\mathrm{H}}$ 的正特征值 $\lambda_{i}, \boldsymbol{A}^{\mathrm{H}} \boldsymbol{A}$ 的正特征值 $\mu_{i}$ ,称


\begin{equation*}
\alpha_{i}=\sqrt{\lambda_{i}}=\sqrt{\mu_{i}} \quad(i=1,2, \cdots, r) \tag{4.3.1}
\end{equation*}


是 $\boldsymbol{A}$ 的正奇异值,简称奇异值.\\
例4.3.1 已知

$$
A=\left[\begin{array}{ll}
1 & 2 \\
0 & 0 \\
0 & 0
\end{array}\right]
$$

求 $\boldsymbol{A}$ 的奇异值。\\
解 因为

$$
\boldsymbol{A} \boldsymbol{A}^{\mathrm{H}}=\left[\begin{array}{lll}
5 & 0 & 0 \\
0 & 0 & 0 \\
0 & 0 & 0
\end{array}\right]
$$

$\boldsymbol{A} \boldsymbol{A}^{\mathrm{H}}$ 的特征值为 $5,0,0$ ,故 $\boldsymbol{A}$ 的奇异值为 $\sqrt{5}$ .\\
定理4.3.2 若 $\boldsymbol{A}$ 是正规矩阵,则 $\boldsymbol{A}$ 的奇异值是 $\boldsymbol{A}$ 的非零特征值的模长。\\
证明 对于正规矩阵 $\boldsymbol{A}$ ,存在西矩阵 $\boldsymbol{U}$ ,满足

故

$$
\begin{gathered}
\boldsymbol{A}=\boldsymbol{U} \operatorname{diag}\left(\lambda_{1}, \lambda_{2}, \cdots, \lambda_{n}\right) \boldsymbol{U}^{\mathrm{H}} \\
\boldsymbol{A}^{\mathrm{H}}=\boldsymbol{U} \operatorname{diag}\left(\bar{\lambda}_{1}, \bar{\lambda}_{2}, \cdots, \bar{\lambda}_{n}\right) \boldsymbol{U}^{\mathrm{H}} \\
\boldsymbol{A} \boldsymbol{A}^{\mathrm{H}}=\boldsymbol{U} \operatorname{diag}\left(\lambda_{1} \bar{\lambda}_{1}, \lambda_{2} \bar{\lambda}_{2}, \cdots, \lambda_{n} \bar{\lambda}_{n}\right) \boldsymbol{U}^{\mathrm{H}}
\end{gathered}
$$

于是 $\boldsymbol{A} \boldsymbol{A}^{\mathrm{H}}$ 的特征值为 $\lambda_{1} \bar{\lambda}_{1}, \lambda_{2} \bar{\lambda}_{2}, \cdots, \lambda_{n} \bar{\lambda}_{n}$ .证毕.\\
定理 4.3.3 若 $\boldsymbol{A} \in C_{r}^{m \times n}, \delta_{1} \geqslant \delta_{2} \geqslant \cdots \geqslant \delta_{r}$ 是 $\boldsymbol{A}$ 的 $r$ 个正奇异值,则存在 $m$ 阶酉矩阵 $\boldsymbol{U}$ 和 $n$ 阶酉矩阵 $\boldsymbol{V}$ ,满足

\[
A=U D V^{\mathrm{H}}=U\left[\begin{array}{ll}
\Delta & 0  \tag{4.3.2}\\
0 & 0
\end{array}\right] V^{\mathrm{H}}
\]

其中, $\boldsymbol{\Delta}=\operatorname{diag}\left(\delta_{1}, \delta_{2}, \cdots, \delta_{r}\right), \boldsymbol{U}$ 满足 $\boldsymbol{U}^{\mathrm{H}} \boldsymbol{A} \boldsymbol{A}^{\mathrm{H}} \boldsymbol{U}$ 是对角矩阵, $\boldsymbol{V}$ 满足 $\boldsymbol{V}^{\mathrm{H}} \boldsymbol{A}^{\mathrm{H}} \boldsymbol{A} \boldsymbol{V}$ 是对角矩阵。

证明 $\boldsymbol{A} \boldsymbol{A}^{\mathrm{H}}$ 是 Hermite 矩阵,故存在 $m$ 阶酉矩阵 $\boldsymbol{U}$ ,满足

$$
\boldsymbol{U}^{\mathrm{H}} \boldsymbol{A} \boldsymbol{A}^{\mathrm{H}} \boldsymbol{U}=\left[\begin{array}{cc}
\boldsymbol{\Delta} \boldsymbol{\Delta}^{\mathrm{H}} & 0 \\
0 & 0
\end{array}\right]
$$

令 $\boldsymbol{U}=\left(\boldsymbol{U}_{1}, \boldsymbol{U}_{2}\right)$ ,其中 $\boldsymbol{U}_{1}$ 是 $m \times r$ 矩阵, $\boldsymbol{U}_{2}$ 是 $m \times(m-r)$ 矩阵,则

$$
\left[\begin{array}{l}
\boldsymbol{U}_{1}^{\mathrm{H}} \\
\boldsymbol{U}_{2}^{\mathrm{H}}
\end{array}\right] \boldsymbol{A} \boldsymbol{A}^{\mathrm{H}}\left[\begin{array}{ll}
\boldsymbol{U}_{1} & \boldsymbol{U}_{2}
\end{array}\right]=\left[\begin{array}{cc}
\Delta \boldsymbol{\Delta}^{\mathrm{H}} & 0 \\
0 & 0
\end{array}\right]
$$

比较上式两端得

$$
\begin{array}{ll}
\boldsymbol{U}_{1}^{\mathrm{H}} \boldsymbol{A} \boldsymbol{A}^{\mathrm{H}} \boldsymbol{U}_{1}=\boldsymbol{\Delta} \boldsymbol{\Delta}^{\mathrm{H}} & \text { (1) } \quad \boldsymbol{U}_{1}^{\mathrm{H}} \boldsymbol{A} \boldsymbol{A}^{\mathrm{H}} \boldsymbol{U}_{2}=0 \\
\boldsymbol{U}_{2}^{\mathrm{H}} \boldsymbol{A} \boldsymbol{A}^{\mathrm{H}} \boldsymbol{U}_{1}=0 & \text { (3) } \quad \boldsymbol{U}_{2}^{\mathrm{H}} \boldsymbol{A} \boldsymbol{A}^{\mathrm{H}} \boldsymbol{U}_{2}=0
\end{array}
$$

令 $V_{1}=A^{\mathrm{H}} U_{1} A^{-H}$ ,则

$$
\boldsymbol{V}_{1}^{\mathrm{H}} \boldsymbol{V}_{1}=\boldsymbol{\Delta}^{-1} \boldsymbol{U}_{1}^{\mathrm{H}} \boldsymbol{A} \boldsymbol{A}^{\mathrm{H}} \boldsymbol{U}_{1} \boldsymbol{\Delta}^{-\boldsymbol{H}}=\boldsymbol{E}_{r}
$$

所以 $\boldsymbol{V}_{1}$ 是 $n \times r$ 的次酉矩阵, $\boldsymbol{V}_{1} \in U_{r}^{n \times r}$ 。于是存在 $\boldsymbol{V}_{2}$ ,使得 $\boldsymbol{V}=\left(\boldsymbol{V}_{1}, \boldsymbol{V}_{2}\right)$ 为 $n$ 阶酉矩阵。故

$$
\boldsymbol{U}^{\mathrm{H}} \boldsymbol{A} \boldsymbol{V}=\left[\begin{array}{c}
\boldsymbol{U}_{1}^{\mathrm{H}} \\
\boldsymbol{U}_{2}^{\mathrm{H}}
\end{array}\right] \boldsymbol{A}\left(\boldsymbol{V}_{1}, \boldsymbol{V}_{2}\right)=\left[\begin{array}{ll}
\boldsymbol{U}_{1}^{\mathrm{H}} \boldsymbol{A} \boldsymbol{V}_{1} & \boldsymbol{U}_{1}^{\mathrm{H}} \boldsymbol{A} \boldsymbol{V}_{2} \\
\boldsymbol{U}_{2}^{\mathrm{H}} \boldsymbol{A} \boldsymbol{V}_{1} & \boldsymbol{U}_{2}^{\mathrm{H}} \boldsymbol{A} \boldsymbol{V}_{2}
\end{array}\right]
$$

而 $\boldsymbol{U}_{1}^{\mathrm{H}} \boldsymbol{A} \boldsymbol{V}_{1}=\boldsymbol{U}_{1}^{\mathrm{H}} \boldsymbol{A} \boldsymbol{A}^{\mathrm{H}} \boldsymbol{U}_{1} \boldsymbol{\Delta}^{-\mathrm{H}}=\boldsymbol{\Delta} \boldsymbol{\Delta}^{\mathrm{H}} \boldsymbol{\Delta}^{-\mathrm{H}}=\boldsymbol{\Delta}$ ,又因 $0=\boldsymbol{V}_{1}^{\mathrm{H}} \boldsymbol{V}_{2}=\boldsymbol{\Delta}^{-1} \boldsymbol{U}_{1}^{\mathrm{H}} \boldsymbol{A} \boldsymbol{V}_{2}$ ,故 $\boldsymbol{U}_{1}^{\mathrm{H}} \boldsymbol{A} \boldsymbol{V}_{2}=0$ ,由式(4)得 $\boldsymbol{U}_{2}^{\mathrm{H}} \boldsymbol{A}=0$ ,故 $\boldsymbol{U}_{2}^{\mathrm{H}} \boldsymbol{A} \boldsymbol{V}_{1}=0, \boldsymbol{U}_{2}^{\mathrm{H}} \boldsymbol{A} \boldsymbol{V}_{2}=0$ ,最后可得

$$
U^{\mathrm{H}} A V=\left[\begin{array}{ll}
\Delta & 0 \\
0 & 0
\end{array}\right]
$$

定理 4.3.4 若 $\boldsymbol{A} \in C_{r}^{m \times n}, \delta_{1} \geqslant \delta_{2} \geqslant \cdots \geqslant \delta_{r}$ 是 $\boldsymbol{A}$ 的正奇异值,则总有次西矩阵 $U_{r} \in U_{r}^{m \times r}, V_{1} \in U_{r}^{n \times r}$ 满足


\begin{equation*}
\boldsymbol{A}=\boldsymbol{U}_{r} \boldsymbol{\Delta} \boldsymbol{V}_{r}^{\mathrm{H}} \tag{4.3.3}
\end{equation*}


其中 $\boldsymbol{\Delta}=\operatorname{diag}\left(\delta_{1}, \delta_{2}, \cdots, \delta_{r}\right)$ .\\
证明 由定理 4.3.3 知

$$
A=U D V^{\mathrm{H}}
$$

若设 $U=\left(u_{1}, u_{2}, \cdots, u_{m}\right), V=\left(v_{1}, v_{2}, \cdots, v_{n}\right)$ ,则

$$
\begin{aligned}
\boldsymbol{A} & =\left(\boldsymbol{u}_{1}, \boldsymbol{u}_{2}, \cdots, \boldsymbol{u}_{m}\right)\left[\begin{array}{lllll}
\delta_{1} & & & & \\
& \delta_{2} & & & \\
& & \ddots & & \\
& & & \delta_{r} & \\
& & & & \ddots \\
& & & & 0
\end{array}\right]\left[\begin{array}{c}
\boldsymbol{v}_{1}^{\mathrm{H}} \\
\boldsymbol{v}_{2}^{\mathrm{H}} \\
\vdots \\
\boldsymbol{v}_{n}^{\mathrm{H}}
\end{array}\right] \\
& =\delta_{1} \boldsymbol{u}_{1} \boldsymbol{v}_{1}^{\mathrm{H}}+\delta_{2} \boldsymbol{u}_{2} \boldsymbol{v}_{2}^{\mathrm{H}}+\cdots+\delta_{r} \boldsymbol{u}_{r} \boldsymbol{v}_{r}^{\mathrm{H}} \\
& =\left(\boldsymbol{u}_{1}, \boldsymbol{u}_{2}, \cdots, \boldsymbol{u}_{r}\right)\left[\begin{array}{llll}
\delta_{1} & & & \\
& \delta_{2} & & \\
& & \ddots & \\
& & & \delta_{r}
\end{array}\right]\left[\begin{array}{c}
\boldsymbol{v}_{1}^{\mathrm{H}} \\
\boldsymbol{v}_{2}^{\mathrm{H}} \\
\vdots \\
\boldsymbol{v}_{r}^{\mathrm{H}}
\end{array}\right] \\
& =\boldsymbol{U}_{r} \Delta \boldsymbol{V}_{r}^{\mathrm{H}}
\end{aligned}
$$

注 1 由定理4.3.3证明过程中可以看出,虽然西矩阵 $\boldsymbol{U}$ 的列向量是 $\boldsymbol{A A}^{\mathrm{H}}$ 的特征向量,西矩阵 $\boldsymbol{V}$ 的列向量是 $\boldsymbol{A}^{\mathrm{H}} \boldsymbol{A}$ 的特征向量,但绝对不是任取 $n$ 个 $\boldsymbol{A}^{\mathrm{H}} \boldsymbol{A}$ 的两两正交单位长的特征向量都可作为 $\boldsymbol{V}$ 的列向量,而必须与 $\boldsymbol{U}$ 的列向量所"匹配",它们之间的匹配关系由关系式 $\boldsymbol{V}_{1}=\boldsymbol{A}^{\mathrm{H}} \boldsymbol{U}_{1} \boldsymbol{\Delta}^{-\mathrm{H}}$ 保证。这是在求 $\boldsymbol{A}$ 的奇异值分解时,不能忽视的。

注2 定理4.3.3中的次西矩阵 $\boldsymbol{U}_{1}$ 和次西矩阵 $\boldsymbol{V}_{1}$ 的列向量分别是 $\boldsymbol{A} \boldsymbol{A}^{\mathrm{H}}$ 与 $\boldsymbol{A}^{\mathrm{H}} \boldsymbol{A}$ 非零特征值所对应的特征向量.且 $\boldsymbol{V}_{1}=\boldsymbol{A}^{\mathrm{H}} \boldsymbol{U}_{1} \boldsymbol{A}^{-\mathrm{H}}$ .

注3 由定理4.3.3的证明可以看出,定理4.3.3中的次酉矩阵 $\boldsymbol{U}_{2}$ 与 $\boldsymbol{V}_{2}$ 中的列向量分别是 $\boldsymbol{A} \boldsymbol{A}^{\mathrm{H}}$ 与 $\boldsymbol{A}^{\mathrm{H}} \boldsymbol{A}$ 零特征值所对应的特征向量。

熟读上述三个注,对于阅读下述例题与求矩阵的奇异值分解会有好处的。\\
例 4.3.2 已知 $\boldsymbol{A}=\left[\begin{array}{ll}1 & 1 \\ 0 & 0 \\ 1 & 1\end{array}\right]$ .求 $\boldsymbol{A}$ 的奇异值分解.\\
解 $\boldsymbol{A} \boldsymbol{A}^{\mathrm{H}}=\left[\begin{array}{lll}2 & 0 & 2 \\ 0 & 0 & 0 \\ 2 & 0 & 2\end{array}\right],\left|\lambda \boldsymbol{E}-\boldsymbol{A} \boldsymbol{A}^{\mathrm{H}}\right|=\lambda^{2}(\lambda-4)$ ,所以 $\boldsymbol{A} \boldsymbol{A}^{\mathrm{H}}$ 的特征值 $\lambda_{1}=4$ , $\lambda_{2}=\lambda_{3}=0, \boldsymbol{A}$ 的奇异值 $\alpha=2, \Delta=2$ .\\
$\boldsymbol{A} \boldsymbol{A}^{\mathrm{H}}$ 的特征值 $\lambda_{1}=4$ 的单位特征向量 $\boldsymbol{u}_{1}=\left(\frac{1}{\sqrt{2}}, 0, \frac{1}{\sqrt{2}}\right)^{\mathrm{T}} \boldsymbol{U}_{1}=\boldsymbol{u}_{1}=\left(\frac{1}{\sqrt{2}}, 0, \frac{1}{\sqrt{2}}\right)^{\mathrm{T}}$ ,因此

$$
\boldsymbol{V}_{\mathrm{I}}=\boldsymbol{A}^{\mathrm{H}} \boldsymbol{U}_{1} \boldsymbol{\Delta}^{-\mathrm{H}}=\left[\begin{array}{lll}
1 & 0 & 1 \\
1 & 0 & 1
\end{array}\right]\left[\begin{array}{l}
\frac{1}{\sqrt{2}} \\
0 \\
\frac{1}{\sqrt{2}}
\end{array}\right] \cdot \frac{1}{2}=\left[\begin{array}{l}
\frac{1}{\sqrt{2}} \\
\frac{1}{\sqrt{2}}
\end{array}\right]
$$

不难验证

$$
\boldsymbol{A}=\boldsymbol{U}_{1} \boldsymbol{\Delta} \boldsymbol{V}_{1}^{\mathrm{H}}=\left[\begin{array}{c}
\frac{1}{\sqrt{2}} \\
0 \\
\frac{1}{\sqrt{2}}
\end{array}\right] 2\left[\begin{array}{c}
\frac{1}{\sqrt{2}} \\
\frac{1}{\sqrt{2}}
\end{array}\right]^{\mathbf{H}}
$$

这是定理4.3.4 表达形式.下面介绍定理4.3.3表述的形式.\\
又 $\boldsymbol{A} \boldsymbol{A}^{\mathrm{H}}$ 的零特征值所对应的次西矩阵

$$
\boldsymbol{U}_{2}=\left[\begin{array}{cc}
-\frac{1}{\sqrt{2}} & 0 \\
0 & 1 \\
\frac{1}{\sqrt{2}} & 0
\end{array}\right]
$$

$\boldsymbol{A}^{\mathrm{H}} \boldsymbol{A}$ 的零特征值所对应的次西矩阵

$$
V_{2}=\left[\begin{array}{c}
-\frac{1}{\sqrt{2}} \\
\frac{1}{\sqrt{2}}
\end{array}\right]
$$

于是 $\boldsymbol{A} \boldsymbol{A}^{\mathrm{H}}$ 的西矩阵 $\boldsymbol{U}$ 与 $\boldsymbol{A}^{\mathrm{H}} \boldsymbol{A}$ 的西矩阵 $\boldsymbol{V}$ 分别为

$$
U=\left[\begin{array}{ccc}
\frac{1}{\sqrt{2}} & -\frac{1}{\sqrt{2}} & 0 \\
0 & 0 & 1 \\
\frac{1}{\sqrt{2}} & \frac{1}{\sqrt{2}} & 0
\end{array}\right], V=\left[\begin{array}{cc}
\frac{1}{\sqrt{2}} & -\frac{1}{\sqrt{2}} \\
\frac{1}{\sqrt{2}} & \frac{1}{\sqrt{2}}
\end{array}\right]
$$

且

$$
\boldsymbol{D}=\left[\begin{array}{ll}
\boldsymbol{\Delta} & \\
& 0
\end{array}\right]=\left[\begin{array}{ll}
2 & 0 \\
0 & 0 \\
0 & 0
\end{array}\right]
$$

不难验证

$$
\boldsymbol{A}=\boldsymbol{U} \boldsymbol{D} \boldsymbol{V}^{\mathrm{H}}
$$

例4.3.3 求矩阵

$$
\boldsymbol{A}=\left[\begin{array}{cc}
2 & 0 \\
0 & -i \\
0 & 0
\end{array}\right]
$$

的奇异值分解表达式。\\
解 $\boldsymbol{A} \boldsymbol{A}^{\mathrm{H}}=\left[\begin{array}{lll}4 & 0 & 0 \\ 0 & 1 & 0 \\ 0 & 0 & 0\end{array}\right],\left|\lambda \boldsymbol{E}-\boldsymbol{A} \boldsymbol{A}^{\mathrm{H}}\right|=\lambda(\lambda-1)(\lambda-4), \boldsymbol{A} \boldsymbol{A}^{\mathrm{H}}$ 的特征值 $\lambda_{1}= 4, \lambda_{2}=1, \lambda_{3}=0$\\
所以 $\boldsymbol{A}$ 的奇异值 $\alpha_{1}=2, \alpha_{2}=1, \Delta=\left[\begin{array}{ll}2 & \\ & 1\end{array}\right]$\\
$\boldsymbol{A} \boldsymbol{A}^{\mathrm{H}}$ 的特征值为 4 的单位特征向量

$$
\boldsymbol{u}_{1}=\left(\begin{array}{lll}
1 & 0 & 0
\end{array}\right)^{\mathrm{T}}
$$

$\boldsymbol{A} \boldsymbol{A}^{\mathrm{H}}$ 的特征值为 1 的单位特征向量

$$
\boldsymbol{u}_{2}=\left(\begin{array}{lll}
0 & 1 & 0
\end{array}\right)^{\mathrm{T}}
$$

于是

$$
\boldsymbol{U}_{1}=\left(\begin{array}{ll}
\boldsymbol{u}_{1} & \boldsymbol{u}_{2}
\end{array}\right)=\left[\begin{array}{ll}
1 & 0 \\
0 & 1 \\
0 & 0
\end{array}\right]
$$

因此

$$
\boldsymbol{V}_{1}=\boldsymbol{A}^{\mathbf{H}} \boldsymbol{U}_{1} \boldsymbol{\Delta}^{-\mathbf{H}}=\left[\begin{array}{ll}
1 & 0 \\
0 & i
\end{array}\right]
$$

所以

$$
A=U_{1} \Delta V_{1}^{\mathbf{H}}=\left[\begin{array}{ll}
1 & 0 \\
0 & 1 \\
0 & 0
\end{array}\right]\left[\begin{array}{ll}
2 & \\
& 1
\end{array}\right]\left[\begin{array}{ll}
1 & 0 \\
0 & i
\end{array}\right]^{\mathrm{H}}
$$

若要写成定理 4.3.3 所形式还得计算 $\boldsymbol{U}, \boldsymbol{V}$ .\\
$\boldsymbol{A} \boldsymbol{A}^{\mathrm{H}}$ 特征值为 0 的单位特征向量

$$
u_{3}=\left[\begin{array}{l}
0 \\
0 \\
1
\end{array}\right]=U_{2},
$$

故

$$
U=\left(\begin{array}{ll}
U_{1} & U_{2}
\end{array}\right)=\left[\begin{array}{lll}
1 & 0 & 0 \\
0 & 1 & 0 \\
0 & 0 & 1
\end{array}\right] \quad V=V_{1}=\left[\begin{array}{ll}
1 & 0 \\
0 & i
\end{array}\right]
$$

所以

$$
\boldsymbol{A}=\boldsymbol{U} \boldsymbol{D} \boldsymbol{V}=\left[\begin{array}{lll}
1 & 0 & 0 \\
0 & 1 & 0 \\
0 & 0 & 1
\end{array}\right]\left[\begin{array}{ll}
2 & 0 \\
0 & 1 \\
0 & 0
\end{array}\right]\left[\begin{array}{ll}
1 & 0 \\
0 & i
\end{array}\right]^{\mathbf{H}}
$$

\section*{§4.4 矩阵的极分解}
任何一个非零的复数 $z$ 总可以写成


\begin{equation*}
z=\rho(\cos \theta+\mathrm{i} \sin \theta) \tag{4.4.1}
\end{equation*}


的形式,式中 $\rho>0$ 是 $z$ 的模(或称极径),$\theta$ 是 $z$ 的幅角.把复数 $z$ 写成这样的形式是唯一的,并称为复数的极分解。

若把数看成是一阶矩阵,则 $\rho$ 是一阶正定 Hermite 矩阵, $\cos \theta+\mathrm{i} \sin \theta$ 是一阶酉矩阵,称式(4.4.1)是一阶复矩阵的极分解。

定理 4.4.1 设 $\boldsymbol{A} \in C_{n}^{n \times n}$ ,则存在 $\boldsymbol{U} \in U^{n \times n}$ ,与正定 Hermite 矩阵 $\boldsymbol{H}_{1}$ 与 $\boldsymbol{H}_{2}$ ,且满足


\begin{equation*}
A=H_{1} U=U H_{2} \tag{4.4.2}
\end{equation*}


且这样的分解式是唯一的。同时有,$A^{\mathrm{H}} A=H_{2}^{2}, A A^{\mathrm{H}}=H_{1}^{2}$ .\\
分解式(4.4.2)称为矩阵 $\boldsymbol{A}$ 的极分解表达式。\\
证明 因为 $\boldsymbol{A}$ 是满秩的, $\boldsymbol{A}^{\mathrm{H}} \boldsymbol{A}$ 是正定 Hermite 矩阵,故存在唯一的正定 Hermite 矩阵 $\boldsymbol{H}_{2}$ ,使得

且

$$
\begin{gathered}
\boldsymbol{A}^{\mathrm{H}} \boldsymbol{A}=\boldsymbol{H}_{2}^{2} \\
\left(\boldsymbol{A} \boldsymbol{H}_{2}^{-1}\right)^{\mathrm{H}}\left(\boldsymbol{A} \boldsymbol{H}_{2}^{-1}\right)=\boldsymbol{E}
\end{gathered}
$$

这表明 $\boldsymbol{A H}_{2}^{-1}$ 是酉矩阵,命 $\boldsymbol{A H}_{2}^{-1}=\boldsymbol{U} \in \boldsymbol{U}^{n \times n}$ ,故

$$
\boldsymbol{A}=\boldsymbol{U} \boldsymbol{H}_{2}
$$

又

$$
\begin{aligned}
\boldsymbol{A} & =\boldsymbol{U} \boldsymbol{H}_{2}=\boldsymbol{U} \boldsymbol{H}_{2} \boldsymbol{U}^{\mathrm{H}} \boldsymbol{U} \\
& =\left(\boldsymbol{U} \boldsymbol{H}_{2} \boldsymbol{U}^{\mathrm{H}}\right) \boldsymbol{U}=\boldsymbol{H}_{1} \boldsymbol{U}
\end{aligned}
$$

其中 $\boldsymbol{H}_{1}=\boldsymbol{U} \boldsymbol{H}_{2} \boldsymbol{U}^{\mathbf{H}}$ ,显然 $\boldsymbol{H}_{1}$ 是正定 Hermite 矩阵。由于 $\boldsymbol{H}_{2}$ 是唯一的,故 $\boldsymbol{U}$ 也是确定的,因此极分解是唯一的。

定理 4.4.2 设 $\boldsymbol{A} \in C^{n \times n}$ ,则存在 $\boldsymbol{U} \in U^{n \times n}$ 与半正定 Hermite 矩阵 $\boldsymbol{H}_{1}$ 与 $\boldsymbol{H}_{2}$ ,满足

$$
A=H_{1} U=U H_{2}
$$

且 $\boldsymbol{H}_{1}^{2}=\boldsymbol{A} \boldsymbol{A}^{\mathrm{H}}, \boldsymbol{H}_{2}^{2}=\boldsymbol{A}^{\mathrm{H}} \boldsymbol{A}$ 。\\
证明 根据 $\boldsymbol{A}$ 的奇异值分解知,存在西矩阵 $U_{1}, U_{2}$ 使得

$$
A=U_{1}\left[\begin{array}{llll}
\alpha_{1} & & & \\
& \alpha_{2} & & \\
& & \ddots & \\
& & & \alpha_{n}
\end{array}\right] U_{2}
$$

其中 $\alpha_{1} \geqslant \alpha_{2} \geqslant \cdots \geqslant \alpha_{r}>0$ 是 $\boldsymbol{A}$ 的 $r$ 个奇异值.因此

$$
\begin{aligned}
\boldsymbol{A} & =\left(\boldsymbol{U}_{1}\left[\begin{array}{llll}
\boldsymbol{\alpha}_{1} & & & \\
& \boldsymbol{\alpha}_{2} & & \\
& & \ddots & \\
& & & \boldsymbol{\alpha}_{n}
\end{array}\right] \boldsymbol{U}_{1}^{\mathrm{H}}\right)\left(\boldsymbol{U}_{1} \boldsymbol{U}_{2}\right) \\
& =\left(\boldsymbol{U}_{1} \boldsymbol{U}_{2}\right)\left(\boldsymbol{U}_{2}^{\mathrm{H}}\left[\begin{array}{llll}
\boldsymbol{\alpha}_{1} & & & \\
& \boldsymbol{\alpha}_{2} & & \\
& & \ddots & \\
& & & \boldsymbol{\alpha}_{n}
\end{array}\right] \boldsymbol{U}_{2}\right)
\end{aligned}
$$

若命

$$
\boldsymbol{H}_{1}=\boldsymbol{U}_{1}\left[\begin{array}{llll}
\alpha_{1} & & & \\
& \alpha_{2} & & \\
& & \ddots & \\
& & & \alpha_{n}
\end{array}\right] \boldsymbol{U}_{1}^{\mathrm{H}}, \boldsymbol{H}_{2}=\boldsymbol{U}_{2}^{\mathrm{H}}\left[\begin{array}{llll}
\alpha_{1} & & & \\
& \alpha_{2} & & \\
& & \ddots & \\
& & & \alpha_{n}
\end{array}\right] \boldsymbol{U}_{2}
$$

则

$$
\begin{gathered}
U=U_{1} U_{2} \\
A=H_{1} U=U H_{2}
\end{gathered}
$$

其中 $\boldsymbol{H}_{1}, \boldsymbol{H}_{2}$ 是半正定 Hermite 矩阵, $\boldsymbol{U}$ 是西矩阵。\\
定理4.4.3 设 $\boldsymbol{A} \in C^{n \times n}$ ,则 $\boldsymbol{A}$ 是正规矩阵的充要条件是

$$
A=H U=U H
$$

其中 $\boldsymbol{U}$ 为西矩阵, $\boldsymbol{H}$ 为半正定 Hermite 矩阵,且 $\boldsymbol{H}^{2}=\boldsymbol{A} \boldsymbol{A}^{\mathrm{H}}$ .\\
证明 必要性 根据定理4.4.2知,$A \boldsymbol{A}^{\mathrm{H}}=\boldsymbol{H}_{1}^{2}, \boldsymbol{A}^{\mathrm{H}} \boldsymbol{A}=\boldsymbol{H}_{2}^{2}$ .因为 $\boldsymbol{A} \boldsymbol{A}^{\mathrm{H}}=\boldsymbol{A}^{\mathrm{H}} \boldsymbol{A}$ ,故 $\boldsymbol{H}_{1}^{2}=\boldsymbol{H}_{2}^{2}, \boldsymbol{H}_{1}=\boldsymbol{H}_{2}$ 。

充分性 设 $\boldsymbol{A}=\boldsymbol{H} \boldsymbol{U}=\boldsymbol{U} \boldsymbol{H}, \boldsymbol{A}^{\mathrm{H}}=\boldsymbol{U}^{\mathrm{H}} \boldsymbol{H}, \boldsymbol{A} \boldsymbol{A}^{\mathrm{H}}=\boldsymbol{H}^{2}$ ,且 $\boldsymbol{A}^{\mathrm{H}} \boldsymbol{A}=\boldsymbol{U}^{\mathrm{H}} \boldsymbol{H} \boldsymbol{U} \boldsymbol{H}=\boldsymbol{U}^{\mathrm{H}} \boldsymbol{U} \boldsymbol{H}^{2}=$\\
$\boldsymbol{H}^{\mathbf{2}}$ .故 $\boldsymbol{A} \boldsymbol{A}^{\mathrm{H}}=\boldsymbol{A}^{\mathrm{H}} \boldsymbol{A}$ .

\section*{§4.5 矩阵的谱分解}
\section*{一、正规矩阵的谱分解}
首先介绍正规矩阵的谱分解,然后再介绍单纯矩阵的谱分解。\\
设 $\boldsymbol{A}$ 为正规矩阵,那么存在 $\boldsymbol{U} \in U^{n \times n}$ ,满足

$$
\boldsymbol{A}=\boldsymbol{U} \operatorname{diag}\left(\lambda_{1}, \lambda_{2}, \cdots, \lambda_{n}\right) \boldsymbol{U}^{\mathrm{H}}
$$

若命 $\boldsymbol{U}=\left(\boldsymbol{\alpha}_{1}, \boldsymbol{\alpha}_{2}, \cdots, \boldsymbol{\alpha}_{n}\right)$ ,则


\begin{align*}
\boldsymbol{A} & =\left(\boldsymbol{\alpha}_{1}, \boldsymbol{\alpha}_{2}, \cdots, \boldsymbol{\alpha}_{n}\right) \operatorname{diag}\left(\lambda_{1}, \lambda_{2}, \cdots, \lambda_{n}\right)\left[\begin{array}{c}
\boldsymbol{\alpha}_{1}^{\mathrm{H}} \\
\boldsymbol{\alpha}_{2}^{\mathrm{H}} \\
\vdots \\
\boldsymbol{\alpha}_{n}^{\mathrm{H}}
\end{array}\right] \\
& =\lambda_{1} \boldsymbol{\alpha}_{1} \boldsymbol{\alpha}_{1}^{\mathrm{H}}+\lambda_{2} \boldsymbol{\alpha}_{2} \boldsymbol{\alpha}_{2}^{\mathrm{H}}+\cdots+\lambda_{n} \boldsymbol{\alpha}_{n} \boldsymbol{\alpha}_{n}^{\mathrm{H}} \tag{4.5.1}
\end{align*}


其中 $\boldsymbol{\alpha}_{i}$ 是矩阵 $\boldsymbol{A}$ 的特征值为 $\lambda_{i}$ 所对应的单位特征向量, $\boldsymbol{\alpha}_{i} \boldsymbol{\alpha}_{i}^{\mathrm{H}}$ 是 $n$ 阶矩阵。式 (4.5.1)称为矩阵 $\boldsymbol{A}$ 的谱分解。由于 $\boldsymbol{A}$ 的特征值 $\lambda_{i}$ 有重根,可以把谱分解式再简化。

设正规矩阵 $\boldsymbol{A}$ 有 $r$ 个相异特征值 $\lambda_{1}, \lambda_{2}, \cdots, \lambda_{r}$ .特征值 $\lambda_{i}$ 的代数重复度为 $n_{i}, \lambda_{i}$ 所对应的 $n_{i}$ 个两两正交单位的特征向量记为 $\boldsymbol{\alpha}_{i 1}, \boldsymbol{\alpha}_{i 2}, \cdots, \alpha_{i n_{i}}$ ,则 $\boldsymbol{A}$ 的谱分解式可写成


\begin{equation*}
\boldsymbol{A}=\sum_{j=1}^{r} \lambda_{j} \sum_{i=1}^{n_{j}} \boldsymbol{\alpha}_{j i} \boldsymbol{\alpha}_{j i}^{\mathrm{H}}=\sum_{j=1}^{r} \lambda_{j} \boldsymbol{G}_{j} \tag{4.5.2}
\end{equation*}


其中 $\boldsymbol{G}_{j}=\sum_{i=1}^{n_{j}} \boldsymbol{\alpha}_{j i} \boldsymbol{\alpha}_{j i}^{\mathrm{H}}$ 。显然, $\boldsymbol{G}_{j}^{\mathrm{H}}=\boldsymbol{G}_{j}=\boldsymbol{G}_{j}^{2}, \boldsymbol{G}_{j} \boldsymbol{G}_{k}=0(j \neq k)$ 。\\
定理4.5.1 设 $n$ 阶矩阵 $\boldsymbol{A}$ 有 $r$ 个相异特征值 $\lambda_{1}, \lambda_{2}, \cdots, \lambda_{r}, \lambda_{i}$ 的代数重复度为 $n_{i}$ ,则 $\boldsymbol{A}$ 为正规矩阵的充要条件是存在 $r$ 个 $n$ 阶矩阵 $\boldsymbol{G}_{1}, \boldsymbol{G}_{2}, \cdots, \boldsymbol{G}_{r}$ ,满足\\
(1) $\boldsymbol{A}=\sum_{j=1}^{r} \lambda_{j} \boldsymbol{G}_{j} ;$\\
(2) $\boldsymbol{G}_{j}=\boldsymbol{G}_{j}^{2}=\boldsymbol{G}_{j}^{\mathrm{H}}$\\
(3) $\boldsymbol{G}_{j} \boldsymbol{G}_{k}=0 \quad(j \neq k)$ ;\\
(4)$\sum_{j=1}^{r} \boldsymbol{G}_{j}=\boldsymbol{E}$\\
(5)满足上述性质的 $\boldsymbol{G}_{j}$ 是唯一的;\\
(6) $\operatorname{rank} G_{j}=n_{j}$ .\\
常称 $\boldsymbol{G}_{j}$ 为正交投影矩阵。\\
证明 必要性(1),(2),(3)请读者自证。\\
(4)命 $\boldsymbol{U}_{j}=\left(\boldsymbol{\alpha}_{j 1}, \boldsymbol{\alpha}_{j 2}, \cdots, \boldsymbol{\alpha}_{j n_{j}}\right)$ ,则

$$
\boldsymbol{G}_{j}=\boldsymbol{U}_{j} \boldsymbol{U}_{j}^{\mathrm{H}}
$$

于是

$$
\begin{gathered}
\boldsymbol{G}_{1}+\boldsymbol{G}_{2}+\cdots+\boldsymbol{G}_{r}=\boldsymbol{U}_{1} \boldsymbol{U}_{1}^{\mathrm{H}}+\boldsymbol{U}_{2} \boldsymbol{U}_{2}^{\mathrm{H}}+\cdots+\boldsymbol{U}_{r} \boldsymbol{U}_{r}^{\mathrm{H}} \\
=\left(\boldsymbol{U}_{1} \boldsymbol{U}_{2} \cdots \boldsymbol{U}_{r}\right)\left(\begin{array}{c}
\boldsymbol{U}_{1}^{\mathrm{H}} \\
\boldsymbol{U}_{2}^{\mathrm{H}} \\
\vdots \\
\boldsymbol{U}_{r}^{\mathrm{H}}
\end{array}\right) \\
=\boldsymbol{U} \boldsymbol{U}^{\mathrm{H}}=\boldsymbol{E}
\end{gathered}
$$

(5)现证 $\boldsymbol{G}_{j}$ 是唯一的,不难证明

$$
\boldsymbol{G}_{j} \boldsymbol{A}=\lambda_{j} \boldsymbol{G}_{j}=\boldsymbol{A} \boldsymbol{G}_{j}
$$

若又有 $\tilde{\boldsymbol{G}}_{j}$ 满足性质(1)$\sim(4)$ ,故 $\tilde{\boldsymbol{G}}_{j}$ 也有

$$
\tilde{\boldsymbol{G}}_{j} \boldsymbol{A}=\lambda_{j} \tilde{\boldsymbol{G}}_{j}=\boldsymbol{A} \tilde{\boldsymbol{G}}_{j}
$$

因此

$$
\begin{aligned}
\left(\lambda_{i}-\lambda_{j}\right) \boldsymbol{G}_{j} \tilde{\boldsymbol{G}}_{i} & =\lambda_{i} \boldsymbol{G}_{j} \tilde{\boldsymbol{G}}_{i}-\lambda_{j} \boldsymbol{G}_{j} \tilde{\boldsymbol{G}}_{i} \\
& =\boldsymbol{G}_{j}\left(\lambda_{i} \tilde{\boldsymbol{G}}_{i}\right)-\left(\lambda_{j} \boldsymbol{G}_{j}\right) \tilde{\boldsymbol{G}}_{i} \\
& =\boldsymbol{G}_{j}\left(A \tilde{\boldsymbol{G}}_{i}\right)-\left(\boldsymbol{G}_{j} A\right) \tilde{\boldsymbol{G}}_{i}=0
\end{aligned}
$$

由于 $\lambda_{\mathrm{i}}-\lambda_{\mathrm{j}} \neq 0$ ,故 $\boldsymbol{G}_{j} \tilde{\boldsymbol{G}}_{i}=0$ .于是

$$
\begin{aligned}
\boldsymbol{G}_{j} & =\boldsymbol{G}_{j} \boldsymbol{E}=\boldsymbol{G}_{j}\left(\sum_{i=1}^{r} \tilde{\boldsymbol{G}}_{i}\right)=\boldsymbol{G}_{j} \tilde{\boldsymbol{G}}_{j} \\
& =\left(\sum_{i=1}^{r} \boldsymbol{G}_{i}\right) \tilde{\boldsymbol{G}}_{j}=\boldsymbol{E} \tilde{\boldsymbol{G}}_{j}=\tilde{\boldsymbol{G}}_{j}
\end{aligned}
$$

(6)因为 $\boldsymbol{G}_{j}=\boldsymbol{U}_{j} \boldsymbol{U}_{j}^{\mathrm{H}}$ ,故根据 §4.3节引理 4.3.1 可知

$$
\operatorname{rank} G_{j}=\operatorname{rank} U_{j}=n_{j}
$$

充分性 易得

$$
\begin{aligned}
\boldsymbol{A} \boldsymbol{A}^{\mathrm{H}} & =\left(\sum_{i=1}^{r} \lambda_{i} \boldsymbol{G}_{i}\right)\left(\sum_{i=1}^{r} \bar{\lambda}_{i} \boldsymbol{G}_{i}^{\mathrm{H}}\right) \\
& =\sum_{i=1}^{r} \lambda_{i} \bar{\lambda}_{i} \boldsymbol{G}_{i} \boldsymbol{G}_{i}^{\mathrm{H}}=\sum_{i=1}^{r} \lambda_{i} \bar{\lambda}_{i} \boldsymbol{G}_{i} \\
\boldsymbol{A}^{\mathrm{H}} \boldsymbol{A} & =\sum_{i=1}^{r} \lambda_{i} \bar{\lambda}_{i} \boldsymbol{G}_{i} \\
\boldsymbol{A} \boldsymbol{A}^{\mathrm{H}} & =\boldsymbol{A}^{\mathrm{H}} \boldsymbol{A}
\end{aligned}
$$

且\\
故\\
例 4.5 .1 已知

$$
A=\left[\begin{array}{rrr}
-2 \mathrm{i} & 4 & -2 \\
-4 & -2 \mathrm{i} & -2 \mathrm{i} \\
2 & -2 \mathrm{i} & -5 \mathrm{i}
\end{array}\right]
$$

验证 $\boldsymbol{A}$ 是正规矩阵,写出 $\boldsymbol{A}$ 的谱分解表达式。\\
解 由于

$$
\boldsymbol{A}^{\mathrm{H}}=\left[\begin{array}{rrc}
2 \mathrm{i} & -4 & 2 \\
4 & 2 \mathrm{i} & 2 \mathrm{i} \\
2 & 2 \mathrm{i} & 5 \mathrm{i}
\end{array}\right]=-\left[\begin{array}{rrr}
-2 \mathrm{i} & 4 & -2 \\
-4 & -2 \mathrm{i} & -2 \mathrm{i} \\
2 & -2 \mathrm{i} & -5 \mathrm{i}
\end{array}\right]=-\boldsymbol{A}
$$

所以 $\boldsymbol{A}$ 是反 Hermite 矩阵.

$$
\begin{aligned}
|\lambda \boldsymbol{E}-\boldsymbol{A}| & =\left|\begin{array}{ccc}
\lambda+2 \mathrm{i} & -4 & 2 \\
4 & \lambda+2 \mathrm{i} & 2 \mathrm{i} \\
-2 & 2 \mathrm{i} & \lambda+5 \mathrm{i}
\end{array}\right| \\
& =\left|\begin{array}{ccc}
\lambda+2 \mathrm{i} & -4 & 2 \\
0 & \lambda+6 \mathrm{i} & 2(\lambda+6 \mathrm{i}) \\
-2 & 2 \mathrm{i} & \lambda+5 \mathrm{i}
\end{array}\right| \\
& =(\lambda+6 \mathrm{i})\left|\begin{array}{ccc}
\lambda+2 \mathrm{i} & -4 & 2 \\
0 & 1 & 2 \\
-2 & 2 \mathrm{i} & \lambda+5 \mathrm{i}
\end{array}\right| \\
& =(\lambda+6 \mathrm{i})\left|\begin{array}{ccc}
\lambda+2 \mathrm{i} & -4 & 10 \\
0 & 1 & 0 \\
-2 & 2 \mathrm{i} & \lambda+\mathrm{i}
\end{array}\right| \\
& =(\lambda+6 \mathrm{i})\left(\lambda^{2}+3 \lambda \mathrm{i}+18\right)=(\lambda+6 \mathrm{i})^{2}(\lambda-3 \mathrm{i})
\end{aligned}
$$

$\boldsymbol{A}$ 的特征值 $\lambda_{1}=\lambda_{2}=-6 \mathrm{i}, \lambda_{3}=3 \mathrm{i}$ 。\\
对于 $\lambda_{1}=\lambda_{2}=-6 \mathrm{i}$ 的特征矩阵

$$
\lambda \boldsymbol{E}-\boldsymbol{A}=\left[\begin{array}{rrr}
-4 \mathrm{i} & -4 & 2 \\
4 & -4 \mathrm{i} & 2 \mathrm{i} \\
-2 & 2 \mathrm{i} & -\mathrm{i}
\end{array}\right] \rightarrow\left[\begin{array}{rrr}
-2 \mathrm{i} & -2 & 1 \\
0 & 0 & 0 \\
0 & 0 & 0
\end{array}\right]
$$

所以属于 $\lambda_{1}=\lambda_{2}=6 \mathrm{i}$ 的正交单位特征向量

$$
\alpha_{1}=\left(0, \frac{1}{\sqrt{5}}, \frac{2}{\sqrt{5}}\right)^{\mathrm{T}}, \quad \alpha_{2}=\left(\frac{5 \mathrm{i}}{3 \sqrt{5}}, \frac{4}{3 \sqrt{5}},-\frac{2}{3 \sqrt{5}}\right)^{\mathrm{T}}
$$

对于 $\lambda_{3}=3 \mathrm{i}$ 的特征矩阵

$$
\lambda \boldsymbol{E}-\boldsymbol{A}=\left[\begin{array}{rrr}
5 \mathrm{i} & -4 & 2 \\
4 & 5 \mathrm{i} & 2 \mathrm{i} \\
-2 & 2 \mathrm{i} & 8 \mathrm{i}
\end{array}\right] \rightarrow\left[\begin{array}{rrr}
1 & 0 & -2 \mathrm{i} \\
0 & 1 & 2 \\
0 & 0 & 0
\end{array}\right]
$$

所以属于 $\lambda_{3}=3 \mathrm{i}$ 的单位特征向量

$$
\boldsymbol{\alpha}_{3}=\left(\frac{2}{3} \mathrm{i},-\frac{2}{3}, \frac{1}{3}\right)^{\mathrm{T}}
$$

因此 $\boldsymbol{A}$ 的正交投影矩阵为

$$
\boldsymbol{G}_{1}=\boldsymbol{\alpha}_{1} \boldsymbol{\alpha}_{1}^{\mathrm{H}}+\boldsymbol{\alpha}_{2} \boldsymbol{\alpha}_{2}^{\mathrm{H}}
$$

$$
\begin{aligned}
& =\left[\begin{array}{ccc}
\frac{5}{9} & \frac{4 \mathrm{i}}{9} & -\frac{2 \mathrm{i}}{9} \\
-\frac{4 \mathrm{i}}{9} & \frac{5}{9} & \frac{2}{9} \\
\frac{2 \mathrm{i}}{9} & \frac{2}{9} & \frac{8}{9}
\end{array}\right] \\
& \boldsymbol{G}_{2}=\boldsymbol{a}_{3} \boldsymbol{a}_{3}^{\mathrm{H}}=\left[\begin{array}{rrr}
\frac{4}{9} & -\frac{4 \mathrm{i}}{9} & \frac{2 \mathrm{i}}{9} \\
\frac{4 \mathrm{i}}{9} & \frac{4}{9} & -\frac{2}{9} \\
-\frac{2 \mathrm{i}}{9} & -\frac{2}{9} & \frac{1}{9}
\end{array}\right]
\end{aligned}
$$

所以 $\boldsymbol{A}$ 的谱分解表达式为

$$
A=-6 \mathrm{i} G_{1}+3 \mathrm{i} G_{2}
$$

例4.5.2 已知

$$
A=\left[\begin{array}{rrr}
\frac{1}{2} & 0 & \frac{3}{2} \mathrm{i} \\
0 & 2 & 0 \\
-\frac{3}{2} \mathrm{i} & 0 & \frac{1}{2}
\end{array}\right]
$$

验证 $\boldsymbol{A}$ 是正规矩阵,对 $\boldsymbol{A}$ 作谱分解.

解 由于

$$
A^{\mathrm{H}}=\left[\begin{array}{rrr}
\frac{1}{2} & 0 & \frac{3}{2} \mathrm{i} \\
0 & 2 & 0 \\
-\frac{3}{2} \mathrm{i} & 0 & \frac{1}{2}
\end{array}\right]=A
$$

所以 $\boldsymbol{A}$ 是 Hermite 矩阵。

$$
|\lambda \boldsymbol{E}-\boldsymbol{A}|=\left|\begin{array}{ccc}
\lambda-\frac{1}{2} & 0 & -\frac{3}{2} \mathrm{i} \\
0 & \lambda-2 & 0 \\
\frac{3}{2} \mathrm{i} & 0 & \lambda-\frac{1}{2}
\end{array}\right|=(\lambda-2)^{2}(\lambda+1)
$$

$\boldsymbol{A}$ 的特征值 $\lambda_{1}=\lambda_{2}=2, \lambda_{3}=-1$\\
$\boldsymbol{A}$ 的属于特征值 $\lambda_{1}=\lambda_{2}=2$ 的正交单位特征向量 $\boldsymbol{\alpha}_{1}=(0,1,0)^{\mathrm{T}}$ , $\boldsymbol{\alpha}_{2}=\left(\frac{\mathrm{i}}{\sqrt{2}}, 0, \frac{1}{\sqrt{2}}\right)^{\mathrm{T}}: \boldsymbol{A}$ 的属于特征值 $\lambda_{3}=-1$ 的单位特征向量 $\boldsymbol{\alpha}_{3}=\left(\frac{\mathrm{i}}{\sqrt{2}}, 0,-\frac{1}{\sqrt{2}}\right)^{\mathrm{T}}$ ,因此 $\boldsymbol{A}$ 的正交投影矩阵

$$
\boldsymbol{G}_{1}=\boldsymbol{\alpha}_{1} \boldsymbol{\alpha}_{1}^{\mathrm{H}}+\boldsymbol{\alpha}_{2} \boldsymbol{\alpha}_{2}^{\mathrm{H}}
$$

$$
\begin{aligned}
= & {\left[\begin{array}{l}
0 \\
1 \\
0
\end{array}\right]\left[\begin{array}{lll}
0 & 1 & 0
\end{array}\right]+\left[\begin{array}{c}
\frac{\mathrm{i}}{\sqrt{2}} \\
0 \\
\frac{1}{\sqrt{2}}
\end{array}\right]\left[-\frac{\mathrm{i}}{\sqrt{2}}, 0, \frac{1}{\sqrt{2}}\right] } \\
= & {\left[\begin{array}{lll}
0 & 0 & 0 \\
0 & 1 & 0 \\
0 & 0 & 0
\end{array}\right]+\left[\begin{array}{ccc}
\frac{1}{2} & 0 & \frac{\mathrm{i}}{2} \\
0 & 0 & 0 \\
-\frac{\mathrm{i}}{2} & 0 & \frac{1}{2}
\end{array}\right]=\left[\begin{array}{ccc}
\frac{1}{2} & 0 & \frac{\mathrm{i}}{2} \\
0 & 1 & 0 \\
-\frac{\mathrm{i}}{2} & 0 & \frac{1}{2}
\end{array}\right] } \\
\boldsymbol{G}_{2}=\boldsymbol{\alpha}_{3} \boldsymbol{\alpha}_{3}^{\mathrm{H}} & =\left[\begin{array}{c}
\frac{\mathrm{i}}{\sqrt{2}} \\
0 \\
-\frac{1}{\sqrt{2}}
\end{array}\right]\left[\begin{array}{ll}
-\frac{\mathrm{i}}{\sqrt{2}}, 0,-\frac{1}{\sqrt{2}}
\end{array}\right] \\
& =\left[\begin{array}{ccc}
\frac{1}{2} & 0 & -\frac{\mathrm{i}}{2} \\
0 & 0 & 0 \\
\frac{\mathrm{i}}{2} & 0 & \frac{1}{2}
\end{array}\right]
\end{aligned}
$$

所以 $\boldsymbol{A}$ 的谱分解表达式为

$$
A=2 G_{1}-G_{2}
$$

正规矩阵之所以可以谱分解关键在于正规矩阵可以西对角化,那么可以对角化但不可酉对角化矩阵的谱分解会有怎样的结果呢?这就是下述单纯矩阵的谱分解。

\section*{二、单纯矩阵的谱分解}
设 $\boldsymbol{A}$ 为 $n$ 阶单纯矩阵(即 $\boldsymbol{A}$ 可以对角化),特征值为 $\lambda_{1}, \lambda_{2}, \cdots, \lambda_{n}$ ,其对应的特征向量分别为 $\boldsymbol{\alpha}_{1}, \boldsymbol{\alpha}_{2}, \cdots, \boldsymbol{\alpha}_{n}$ .若命 $\boldsymbol{P}=\left(\boldsymbol{\alpha}_{1}, \boldsymbol{\alpha}_{2}, \cdots, \boldsymbol{\alpha}_{n}\right)$ ,则


\begin{align*}
\boldsymbol{A} & =\boldsymbol{P} \operatorname{diag}\left(\lambda_{1}, \lambda_{2}, \cdots, \lambda_{n}\right) \boldsymbol{P}^{-1} \\
& =\left(\boldsymbol{\alpha}_{1}, \boldsymbol{\alpha}_{2}, \cdots, \boldsymbol{\alpha}_{n}\right)\left[\begin{array}{llll}
\lambda_{1} & & & \\
& \lambda_{2} & & \\
& & \ddots & \\
& & & \lambda_{n}
\end{array}\right]\left[\begin{array}{c}
\boldsymbol{\beta}_{1}^{\mathrm{T}} \\
\boldsymbol{\beta}_{2}^{\mathrm{T}} \\
\vdots \\
\boldsymbol{\beta}_{n}^{\mathrm{T}}
\end{array}\right] \\
& =\lambda_{1} \boldsymbol{\alpha}_{1} \boldsymbol{\beta}_{1}^{\mathrm{T}}+\lambda_{2} \boldsymbol{\alpha}_{2} \boldsymbol{\beta}_{2}^{\mathrm{T}}+\cdots+\lambda_{n} \boldsymbol{\alpha}_{n} \boldsymbol{\beta}_{n}^{\mathrm{T}} \tag{4.5.3}
\end{align*}


其中

$$
\boldsymbol{P}^{-\mathbf{1}}=\left[\begin{array}{c}
\boldsymbol{\beta}_{1}^{\mathrm{T}} \\
\boldsymbol{\beta}_{2}^{\mathrm{T}} \\
\vdots \\
\boldsymbol{\beta}_{n}^{\mathrm{T}}
\end{array}\right]
$$

由于 $\boldsymbol{P P}^{-1}=\boldsymbol{E}$ ,故


\begin{equation*}
\boldsymbol{\alpha}_{1} \boldsymbol{\beta}_{1}^{\mathrm{T}}+\boldsymbol{\alpha}_{2} \boldsymbol{\beta}_{2}^{\mathrm{T}}+\cdots+\boldsymbol{\alpha}_{n} \boldsymbol{\beta}_{n}^{\mathrm{T}}=\boldsymbol{E} \tag{4.5.4}
\end{equation*}


又由 $\boldsymbol{P}^{-1} \boldsymbol{P}=\boldsymbol{E}$ ,故


\begin{equation*}
\boldsymbol{\beta}_{i}^{\mathrm{T}} \boldsymbol{\alpha}_{j}=\boldsymbol{\delta}_{i j} \quad(i, j=1,2, \cdots, n) \tag{4.5.5}
\end{equation*}


现在分析 $\boldsymbol{\beta}_{i}$ 与矩阵 $A$ 的关系。因为

$$
\begin{aligned}
\boldsymbol{A}^{\mathrm{T}} & =\left(\boldsymbol{P}^{-1}\right)^{\mathrm{T}} \operatorname{diag}\left(\lambda_{1}, \lambda_{2}, \cdots, \lambda_{n}\right) \boldsymbol{P}^{\mathrm{T}} \\
& =\left(\boldsymbol{P}^{\mathrm{T}}\right)^{-1} \operatorname{diag}\left(\lambda_{1}, \lambda_{2}, \cdots, \lambda_{n}\right) \boldsymbol{P}^{\mathrm{T}}
\end{aligned}
$$

所以 $\left(\boldsymbol{P}^{\mathrm{T}}\right)^{-1}$ 的列向量应该是矩阵 $\boldsymbol{A}^{\mathrm{T}}$ 的特征向量.而

$$
\left(\boldsymbol{P}^{\mathbf{T}}\right)^{-1}=\left(\boldsymbol{P}^{-1}\right)^{\mathbf{T}}=\left(\boldsymbol{\beta}_{1}, \boldsymbol{\beta}_{2}, \cdots, \boldsymbol{\beta}_{n}\right)
$$

因此 $\boldsymbol{\beta}_{i}$ 是 $\boldsymbol{A}^{\mathrm{T}}$ 的对应于特征值为 $\lambda_{i}$ 的特征向量,即

或

$$
\begin{gathered}
\boldsymbol{A}^{\mathrm{T}} \boldsymbol{\beta}_{i}=\lambda_{i} \boldsymbol{\beta}_{i} \\
\boldsymbol{\beta}_{i}^{\mathrm{T}} \boldsymbol{A}=\lambda_{i} \boldsymbol{\beta}_{i}^{\mathrm{T}} .
\end{gathered}
$$

所以我们经常称 $\boldsymbol{\beta}_{i}^{\mathbf{T}}$ 是矩阵 $\boldsymbol{A}$ 的左特征向量,称 $\boldsymbol{\alpha}_{i}$ 是矩阵 $\boldsymbol{A}$ 的右特征向量。由式 (4.5.5)可知 $\boldsymbol{A}$ 的相异特征值所对应的左、右特征向量是正交的。

在式(4.5.3)中把相同特征值的项合并。若 $\boldsymbol{A}$ 的相异特征值 $\lambda_{1}, \lambda_{2}, \cdots, \lambda_{r}, \lambda_{i}$所对应的线性无关特征向量为 $\boldsymbol{\alpha}_{i 1}, \boldsymbol{\alpha}_{i 2}, \cdots, \boldsymbol{\alpha}_{i n_{i}}, \boldsymbol{A}^{\mathrm{T}}$ 的特征值 $\lambda_{1}$ 所对应的线性无关特征向量为 $\boldsymbol{\beta}_{i 1}, \boldsymbol{\beta}_{i 2}, \cdots, \boldsymbol{\beta}_{i n_{i}}$ ,且所选取的 $\boldsymbol{\beta}_{i j}$ 满足

$$
\boldsymbol{\beta}_{i j}^{\mathrm{T}} \boldsymbol{\alpha}_{i j}=1, \quad \boldsymbol{\beta}_{i j}^{T} \boldsymbol{\alpha}_{i k}=0 \quad(j \neq k)
$$

则

$$
\begin{aligned}
\boldsymbol{A}= & \lambda_{1}\left(\boldsymbol{\alpha}_{11} \boldsymbol{\beta}_{11}^{\mathrm{T}}+\boldsymbol{\alpha}_{12} \boldsymbol{\beta}_{12}^{\mathrm{T}}+\cdots+\boldsymbol{\alpha}_{1 n_{1}} \boldsymbol{\beta}_{1 n_{1}}^{\mathrm{T}}\right)+ \\
& \lambda_{2}\left(\boldsymbol{\alpha}_{21} \boldsymbol{\beta}_{21}^{\mathrm{T}}+\boldsymbol{\alpha}_{22} \boldsymbol{\beta}_{22}^{\mathrm{T}}+\cdots+\boldsymbol{\alpha}_{2 n_{2}} \boldsymbol{\beta}_{2 n_{2}}^{\mathrm{T}}\right)+\cdots+ \\
& \lambda_{r}\left(\boldsymbol{\alpha}_{r 1} \boldsymbol{\beta}_{r 1}^{\mathrm{T}}+\boldsymbol{\alpha}_{r 2} \boldsymbol{\beta}_{r 2}^{\mathrm{T}}+\cdots+\boldsymbol{\alpha}_{r n} \boldsymbol{\beta}_{r n_{r}}^{\mathrm{T}}\right) \\
= & \lambda_{1} \boldsymbol{G}_{1}+\lambda_{2} \boldsymbol{G}_{2}+\cdots+\lambda_{r} \boldsymbol{G}_{r}
\end{aligned}
$$

类似于正规矩阵的谱分解定理,有单纯矩阵的谱分解定理:\\
定理 4.5.2 设 $\boldsymbol{A}$ 为单纯矩阵,$\lambda_{1}, \lambda_{2}, \cdots, \lambda_{r}$ 为其相异特征值,$\lambda_{\mathrm{i}}$ 的代数重复度为 $n_{i}$ .则存在 $r$ 个 $n$ 阶矩阵 $\boldsymbol{G}_{1}, \boldsymbol{G}_{2}, \cdots, \boldsymbol{G}_{r}$ ,满足\\
(1)$A=\sum_{i=1}^{r} \lambda_{i} G_{i} ;$\\
(2) $\boldsymbol{G}_{j}^{2}=\boldsymbol{G}_{j}$ ;\\
(3) $\boldsymbol{G}_{i} \boldsymbol{G}_{j}=0 \quad(i \neq j)$ ;\\
(4)$\sum_{i=1}^{r} \boldsymbol{G}_{i}=\boldsymbol{E}$ ;\\
(5)满足上述性质的 $\boldsymbol{G}_{j}$ 是唯一的;\\
(6) $\operatorname{rank} \boldsymbol{G}_{j}=n_{j}$ .\\
证明从略。\\
单纯矩阵谱分解步骤:\\
(1)先求出矩阵 $\boldsymbol{A}$ 的特征值 $\lambda_{1}$ 与特征向量 $\boldsymbol{\alpha}_{i}$ .不妨设相异特征值为 $\lambda_{1}, \lambda_{2}$ , $\cdots, \lambda_{r}$ .特征值 $\lambda_{\mathrm{i}}$ 所对应的线性无关特征向量 $\boldsymbol{\alpha}_{i 1}, \boldsymbol{\alpha}_{i 2}, \cdots, \boldsymbol{\alpha}_{i n_{i}}$ .于是 $\boldsymbol{P}=\left(\boldsymbol{\alpha}_{11}, \cdots\right.$ , $\left.\boldsymbol{\alpha}_{1 n_{1}}, \boldsymbol{\alpha}_{21}, \cdots, \boldsymbol{\alpha}_{2 n_{2}}, \cdots, \boldsymbol{\alpha}_{r n_{r}}\right)$\\
(2)根据矩阵转置的性质得到

$$
\left(\boldsymbol{P}^{-1}\right)^{\mathrm{T}}=\left(\boldsymbol{\beta}_{1}, \boldsymbol{\beta}_{2}, \cdots, \boldsymbol{\beta}_{n}\right)
$$

此即 $\boldsymbol{\beta}_{11}, \cdots, \boldsymbol{\beta}_{1 n_{1}}, \boldsymbol{\beta}_{21}, \cdots, \boldsymbol{\beta}_{2 n_{2}}, \cdots, \boldsymbol{\beta}_{m n_{r}}$ .\\
(3)令 $\boldsymbol{G}_{i}=\boldsymbol{\alpha}_{i 1} \boldsymbol{\beta}_{i 1}^{\mathbf{T}}+\boldsymbol{\alpha}_{i 2} \boldsymbol{\beta}_{i 2}^{\mathbf{T}}+\cdots+\boldsymbol{\alpha}_{i n} \boldsymbol{\beta}_{i n_{i}}^{\mathbf{T}}$\\
则

$$
A=\lambda_{1} G_{1}+\lambda_{2} G_{2}+\cdots+\lambda_{r} G_{r}
$$

常称 $\boldsymbol{G}_{\boldsymbol{i}}$ 为 $\boldsymbol{A}$ 的投影矩阵。\\
例4.5.3 已知矩阵

$$
A=\left[\begin{array}{rrr}
4 & 6 & 0 \\
-3 & -5 & 0 \\
-3 & -6 & 1
\end{array}\right]
$$

为单纯矩阵,试作 $\boldsymbol{A}$ 的谱分解。\\
解 先求 $\boldsymbol{A}$ 的特征值和特征向量,由

$$
|\lambda \boldsymbol{E}-\boldsymbol{A}|=\left|\begin{array}{ccc}
\lambda-4 & -6 & 0 \\
3 & \lambda+5 & 0 \\
3 & 6 & \lambda-1
\end{array}\right|=(\lambda-1)^{2}(\lambda+2)
$$

故 $A$ 的特征值为:$\lambda_{1}=\lambda_{2}=1, \lambda_{3}=-2$ .\\
当 $\lambda=1$ 时,由方程组

$$
\left[\begin{array}{rrr}
-3 & -6 & 0 \\
3 & 6 & 0 \\
3 & 6 & 0
\end{array}\right]\left[\begin{array}{l}
x_{1} \\
x_{2} \\
x_{3}
\end{array}\right]=0
$$

求得特征向量为

$$
\begin{aligned}
& \boldsymbol{\alpha}_{1}=(2,-1,0)^{\mathrm{T}} \\
& \boldsymbol{\alpha}_{2}=(0,0,1)^{\mathrm{T}}
\end{aligned}
$$

当 $\lambda=-2$ 时,由方程组

$$
\left[\begin{array}{rrr}
-6 & -6 & 0 \\
3 & 3 & 0 \\
3 & 6 & -3
\end{array}\right]\left[\begin{array}{l}
x_{1} \\
x_{2} \\
x_{3}
\end{array}\right]=0
$$

求得特征向量为

$$
\alpha_{3}=(-1,1,1)^{\mathrm{T}}
$$

所以

$$
\begin{gathered}
\boldsymbol{P}=\left(\boldsymbol{\alpha}_{1}, \boldsymbol{\alpha}_{2}, \boldsymbol{\alpha}_{3}\right)=\left[\begin{array}{rrr}
2 & 0 & -1 \\
-1 & 0 & 1 \\
0 & 1 & 1
\end{array}\right] \\
\boldsymbol{P}^{-1}=\left[\begin{array}{rrr}
1 & 1 & 0 \\
-1 & -2 & 1 \\
1 & 2 & 0
\end{array}\right] \\
\left(\boldsymbol{P}^{-1}\right)^{\mathrm{T}}=\left[\begin{array}{rrr}
1 & -1 & 1 \\
1 & -2 & 2 \\
0 & 1 & 0
\end{array}\right]
\end{gathered}
$$

因此

$$
\begin{gathered}
\boldsymbol{\beta}_{1}=(1,1,0)^{\mathrm{T}} \\
\boldsymbol{\beta}_{2}=(-1,-2,1)^{\mathrm{T}} \\
\boldsymbol{\beta}_{3}=(1,2,0)^{\mathrm{T}}
\end{gathered}
$$

于是所求投影矩阵为

$$
\begin{aligned}
& \boldsymbol{G}_{1}=\boldsymbol{\alpha}_{1} \boldsymbol{\beta}_{1}^{\mathrm{T}}+\boldsymbol{\alpha}_{2} \boldsymbol{\beta}_{2}^{\mathrm{T}}=\left[\begin{array}{rrr}
2 & 2 & 0 \\
-1 & -1 & 0 \\
-1 & -2 & 1
\end{array}\right] \\
& \boldsymbol{G}_{2}=\boldsymbol{\alpha}_{3} \boldsymbol{\beta}_{3}^{\mathrm{T}}=\left[\begin{array}{rrr}
-1 & -2 & 0 \\
1 & 2 & 0 \\
1 & 2 & 0
\end{array}\right]
\end{aligned}
$$

$\boldsymbol{A}$ 的谱分解表达式为

$$
A=G_{1}-2 G_{2}
$$

例 4.5.4 设 $n$ 阶单纯矩阵 $\boldsymbol{A}$ 有 $r$ 个相异特征值 $\lambda_{1}, \lambda_{2}, \cdots, \lambda_{r} . \boldsymbol{G}_{1}, \boldsymbol{G}_{2}, \cdots, \boldsymbol{G}$ r为其投影矩阵,则

$$
f(\boldsymbol{A})=f\left(\lambda_{1}\right) \boldsymbol{G}_{1}+f\left(\lambda_{2}\right) \boldsymbol{G}_{2}+\cdots+f\left(\lambda_{r}\right) \boldsymbol{G}_{r}
$$

其中 $f(\lambda)$ 为 $\lambda$ 的某一多项式。\\
解 因为

$$
\begin{aligned}
\boldsymbol{A} & =\lambda_{1} \boldsymbol{G}_{1}+\lambda_{2} \boldsymbol{G}_{2}+\cdots+\lambda_{r} \boldsymbol{G}_{r} \\
\boldsymbol{A}^{k} & =\lambda_{1}^{k} \boldsymbol{G}_{1}+\lambda_{2}^{k} \boldsymbol{G}_{2}+\cdots+\lambda_{r}^{k} \boldsymbol{G}_{r}
\end{aligned}
$$

若

$$
f(\lambda)=a_{0}+a_{1} \lambda+a_{2} \lambda^{2}+\cdots+a_{s} \lambda^{s}
$$

则

$$
\begin{aligned}
f(A) & =a_{0} E+a_{1} A+a_{2} A^{2}+\cdots+a_{s} A^{s} \\
& =a_{0}\left(G_{1}+G_{2}+\cdots+G_{r}\right)+a_{1}\left(\lambda_{1} G_{1}+\lambda_{2} G_{2}+\cdots+\right.
\end{aligned}
$$

$$
\begin{aligned}
& \left.\lambda_{r} \boldsymbol{G}_{r}\right)+\cdots+a_{s}\left(\lambda_{1}^{s} \boldsymbol{G}_{1}+\lambda_{2}^{s} \boldsymbol{G}_{2}+\cdots+\lambda_{r}^{s} \boldsymbol{G}_{r}\right) \\
= & \left(a_{0}+a_{1} \lambda_{1}+\cdots+a_{s} \lambda_{1}^{s}\right) \boldsymbol{G}_{1}+\left(a_{0}+a_{1} \lambda_{2}+\cdots+a_{s} \lambda_{2}^{s}\right) \boldsymbol{G}_{2} \\
& +\cdots+\left(a_{0}+a_{1} \lambda_{r}+\cdots+a_{s} \lambda_{r}^{s}\right) \boldsymbol{G}_{r} \\
= & f\left(\lambda_{1}\right) \boldsymbol{G}_{1}+f\left(\lambda_{2}\right) \boldsymbol{G}_{2}+\cdots+f\left(\lambda_{r}\right) \boldsymbol{G}_{r}
\end{aligned}
$$

例4.5.5 对实对称矩阵

$$
\boldsymbol{A}=\left[\begin{array}{rrrr}
0 & 1 & 1 & -1 \\
1 & 0 & -1 & 1 \\
1 & -1 & 0 & 1 \\
-1 & 1 & 1 & 0
\end{array}\right]
$$

作谱分解。\\
解 $\boldsymbol{A}$ 既是正规矩阵,又是单纯矩阵,我们用两种观点对矩阵 $\boldsymbol{A}$ 作谱分解.\\
方法一: $\boldsymbol{A}$ 是单纯矩阵

$$
|\lambda E-A|=\left|\begin{array}{rrrr}
\lambda & -1 & -1 & 1 \\
-1 & \lambda & 1 & -1 \\
-1 & 1 & \lambda & -1 \\
1 & -1 & -1 & \lambda
\end{array}\right|=(\lambda-1)^{3}(\lambda+3)
$$

$\boldsymbol{A}$ 的特征值 $\lambda_{1}=\lambda_{2}=\lambda_{3}=1, \lambda_{4}=-3$ 。\\
当 $\lambda=1$ 时,可以求得特征向量为

$$
\begin{aligned}
& \boldsymbol{\alpha}_{1}=(1,1,0,0)^{\mathrm{T}}, \quad \boldsymbol{\alpha}_{2}=(1,0,1,0)^{\mathrm{T}}, \\
& \boldsymbol{\alpha}_{3}=(-1,0,0,1)^{\mathrm{T}}
\end{aligned}
$$

当 $\lambda=-3$ 时,可以求得特征向量为

$$
\lambda_{4}=(1,-1,-1,1)^{\mathrm{T}}
$$

所以

$$
\begin{gathered}
\boldsymbol{P}=\left(\boldsymbol{\alpha}_{1}, \boldsymbol{\alpha}_{2}, \boldsymbol{\alpha}_{3}, \boldsymbol{\alpha}_{4}\right)=\left[\begin{array}{rrrr}
1 & 1 & -1 & 1 \\
1 & 0 & 0 & -1 \\
0 & 1 & 0 & -1 \\
0 & 0 & 1 & 1
\end{array}\right] \\
\boldsymbol{P}^{-1}=\left[\begin{array}{rrrr}
\frac{1}{4} & \frac{3}{4} & -\frac{1}{4} & \frac{1}{4} \\
\frac{1}{4} & -\frac{1}{4} & \frac{3}{4} & \frac{1}{4} \\
-\frac{1}{4} & \frac{1}{4} & \frac{1}{4} & \frac{3}{4} \\
\frac{1}{4} & -\frac{1}{4} & -\frac{1}{4} & \frac{1}{4}
\end{array}\right]
\end{gathered}
$$

$$
\left(\boldsymbol{P}^{-1}\right)^{\mathrm{T}}=\left[\begin{array}{rrrr}
\frac{1}{4} & \frac{1}{4} & -\frac{1}{4} & \frac{1}{4} \\
\frac{3}{4} & -\frac{1}{4} & \frac{1}{4} & -\frac{1}{4} \\
-\frac{1}{4} & \frac{3}{4} & \frac{1}{4} & -\frac{1}{4} \\
\frac{1}{4} & \frac{1}{4} & \frac{3}{4} & \frac{1}{4}
\end{array}\right]
$$

因此

$$
\begin{aligned}
& \boldsymbol{\beta}_{1}=\left(\frac{1}{4}, \frac{3}{4},-\frac{1}{4}, \frac{1}{4}\right)^{\mathrm{T}} \\
& \boldsymbol{\beta}_{2}=\left(\frac{1}{4},-\frac{1}{4}, \frac{3}{4}, \frac{1}{4}\right)^{\mathrm{T}} \\
& \boldsymbol{\beta}_{3}=\left(-\frac{1}{4}, \frac{1}{4}, \frac{1}{4}, \frac{3}{4}\right)^{\mathrm{T}} \\
& \boldsymbol{\beta}_{4}=\left(\frac{1}{4},-\frac{1}{4},-\frac{1}{4}, \frac{1}{4}\right)^{\mathrm{T}}
\end{aligned}
$$

于是 $\boldsymbol{A}$ 的投影矩阵为

$$
\begin{aligned}
\boldsymbol{G}_{1} & =\boldsymbol{\alpha}_{1} \boldsymbol{\beta}_{1}^{\mathrm{T}}+\boldsymbol{\alpha}_{2} \boldsymbol{\beta}_{2}^{\mathrm{T}}+\boldsymbol{\alpha}_{3} \boldsymbol{\beta}_{3}^{\mathrm{T}} \\
& =\left[\begin{array}{rrrr}
\frac{3}{4} & \frac{1}{4} & \frac{1}{4} & -\frac{1}{4} \\
\frac{1}{4} & \frac{3}{4} & -\frac{1}{4} & \frac{1}{4} \\
\frac{1}{4} & -\frac{1}{4} & \frac{3}{4} & \frac{1}{4} \\
-\frac{1}{4} & \frac{1}{4} & -\frac{1}{4} & \frac{3}{4}
\end{array}\right] \\
\boldsymbol{G}_{2} & =\boldsymbol{\alpha}_{4} \boldsymbol{\beta}_{4}^{\mathrm{T}} \\
& =\left[\begin{array}{rrrr}
\frac{1}{4} & -\frac{1}{4} & -\frac{1}{4} & \frac{1}{4} \\
-\frac{1}{4} & \frac{1}{4} & \frac{1}{4} & -\frac{1}{4} \\
-\frac{1}{4} & \frac{1}{4} & \frac{1}{4} & -\frac{1}{4} \\
\frac{1}{4} & -\frac{1}{4} & -\frac{1}{4} & -\frac{1}{4}
\end{array}\right]
\end{aligned}
$$

$\boldsymbol{A}$ 的谱分解表达式为

$$
A=G_{1}-3 G_{2}
$$

方法二: $\boldsymbol{A}$ 是正规矩阵。

由方法一中已知 $\boldsymbol{A}$ 的特征值 $\lambda_{1}=\lambda_{2}=\lambda_{3}=1, \lambda_{4}=-3$ ,把 $\boldsymbol{\alpha}_{1}, \boldsymbol{\alpha}_{2}, \boldsymbol{\alpha}_{3}$ 用 Schmidt 方法标准正交化得

$$
\begin{gathered}
v_{1}=\left(\frac{1}{\sqrt{2}}, \frac{1}{\sqrt{2}}, 0,0\right)^{\mathrm{T}} \\
v_{2}=\left(\frac{1}{\sqrt{6}},-\frac{1}{\sqrt{6}}, \frac{2}{\sqrt{6}}, 0\right)^{\mathrm{T}} \\
v_{3}=\left(-\frac{1}{\sqrt{12}}, \frac{1}{\sqrt{12}}, \frac{1}{\sqrt{12}}, \frac{3}{\sqrt{12}}\right)^{\mathrm{T}}
\end{gathered}
$$

把 $\boldsymbol{\alpha}_{4}$ 单位化得

$$
v_{4}=\left(\frac{1}{2},-\frac{1}{2},-\frac{1}{2}, \frac{1}{2}\right)^{\mathrm{T}}
$$

正交投影矩阵

$$
\begin{aligned}
G_{1} & =v_{1} v_{1}^{\mathrm{H}}+v_{2} v_{2}^{\mathrm{H}}+v_{3} v_{3}^{\mathrm{H}} \\
& =\left[\begin{array}{rrrr}
\frac{3}{4} & \frac{1}{4} & \frac{1}{4} & -\frac{1}{4} \\
\frac{1}{4} & \frac{3}{4} & -\frac{1}{4} & \frac{1}{4} \\
\frac{1}{4} & -\frac{1}{4} & \frac{3}{4} & \frac{1}{4} \\
-\frac{1}{4} & \frac{1}{4} & -\frac{1}{4} & \frac{3}{4}
\end{array}\right] \\
G_{2} & =v_{4} v_{4}^{\mathrm{H}} \\
& =\left[\begin{array}{rrrr}
\frac{1}{4} & -\frac{1}{4} & -\frac{1}{4} & \frac{1}{4} \\
-\frac{1}{4} & \frac{1}{4} & \frac{1}{4} & -\frac{1}{4} \\
-\frac{1}{4} & \frac{1}{4} & \frac{1}{4} & -\frac{1}{4} \\
\frac{1}{4} & -\frac{1}{4} & -\frac{1}{4} & \frac{1}{4}
\end{array}\right]
\end{aligned}
$$

$\boldsymbol{A}$ 的谱分解表达式为

$$
A=G_{1}-3 G_{2}
$$

用两种角度作出的谱分解结果是一样的,为什么?请读者自己分析。

\section*{习 题}
\section*{4-1 求矩阵 $\boldsymbol{A}$ 的满秩分解}
(1) $\boldsymbol{A}=\left[\begin{array}{lllll}2 & 1 & -2 & 3 & 1 \\ 2 & 5 & -1 & 4 & 1 \\ 1 & 3 & -1 & 2 & 1\end{array}\right]$\\
(2) $\boldsymbol{A}=\left[\begin{array}{lllll}1 & 1 & 0 & 1 & 0 \\ 0 & 1 & 1 & 1 & 1 \\ 2 & 3 & 1 & 3 & 1\end{array}\right]$\\
(3) $\boldsymbol{A}=\left[\begin{array}{cccccc}1 & 2 & 1 & 0 & 1 & 2 \\ 1 & 2 & 2 & 1 & 3 & 3 \\ 2 & 4 & 3 & 1 & 4 & 5 \\ 4 & 8 & 6 & 2 & 8 & 10\end{array}\right]$\\
(4) $\boldsymbol{A}=\left[\begin{array}{cccccc}1 & 2 & 0 & 1 & 1 & 10 \\ 3 & 6 & 1 & 4 & 2 & 36 \\ 2 & 4 & 0 & 2 & 2 & 27 \\ 6 & 12 & 1 & 7 & 5 & 73\end{array}\right]$

4-2 已知

$$
A=\left[\begin{array}{lll}
1 & 1 & 1 \\
1 & 1 & 1
\end{array}\right]
$$

求 $\boldsymbol{A}$ 的奇异值分解。\\
4-3 已知矩阵

$$
A=\left[\begin{array}{rrr}
3 & -1 & 0 \\
-1 & 2 & -1 \\
0 & -1 & 3
\end{array}\right] \quad B=\left[\begin{array}{lll}
0 & 1 & 1 \\
1 & 0 & 1 \\
1 & 1 & 0
\end{array}\right]
$$

验证 $\boldsymbol{A}$ 与 $\boldsymbol{B}$ 是正规矩阵,并求 $\boldsymbol{A}$ 与 $\boldsymbol{B}$ 的谱分解.\\
4-4 已知矩阵

$$
A=\left[\begin{array}{ccc}
0 & 2 & 4 \\
\frac{1}{2} & 0 & 2 \\
\frac{1}{4} & \frac{1}{2} & 0
\end{array}\right]
$$

验证 $\boldsymbol{A}$ 是单纯矩阵,并求 $\boldsymbol{A}$ 的谱分解.\\
4-5 求矩阵 $\boldsymbol{A}=\left[\begin{array}{ll}2 & i \\ 0 & 0 \\ 0 & 0\end{array}\right]$ 的奇异值分解.\\
4-6 求矩阵 $\boldsymbol{A}=\left[\begin{array}{cc}2 & 0 \\ 0 & i \\ 0 & 0\end{array}\right]$ 的奇异值分解.\\
4-7 已知 $A=\left[\begin{array}{ccc}2 & 2 & 0 \\ 8 & 2 & a \\ 0 & 0 & 6\end{array}\right]$ 是单纯矩阵.\\
(1)求 $a$ ;\\
(2)求 $\boldsymbol{A}$ 的谱分解表达式.\\
4-8 已知 $\boldsymbol{A}=\left[\begin{array}{ccc}3 & i & -1 \\ -i & 0 & i \\ -1 & -i & 0\end{array}\right]$\\
(1)试证: $\boldsymbol{A}$ 是正规矩阵.\\
(2)求 $\boldsymbol{A}$ 的谱分解表达式。

\section*{第五章}
\section*{范数、序列、级数}
向量(或矩阵)范数是向量(或矩阵)的数字特征,在某种意义上范数相当于实数和复数的绝对值,它在研究序列、级数、极限范围内起到基础的作用。

\section*{§5.1 向 量 范 数}
在初等数学与微积分学中,对于实数和复数,由于定义了绝对值用来表示其大小,带来了许多方便。在解析几何及力学中讨论的向量也是根据绝对值来定义向量的大小。在第一章中把向量概念推广到线性空间中。在第三章线性空间中引进内积的概念,对向量赋予向量长度,向量的夹角等概念。向量范数是比向量长度更一般的概念。引入一个内积向量长度是唯一的,向量的范数对于一个向量可以有多个向量范数,只要它满足向量范数定义即可。

为了更好地理解向量范数定义,先看例5.1.1.\\
例 5.1.1 $n$ 维欧氏空间中向量的模为 $|x|=(x, x)^{\frac{1}{2}}$ ,它具有如下三个性质:\\
(1)若 $\boldsymbol{x} \neq 0$ ,则 $|\boldsymbol{x}|>0$ ;若 $\boldsymbol{x}=0$ ,则 $|\boldsymbol{x}|=0$ ;\\
(2)$|k x|=|k||x|, k$ 为任意实数;\\
(3)对于任何向量 $\boldsymbol{x}$ 和 $\boldsymbol{y}$ ,有三角不等式 $|\boldsymbol{x}+\boldsymbol{y}| \leqslant|\boldsymbol{x}|+|\boldsymbol{y}|$ 。\\
定义 5.1.1 设 $V$ 是数域 $F$(一般为实数域 $\mathbf{R}$ 或复数域 $\mathbf{C})$ 上的线性空间,用 $\|x\|$ 表示按照某个法则确定的与向量 $x$ 对应的实数,且满足\\
(1)非负性:当 $x \neq 0,\|x\|>0$ ;当且仅当 $x=0$ 时,$\|x\|=0$ ;\\
(2)齐次性:$\|k x\|=|k|\|x\|, k$ 为任意数;\\
(3)三角不等式:对于 $V$ 中任何向量 $\boldsymbol{x}, \boldsymbol{y}$ 都有

$$
\|x+y\| \leqslant\|x\|+\|y\|
$$

则称实数 $\|x\|$ 是向量 $x$ 的范数.\\
由向量范数定义不难验证:\\
(1)$\|-\boldsymbol{x}\|=\|\boldsymbol{x}\|$\\
(2)$\|x-y\| \geqslant|\|x\|-\|y\||$\\
(3)$\|x+y\| \geqslant|\|x\|-\|y\||$\\
由齐次性易证(1)。现只证(2),(3)作为练习留给读者。

事实上,

$$
\|x\|=\|x-y+y\| \leqslant\|x-y\|+\|y\|
$$

故


\begin{equation*}
\|x\|-\|y\| \leqslant\|x-y\| \tag{*}
\end{equation*}


又

$$
\begin{aligned}
\|y\| & =\|x-x+y\| \leqslant\|x\|+\|-x+y\| \\
& =\|x\|+\|x-y\|
\end{aligned}
$$

故


\begin{equation*}
\|x\|-\|y\| \geqslant-\|x-y\| \tag{**}
\end{equation*}


综合式( $*$ )( $* *$ )得

$$
\|x-y\| \geqslant|\|x\|-\|y\|| . \quad \text { (证毕) }
$$

为了进一步讨论的需要,先证明 Hölder 不等式与 Minkowski 不等式。\\
Hölder 不等式:设 $p>1, q=\frac{p}{(p-1)}$ ,则


\begin{equation*}
\sum_{k=1}^{n} a_{k} b_{k} \leqslant\left(\sum_{k=1}^{n} a_{k}^{p}\right)^{\frac{1}{p}}\left(\sum_{k=1}^{n} b_{k}^{q}\right)^{\frac{1}{q}} \tag{5.1.1}
\end{equation*}


其中 $a_{k}, b_{k} \geqslant 0$ .\\
证明 首先证明:若 $u, v$ 均非负,则总有


\begin{equation*}
u v \leqslant \frac{u^{p}}{p}+\frac{v^{q}}{q} \tag{5.1.2}
\end{equation*}


事实上,若令 $\varphi(v)=\frac{u^{p}}{p}+\frac{v^{q}}{q}-u v$ ,则因 $\varphi(0) \geqslant 0, \varphi(v) \rightarrow+\infty($ 当 $v \rightarrow+\infty)$ ,并且

$$
\varphi^{\prime}(v)=v^{q-1}-u
$$

故当 $v=u^{\frac{1}{q-1}}$ 时,$\varphi^{\prime}(v)=0$ ,容易验证 $\varphi\left(u^{\frac{1}{q-1}}\right)=0$ 是 $\varphi(v)$ 的最小值,所以 $\varphi(v) \geqslant 0$ ,此即式(5.1.2)成立。\\
令

$$
u=\frac{a_{k}}{a}, \quad v=\frac{b_{k}}{b}
$$

其中

$$
a=\left(\sum_{k=1}^{n} a_{k}^{p}\right)^{\frac{1}{p}}, \quad b=\left(\sum_{k=1}^{n} b_{k}^{q}\right)^{\frac{1}{q}}
$$

代人式(5.1.2),得

$$
a_{k} b_{k} \leqslant a b\left(\frac{1}{p} \frac{a_{k}^{p}}{a^{p}}+\frac{1}{q} \frac{b_{k}^{q}}{b^{q}}\right)
$$

于是

$$
\begin{aligned}
\sum_{k=1}^{n} a_{k} b_{k} & \leqslant a b\left(\frac{1}{p a^{p}} \sum_{k=1}^{n} a_{k}^{p}+\frac{1}{q b^{q}} \sum_{k=1}^{n} b_{k}^{q}\right) \\
& =a b\left(\frac{1}{p}+\frac{1}{q}\right)=a b \\
& =\left(\sum_{k=1}^{n} a_{k}^{p}\right)^{\frac{1}{p}}\left(\sum_{k=1}^{n} b_{k}^{q}\right)^{\frac{1}{q}}
\end{aligned}
$$

特别若 $p=q=2$ 便成为著名的 Schwarz 不等式


\begin{equation*}
\sum_{k=1}^{n} a_{k} b_{k} \leqslant\left(\sum_{k=1}^{n} a_{k}^{2}\right)^{\frac{1}{2}}\left(\sum_{k=1}^{n} b_{k}^{2}\right)^{\frac{1}{2}} \tag{5.1.3}
\end{equation*}


Minkowski 不等式:对任何 $p \geqslant 1$ ,有


\begin{equation*}
\left(\sum_{i=1}^{n}\left|a_{i}+b_{i}\right|^{p}\right)^{\frac{1}{p}} \leqslant\left(\sum_{i=1}^{n}\left|a_{i}\right|^{p}\right)^{\frac{1}{p}}+\left(\sum_{i=1}^{n}\left|b_{i}\right|^{p}\right)^{\frac{1}{p}} \tag{5.1.4}
\end{equation*}


证明 :当 $p=1$ 时,不等式显然成立,当 $p \neq 1$ 时,以 $q=p /(p-1)$ 代入下式

得

$$
\begin{aligned}
\sum_{i=1}^{n}\left|a_{i}+b_{i}\right|^{p} & =\sum_{i=1}^{n}\left|a_{i}+b_{i}\right|\left|a_{i}+b_{i}\right|^{p-1} \\
\sum_{i=1}^{n}\left|a_{i}+b_{i}\right|^{p} & =\sum_{i=1}^{n}\left|a_{i}+b_{i}\right|\left|a_{i}+b_{i}\right|^{\frac{p}{q}} \\
& \leqslant \sum_{i=1}^{n}\left|a_{i}\right|\left|a_{i}+b_{i}\right|^{\frac{p}{q}}+\sum_{i=1}^{n}\left|b_{i}\right|\left|a_{i}+b_{i}\right|^{\frac{p}{q}}
\end{aligned}
$$

按 Hölder 不等式得

$$
\begin{aligned}
\sum_{i=1}^{n}\left|a_{i}+b_{i}\right|^{p} \leqslant & \left(\sum_{i=1}^{n}\left|a_{i}\right|^{p}\right)^{\frac{1}{p}}\left(\sum_{i=1}^{n}\left|a_{i}+b_{i}\right|^{p}\right)^{\frac{1}{q}}+ \\
& \left(\sum_{i=1}^{n}\left|b_{i}\right|^{p}\right)^{\frac{1}{p}}\left(\sum_{i=1}^{n}\left|a_{i}+b_{i}\right|^{p}\right)^{\frac{1}{q}} \\
=\left[\left(\sum_{i=1}^{n}\left|a_{i}\right|^{p}\right)^{\frac{1}{p}}+\right. & \left.\left(\sum_{i=1}^{n}\left|b_{i}\right|^{p}\right)^{\frac{1}{p}}\right]\left(\sum_{i=1}^{n}\left|a_{i}+b_{i}\right|^{p}\right)^{\frac{1}{q}}
\end{aligned}
$$

不等式两端除以 $\left(\sum_{i=1}^{n}\left|a_{i}+b_{i}\right|^{p}\right)^{\frac{1}{q}}$ ,根据 $1-\left(\frac{1}{q}\right)=\frac{1}{p}$ 得

$$
\left(\sum_{i=1}^{n}\left|a_{i}+b_{i}\right|^{p}\right)^{\frac{1}{p}} \leqslant\left(\sum_{i=1}^{n}\left|a_{i}\right|^{p}\right)^{\frac{1}{p}}+\left(\sum_{i=1}^{n}\left|b_{i}\right|^{p}\right)^{\frac{1}{p}}
$$

由 Minkowski 不等式,可以引人常用的 $p$-范数。\\
定义5.1.2 设向量 $x=\left(x_{1}, x_{2}, \cdots, x_{n}\right)^{\mathrm{T}}$ ,对任意数 $p \geqslant 1$ ,称量


\begin{equation*}
\|x\|_{p}=\left(\sum_{i=1}^{n}\left|x_{i}\right|^{p}\right)^{\frac{1}{p}} \tag{5.1.5}
\end{equation*}


为向量 $\boldsymbol{x}$ 的 $p$-范数。\\
易知,$\|\boldsymbol{x}\|_{p}$ 满足非负性、齐次性。Minkowski 不等式(5.1.4)就是 $\|\boldsymbol{x}+\boldsymbol{y}\|_{p} \leqslant \|\boldsymbol{x}\|_{p}+\|\boldsymbol{y}\|_{p}$ .所以由式(5.1.5)所确定的 $\|\boldsymbol{x}\|_{p}$ 确实是向量范数。

常用的 $p$-范数有下述三种:\\
(1)1-范数 $\quad\|\boldsymbol{x}\|_{1}=\sum_{i=1}^{n}\left|x_{i}\right|$\\
(2)2-范数 $\quad\|\boldsymbol{x}\|_{2}=\left(\sum_{i=1}^{n}\left|x_{i}\right|^{2}\right)^{\frac{1}{2}}=\left(\boldsymbol{x}^{\mathrm{H}} \boldsymbol{x}\right)^{\frac{1}{2}}$

也称为欧氏范数。\\
(3)$\infty$-范数 $\quad\|x\|_{\infty}=\lim _{p \rightarrow \infty}\|x\|_{p}$\\
定理 5.1.1 $\|\boldsymbol{x}\|_{\infty}=\boldsymbol{\operatorname { m a x }}\left|x_{i}\right|,(i=1,2, \cdots, n)$

证明 令 $\alpha=\max _{i}\left|x_{i}\right|$ ,则

$$
\beta_{i}=\frac{\left|x_{i}\right|}{\alpha} \leqslant 1 \quad(i=1,2, \cdots, n)
$$

于是

$$
\|\boldsymbol{x}\|_{p}=\alpha\left(\sum_{i=1}^{n} \beta_{i}^{p}\right)^{\frac{1}{p}}
$$

由于

$$
1 \leqslant\left(\sum_{i=1}^{n} \beta_{i}^{p}\right)^{\frac{1}{p}} \leqslant n^{\frac{1}{p}}
$$

故

$$
\lim _{p \rightarrow \infty}\left(\sum_{i=1}^{n} \beta_{i}^{p}\right)^{\frac{1}{p}}=1
$$

因此

$$
\|\boldsymbol{x}\|_{\infty}=\lim _{p \rightarrow \infty}\|\boldsymbol{x}\|_{p}=\alpha=\max _{i}\left|x_{i}\right|
$$

在一个线性空间中,可以引进各种范数,按照不同法则规定的向量范数,其大小一般不等、例如对 $R^{n}$ 中的向量 $x=(1,1, \cdots, 1)^{\mathrm{T}}$ ,有

$$
\|\boldsymbol{x}\|_{1}=n,\|\boldsymbol{x}\|_{2}=\sqrt{n},\|\boldsymbol{x}\|_{\infty}=1
$$

虽然一个向量不同的范数有不同的值,但是这些范数之间有着重要的关系。譬如,在考虑向量序列收敛性时,它们就表现出明显的一致性。这种性质称为范数的等价性。

定理5.1.2 设 $V$ 是 $n$ 维线性空间,$\|\boldsymbol{x}\|_{\boldsymbol{\alpha}}$ 和 $\|\boldsymbol{x}\|_{\boldsymbol{\beta}}$ 为任意两种向量范数 (不限于 $p$-范数),则总存在正数 $c_{1}, c_{2}$ ,对 $V$ 中所有向量 $\boldsymbol{x} \in V$ 恒有


\begin{equation*}
c_{1}\|\boldsymbol{x}\|_{\beta} \leqslant\|\boldsymbol{x}\|_{\alpha} \leqslant c_{2}\|\boldsymbol{x}\|_{\beta} \tag{5.1.9}
\end{equation*}


证略.

\section*{§5.2 矩阵范数}
$\boldsymbol{m} \times \boldsymbol{n}$ 矩阵 $\boldsymbol{A}$ 可以看做是 $\boldsymbol{m} \boldsymbol{n}$ 维向量空间 $\boldsymbol{C}^{m \times n}$ 中向量,所以矩阵范数完全可以仿照向量范数定义而不必重新给出。但是矩阵作为线性映射的表示,还必须考虑矩阵的乘法。因此矩阵范数必须考虑到乘积范数与因子范数之间的限制要求。此即不等式:


\begin{equation*}
\|A B\| \leqslant\|A\|\|B\| \tag{5.2.1}
\end{equation*}


若把向量范数形式上推广到矩阵范数时,发现有的范数符合式(5.2.1)。有的不符合式(5.2.1)。例如向量 1 -范数形式上推广到矩阵范数 $\|\boldsymbol{A}\|=\sum_{i=1}^{m} \sum_{j=1}^{n}\left|\boldsymbol{a}_{i j}\right|$ ,向量 2 -范数形式上推广到矩阵范数 $\|\boldsymbol{A}\|_{\boldsymbol{F}}=\left(\sum_{i=1}^{m} \sum_{j=1}^{n}\left|a_{i j}\right|^{2}\right)^{1 / 2}$ 都满足式 (5.2.1),(见例5.2.1,例5.2.2)但是向量 $\infty$-范数形式上推广到矩阵范数\\
$\|\boldsymbol{A}\|=\max _{i, j}\left|a_{i j}\right|$ 不满足式(5.2.1).取 $\boldsymbol{A}=\left[\begin{array}{ll}1 & 1 \\ 1 & 1\end{array}\right], \boldsymbol{B}=\left[\begin{array}{ll}1 & 1 \\ 0 & 1\end{array}\right]$ ,则 $\boldsymbol{A B}=\left[\begin{array}{ll}1 & 2 \\ 1 & 2\end{array}\right]$ , $\|\boldsymbol{A}\|=1,\|\boldsymbol{B}\|=1,\|\boldsymbol{A} \boldsymbol{B}\|=2$ ,不满足式(5.2.1).所以,矩阵范数的定义必须在向量范数定义的三个条件上添加式(5.2.1)。此即定义5.2.1。

定义 5.2.1 对于任何一个矩阵 $\boldsymbol{A} \in C^{m \times n}$ ,用 $\|\boldsymbol{A}\|$ 表示按照某个法则确定的与矩阵 $\boldsymbol{A}$ 对应的实数,且满足\\
(1)非负性:当 $A \neq 0$ 时,$\|A\|>0$ ;当且仅当 $A=0$ 时,$\|A\|=0$\\
(2)齐次性:$\|k \boldsymbol{A}\|=|k|\|\boldsymbol{A}\|, k$ 为任意复数\\
(3)三角不等式:对于任何两个同类型矩阵 $A, B$ 都有

$$
\|A+B\| \leqslant\|A\|+\|B\|
$$

(4)矩阵乘法相容性:若 $\boldsymbol{A}$ 与 $\boldsymbol{B}$ 可乘,有

$$
\|A B\| \leqslant\|A\|\|B\|
$$

则称对应于 $\boldsymbol{A}$ 的这个实数 $\|\boldsymbol{A}\|$ 是矩阵 $\boldsymbol{A}$ 的矩阵范数。\\
矩阵范数的齐次性、三角不等式是针对矩阵与数相乘、矩阵加法运算以后范数的变化特性,这与向量范数相同。由于矩阵运算还有矩阵与矩阵相乘,因此在矩阵范数定义中比向量范数定义增加了相容性的要求。

例 5.2.1 试证:对于 $\boldsymbol{A}=\left(a_{i j}\right) \in C^{m \times n},\|\boldsymbol{A}\|=\sum_{i=1}^{m} \sum_{j=1}^{n}\left|a_{i j}\right|$ 是矩阵范数。它是向量1-范数的形式推广。

解 需要验证给出的公式满足矩阵范数的四个性质。非负性与齐次性容易验证,现证三角不等式.若设 $B=\left(b_{i j}\right) \in C^{m \times n}$ ,则

$$
\begin{aligned}
\|\boldsymbol{A}+\boldsymbol{B}\| & =\sum_{i=1}^{m} \sum_{j=1}^{n}\left|a_{i j}+b_{i j}\right| \\
& \leqslant \sum_{i=1}^{m} \sum_{j=1}^{n}\left(\left|a_{i j}\right|+\left|b_{i j}\right|\right) \\
& =\sum_{i=1}^{m} \sum_{j=1}^{n}\left|a_{i j}\right|+\sum_{i=1}^{m} \sum_{j=1}^{n}\left|b_{i j}\right| \\
& =\|\boldsymbol{A}\|+\|\boldsymbol{B}\|
\end{aligned}
$$

最后证矩阵乘法的相容性.若 $\boldsymbol{A}=\left(a_{i j}\right) \in C^{m \times p}, \boldsymbol{B}=\left(b_{i j}\right) \in C^{p \times n}$ ,则

$$
\begin{aligned}
\|\boldsymbol{A} \boldsymbol{B}\| & =\sum_{i=1}^{m} \sum_{j=1}^{n}\left|\sum_{k=1}^{p} a_{i k} b_{k j}\right| \\
& \leqslant \sum_{i=1}^{m} \sum_{j=1}^{n} \sum_{k=1}^{p}\left|a_{i k}\right|\left|b_{k j}\right| \\
& \leqslant \sum_{i=1}^{m} \sum_{j=1}^{n}\left[\left(\sum_{k=1}^{p}\left|a_{i k}\right|\right)\left(\sum_{k=1}^{p}\left|b_{k j}\right|\right)\right]
\end{aligned}
$$

$$
\begin{aligned}
& =\left(\sum_{i=1}^{m} \sum_{k=1}^{p}\left|a_{i k}\right|\right)\left(\sum_{j=1}^{n} \sum_{k=1}^{p}\left|b_{k j}\right|\right) \\
& =\|\boldsymbol{A}\|\|\boldsymbol{B}\|
\end{aligned}
$$

因此所给计算公式的确是矩阵范数。\\
例 5.2.2 Frobenius 范数:若 $\boldsymbol{A}=\left(a_{i j}\right) \in C^{m \times n}$ ,规定


\begin{equation*}
\|\boldsymbol{A}\|_{F}=\left(\sum_{i=1}^{m} \sum_{j=1}^{n}\left|a_{i j}\right|^{2}\right)^{\frac{1}{2}} \tag{5.2.2}
\end{equation*}


可以证明:由式(5.2.2)所确定的数满足范数定义中四个性质。称 $\|\boldsymbol{A}\|_{\boldsymbol{F}}$ 为矩阵 $\boldsymbol{A}$ 的 Frobenius 范数,是向量范数中欧氏范数的形式推广。

解 矩阵范数的非负性、齐次性易证。现证三角不等式与相容性。\\
根据 Minkowski 不等式易证三角不等式。设 $\boldsymbol{A}=\left(a_{i j}\right)_{m \times n}, \boldsymbol{B}=\left(b_{i j}\right)_{m \times n}$ ,则

$$
\begin{aligned}
\|\boldsymbol{A}+\boldsymbol{B}\|_{F} & =\left(\sum_{i=1}^{m} \sum_{j=1}^{n}\left|a_{i j}+b_{i j}\right|^{2}\right)^{\frac{1}{2}} \\
& \leqslant\left(\sum_{i=1}^{m} \sum_{j=1}^{n}\left|a_{i j}\right|^{2}\right)^{\frac{1}{2}}+\left(\sum_{i=1}^{m} \sum_{j=1}^{n}\left|b_{i j}\right|^{2}\right)^{\frac{1}{2}} \\
& =\|\boldsymbol{A}\|_{F}+\|\boldsymbol{B}\|_{F}
\end{aligned}
$$

设 $\boldsymbol{A}=\left(a_{i j}\right)_{m \times l}, \boldsymbol{B}=\left(b_{i j}\right)_{l \times n}$ ,则

$$
\begin{aligned}
\|\boldsymbol{A} \boldsymbol{B}\|_{F}^{2} & =\sum_{i=1}^{m} \sum_{j=1}^{n}\left|\sum_{k=1}^{l} a_{i k} b_{k j}\right|^{2} \\
& \leqslant \sum_{i=1}^{m} \sum_{j=1}^{n}\left(\sum_{k=1}^{l}\left|a_{i k}\right|\left|b_{k j}\right|\right)^{2}
\end{aligned}
$$

根据 Hölder 不等式得

于是

$$
\begin{aligned}
\|\boldsymbol{A} \boldsymbol{B}\|_{F}^{2} \leqslant & \sum_{i=1}^{m} \sum_{j=1}^{n}\left[\left(\sum_{k=1}^{l}\left|a_{i k}\right|^{2}\right)\left(\sum_{k=1}^{l}\left|b_{k j}\right|^{2}\right)\right] \\
= & \left(\sum_{i=1}^{m} \sum_{k=1}^{l}\left|a_{i k}\right|^{2}\right)\left(\sum_{j=1}^{n} \sum_{k=1}^{l}\left|b_{k j}\right|^{2}\right) \\
= & \|\boldsymbol{A}\|_{F}^{2}\|\boldsymbol{B}\|_{F}^{2} \\
& \|A B\|_{F} \leqslant\|\boldsymbol{A}\|_{F}\|\boldsymbol{B}\|_{F}
\end{aligned}
$$

定理 5.2.1 Frobenius 范数性质。\\
(1)若 $\boldsymbol{A}=\left(\boldsymbol{\alpha}_{1}, \boldsymbol{\alpha}_{2}, \cdots, \boldsymbol{\alpha}_{n}\right)$ ,则

$$
\|\boldsymbol{A}\|_{F}^{2}=\sum_{i=1}^{n}\left\|\boldsymbol{\alpha}_{i}\right\|_{2}^{2}
$$

(2)$\|\boldsymbol{A}\|_{F}^{2}=\operatorname{tr}\left(\boldsymbol{A}^{\mathrm{H}} \boldsymbol{A}\right)=\sum_{i=1}^{n} \lambda_{\mathrm{i}}\left(\boldsymbol{A}^{\mathrm{H}} \boldsymbol{A}\right)$ 。其中 $\lambda_{\mathrm{i}}\left(\boldsymbol{A}^{\mathrm{H}} \boldsymbol{A}\right)$ 表示 $n$ 阶方阵 $\boldsymbol{A}^{\mathrm{H}} \boldsymbol{A}$第 $i$ 个特征值. $\operatorname{tr}\left(\boldsymbol{A}^{\mathrm{H}} \boldsymbol{A}\right)$ 是 $\boldsymbol{A}^{\mathrm{H}} \boldsymbol{A}$ 的迹.\\
(3)对于任何 $m$ 阶酉矩阵 $\boldsymbol{U}$ 与 $n$ 阶酉矩阵 $\boldsymbol{V}$ ,都有等式

$$
\|A\|_{F}=\|U A\|_{F}=\left\|A^{\mathrm{H}}\right\|_{F}=\|A V\|_{F}=\|U A V\|_{F}
$$

证明(1),(2)显然,现证(3).若 $\boldsymbol{A}=\left(\boldsymbol{\alpha}_{1}, \boldsymbol{\alpha}_{2}, \cdots, \boldsymbol{\alpha}_{n}\right)$ ,则

$$
\begin{aligned}
\|\boldsymbol{U} \boldsymbol{A}\|_{F}^{2} & =\left\|\boldsymbol{U}\left(\boldsymbol{\alpha}_{1}, \boldsymbol{\alpha}_{2}, \cdots, \boldsymbol{\alpha}_{n}\right)\right\|_{F}^{2}=\left\|\left(\boldsymbol{U} \boldsymbol{\alpha}_{1}, \boldsymbol{U} \boldsymbol{\alpha}_{2}, \cdots, \boldsymbol{U} \boldsymbol{\alpha}_{n}\right)\right\|_{F}^{2} \\
& =\left\|\boldsymbol{U} \boldsymbol{\alpha}_{1}\right\|_{2}^{2}+\left\|\boldsymbol{U} \boldsymbol{\alpha}_{2}\right\|_{2}^{2}+\cdots+\left\|\boldsymbol{U} \boldsymbol{\alpha}_{n}\right\|_{2}^{2} \\
& =\left\|\boldsymbol{\alpha}_{1}\right\|_{2}^{2}+\left\|\boldsymbol{\alpha}_{2}\right\|_{2}^{2}+\cdots+\left\|\boldsymbol{\alpha}_{n}\right\|_{2}^{2}=\|\boldsymbol{A}\|_{F}^{2}
\end{aligned}
$$

故

$$
\|\boldsymbol{A}\|_{F}=\|\boldsymbol{U} \boldsymbol{A}\|_{F}
$$

又由 $\|\boldsymbol{A}\|_{F}=\left\|\boldsymbol{A}^{\mathrm{H}}\right\|_{F}$ ,易得其余几个等式。\\
矩阵范数也有等价性定理,此即:\\
定理 5.2.2 若 $\|\boldsymbol{A}\|_{\alpha}$ 与 $\|\boldsymbol{A}\|_{\beta}$ 是任意两种矩阵范数,则总存在正数 $c_{1}, c_{2}$ ,对于任意矩阵 $\boldsymbol{A}$ 恒有


\begin{equation*}
c_{1}\|\boldsymbol{A}\|_{\beta} \leqslant\|\boldsymbol{A}\|_{\alpha} \leqslant c_{2}\|\boldsymbol{A}\|_{\beta} \tag{5.2.3}
\end{equation*}


证明从略。

\section*{§5.3 诱导范数(算子范数)}
设 $\boldsymbol{A} \in C^{m \times n}, ~ \boldsymbol{x} \in C^{n}$ ,若已知矩阵范数 $\|\boldsymbol{A}\|$ ,可将 $\boldsymbol{x}$ 与 $\boldsymbol{A x}$ 作为 $n \times 1$ 与 $m \times 1$矩阵,因此根据矩阵范数的相容性应有

$$
\|A x\| \leqslant\|A\|\|x\|
$$

但是,$x$ 与 $A x$ 终究是向量,若取 $\|x\|_{\alpha}$ 与 $\|A x\|_{\alpha}$ 是向量范数,$\|A\|$ 是矩阵范数,则不等式

$$
\|A x\|_{\alpha} \leqslant\|A\|\|x\|_{\alpha}
$$

是否仍能成立?这就是向量范数与矩阵范数相容性。\\
定义 5.3.1 设 $\|\boldsymbol{x}\|_{\alpha}$ 是向量范数,$\|\boldsymbol{A}\|_{\beta}$ 是矩阵范数,若对于任何矩阵 $\boldsymbol{A}$与向量 $\boldsymbol{x}$ 都有


\begin{equation*}
\|A x\|_{\alpha} \leqslant\|A\|_{\beta}\|x\|_{\alpha} \tag{5.3.1}
\end{equation*}


则称 $\|\boldsymbol{A}\|_{\boldsymbol{\beta}}$ 为与向量范数 $\|\boldsymbol{x}\|_{\alpha}$ 相容的矩阵范数。\\
例 5.3.1 试证:矩阵的 Frobenius 范数与向量的2-范数相容。\\
解 因为

$$
\|\boldsymbol{A}\|_{F}=\left(\sum_{i=1}^{m} \sum_{j=1}^{n}\left|a_{i j}\right|^{2}\right)^{\frac{1}{2}},\|x\|_{2}=\left(\sum_{i=1}^{n}\left|x_{i}\right|^{2}\right)^{\frac{1}{2}}
$$

根据 Hölder 不等式

$$
\begin{aligned}
\|A x\|_{2}^{2} & =\sum_{i=1}^{m}\left|\sum_{j=1}^{n} a_{i j} x_{j}\right|^{2} \\
& \leqslant \sum_{i=1}^{m}\left[\left(\sum_{j=1}^{n}\left|a_{i j}\right|^{2}\right)\left(\sum_{j=1}^{n}\left|x_{j}\right|^{2}\right)\right]
\end{aligned}
$$

$$
=\left(\sum_{i=1}^{m} \sum_{j=1}^{n}\left|a_{i j}\right|^{2}\right)\left(\sum_{j=1}^{n}\left|x_{j}\right|^{2}\right)=\|\boldsymbol{A}\|_{F}^{2}\|\boldsymbol{x}\|_{2}^{2}
$$

于是 $\quad\|\boldsymbol{A} \boldsymbol{x}\|_{2} \leqslant\|\boldsymbol{A}\|_{F}\|\boldsymbol{x}\|_{2}$\\
定理 5.3.1 设 $\|x\|_{\alpha}$ 是向量范数,则


\begin{equation*}
\|A\|_{i}=\max _{x \neq 0} \frac{\|A x\|_{\alpha}}{\|x\|_{\alpha}} \tag{5.3.2}
\end{equation*}


满足矩阵范数定义,且 $\|\boldsymbol{A}\|_{i}$ 是与向量范数 $\|\boldsymbol{x}\|_{\alpha}$ 相容的矩阵范数。\\
证明 非负性、齐次性显然。\\
根据向量范数三角不等式可得

$$
\begin{aligned}
\|A+B\|_{i} & =\max _{x \neq 0} \frac{\|(A+B) x\|_{\alpha}}{\|x\|_{\alpha}}=\max _{x \neq 0} \frac{\|A x+B x\|_{\alpha}}{\|x\|_{\alpha}} \\
& \leqslant \max _{x \neq 0} \frac{\|A x\|_{\alpha}+\|B x\|_{\alpha}}{\|x\|_{\alpha}} \\
& =\max _{x \neq 0} \frac{\|A x\|_{\alpha}}{\|x\|_{\alpha}}+\max _{x \neq 0} \frac{\|B x\|_{\alpha}}{\|x\|_{\alpha}} \\
& =\|A\|_{i}+\|B\|_{i}
\end{aligned}
$$

现证矩阵范数的相容性.设 $\boldsymbol{B} \neq 0$ ,则

$$
\begin{aligned}
\|A B\|_{i} & =\max _{x \neq 0} \frac{\|A B x\|_{\alpha}}{\|x\|_{\alpha}}=\max _{x \neq 0}\left(\frac{\|A(B x)\|_{\alpha}}{\|B x\|_{\alpha}} \frac{\|B x\|_{\alpha}}{\|x\|_{\alpha}}\right) \\
& \leqslant \max _{x \neq 0} \frac{\|A(B x)\|_{\alpha}}{\|B x\|_{\alpha}} \max _{x \neq 0} \frac{\|B x\|_{\alpha}}{\|x\|_{\alpha}} \\
& =\|A\|_{i}\|B\|_{i}
\end{aligned}
$$

因此 $\|\boldsymbol{A}\|_{i}$ 确实是矩阵范数。最后证明 $\|\boldsymbol{A}\|_{i}$ 与 $\|\boldsymbol{x}\|_{\alpha}$ 相容。\\
由式(5.3.2)得

此即

$$
\begin{gathered}
\|A\|_{i} \geqslant \frac{\|A x\|_{\alpha}}{\|x\|_{\alpha}} \\
\|A x\|_{\alpha} \leqslant\|A\|_{i}\|x\|_{\alpha}
\end{gathered}
$$

这表明 $\|\boldsymbol{A}\|_{i}$ 与 $\|\boldsymbol{x}\|_{\alpha}$ 相容。\\
定义5.3.2 由式(5.3.2)所定义的矩阵范数称为由向量范数 $\|\boldsymbol{x}\|_{\alpha}$ 所诱导的诱导范数,也称算子范数。

显然,单位矩阵 $\boldsymbol{E}$ 的(任何)诱导范数 $\|\boldsymbol{E}\|_{i}=1$ ,其他矩阵范数就不一定有此性质。

由向量 $p$-范数 $\|\boldsymbol{x}\|_{p}$ 所诱导的矩阵范数称为矩阵 $\boldsymbol{p}$-范数,即

$$
\|\boldsymbol{A}\|_{p}=\max _{x \neq 0} \frac{\|\boldsymbol{A} \boldsymbol{x}\|_{p}}{\|\boldsymbol{x}\|_{p}}
$$

常用的矩阵 $p$-范数为 $\|\boldsymbol{A}\|_{1},\|\boldsymbol{A}\|_{2}$ 与 $\|\boldsymbol{A}\|_{\infty}$ 。如何计算这三个范数有下面定

理.\\
定理 5.3.2 设 $A=\left(a_{i j}\right)_{m \times n}$ ,则\\
(1)$\|\boldsymbol{A}\|_{1}=\max _{j}\left(\sum_{i=1}^{m}\left|a_{i j}\right|\right) \quad(j=1,2, \cdots, n)$\\
称 $\|\boldsymbol{A}\|_{1}$ 是列和范数.\\
(2)$\|\boldsymbol{A}\|_{2}=\max _{j}\left(\lambda_{\mathrm{j}}\left(\boldsymbol{A}^{\mathrm{H}} \boldsymbol{A}\right)\right)^{\frac{1}{2}}, \lambda_{\mathrm{j}}\left(\boldsymbol{A}^{\mathrm{H}} \boldsymbol{A}\right)$ 表示矩阵 $\boldsymbol{A}^{\mathrm{H}} \boldsymbol{A}$ 的第 $j$ 个特征值。称 $\|\boldsymbol{A}\|_{2}$ 是谱范数.即 $\|\boldsymbol{A}\|_{2}$ 是 $\boldsymbol{A}$ 的最大正奇异值.\\
(3)$\|\boldsymbol{A}\|_{\infty}=\max _{i}\left(\sum_{j=1}^{n}\left|a_{i j}\right|\right) \quad(i=1,2, \cdots, m)$\\
称 $\|\boldsymbol{A}\|_{\infty}$ 是行和范数.\\
证明(1)命 $w=\max _{j}\left(\sum_{i=1}^{m}\left|a_{i j}\right|\right)$\\
设 $A=\left(\alpha_{1}, \alpha_{2}, \cdots, \alpha_{n}\right), x=\left(x_{1}, x_{2}, \cdots, x_{n}\right)^{\mathrm{T}}$ ,则

$$
w=\max _{j}\left\|\alpha_{j}\right\|_{1}
$$

且 $\quad\|A x\|_{1}=\left\|x_{1} \alpha_{1}+x_{2} \alpha_{2}+\cdots+x_{n} \alpha_{n}\right\|_{1}$

$$
\begin{aligned}
& \leqslant\left|x_{1}\right|\left\|\alpha_{1}\right\|_{1}+\left|x_{2}\right|\left\|\alpha_{2}\right\|_{1}+\cdots+\left|x_{n}\right|\left\|\alpha_{n}\right\|_{1} \\
& \leqslant\left(\left|x_{1}\right|+\left|x_{2}\right|+\cdots+\left|x_{n}\right|\right) w=\|x\|_{1} w
\end{aligned}
$$

故

$$
\|A\|_{1}=\max _{x \neq 0} \frac{\|A x\|_{1}}{\|x\|_{1}} \leqslant w
$$

另一方面,设

$$
\sum_{i=1}^{m}\left|a_{i r}\right|=w
$$

取 $x_{r}=(0,0, \cdots, 1,0, \cdots, 0)^{\mathrm{T}}$ ,第 $r$ 个位置为 1 ,则 $\left\|x_{r}\right\|_{1}=1$ ,且

$$
\|A\|_{1} \geqslant\left\|A x_{r}\right\|_{1}=\sum_{i=1}^{m}\left|a_{i r}\right|=w
$$

合并以上结果,有

$$
\|A\|_{1}=\max \left(\sum_{i=1}^{m}\left|a_{i j}\right|\right) \quad(j=1,2, \cdots, n)
$$

(2)当 $x \neq 0$ 时

$$
\frac{\|A x\|_{2}}{\|x\|_{2}}=\frac{\left(x^{\mathrm{H}} A^{\mathrm{H}} A x\right)^{\frac{1}{2}}}{\left(x^{\mathrm{H}} x\right)^{\frac{1}{2}}}
$$

由于 $\boldsymbol{A}^{\mathrm{H}} \boldsymbol{A}$ 是 Hermite 矩阵,根据 Rayleigh 商知

$$
\max _{x \neq 0} \frac{\|A x\|_{2}}{\|x\|_{2}}=\max _{j}\left(\lambda_{j}\left(A^{\mathrm{H}} A\right)\right)^{\frac{1}{2}}
$$

(3)证略.

显然,若 $\boldsymbol{A}$ 是正规矩阵,则 $\|\boldsymbol{A}\|_{2}=\max _{i}\left|\lambda_{i}(\boldsymbol{A})\right|$ 。\\
从一个向量范数可按式(5.3.2)诱导出一个算子范数,这两个范数是相容的。而一个矩阵范数虽然不是某个向量范数所诱导的,但它们之间也有可能是相容的 (见例5.3.1)。

若已给矩阵范数,可以构造向量范数使这两个范数是相容的,这就是下述定理。

定理 5.3.3 设 $\|\boldsymbol{A}\|_{*}$ 是矩阵范数,则存在向量范数 $\|\boldsymbol{x}\|$ ,满足

$$
\|\boldsymbol{A} \boldsymbol{x}\| \leqslant\|\boldsymbol{A}\|_{*}\|\boldsymbol{x}\|
$$

证明 任给非零向量 $\boldsymbol{\alpha}$ ,定义向量范数 $\|\boldsymbol{x}\|=\left\|\boldsymbol{x} \boldsymbol{\alpha}^{\mathrm{H}}\right\|_{*}$ .不难验证它满足向量范数三个性质,且

$$
\|A x\|=\left\|A x \alpha^{\mathrm{H}}\right\|_{*} \leqslant\|A\|_{*}\left\|x \alpha^{\mathrm{H}}\right\|_{*}=\|A\|_{*}\|x\|
$$

例 5.3 . 2 已知矩阵范数

$$
\|A\|_{*}=\|A\|_{F}=\left(\sum_{i=1}^{m} \sum_{j=1}^{n}\left|a_{i j}\right|^{2}\right)^{\frac{1}{2}}
$$

求与之相容的一个向量范数.\\
解 取 $\boldsymbol{\alpha}=(1,0, \cdots, 0)^{\mathrm{T}}$ ,设 $\boldsymbol{x}=\left(x_{1}, x_{2}, \cdots, x_{n}\right)^{\mathrm{T}}$ ,则

$$
\|x\|=\left\|x \alpha^{\mathrm{H}}\right\|_{*}=\left(\sum_{i=1}^{n}\left|x_{i}\right|^{2}\right)^{\frac{1}{2}}=\|x\|_{2}
$$

根据定理 5.3.3 的证明可知,满足定理 5.3.3 的向量范数不止一个。\\
范数是矩阵理论的一个重要概念,在许多方面有广泛的应用。\\
定义5.3.3 设 $A \in C^{n \times n}, A$ 的 $n$ 个特征值为 $\lambda_{1}, \lambda_{2}, \cdots, \lambda_{n}$ ,称 $\rho(A)=\max \left\{\left|\lambda_{1}\right|,\left|\lambda_{2}\right|, \cdots,\left|\lambda_{n}\right|\right\}$ 是 $A$ 的谱半径。

定理 5.3.4 $A \in C^{n \times n}$ ,则

$$
\rho(A) \leqslant\|A\|
$$

其中 $\|\boldsymbol{A}\|$ 是 $\boldsymbol{A}$ 的任何一种范数。\\
证明 设 $\lambda$ 是 $\boldsymbol{A}$ 的任何一个特征值,即

$$
\boldsymbol{A} \boldsymbol{x}=\lambda \boldsymbol{x} \quad x \neq 0
$$

故

$$
\|\lambda x\|=|\lambda|\|x\|=\|A x\| \leqslant\|A\|\|x\|
$$

于是

$$
|\lambda| \leqslant\|\boldsymbol{A}\|
$$

由于 $\lambda$ 是 $\boldsymbol{A}$ 的任一个特征值,故

$$
\rho(A) \leqslant\|A\|
$$

定理 5.3.5 设 $\boldsymbol{A}$ 是正规矩阵,则

$$
\rho(\boldsymbol{A})=\|\boldsymbol{A}\|_{2}
$$

证明 因为

$$
\|A\|_{2}^{2}=\max _{x \neq 0} \frac{\|A x\|_{2}^{2}}{\|x\|_{2}^{2}}=\max _{x \neq 0} \frac{x^{\mathrm{H}} A^{\mathrm{H}} A x}{x^{\mathrm{H}} x}=\rho\left(A^{\mathrm{H}} A\right)=\rho^{2}(A)
$$

故

$$
\rho(\boldsymbol{A})=\|\boldsymbol{A}\|_{2}
$$

定理 5.3.6 若 $\|\boldsymbol{A}\|<1$ ,则 $\boldsymbol{E} \pm \boldsymbol{A}$ 都为非奇异,且

$$
\frac{1}{1+\|A\|} \leqslant\left\|(E \pm A)^{-1}\right\| \leqslant \frac{1}{1-\|A\|}
$$

其中范数 $\|\boldsymbol{A}\|$ 是矩阵 $\boldsymbol{A}$ 的算子范数。\\
证明 因为把 $\boldsymbol{A}$ 用- $\boldsymbol{A}$ 替换时 $\boldsymbol{E}+\boldsymbol{A}$ 变成 $\boldsymbol{E}-\boldsymbol{A}$ ,故只对 $\boldsymbol{E}+\boldsymbol{A}$ 证明不等式。\\
设 $\boldsymbol{x}$ 为非零向量,则

$$
\begin{aligned}
\|(E+A) x\| & =\|x+A x\| \\
& =\|x-(-A x)\| \\
& \geqslant|\|x\|-\|-A x\|| \\
& =|\|x\|-\|A x\|| \\
& \geqslant|\|x\|-\|A\|\|x\|| \\
& =|(1-\|A\|)\|x\||
\end{aligned}
$$

因为 $\|\boldsymbol{A}\|<1$ ,故 $1-\|\boldsymbol{A}\|>0$ ,于是 $\|(\boldsymbol{E}+\boldsymbol{A}) \boldsymbol{x}\|>0$ 。从而方程 $(\boldsymbol{E}+\boldsymbol{A}) \boldsymbol{x}=0$没有非零解, $\boldsymbol{E}+\boldsymbol{A}$ 非奇异。又因

$$
(\boldsymbol{E}+\boldsymbol{A})^{-1}=\boldsymbol{E}-\boldsymbol{A}(\boldsymbol{E}+\boldsymbol{A})^{-1}
$$

故

$$
\begin{aligned}
\left\|(\boldsymbol{E}+\boldsymbol{A})^{-1}\right\| & =\left\|\boldsymbol{E}-\boldsymbol{A}(\boldsymbol{E}+\boldsymbol{A})^{-1}\right\| \\
& \leqslant\|\boldsymbol{E}\|+\left\|-\boldsymbol{A}(\boldsymbol{E}+\boldsymbol{A})^{-1}\right\| \\
& =1+\left\|\boldsymbol{A}(\boldsymbol{E}+\boldsymbol{A})^{-1}\right\| \\
& \leqslant 1+\|\boldsymbol{A}\|\left\|(\boldsymbol{E}+\boldsymbol{A})^{-1}\right\|
\end{aligned}
$$

所以

$$
\left\|(E+A)^{-1}\right\| \leqslant \frac{1}{1-\|A\|}
$$

又因

$$
(\boldsymbol{E}+\boldsymbol{A})^{-1}(\boldsymbol{E}+\boldsymbol{A})=\boldsymbol{E}
$$

故

$$
\begin{aligned}
& 1=\|E\|=\left\|(E+A)^{-1}(E+A)\right\| \\
& \leqslant\left\|(E+A)^{-1}\right\|\|(E+A)\| \\
& \leqslant\left\|(E+A)^{-1}\right\|(1+\|A\|)
\end{aligned}
$$

所以

$$
\left\|(E+A)^{-1}\right\| \geqslant \frac{1}{1+\|A\|}
$$

\section*{§5.4 矩阵序列与极限}
定义5.4.1 设矩阵序列 $\left\{\boldsymbol{A}_{k}\right\}$ ,其中 $\boldsymbol{A}_{k}=\left(\boldsymbol{a}_{i j}^{(k)}\right) \in C^{m \times n}$ ,若 $m \times n$ 个数列 $\left\{a_{i j}^{(k)}\right\}(i=1,2, \cdots, m ; j=1,2, \cdots, n)$ 都收敛,便称矩阵序列 $\left\{\boldsymbol{A}_{k}\right\}$ 收敛。若 $\lim _{k \rightarrow \infty} a_{i j}^{(k)}= a_{i j}$ ,则 $\lim _{k \rightarrow \infty} A_{k}=A=\left(a_{i j}\right)$ ,称 $A$ 为矩阵序列 $\left\{A_{k}\right\}$ 的极限。

若把向量看成矩阵的特例,向量序列收敛的定义类似可得。

定理 5.4.1 矩阵序列 $\left\{\boldsymbol{A}_{k}\right\}$ 收敛于 $\boldsymbol{A}$ 的充要条件是 $\lim _{k \rightarrow \infty}\left\|\boldsymbol{A}_{k}-\boldsymbol{A}\right\|=0$ .其中矩阵范数 $\left\|\boldsymbol{A}_{k}-\boldsymbol{A}\right\|$ 为任何一种范数。

证明 取矩阵范数 $\|\boldsymbol{A}\|=\sum_{i=1}^{m} \sum_{j=1}^{n}\left|a_{i j}\right|$\\
必要性:设 $\lim _{k \rightarrow \infty} A_{k}=A=\left(a_{i j}\right)$ ,由定义知,对于每一个 $i, j$ 都有

$$
\lim _{k \rightarrow \infty}\left|a_{i j}^{(k)}-a_{i j}\right|=0(i=1,2, \cdots, m ; j=1,2, \cdots, n)
$$

于是

$$
\lim _{k \rightarrow \infty} \sum_{i=1}^{m} \sum_{j=1}^{n}\left|a_{i j}^{(k)}-a_{i j}\right|=0
$$

此即

$$
\lim _{k \rightarrow \infty}\left\|\boldsymbol{A}_{k}-\boldsymbol{A}\right\|=0
$$

充分性:设 $\lim _{k \rightarrow \infty}\left\|A_{k}-A\right\|=\lim _{k \rightarrow \infty} \sum_{i=1}^{m} \sum_{j=1}^{n}\left|a_{i j}^{(k)}-a_{i j}\right|=0$ ,因此,对于每一个 $i$, $j$ 都有

$$
\lim _{k \rightarrow \infty}\left|a_{i j}^{(k)}-a_{i j}\right|=0
$$

此即

$$
\lim _{k \rightarrow \infty} a_{i j}^{(k)}=a_{i j}
$$

于是

$$
\lim _{k \rightarrow \infty} A_{k}=A
$$

现在已经证明了对于所设的范数时定理成立,若 $\|\boldsymbol{A}\|_{\alpha}$ 是其他某一种范数,则由范数等价性定理知

$$
c_{1}\left\|\boldsymbol{A}_{k}-\boldsymbol{A}\right\| \leqslant\left\|\boldsymbol{A}_{k}-\boldsymbol{A}\right\|_{\alpha} \leqslant c_{2}\left\|\boldsymbol{A}_{k}-\boldsymbol{A}\right\|
$$

由 $\lim _{k \rightarrow \infty}\left\|A_{k}-A\right\|=0$ 便得 $\lim _{k \rightarrow \infty}\left\|A_{k}-A\right\|_{\alpha}=0$ 。因此对任何一种范数定理成立。\\
利用数列收敛的概念和性质容易验证:\\
(1)一个收敛矩阵序列的极限是唯一的\\
(2)设 $\lim _{k \rightarrow \infty} A_{k}=A, ~ \lim _{k \rightarrow \infty} B_{k}=B$ ,则

$$
\lim _{k \rightarrow \infty}\left(a \boldsymbol{A}_{k}+b \boldsymbol{B}_{k}\right)=a \boldsymbol{A}+b \boldsymbol{B} \quad a, b \in \mathbf{C}
$$

(3)设 $\lim _{k \rightarrow \infty} \boldsymbol{A}_{k}=\boldsymbol{A}, \lim _{k \rightarrow \infty} \boldsymbol{B}_{k}=\boldsymbol{B}$ ,其中 $\boldsymbol{A}, \boldsymbol{B} \in C^{n \times n}$ ,则

$$
\lim _{k \rightarrow \infty} \boldsymbol{A}_{k} \boldsymbol{B}_{k}=\boldsymbol{A} \boldsymbol{B}
$$

(4)设 $\lim _{k \rightarrow \infty} A_{k}=A$ ,取 $P, Q \in C^{n \times n}$ 则

$$
\lim _{k \rightarrow \infty} P A_{k} Q=P A Q
$$

(5)设 $\lim _{k \rightarrow \infty} A_{k}=A, A_{k}, A$ 均可逆,则 $\left\{A_{k}^{-1}\right\}$ 也收敛,且 $\lim _{k \rightarrow \infty} A_{k}^{-1}=A^{-1}$\\
证明 设 $\tilde{\boldsymbol{A}}_{k}$ 为 $\boldsymbol{A}_{k}$ 的伴随矩阵,则

$$
\boldsymbol{A}_{k}^{-1}=\frac{\tilde{\boldsymbol{A}}_{k}}{\left|\boldsymbol{A}_{k}\right|}
$$

其中 $\tilde{\boldsymbol{A}}_{k}$ 中的元素是 $\left|\boldsymbol{A}_{k}\right|$ 中元素的代数余子式,它是 $\boldsymbol{A}_{k}$ 中元素的 $n-1$ 次多项式。

因此

$$
\lim _{k \rightarrow \infty} \tilde{A}_{k}=\tilde{A} \text { 和 } \lim _{k \rightarrow \infty}\left|A_{k}\right|=|A| \neq 0
$$

于是

$$
\lim _{k \rightarrow \infty} A_{k}^{-1}=\lim _{k \rightarrow \infty} \frac{\tilde{A}_{k}}{\left|A_{k}\right|}=\frac{\tilde{A}}{|A|}=A^{-1}
$$

下面研究由 $n$ 阶方阵 $\boldsymbol{A}$ 的幂组成的矩阵序列

$$
A, A^{2}, A^{3}, \cdots, A^{k}, \cdots
$$

定理 5.4.2 若对矩阵 $\boldsymbol{A}$ 的某一种范数 $\|\boldsymbol{A}\|<1$ ,则 $\lim _{k \rightarrow \infty} A^{k}=0$ .

证明 由 $\left\|\boldsymbol{A}^{k}\right\| \leqslant\|\boldsymbol{A}\|^{k}$ 即可证得。\\
定理 5.4.3 已知矩阵序列: $\boldsymbol{A}, \boldsymbol{A}^{2}, \cdots, \boldsymbol{A}^{k}, \cdots$ ,则 $\lim _{k \rightarrow \infty} \boldsymbol{A}^{k}=0$ 的充要条件是 $\rho(A)<1$ .

证明 设 $\boldsymbol{A}$ 的 Jordan 标准形

$$
J=\operatorname{diag}\left(J_{1}\left(\lambda_{1}\right), J_{2}\left(\lambda_{2}\right), \cdots, J_{r}\left(\lambda_{r}\right)\right)
$$

其中

$$
J_{i}\left(\lambda_{i}\right)=\left[\begin{array}{ccccc}
\lambda_{i} & 1 & & & \\
& \lambda_{i} & 1 & & \\
& & \ddots & \ddots & \\
& 0 & & \ddots & 1 \\
& & & & \lambda_{i}
\end{array}\right]_{d_{i} \times d_{i}} \quad(i=1,2, \cdots, r)
$$

于是

$$
\boldsymbol{A}^{k}=\boldsymbol{P} \operatorname{diag}\left(\boldsymbol{J}_{1}^{k}\left(\lambda_{1}\right), \boldsymbol{J}_{2}^{k}\left(\lambda_{2}\right), \cdots, \boldsymbol{J}_{r}^{k}\left(\lambda_{r}\right)\right) \boldsymbol{P}^{-1}
$$

显然, $\lim _{k \rightarrow \infty} A^{k}=0$ 的充要条件是 $\lim _{k \rightarrow \infty} J_{i}^{k}\left(\lambda_{\mathrm{i}}\right)=0, i=1,2, \cdots, r$ .又因

\[
\boldsymbol{J}_{i}^{k}\left(\lambda_{i}\right)=\left[\begin{array}{ccccc}
\lambda_{i}^{k} & & \mathrm{C}_{k}^{1} \lambda_{i}^{k-1} & \cdots &  \tag{5.4.1}\\
& \lambda_{i}^{k} & & \mathrm{C}_{k}^{1} \lambda_{i}^{k-1} & \\
& & \ddots & & \ddots \\
& 0 & & \ddots & \\
& & & & \\
{ }_{k}^{i} \lambda_{i}^{k-1} \lambda_{i}^{k-d_{i}+1} \\
\\
\lambda_{i}^{k}
\end{array}\right]_{d_{i} \times d_{i}}
\]

其中

$$
\begin{array}{ll}
\mathrm{C}_{k}^{l}=\frac{k(k-1) \cdots(k-l+1)}{l!} & (\text { 当 } l \leqslant k) \\
\mathrm{C}_{k}^{l}=0 & (\text { 当 } l>k)
\end{array}
$$

于是 $\lim _{k \rightarrow \infty} J_{i}^{k}\left(\lambda_{\mathrm{i}}\right)=0$ 的充要条件是 $\left|\lambda_{\mathrm{i}}\right|<1$ .因此 $\lim _{k \rightarrow \infty} A^{k}=0$ 的充要条件是 $\rho(A)<1$ .\\
例 5.4.1 判别矩阵序列 $\boldsymbol{A}^{k}$ 的敛散性。\\
(1) $\boldsymbol{A}=\left[\begin{array}{ll}1 & 1 \\ 0 & 1\end{array}\right]$\\
(2)$A=\left[\begin{array}{cr}0.9 & 1 \\ 0 & 0.9\end{array}\right]$\\
(3)$A=\left[\begin{array}{rrr}1 & 0 & 0 \\ 0 & 0.9 & 1 \\ 0 & 0 & 0.9\end{array}\right]$\\
(4)$A=\left[\begin{array}{ll}0.3 & 0.8 \\ 0.6 & 0.1\end{array}\right]$

解(1) $\boldsymbol{A}^{k}=\left[\begin{array}{ll}1 & k \\ 0 & 1\end{array}\right]$ ,故 $\lim _{k \rightarrow \infty} \boldsymbol{A}^{k}$ 发散。\\
(2) $\boldsymbol{A}$ 的特征值 $\lambda_{1}=\lambda_{2}=0.9<1$ ,故 $\boldsymbol{A}^{k}$ 收敛,且 $\lim _{k \rightarrow \infty} \boldsymbol{A}^{k}=0$ .\\
(3)$A^{k}=\left[\begin{array}{llc}1 & 0 & 0 \\ 0 & 0.9^{k} & k 0.9^{k-1} \\ 0 & 0 & 0.9^{k}\end{array}\right]$\\
由于 $\lim _{k \rightarrow \infty} 0.9^{k}=0, \lim _{k \rightarrow \infty} k 0.9^{k-1}=0$ .\\
故 $\boldsymbol{A}^{k}$ 收敛,且

$$
\lim _{k \rightarrow \infty} A^{k}=\left[\begin{array}{lll}
1 & 0 & 0 \\
0 & 0 & 0 \\
0 & 0 & 0
\end{array}\right]
$$

(4)由于 $\|\boldsymbol{A}\|_{1}=0.9<1$ ,由定理 5.4.2 知 $\boldsymbol{A}^{k}$ 收敛,且 $\lim _{k \rightarrow \infty} \boldsymbol{A}^{k}=0$\\
例 5.4 .2 判别矩阵序列 $A^{k}$ 的敛散性:\\
(1) $\boldsymbol{A}=\left[\begin{array}{ccc}1 & 1 & 0 \\ 0 & 1 & 0 \\ 0 & -1 & 1\end{array}\right]$\\
(2)$A=\left[\begin{array}{ccc}\frac{7}{3} & \frac{1}{3} & -\frac{4}{3} \\ 0 & 1 & 0 \\ -\frac{2}{3} & \frac{1}{3} & \frac{5}{3}\end{array}\right]$\\
(3) $\boldsymbol{A}=\left[\begin{array}{ccc}\frac{2}{3} & 0 & -\frac{1}{3} \\ 0 & 1 & 0 \\ -\frac{1}{6} & 0 & \frac{5}{6}\end{array}\right]$\\
(4) $\boldsymbol{A}=\left[\begin{array}{ccc}3 & 1 & -4 \\ 0 & 1 & 0 \\ -2 & 1 & 5\end{array}\right]$

解(1) $\boldsymbol{A}$ 的特征值 $\lambda_{1}=\lambda_{2}=\lambda_{3}=1, \rho(\boldsymbol{A})=1$ ,若 $\mathbf{J}$ 是 $\mathbf{A}$ 的 Jordan 标准形,由于 $\boldsymbol{A}^{\mathbf{k}}=\boldsymbol{P} \boldsymbol{J}^{\mathbf{k}} \boldsymbol{P}^{-\mathbf{1}}$ ,所以只需判别 $\boldsymbol{J}^{\mathbf{k}}$ 的玫散性即可.当 $\lambda=1$ 时 $\boldsymbol{A}$ 的特征向量 $\alpha_{1}= (1,0,0)^{\mathrm{T}}, \alpha_{2}=(0,0,1)^{\mathrm{T}}$ ,所以 $J=\left[\begin{array}{lll}1 & & \\ & 1 & 1 \\ & & 1\end{array}\right], J^{k}=\left[\begin{array}{lll}1 & & \\ & 1 & \mathrm{k} \\ & & 1\end{array}\right] \rightarrow \infty$ ,所以 $A^{k}$ 发散.\\
(2) $\boldsymbol{A}$ 的特征值 $\lambda_{1}=\lambda_{2}=1, \lambda_{3}=3, \rho(\boldsymbol{A})=3>1$ ,故 $\boldsymbol{A}^{k}$ 发散。\\
(3) $\boldsymbol{A}$ 的特征值 $\lambda_{1}=\lambda_{2}=1, \lambda_{3}=\frac{1}{2}, \rho(\boldsymbol{A})=1$ ,所以需要判别 $\boldsymbol{J}^{k}$ 的敛散性,\\
当 $\lambda=1$ 时,$A$ 的特征向量 $\alpha_{1}=(0,1,0)^{\mathrm{T}}, \alpha_{2}=(-1,0,1)^{\mathrm{T}}$ ,所以 $J=\left[\begin{array}{lll}1 & & \\ & 1 & \\ & & \frac{1}{2}\end{array}\right]$ ,\\
$\boldsymbol{J}^{k}=\left[\begin{array}{lll}1 & & \\ & 1 & \\ & & \left(\frac{1}{2}\right)^{k}\end{array}\right] \rightarrow\left[\begin{array}{lll}1 & & \\ & 1 & \\ & & 0\end{array}\right]$ ,所以 $\boldsymbol{A}^{k}$ 发散.\\
(4) $\boldsymbol{A}$ 的特征值 $\lambda_{1}=\lambda_{2}=1, \lambda_{3}=-\frac{1}{2}, \rho(\boldsymbol{A})=1$ .所以需要判别 $\boldsymbol{J}^{k}$ 的敛散性.当 $\lambda=1$ 时, $\boldsymbol{A}$ 的特征向量 $\alpha=(1,0,-2)^{\mathrm{T}}$ .所以 $\boldsymbol{J}=\left[\begin{array}{lll}1 & 1 & \\ & 1 & \\ & & -\frac{1}{2}\end{array}\right], \boldsymbol{J}^{k}= \left[\begin{array}{lll}1 & k & \\ & 1 & \\ & & \left(-\frac{1}{2}\right)^{k}\end{array}\right] \rightarrow \infty$ ,所以 $\boldsymbol{A}^{k}$ 发散.

例 5.4 . 3 已知

$$
A=\left[\begin{array}{lll}
1 & 2 & 2 \\
2 & 1 & 2 \\
2 & 2 & 1
\end{array}\right]
$$

求 $\lim _{k \rightarrow \infty}\left(\frac{1}{\rho(\boldsymbol{A})} \boldsymbol{A}\right)^{k}$ .\\
解 由于 $|\lambda \boldsymbol{E}-\boldsymbol{A}|=(\lambda-5)(\lambda+1)^{2}=0$ ,所以 $\boldsymbol{A}$ 有三个特征值 $\lambda_{1}=5, \lambda_{2}= \lambda_{3}=-1$ .又 $\boldsymbol{A}$ 为实对称矩阵,于是一定存在可逆矩阵 $\boldsymbol{P}$ 使得

$$
\boldsymbol{A}=\boldsymbol{P}\left[\begin{array}{rrr}
5 & 0 & 0 \\
0 & -1 & 0 \\
0 & 0 & -1
\end{array}\right] \boldsymbol{P}^{-1}, \quad \boldsymbol{P}=\left[\begin{array}{rrr}
1 & -1 & -1 \\
1 & 1 & 0 \\
1 & 0 & 1
\end{array}\right]
$$

显然 $\rho(A)=5$ .那么

$$
\begin{aligned}
\frac{1}{\rho(\boldsymbol{A})} \boldsymbol{A} & =\boldsymbol{P}\left[\begin{array}{rrr}
1 & 0 & 0 \\
0 & -\frac{1}{5} & 0 \\
0 & 0 & -\frac{1}{5}
\end{array}\right] \boldsymbol{P}^{-1} \\
\left(\frac{1}{\rho(\boldsymbol{A})} \boldsymbol{A}\right)^{k} & =\boldsymbol{P}\left[\begin{array}{rrr}
1 & 0 & 0 \\
0 & \left(-\frac{1}{5}\right)^{k} & 0 \\
0 & 0 & \left(-\frac{1}{5}\right)^{k}
\end{array} \boldsymbol{P}^{-1}\right.
\end{aligned}
$$

从而

$$
\lim _{k \rightarrow \infty}\left(\frac{1}{\rho(\boldsymbol{A})} \boldsymbol{A}\right)^{k}=\boldsymbol{P}\left[\begin{array}{lll}
1 & 0 & 0 \\
0 & 0 & 0 \\
0 & 0 & 0
\end{array}\right] \boldsymbol{P}^{-1}=\left[\begin{array}{lll}
\frac{1}{3} & \frac{1}{3} & \frac{1}{3} \\
\frac{1}{3} & \frac{1}{3} & \frac{1}{3} \\
\frac{1}{3} & \frac{1}{3} & \frac{1}{3}
\end{array}\right]
$$

\section*{§5.5 矩阵冪级数}
\section*{一、矩阵级数}
定义 5.5.1 设 $A_{k}=\left(a_{i j}^{(k)}\right) \in C^{m \times n}$ ,若 $m \times n$ 个常数项级数


\begin{gather*}
\sum_{k=1}^{\infty} a_{i j}^{(k)}=a_{i j}^{(1)}+a_{i j}^{(2)}+\cdots+a_{i j}^{(k)}+\cdots \\
(i=1,2, \cdots, m ; j=1,2, \cdots, n) \tag{5.5.1}
\end{gather*}


都收玫时,称矩阵级数


\begin{equation*}
\sum_{k=1}^{\infty} A_{k}=A_{1}+A_{2}+\cdots+A_{k}+\cdots \tag{5.5.2}
\end{equation*}


收敛.若常数项级数(5.5.1)的和为 $a_{i j}$ ,则矩阵级数(5.5.2)的和为 $\boldsymbol{A}=\left(a_{i j}\right)$ 。\\
若 $m \times n$ 个常数项级数(5.5.1)都绝对收敛,则称矩阵级数(5.5.2)绝对收敛。\\
注:也可以利用矩阵级数的部分和序列极限的概念来定义矩阵级数收敛。即有矩阵级数(5.5.2)的部分和作成的矩阵序列 $\boldsymbol{S}_{k}: \boldsymbol{A}_{1}+\boldsymbol{A}_{2}+\cdots+\boldsymbol{A}_{k}$ ,若矩阵序列 $\left\{\boldsymbol{S}_{k}\right\}$ 收敛于矩阵 $\boldsymbol{A}$ 。则称矩阵级数(5.5.2)收敛且其和为 $\boldsymbol{A}$ ,这个定义与定义 5.5.1是等价的,限于篇幅不介绍证明。

定理 5.5.1 设 $A_{k}=\left(a_{i j}^{(k)}\right) \in C^{m \times n}$ ,则矩阵级数 $\sum_{k=1}^{\infty} A_{k}$ 绝对收敛的充要条件是正项数项级数 $\sum_{k=1}^{\infty}\left\|\boldsymbol{A}_{k}\right\|$ 收敛,其中 $\|\boldsymbol{A}\|$ 为任何一种矩阵范数。

证明 取矩阵范数 $\left\|\boldsymbol{A}_{k}\right\|=\sum_{i=1}^{m} \sum_{j=1}^{n}\left|a_{i j}^{(k)}\right|$ ,对于每一个 $i, j$ 都有

$$
\left\|\boldsymbol{A}_{k}\right\| \geqslant\left|a_{i j}^{(k)}\right|
$$

因此,若 $\sum_{k=1}^{\infty}\left\|A_{k}\right\|$ 收敛,则对于每一个 $i, j$ ,常数项级数 $\sum_{k=1}^{\infty}\left|a_{i j}^{(k)}\right|$ 都收敛,于是 $\sum_{k=1}^{\infty} A_{k}$ 绝对收敛。

反之,若矩阵级数 $\sum_{k=1}^{\infty} A_{k}$ 绝对收玫,则对于每一个 $i, j$ 都有 $\sum_{k=1}^{\infty}\left|a_{i j}^{(k)}\right|<\infty$ .于是

$$
\sum_{k=1}^{\infty}\left\|A_{k}\right\|=\sum_{k=1}^{\infty} \sum_{i=1}^{m} \sum_{j=1}^{n}\left|a_{i j}^{(k)}\right|=\sum_{i=1}^{m} \sum_{j=1}^{n} \sum_{k=1}^{\infty}\left|a_{i j}^{(k)}\right|<\infty
$$

根据范数等价性定理知此结论对任何一种范数都正确。\\
注 由定理 5.5.1 可以看出,$\sum_{k=1}^{\infty}\left\|\boldsymbol{A}_{k}\right\|$ 收敛等价于矩阵级数 $\sum_{k=1}^{\infty} \boldsymbol{A}_{k}$ 绝对收敛,不少书上用此定义矩阵级数绝对收敛。

定理 5.5.2 设 $C^{n \times n}$ 中的两个矩阵级数

$$
S_{1}: A_{1}+A_{2}+\cdots+A_{k}+\cdots ; \quad S_{2}: B_{1}+B_{2}+\cdots+B_{k}+\cdots
$$

都绝对收敛,其和分别为 $\boldsymbol{A}, \boldsymbol{B}$ .这时将 $\boldsymbol{S}_{1}$ 和 $\boldsymbol{S}_{2}$ 按项相乘(称柯西乘积)后所组成的矩阵级数


\begin{align*}
S_{3}: & A_{1} B_{1}+\left(A_{1} B_{2}+A_{2} B_{1}\right)+\left(A_{1} B_{3}+A_{2} B_{2}+A_{3} B_{1}\right) \\
& +\cdots+\left(A_{1} B_{k}+A_{2} B_{k-1}+\cdots+A_{k} B_{1}\right)+\cdots \tag{5.5.3}
\end{align*}


绝对收敛,且其和为 $A B$ 。\\
与矩阵级数(5.5.3)相应的范数级数为


\begin{align*}
& \left\|A_{1} B_{1}\right\|+\left\|A_{1} B_{2}+A_{2} B_{1}\right\|+\left\|A_{1} B_{3}+A_{2} B_{2}+A_{3} B_{1}\right\| \\
& +\cdots+\left\|A_{1} B_{k}+A_{2} B_{k-1}+\cdots+A_{k} B_{1}\right\|+\cdots \tag{5.5.4}
\end{align*}


对此范数级数,运用矩阵范数三角不等式、乘法相容性可得范数级数不大于如下级数


\begin{align*}
& \left\|A_{1}\right\|\left\|B_{1}\right\|+\left(\left\|A_{1}\right\|\left\|B_{2}\right\|+\left\|A_{2}\right\|\left\|B_{1}\right\|\right)+ \\
& \left(\left\|A_{1}\right\|\left\|B_{3}\right\|+\left\|A_{2}\right\|\left\|B_{2}\right\|+\left\|A_{3}\right\|\left\|B_{1}\right\|\right)+\cdots+ \\
& \left(\left\|A_{k}\right\|\left\|B_{1}\right\|+\left\|A_{2}\right\|\left\|B_{k-1}\right\|+\cdots+\left\|A_{1}\right\|\left\|B_{k}\right\|\right)+\cdots \tag{5.5.5}
\end{align*}


根据定理 5.5.1 知 $\sum_{k=1}^{\infty}\left\|\boldsymbol{A}_{k}\right\|$ 与 $\sum_{k=1}^{\infty}\left\|\boldsymbol{B}_{k}\right\|$ 收敛,于是根据常数项级数理论知级数式(5.5.5)收敛,因此级数(5.5.4)收敛,于是由定理 5.5.1知矩阵级数(5.5.3)绝对收敛。并且和为 $A B$ ,可类似证明(证略)。

\section*{二、矩阵幂级数}
下面对矩阵幂级数作深入讨论,它是研究矩阵函数的重要工具.\\
定义 5.5.2 设 $\boldsymbol{A}=\left(a_{i j}\right) \in C^{n \times n}$ ,称形如

$$
\sum_{k=0}^{\infty} c_{k} A^{k}=c_{0} E+c_{1} A+c_{2} A^{2}+\cdots+c_{k} A^{k}+\cdots
$$

的矩阵级数为矩阵霫级数。\\
把定理5.5.1运用到幂级数上,便得\\
定理 5.5.3 若矩阵 $\boldsymbol{A}$ 的某一种范数 $\|\boldsymbol{A}\|$ 在幂级数

$$
c_{0}+c_{1} x+c_{2} x^{2}+\cdots+c_{k} x^{k}+\cdots
$$

的收玫域内,则矩阵幂级数

$$
c_{0} \boldsymbol{E}+c_{1} \boldsymbol{A}+c_{2} \boldsymbol{A}^{2}+\cdots+c_{k} \boldsymbol{A}^{k}+\cdots
$$

绝对收敛。\\
例 5.5.1 设

$$
A=\left[\begin{array}{lll}
0.2 & 0.5 & 0.1 \\
0.1 & 0.5 & 0.3 \\
0.2 & 0.4 & 0.2
\end{array}\right]
$$

则 $\boldsymbol{E}+\boldsymbol{A}+\boldsymbol{A}^{2}+\cdots+\boldsymbol{A}^{k}+\cdots$ .绝对收敛.\\
解 因为级数 $1+x+x^{2}+\cdots+x^{k}+\cdots$ 的收敛半径为 1 ,而 $\|A\|_{\infty}=0.9<1$ ,故矩阵幂级数 $\boldsymbol{E}+\boldsymbol{A}+\boldsymbol{A}^{2}+\cdots+\boldsymbol{A}^{k}+\cdots$ 绝对收敛.

定理 5.5.4 设幂级数 $\sum_{k=0}^{\infty} c_{k} x^{k}$ 的收敛半径为 $R, A$ 为 $n$ 阶方阵。若 $\rho(A)< R$ ,则矩阵幂级数 $\sum_{k=0}^{\infty} c_{k} A^{k}$ 绝对收敛;若 $\rho(A)>R$ ,则 $\sum_{k=0}^{\infty} c_{k} A^{k}$ 发散。

证明 设 $\boldsymbol{J}$ 是 $\boldsymbol{A}$ 的 Jordan 标准形,则

$$
A=P J P^{-1}=P \operatorname{diag}\left(J_{1}\left(\lambda_{1}\right), J_{2}\left(\lambda_{2}\right), \cdots, J_{r}\left(\lambda_{r}\right)\right) P^{-1}
$$

其中

$$
J_{i}=\left[\begin{array}{ccccc}
\lambda_{i} & 1 & & & \\
& \lambda_{i} & 1 & & \\
& & \ddots & \ddots & \\
& & & \ddots & 1 \\
& & & & \lambda_{i}
\end{array}\right]_{d_{i} \times d_{i}}
$$

于是

$$
\boldsymbol{A}^{k}=P \operatorname{diag}\left(\boldsymbol{J}_{1}^{k}\left(\lambda_{1}\right), \boldsymbol{J}_{2}^{k}\left(\lambda_{2}\right), \cdots, \boldsymbol{J}_{r}^{k}\left(\lambda_{r}\right)\right) \boldsymbol{P}^{-1}
$$

\[
J_{i}^{k}\left(\lambda_{i}\right)=\left[\begin{array}{cccc}
\lambda_{i}^{k} & \mathbb{C}_{k}^{1} \lambda_{i}^{k-1} & \cdots & \mathbb{C}_{k}^{d_{i}-1} \lambda_{i}^{k-d_{i}+1}  \tag{5.5.6}\\
& \lambda_{i}^{k} & & \vdots \\
& & \ddots & \\
& & & \mathbb{C}_{k}^{1} \lambda_{i}^{k-1} \\
& & & \lambda_{i}^{k}
\end{array}\right]_{d_{i} \times d_{i}}
\]

所以


\begin{align*}
\sum_{k=0}^{\infty} c_{k} \boldsymbol{A}^{k} & =\sum_{k=0}^{\infty} c_{k}\left(\boldsymbol{P} \boldsymbol{J}^{k} \boldsymbol{P}^{-1}\right)=\boldsymbol{P}\left(\sum_{k=0}^{\infty} c_{k} \boldsymbol{J}^{k}\right) \boldsymbol{P}^{-1} \\
& =\boldsymbol{P} \operatorname{diag}\left(\sum_{k=0}^{\infty} c_{k} \boldsymbol{J}_{1}^{k}\left(\lambda_{1}\right), \sum_{k=0}^{\infty} c_{k} \boldsymbol{J}_{2}^{k}\left(\lambda_{2}\right), \cdots, \sum_{k=0}^{\infty} c_{k} \boldsymbol{J}_{r}^{k}\left(\lambda_{r}\right)\right) \boldsymbol{P}^{-1} \tag{5.5.7}
\end{align*}


其中

$$
\sum_{k=0}^{\infty} c_{k} J_{i}^{k}\left(\lambda_{\mathrm{i}}\right)
$$

$$
\begin{aligned}
& =\left[\begin{array}{cccc}
\sum_{k=0}^{\infty} c_{k} \lambda_{i}^{k} & \sum_{k=0}^{\infty} c_{k} \mathrm{C}_{k}^{1} \lambda_{i}^{k-1} & \cdots & \sum_{k=0}^{\infty} c_{k} \mathrm{C}_{k}^{d_{i}-1} \lambda_{i}^{k-d_{i}+1} \\
& \ddots & & \vdots \\
& \ddots & \ddots & \\
& & \ddots & \sum_{k=0}^{\infty} c_{k} \mathrm{C}_{k}^{1} \lambda_{i}^{k-1} \\
& & & \sum_{k=0}^{\infty} c_{k} \lambda_{i}^{k}
\end{array}\right]_{d_{i} \times d_{i}} \\
& \mathrm{C}_{k}^{l}=\frac{k(k-1) \cdots(k-l+1)}{l!} \\
& \mathrm{C}_{k}^{l}=0
\end{aligned}
$$

当 $\rho(\boldsymbol{A})<R$ 时,幂级数 $\sum_{k=0}^{\infty} c_{k} \lambda_{i}^{k}, \sum_{k=0}^{\infty} c_{k} \mathrm{C}_{k}^{1} \lambda_{i}^{k-1}, \cdots, \sum_{k=0}^{\infty} c_{k} \mathrm{C}_{k}^{d_{i}-1} \lambda_{i}^{k-d_{i}+1}$ 都绝对收敛,故矩阵幂级数 $\sum_{k=0}^{\infty} c_{k} A^{k}$ 绝对收敛。

当 $\rho(\boldsymbol{A})>R$ 时,幂级数 $\sum_{k=0}^{\infty} c_{k} \lambda_{i}^{k}$ 发散,故 $\sum_{k=0}^{\infty} c_{k} \boldsymbol{A}^{k}$ 发散。\\
例 5.5 .2 由于幂级数

$$
\begin{aligned}
& 1+x+\frac{1}{2!} x^{2}+\cdots+\frac{1}{k!} x^{k}+\cdots \\
& 1-\frac{1}{2!} x^{2}+\frac{1}{4!} x^{4}-\cdots+(-1)^{k} \frac{1}{(2 k)!} x^{2 k}+\cdots \\
& x-\frac{1}{3!} x^{3}+\frac{1}{5!} x^{5}-\cdots+(-1)^{k} \frac{1}{(2 k+1)!} x^{2 k+1}+\cdots
\end{aligned}
$$

的收敛半径 $R=\infty$ ,所以对于任意 $n$ 阶矩阵 $\boldsymbol{A}$ ,矩阵幂级数

$$
\begin{aligned}
& E+A+\frac{1}{2!} A^{2}+\cdots+\frac{1}{k!} A^{k}+\cdots \\
& E-\frac{1}{2!} A^{2}+\frac{1}{4!} A^{4}-\cdots+(-1)^{k} \frac{1}{(2 k)!} A^{2 k}+\cdots \\
& A-\frac{1}{3!} A^{3}+\frac{1}{5!} A^{5}-\cdots+(-1)^{k} \frac{1}{(2 k+1)!} A^{2 k+1}+\cdots
\end{aligned}
$$

都绝对收敛。

\section*{定理 5.5 .5 矩阵幂级数}
$$
E+A+A^{2}+\cdots+A^{k}+\cdots
$$

绝对收敛的充要条件是 $\rho(A)<1$ .且其和为 $(E-A)^{-1}$ .\\
证明 幂级数 $1+x+x^{2}+\cdots+x^{k}+\cdots$ 的收敛半径 $R=1$ .故当 $\rho(A)<1$ 时, $\boldsymbol{E}+\boldsymbol{A}+\boldsymbol{A}^{2}+\cdots+\boldsymbol{A}^{k}+\cdots$ 绝对收玫.反之,若所给矩阵幂级数绝对收敛,则 $\|\boldsymbol{E}\|+$\\
$\|A\|+\left\|A^{2}\right\|+\cdots+\left\|A^{k}\right\|+\cdots$ 绝对收敛,故 $\left\|A^{k}\right\| \rightarrow 0, A^{k} \rightarrow 0$ ,从而 $\rho(A)<1$ .\\
现在来求其和.因为

$$
(\boldsymbol{E}-\boldsymbol{A})\left(\boldsymbol{E}+\boldsymbol{A}+\boldsymbol{A}^{2}+\cdots+\boldsymbol{A}^{k}+\cdots\right)=\boldsymbol{E}
$$

故

$$
\boldsymbol{E}+\boldsymbol{A}+\boldsymbol{A}^{2}+\cdots+\boldsymbol{A}^{k}+\cdots=(\boldsymbol{E}-\boldsymbol{A})^{-1}
$$

例 5.5.3 设

$$
A=\left[\begin{array}{lll}
0.2 & 0.5 & 0.1 \\
0.1 & 0.5 & 0.3 \\
0.2 & 0.4 & 0.2
\end{array}\right]
$$

求 $\boldsymbol{E}+\boldsymbol{A}+\boldsymbol{A}^{2}+\cdots+\boldsymbol{A}^{k}+\cdots$ 的和.\\
解 由定理 5.5.3 知,因 $\|A\|_{\infty}=0.9<1$ .故 $\rho(A)<1$ .由定理 5.5.5知所求矩阵幂级数的和是 $(\boldsymbol{E}-\boldsymbol{A})^{-1}$ .因此

$$
\begin{gathered}
E-A=\left[\begin{array}{rrr}
0.8 & -0.5 & -0.1 \\
-0.1 & 0.5 & -0.3 \\
-0.2 & -0.4 & 0.8
\end{array}\right] \\
(E-A)^{-1}=\left[\begin{array}{lll}
2 & \frac{22}{7} & \frac{10}{7} \\
1 & \frac{31}{7} & \frac{25}{14} \\
1 & 3 & \frac{5}{2}
\end{array}\right]
\end{gathered}
$$

于是

$$
E+A+A^{2}+\cdots+A^{k}+\cdots=\left[\begin{array}{ccc}
2 & \frac{22}{7} & \frac{10}{7} \\
1 & \frac{31}{7} & \frac{25}{14} \\
1 & 3 & \frac{5}{2}
\end{array}\right]
$$

例5.5.4 已知

$$
A=\left[\begin{array}{ll}
-2 & 1 \\
-1 & 0
\end{array}\right]
$$

分别讨论矩阵幂级数 $\sum_{k=1}^{\infty} \frac{1}{k} A^{k}$ 与 $\sum_{k=1}^{\infty} \frac{1}{k^{2}} A^{k}$ 的敛散性。\\
解 首先求得 $|\lambda E-A|=(\lambda+1)^{2}=0$ ,即有两个特征值 $\lambda_{1}=\lambda_{2}=1$ 于是 $A$ 的谱半径 $\rho(A)=1$ .而幂级数 $\sum_{k=1}^{\infty} \frac{1}{k} x^{k}$ 的收敛半径 $R=1$ ,所以不能用定理 5.5.3 来判

断矩阵幂级数 $\sum_{k=1}^{\infty} \frac{1}{k} A^{k}$ 的玫散性,只能用定义来验证其玫散性。求出矩阵 $A$ 的 Jordan 标准形 J 及可逆矩阵 $\boldsymbol{P}$ 且使得

$$
\boldsymbol{P}^{-1} \boldsymbol{A} \boldsymbol{P}=\boldsymbol{J}=\left[\begin{array}{rr}
-1 & 1 \\
0 & -1
\end{array}\right], \quad \boldsymbol{A}^{k}=\boldsymbol{P} \boldsymbol{J}^{k} \boldsymbol{P}^{-1}
$$

对于矩阵幂级数

$$
\begin{aligned}
\sum_{k=1}^{\infty} \frac{1}{k} A^{k} & =P\left(\sum_{k=1}^{\infty} \frac{1}{k} J^{k}\right) P^{-1} \\
& =P\left[\begin{array}{cc}
\sum_{k=1}^{\infty} \frac{(-1)^{k}}{k} & \sum_{k=1}^{\infty}(-1)^{k-1} \\
0 & \sum_{k=1}^{\infty} \frac{(-1)^{k}}{k}
\end{array}\right] P^{-1}
\end{aligned}
$$

容易看出上式右端只有数项级数 $\sum_{k=1}^{\infty}(-1)^{k-1}$ 是发散的,尽管其余位置的数项级数收敛,也导致此矩阵暮级数 $\sum_{k=1}^{\infty} \frac{1}{k} A^{k}$ 发散。

对于矩阵幂级数

$$
\begin{aligned}
\sum_{k=1}^{\infty} \frac{1}{k^{2}} \boldsymbol{A}^{k} & =\boldsymbol{P}\left(\sum_{k=1}^{\infty} \frac{1}{k^{2}} \boldsymbol{J}^{k}\right) \boldsymbol{P}^{-1} \\
& =\boldsymbol{P}\left[\begin{array}{cc}
\sum_{k=1}^{\infty} \frac{(-1)^{k}}{k^{2}} & \sum_{k=1}^{\infty} \frac{(-1)^{k-1}}{k} \\
0 & \sum_{k=1}^{\infty} \frac{(-1)^{k}}{k^{2}}
\end{array}\right] \boldsymbol{P}^{-1}
\end{aligned}
$$

由于幂级数 $\sum_{k=1}^{\infty} \frac{1}{k^{2}} x^{k}$ ,其收玫半径为 $R=1=\rho(A)$ .同样不能用定理 5.5.4 来判断矩阵幂级数 $\sum_{k=1}^{\infty} \frac{1}{k^{2}} A^{k}$ 的玫散性,只能用定义来判断,即

$$
\begin{aligned}
\sum_{k=1}^{\infty} \frac{1}{k^{2}} A^{k} & =P\left(\sum_{k=1}^{\infty} \frac{1}{k^{2}} J^{k}\right) P^{-1} \\
& =P\left[\begin{array}{cc}
\sum_{k=1}^{\infty} \frac{(-1)^{k}}{k^{2}} & \sum_{k=1}^{\infty} \frac{(-1)^{k-1}}{k} \\
0 & \sum_{k=1}^{\infty} \frac{(-1)^{k}}{k^{2}}
\end{array}\right] P^{-1}
\end{aligned}
$$

容易看出上面矩阵序列中四个位置元素所构成的数项级数均收玫。从而矩阵幂级

数 $\sum_{k=1}^{\infty} \frac{1}{k^{2}} A^{k}$ 也收敛。\\
例 5.5 .5 判别矩阵幂级数的敛散性。\\
(1)$\sum_{k=1}^{\infty} \frac{1}{k^{2}}\left[\begin{array}{ll}1 & 1 \\ 0 & 1\end{array}\right]^{k}$\\
(2)$\sum_{k=1}^{\infty} \frac{1}{k^{2}}\left[\begin{array}{rr}-1 & 1 \\ 0 & -1\end{array}\right]^{k}$\\
(3)$\sum_{k=1}^{\infty} \frac{1}{k^{2}}\left[\begin{array}{rrr}-1 & 1 & 0 \\ 0 & -1 & 1 \\ 0 & 0 & -1\end{array}\right]^{k}$\\
(4)$\sum_{k=1}^{\infty} \frac{1}{k^{2}}\left[\begin{array}{rrr}1 & 0 & 0 \\ 0 & -1 & 1 \\ 0 & 0 & -1\end{array}\right]^{k}$

解(1)根据式(5.5.6)与式(5.5.7)得

$$
\begin{aligned}
\sum_{k=1}^{\infty} \frac{1}{k^{2}}\left[\begin{array}{ll}
1 & 1 \\
0 & 1
\end{array}\right]^{k} & =\sum_{k=1}^{\infty} \frac{1}{k^{2}}\left[\begin{array}{ll}
1 & k \\
0 & 1
\end{array}\right] \\
& =\left[\begin{array}{cc}
\sum_{k=1}^{\infty} \frac{1}{k^{2}} & \sum_{k=1}^{\infty} \frac{1}{k} \\
0 & \sum_{k=1}^{\infty} \frac{1}{k^{2}}
\end{array}\right]
\end{aligned}
$$

由于数项级数 $\sum_{k=1}^{\infty} \frac{1}{k^{2}}$ 收玫,$\sum_{k=1}^{\infty} \frac{1}{k}$ 发散,故此矩阵幂级数发散。\\
(2)根据式(5.5.6)与(5.5.7)可得

$$
\begin{aligned}
\sum_{k=1}^{\infty} \frac{1}{k^{2}}\left[\begin{array}{rr}
-1 & 1 \\
0 & -1
\end{array}\right]^{k} & =\sum_{k=1}^{\infty} \frac{1}{k^{2}}\left[\begin{array}{cc}
(-1)^{k} & k(-1)^{k-1} \\
0 & (-1)^{k}
\end{array}\right] \\
& =\left[\begin{array}{cc}
\sum_{k=1}^{\infty} \frac{(-1)^{k}}{k^{2}} & \sum_{k=1}^{\infty} \frac{(-1)^{k-1}}{k} \\
0 & \sum_{k=1}^{\infty} \frac{(-1)^{k}}{k^{2}}
\end{array}\right]
\end{aligned}
$$

由于数项级数 $\sum_{k=1}^{\infty} \frac{(-1)^{k}}{k^{2}}, \sum_{k=1}^{\infty} \frac{(-1)^{k-1}}{k}$ 都收敛,故此矩阵幂级数收敛(不是绝对收敛)。\\
(3)根据式(5.5.6)与式(5.5.7)得

$$
\begin{aligned}
& \sum_{k=1}^{\infty} \frac{1}{k^{2}}\left[\begin{array}{rrr}
-1 & 1 & 0 \\
0 & -1 & 1 \\
0 & 0 & -1
\end{array}\right]^{k} \\
= & \sum_{k=1}^{\infty} \frac{1}{k^{2}}\left[\begin{array}{ccc}
(-1)^{k} & k(-1)^{k-1} & \frac{1}{2} k(k-1)(-1)^{k-2} \\
0 & (-1)^{k} & k(-1)^{k-1} \\
0 & 0 & (-1)^{k}
\end{array}\right]
\end{aligned}
$$

$$
=\left[\begin{array}{ccc}
\sum_{k=1}^{\infty} \frac{(-1)^{k}}{k^{2}} & \sum_{k=1}^{\infty} \frac{(-1)^{k-1}}{k} & \frac{1}{2} \sum_{k=1}^{\infty} \frac{k-1}{k}(-1)^{k-2} \\
0 & \sum_{k=1}^{\infty} \frac{(-1)^{k}}{k^{2}} & \sum_{k=1}^{\infty} \frac{(-1)^{k-1}}{k} \\
0 & 0 & \sum_{k=1}^{\infty} \frac{(-1)^{k}}{k^{2}}
\end{array}\right]
$$

由于数项级数 $\sum_{k=1}^{\infty} \frac{k-1}{k}(-1)^{k-2}$ 发散。故此矩阵幂级数发散。\\
(4)根据式(5.5.6)与式(5.5.7)得

$$
\begin{aligned}
\sum_{k=1}^{\infty} \frac{1}{k^{2}}\left[\begin{array}{rrr}
1 & 0 & 0 \\
0 & -1 & 1 \\
0 & 0 & -1
\end{array}\right]^{k} & =\sum_{k=1}^{\infty} \frac{1}{k^{2}}\left[\begin{array}{ccc}
1 & 0 & 0 \\
0 & (-1)^{k} & k(-1)^{k-1} \\
0 & 0 & (-1)^{k}
\end{array}\right] \\
& =\left[\begin{array}{ccc}
\sum_{k=1}^{\infty} \frac{1}{k^{2}} & 0 & 0 \\
0 & \sum_{k=1}^{\infty} \frac{(-1)^{k}}{k^{2}} & \sum_{k=1}^{\infty} \frac{(-1)^{k-1}}{k} \\
0 & 0 & \sum_{k=1}^{\infty} \frac{(-1)^{k}}{k^{2}}
\end{array}\right]
\end{aligned}
$$

由于数项级数 $\sum_{k=1}^{\infty} \frac{1}{k^{2}}, \sum_{k=1}^{\infty} \frac{(-1)^{k}}{k^{2}}, \sum_{k=1}^{\infty} \frac{(-1)^{k-1}}{k}$ 都收敛,故此矩阵幂级数收玫(不是绝对收敛).

\section*{§5.6 矩阵的测度}
作为矩阵范数的应用.本节讨论矩阵的测度.\\
定义 5.6.1 设 $A \in C^{n \times n},\|\cdot\|$ 是给定的算子范数。如果极限

$$
\mu(A)=\lim _{x \rightarrow 0_{+}} \frac{\|E+x A\|-1}{x}
$$

存在,那么称 $\mu(\boldsymbol{A})$ 是矩阵 $\boldsymbol{A}$ 关于范数 $\|\cdot\|$ 的测度\\
例 5.6.1 设 $A \in C^{n \times n}$ ,试证矩阵 $A$ 关于算子范数 $\|\cdot\|_{1}$ 和 $\|\cdot\|_{\infty}$ 的测度分别是

$$
\begin{aligned}
& \mu_{1}(A)=\max _{j}\left(\operatorname{Re}\left(a_{i j}\right)+\sum_{\substack{i=1 \\
i \neq j}}^{n}\left|a_{i j}\right|\right) \\
& \mu_{\infty}(A)=\max _{i}\left(\operatorname{Re}\left(a_{i i}\right)+\sum_{\substack{j=1 \\
j \neq i}}^{n}\left|a_{i j}\right|\right)
\end{aligned}
$$

证明 由定义可知

$$
\begin{aligned}
\mu_{1}(A) & =\lim _{x \rightarrow 0+} \frac{\|E+x A\|_{1}-1}{x} \\
& =\lim _{x \rightarrow 0+} \frac{\max _{j}\left(\sum_{i=1}^{n}\left|\delta_{i j}+x a_{i j}\right|\right)-1}{x} \\
& =\lim _{x \rightarrow 0+} \frac{\max _{j}\left(\left|1+x a_{i j}\right|+x \sum_{\substack{i=1 \\
i \neq j}}^{n}\left|a_{i j}\right|\right)-1}{x} \\
& =\lim _{x \rightarrow 0+} \frac{\max _{j}\left(1+x \operatorname{Re}\left(a_{i j}\right)+o(x)+x \sum_{\substack{i=1 \\
i \neq j}}^{n}\left|a_{i j}\right|\right)-1}{x} \\
& =\max _{j}\left(\operatorname{Re}\left(a_{i j}\right)+\sum_{\substack{i=1 \\
i \neq j}}^{n}\left|a_{i j}\right|\right) \\
\mu_{\infty}(A) & =\lim _{x \rightarrow 0+} \frac{\|E+x A\| \infty-1}{x} \\
& =\lim _{x \rightarrow 0+} \frac{\max _{i}\left(\sum_{j=1}^{n}\left|\delta_{i j}+x a_{i j}\right|\right)-1}{x} \\
& =\lim _{x \rightarrow 0+} \frac{\max _{i}\left(\left|1+x a_{i i}\right|+x \sum_{j=1, i \neq j}^{n}\left|a_{i j}\right|\right)-1}{x} \\
& =\lim _{x \rightarrow 0+} \frac{\max _{i}\left(1+x \operatorname{Re}\left(a_{i i}\right)+o(x)+x \sum_{\substack{j=1 \\
j \neq i}}^{n}\left|a_{i j}\right|\right)-1}{x} \\
& =\max _{i}\left(\operatorname{Re}\left(a_{i i}\right)+\sum_{\substack{j=1 \\
i \neq j}}^{n}\left|a_{i j}\right|\right)
\end{aligned}
$$

例 5.6.2 设 $\boldsymbol{A} \in C^{n \times n}$ ,矩阵 $\boldsymbol{A}$ 关于算子范数 $\|\cdot\|_{2}$ 的测度是 $\mu_{2}(\boldsymbol{A})=\max _{i} \lambda_{i}\left(\frac{\boldsymbol{A}+\boldsymbol{A}^{\mathrm{H}}}{2}\right)$ ,其中 $\lambda_{i}\left(\frac{\boldsymbol{A}+\boldsymbol{A}^{\mathrm{H}}}{2}\right)$ 表示矩阵 $\frac{\boldsymbol{A}+\boldsymbol{A}^{\mathrm{H}}}{2}$ 的第 $i$ 个特征值。

证明 由定义可知

$$
\begin{aligned}
\mu_{2}(A) & =\lim _{x \rightarrow 0+} \frac{\|E+x A\|_{2}-1}{x} \\
& =\lim _{x \rightarrow 0+} \frac{\sqrt{\max _{i} \lambda_{i}\left((E+x A)\left(E+x A^{\mathrm{H}}\right)\right)}-1}{x}
\end{aligned}
$$

$$
\begin{aligned}
& =\lim _{x \rightarrow 0+} \frac{\sqrt{\max _{i} \lambda_{i}\left(E+\frac{\boldsymbol{A}+\boldsymbol{A}^{\mathrm{H}}}{2} 2 x+x^{2} \boldsymbol{A} \boldsymbol{A}^{\mathrm{H}}\right)}-1}{x} \\
& =\lim _{x \rightarrow 0+} \frac{\max _{i}\left(1+\lambda_{i}\left(\frac{\boldsymbol{A}+\boldsymbol{A}^{\mathrm{H}}}{2}\right) x+o(x)\right)-1}{x} \\
& =\max _{i} \lambda_{i}\left(\frac{\boldsymbol{A}+\boldsymbol{A}^{\mathrm{H}}}{2}\right)
\end{aligned}
$$

矩阵的测度有如下性质。\\
定理 5.6.1 设 $A \in C^{n \times n},\|\cdot\|$ 为算子范数,那么\\
(1)$\mu(0)=0, \mu(E)=1, \mu(-E)=-1$ ;\\
(2)$\mu(k \boldsymbol{A})=k \mu(\boldsymbol{A}), k \in \mathbf{R}$ 且 $k \geqslant 0$ ;\\
(3)$\mu(k \boldsymbol{E}+\boldsymbol{A})=k+\mu(\boldsymbol{A}), \quad k \in \mathbf{C}$ ;\\
(4)$\mu(A+B) \leqslant \mu(A)+\mu(B), A, B \in C^{n \times n}$ ;\\
(5)$-\|\boldsymbol{A}\| \leqslant-\mu(-\boldsymbol{A}) \leqslant \mu(\boldsymbol{A}) \leqslant\|\boldsymbol{A}\|$ ;\\
(6)$-\mu(-A) \leqslant \operatorname{Re} \lambda_{\mathrm{i}}(A) \leqslant \mu(A), \lambda_{\mathrm{i}}(A)$ 表示矩阵 $A$ 的第 $i$ 个特征值;\\
(7)max $\{-\mu(-\boldsymbol{A}),-\mu(\boldsymbol{A})\}\|\boldsymbol{\alpha}\| \leqslant\|\boldsymbol{A} \boldsymbol{\alpha}\|, \boldsymbol{\alpha} \in C^{n}$ 。\\
证明(1),(2)由矩阵测度的定义立即可得\\
(3)

$$
\begin{aligned}
\mu(k \boldsymbol{E}+\boldsymbol{A}) & =\lim _{x \rightarrow 0+} \frac{\|\boldsymbol{E}+x(k \boldsymbol{E}+\boldsymbol{A})\|-1}{x} \\
& =\lim _{x \rightarrow 0+} \frac{\left\|\boldsymbol{E}+\frac{x}{1+k x} \boldsymbol{A}\right\|-1+\left(1-\frac{1}{1+k x}\right)}{\frac{x}{1+k x}} \\
& =\mu(\boldsymbol{A})+k
\end{aligned}
$$

(4)

$$
\begin{aligned}
\mu(A+B) & =\lim _{x \rightarrow 0+} \frac{\|E+x(A+B)\|-1}{x} \\
& =\lim _{x \rightarrow 0+} \frac{\|(E+2 x A)+(E+2 x B)\|-2}{2 x} \\
& \leqslant \lim _{x \rightarrow 0+}\left(\frac{\|E+2 x A\|-1}{2 x}+\frac{\|E+2 x B\|-1}{2 x}\right) \\
& =\mu(A)+\mu(B)
\end{aligned}
$$

(5)

$$
\begin{aligned}
0 & =\frac{\|2 E+x A-x A\|-2}{x} \\
& \leqslant \frac{\|E+x A\|-1}{x}+\frac{\|E-x A\|-1}{x}
\end{aligned}
$$

从而

$$
\begin{aligned}
-\|A\| & =\frac{-x\|A\|+1-1}{x} \leqslant-\frac{\|E-x A\|-1}{x} \\
& \leqslant \frac{\|E+x A\|-1}{x} \leqslant \frac{\|E\|+x\|A\|-1}{x}=\|A\|
\end{aligned}
$$

于是有

$$
-\|\boldsymbol{A}\| \leqslant-\mu(-\boldsymbol{A}) \leqslant \mu(\boldsymbol{A}) \leqslant\|\boldsymbol{A}\|
$$

(6)任取 $\boldsymbol{\alpha} \in C^{n}$ ,并设 $\boldsymbol{A} \boldsymbol{\alpha}=\lambda_{i} \boldsymbol{\alpha}$ 且 $\|\boldsymbol{\alpha}\|=1$ ,那么

$$
\begin{aligned}
\operatorname{Re} \lambda_{i}(\boldsymbol{A}) & =\operatorname{Re} \lambda_{i}(\boldsymbol{A})+\frac{o(x)}{x}=\frac{1+x \operatorname{Re} \lambda_{i}(\boldsymbol{A})+o(x)-1}{x} \\
& =\frac{\left|1+x \lambda_{i}\right|-1}{x}=\frac{\left\|\left(1+x \lambda_{i}\right) \boldsymbol{\alpha}\right\|-1}{x} \\
& =\frac{\|\boldsymbol{\alpha}+x \boldsymbol{A} \boldsymbol{\alpha}\|-1}{x} \leqslant \frac{\|\boldsymbol{E}+x \boldsymbol{A}\|\|\boldsymbol{\alpha}\|-1}{x} \\
& =\frac{\|\boldsymbol{E}+x \boldsymbol{A}\|-1}{x}
\end{aligned}
$$

于是有

$$
\operatorname{Re} \lambda_{i}(\boldsymbol{A}) \leqslant \mu(\boldsymbol{A})
$$

由此立即可得

$$
-\mu(-\boldsymbol{A}) \leqslant \operatorname{Re} \lambda_{i}(\boldsymbol{A})
$$

(7)

$$
\begin{aligned}
\|\boldsymbol{A} \boldsymbol{\alpha}\| & =\frac{\|(\boldsymbol{E}-x \boldsymbol{A}) \boldsymbol{\alpha}-\boldsymbol{\alpha}\|}{x} \geqslant \frac{\|\boldsymbol{\alpha}\|-\|(\boldsymbol{E}-x \boldsymbol{A}) \boldsymbol{\alpha}\|}{x} \\
& \geqslant \frac{\|\boldsymbol{\alpha}\|-\|\boldsymbol{E}-x \boldsymbol{A}\|\|\boldsymbol{\alpha}\|}{x} \\
& =-\frac{(\|\boldsymbol{E}-x \boldsymbol{A}\|-1)\|\boldsymbol{\alpha}\|}{x}
\end{aligned}
$$

于是有

$$
\|\boldsymbol{A} \boldsymbol{\alpha}\| \geqslant-\mu(-\boldsymbol{A})\|\boldsymbol{\alpha}\| .
$$

由此立即可得

$$
\|\boldsymbol{A} \boldsymbol{\alpha}\|=\|-\boldsymbol{A} \boldsymbol{\alpha}\| \geqslant-\mu(\boldsymbol{A})\|\boldsymbol{\alpha}\| .
$$

\section*{习 题}
5-1 设 $\boldsymbol{\alpha}, \boldsymbol{\beta} \in C^{n}, \boldsymbol{A}, \boldsymbol{B} \in C^{n \times n}$ ,试证:\\
(1)$\|\boldsymbol{\alpha}-\boldsymbol{\beta}\| \geqslant\|\boldsymbol{\alpha}\|-\|\boldsymbol{\beta}\|$ ;\\
(2)$\|\boldsymbol{A}-\boldsymbol{B}\| \geqslant\|\boldsymbol{A}\|-\|\boldsymbol{B}\|$ .

5-2 设 $A=\left[a_{i j}\right] \in C^{n \times n}$ ,试证:\\
(1)$\|\boldsymbol{A}\|=\left[\operatorname{tr}\left(\boldsymbol{A}^{\mathbf{H}} \boldsymbol{A}\right)\right]^{\frac{1}{2}}$ 是矩阵范数;\\
(2)$\|\boldsymbol{A}\|=n \max _{i, j}\left|a_{i j}\right|$ 是矩阵范数。\\
5-3 对 $\boldsymbol{\alpha} \in C^{n}, ~ \boldsymbol{A} \in C^{n \times n}$ ,设 $\|\boldsymbol{A}\|$ 是诱导范数,且 $\operatorname{det} \boldsymbol{A} \neq 0$ ,试证:\\
(1)$\left\|\boldsymbol{A}^{-1}\right\| \geqslant\|\boldsymbol{A}\|^{-1}$ ;\\
(2)$\left\|\boldsymbol{A}^{-1}\right\|^{-1}=\min _{\alpha \neq 0} \frac{\|\boldsymbol{A} \boldsymbol{\alpha}\|}{\|\boldsymbol{\alpha}\|}$ .\\
5-4 设 $\alpha \in C^{n}$ ,试证:

$$
\frac{1}{n}\|\alpha\|_{1} \leqslant\|\alpha\|_{\infty} \leqslant\|\alpha\|_{2} \leqslant\|\alpha\|_{1} .
$$

5-5 设 $a_{1}, a_{2}, \cdots, a_{n}$ 是正实数,$\alpha=\left(x_{1}, x_{2}, \cdots, x_{n}\right)^{\mathrm{T}} \in R^{n}$ ,试证:$\|\boldsymbol{\alpha}\|=$ ( $\left.\sum_{i=1}^{n} a_{i}\left|x_{i}\right|^{2}\right)^{\frac{1}{2}}$ 是向量范数。

5-6 设 $\boldsymbol{A}$ 是正定 Hermite 矩阵,试证:若 $\boldsymbol{\alpha} \in C^{n}$ ,则 $\|\boldsymbol{\alpha}\|=\left(\boldsymbol{\alpha}^{\mathrm{H}} \boldsymbol{A} \boldsymbol{\alpha}\right)^{\frac{1}{2}}$ 是 $\boldsymbol{\alpha}$的向量范数.称此范数为椭圆范数.

5-7 试找一个收敛的二阶可逆的矩阵序列,但其极限矩阵不可逆。\\
5-8 讨论矩阵幂级数 $\sum_{k=1}^{\infty} \frac{1}{k^{2}}\left[\begin{array}{rr}1 & 4 \\ -1 & -3\end{array}\right]^{k}$ 的玫散性.\\
5-9 讨论矩阵幂级数 $\sum_{k=1}^{\infty} \frac{1}{k^{2}} A^{k}$ 的玫散性.\\
(1)$A=\left[\begin{array}{ccc}-1 & -2 & 6 \\ -1 & 0 & 3 \\ -1 & -1 & 4\end{array}\right]$\\
(2)$A=\left[\begin{array}{ccc}-2 & 1 & -1 \\ 0 & 1 & 0 \\ 1 & 1 & 0\end{array}\right]$

5-10 求矩阵幂级数 $\sum_{k=2}^{\infty}\left[\begin{array}{ll}0.1 & 0.7 \\ 0.3 & 0.6\end{array}\right]^{k}$ 的和.

\section*{第突章}
\section*{矩 阵 函 数}
矩阵函数定义一般有两种方法,本书采用矩阵函数的解析定义,而不采用幂级数法定义。矩阵函数有多种表示及计算方法,我们只介绍了矩阵函数的 Jordan 表示,多项式表示和幂级数表示及其相应计算方法。

\section*{§6.1 矩阵多项式、最小多项式}
\section*{一、矩阵多项式}
已知 $\boldsymbol{A} \in C^{n \times n}$ 和变量 $\lambda$ 的多项式


\begin{equation*}
p(\lambda)=a_{m} \lambda^{m}+a_{m-1} \lambda^{m-1}+\cdots+a_{1} \lambda+a_{0} \tag{6.1.1}
\end{equation*}


则称


\begin{equation*}
p(A)=a_{m} A^{m}+a_{m-1} A^{m-1}+\cdots+a_{1} A+a_{0} E \tag{6.1.2}
\end{equation*}


是 $\boldsymbol{A}$ 的矩阵多项式。 $p(\boldsymbol{A})$ 和 $\boldsymbol{A}$ 同为 $n$ 阶方阵。\\
若 $J_{i}$ 为 $d_{i}$ 阶 Jordan 块矩阵

\[
\boldsymbol{J}_{i}=\left[\begin{array}{ccccc}
\lambda_{i} & 1 & & &  \tag{6.1.3}\\
& \lambda_{i} & 1 & & \\
& & \ddots & & \\
& & & \ddots & 1 \\
& & & & \lambda_{i}
\end{array}\right]_{d_{i} \times d_{i}}
\]

则

\[
J_{i}^{k}\left(\lambda_{i}\right)=\left[\begin{array}{cccc}
\lambda_{i}^{k} & \mathrm{C}_{k}^{1} \lambda_{i}^{k-1} & \cdots & \mathrm{C}_{k}^{d-1} \lambda_{i}^{k-d_{i}+1}  \tag{6.1.4}\\
& \lambda_{i}^{k} & \mathrm{C}_{k}^{1} \lambda_{i}^{k-1} & \vdots \\
& \ddots & \ddots & \\
& & \ddots & \mathrm{C}_{k}^{1} \lambda_{i}^{k-1} \\
& & & \lambda_{i}^{k}
\end{array}\right]
\]

其中

$$
\begin{gathered}
\mathrm{C}_{k}^{l}=\frac{k(k-1) \cdots(k-l+1)}{l!} \\
\mathrm{C}_{k}^{l}=0
\end{gathered}
$$

(当 $l \leqslant k$ 时)\\
(当 $l>k$ 时)\\
于是关于 $d_{i}$ 阶矩阵 $\boldsymbol{J}_{i}\left(\lambda_{i}\right)$ 的矩阵多项式

$$
p\left(\boldsymbol{J}_{i}\right)=a_{m} \boldsymbol{J}_{i}^{m}+a_{m-1} \boldsymbol{J}_{i}^{m-1}+\cdots+a_{1} \boldsymbol{J}_{i}+a_{0} \boldsymbol{E}
$$

根据式(6.1.4)引人多项式 $p(\lambda)$ 的各阶导数记号后可写成

\[
p\left(\boldsymbol{J}_{i}\right)=\left[\begin{array}{ccccc}
p\left(\lambda_{i}\right) & p^{\prime}\left(\lambda_{i}\right) & \frac{p^{\prime \prime}\left(\lambda_{i}\right)}{2!} & \cdots & \frac{p^{\left(d_{i}-1\right)}\left(\lambda_{i}\right)}{\left(d_{i}-1\right)!}  \tag{6.1.5}\\
& p\left(\lambda_{i}\right) & p^{\prime}\left(\lambda_{i}\right) & & \vdots \\
& & p\left(\lambda_{i}\right) & \ddots & \\
& & & \ddots & p^{\prime}\left(\lambda_{i}\right) \\
& & & & p\left(\lambda_{i}\right)
\end{array}\right]_{d_{i} \times d_{i}}
\]

若 $\boldsymbol{J}$ 为 Jordan 标准形式, $\boldsymbol{J}=\operatorname{diag}\left(\boldsymbol{J}_{1}, \boldsymbol{J}_{2}, \cdots, \boldsymbol{J}_{r}\right)$ ,则


\begin{equation*}
p(J)=\operatorname{diag}\left(p\left(J_{1}\right), p\left(J_{2}\right), \cdots, p\left(J_{r}\right)\right. \tag{6.1.6}
\end{equation*}


若 $\boldsymbol{A}$ 为 $n$ 阶矩阵, $\boldsymbol{J}$ 是它的 Jordan 标准形式,则存在满秩矩阵 $\boldsymbol{P}$ ,使得

$$
\boldsymbol{A}=\boldsymbol{P} \boldsymbol{J} \boldsymbol{P}^{-1}=\boldsymbol{P} \operatorname{diag}\left(\boldsymbol{J}_{1}, \boldsymbol{J}_{2}, \cdots, \boldsymbol{J}_{r}\right) \boldsymbol{P}^{-1}
$$

因此


\begin{equation*}
p(\boldsymbol{A})=\boldsymbol{P} \operatorname{diag}\left(p\left(\boldsymbol{J}_{1}\right), p\left(\boldsymbol{J}_{2}\right), \cdots, p\left(\boldsymbol{J}_{r}\right)\right) \boldsymbol{P}^{-1} \tag{6.1.7}
\end{equation*}


式(6.1.7)称为矩阵多项式 $p(\boldsymbol{A})$ 的 Jordan 表示,用它可以计算 $p(\boldsymbol{A})$ 。\\
根据矩阵多项式的 Jordan 表示式(6.1.7)可以得到下面命题。\\
命题 6.1.1 已知 $n$ 阶矩阵 $\boldsymbol{A}$ 的特征值为 $\lambda_{i} \quad(i=1,2, \cdots, n)$ ,则矩阵多项式 $p(\boldsymbol{A})$ 的特征值为 $p\left(\lambda_{i}\right) \quad(i=1,2, \cdots, n)$(证略)。

实际上利用矩阵特征值、特征向量定义还可以证明下述定理。\\
定理6.1.1 设 $\lambda$ 是 $n$ 阶矩阵 $\boldsymbol{A}$ 的特征值, $\boldsymbol{\alpha}$ 是 $\boldsymbol{A}$ 的属于 $\lambda$ 的特征向量,则 $p(\lambda)$ 是矩阵多项式 $p(\boldsymbol{A})$ 的特征值, $\boldsymbol{\alpha}$ 是 $p(\boldsymbol{A})$ 的属于 $p(\lambda)$ 的特征向量。(证略)

例6.1.1 已知多项式 $p(\lambda)=\lambda^{4}-2 \lambda^{3}+\lambda-1$ 与矩阵

$$
A=\left[\begin{array}{rrr}
2 & 0 & 0 \\
1 & 1 & 1 \\
1 & -1 & 3
\end{array}\right]
$$

计算 $p(\boldsymbol{A})$ .\\
解 $\boldsymbol{A}$ 的 Jordan 标准形是

$$
J=\left[\begin{array}{lll}
2 & 1 & 0 \\
0 & 2 & 0 \\
0 & 0 & 2
\end{array}\right]
$$

变换矩阵 $\boldsymbol{P}$ 和 $\boldsymbol{P}^{-1}$ 分别为

$$
\boldsymbol{P}=\left[\begin{array}{rrr}
0 & 1 & 1 \\
1 & 0 & 0 \\
1 & 0 & -1
\end{array}\right], \boldsymbol{P}^{-1}=\left[\begin{array}{rrr}
0 & 1 & 0 \\
1 & -1 & 1 \\
0 & 1 & -1
\end{array}\right]
$$

所以由式(6.1.7)与式(6.1.5)得

$$
\begin{aligned}
p(\boldsymbol{A}) & =\boldsymbol{P}_{p}(\boldsymbol{J}) \boldsymbol{P}^{-1} \\
& =\left[\begin{array}{lll}
0 & 1 & 1 \\
1 & 0 & 0 \\
1 & 0 & -1
\end{array}\right]\left[\begin{array}{ccc}
p(2) & p^{\prime}(2) & 0 \\
0 & p(2) & 0 \\
0 & 0 & p(2)
\end{array}\right]\left[\begin{array}{rrr}
0 & 1 & 0 \\
1 & -1 & 1 \\
0 & 1 & -1
\end{array}\right] \\
& =\left[\begin{array}{ccc}
p(2) & 0 & 0 \\
p^{\prime}(2) & p(2)-p^{\prime}(2) & p^{\prime}(2) \\
p^{\prime}(2) & -p^{\prime}(2) & p^{\prime}(2)+p(2)
\end{array}\right] \\
& =\left[\begin{array}{rrr}
1 & 0 & 0 \\
9 & -8 & 9 \\
9 & -9 & 10
\end{array}\right]
\end{aligned}
$$

\section*{二、化零多项式、最小多项式}
定义6.1.1 给定矩阵 $A \in C^{n \times n}$ ,如果多项式

$$
p(\lambda)=a_{m} \lambda^{m}+a_{m-1} \lambda^{m-1}+\cdots+a_{1} \lambda+a_{0}
$$

满足 $p(\boldsymbol{A})=0$ ,则称 $p(\lambda)$ 是 $\boldsymbol{A}$ 的化零多项式。\\
定理 6.1.2(Hamilton-Cayley 定理)$n$ 阶方阵 $A$ 的特征多项式

$$
\begin{aligned}
D(\lambda) & =\operatorname{det}(\lambda \boldsymbol{E}-\boldsymbol{A}) \\
& =\lambda^{n}-\operatorname{tr} \boldsymbol{A} \lambda^{n-1}+\cdots+(-1)^{n} \operatorname{det} \boldsymbol{A}
\end{aligned}
$$

是 $\boldsymbol{A}$ 的化零多项式,即 $D(\boldsymbol{A})=0$ .\\
证明 设 $J=\operatorname{diag}\left(J_{1}\left(\lambda_{1}\right), J_{2}\left(\lambda_{2}\right), \cdots, J_{r}\left(\lambda_{r}\right)\right)$ 是 $A$ 的 Jordan 标准形,故存在满秩方阵 $\boldsymbol{P}$ ,满足

$$
\boldsymbol{A}=\boldsymbol{P} J \boldsymbol{P}^{-1}=\boldsymbol{P} \operatorname{diag}\left(J_{1}\left(\lambda_{1}\right), J_{2}\left(\lambda_{2}\right), \cdots, J_{r}\left(\lambda_{r}\right)\right) \boldsymbol{P}^{-1}
$$

于是根据式(6.1.6)得

$$
D(\boldsymbol{A})=\boldsymbol{P} D(\boldsymbol{J}) \boldsymbol{P}^{-1}=\boldsymbol{P} \operatorname{diag}\left(D\left(\boldsymbol{J}_{1}\right), D\left(\boldsymbol{J}_{2}\right), \cdots, D\left(\boldsymbol{J}_{r}\right)\right) \boldsymbol{P}^{-1}
$$

其中

$$
D\left(\boldsymbol{J}_{i}\right)=\left[\begin{array}{ccccc}
D\left(\lambda_{i}\right) & D^{\prime}\left(\lambda_{i}\right) & \frac{1}{2!} D^{\prime \prime}\left(\lambda_{i}\right) & \cdots & \frac{1}{\left(d_{i}-1\right)!} D^{\left(d_{i}-1\right)}\left(\lambda_{i}\right) \\
& D\left(\lambda_{i}\right) & \ddots & & \vdots \\
& & \ddots & \ddots & D^{\prime}\left(\lambda_{i}\right) \\
& & & \ddots & D\left(\lambda_{i}\right)
\end{array}\right]_{d_{i} \times d_{i}}
$$

由于 $\lambda_{i}$ 的代数重复度 $\geqslant d_{i}$ ,故

$$
D\left(\lambda_{i}\right)=D^{\prime}\left(\lambda_{i}\right)=D^{\left(d_{i}-1\right)}\left(\lambda_{i}\right)=0
$$

于是 $D\left(J_{i}\right)=0(i=1,2, \cdots, r)$ ,因此 $D(J)=0$ ,证毕。\\
若 $f(\lambda)$ 是 $\boldsymbol{A}$ 的化零多项式,$h(\lambda)$ 是任一多项式,则 $f(\lambda) h(\lambda)$ 也是 $\boldsymbol{A}$ 的化零多项式。因此不存在 $A$ 的次数最高的化零多项式。

定义6.1.2 在 $\boldsymbol{A}$ 的化零多项式中,次数最低且首项系数为 1 的化零多项式称为 $\boldsymbol{A}$ 的最小多项式,记为 $\psi_{\boldsymbol{A}}(\lambda)$ 。

根据定理6.1.3知, $\boldsymbol{A}$ 的最小多项式是其特征多项式的因子。

\section*{定理6.1.3 设 $A \in C^{n \times n}$ ,则}
(1) $\boldsymbol{A}$ 的任一化零多项式都能被 $\psi_{A}(\lambda)$ 整除;\\
(2) $\boldsymbol{A}$ 的最小多项式 $\psi_{A}(\lambda)$ 是唯一的;\\
(3)相似矩阵的最小多项式相同。\\
证明(1)设 $f(\lambda)$ 为 $\boldsymbol{A}$ 的化零多项式,根据多项式理论可知

$$
f(\lambda)=\psi_{A}(\lambda) q(\lambda)+r(\lambda)
$$

其中余项 $r(\lambda)$ 的次数 $<\psi_{A}(\lambda)$ 的次数。把 $\boldsymbol{A}$ 代入上式 $\lambda$ 得

$$
f(\boldsymbol{A})=\psi_{\boldsymbol{A}}(\boldsymbol{A}) q(\boldsymbol{A})+r(\boldsymbol{A})
$$

由于 $f(\boldsymbol{A})=0, \psi_{A}(\boldsymbol{A})=0$ ,得 $r(\boldsymbol{A})=0$ .又因已知 $\psi_{\boldsymbol{A}}(\lambda)$ 是 $\boldsymbol{A}$ 的最小多项式,因此 $r(\lambda)=0$ ,证毕。\\
(2)应用(1)的结果立即可证。\\
(3)设 $\boldsymbol{B}=\boldsymbol{P}^{-1} \boldsymbol{A P}$ ,则由 $p(\boldsymbol{B})=\boldsymbol{P}^{-1} p(\boldsymbol{A}) \boldsymbol{P}$ 立即可得 $\psi_{\boldsymbol{A}}(\lambda)=\psi_{\boldsymbol{B}}(\lambda)$ 。\\
根据刚才证明的定理 6.1.3 知,矩阵 $\boldsymbol{A}$ 的最小多项式等于 $\boldsymbol{A}$ 的 Jordan 标准形的最小多项式,而 Jordan 标准形是准对角形矩阵,为此先介绍下述定理。

定理6.1.4 设 $\boldsymbol{A}=\operatorname{diag}\left(\boldsymbol{A}_{1}, \boldsymbol{A}_{2}, \cdots, \boldsymbol{A}_{s}\right), \psi_{1}(\lambda), \psi_{2}(\lambda), \cdots, \psi_{s}(\lambda)$ 分别是 $\boldsymbol{A}_{1}, \boldsymbol{A}_{2}, \cdots, \boldsymbol{A}_{s}$ 的最小多项式,则 $\boldsymbol{A}$ 的最小多项式 $\psi_{A}(\lambda)$ 是 $\psi_{1}(\lambda), \psi_{2}(\lambda), \cdots$ , $\psi_{s}(\lambda)$ 的最低公倍式。

证明 设 $\psi_{A}(\lambda)$ 是 $A$ 的最小多项式,则

$$
\psi_{A}(A)=\operatorname{diag}\left(\psi_{A}\left(A_{1}\right), \psi_{A}\left(A_{2}\right), \cdots, \psi_{A}\left(A_{s}\right)\right)=0
$$

于是 $\psi_{A}\left(\boldsymbol{A}_{1}\right)=0, \psi_{A}\left(\boldsymbol{A}_{2}\right)=0, \cdots, \psi_{A}\left(\boldsymbol{A}_{s}\right)=0$ ,即 $\psi_{A}(\lambda)$ 是 $\boldsymbol{A}_{1}, \boldsymbol{A}_{2}, \cdots, \boldsymbol{A}_{s}$ 的化零多项式。因此 $\psi_{A}(\lambda)$ 是 $\psi_{1}(\lambda), \psi_{2}(\lambda), \cdots, \psi_{s}(\lambda)$ 的公倍式。

另一方面,若 $\psi_{A}(\lambda)$ 是 $\psi_{1}(\lambda), \psi_{2}(\lambda), \cdots, \psi_{s}(\lambda)$ 的最低公倍式,则 $\psi_{A}(\boldsymbol{A})=$ 0 ,若 $\psi_{A}(\lambda)$ 不是 $\psi_{1}(\lambda), \psi_{2}(\lambda), \cdots, \psi_{s}(\lambda)$ 的公倍式,则 $\psi_{A}(\boldsymbol{A}) \neq 0$ .定理得证.

例6.1.2 求 $d_{i}$ 阶 Jordan 块

$$
J_{i}=\left[\begin{array}{ccccc}
\lambda_{i} & 1 & & & \\
& \lambda_{i} & 1 & & \\
& & & \ddots & \\
& & & \ddots & 1 \\
& & & & \lambda_{i}
\end{array}\right]_{d_{i} \times d_{i}}
$$

的最小多项式。\\
解 不难验算得

$$
\begin{gathered}
\left(\boldsymbol{J}_{i}-\lambda_{i} \boldsymbol{E}\right)^{d_{i}-1}=\left[\begin{array}{cccc}
0 & \cdots & 0 & 1 \\
0 & \cdots & 0 & 0 \\
\vdots & & \vdots & \vdots \\
0 & \cdots & 0 & 0
\end{array}\right] \neq 0 \\
\left(\boldsymbol{J}_{i}-\lambda_{i} \boldsymbol{E}\right)^{d_{i}}=0
\end{gathered}
$$

因此,Jordan 块 $J_{i}$ 的最小多项式为 $\left(\lambda-\lambda_{i}\right)^{d_{i}}$ ,显然它等于 $J_{i}$ 的初等因子。\\
根据定理6.1.3、定理6.1.4与例6.1.2可以得到用矩阵 $\boldsymbol{A}$ 的 Jordan 标准形 (初等因子)求 $\boldsymbol{A}$ 的最小多项式的方法。

例6.1.3 求矩阵 $\boldsymbol{A}$ 的最小多项式。若\\
(1)$A=\left[\begin{array}{rrr}-1 & -2 & 6 \\ -1 & 0 & 3 \\ -1 & -1 & 4\end{array}\right]$\\
(2)$A=\left[\begin{array}{rrr}-1 & 1 & 1 \\ -5 & 21 & 17 \\ 6 & -26 & -21\end{array}\right]$\\
(3)$A=\left[\begin{array}{rrr}-7 & -12 & -6 \\ 3 & 5 & 3 \\ 3 & 6 & 2\end{array}\right]$\\
(4) $\boldsymbol{A}=\left[\begin{array}{rrrc}3 & -1 & -3 & 1 \\ -1 & 3 & 1 & -3 \\ 3 & -1 & -3 & 1 \\ -1 & 3 & 1 & -3\end{array}\right]$

解(1) $\boldsymbol{A}$ 的 Jordan 标准形为

$$
J=\left[\begin{array}{lll}
1 & & \\
& 1 & 1 \\
& & 1
\end{array}\right]
$$

故 $\boldsymbol{A}$ 的最小多项式为 $(\lambda-1)^{2}$ .\\
(2) $\boldsymbol{A}$ 的 Jordan 标准形为

$$
J=\left[\begin{array}{ccc}
-1 & & \\
& 0 & 1 \\
& & 0
\end{array}\right]
$$

故 $\boldsymbol{A}$ 的最小多项式为 $\lambda^{2}(\lambda+1)$ .\\
(3) $\boldsymbol{A}$ 的 Jordan 标准形为

$$
J=\left[\begin{array}{lll}
-1 & & \\
& -1 & \\
& & 2
\end{array}\right]
$$

故 $\boldsymbol{A}$ 的最小多项式为 $(\lambda+1)(\lambda-2)$ .\\
(4) $\boldsymbol{A}$ 的 Jordan 标准形为

$$
J=\left[\begin{array}{llll}
0 & 1 & & \\
& 0 & & \\
& & 0 & 1 \\
& & & 0
\end{array}\right]
$$

故 $\boldsymbol{A}$ 的最小多项式为 $\lambda^{2}$ 。\\
设 $p(\lambda)$ 与 $q(\lambda)$ 是两个不同的多项式,对于 $n$ 阶矩阵 $A$ 满足什么条件使得 $p(\boldsymbol{A})=q(\boldsymbol{A})$ 。为了研究这个问题及引进矩阵函数定义的需要,我们首先给出关于函数在矩阵 $\boldsymbol{A}$ 的谱上的定义。

定义6.1.3 设 $\boldsymbol{A} \in C^{n \times n}, \lambda_{1}, \lambda_{2}, \cdots, \lambda_{s}$ 为 $\boldsymbol{A}$ 的相异特征值, $\boldsymbol{A}$ 的最小多项式

$$
\psi_{A}(\lambda)=\left(\lambda-\lambda_{1}\right)^{d_{1}}\left(\lambda-\lambda_{2}\right)^{d_{2}} \cdots\left(\lambda-\lambda_{s}\right)^{d_{s}}
$$

若函数 $f(x)$ 具有足够多阶的导数值,且下列 $m$ 个值(称 $f(x)$ 在影谱上的值)

$$
f\left(\lambda_{i}\right), f^{\prime}\left(\lambda_{i}\right), \cdots, f^{\left(d_{i}-1\right)}\left(\lambda_{i}\right) \quad(i=1,2, \cdots, s)
$$

都有确定的值,便称函数 $f(x)$ 在矩阵 $\boldsymbol{A}$ 的影谱上有定义。\\
例如

$$
f(x)=\frac{1}{(x-3)(x-4)}
$$

若

$$
A=\left[\begin{array}{rrr}
8 & -3 & 6 \\
3 & -2 & 0 \\
4 & 2 & -2
\end{array}\right], \quad B=\left[\begin{array}{rrr}
3 & -1 & 0 \\
-1 & 2 & -1 \\
0 & -1 & 3
\end{array}\right]
$$

$\boldsymbol{A}$ 与 $\boldsymbol{B}$ 的最小多项式分别为

$$
\psi_{A}(\lambda)=(\lambda-2)(\lambda-1)^{2}, \psi_{B}(\lambda)=(\lambda-1)(\lambda-3)(\lambda-4)
$$

因为 $f(2)=\frac{1}{2}, f(1)=\frac{1}{6}, f^{\prime}(1)=\frac{5}{36}$ ,故 $f(x)$ 在 $\boldsymbol{A}$ 的影谱上有定义。而 $f(3)$ 无意义,故 $f(x)$ 在 $\boldsymbol{B}$ 的影谱上无定义。

定理6.1.5 设 $p(\lambda)$ 与 $q(\lambda)$ 为两个不同的多项式, $\boldsymbol{A}$ 为 $n$ 阶矩阵,则 $p(\boldsymbol{A})= q(\boldsymbol{A})$ 的充分必要条件 $p(\lambda)$ 与 $q(\lambda)$ 在 $\boldsymbol{A}$ 的影谱上的值对应相等,即

$$
p^{(k)}\left(\lambda_{i}\right)=q^{(k)}\left(\lambda_{i}\right) \quad\left(i=1,2, \cdots, s ; k=0,1,2, \cdots, d_{i}-1 ;\right)
$$

证明 根据式(6.1.5)可得。

\section*{§6.2 矩阵函数及其 Jordan 表示}
若 $f(x)$ 为 $x$ 的函数(例如 $\mathrm{e}^{x}, \sin x, \cos x, \cdots$ ), $\boldsymbol{A}$ 为 $n$ 阶矩阵,如何定义 $f(\boldsymbol{A})$ ?显然用这些函数的原来含义来定义 $f(\boldsymbol{A})$ 就不合适了,为此需要用新的思路。

定义6.2.1 设函数 $f(x)$ 在 $n$ 阶矩阵 $\boldsymbol{A}$ 的影谱上有定义,即

$$
f^{(k)}\left(\lambda_{j}\right) \quad\left(j=1,2, \cdots, s ; k=0,1,2, \cdots, d_{j}-1\right)
$$

是确定的值.若 $p(\lambda)$ 为一多项式,且满足


\begin{equation*}
f^{(k)}\left(\lambda_{j}\right)=p^{(k)}\left(\lambda_{j}\right) \quad\left(j=1,2, \cdots, s ; k=0,1,2, \cdots, d_{j}-1\right) \tag{6.2.1}
\end{equation*}


则矩阵函数 $f(\boldsymbol{A})$ 定义为

$$
f(\boldsymbol{A})=p(\boldsymbol{A})
$$

注1 由定理6.1.5知满足上述定义的 $p(\lambda)$ 是不唯一的。\\
注2 矩阵函数 $f(\boldsymbol{A})$ 是与 $\boldsymbol{A}$ 相同阶数的矩阵。\\
定义6.2.2 把函数 $f(x)$ 的矩阵函数 $f(\boldsymbol{A})$ 用一个矩阵多项式 $p(\boldsymbol{A})$ 来确定,于是根据矩阵多项式 $p(\boldsymbol{A})$ 的 Jordan 表示式(6.1.7)得到矩阵函数 $f(\boldsymbol{A})$ 的 Jordan表示式。

定理 6.2.1 设 $A \in C^{n \times n}, ~ J$ 是 $A$ 的 Jordan 标准形,$P \in C_{n}^{n \times n}, A=P J P^{-1}$ ,若函数 $f(x)$ 在 $\boldsymbol{A}$ 的影谱上有定义,则


\begin{align*}
f(A) & =P f(J) P^{-1}  \tag{6.2.2}\\
& =P \operatorname{diag}\left(f\left(J_{1}\right), f\left(J_{2}\right), \cdots, f\left(J_{r}\right)\right) P^{-1}
\end{align*}


其中

\[
f\left(\boldsymbol{J}_{i}\right)=\left[\begin{array}{cccccc}f\left(\lambda_{i}\right) & f^{\prime}\left(\lambda_{i}\right) & \frac{1}{2!} f^{\prime \prime}\left(\lambda_{i}\right) & \cdots & \frac{1}{\left(d_{i}-1\right)!} & f^{\left(d_{i}-1\right)}\left(\lambda_{i}\right)  \tag{6.2.3}\\ & f\left(\lambda_{i}\right) & f^{\prime}\left(\lambda_{i}\right) & & & \\ & & \ddots & \ddots & \ddots & \vdots \\ & & & \ddots & \ddots & \frac{1}{2!} f^{\prime \prime}\left(\lambda_{i}\right) \\ & & & & \ddots & f^{\prime}\left(\lambda_{i}\right) \\ & & & & f\left(\lambda_{i}\right)\end{array}\right]_{d_{i} \times d_{i}}
\]

证略。\\
根据矩阵函数的定义6.2.2与定理6.1.1可得定理6.2.2.\\
定理6.2.2 设 $\lambda$ 是 $n$ 阶矩阵 $\boldsymbol{A}$ 的特征值, $\boldsymbol{\alpha}$ 是 $\boldsymbol{A}$ 的属于 $\lambda$ 的特征向量,则 $f$ ( $\lambda$ )是矩阵函数 $f(\boldsymbol{A})$ 的特征值, $\boldsymbol{\alpha}$ 是 $f(\boldsymbol{A})$ 的属于 $f(\lambda)$ 的特征向量。

例6.2.1 已知

$$
A=\left[\begin{array}{ccc}
17 & 0 & -25 \\
0 & 3 & 0 \\
9 & 0 & -13
\end{array}\right]
$$

求 $f(\boldsymbol{A})$ 的 Jordan 表示,并计算 $\mathrm{e}^{\boldsymbol{A}}, \mathrm{e}^{\boldsymbol{A} t}, \cos \boldsymbol{A}, \sin \frac{3}{2} \pi \boldsymbol{A}$ .\\
解 由例 2.3.2 知 $\boldsymbol{A}$ 的 Jordan 标准形为

$$
J=\left[\begin{array}{lll}
1 & 0 & 0 \\
0 & 2 & 1 \\
0 & 0 & 2
\end{array}\right]
$$

变换矩阵 $\boldsymbol{P}$ 和 $\boldsymbol{P}^{-1}$ 分别为

$$
\boldsymbol{P}=\left[\begin{array}{lll}
0 & 5 & 2 \\
1 & 0 & 0 \\
0 & 3 & 1
\end{array}\right], \boldsymbol{P}^{-1}=\left[\begin{array}{rrr}
0 & 1 & 0 \\
-1 & 0 & 2 \\
3 & 0 & -5
\end{array}\right]
$$

且

$$
f(J)=\left[\begin{array}{ccc}
f(1) & 0 & 0 \\
0 & f(2) & f^{\prime}(2) \\
0 & 0 & f(2)
\end{array}\right]
$$

所以 $f(\boldsymbol{A})$ 的 Jordan 表示为

$$
\begin{aligned}
f(\boldsymbol{A}) & =\boldsymbol{P} f(\boldsymbol{J}) \boldsymbol{P}^{-1} \\
& =\left[\begin{array}{lll}
0 & 5 & 2 \\
1 & 0 & 0 \\
0 & 3 & 1
\end{array}\right]\left[\begin{array}{ccc}
f(1) & 0 & 0 \\
0 & f(2) & f^{\prime}(2) \\
0 & 0 & f(2)
\end{array}\right]\left[\begin{array}{rrr}
0 & 1 & 0 \\
-1 & 0 & 2 \\
3 & 0 & -5
\end{array}\right]
\end{aligned}
$$

为了计算所求矩阵函数,$f(\boldsymbol{A})$ 可写为

$$
f(\boldsymbol{A})=\left[\begin{array}{ccc}
f(2)+15 f^{\prime}(2) & 0 & -25 f^{\prime}(2) \\
0 & f(1) & 0 \\
9 f^{\prime}(2) & 0 & f(2)-15 f^{\prime}(2)
\end{array}\right]
$$

当 $f(x)=\mathrm{e}^{x}$ 时,$f(2)=f^{\prime}(2)=\mathrm{e}^{2}, f(1)=\mathrm{e}$ ,故

$$
\mathrm{e}^{A}=\left[\begin{array}{ccc}
16 \mathrm{e}^{2} & 0 & -25 \mathrm{e}^{2} \\
0 & \mathrm{e} & 0 \\
9 \mathrm{e}^{2} & 0 & -14 \mathrm{e}^{2}
\end{array}\right]
$$

当 $f(x)=\mathrm{e}^{t x}$ 时,$f(2)=\mathrm{e}^{2 t}, f^{\prime}(2)=t \mathrm{e}^{2 t}, f(1)=\mathrm{e}^{t}$ .故

$$
\mathbf{e}^{A t}=\left[\begin{array}{ccc}
\mathrm{e}^{2 t}(1+15 t) & 0 & -25 t \mathrm{e}^{2 t} \\
0 & \mathrm{e}^{t} & 0 \\
9 t \mathrm{e}^{2 t} & 0 & \mathrm{e}^{2 t}(1-15 t)
\end{array}\right]
$$

当 $f(x)=\cos x$ 时,$f(2)=\cos 2, f^{\prime}(2)=-\sin 2, f(1)=\cos 1$ ,故

$$
\cos A=\left[\begin{array}{ccc}
\cos 2-15 \sin 2 & 0 & 25 \sin 2 \\
0 & \cos 3 & 0 \\
-9 \sin 2 & 0 & \cos 2+15 \sin 2
\end{array}\right]
$$

当 $f(x)=\sin \frac{3 \pi}{2} x$ 时,$f(2)=0, f^{\prime}(2)=-\frac{3 \pi}{2}, f(1)=-1$

$$
\sin \frac{3 \pi}{2} A=\left[\begin{array}{ccc}
-\frac{45 \pi}{2} & 0 & \frac{75 \pi}{2} \\
0 & -1 & 0 \\
-\frac{27 \pi}{2} & 0 & \frac{45 \pi}{2}
\end{array}\right]
$$

注:确定函数 $f(x)$ 在影谱上的值时,需计算 $f^{\prime}\left(\lambda_{i}\right), f^{\prime \prime}\left(\lambda_{i}\right), \cdots$ ,这里的导数是对 $x$ 的

导数,因此若函数 $f(x)$ 带有参数 $t$ 时,求 $x$ 的导数需处理好参数 $t$ .例如,若 $f(x)= \mathrm{e}^{t x}$ ,则 $f\left(\lambda_{i}\right)=\mathrm{e}^{t \lambda_{i}}, f^{\prime}\left(\lambda_{i}\right)=t \mathrm{e}^{t \lambda_{i}}, f^{\prime \prime}\left(\lambda_{i}\right)=t^{2} \mathrm{e}^{t \lambda_{i}}, \cdots$ .若 $f(x)=\sin t x$ ,则 $f\left(\lambda_{i}\right)= \sin \lambda_{i} t, f^{\prime}\left(\lambda_{i}\right)=t \cos \lambda_{i} t, f^{\prime \prime}\left(\lambda_{i}\right)=-t^{2} \sin \lambda_{i} t, \cdots$ ,故

\[
\mathrm{e}^{J_{i} t}=\left[\begin{array}{ccccc}
\mathrm{e}^{\lambda_{i} t} & t \mathrm{e}^{\lambda_{i} t} & \frac{t^{2}}{2!} \mathrm{e}^{\lambda_{i} t} & \cdots & \frac{t^{d_{i}-1}}{\left(d_{i}-1\right)}  \tag{6.2.4}\\
& \mathrm{e}^{\lambda_{i} t} & t \mathrm{e}^{\lambda_{i} t} & & \vdots \\
& & \ddots & \ddots & \\
& & & \ddots & t \mathrm{e}^{\lambda_{i} t} \\
& & & & \mathrm{e}^{\lambda_{i} t}
\end{array}\right]_{d_{i} \times d_{i}}
\]

用矩阵函数的 Jordan 表示计算矩阵函数的步骤:\\
1.求 $A$ 的 Jordan 标准形,$J=\operatorname{diag}\left(J_{1}, J_{2}, \cdots, J_{r}\right)$ 。\\
2.由 $\boldsymbol{J}$ 写出 $f(\boldsymbol{J})=\operatorname{diag}\left(f\left(\boldsymbol{J}_{1}\right), f\left(\boldsymbol{J}_{2}\right), \cdots, f\left(\boldsymbol{J}_{r}\right)\right)$ .\\
3.由 $\boldsymbol{J}$ 计算变换矩阵 $\boldsymbol{P}$ ,满足 $\boldsymbol{A P}=\boldsymbol{P J}$ 。\\
4.写出 $f(\boldsymbol{A})$ 的 Jordan 表示式

$$
f(\boldsymbol{A})=\boldsymbol{P} f(\boldsymbol{J}) \boldsymbol{P}^{-1}
$$

5.把所求矩阵函数 $f(A)$ 所对应的函数 $f(x)$ 代入 $P f(J) P^{-1}$ 即可。\\
根据矩阵函数的定义,矩阵函数也有与定理6.1.5相类似的结论。\\
定理6.2.3 设函数 $f(x)$ 与 $g(x)$ 在矩阵 $\boldsymbol{A}$ 的影谱上有定义,则矩阵函数 $f(\boldsymbol{A})=g(\boldsymbol{A})$ 的充分必要条件是 $f(x)$ 与 $g(x)$ 在 $\boldsymbol{A}$ 的影谱上的值全相同。

\section*{§6.3 矩阵函数的内插多项式表示与多项式表示}
\section*{一、矩阵函数的拉格朗日—西勒维斯特内插多项式表示}
根据矩阵函数定义知道,函数 $f(x)$ 的矩阵函数 $f(\boldsymbol{A})$ 是用一个多项式 $p(x)$ 的矩阵多项式 $p(\boldsymbol{A})$ 来定义的,只要 $p(x)$ 与 $f(x)$ 在 $\boldsymbol{A}$ 的影谱上的值全相同,而根据数值计算课程知道,在诸多满足要求的多项式中有一个次数最低的称为拉格朗日—西勒维斯特(Lagrange-Sylvester)内插多项式。

设 $n$ 阶矩阵 $\boldsymbol{A}$ 的最小多项式

$$
\begin{aligned}
\varphi_{A}(x)= & \left(x-\lambda_{1}\right)^{d_{1}}\left(x-\lambda_{2}\right)^{d_{2}} \cdots\left(x-\lambda_{s}\right)^{d_{s}} \\
& \left(d_{1}+d_{2}+\cdots+d_{s}=m\right)
\end{aligned}
$$

若函数 $f(x)$ 在 $\boldsymbol{A}$ 的影谱上的值有定义,则 $f(x)$ 的拉格朗日一西勒维斯特内插多项式是


\begin{equation*}
p(x)=\sum_{k=1}^{s}\left[a_{k 1}+a_{k 2}\left(x-\lambda_{k}\right)+\cdots+a_{k d_{k}}\left(x-\lambda_{k}\right)^{d_{k}-1}\right] \varphi_{k}(x) \tag{6.3.1}
\end{equation*}


其中


\begin{align*}
\varphi_{k}(x)= & \frac{\varphi_{A}(x)}{\left(x-\lambda_{k}\right)^{d_{k}}} \\
= & \left(x-\lambda_{1}\right)^{d_{1}} \cdots\left(x-\lambda_{k-1}\right)^{d_{k-1}} \cdot\left(x-\lambda_{k+1}\right)^{d_{k+1}} \cdots\left(x-\lambda_{s}\right)^{d_{s}} \\
& \quad\left(k=1,2, \cdots, s ; l=1,2, \cdots, d_{k}\right)  \tag{6.3.2}\\
a_{k l}= & \left.\frac{1}{(l-1)!}\left[\frac{d^{l-1}}{d x^{l-1}}\left(\frac{f(x)}{\varphi_{k}(x)}\right)\right]\right|_{x=\lambda_{k}} \tag{6.3.3}
\end{align*}


容易验证,多项式 $p(x)$ 的次数为 $m-1$ ,且 $p(x)$ 与 $f(x)$ 在 $\boldsymbol{A}$ 的影谱上的值全相同。因此根据矩阵函数定义便有


\begin{align*}
f(\boldsymbol{A})=p(\boldsymbol{A})= & \sum_{k=1}^{s}\left[a_{k 1} \boldsymbol{E}+a_{k 2}\left(\boldsymbol{A}-\lambda_{k} \boldsymbol{E}\right)+\cdots\right. \\
& \left.+a_{k d_{k}}\left(\boldsymbol{A}-\lambda_{k} \boldsymbol{E}\right)^{d_{k}-1}\right] \varphi_{k}(\boldsymbol{A}) \tag{6.3.4}
\end{align*}


称式(6.3.4)是矩阵函数 $f(\boldsymbol{A})$ 的拉格朗日一西勒维斯特内插多项式表示。\\
例 6.3 . 1 设

$$
A=\left[\begin{array}{rrr}
2 & 0 & 0 \\
1 & 1 & 1 \\
1 & -1 & 3
\end{array}\right]
$$

试求矩阵函数 $f(\boldsymbol{A})$ 的拉格朗日—西勒维斯特内插多项式表示,并计算 $\mathrm{e}^{\boldsymbol{A} t}, \boldsymbol{\operatorname { s i n }} \boldsymbol{A}$ , $\cos \pi A,(E-A)^{-1}, \arctan \frac{A}{2}, A^{10}$.

解 $\boldsymbol{A}$ 的最小多项式

$$
\varphi_{A}(x)=(x-2)^{2}
$$

由式(6.3.3)得( $k=1, l=1,2$ )

$$
\varphi_{1}(x)=1
$$

由式(6.3.2)得

$$
\begin{gathered}
a_{11}=\left.f(x)\right|_{x=2}=f(2) \\
a_{12}=\left.\frac{\mathrm{d}}{\mathrm{~d} x} f(x)\right|_{x=2}=f^{\prime}(2)
\end{gathered}
$$

由式(6.3.1)得

$$
p(x)=f(2)+f^{\prime}(2)(x-2)
$$

因此 $f(\boldsymbol{A})$ 的拉格朗日—西勒维斯特内插多项式表示是

$$
f(\boldsymbol{A})=p(\boldsymbol{A})=f(2) \boldsymbol{E}+f^{\prime}(2)(\boldsymbol{A}-2 \boldsymbol{E})
$$

为了计算所求矩阵函数,把 $\boldsymbol{A}$ 代人上式得

$$
f(\boldsymbol{A})=\left[\begin{array}{ccc}
f(2) & 0 & 0 \\
f^{\prime}(2) & f(2)-f^{\prime}(2) & f^{\prime}(2) \\
f^{\prime}(2) & -f^{\prime}(2) & f(2)+f^{\prime}(2)
\end{array}\right]
$$

当 $f(x)=\mathrm{e}^{t x}$ 时,$f(2)=\mathrm{e}^{2 t}, f^{\prime}(2)=t \mathrm{e}^{2 t}$ ,所以

$$
\mathrm{e}^{t A}=\left[\begin{array}{ccc}
\mathrm{e}^{2 t} & 0 & 0 \\
t \mathrm{e}^{2 t} & \mathrm{e}^{2 t}(1-t) & t \mathrm{e}^{2 t} \\
t \mathrm{e}^{2 t} & -t \mathrm{e}^{2 t} & \mathrm{e}^{2 t}(1+t)
\end{array}\right]
$$

当 $f(x)=\sin x$ 时,$f(2)=\sin 2, f^{\prime}(2)=\cos 2$ ,所以

$$
\sin A=\left[\begin{array}{ccc}
\sin 2 & 0 & 0 \\
\cos 2 & \sin 2-\cos 2 & \cos 2 \\
\cos 2 & -\cos 2 & \sin 2+\cos 2
\end{array}\right]
$$

当 $f(x)=\cos \pi x$ 时,$f(2)=1, f^{\prime}(2)=0$ ,故

$$
\cos \pi A=\left[\begin{array}{lll}
1 & 0 & 0 \\
0 & 1 & 0 \\
0 & 0 & 1
\end{array}\right]=E
$$

当 $f(x)=(1-x)^{-1}$ 时,$f(2)=-1, f^{\prime}(2)=1$ ,故

$$
(\boldsymbol{E}-\boldsymbol{A})^{-1}=\left[\begin{array}{rrr}
-1 & 0 & 0 \\
1 & -2 & 1 \\
1 & -1 & 0
\end{array}\right]
$$

当 $f(x)=\arctan \frac{x}{2}$ 时,$f(2)=\frac{\pi}{4}, f^{\prime}(2)=\frac{1}{4}$ ,故

$$
\arctan \frac{A}{2}=\frac{1}{4}\left[\begin{array}{ccc}
\pi & 0 & 0 \\
1 & \pi-1 & 1 \\
1 & -1 & \pi+1
\end{array}\right]
$$

当 $f(x)=x^{10}$ 时,$f(2)=2^{10}, f^{\prime}(2)=10 \times 2^{9}$ ,故

$$
A^{10}=2^{9}\left[\begin{array}{ccc}
2 & 0 & 0 \\
10 & -8 & 10 \\
10 & -10 & 12
\end{array}\right]
$$

例 6.3 . 2 设

$$
A=\left[\begin{array}{rrr}
-1 & 1 & 0 \\
-4 & 3 & 0 \\
1 & 0 & 2
\end{array}\right]
$$

试求矩阵函数 $f(\boldsymbol{A})$ 的拉格朗日—西勒维斯特内插多项式表示并计算 $\sin \frac{\pi}{2} \boldsymbol{A}, \mathrm{e}^{\boldsymbol{i} \boldsymbol{A}}$ .\\
解 $\boldsymbol{A}$ 的最小多项式

$$
\varphi_{A}(x)=(x-1)^{2}(x-2)
$$

由式(6.3.3)得( $k=1,2$ )

$$
\varphi_{1}(x)=x-2, \quad \varphi_{2}(x)=(x-1)^{2}
$$

由式(6.3.2)得( $\lambda_{1}=1, \lambda_{2}=2$ )

$$
\begin{gathered}
a_{11}=\left.\frac{f(x)}{\varphi_{1}(x)}\right|_{x=1}=-f(1) \\
a_{12}=\left.\frac{\mathrm{d}}{\mathrm{~d} x}\left(\frac{f(x)}{\varphi_{1}(x)}\right)\right|_{x=1}=-f^{\prime}(1)-f(1) \\
a_{21}=\left.\frac{f(x)}{\varphi_{2}(x)}\right|_{x=2}=f(2)
\end{gathered}
$$

代人式(6.3.1)得

$$
\begin{aligned}
p(x)= & {\left[-f(1)-\left(f^{\prime}(1)+f(1)\right)(x-1)\right] } \\
& (x-2)+f(2)(x-1)^{2}
\end{aligned}
$$

因此 $f(\boldsymbol{A})$ 的拉格朗日—西勒维斯特内插多项式表示为


\begin{align*}
f(\boldsymbol{A})= & P(\boldsymbol{A})=\left[-f(1) \boldsymbol{E}-\left(f^{\prime}(1)+f(1)\right)(\boldsymbol{A}-\boldsymbol{E})\right] \\
& (\boldsymbol{A}-2 \boldsymbol{E})+f(2)(\boldsymbol{A}-\boldsymbol{E})^{2} \tag{*}
\end{align*}


当 $f(x)=\sin \frac{\pi}{2} x$ 时,$f(1)=1, f^{\prime}(1)=0, f(2)=0$ ,代人式( $*$ )得

$$
\sin \frac{\pi}{2} A=-A(A-2 E)=\left[\begin{array}{rrr}
1 & 0 & 0 \\
0 & 1 & 0 \\
1 & -1 & 0
\end{array}\right]
$$

当 $f(x)=\mathrm{e}^{t x}$ 时,$f(1)=\mathrm{e}^{t}, f^{\prime}(1)=t \mathrm{e}^{t}, f(2)=\mathrm{e}^{2 t}$ ,代入式(*)得

$$
\begin{aligned}
\mathrm{e}^{t A} & =\left[-\mathrm{e}^{t} \boldsymbol{E}-\mathrm{e}^{t}(1+t)(\boldsymbol{A}-\boldsymbol{E})\right](\boldsymbol{A}-2 \boldsymbol{E})+\mathrm{e}^{2 t}(\boldsymbol{A}-\boldsymbol{E})^{2} \\
& =\mathrm{e}^{t}[t \boldsymbol{E}-(t+1) \boldsymbol{A}](\boldsymbol{A}-2 \boldsymbol{E})+\mathrm{e}^{2 t}(\boldsymbol{A}-\boldsymbol{E})^{2} \\
& =\mathrm{e}^{t}\left[\begin{array}{lcc}
-10 t-7 & 3 t+2 & 0 \\
-4 t & -1 & 0 \\
-2 & t+1 & 0
\end{array}\right]+\mathrm{e}^{2 t}\left[\begin{array}{rcc}
0 & 0 & 0 \\
0 & 0 & 0 \\
-1 & 1 & 1
\end{array}\right]
\end{aligned}
$$

\section*{二、矩阵函数的多项式表示}
矩阵函数的拉格朗日—西勒维斯特内插多项式表示有其独特的形式与作用,缺点是公式不易记。若要计算矩阵函数,计算量不小。

在公式(6.3.1)的基础上进一步简化计算,不难看到,式(6.3.1)的多项式是一个 $m-1$ 次多项式,即


\begin{equation*}
p(x)=a_{0}+a_{1} x+a_{2} x^{2}+\cdots+a_{m-1} x^{m-1} \tag{6.3.5}
\end{equation*}


此式中的系数 $a_{0}, a_{1}, \cdots, a_{m-1}$ 可由


\begin{equation*}
p^{k}\left(\lambda_{i}\right)=f^{(k)}\left(\lambda_{i}\right) \quad\left(i=1,2, \cdots, s ; k=1,2, \cdots, d_{i}-1\right) \tag{6.3.6}
\end{equation*}


确定,这时所求的矩阵函数


\begin{equation*}
f(A)=p(A)=a_{0} E+a_{1} A+a_{2} A^{2}+\cdots+a_{m-1} A^{m-1} \tag{6.3.7}
\end{equation*}


式(6.3.6)称为矩阵函数的多项式表示。

例6.3.3 设

$$
\boldsymbol{A}=\left[\begin{array}{rrr}
2 & 0 & 0 \\
1 & 1 & 1 \\
1 & -1 & 3
\end{array}\right]
$$

试求矩阵函数 $f(\boldsymbol{A})$ 的多项式表示并计算 $\mathbf{e}^{\boldsymbol{A} t}$ 。\\
解 $\boldsymbol{A}$ 的最小多项式

$$
\varphi_{A}(x)=(x-2)^{2}
$$

由式(6.3.5)知

$$
p(x)=a_{0}+a_{1} x, \quad p^{\prime}(x)=a_{1}
$$

把 $f(2), f^{\prime}(2)$ 代入上二式

$$
f(2)=p(2)=a_{0}+2 a_{1}, f^{\prime}(2)=p^{\prime}(2)=a_{1}
$$

解之得

$$
a_{0}=f(2)-2 f^{\prime}(2), \quad a_{1}=f^{\prime}(2)
$$

于是矩阵函数 $f(\boldsymbol{A})$ 的多项式表示为

$$
f(\boldsymbol{A})=p(\boldsymbol{A})=\left[f(2)-2 f^{\prime}(2)\right] \boldsymbol{E}+f^{\prime}(2) \boldsymbol{A}
$$

当 $f(x)=\mathrm{e}^{t x}$ 时,$f(2)=\mathrm{e}^{2 t}, f^{\prime}(2)=t \mathrm{e}^{2 t}$ 将其代入上式可得

$$
\mathrm{e}^{A t}=\mathrm{e}^{2 t}\left[\begin{array}{ccc}
1 & 0 & 0 \\
t & 1-t & t \\
t & -t & 1+t
\end{array}\right]
$$

例6.3.4 设

$$
A=\left[\begin{array}{rrr}
-1 & 1 & 0 \\
-4 & 3 & 0 \\
1 & 0 & 2
\end{array}\right]
$$

试求矩阵函数 $f(\boldsymbol{A})$ 的多项式表示,并计算 $\sin \frac{\pi}{2} \boldsymbol{A}$ .\\
解 $\boldsymbol{A}$ 的最小多项式

$$
\varphi_{A}(x)=(x-1)^{2}(x-2)
$$

由式(6.3.5)知

$$
p(x)=a_{0}+a_{1} x+a_{2} x^{2}, \quad p^{\prime}(x)=a_{1}+2 a_{2} x
$$

把 $f(1), f(2), f^{\prime}(1)$ 代入上二式得

$$
\begin{gathered}
f(1)=p(1)=a_{0}+a_{1}+a_{2}, \quad f^{\prime}(1)=p^{\prime}(1)=a_{1}+2 a_{2} \\
f(2)=p(2)=a_{0}+2 a_{1}+4 a_{2}
\end{gathered}
$$

解之得

$$
\begin{gathered}
a_{0}=f(2)-2 f^{\prime}(1), \quad a_{1}=2 f(1)+3 f^{\prime}(1)-2 f(2) \\
a_{2}=f(2)-f(1)-f^{\prime}(1)
\end{gathered}
$$

所以,$f(\boldsymbol{A})$ 的多项式表示为

$$
\begin{aligned}
f(\boldsymbol{A})= & {\left[f(2)-2 f^{\prime}(1)\right] \boldsymbol{E}+\left[2 f(1)+3 f^{\prime}(1)-2 f(2)\right] \boldsymbol{A}+} \\
& {\left[f(2)-f(1)-f^{\prime}(1)\right] \boldsymbol{A}^{2} }
\end{aligned}
$$

当 $f(x)=\sin \frac{\pi}{2} x$ 时,$f(1)=1, f^{\prime}(1)=0, f(2)=0$ ,故

$$
f(\boldsymbol{A})=2 \boldsymbol{A}-\boldsymbol{A}^{2}=\left[\begin{array}{rrr}
1 & 0 & 0 \\
0 & 1 & 0 \\
1 & -1 & 0
\end{array}\right]
$$

\section*{§6.4 矩阵函数的冪级数表示}
正如微积分学的幂级数理论一样,在矩阵分析中用矩阵幂级数表示矩阵函数是常用的方法。

在定理5.5.4中给出矩阵幂级数


\begin{gather*}
\sum_{k=0}^{\infty} c_{k} \boldsymbol{A}^{k}=\boldsymbol{P} \operatorname{diag}\left(\sum_{k=0}^{\infty} c_{k} \boldsymbol{J}_{1}^{k}\left(\lambda_{1}\right), \sum_{k=0}^{\infty} c_{k} \boldsymbol{J}_{2}^{k}\left(\lambda_{2}\right), \cdots,\right. \\
\sum_{k=0}^{\infty} c_{k} \boldsymbol{J}_{i}^{k}\left(\lambda_{i}\right)=\left[\begin{array}{ccc}
\left.\sum c_{k} \lambda_{i}^{k} c_{k} \boldsymbol{J}_{r}^{k}\left(\lambda_{r}\right)\right) \boldsymbol{P}^{-1} & & \\
\sum c_{k} \mathrm{C}_{k}^{1} \lambda_{i}^{k-1} & \cdots & \sum c_{k} \mathrm{C}_{k}^{d_{i}-1} \lambda_{i}^{k-d_{i}+1} \\
\sum c_{k} \lambda_{i}^{k} & \ddots & \vdots \\
& \ddots & \sum c_{k} \mathrm{C}_{k}^{1} \lambda_{i}^{k-1} \\
& & \sum c_{k} \lambda_{i}^{k}
\end{array}\right]_{d_{i} \times d_{i}} \tag{6.4.1}
\end{gather*}


设函数 $f(x)$ 在 $|x|<R$ 可以展开成幂级数

$$
f(x)=\sum_{k=0}^{\infty} c_{k} x^{k} \quad|x|<R
$$

若 $\boldsymbol{A}$ 的谱半径 $\rho$ 满足 $\rho<R$ ,则下列等式

$$
\begin{gathered}
f\left(\lambda_{i}\right)=\sum_{k=0}^{\infty} c_{k} \lambda_{i}^{k} \\
f^{\prime}\left(\lambda_{i}\right)=\sum_{k=0}^{\infty} c_{k} \mathrm{C}_{k}^{1} \lambda_{i}^{k-1} \\
\frac{1}{2!} f^{\prime \prime}\left(\lambda_{i}\right)=\sum_{k=0}^{\infty} c_{k} \mathrm{C}_{k}^{2} \lambda_{i}^{k-2} \\
\vdots \\
\frac{1}{\left(d_{i}-1\right)!} f^{\left(d_{i}-1\right)}\left(\lambda_{i}\right)=\sum_{k=0}^{\infty} c_{k} \mathrm{C}_{k}^{d_{i}-1} \lambda_{i}^{k-d_{i}+1}
\end{gathered}
$$

成立.其中

$$
\begin{array}{cl}
\mathrm{C}_{k}^{l}=\frac{k(k-1) \cdots(k-l+1)}{l!} & (k \geqslant l) \\
\mathrm{C}_{k}^{l}=0 & (k<l)
\end{array}
$$

因此式(6.4.2)右端矩阵可写为

\[
\left[\begin{array}{ccccc}
f\left(\lambda_{i}\right) & f^{\prime}\left(\lambda_{i}\right) & \frac{1}{2!} f^{\prime \prime}\left(\lambda_{i}\right) & \cdots & \frac{1}{\left(d_{i}-1\right)!} f^{\left(d_{i}-1\right)}\left(\lambda_{i}\right)  \tag{6.4.3}\\
& f\left(\lambda_{i}\right) & f^{\prime}\left(\lambda_{i}\right) & & \vdots \\
& & \ddots & \ddots & f^{\prime}\left(\lambda_{i}\right) \\
& & & \ddots & f\left(\lambda_{i}\right)
\end{array}\right]_{d_{i} \times d_{i}}
\]

根据式(6.2.3)知它恰等于 $f\left(J_{i}\right)$ 。因此由式(6.4.1)知


\begin{equation*}
\sum_{k=0}^{\infty} c_{k} A^{k}=f(A) \tag{6.4.4}
\end{equation*}


此即\\
定理 6.4.1 设 $\boldsymbol{A} \in C^{n \times n}, \boldsymbol{A}$ 的谱半径为 $\rho$ ,若函数 $f(x)$ 的幂级数表示式为

$$
f(x)=\sum_{k=0}^{\infty} c_{k} x^{k} \quad|x|<R
$$

则当 $\rho<R$ 时

$$
f(\boldsymbol{A})=\sum_{k=0}^{\infty} c_{k} \boldsymbol{A}^{k}
$$

根据定理6.4.1可以得到一系列矩阵函数的幂级数表示式。因为

$$
\begin{array}{cr}
\mathrm{e}^{x}=1+x+\frac{1}{2!} x^{2}+\cdots+\frac{1}{n!} x^{n}+\cdots & |x|<\infty \\
\sin x=x-\frac{1}{3!} x^{3}+\frac{1}{5!} x^{5}-\cdots+(-1)^{n} \frac{1}{(2 n+1)!} x^{2 n+1}+\cdots & |x|<\infty \\
\cos x=1-\frac{1}{2!} x^{2}+\frac{1}{4!} x^{4}-\cdots+(-1)^{n} \frac{1}{(2 n)!} x^{2 n}+\cdots & |x|<\infty \\
(1+x)^{-1}=1-x+x^{2}-x^{3}+\cdots+(-1)^{n} x^{n}+\cdots & |x|<1 \\
\ln (1+x)=x-\frac{1}{2} x^{2}+\frac{1}{3} x^{3}-\cdots+(-1)^{n+1} \frac{1}{n} x^{n}+\cdots & -1<x \leqslant 1 \\
(1-x)^{-1}=1+x+x^{2}+\cdots+x^{n}+\cdots & |x|<1
\end{array}
$$

故

$$
\begin{gathered}
\mathrm{e}^{A}=E+A+\frac{1}{2!} A^{2}+\cdots+\frac{1}{n!} A^{n}+\cdots \quad(\rho<\infty) \\
\sin A=A-\frac{1}{3!} A^{3}+\frac{1}{5!} A^{5}-\cdots+(-1)^{n} \frac{1}{(2 n+1)!} A^{2 n+1}+\cdots \\
(\rho<\infty)
\end{gathered}
$$

$$
\begin{array}{cc}
\cos A=E-\frac{1}{2!} A^{2}+\frac{1}{4!} A^{4}-\cdots+(-1)^{n} \frac{1}{(2 n)!} A^{2 n}+\cdots & (\rho<\infty) \\
(E+A)^{-1}=E-A+A^{2}-A^{3}+\cdots+(-1)^{n} A^{n}+\cdots & (\rho<1) \\
\ln (E+A)=A-\frac{1}{2} A^{2}+\frac{1}{3} A^{3}-\cdots+(-1)^{n+1} \frac{1}{n} A^{n}+\cdots & (\rho<1) \\
(E-A)^{-1}=E+A+A^{2}+\cdots+\cdots+A^{n}+\cdots & (\rho<1)
\end{array}
$$

注1 需要说明的是 $(\boldsymbol{E}-\boldsymbol{A})^{-1}=\boldsymbol{E}+\boldsymbol{A}+\boldsymbol{A}^{2}+\cdots+\boldsymbol{A}^{n}+\cdots(\boldsymbol{\rho}<1)$ 就是上一章定理 5.5.5 的结果。

注2 结合微积分学中函数幂级数展开式理论可知下列结果也对。

$$
\begin{aligned}
\sin 2 A=2 A & -\frac{1}{3!}(2 A)^{3}+\frac{1}{5!}(2 A)^{5}+\cdots+(-1)^{n} \frac{1}{(2 n+1)!}(2 A)^{2 n+1}+\cdots(\rho<\infty) \\
(2 E+A)^{-1} & =\frac{1}{2}\left(E+\frac{1}{2} A\right)^{-1} \\
& =\frac{1}{2}\left(E-\frac{1}{2} A+\frac{1}{2^{2}} A^{2}-\frac{1}{2^{3}} A^{3}+\cdots+(-1)^{n} \frac{1}{2^{n}} A^{n}+\cdots\right) \\
& =\frac{1}{2} E-\frac{1}{2^{2}} A+\frac{1}{2^{3}} A^{2}-\frac{1}{2^{4}} A^{3}+\cdots+(-1)^{n} \frac{1}{2^{n+1}} A^{n}+\cdots(\rho<2) \\
(E & -A)^{-2}=E+2 A+3 A^{2}+4 A^{3}+\cdots+n A^{n-1}+\cdots(\rho<1)
\end{aligned}
$$

例6.4.1 已知

$$
A=\left[\begin{array}{ccc}
\frac{1}{6} & 0 & 0 \\
0 & \frac{1}{2} & 0 \\
0 & 1 & \frac{1}{2}
\end{array}\right]
$$

试证级数 $\sum_{k=1}^{\infty} k A^{k-1}$ 收玫,且求级数和 $\sum_{k=1}^{\infty} k A^{k-1}$ .\\
解 幂级数 $\sum_{k=1}^{\infty} k x^{k-1}$ 的收敛半径 $R=1$ ,而 $A$ 的特征值为 $\lambda_{1}=\lambda_{2}=\frac{1}{2}, \lambda_{3}= \frac{1}{6}$ ,此即 $A$ 的谱半径 $\rho(A)=\frac{1}{2}<R$ ,所以矩阵幂级数 $\sum_{k=1}^{\infty} k A^{K-1}$ 收玫。

根据微积分学知 $\sum_{k=0}^{\infty} x^{k}=(1-x)^{-1} .|x|<1$ ,且 $\sum_{k=1}^{\infty} k x^{k-1}=(1-x)^{-2} .|x|<1$.命 $f(x)=(1-x)^{-2}$ .于是 $\sum_{k=1}^{\infty} k \boldsymbol{A}^{k-1}=f(\boldsymbol{A})$ .

不难验证 $\boldsymbol{P}^{-1} \boldsymbol{A} \boldsymbol{P}=\boldsymbol{J}=\left[\begin{array}{ccc}\frac{1}{6} & 0 & 0 \\ 0 & \frac{1}{2} & 1 \\ 0 & 0 & \frac{1}{2}\end{array}\right] \quad$ 其中 $\boldsymbol{P}=\boldsymbol{P}^{-1}=\left[\begin{array}{ccc}1 & 0 & 0 \\ 0 & 0 & 1 \\ 0 & 1 & 0\end{array}\right]$ ,\\
$f\left(\frac{1}{6}\right)=\frac{36}{25}, f\left(\frac{1}{2}\right)=4, f^{\prime}\left(\frac{1}{2}\right)=16$.

$$
f(J)=\left[\begin{array}{ccc}
f\left(\frac{1}{6}\right) & 0 & 0 \\
0 & f\left(\frac{1}{2}\right) & f^{\prime}\left(\frac{1}{2}\right) \\
0 & 0 & f\left(\frac{1}{2}\right)
\end{array}\right]=\left[\begin{array}{ccc}
\frac{36}{25} & 0 & 0 \\
0 & 4 & 16 \\
0 & 0 & 4
\end{array}\right]
$$

于是

$$
f(A)=P f(J) P^{-1}=\left[\begin{array}{ccc}
\frac{36}{25} & 0 & 0 \\
0 & 4 & 0 \\
0 & 16 & 4
\end{array}\right]
$$

注 本题若用上述注 2.可知 $\sum_{k=1}^{\infty} k A^{k-1}=(\boldsymbol{E}-\boldsymbol{A})^{-2}$\\
于是

$$
\boldsymbol{E}-\boldsymbol{A}=\left[\begin{array}{ccc}
\frac{5}{6} & 0 & 0 \\
0 & \frac{1}{2} & 0 \\
0 & -1 & \frac{1}{2}
\end{array}\right],(\boldsymbol{E}-\boldsymbol{A})^{-1}=\left[\begin{array}{ccc}
\frac{6}{5} & 0 & 0 \\
0 & 2 & 0 \\
0 & 4 & 2
\end{array}\right]
$$

所以

$$
(E-A)^{-2}=\left[\begin{array}{ccc}
\frac{36}{25} & 0 & 0 \\
0 & 4 & 0 \\
0 & 16 & 4
\end{array}\right]
$$

例 6.4 .2 已知

$$
A=\left[\begin{array}{ccc}
2 & 0 & 0 \\
1 & 1 & 1 \\
1 & -1 & 3
\end{array}\right]
$$

试求 $\sum_{k=0}^{\infty} \frac{k+1}{10^{k+1}} A^{k}$ 之和.

解 由微积分学知

$$
\begin{aligned}
\sum_{k=0}^{\infty} \frac{k+1}{10^{k+1}} x^{k} & =\frac{1}{10}+\frac{2}{10^{2}} x^{2}+\frac{3}{10^{3}} x^{3}+\cdots+\frac{k+1}{10^{k+1}} x^{k+1}+\cdots \\
& =\left\{\frac{x}{10}+\left(\frac{x}{10}\right)^{2}+\cdots+\left(\frac{x}{10}\right)^{k+1}+\cdots\right\}^{\prime} \\
& =\left\{1+\frac{x}{10}+\left(\frac{x}{10}\right)^{2}+\cdots+\left(\frac{x}{10}\right)^{k+1}+\cdots\right\}^{\prime} \\
& =\left\{\left(1-\frac{x}{10}^{-1}\right)\right\}^{\prime}=\frac{1}{10}\left(1-\frac{x}{10}\right)^{-2} \quad\left|\frac{x}{10}\right|<1
\end{aligned}
$$

因此命 $f(x)=\frac{1}{10}\left(1-\frac{x}{10}\right)^{-2}$\\
则

$$
\sum_{k=0}^{\infty} \frac{k+1}{10^{k+1}} A^{k}=f(A) \quad(\rho(A)<10)
$$

由例 6.3.1 不难知 $\rho(\boldsymbol{A})=2<10$ ,所以 $\sum_{k=0}^{\infty} \frac{k+1}{10^{k+1}}=f(\boldsymbol{A})$ .\\
由例6.3.1知

$$
f(\boldsymbol{A})=\left[\begin{array}{ccc}
f(2) & 0 & 0 \\
f^{\prime}(2) & f(2)-f^{\prime}(2) & f^{\prime}(2) \\
f^{\prime}(2) & -f^{\prime}(2) & f(2)+f^{\prime}(2)
\end{array}\right]
$$

$f(2)=\frac{5}{32}, f^{\prime}(2)=\frac{5}{128}$ ,于是

$$
f(A)=\frac{5}{128}\left[\begin{array}{ccc}
4 & 0 & 0 \\
1 & 3 & 1 \\
1 & -1 & 5
\end{array}\right]
$$

本题由前文介绍知

$$
\sum_{k=0}^{\infty} \frac{k+1}{10^{k+1}} A^{k}=\frac{1}{10}\left(E-\frac{A}{10}\right)^{-2}
$$

不难计算可得

$$
\begin{gathered}
\boldsymbol{E}-\frac{\boldsymbol{A}}{10}=\left[\begin{array}{ccc}
\frac{8}{10} & 0 & 0 \\
-\frac{1}{10} & \frac{9}{10} & -\frac{1}{10} \\
-\frac{1}{10} & \frac{1}{10} & \frac{7}{10}
\end{array}\right] \\
\frac{1}{10}\left(\boldsymbol{E}-\frac{\boldsymbol{A}}{10}^{-2}\right)=\frac{5}{128}\left[\begin{array}{ccc}
4 & 0 & 0 \\
1 & 3 & 1 \\
1 & -1 & 5
\end{array}\right]
\end{gathered}
$$

\section*{§6.5 矩阵指数函数与矩阵三角函数}
在§6.4节中知,对于任何 $n$ 阶方阵 $A$ 都有


\begin{gather*}
\mathbf{e}^{A t}=\sum_{k=0}^{\infty} \frac{A^{k} t^{k}}{k!}  \tag{6.5.1}\\
\sin A t=\sum_{k=0}^{\infty} \frac{(-1)^{k} A^{2 k+1} t^{2 k+1}}{(2 k+1)!}  \tag{6.5.2}\\
\cos A t=\sum_{k=0}^{\infty} \frac{(-1)^{k} A^{2 k} t^{2 k}}{(2 k)!} \tag{6.5.3}
\end{gather*}


以上三个矩阵幂级数对所有有限参数 $t$ 都是绝对收敛的.\\
定理6.5.1 设 $A, B \in C^{n \times n}$ ,则\\
(1) $\mathrm{e}^{A \lambda} \mathrm{e}^{A \mu}=\mathrm{e}^{A(\lambda+\mu)} \quad \lambda, \mu \in \mathrm{C}$\\
(2) $\mathrm{e}^{\mathrm{i} \boldsymbol{A}}=\cos \boldsymbol{A}+\mathrm{i} \sin \boldsymbol{A} \quad(\mathrm{i}=\sqrt{-1})$\\
(3)当 $\boldsymbol{A B}=\boldsymbol{B A}$ 时,有

$$
\mathrm{e}^{A+B}=\mathrm{e}^{A} \mathrm{e}^{B}=\mathrm{e}^{B} \mathrm{e}^{A}
$$

(4)对于任何矩阵 $A, \mathrm{e}^{A}$ 总是可逆的,且 $\left(\mathrm{e}^{A}\right)^{-1}=\mathrm{e}^{-A}$\\
(5)$\frac{\mathrm{d}}{\mathrm{d} t}\left(\mathrm{e}^{A t}\right)=A \mathrm{e}^{A t}=\mathrm{e}^{A t} A$\\
(6) $\operatorname{det} \mathrm{e}^{A}=\mathrm{e}^{\operatorname{tr} A}$ ,其中 $\operatorname{tr} A=a_{11}+a_{22}+\cdots+a_{n n}$ 是 $A$ 的迹\\
证明(1)由式(6.5.1)知

$$
\begin{aligned}
\mathrm{e}^{A(\lambda+\mu)} & =\sum_{k=0}^{\infty} \frac{A^{k}(\lambda+\mu)^{k}}{k!} \\
& =\sum_{k=0}^{\infty} \frac{1}{k!}\left[\sum_{m=0}^{k} \mathrm{C}_{k}^{m}(A \lambda)^{m}(A \mu)^{k-m}\right]
\end{aligned}
$$

若命 $k-m=l$ ,则

$$
\mathbf{e}^{\boldsymbol{A}(\lambda+\mu)}=\sum_{m=0}^{\infty} \sum_{l=0}^{\infty} \mathrm{C}_{l+m}^{m}(\boldsymbol{A} \lambda)^{m}(\boldsymbol{A} \mu)^{l} \frac{1}{(l+m)!}
$$

但由于 $\mathrm{C}_{l+m}^{m}=\frac{(l+m)!}{l!m!}$ ,于是有

$$
\mathrm{e}^{A(\lambda+\mu)}=\left(\sum_{m=0}^{\infty} \frac{(A \lambda)^{m}}{m!}\right)\left(\sum_{l=0}^{\infty} \frac{(A \mu)^{l}}{l!}\right)=\mathrm{e}^{A \lambda} \mathrm{e}^{A \mu}
$$

(2)由式(6.5.1)~式(6.5.3)知

$$
\begin{aligned}
\mathrm{e}^{\mathrm{i} A} & =\sum_{k=0}^{\infty} \frac{A^{k} \mathrm{i}^{k}}{k!} \\
& =E+\mathrm{i} A-\frac{1}{2!} A^{2}-\frac{1}{3!} \mathrm{i} A^{3}+\frac{1}{4!} A^{4}-\cdots
\end{aligned}
$$

$$
\begin{aligned}
& =\left(E-\frac{1}{2!} A^{2}+\frac{1}{4!} A^{4}-\cdots\right)+\mathrm{i}\left(A-\frac{1}{3!} A^{3}+\frac{1}{5!} A^{5}-\cdots\right) \\
& =\cos A+\mathrm{i} \sin A
\end{aligned}
$$

(3)因为当 $\boldsymbol{A B}=\boldsymbol{B A}$ 时,二项式公式

$$
(\boldsymbol{A}+\boldsymbol{B})^{k}=\sum_{m=0}^{k} \mathrm{C}_{k}^{m} \boldsymbol{A}^{k-m} \boldsymbol{B}^{m}
$$

成立,因此

$$
\begin{aligned}
\mathrm{e}^{A+B} & =\sum_{m=0}^{\infty} \frac{1}{k!}(A+B)^{k} \\
& =\sum_{m=0}^{\infty} \frac{1}{k!}\left(\sum_{m=0}^{k} \mathrm{C}_{k}^{m} A^{k-m} B^{m}\right)
\end{aligned}
$$

由证明(1)过程中知上式右端可写成

故

$$
\begin{gathered}
\sum_{m=0}^{\infty} \frac{A^{m}}{m!} \sum_{l=0}^{\infty} \frac{B^{l}}{l!} \text { 或 } \sum_{l=0}^{\infty} \frac{B^{l}}{l!} \sum_{m=0}^{\infty} \frac{A^{m}}{m!} \\
\mathrm{e}^{A+B}=\mathrm{e}^{A} \mathrm{e}^{B}=\mathrm{e}^{B} \mathrm{e}^{A}
\end{gathered}
$$

(4)因为矩阵指数函数 $\mathrm{e}^{0}=\boldsymbol{E}$ ,由(1)得

故

$$
\begin{gathered}
\mathrm{e}^{A+(-A)}=\mathrm{e}^{A} \mathrm{e}^{-A}=E \\
\left(\mathrm{e}^{A}\right)^{-1}=\mathrm{e}^{-A}
\end{gathered}
$$

(5)由于式(6.5.1)给出的幂级数对给定 $A$ 与对一切 $t$ 都是绝对收敛的,且对 $t$ 是一致收敛,从而可对式(6.5.1)逐项求导,则

$$
\begin{aligned}
\frac{\mathrm{d}}{\mathrm{~d} t}\left(\mathrm{e}^{A t}\right) & =\sum_{k=0}^{\infty} \frac{\mathrm{d}}{\mathrm{~d} t}\left(\frac{A^{k} t^{k}}{k!}\right)=\sum_{k=1}^{\infty} \frac{A^{k} t^{k-1}}{(k-1)!} \\
& =A \sum_{l=0}^{\infty} \frac{A^{l} t^{l}}{l!}=\left(\sum_{l=0}^{\infty} \frac{A^{l} t^{l}}{l!}\right) A \\
& =A \mathrm{e}^{A t}=\mathrm{e}^{A t} A
\end{aligned}
$$

(6)设 $A=P J P^{-1}=P \operatorname{diag}\left(J_{1}, J_{2}, \cdots, J_{r}\right) P^{-1}$ ,则

$$
\mathrm{e}^{A}=P \operatorname{diag}\left(\mathrm{e}^{J_{1}}, \mathrm{e}^{J_{2}}, \cdots, \mathrm{e}^{J_{r}}\right) P^{-1}
$$

而

$$
\mathbf{e}^{\boldsymbol{J}_{i}}=\left[\begin{array}{ccccc}
\mathrm{e}^{\lambda_{i}} & \mathrm{e}^{\lambda_{i}} & \frac{1}{2!} \mathrm{e}^{\lambda_{i}} & \cdots & \frac{1}{\left(d_{i}-1\right)!} \mathrm{e}^{\lambda_{i}} \\
& \mathrm{e}^{\lambda_{i}} & \mathrm{e}^{\lambda_{i}} & \ddots & \vdots \\
& & \ddots & \ddots & \mathrm{e}^{\lambda_{i}}
\end{array}\right]_{d_{i} \times d_{i}}
$$

所以

$$
\begin{aligned}
\operatorname{det} \mathrm{e}^{\boldsymbol{A}} & =\operatorname{det} \boldsymbol{P} \operatorname{det}\left(\operatorname{diag}\left(\mathrm{e}^{J_{1}}, \mathrm{e}^{J_{2}}, \cdots, \mathrm{e}^{J_{r}}\right)\right) \operatorname{det} \boldsymbol{P}^{-1} \\
& =\operatorname{det} \mathrm{e}^{J_{1}} \operatorname{det} \mathrm{e}^{J_{2} \cdots \operatorname{det}} \mathrm{e}^{J_{r}} \\
& =\mathrm{e}^{d_{1} \lambda_{1}} \mathrm{e}^{d_{2} \lambda_{2} \cdots} \mathrm{e}^{d_{r} \lambda_{r}}=\mathrm{e}^{d_{1} \lambda_{1}+d_{2} \lambda_{2}+\cdots+d_{r} \lambda_{r}}=\mathrm{e}^{\mathrm{tr} \boldsymbol{A}}
\end{aligned}
$$

根据定理6.5.1(2)可知

$$
\begin{gathered}
\mathrm{e}^{\mathrm{i} A}=\cos A+\mathrm{i} \sin A \\
\mathrm{e}^{-\mathrm{i} A}=\cos A-\mathrm{i} \sin A
\end{gathered}
$$

于是


\begin{align*}
& \cos A=\frac{1}{2}\left(\mathrm{e}^{\mathrm{i} A}+\mathrm{e}^{-\mathrm{i} A}\right)  \tag{6.5.4}\\
& \sin A=\frac{1}{2 \mathrm{i}}\left(\mathrm{e}^{\mathrm{i} A}-\mathrm{e}^{-\mathrm{i} A}\right) \tag{6.5.5}
\end{align*}


类似地


\begin{align*}
& \cos A t=\frac{1}{2}\left(\mathrm{e}^{\mathrm{i} A t}+\mathrm{e}^{-\mathrm{i} A t}\right)  \tag{6.5.6}\\
& \sin A t=\frac{1}{2 \mathrm{i}}\left(\mathrm{e}^{\mathrm{i} A t}-\mathrm{e}^{-\mathrm{i} A t}\right) \tag{6.5.7}
\end{align*}


根据这几个等式与定理6.5.1可得定理6.5.2.\\
定理 6.5.2 设 $A 、 B \in C^{n \times n}$ 则\\
(1)$\frac{\mathrm{d}(\sin \boldsymbol{A} t)}{\mathrm{d} t}=\boldsymbol{A} \cos \boldsymbol{A} t=(\cos \boldsymbol{A} t) \boldsymbol{A}$

$$
\frac{\mathrm{d}(\cos A t)}{\mathrm{d} t}=-A \sin A t=-(\sin A t) A
$$

(2) $\sin ^{2} A+\cos ^{2} A=E$\\
$\sin (-A)=-\sin A$\\
$\cos (-\boldsymbol{A})=\cos \boldsymbol{A}$\\
(3)若 $\boldsymbol{A B}=\boldsymbol{B A}$ ,有

$$
\begin{aligned}
& \sin (A+B)=\sin A \cos B+\cos A \sin B \\
& \sin (A-B)=\sin A \cos B-\cos A \sin B \\
& \cos (A+B)=\cos A \cos B-\sin A \sin B \\
& \cos (A-B)=\cos A \cos B+\sin A \sin B
\end{aligned}
$$

证明(1)由式(6.5.7)得

$$
\begin{aligned}
\frac{\mathrm{d}}{\mathrm{~d} t}(\sin A t) & =\frac{1}{2 \mathrm{i}} \frac{\mathrm{~d}}{\mathrm{~d} t}\left(\mathrm{e}^{\mathrm{i} A t}-\mathrm{e}^{-\mathrm{i} A t}\right) \\
& =\frac{1}{2 \mathrm{i}}\left(\mathrm{i} A \mathrm{e}^{\mathrm{i} A t}+\mathrm{i} A \mathrm{e}^{-\mathrm{i} A t}\right) \\
& =A \cdot \frac{1}{2}\left(\mathrm{e}^{\mathrm{i} A t}+\mathrm{e}^{-\mathrm{i} A t}\right) \\
& =A \cos A t
\end{aligned}
$$

类似地

$$
\begin{aligned}
\frac{\mathrm{d}}{\mathrm{~d} t}(\cos \boldsymbol{A} t) & =\frac{1}{2 \mathrm{i}}\left(\mathrm{e}^{\mathrm{i} \boldsymbol{A} t} \mathrm{i} \boldsymbol{A}+\mathrm{e}^{-\mathrm{i} \boldsymbol{A} t} \mathrm{i} \boldsymbol{A}\right) \\
& =\frac{1}{2}\left(\mathrm{e}^{\mathrm{i} \boldsymbol{A} t}+\mathrm{e}^{-\mathrm{i} \boldsymbol{A} t}\right) \boldsymbol{A} \\
& =(\cos \boldsymbol{A} t) \boldsymbol{A}
\end{aligned}
$$

同理可证 $\frac{\mathrm{d}}{\mathrm{d} t}(\cos A t)=-A \sin A t=-(\sin A t) A$\\
(2)由式(6.5.4)与式(6.5.5)得

$$
\begin{aligned}
& \sin ^{2} \boldsymbol{A}+\cos ^{2} \boldsymbol{A}=\frac{1}{4}\left(\mathrm{e}^{\mathrm{i} \boldsymbol{A}}+\mathrm{e}^{-\mathrm{i} \boldsymbol{A}}+2 \boldsymbol{E}\right)-\frac{1}{4}\left(\mathrm{e}^{\mathrm{i} \boldsymbol{A}}+\mathrm{e}^{-\mathrm{i} \boldsymbol{A}}-2 \boldsymbol{E}\right)=\boldsymbol{E} \\
& \sin (-\boldsymbol{A})=\frac{1}{2 \mathrm{i}}\left(\mathrm{e}^{-\mathrm{i} \boldsymbol{A}}-\mathrm{e}^{\mathrm{i} \boldsymbol{A}}\right) \\
&=-\frac{1}{2 \mathrm{i}}\left(\mathrm{e}^{\mathrm{i} \boldsymbol{A}}-\mathrm{e}^{-\mathrm{i} \boldsymbol{A}}\right) \\
&=-\sin \boldsymbol{A} \\
& \cos (-\boldsymbol{A})=\frac{1}{2}\left(\mathrm{e}^{-\mathrm{i} \boldsymbol{A}}+\mathrm{e}^{\mathrm{i} \boldsymbol{A}}\right)=\cos \boldsymbol{A}
\end{aligned}
$$

(3)由式(6.5.4)与式(6.5.5)得

$$
\begin{aligned}
& \sin A \cos B+\cos A \sin B \\
& =\frac{1}{2 \mathrm{i}}\left(\mathrm{e}^{\mathrm{i} A}-\mathrm{e}^{-\mathrm{i} A}\right) \cdot \frac{1}{2}\left(\mathrm{e}^{\mathrm{i} B}+\mathrm{e}^{-\mathrm{i} B}\right)+\frac{1}{2}\left(\mathrm{e}^{\mathrm{i} A}+\mathrm{e}^{-\mathrm{i} A}\right) \cdot \frac{1}{2 \mathrm{i}}\left(\mathrm{e}^{\mathrm{i} B}-\mathrm{e}^{-\mathrm{i} B}\right) \\
& =\frac{1}{2 \mathrm{i}} \cdot \frac{1}{2}\left(\mathrm{e}^{\mathrm{i} A} \cdot \mathrm{e}^{\mathrm{i} B}+\mathrm{e}^{\mathrm{i} A} \mathrm{e}^{-\mathrm{i} B}-\mathrm{e}^{-\mathrm{i} A} \cdot \mathrm{e}^{\mathrm{i} B}-\mathrm{e}^{-\mathrm{i} A} \mathrm{e}^{-\mathrm{i} B}\right. \\
& \left.+\mathrm{e}^{\mathrm{i} A} \mathrm{e}^{\mathrm{i} B}-\mathrm{e}^{\mathrm{i} A} \mathrm{e}^{-\mathrm{i} B}+\mathrm{e}^{-\mathrm{i} A} \mathrm{e}^{\mathrm{i} B}-\mathrm{e}^{-\mathrm{i} A} \mathrm{e}^{-\mathrm{i} B}\right) \\
& =\frac{1}{2 \mathrm{i}}\left(\mathrm{e}^{\mathrm{i} A} \mathrm{e}^{\mathrm{i} B}-\mathrm{e}^{-\mathrm{i} A} \mathrm{e}^{-\mathrm{i} B}\right) \\
& =\frac{1}{2 \mathrm{i}}\left(\mathrm{e}^{\mathrm{i}(A+B)}-\mathrm{e}^{-\mathrm{i}(A+B)}\right) \\
& =\sin (A+B)
\end{aligned}
$$

其余三式可类似证得。

\section*{习 题}
6-1 设 $\boldsymbol{A}$ 为 $n$ 阶矩阵,$f(x)$ 为关于 $x$ 的函数,$f(\boldsymbol{A})$ 有定义,\\
(1)试证:$f\left(\boldsymbol{A}^{\mathrm{T}}\right)$ 有定义,且 $f\left(\boldsymbol{A}^{\mathrm{T}}\right)=[f(\boldsymbol{A})]^{\mathrm{T}}$ 。\\
(2)已知

$$
A=\left[\begin{array}{llll}
1 & 0 & 0 & 0 \\
1 & 1 & 0 & 0 \\
0 & 1 & 1 & 0 \\
0 & 0 & 1 & 1
\end{array}\right]
$$

求 $\mathrm{e}^{i A}, \sin A, \cos A, \ln A$ .\\
6-2 若 $\boldsymbol{A}$ 是可逆矩阵,$f(\boldsymbol{A})$ 有定义,那么 $f\left(\boldsymbol{A}^{-1}\right)$ 是否有定义?试举例说明.

6-3 设 $\boldsymbol{A}$ 是可逆矩阵,$f(x)=x^{\frac{s}{p}}$(其中,$p, q$ 均为正整数),求证:$f(\boldsymbol{A})$ 有定义,且 $[f(\boldsymbol{A})]^{p}=A^{q}$ 。

6-4 已知

$$
J_{0}\left(\lambda_{0}\right)=\left[\begin{array}{cccc}
\lambda_{0} & 1 & & \\
& \lambda_{0} & 1 & \\
& & \lambda_{0} & 1 \\
& & & \lambda_{0}
\end{array}\right]
$$

试求 $J_{0}^{2}\left(\lambda_{0}\right), J_{0}^{3}\left(\lambda_{0}\right), J_{0}^{4}\left(\lambda_{0}\right), J_{0}^{k}\left(\lambda_{0}\right),(k \geqslant 5)$ .\\
6-5 求矩阵 $\boldsymbol{A}$ 的最小多项式。已知\\
(1)$A=\left[\begin{array}{rrr}1 & 0 & 0 \\ 2 & 3 & -4 \\ 1 & 1 & -1\end{array}\right]$ ,\\
(2)$A=\left[\begin{array}{lll}1 & 1 & -1 \\ 2 & 4 & -5 \\ 1 & 2 & -2\end{array}\right]$\\
(3)$A=\left[\begin{array}{rrr}0 & -1 & 0 \\ 1 & 2 & 0 \\ 2 & 1 & 2\end{array}\right]$ ,\\
(4)$A=\left[\begin{array}{lll}0 & 1 & 1 \\ 1 & 0 & 1 \\ 1 & 1 & 0\end{array}\right]$

6-6 已知矩阵

$$
A=\left[\begin{array}{rrr}
1 & 0 & 0 \\
-1 & 2 & -1 \\
0 & 0 & 2
\end{array}\right]
$$

试求:(1)$f(A)$ 的 Jordan 的表示式;\\
(2)计算 $\mathrm{e}^{A}, \mathrm{e}^{A t}, \arctan \frac{A}{4}$ .\\
6-7 已知矩阵 $A$ ,求矩阵函数 $f(A)$ 的 Jordan 表示,并计算 $\mathrm{e}^{i A}, \sin \frac{\pi}{2} A$ , $\cos \pi A, \ln (E+A)$.\\
(1)$A=\left[\begin{array}{llll}2 & 1 & 0 & 0 \\ 0 & 2 & 0 & 0 \\ 0 & 0 & 1 & 1 \\ 0 & 0 & 0 & 1\end{array}\right]$ ,\\
(2)$A=\left[\begin{array}{llll}2 & 1 & 0 & 0 \\ 0 & 2 & 1 & 0 \\ 0 & 0 & 2 & 0 \\ 0 & 0 & 0 & 2\end{array}\right]$

6-8 已知矩阵

$$
A=\left[\begin{array}{rrr}
2 & 2 & 1 \\
-2 & 6 & 1 \\
0 & 0 & 4
\end{array}\right]
$$

试求:(1)$f(\boldsymbol{A})$ 的多项式表示式;\\
(2)计算 $\mathrm{e}^{A}, \mathrm{e}^{A t}, \sin A$ .\\
6-9 已知 $n$ 阶矩阵 $\boldsymbol{A}$ 的最小多项式为 $\varphi_{\boldsymbol{A}}(\lambda)$ ,写出矩阵函数 $f(\boldsymbol{A})$ 的 Lagrange-Sylvester 内插多项式表示与多项式表示。\\
(1)$\varphi_{A}(\lambda)=(\lambda-3)^{3}$ ,\\
(2)$\varphi_{A}(\lambda)=(\lambda-2)^{2}(\lambda-5)$

6-10 已知矩阵

$$
A=\left[\begin{array}{rrr}
2 & 1 & 0 \\
-1 & 0 & 0 \\
-2 & -1 & 2
\end{array}\right]
$$

试求矩阵函数 $f(\boldsymbol{A})$ 的 Lagrange-Sylvester 内插多项式表示,并用其计算矩阵函数 $\mathrm{e}^{t A}, \sin \pi A$ .

6-11 设 $\boldsymbol{A}$ 为 $n$ 阶矩阵,试证:\\
(1) $\mathrm{e}^{2 \pi \mathrm{i} E}=\boldsymbol{E}, \mathrm{e}^{2 \pi \mathrm{i} E+A}=\mathrm{e}^{A}$ ,这里 $\mathrm{i}^{2}=-1$ .\\
(2) $\sin 2 \pi E=0, \cos 2 \pi E=E$\\
(3) $\sin (2 \pi E+A)=\sin A$\\
(4)$\left|\mathrm{e}^{A}\right|=\mathrm{e}^{\operatorname{tr}(A)}$\\
(5)$\left\|\mathrm{e}^{A}\right\| \leqslant \mathrm{e}^{\|A\|}$\\
6-12 试证:如果 $\boldsymbol{A}$ 为反对称(反 Hermite)矩阵,那么 $\mathrm{e}^{\boldsymbol{A}}$ 为正交(酉)矩阵。\\
6-13 试证:如果 $\boldsymbol{A}$ 为 Hermite 矩阵,那么 $\mathrm{e}^{\mathrm{i} \boldsymbol{A}}$ 为西矩阵。\\
6-14 设 $\boldsymbol{A}, \boldsymbol{B} \in C^{n \times n}$ ,且 $\boldsymbol{A B}=\boldsymbol{B A}$ ,证明:\\
(1) $\sin (A+B)=\sin A \cos B+\cos A \sin B$\\
(2) $\cos (A+B)=\cos A \cos B-\sin A \sin B$\\
6-15 求矩阵幂级数 $\sum_{k=0}^{\infty} \frac{k+1}{10^{k}}\left[\begin{array}{ll}1 & 1 \\ 1 & 1\end{array}\right]^{k}$ 的和.\\
6-16 求矩阵幂级数 $\sum_{k=0}^{\infty} \frac{k+1}{10^{k}}\left[\begin{array}{ll}1 & 2 \\ 8 & 1\end{array}\right]^{k}$ 的和.\\
6-17 已知矩阵 $\boldsymbol{A}=\left[\begin{array}{cc}\frac{1}{2} & -a \\ -a & \frac{1}{2}\end{array}\right]$\\
(1)问当 $a$ 满足什么条件时,矩阵幂级数 $\sum_{k=1}^{\infty}(2 k+1) A^{k}$ 绝对收敛.\\
(2)取 $a=0$ 时,求上述幂级数的和.\\
6-18 已知矩阵 $\boldsymbol{A}=(10-\sqrt{10}) \boldsymbol{E}_{3}$ 。这里 $\boldsymbol{E}_{3}$ 是 3 阶单位矩阵。\\
(1)求证 矩阵幂级数 $\sum_{k=0}^{\infty} \frac{k+1}{10^{k}} \boldsymbol{A}^{k}$ 绝对收敛.\\
(2)求矩阵幂级数 $\sum_{k=0}^{\infty} \frac{k+1}{10^{k}} A^{k}$ 的收玫和.

\section*{第 $\leftarrow$ 章}
\section*{函数矩阵与矩阵微分方程}
函数矩阵、函数向量有不同于常数矩阵、常数向量的特殊性质。本章介绍函数矩阵(包括函数向量)的基本运算,函数矩阵的导数与积分,函数向量线性相关性的判别法则,最后作为应用简单介绍 $n$ 阶矩阵微分方程。

\section*{§7.1 函数矩阵对纯量的导数与积分}
\section*{一、函数矩阵的基本运算}
以实变量 $x$ 的实函数 $a_{i j}(x)$ 为元素的矩阵

$$
A(x)=\left[\begin{array}{cccc}
a_{11}(x) & a_{12}(x) & \cdots & a_{1 n}(x) \\
a_{21}(x) & a_{22}(x) & \cdots & a_{2 n}(x) \\
\vdots & \vdots & & \vdots \\
a_{m 1}(x) & a_{m 2}(x) & \cdots & a_{m n}(x)
\end{array}\right]
$$

称为函数矩阵,其中所有的元素 $a_{i j}(x)(i=1,2, \cdots, m ; j=1,2, \cdots, n)$ 都是定义在区间 $[a, b]$ 上的实函数。

当 $m=1$ 时,$A(x)$ 是函数行向量,当 $n=1$ 时,$A(x)$ 是函数列向量。\\
函数矩阵的加法、数量乘法、矩阵与矩阵的乘法、矩阵的转置与常数矩阵的相应运算完全相同。即

加法:设 $\boldsymbol{A}(x)=\left(a_{i j}(x)\right)_{m \times n}, \boldsymbol{B}(x)=\left(b_{i j}(x)\right)_{m \times n}$ ,它们的和 $\boldsymbol{A}(x)+\boldsymbol{B}(x)$定义为

$$
\boldsymbol{A}(x)+\boldsymbol{B}(x)=\left(a_{i j}(x)+b_{i j}(x)\right)_{m \times n}
$$

数量乘法:设 $k(x)$ 为 $x$ 的函数,$A(x)=\left(a_{i j}(x)\right)_{m \times n}, k(x)$ 与 $A(x)$ 的数乘积定义为

$$
k(x) A(x)=\left(k(x) a_{i j}(x)\right)_{m \times n}
$$

矩阵与矩阵的乘法:设 $\boldsymbol{A}(x)=\left(a_{i j}(x)\right)_{m \times k}, \boldsymbol{B}(x)=\left(b_{i j}(x)\right)_{k \times n}$ ,它们的积 $\boldsymbol{A}(x) \boldsymbol{B}(x)$ 定义为

$$
\boldsymbol{A}(x) \boldsymbol{B}(x)=\left(c_{i j}(x)\right)_{m \times n}
$$

其中 $c_{i j}(x)=\sum_{l=1}^{k} a_{i l}(x) b_{l j}(x) \quad(i=1,2, \cdots m ; j=1,2, \cdots, n)$ ,称 $\boldsymbol{A}(x)$ 与 $\boldsymbol{B}(x)$ 是可乘的。

转置:设 $m \times n$ 阶矩阵

$$
\boldsymbol{A}(x)=\left[\begin{array}{cccc}
a_{11}(x) & a_{12}(x) & \cdots & a_{1 n}(x) \\
a_{21}(x) & a_{22}(x) & \cdots & a_{2 n}(x) \\
\vdots & \vdots & & \vdots \\
a_{m 1}(x) & a_{m 2}(x) & \cdots & a_{m n}(x)
\end{array}\right]_{m \times n}
$$

它的转置矩阵 $\boldsymbol{A}^{\mathrm{T}}(x)$ 定义为

$$
\boldsymbol{A}^{\mathrm{T}}(x)=\left[\begin{array}{cccc}
a_{11}(x) & a_{12}(x) & \cdots & a_{m 1}(x) \\
a_{12}(x) & a_{22}(x) & \cdots & a_{m 2}(x) \\
\vdots & \vdots & & \vdots \\
a_{1 n}(x) & a_{2 n}(x) & \cdots & a_{m n}(x)
\end{array}\right]_{n \times m}
$$

由这些定义可知,它们的运算性质与常数矩阵的运算性质相同,这里不再一一列举。

定义7.1.1 设 $\boldsymbol{A}(x)=\left(a_{i j}(x)\right)$ 为 $n$ 阶函数矩阵,若存在 $n$ 阶函数矩阵 $\boldsymbol{B}(x)= \left(b_{i j}(x)\right)$ 使得对于任何 $x \in[a, b]$ 都有

$$
\boldsymbol{A}(x) \boldsymbol{B}(x)=\boldsymbol{B}(x) \boldsymbol{A}(x)=\boldsymbol{E}
$$

则称 $\boldsymbol{A}(x)$ 在 $[a, b]$ 上可逆, $\boldsymbol{B}(x)$ 是 $\boldsymbol{A}(x)$ 的逆矩阵,记为 $\boldsymbol{A}^{-1}(x)$ .\\
定理7.1.1 $n$ 阶矩阵 $A(x)$ 在 $[a, b]$ 上可逆的充要条件是 $|\boldsymbol{A}(x)|$ 在 $[a, b]$ 上处处不为零,并且

$$
\boldsymbol{A}^{-1}(x)=\frac{1}{|\boldsymbol{A}(x)|} \operatorname{adj} \boldsymbol{A}(x)
$$

其中

$$
\operatorname{adj} \boldsymbol{A}(x)=\left[\begin{array}{cccc}
\boldsymbol{A}_{11}(x) & \boldsymbol{A}_{21}(x) & \cdots & \boldsymbol{A}_{n 1}(x) \\
\boldsymbol{A}_{12}(x) & \boldsymbol{A}_{22}(x) & \cdots & \boldsymbol{A}_{n 2}(x) \\
\vdots & \vdots & & \vdots \\
\boldsymbol{A}_{1 n}(x) & \boldsymbol{A}_{2 n}(x) & \cdots & \boldsymbol{A}_{n n}(x)
\end{array}\right]
$$

$\boldsymbol{A}_{i j}(x)$ 是 $\boldsymbol{A}(x)$ 中元素 $a_{i j}(x)$ 的代数余子式。(证明从略)\\
例如函数矩阵

$$
A(x)=\left[\begin{array}{ll}
x & 1 \\
1 & x
\end{array}\right]
$$

在 $[2,3]$ 上的逆矩阵 $\boldsymbol{A}^{-1}(x)=\frac{1}{x^{2}-1}\left[\begin{array}{cc}x & -1 \\ -1 & x\end{array}\right]$ .在 $[0,2]$ 上 $\boldsymbol{A}(x)$ 不可逆(因为在 $x=1$ 时,$|A(x)|=0$ ).

\section*{二、函数矩阵的极限}
定义7.1.2 若 $\boldsymbol{A}(x)=\left(a_{i j}(x)\right)_{m \times n}$ 的所有各元素 $a_{i j}(x)$ 在 $x=x_{0}$ 处有极限,即

$$
\lim _{x \rightarrow x_{0}} a_{i j}(x)=a_{i j} \quad(i=1,2, \cdots, m ; j=1,2, \cdots, n)
$$

其中 $a_{i j}$ 为固定常数.则称 $A(x)$ 在 $x=x_{0}$ 处有极限,且记为

$$
\lim _{x \rightarrow x_{0}} A(x)=A
$$

其中

$$
A=\left[\begin{array}{cccc}
a_{11} & a_{12} & \cdots & a_{1 n} \\
a_{21} & a_{22} & \cdots & a_{2 n} \\
\vdots & \vdots & & \vdots \\
a_{m 1} & a_{m 2} & \cdots & a_{m n}
\end{array}\right]
$$

若 $\boldsymbol{A}(x)$ 的所有各元素 $a_{i j}(x)$ 在 $x=x_{0}$ 处连续,即

$$
\lim _{x \rightarrow x_{0}} a_{i j}(x)=a_{i j}\left(x_{0}\right) \quad(i=1,2, \cdots, m ; j=1,2, \cdots, n)
$$

则称 $\boldsymbol{A}(x)$ 在 $x=x_{0}$ 处连续,且记为

$$
\lim _{x \rightarrow x_{0}} \boldsymbol{A}(x)=\boldsymbol{A}\left(x_{0}\right)
$$

其中

$$
A\left(x_{0}\right)=\left[\begin{array}{cccc}
a_{11}\left(x_{0}\right) & a_{12}\left(x_{0}\right) & \cdots & a_{1 n}\left(x_{0}\right) \\
a_{21}\left(x_{0}\right) & a_{22}\left(x_{0}\right) & \cdots & a_{2 n}\left(x_{0}\right) \\
\vdots & \vdots & & \vdots \\
a_{m 1}\left(x_{0}\right) & a_{m 2}\left(x_{0}\right) & \cdots & a_{m n}\left(x_{0}\right)
\end{array}\right]_{m \times n}
$$

容易验证有下列性质:\\
设 $\lim _{x \rightarrow x_{0}} A(x)=A, \lim _{x \rightarrow x_{0}} B(x)=B$\\
则\\
(1) $\lim _{x \rightarrow x_{0}}(\boldsymbol{A}(x) \pm \boldsymbol{B}(x))=\boldsymbol{A} \pm \boldsymbol{B}$\\
(2) $\lim _{x \rightarrow x_{0}}(k A(x))=k A$\\
(3) $\lim _{x \rightarrow x_{0}}(\boldsymbol{A}(x) \boldsymbol{B}(x))=\boldsymbol{A} \boldsymbol{B} \quad$(当 $\boldsymbol{A}(x)$ 与 $\boldsymbol{B}(x)$ 可乘时)

\section*{三、函数矩阵对纯量的导数}
定义7.1.3 若 $\boldsymbol{A}(x)=\left(a_{i j}(x)\right)_{m \times n}$ 的所有各元素 $a_{i j}(x)(i=1,2, \cdots, m ; j= 1,2, \cdots, n)$ 在点 $x=x_{0}$ 处(或在区间 $[a, b]$ 上)可导,便称函数矩阵 $\boldsymbol{A}(x)$ 在点 $x=x_{0}$处(或在区间 $[a, b]$ 上)可导,并且记为

$$
\begin{aligned}
\boldsymbol{A}^{\prime}\left(x_{0}\right) & =\left.\frac{\mathrm{d} \boldsymbol{A}(x)}{\mathrm{d} x}\right|_{x=x_{0}}=\lim _{\Delta x \rightarrow 0} \frac{\boldsymbol{A}\left(x_{0}+\Delta x\right)-\boldsymbol{A}(x)}{\Delta x} \\
& =\left[\begin{array}{cccc}
{a^{\prime}}_{11}\left(x_{0}\right) & a_{12}^{\prime}\left(x_{0}\right) & \cdots & a_{1 n}^{\prime}\left(x_{0}\right) \\
a_{21}^{\prime}\left(x_{0}\right) & a_{22}^{\prime}\left(x_{0}\right) & \cdots & a_{2 n}^{\prime}\left(x_{0}\right) \\
\vdots & \vdots & & \vdots \\
a_{m 1}^{\prime}\left(x_{0}\right) & a_{m 2}^{\prime}\left(x_{0}\right) & \cdots & a_{m n}^{\prime}\left(x_{0}\right)
\end{array}\right]
\end{aligned}
$$

\section*{导数运算的性质}
(1) $\boldsymbol{A}(x)$ 是常数矩阵的充要条件是

$$
\frac{\mathrm{d} \boldsymbol{A}(x)}{\mathrm{d} x}=0
$$

(2)设 $\boldsymbol{A}(x)=\left(a_{i j}(x)\right)_{m \times n}, \boldsymbol{B}(x)=\left(b_{i j}(x)\right)_{m \times n}$ 均可导,则

$$
\frac{\mathrm{d}}{\mathrm{~d} x}[\boldsymbol{A}(x)+\boldsymbol{B}(x)]=\frac{\mathrm{d} \boldsymbol{A}(x)}{\mathrm{d} x}+\frac{\mathrm{d} \boldsymbol{B}(x)}{\mathrm{d} x}
$$

(3)设 $k(x)$ 是 $x$ 的纯量函数,$A(x)$ 是函数矩阵,$k(x)$ 与 $A(x)$ 均可导,则

$$
\frac{\mathrm{d}}{\mathrm{~d} x}[k(x) \boldsymbol{A}(x)]=\frac{\mathrm{d} k(x)}{\mathrm{d} x} \boldsymbol{A}(x)+k(x) \frac{\mathrm{d} \boldsymbol{A}(x)}{\mathrm{d} x}
$$

特别地,当 $k(x)$ 是常数 $k$ 时,有

$$
\frac{\mathrm{d}}{\mathrm{~d} x}[k \boldsymbol{A}(x)]=k \frac{\mathrm{~d} \boldsymbol{A}(x)}{\mathrm{d} x}
$$

(4)设 $\boldsymbol{A}(x), \boldsymbol{B}(x)$ 均可导,且 $\boldsymbol{A}(x)$ 与 $\boldsymbol{B}(x)$ 是可乘的,则

$$
\frac{\mathrm{d}}{\mathrm{~d} x}[\boldsymbol{A}(x) \boldsymbol{B}(x)]=\frac{\mathrm{d} \boldsymbol{A}(x)}{\mathrm{d} x} \boldsymbol{B}(x)+\boldsymbol{A}(x) \frac{\mathrm{d} \boldsymbol{B}(x)}{\mathrm{d} x}
$$

因为矩阵乘法没有交换律,所以

$$
\begin{aligned}
& \frac{\mathrm{d}}{\mathrm{~d} x} \boldsymbol{A}^{2}(x) \neq 2 \boldsymbol{A}(x) \frac{\mathrm{d} \boldsymbol{A}(x)}{\mathrm{d} x} \\
& \frac{\mathrm{~d}}{\mathrm{~d} x} \boldsymbol{A}^{3}(x) \neq 3 \boldsymbol{A}^{2}(x) \frac{\mathrm{d} \boldsymbol{A}(x)}{\mathrm{d} x}
\end{aligned}
$$

(5)若 $\boldsymbol{A}(x)$ 与 $\boldsymbol{A}^{-1}(x)$ 都可导,则

$$
\frac{\mathrm{d} \boldsymbol{A}^{-1}(x)}{\mathrm{d} x}=-\boldsymbol{A}^{-1}(x) \frac{\mathrm{d} \boldsymbol{A}(x)}{\mathrm{d} x} \boldsymbol{A}^{-1}(x)
$$

证明:因为

$$
A^{-1}(x) A(x)=E
$$

所以

$$
\frac{\mathrm{d}}{\mathrm{~d} x}\left[\boldsymbol{A}^{-1}(x) \boldsymbol{A}(x)\right]=\frac{\mathrm{d} \boldsymbol{A}^{-1}(x)}{\mathrm{d} x} \boldsymbol{A}(x)+\boldsymbol{A}^{-1}(x) \frac{\mathrm{d} \boldsymbol{A}(x)}{\mathrm{d} x}=\frac{\mathrm{d}}{\mathrm{~d} x} \boldsymbol{E}=0
$$

于是

$$
\frac{\mathrm{d} \boldsymbol{A}^{-1}(x)}{\mathrm{d} x}=-\boldsymbol{A}^{-1}(x) \frac{\mathrm{d} \boldsymbol{A}(x)}{\mathrm{d} x} \boldsymbol{A}^{-1}(x)
$$

例 7. 1.1 已知 $\boldsymbol{A}=\left[\begin{array}{ll}x & 1 \\ 1 & x\end{array}\right]$ 在 $[2,3]$ 上有逆矩阵 $\boldsymbol{A}^{-1}(x)=\frac{1}{x^{2}-1}\left[\begin{array}{cc}x & -1 \\ -1 & x\end{array}\right]$ ,试用性质(3)与性质(5)分别计算 $\frac{\mathrm{d} \boldsymbol{A}^{-1}(\boldsymbol{x})}{\mathrm{d} \boldsymbol{x}}$ .

解 由性质(3)知

$$
\begin{aligned}
& \frac{\mathrm{d}}{\mathrm{~d} x}\left\{\frac{1}{x^{2}-1}\left[\begin{array}{cc}
x & -1 \\
-1 & x
\end{array}\right]\right\} \\
& =\frac{\mathrm{d}}{\mathrm{~d} x}\left(\frac{1}{x^{2}-1}\right) \cdot\left[\begin{array}{cc}
x & -1 \\
-1 & x
\end{array}\right]+\frac{1}{x^{2}-1 \mathrm{~d} x}\left[\begin{array}{cc}
x & -1 \\
-1 & x
\end{array}\right] \\
& =-\frac{2 x}{\left(x^{2}-1\right)^{2}}\left[\begin{array}{cc}
x & -1 \\
-1 & x
\end{array}\right]+\frac{1}{x^{2}-1}\left[\begin{array}{ll}
1 & 0 \\
0 & 1
\end{array}\right] \\
& =\frac{1}{\left(x^{2}-1\right)^{2}}\left[\begin{array}{cc}
-x^{2}-1 & 2 x \\
2 x & -x^{2}-1
\end{array}\right]
\end{aligned}
$$

由性质(5)知

$$
\begin{aligned}
\frac{\mathrm{d} \boldsymbol{A}^{-1}(x)}{\mathrm{d} x} & =-\boldsymbol{A}^{-1}(x) \frac{\mathrm{d} \boldsymbol{A}(x)}{\mathrm{d} x} \boldsymbol{A}^{-1}(x) \\
& =-\frac{1}{x^{2}-1}\left[\begin{array}{cc}
x & -1 \\
-1 & x
\end{array}\right]\left[\begin{array}{cc}
1 & 0 \\
0 & 1
\end{array}\right] \frac{1}{x^{2}-1}\left[\begin{array}{cc}
x & -1 \\
-1 & x
\end{array}\right] \\
& =-\frac{1}{\left(x^{2}-1\right)^{2}}\left[\begin{array}{cc}
x^{2}+1 & -2 x \\
-2 x & x^{2}+1
\end{array}\right] \\
& =\frac{1}{\left(x^{2}-1\right)^{2}}\left[\begin{array}{cc}
-x^{2}-1 & 2 x \\
2 x & -x^{2}-1
\end{array}\right]
\end{aligned}
$$

(6)设 $\boldsymbol{A}(x)$ 为函数矩阵,$x=f(t)$ 是 $t$ 的纯量函数, $\boldsymbol{A}(x)$ 与 $f(t)$ 均可导,则

$$
\frac{\mathrm{d}}{\mathrm{~d} t}(\boldsymbol{A}(x))=\frac{\mathrm{d} A(x)}{\mathrm{d} x} f^{\prime}(t)=f^{\prime}(t) \frac{\mathrm{d} \boldsymbol{A}(x)}{\mathrm{d} x}
$$

函数矩阵的导数也是一个函数矩阵,它可以再求导,因此就有函数矩阵的高阶导数

$$
\begin{gathered}
\frac{\mathrm{d}^{2} \boldsymbol{A}(x)}{\mathrm{d} x^{2}}=\frac{\mathrm{d}}{\mathrm{~d} x}\left(\frac{\mathrm{~d} \boldsymbol{A}(x)}{\mathrm{d} x}\right) \\
\frac{\mathrm{d}^{3} \boldsymbol{A}(x)}{\mathrm{d} x^{3}}=\frac{\mathrm{d}}{\mathrm{~d} x}\left(\frac{\mathrm{~d}^{2} \boldsymbol{A}(x)}{\mathrm{d} x^{2}}\right) \\
\vdots \\
\frac{\mathrm{d}^{k} \boldsymbol{A}(x)}{\mathrm{d} x^{k}}=\frac{\mathrm{d}}{\mathrm{~d} x}\left(\frac{\mathrm{~d}^{k-1} \boldsymbol{A}(x)}{\mathrm{d} x^{k-1}}\right)
\end{gathered}
$$

\section*{四、函数矩阵的积分}
定义7.1.4 若函数矩阵 $\boldsymbol{A}(x)=\left(a_{i j}(x)\right)_{m \times n}$ 的所有各元 $a_{i j}(x) \quad(i=1,2, \cdots$ ,\\
$m ; j=1,2, \cdots, n)$ 都在 $[a, b]$ 上可积,则称 $\boldsymbol{A}(x)$ 在 $[a, b]$ 上可积,且

$$
\int_{a}^{b} \boldsymbol{A}(x) \mathrm{d} x=\left[\begin{array}{cccc}
\int_{a}^{b} a_{11}(x) \mathrm{d} x & \int_{a}^{b} a_{12}(x) \mathrm{d} x & \cdots & \int_{a}^{b} a_{1 n}(x) \mathrm{d} x \\
\int_{a}^{b} a_{21}(x) \mathrm{d} x & \int_{a}^{b} a_{22}(x) \mathrm{d} x & \cdots & \int_{a}^{b} a_{2 n}(x) \mathrm{d} x \\
\vdots & \vdots & & \vdots \\
\int_{a}^{b} a_{m 1}(x) \mathrm{d} x & \int_{a}^{b} a_{m 2}(x) \mathrm{d} x & \cdots & \int_{a}^{b} a_{m n}(x) \mathrm{d} x
\end{array}\right]
$$

函数矩阵的定积分有如下简单性质:\\
(1) $\int_{a}^{b} k \boldsymbol{A}(x) \mathrm{d} x=k \int_{a}^{b} \boldsymbol{A}(x) \mathrm{d} x \quad k \in \mathbf{R}$ .\\
(2) $\int_{a}^{b}[\boldsymbol{A}(x)+\boldsymbol{B}(x)] \mathrm{d} x=\int_{a}^{b} \boldsymbol{A}(x) \mathrm{d} x+\int_{a}^{b} \boldsymbol{B}(x) \mathrm{d} x$.

\section*{§7.2 函数向量的线性相关性}
常数向量组线性相关性判别方法不能用来判断函数向量组.本节主要介绍两个判断函数向量组线性相关性的方法。

定义7.2.1 设有定义在 $[a, b]$ 上的 $m$ 个连续函数向量

$$
\alpha_{i}(x)=\left(a_{i 1}(x), a_{i 2}(x), \cdots, a_{i n}(x)\right) \quad(i=1,2, \cdots, m)
$$

若存在一组不全为零的实常数 $k_{1}, k_{2}, \cdots, k_{m}$ ,使得对于所有的 $x \in[a, b]$ ,等式


\begin{equation*}
k_{1} \boldsymbol{\alpha}_{1}(\boldsymbol{x})+k_{2} \boldsymbol{\alpha}_{2}(\boldsymbol{x})+\cdots+k_{m} \boldsymbol{\alpha}_{m}(\boldsymbol{x})=0 \tag{7.2.1}
\end{equation*}


成立,那么,在 $[a, b]$ 上 $\boldsymbol{\alpha}_{1}(x), \boldsymbol{\alpha}_{2}(x), \cdots, \boldsymbol{\alpha}_{m}(x)$ 线性相关,否则 $\boldsymbol{\alpha}_{1}(x), \boldsymbol{\alpha}_{2}(x), \cdots \boldsymbol{\alpha}_{m}(x)$ 线性无关。即如果只有在 $k_{1}=k_{2}=k_{3}=\cdots=k_{m}=0$ 时,式(7.2.1)才成立。便说 $\boldsymbol{\alpha}_{1}(x), \boldsymbol{\alpha}_{2}(x), \cdots, \boldsymbol{\alpha}_{m}(x)$ 线性无关。

定义7.2.2 设 $\boldsymbol{\alpha}_{1}(x), \boldsymbol{\alpha}_{2}(x), \cdots, \boldsymbol{\alpha}_{m}(x)$ 是 $m$ 个定义在 $[a, b]$ 上的连续函数向量,且

$$
\boldsymbol{\alpha}_{i}(x)=\left(a_{i 1}(x), a_{i 2}(x), \cdots, a_{i n}(x)\right) \quad(i=1,2, \cdots, m)
$$

记


\begin{equation*}
g_{i j}=\int_{a}^{b} \boldsymbol{\alpha}_{i}(x) \boldsymbol{\alpha}_{j}^{\mathrm{T}}(x) \mathrm{d} x \quad(i, j=1,2, \cdots, m) \tag{7.2.2}
\end{equation*}


以 $g_{i j}$ 为元素的常数矩阵

$$
\boldsymbol{G}=\left(g_{i j}\right)_{m \times m}
$$

称为 $\boldsymbol{\alpha}_{1}(x), ~ \boldsymbol{\alpha}_{2}(x), \cdots, \boldsymbol{\alpha}_{m}(x)$ 的 Gram 矩阵, $\operatorname{det} \boldsymbol{G}$ 称为 Gram 行列式。\\
定理7.2.1 定义在 $[a, b]$ 上的连续函数向量 $\boldsymbol{\alpha}_{1}(x), \boldsymbol{\alpha}_{2}(x), \boldsymbol{\alpha}_{m}(x)$ 线性无关的充要条件是它的 Gram 矩阵为满秩矩阵。

证明 设

$$
k_{1} \alpha_{1}(x)+k_{2} \alpha_{2}(x)+\cdots+k_{m} \alpha_{m}(x)=0
$$

在上式两边右乘 $\boldsymbol{\alpha}_{i}^{\mathrm{T}}(x)$ 以后,对 $x$ 积分得


\begin{equation*}
\int_{a}^{b}\left[\sum_{j=1}^{m} k_{j} \boldsymbol{\alpha}_{j}(x)\right] \boldsymbol{\alpha}_{i}^{\mathrm{r}}(x) \mathrm{d} x=0 \quad(i=1,2, \cdots, m) \tag{7.2.3}
\end{equation*}


此即

$$
\sum_{j=1}^{m} k_{j} g_{j i}=0 \quad(i=1,2, \cdots, m)
$$

命

$$
\boldsymbol{u}=\left(k_{1}, k_{2}, \cdots, k_{m}\right)^{\mathrm{T}}
$$

式(7.2.3)可以写成


\begin{equation*}
G^{\mathbf{T}} u=0 \tag{7.2.4}
\end{equation*}


若 $\boldsymbol{G}^{\mathrm{T}}$ 是满秩的,式(7.2.4)只有零解,这时 $k_{1}=k_{2}=\cdots=k_{m}=0$ ,故 $\boldsymbol{\alpha}_{1}(x), ~ \boldsymbol{\alpha}_{2}(x), \cdots$ , $\boldsymbol{\alpha}_{m}(x)$ 是线性无关的。

若 $\boldsymbol{G}^{\mathrm{T}}$ 不满秩,则式(7.2.4)有非零解,这时至少有一组不全为零的数组 $k_{1}$ , $k_{2}, \cdots, k_{m}$ ,满足式(7.2.3)。以不全为零的数组 $k_{1}, k_{2}, \cdots, k_{m}$ 依次乘方程组 (7.2.3)的第 1 个,第 2 个,$\cdots$ ,第 $m$ 个方程,然后相加得


\begin{equation*}
\int_{a}^{b}\left[\sum_{j=1}^{m} k_{j} \boldsymbol{\alpha}_{j}(x)\right]\left[\sum_{i=1}^{m} k_{i} \boldsymbol{\alpha}_{i}^{\mathrm{T}}(x)\right] \mathrm{d} x=0 \tag{7.2.5}
\end{equation*}


命


\begin{equation*}
\boldsymbol{\alpha}(x)=\sum_{i=1}^{m} k_{i} \boldsymbol{\alpha}_{i}(x) \tag{7.2.6}
\end{equation*}


代人式(7.2.5)得


\begin{equation*}
\int_{a}^{b} \boldsymbol{\alpha}(x) \boldsymbol{\alpha}^{\mathrm{T}}(x) \mathrm{d} x=0 \tag{7.2.7}
\end{equation*}


因为 $\boldsymbol{\alpha}(x)$ 是连续的, $\boldsymbol{\alpha}(x) \boldsymbol{\alpha}^{\mathrm{T}}(x)$ 对于所有的 $x$ 都是非负的,因此只有当

$$
\alpha(x)=\sum_{i=1}^{m} k_{i} \alpha_{i}(x)=0 \quad x \in[a, b]
$$

时式(7.2.7)成立,因而 $\boldsymbol{\alpha}_{1}(x), \boldsymbol{\alpha}_{2}(x), \cdots, \boldsymbol{\alpha}_{m}(x)$ 是线性相关的.\\
定理7.2.1给出了判断函数向量组线性相关性的判断方法,如果根据定义 7.2.1 判断一组函数向量组的线性相关性是十分困难的。因为它需要判断在区间 $[a, b]$ 上每一点(7.2.1)都成立。而定理7.2.1只要计算常数矩阵 $\boldsymbol{G}$ 是否满秩就可以了。

例7.2.1 设

$$
\alpha_{1}(x)=(0, x) \quad \alpha_{2}(x)=(x, 0)
$$

则

$$
\begin{aligned}
& g_{11}=\int_{a}^{b} \boldsymbol{\alpha}_{1}(x) \boldsymbol{\alpha}_{1}^{\mathrm{T}}(x) \mathrm{d} x=\int_{a}^{b} x^{2} \mathrm{~d} x=\frac{1}{3}\left(b^{3}-a^{3}\right) \\
& g_{12}=g_{21}=\int_{a}^{b} \boldsymbol{\alpha}_{1}(x) \boldsymbol{\alpha}_{2}^{\mathrm{T}}(x) \mathrm{d} x=0 \\
& g_{22}=\int_{a}^{b} \boldsymbol{\alpha}_{2}(x) \boldsymbol{\alpha}_{2}^{\mathrm{T}}(x) \mathrm{d} x=\int_{a}^{b} x^{2} \mathrm{~d} x=\frac{1}{3}\left(b^{3}-a^{3}\right)
\end{aligned}
$$

于是 $\boldsymbol{\alpha}_{1}(x), \boldsymbol{\alpha}_{2}(x)$ 的 Gram 矩阵为

$$
\boldsymbol{G}=\left[\begin{array}{cc}
\frac{1}{3}\left(b^{3}-a^{3}\right) & 0 \\
0 & \frac{1}{3}\left(b^{3}-a^{3}\right)
\end{array}\right]
$$

而

$$
\operatorname{det} \boldsymbol{G}=\frac{1}{9}\left(b^{3}-a^{3}\right)^{2}
$$

故当 $a \neq b$ 时, $\operatorname{det} \boldsymbol{G}>0$ ,因此 $\boldsymbol{\alpha}_{1}(x), \boldsymbol{\alpha}_{2}(x)$ 在 $[a, b]$ 上是线性无关的.\\
当函数向量组满足可积条件时,可以用求定积分方法,用定理7.2.1判断函数向量组的线性无(相)关,十分方便。但是函数的积分有时也是十分困难的。下述方法是用微分方法来判断函数向量组的线性相关性。

定义7.2.3 设 $m$ 个函数向量组

$$
\boldsymbol{\alpha}_{i}(x)=\left(a_{i 1}(x), a_{i 2}(x), \cdots, a_{i n}(x)\right) \quad(i=1,2 \cdots, m)
$$

是 $m$ 个定义在 $[a, b]$ 上的有 $m-1$ 阶导数的函数向量,记

\[
\boldsymbol{A}(x)=\left[\begin{array}{c}
\boldsymbol{\alpha}_{1}(x)  \tag{7.2.8}\\
\boldsymbol{\alpha}_{2}(x)^{2} \\
\vdots \\
\boldsymbol{\alpha}_{m}(x)
\end{array}\right]=\left[\begin{array}{cccc}
a_{11}(x) & a_{12}(x) & \cdots & a_{1 n}(x) \\
a_{21}(x) & a_{22}(x) & \cdots & a_{2 n}(x) \\
\vdots & \vdots & & \vdots \\
a_{m 1}(x) & a_{m 2}(x) & \cdots & a_{m n}(x)
\end{array}\right]
\]

那么称矩阵

$$
\begin{aligned}
& \boldsymbol{W}(x)=\left(\boldsymbol{A}(x), \boldsymbol{A}^{\prime}(x), \boldsymbol{A}^{\prime \prime}(x) \cdots, \boldsymbol{A}^{(m-1)}(x)\right)_{m \times m n} \\
& =\left[\begin{array}{cccccc}
a_{11}(x) & a_{12}(x) & \cdots a_{1 n}(x) & \cdots a_{11}^{(m-1)}(x) & a_{12}^{(m-1)}(x) & \cdots a_{1 n}^{(m-1)}(x) \\
a_{21}(x) & a_{22}(x) & \cdots a_{2 n}(x) & \cdots a_{21}^{(m-1)}(x) & a_{22}^{(m-1)}(x) & \cdots a_{2 n}^{(m-1)}(x) \\
\vdots & \vdots & \vdots & \vdots & \vdots & \vdots \\
a_{m 1}(x) & a_{m 2}(x) & \cdots a_{m n}(x) & \cdots a_{m 1}^{(m-1)}(x) & a_{m 2}^{(m-1)}(x) & \cdots a_{m n}^{(m-1)}(x)
\end{array}\right]
\end{aligned}
$$

是函数向量组 $\boldsymbol{\alpha}_{1}(x), \boldsymbol{\alpha}_{2}(x), \cdots, \boldsymbol{\alpha}_{m}(x)$ 的 Wronski 矩阵。其中 $\boldsymbol{A}^{\prime}(x), \boldsymbol{A}^{\prime \prime}(x)$ , $\boldsymbol{A}^{(m-1)}(x)$ 分别是函数矩阵 $\boldsymbol{A}(x)$ 对纯量 $x$ 的 1 阶, 2 阶,$\cdots,(m-1)$ 阶导数矩阵。

下面定理给出了函数向量组线性无关的一个充分条件。\\
定理7.2.2 设 $\boldsymbol{W}(x)$ 是 $\boldsymbol{\alpha}_{1}(x), \boldsymbol{\alpha}_{2}(x), \cdots, \boldsymbol{\alpha}_{m}(x)$ 的 Wronski 矩阵,若在区间\\
$[a, b]$ 上的某个点 $x_{0} \in[a, b]$ ,常数矩阵 $\boldsymbol{W}\left(x_{0}\right)$ 的秩等于 $\boldsymbol{m}$ ,则函数向量组 $\boldsymbol{\alpha}_{1}(x)$ , $\boldsymbol{a}_{2}(x), \cdots, \boldsymbol{a}_{m}(x)$ 在 $[a, b]$ 上线性无关.

证明 用反证法证明:设 $\boldsymbol{\alpha}_{1}(\boldsymbol{x}), \boldsymbol{\alpha}_{2}(x), \cdots, \boldsymbol{\alpha}_{m}(x)$ 在 $[a, b]$ 上线性相关,则存在不全为零的常数 $k_{1}, k_{2}, \cdots, k_{m}$ 使得对于任意 $x \in[a, b]$ 都有


\begin{equation*}
k_{1} \boldsymbol{\alpha}_{1}(x)+k_{2} \boldsymbol{\alpha}_{1}(x)+\cdots+k_{m} \boldsymbol{\alpha}_{m}(x)=0 \tag{7.2.9}
\end{equation*}


将此式逐次对 $x$ 求导数得


\begin{gather*}
k_{1} \boldsymbol{\alpha}_{1}^{\prime}(x)+k_{2} \boldsymbol{\alpha}_{2}^{\prime}(x)+\cdots+k_{m} \boldsymbol{\alpha}_{m}^{\prime}(x)=0 \\
k_{1} \boldsymbol{\alpha}_{1}^{\prime \prime}(x)+k_{2} \boldsymbol{\alpha}_{2}^{\prime \prime}(x)+\cdots+k_{m} \boldsymbol{\alpha}_{m}^{\prime \prime}(x)=0 \\
\vdots  \tag{7.2.10}\\
\vdots \\
k_{1} \boldsymbol{\alpha}_{1}^{(m-1)}(x)+k_{2} \boldsymbol{\alpha}_{2}^{(m-1)}(x)+\cdots+k_{m} \boldsymbol{\alpha}_{m}^{(m-1)}(x)=0
\end{gather*}


式(7.2.9)与式(7.2.10)共 $m$ 个等式对于任意 $x \in[a, b]$ 都成立,以 $x=x_{0}$ 代人式 (7.2.9)与式(7.2.10)并把这 $m$ 个等式写成矩阵形成

\[
\left[\begin{array}{cccc}
\boldsymbol{\alpha}_{1}\left(x_{0}\right) & \boldsymbol{\alpha}_{2}\left(x_{0}\right) & \cdots & \boldsymbol{\alpha}_{m}\left(x_{0}\right)  \tag{7.2.11}\\
\boldsymbol{\alpha}_{1}^{\prime}\left(x_{0}\right) & \boldsymbol{\alpha}_{2}^{\prime}\left(x_{0}\right) & \cdots & \boldsymbol{\alpha}_{m}^{\prime}\left(x_{0}\right) \\
\vdots & \vdots & & \vdots \\
\boldsymbol{\alpha}_{1}^{(m-1)}\left(x_{0}\right) & \boldsymbol{\alpha}_{2}^{(m-1)}\left(x_{0}\right) & \cdots & \boldsymbol{\alpha}_{m}^{(m-1)}\left(x_{0}\right)
\end{array}\right]\left[\begin{array}{c}
k_{1} \\
k_{2} \\
\vdots \\
k_{m}
\end{array}\right]=0
\]

数组 $k_{1}, k_{2}, \cdots, k_{m}$ 不全为零的充要条件是式(7.2.11)的系数矩阵 $\boldsymbol{W}\left(x_{0}\right)$ 的秩小于 $m$ ,这与假设矛盾。故 $\boldsymbol{\alpha}_{1}(x), \boldsymbol{\alpha}_{2}(x), \cdots, \boldsymbol{\alpha}_{m}(x)$ 在 $[a, b]$ 上线性无关。

例 7.2.2 设

$$
\boldsymbol{\alpha}_{1}(x)=\left(1, x, x^{2}\right) \quad \boldsymbol{\alpha}_{2}(x)=\left(\mathrm{e}^{x}, 1, x\right)
$$

则

$$
\begin{gathered}
\boldsymbol{A}(x)=\left[\begin{array}{l}
\boldsymbol{\alpha}_{1}(x) \\
\boldsymbol{\alpha}_{2}(x)
\end{array}\right]=\left[\begin{array}{ccc}
1 & x & x^{2} \\
\mathrm{e}^{x} & 1 & x
\end{array}\right] \\
\boldsymbol{A}^{\prime}(x)=\left[\begin{array}{ccc}
0 & 1 & 2 x \\
\mathrm{e}^{x} & 0 & 1
\end{array}\right]
\end{gathered}
$$

于是 $\boldsymbol{\alpha}_{1}(x), \boldsymbol{\alpha}_{2}(x)$ 的 Wronski 矩阵为

$$
\boldsymbol{W}(x)=\left[\begin{array}{cccccc}
1 & x & x^{2} & 0 & 1 & 2 x \\
\mathrm{e}^{x} & 1 & x & \mathrm{e}^{x} & 0 & 1
\end{array}\right]
$$

不难看出,对于任意实数 $x_{0}, \boldsymbol{W}\left(x_{0}\right)$ 的秩为 2 ,因此 $\boldsymbol{a}_{1}(x), \boldsymbol{\alpha}_{2}(x)$ 线性无关。\\
再看例7.2.1. $\boldsymbol{\alpha}_{1}(x)=(0, x), \boldsymbol{\alpha}_{2}(x)=(x, 0)$ ,则 $\boldsymbol{\alpha}_{1}(x), \boldsymbol{\alpha}_{2}(x)$ 的 Wronski 矩阵为

$$
\boldsymbol{W}(x)=\left[\begin{array}{llll}
0 & x & 0 & 1 \\
x & 0 & 1 & 0
\end{array}\right]
$$

不难看出,对于任何 $x$ ,矩阵 $\boldsymbol{W}(x)$ 的秩为 2 ,所以 $\boldsymbol{\alpha}_{1}(x), \boldsymbol{\alpha}_{2}(x)$ 线性无关。显然,

比用定理7.2.1判断要简便。但不能就此认为定理7.2.1是多余的,因为定理 7.2.1仅仅要求函数向量可积,而定理7.2.2要求函数向量具有 $m-1$ 阶导数,另外,定理7.2.1是充要判别法,而定理7.2.2是充分性判别法。

\section*{§7.3 矩阵微分方程 $\frac{\mathrm{d} \boldsymbol{X}(t)}{\mathrm{d} t}=\boldsymbol{A}(t) \boldsymbol{X}(t)$}
形如


\begin{equation*}
\frac{\mathrm{d} \boldsymbol{X}}{\mathrm{~d} t}=\boldsymbol{A}(t) \boldsymbol{X}(t), \quad \boldsymbol{X}\left(t_{0}\right)=\boldsymbol{C} \tag{7.3.1}
\end{equation*}


的矩阵微分方程实质上是一个 $n^{2}$ 个未知函数的常微分方程组。由常微分方程解的存在与唯一性定理可知:

定理7.3.1 若 $\boldsymbol{A}(t)$ 是定义在 $\left[t_{0}, t_{1}\right]$ 上的分段连续 $n$ 阶函数矩阵, $\boldsymbol{X}(t)$ 是 $n$阶未知函数矩阵, $\boldsymbol{C}$ 是 $n$ 阶常数矩阵,则方程组(7.3.1)的解 $\boldsymbol{X}(t)$ 存在且唯一。\\
(证略)\\
定理7.3.2 当 $\operatorname{det} C \neq 0$ 时,方程(7.3.1)的任何一个解 $\boldsymbol{X}(t)$ 有 Jacobi 等式


\begin{equation*}
\operatorname{det} X(t)=\operatorname{det} C \cdot \exp \int_{t_{0}}^{t}(\operatorname{tr}(A(t))) \mathrm{d} t \tag{7.3.2}
\end{equation*}


证明 设 $X(t)=\left(x_{i j}(t)\right)_{n \times n}$ ,则

$$
\frac{\mathrm{d}}{\mathrm{~d} t} \operatorname{det} \boldsymbol{X}(t)=\sum_{i=1}^{n}\left|\begin{array}{cccc}
x_{11}(t) & x_{12}(t) & \cdots & x_{1 n}(t) \\
x_{21}(t) & x_{22}(t) & \cdots & x_{2 n}(t) \\
\vdots & \vdots & & \vdots \\
\frac{\mathrm{d} x_{i 1}(t)}{\mathrm{d} t} & \frac{\mathrm{~d} x_{i 2}(t)}{\mathrm{d} t} & \cdots & \frac{\mathrm{~d} x_{i n}(t)}{\mathrm{d} t} \\
\vdots & \vdots & & \vdots \\
x_{n 1}(t) & x_{n 2}(t) & \cdots & x_{n n}(t)
\end{array}\right|
$$

把式(7.3.1)代人上式右边得

$$
\frac{\mathrm{d}}{\mathrm{~d} t} \operatorname{det} X(t)=\sum_{i=1}^{n}\left|\begin{array}{cccc}
x_{11}(t) & x_{12}(t) & \cdots & x_{1 n}(t) \\
x_{21}(t) & x_{22}(t) & \cdots & x_{2 n}(t) \\
\vdots & \vdots & & \vdots \\
\sum_{j=1}^{n} a_{i j} x_{j 1}(t) & \sum_{j=1}^{n} a_{i j} x_{j 2}(t) & \cdots & \sum_{j=1}^{n} a_{i j} x_{j n}(t) \\
\vdots & \vdots & & \vdots \\
x_{n 1}(t) & x_{n 2}(t) & \cdots & x_{n n}(t)
\end{array}\right|
$$

根据行列式性质,上式右边为

$$
\begin{aligned}
\frac{\mathrm{d}}{\mathrm{~d} t} \operatorname{det} X(t) & =\sum_{i=1}^{n}\left|\begin{array}{cccc}
x_{11}(t) & x_{12}(t) & \cdots & x_{1 n}(t) \\
x_{21}(t) & x_{22}(t) & \cdots & x_{2 n}(t) \\
\vdots & \vdots & & \vdots \\
a_{i i} x_{i 1}(t) & a_{i i} x_{i 2}(t) & \cdots & a_{i i} x_{i n}(t) \\
\vdots & \vdots & & \vdots \\
x_{n 1}(t) & x_{n 2}(t) & \cdots & x_{n n}(t)
\end{array}\right| \\
& =\sum_{i=1}^{n} a_{i i}|X(t)|=|X(t)| \operatorname{tr}(A(t))
\end{aligned}
$$

此即

$$
\frac{\mathrm{d}|\boldsymbol{X}(t)|}{|\boldsymbol{X}(t)|}=\operatorname{tr}(\boldsymbol{A}(t)) \mathrm{d} t
$$

上式两边从 $t_{0}$ 到 $t$ 积分,最后得

$$
|X(t)|=|C| \exp \int_{t_{0}}^{t}(\operatorname{tr}(A(t))) \mathrm{d} t
$$

\section*{定理7.3.3 设方程}
$$
\left\{\begin{array} { l } 
{ \frac { \mathrm { d } \boldsymbol { X } ( t ) } { \mathrm { d } t } = \boldsymbol { A } ( t ) \boldsymbol { X } ( t ) } \\
{ \boldsymbol { X } ( t _ { 0 } ) = \boldsymbol { C } _ { 1 } }
\end{array} \quad \text { 与 } \left\{\begin{array}{l}
\frac{\mathrm{d} \boldsymbol{X}(t)}{\mathrm{d} t}=\boldsymbol{A}(t) \boldsymbol{X}(t) \\
\boldsymbol{X}\left(t_{0}\right)=\boldsymbol{C}_{2}
\end{array}\right.\right.
$$

的解分别为 $\boldsymbol{X}_{1}(t)$ 和 $\boldsymbol{X}_{2}(t)$ ,则

$$
\boldsymbol{X}_{2}(t)=\boldsymbol{X}_{1}(t) \boldsymbol{T}, \boldsymbol{T}=\boldsymbol{C}_{1}^{-1} \boldsymbol{C}_{2}
$$

证明 若令 $\boldsymbol{X}_{3}(t)=\boldsymbol{X}_{1}(t) \boldsymbol{C}_{1}^{-1} \boldsymbol{C}_{2}=\boldsymbol{X}_{1}(t) \boldsymbol{T}$ ,则

$$
\frac{\mathrm{d} \boldsymbol{X}_{3}(t)}{\mathrm{d} t}=\frac{\mathrm{d} \boldsymbol{X}_{1}(t)}{\mathrm{d} t} \boldsymbol{T}=\boldsymbol{A}(t) \boldsymbol{X}_{1}(t) \boldsymbol{T}=\boldsymbol{A}(t) \boldsymbol{X}_{3}(t)
$$

且

$$
\boldsymbol{X}_{3}\left(t_{0}\right)=\boldsymbol{X}_{1}\left(t_{0}\right) \boldsymbol{C}_{1}^{-1} \boldsymbol{C}_{2}=\boldsymbol{C}_{2}
$$

根据解的存在唯一性知

$$
\boldsymbol{X}_{3}(t)=\boldsymbol{X}_{2}(t) \text {, 于是 }
$$

$$
\boldsymbol{X}_{2}(t)=\boldsymbol{X}_{1}(t) \boldsymbol{T}
$$

称方程

$$
\frac{\mathrm{d} X}{\mathrm{~d} t}=A(t) X(t), \quad X\left(t_{0}\right)=E
$$

的解 $\boldsymbol{X}=\boldsymbol{X}_{0}(t)$ 是 $\boldsymbol{A}(t)$ 的基本解矩阵,根据定理7.3.3知,方程 $\frac{\mathrm{d} \boldsymbol{X}}{\mathrm{d} t}=\boldsymbol{A}(t) \boldsymbol{X}(t)$ 满足任何初始条件 $\boldsymbol{X}\left(t_{0}\right)=\boldsymbol{C}$ 的解 $\boldsymbol{X}(t)$ 都可由基本解矩阵 $\boldsymbol{X}_{0}(t)$ 来表示。

利用矩阵指数函数的性质,不难证明下述定理。\\
定理7.3.4 若 $\boldsymbol{A}$ 为 $n$ 阶常数矩阵,则矩阵微分方程

$$
\frac{\mathrm{d} \boldsymbol{X}(t)}{\mathrm{d} t}=\boldsymbol{A} \boldsymbol{X}(t), \boldsymbol{X}_{0}(t)=\boldsymbol{C}
$$

的解为

$$
\boldsymbol{X}(t)=\mathbf{e}^{\boldsymbol{A}\left(t-t_{0}\right)} \boldsymbol{C}
$$

矩阵微分方程

$$
\frac{\mathrm{d} \boldsymbol{X}(t)}{\mathrm{d} t}=\boldsymbol{X}(t) \boldsymbol{A}, \boldsymbol{X}\left(t_{0}\right)=\boldsymbol{C}
$$

的解为

$$
X(t)=C \mathrm{e}^{A\left(t-t_{0}\right)}
$$

根据定理 7.3.4,当 $\boldsymbol{X}(0)=\boldsymbol{E}$ 时,方程 $\frac{\mathrm{d} \boldsymbol{X}}{\mathrm{d} t}=\boldsymbol{X}(t) \boldsymbol{A}$ 的解为 $\boldsymbol{X}(t)=\mathrm{e}^{\boldsymbol{A} t}$ ,而矩阵指数函数 $\boldsymbol{X}(t)=\mathrm{e}^{\boldsymbol{A} t}$ 具有性质 $\mathrm{e}^{\boldsymbol{A}(t+s)}=\mathrm{e}^{\boldsymbol{A} t} \mathrm{e}^{\boldsymbol{A} s}$ ,也就是说,矩阵函数 $\boldsymbol{X}(t)=\mathrm{e}^{\boldsymbol{A} t}$ 满足矩阵方程


\begin{gather*}
\boldsymbol{X}(t+s)=\boldsymbol{X}(t) \boldsymbol{X}(s) \quad(-\infty<t, s<+\infty) \\
\boldsymbol{X}(0)=\boldsymbol{E} \tag{7.3.3}
\end{gather*}


相反地,可以证明,满足方程(7.3.3)的矩阵 $\boldsymbol{X}(t)$ 只能是矩阵指数函数 $\mathrm{e}^{A t}$ ,这就是下面的定理。

定理 7.3.5 设函数矩阵 $\boldsymbol{X}(t)$ 对于任何有限实数 $t$ 可以求导,且满足方程

$$
\begin{gathered}
\boldsymbol{X}(t+s)=\boldsymbol{X}(t) \boldsymbol{X}(s), \quad-\infty<t, s<+\infty \\
\boldsymbol{X}(0)=\boldsymbol{E}
\end{gathered}
$$

则

$$
\boldsymbol{X}(t)=\mathrm{e}^{\boldsymbol{A} t}
$$

(证略)

\section*{§7.4 线性向量微分方程}
$$
\frac{\mathrm{d} \boldsymbol{x}(t)}{\mathrm{d} t}=\boldsymbol{A}(t) \boldsymbol{x}(t)+\boldsymbol{f}(t)
$$

形如

\[
\left\{\begin{array}{ccc}
\begin{array}{c}
\frac{\mathrm{d} x_{1}}{\mathrm{~d} t}=a_{11}(t) x_{1}(t)+a_{12}(t) x_{2}(t)+\cdots+a_{1 n}(t) x_{n}(t)+f_{1}(t) \\
\frac{\mathrm{d} x_{2}}{\mathrm{~d} t}=a_{21}(t) x_{1}(t)+a_{22}(t) x_{2}(t)+\cdots+a_{2 n}(t) x_{n}(t)+f_{2}(t) \\
\vdots \\
\vdots
\end{array} & \vdots  \tag{7.4.1}\\
\frac{\mathrm{d} x_{n}}{\mathrm{~d} t}= & a_{n 1}(t) x_{1}(t)+a_{n 2}(t) x_{2}(t)+\cdots+a_{n n}(t) x_{n}(t)+f_{n}(t)
\end{array}\right.
\]

的线性微分方程组在引进函数矩阵与函数向量以后可以表示成


\begin{equation*}
\frac{\mathrm{d} \boldsymbol{x}(t)}{\mathrm{d} t}=\boldsymbol{A}(t) \boldsymbol{x}(t)+\boldsymbol{f}(t) \tag{7.4.2}
\end{equation*}


其中


\begin{gather*}
\boldsymbol{A}(t)=\left[\begin{array}{cccc}
a_{11}(t) & a_{12}(t) & \cdots & a_{1 n}(t) \\
a_{21}(t) & a_{22}(t) & \cdots & a_{2 n}(t) \\
\vdots & \vdots & & \vdots \\
a_{n 1}(t) & a_{n 2}(t) & \cdots & a_{n n}(t)
\end{array}\right] \\
\boldsymbol{x}(t)=\left(x_{1}(t), x_{2}(t), \cdots, x_{n}(t)\right)^{\mathrm{T}} \\
\boldsymbol{f}(t)=\left(f_{1}(t), f_{2}(t), \cdots, f_{n}(t)\right)^{\mathrm{T}} \tag{7.4.3}
\end{gather*}


方程组(7.4.1)的初始条件


\begin{equation*}
x_{1}\left(t_{0}\right)=x_{10}, x_{2}\left(t_{0}\right)=x_{20}, \cdots, x_{n}\left(t_{0}\right)=x_{n 0} \tag{7.4.4}
\end{equation*}


可以表示成


\begin{equation*}
x\left(t_{0}\right)=\left(x_{10}, x_{20}, \cdots, x_{n 0}\right)^{\mathrm{T}} \tag{7.4.5}
\end{equation*}


利用矩阵指数函数的性质不难验证下述定理。\\
定理7.4.1 若 $\boldsymbol{A}$ 为 $n$ 阶常数矩阵时,向量微分方程

$$
\frac{\mathrm{d} x}{\mathrm{~d} t}=A x(t), x\left(t_{0}\right)=x_{0}
$$

的解为

$$
\boldsymbol{x}(t)=\mathbf{e}^{\boldsymbol{A}\left(t-t_{0}\right)} \boldsymbol{x}_{0}
$$

定理7.4.2 当 $\boldsymbol{A}$ 为 $n$ 阶常数矩阵时,向量微分方程

$$
\frac{\mathrm{d} \boldsymbol{x}}{\mathrm{~d} t}=\boldsymbol{A x}(t)+\boldsymbol{f}(t), \boldsymbol{x}\left(t_{0}\right)=\boldsymbol{x}_{0}
$$

的解为

$$
\boldsymbol{x}(t)=\mathrm{e}^{\boldsymbol{A}\left(t-t_{0}\right)} \boldsymbol{x}_{0}+\int_{t_{0}}^{t} \mathrm{e}^{\boldsymbol{A}(t-\tau)} f(\tau) \mathrm{d} \tau
$$

例7.4.1 已知微分方程

$$
\left\{\begin{array}{l}
\frac{\mathrm{d} x_{1}}{\mathrm{~d} t}=2 x_{1}+1 \\
\frac{\mathrm{~d} x_{2}}{\mathrm{~d} t}=x_{1}+x_{2}+x_{3}-t \\
\frac{\mathrm{~d} x_{3}}{\mathrm{~d} t}=x_{1}-x_{2}+3 x_{3}+t
\end{array}\right.
$$

及初始条件

$$
x_{1}\left(t_{0}\right)=1, x_{2}\left(t_{0}\right)=0, x_{3}\left(t_{0}\right)=-1
$$

试求方程组解。\\
解 命

$$
\boldsymbol{A}=\left[\begin{array}{rrr}
2 & 0 & 0 \\
1 & 1 & 1 \\
1 & -1 & 3
\end{array}\right], \boldsymbol{x}=\left(x_{1}, x_{2}, x_{3}\right)^{\mathrm{T}}
$$

$$
f=(1,-t, t)^{\mathrm{T}}, x_{0}=(1,0,-1)^{\mathrm{T}}
$$

则原方程组可以表示成

$$
\frac{\mathrm{d} x}{\mathrm{~d} t}=A x+f, x\left(t_{0}\right)=x_{0}
$$

由定理 7.4.1 知

$$
\boldsymbol{x}(t)=\mathrm{e}^{\boldsymbol{A}\left(t-t_{0}\right)} \boldsymbol{x}_{0}+\int_{t_{0}}^{t} \mathrm{e}^{\boldsymbol{A}(t-\tau)} f(\tau) \mathrm{d} \tau
$$

参阅例6.3.1知

$$
\mathrm{e}^{A\left(t-t_{0}\right)}=\mathrm{e}^{2\left(t-t_{0}\right)}\left[\begin{array}{ccc}
1 & 0 & 0 \\
t-t_{0} & 1-t+t_{0} & t-t_{0} \\
t-t_{0} & -t+t_{0} & 1+t-t_{0}
\end{array}\right]
$$

于是

$$
\begin{gathered}
\mathrm{e}^{A\left(t-t_{0}\right)} x_{0}=\mathrm{e}^{2\left(t-t_{0}\right)}\left[\begin{array}{r}
1 \\
0 \\
-1
\end{array}\right] \\
\mathrm{e}^{A(t-\tau)} f(\tau)=\mathrm{e}^{2(t-\tau)}\left[\begin{array}{c}
1 \\
t-2 \tau+2 t \tau-2 \tau^{2} \\
t+2 t \tau-2 \tau^{2}
\end{array}\right]
\end{gathered}
$$

故

$$
\begin{aligned}
\int_{t_{0}}^{t} \mathrm{e}^{A(t-\tau)} f(\tau) \mathrm{d} \tau & =\int_{t_{0}}^{t} \mathrm{e}^{2(t-\tau)}\left[\begin{array}{c}
1 \\
t-2 \tau+2 t \tau-2 \tau^{2} \\
t+2 t \tau-2 \tau^{2}
\end{array}\right] \mathrm{d} \tau \\
& =\mathrm{e}^{2\left(t-t_{0}\right)}\left[\begin{array}{c}
\frac{1}{2} \\
-t_{0}^{2}-2 t_{0}+t t_{0}+t-\frac{3}{2} \\
-t_{0}^{2}-t_{0}+t t_{0}+t-1
\end{array}\right]+\left[\begin{array}{r}
-\frac{1}{2} \\
t+\frac{3}{2} \\
1
\end{array}\right]
\end{aligned}
$$

所以,所求解为

$$
\begin{aligned}
\boldsymbol{x}(t) & =\mathrm{e}^{\boldsymbol{A}\left(t-t_{0}\right)} \boldsymbol{x}_{0}+\int_{t_{0}}^{t} \mathrm{e}^{\boldsymbol{A}(t-\tau)} \boldsymbol{f}(\tau) \mathrm{d} \tau \\
& =\mathrm{e}^{2\left(t-t_{0}\right)}\left[\begin{array}{r}
\frac{3}{2} \\
-t_{0}^{2}-2 t_{0}+t t_{0}+t-\frac{3}{2} \\
-t_{0}^{2}-t_{0}+t t_{0}+t-2
\end{array}\right]+\left[\begin{array}{r}
-\frac{1}{2} \\
t+\frac{3}{2} \\
1
\end{array}\right]
\end{aligned}
$$

\section*{习 题}
7-1 已知线性常系数非齐次微分方程组

$$
\begin{aligned}
& \frac{\mathrm{d} x_{1}}{\mathrm{~d} t}=x_{1}+2 t \\
& \frac{\mathrm{~d} x_{2}}{\mathrm{~d} t}=-x_{1}+2 x_{2}+x_{3}+t \\
& \frac{\mathrm{~d} x_{3}}{\mathrm{~d} t}=2 x_{3}-3
\end{aligned}
$$

以及初始条件

$$
x_{1}(0)=1, x_{2}(0)=-1, x_{3}(0)=1
$$

试求方程组解。\\
7-2 已知线性常系数非齐次微分方程组

$$
\begin{gathered}
\frac{\mathrm{d} x}{\mathrm{~d} t}=\left[\begin{array}{rrr}
2 & 2 & 1 \\
-2 & 6 & 1 \\
0 & 0 & 4
\end{array}\right] x+\left[\begin{array}{r}
2 \\
-1 \\
t
\end{array}\right] \\
x(0)=[1,2,0]^{\mathrm{T}}
\end{gathered}
$$

试求 $x(t)$ .\\
7-3 已知线性常系数非齐次微分方程组

$$
\begin{aligned}
& \frac{\mathrm{d} x_{1}}{\mathrm{~d} t}=x_{1}+x_{2}-x_{3}+1 \\
& \frac{\mathrm{~d} x_{2}}{\mathrm{~d} t}=2 x_{1}+4 x_{2}-5 x_{3}+t \\
& \frac{\mathrm{~d} x_{3}}{\mathrm{~d} t}=x_{1}+2 x_{2}-2 x_{3}-t
\end{aligned}
$$

以及初始条件

$$
x_{1}(0)=1, x_{2}(0)=1, x_{3}(0)=0
$$

试求 $x_{1}(t), x_{2}(t), x_{3}(t)$ .

\section*{矩阵的广义逆}
近几十年来,人们为推广方阵逆矩阵的概念使之适用于研究各类数学问题,做了大量的工作,使得矩阵广义逆的基本知识成为矩阵分析的基础内容之一。本章仅简要介绍广义逆 $\boldsymbol{A}^{-}$和伪逆矩阵 $\boldsymbol{A}^{+}$,最后粗略介绍广义逆与线性方程组 $\boldsymbol{A} \boldsymbol{x}=\boldsymbol{b}$的关系。

\section*{§8.1 广义逆矩阵}
若 $\boldsymbol{A}$ 为可逆方阵,方程组 $\boldsymbol{A} \boldsymbol{x}=\boldsymbol{b}$ 有唯一解 $\boldsymbol{x}=\boldsymbol{A}^{-1} \boldsymbol{b}$ ,在 $\boldsymbol{A}$ 不是可逆方阵,即当 $\boldsymbol{A} \in C^{m \times n}, ~ \boldsymbol{x} \in C^{n}, \boldsymbol{b} \in C^{m}$ ,且 $\boldsymbol{b} \in R(\boldsymbol{A})$ 时,方程组 $\boldsymbol{A x}=\boldsymbol{b}$ 有解。这时是否能用某个矩阵 $\boldsymbol{G}$ 把方程组解表示成 $\boldsymbol{x}=\boldsymbol{G} \boldsymbol{b}$ 的形式。

定义8.1.1 若线性方程组


\begin{equation*}
A x=b \tag{8.1.1}
\end{equation*}


其中 $\boldsymbol{A} \in C^{m \times n}, \boldsymbol{b} \in C^{n}$ ,对于任意 $m$ 维向量 $\boldsymbol{b} \in R(A)$ ,有使解 $\boldsymbol{x}=\boldsymbol{A}^{-} \boldsymbol{b}$ 成立的 $\boldsymbol{A}^{-}$存在时,便称 $\boldsymbol{A}^{-}$是 $\boldsymbol{A}$ 的广义逆矩阵。

定理8.1.1 对于给定的 $\boldsymbol{A} \in C^{m \times n}$ ,广义逆矩阵 $\boldsymbol{A}^{-}$存在的充要条件是有满足


\begin{equation*}
\boldsymbol{A} \boldsymbol{A}^{-} \boldsymbol{A}=\boldsymbol{A} \tag{8.1.2}
\end{equation*}


的 $A^{-} \in C^{n \times m}$ 存在,或者说矩阵方程

$$
\boldsymbol{A} \boldsymbol{A}^{-} \boldsymbol{A}=\boldsymbol{A}
$$

有解。\\
证明 必要性 若 $\boldsymbol{A}^{-}$存在,即当 $\boldsymbol{b} \in R(\boldsymbol{A})$ 时,有 $\boldsymbol{x}=\boldsymbol{A}^{-} \boldsymbol{b}, \boldsymbol{A} \boldsymbol{A}^{-} \boldsymbol{b}=\boldsymbol{b}$ .设 $\boldsymbol{A}= \left(\boldsymbol{\alpha}_{1}, \boldsymbol{\alpha}_{2}, \cdots, \boldsymbol{\alpha}_{n}\right)$ ,取 $\boldsymbol{b}=\boldsymbol{\alpha}_{i}(i=1,2, \cdots, n)$ ,于是 $\boldsymbol{A} \boldsymbol{A}^{-} \boldsymbol{\alpha}_{i}=\boldsymbol{\alpha}_{i}(i=1,2, \cdots, n)$ 所以 $\boldsymbol{A} \boldsymbol{A}^{-} \boldsymbol{A}=\boldsymbol{A}$ .

充分性 若 $x \in C^{n}$ 是 $A x=b$ 的解,将 $x$ 右乘式(8.1.2)两边得,$A A^{-} A x=A x$ ,即 $\boldsymbol{A} \boldsymbol{A}^{-} \boldsymbol{b}=\boldsymbol{b}$ ,这意味着 $\boldsymbol{x}=\boldsymbol{A}^{-} \boldsymbol{b}$ 是 $\boldsymbol{A x}=\boldsymbol{b}$ 的解。

推论 $\operatorname{rank} \boldsymbol{A} \leqslant \operatorname{rank} \boldsymbol{A}^{-}$.\\
证明 $\quad \operatorname{rank} A=\operatorname{rank}\left(A A^{-} A\right) \leqslant \operatorname{rank}\left(A A^{-}\right) \leqslant \operatorname{rank} A^{-}$。\\
根据定理8.1.1,可把式(8.1.2)作为广义逆矩阵 $\boldsymbol{A}^{-}$的定义。\\
根据线性代数知,对于任何一个 $\boldsymbol{A} \in C_{r}^{m \times n}$ ,总存在 $\boldsymbol{P} \in C_{m}^{m \times m}, \boldsymbol{Q} \in C_{n}^{n \times n}$ ,满足

\[
P A Q=\left[\begin{array}{cc}
E_{r} & 0  \tag{8.1.3}\\
0 & 0
\end{array}\right]
\]

(其中, $\boldsymbol{P}$ 与 $\boldsymbol{Q}$ 都不是唯一的)\\
例8.1.1 设 $\boldsymbol{A} \in C_{r}^{m \times n}, ~ \boldsymbol{P}$ 与 $\boldsymbol{Q}$ 满足式(8.1.3),则

\[
M=Q\left[\begin{array}{cc}
E_{r} & X  \tag{8.1.4}\\
Y & Z
\end{array}\right] P
\]

是 $\boldsymbol{A}$ 的广义逆矩阵 $\boldsymbol{A}^{-}$.其中 $\boldsymbol{X} 、 \boldsymbol{Y} 、 \boldsymbol{Z}$ 为任意满足式(8.1.4)的矩阵.\\
解 由式(8.1.3)知

$$
A=P^{-1}\left[\begin{array}{cc}
E_{r} & 0 \\
0 & 0
\end{array}\right] Q^{-1}
$$

于是

$$
\begin{aligned}
A M A= & P^{-1}\left[\begin{array}{cc}
E_{r} & 0 \\
0 & 0
\end{array}\right] Q^{-1} \cdot Q\left[\begin{array}{cc}
E_{r} & X \\
Y & Z
\end{array}\right] P \cdot P^{-1}\left[\begin{array}{cc}
E_{r} & 0 \\
0 & 0
\end{array}\right] Q^{-1}= \\
& P^{-1}\left[\begin{array}{cc}
E_{r} & 0 \\
0 & 0
\end{array}\right] Q^{-1}=A
\end{aligned}
$$

例8.1.2 已知

$$
A=\left[\begin{array}{rrrr}
0 & -1 & 3 & 0 \\
2 & -4 & 1 & 5 \\
-4 & 5 & 7 & -10
\end{array}\right]
$$

试求表示成式(8.1.4)的广义逆矩阵 $\boldsymbol{M}$ 。\\
解 先介绍如何求满足式(8.1.3)的矩阵 $\boldsymbol{P}$ 与 $\boldsymbol{Q}$(什么道理请读者自己分析).

$$
\begin{aligned}
& {\left[\begin{array}{c:c}
\boldsymbol{A} & \boldsymbol{E}_{m} \\
\hdashline \boldsymbol{E}_{n} & 0
\end{array}\right]=} {\left[\begin{array}{rrrr:rrr}
0 & -1 & 3 & 0 & 1 & 0 & 0 \\
2 & -4 & 1 & 5 & 0 & 1 & 0 \\
-4 & 5 & 7 & -10 & 0 & 0 & 1 \\
\hdashline 1 & 0 & 0 & 0 & & \\
0 & 1 & 0 & 0 & 0 & \\
0 & 0 & 1 & 0 & &
\end{array}\right] } \\
& \xrightarrow[\substack{0 \\
\left(\text { 对 } \boldsymbol{A} \text { 作初等列变换 } \boldsymbol{Q} \\
\text { (同时时 } \boldsymbol{E}_{n}\right. \text { 作列变换) }}]{\substack{0 \\
\text { (同) }}}\left[\begin{array}{llll:ccc}
1 & 0 & 0 & 0 & -2 & \frac{1}{2} & 0 \\
0 & 1 & 0 & 0 & -1 & 0 & 0 \\
0 & 0 & 0 & 0 & -3 & 2 & 1 \\
\hdashline 1 & 0 & \frac{11}{2} & -\frac{5}{2} & & & \\
0 & 1 & 3 & 0 & 0 & \\
0 & 0 & 1 & 0 & & \\
0 & 0 & 0 & 1 & &
\end{array}\right]
\end{aligned}
$$

于是

$$
\boldsymbol{P}=\left[\begin{array}{lll}
-2 & \frac{1}{2} & 0 \\
-1 & 0 & 0 \\
-3 & 2 & 1
\end{array}\right], \quad \boldsymbol{Q}=\left[\begin{array}{rrrr}
1 & 0 & \frac{11}{2} & -\frac{5}{2} \\
0 & 1 & 3 & 0 \\
0 & 0 & 1 & 0 \\
0 & 0 & 0 & 1
\end{array}\right]
$$

所以广义逆矩阵

$$
M=Q\left[\begin{array}{cc}
E_{2} & X \\
Y & Z
\end{array}\right] P
$$

其中 $\boldsymbol{X} \in C^{2 \times 1}, \boldsymbol{Y} \in C^{2 \times 2}, \boldsymbol{Z} \in C^{2 \times 1}$ 。\\
例8.1.3 设 $\boldsymbol{A}^{-}$为 $\boldsymbol{A} \in C^{m \times n}$ 的一个广义逆矩阵,则对任意的 $\boldsymbol{V} \in C^{n \times m}$ 与 $\boldsymbol{W} \in C^{m \times m}, n \times m$ 矩阵


\begin{equation*}
\boldsymbol{X}=\boldsymbol{A}^{-}+\boldsymbol{V}\left(\boldsymbol{E}_{m}-\boldsymbol{A} \boldsymbol{A}^{-}\right)+\left(\boldsymbol{E}_{n}-\boldsymbol{A}^{-} \boldsymbol{A}\right) \boldsymbol{W} \tag{8.1.5}
\end{equation*}


是 $\boldsymbol{A}$ 的广义逆矩阵。\\
解

$$
\begin{aligned}
A X A & =A\left[A^{-}+V\left(E_{m}-A A^{-}\right)+\left(E_{n}-A^{-} A\right) W\right] A \\
& =A A^{-} A+A V\left(E_{m}-A A^{-}\right) A+A\left(E_{n}-A-A\right) W A \\
& =A+A V A-A V A A^{-} A+A W A-A A^{-} A W A \\
& =A+A V A-A V A+A W A-A W A \\
& =A
\end{aligned}
$$

例8.1.4 设 $A^{-}$是 $A \in C^{m \times n}$ 的一个广义逆矩阵,$V \in C^{n \times m}$ 为任意矩阵,则


\begin{equation*}
X=A^{-}+V-A^{-} A V A A^{-} \tag{8.1.6}
\end{equation*}


是 $\boldsymbol{A}$ 的广义逆矩阵\\
解 $\boldsymbol{A} \boldsymbol{X A}=\boldsymbol{A}\left(\boldsymbol{A}^{-}+\boldsymbol{V}^{-} \boldsymbol{A}^{-} \boldsymbol{A} \boldsymbol{V A A ^ { - }}\right) \boldsymbol{A}$

$$
\begin{aligned}
& =A A^{-} A+A V A-A A^{-} A V A A^{-} A \\
& =A+A V A-A V A=A
\end{aligned}
$$

关于广义逆矩阵 $\boldsymbol{A}^{-}$的运算性质有定理 8.1.2.\\
定理8.1.2 设 $A \in C^{m \times n}, \lambda \in \mathbf{R}$ 则\\
(1)$\left(\boldsymbol{A}^{\mathrm{T}}\right)^{-}=\left(\boldsymbol{A}^{-}\right)^{\mathrm{T}},\left(\boldsymbol{A}^{\mathrm{H}}\right)^{-}=\left(\boldsymbol{A}^{-}\right)^{\mathrm{H}}$\\
(2)若 $\boldsymbol{A} \in C_{n}^{n \times n}$ ,则 $\boldsymbol{A}^{-}=\boldsymbol{A}^{-1}$ ,且 $\boldsymbol{A}^{-}$唯一\\
(3)$(\lambda \boldsymbol{A})^{-}=\lambda^{+} \boldsymbol{A}^{-}, \quad \lambda \in \mathbf{R}$

$$
\lambda^{+}= \begin{cases}\frac{1}{\lambda} & \text { 当 } \lambda \neq 0 \text { 时 } \\ 0 & \text { 当 } \lambda=0 \text { 时 }\end{cases}
$$

(4)设 $S \in C_{m}^{m \times m}, T \in C_{n}^{n \times n}$ ,且 $B=S A T$ ,则

$$
(S A T)^{-}=T^{-1} A^{-} S^{-1}
$$

(5) $\boldsymbol{A} \boldsymbol{A}^{-}$与 $\boldsymbol{A}^{-} \boldsymbol{A}$ 都是幂等矩阵,且

$$
\operatorname{rank} A=\operatorname{rank} A A^{-}=\operatorname{rank} A^{-} A
$$

证明(1)证略。\\
(2)因为 $\boldsymbol{A}^{-1}=\boldsymbol{A}^{-1} \boldsymbol{A} \boldsymbol{A}^{-1}=\boldsymbol{A}^{-1}\left(\boldsymbol{A} \boldsymbol{A}^{-} \boldsymbol{A}\right) \boldsymbol{A}^{-1}=\boldsymbol{A}^{-}$。\\
(3)当 $\lambda=0$ 时,公式显然成立.当 $\lambda \neq 0$ 时,$(\lambda \boldsymbol{A})\left(\lambda^{+} \boldsymbol{A}^{-}\right)(\lambda \boldsymbol{A})=(\lambda \boldsymbol{A}) \left(\frac{1}{\lambda} \boldsymbol{A}^{-}\right)(\lambda \boldsymbol{A})=\lambda \boldsymbol{A} \boldsymbol{A}^{-} \boldsymbol{A}=\lambda \boldsymbol{A}$ ,故 $(\lambda \boldsymbol{A})^{-}=\lambda^{+} \boldsymbol{A}^{-}$.\\
(4)因为(SAT)$\left(\boldsymbol{T}^{-1} \boldsymbol{A}^{-} \boldsymbol{S}^{-1}\right)(\boldsymbol{S A T})$

$$
=S A A^{-} A T=S A T .
$$

故 $\quad(S A T)^{-}=T^{-1} A^{-} S^{-1}$ .\\
(5)因为 $\quad\left(\boldsymbol{A} \boldsymbol{A}^{-}\right)^{2}=\boldsymbol{A} \boldsymbol{A}^{-} \boldsymbol{A} \boldsymbol{A}^{-}=\boldsymbol{A} \boldsymbol{A}^{-}$

$$
\left(A^{-} A\right)^{2}=A^{-} A A^{-} A=A^{-} A .
$$

故 $\boldsymbol{A} \boldsymbol{A}^{-}, \boldsymbol{A}^{-} \boldsymbol{A}$ 都是幂等矩阵,且

$$
\operatorname{rank} A=\operatorname{rank}\left(A A^{-} A\right) \leqslant \operatorname{rank}\left(A A^{-}\right) \leqslant \operatorname{rank} A
$$

于是

$$
\operatorname{rank} \boldsymbol{A}=\operatorname{rank} \boldsymbol{A} \boldsymbol{A}^{-}
$$

类似可得

$$
\operatorname{rank} \boldsymbol{A}=\operatorname{rank} \boldsymbol{A}^{-} \boldsymbol{A}
$$

定义8.1.2 设 $A \in C^{m \times n}$ ,若存在矩阵 $A_{\mathrm{L}}^{-1} \in C^{n \times m}$(或 $A_{\mathrm{R}}^{-1} \in C^{n \times m}$ ),使得


\begin{equation*}
\boldsymbol{A}_{\mathrm{L}}^{-1} \boldsymbol{A}=\boldsymbol{E}_{n} \quad\left(\text { 或 } \boldsymbol{A} \boldsymbol{A}_{\mathrm{R}}^{-1}=\boldsymbol{E}_{m}\right) \tag{8.1.7}
\end{equation*}


则称 $\boldsymbol{A}_{\mathrm{L}}^{-1}$(或 $\boldsymbol{A}_{\mathrm{R}}^{-1}$ )是 $\boldsymbol{A}$ 的左逆(或右逆)。\\
若 $m=n$ ,且 $\boldsymbol{A}$ 是满秩的,则 $\boldsymbol{A}^{-1}=\boldsymbol{A}_{\mathrm{L}}^{-1}=\boldsymbol{A}_{\mathrm{R}}^{-1}$ .\\
定理8.1.3 设 $A \in C^{m \times n}$ ,则\\
(1) $\boldsymbol{A} \in C_{n}^{m \times n}$ 的充要条件是 $\boldsymbol{A}^{-} \boldsymbol{A}=\boldsymbol{E}_{n}$\\
(2) $\boldsymbol{A} \in C_{m}^{m \times n}$ 的充要条件是 $\boldsymbol{A} \boldsymbol{A}^{-}=\boldsymbol{E}_{m}$\\
证明(1)必要性 因为

$$
\operatorname{rank} A=\operatorname{rank} A^{-} A=n
$$

所以 $\boldsymbol{A}^{-} \boldsymbol{A}$ 可逆,于是

$$
\begin{aligned}
\boldsymbol{E}_{n} & =\left(\boldsymbol{A}^{-} \boldsymbol{A}\right)\left(\boldsymbol{A}^{-} \boldsymbol{A}\right)^{-1}=\left(\boldsymbol{A}^{-} \boldsymbol{A}\right)\left(\boldsymbol{A}^{-} \boldsymbol{A}\right)\left(\boldsymbol{A}^{-} \boldsymbol{A}\right)^{-1}\left(\boldsymbol{A}^{-} \boldsymbol{A}\right)^{-1} \\
& =\boldsymbol{A}^{-} \boldsymbol{A} \boldsymbol{A}^{-} \boldsymbol{A}\left(\boldsymbol{A}^{-} \boldsymbol{A}\right)^{-1}\left(\boldsymbol{A}^{-} \boldsymbol{A}\right)^{-1} \\
& =\boldsymbol{A}^{-} \boldsymbol{A}\left(\boldsymbol{A}^{-} \boldsymbol{A}\right)^{-1}\left(\boldsymbol{A}^{-} \boldsymbol{A}\right)^{-1}=\left(\boldsymbol{A}^{-} \boldsymbol{A}\right)^{-1}
\end{aligned}
$$

故 $\boldsymbol{A}^{-} \boldsymbol{A}=\boldsymbol{E}$\\
充分性 若 $\boldsymbol{A}^{-} \boldsymbol{A}=\boldsymbol{E}_{n}$ ,则

$$
\operatorname{rank} A=\operatorname{rank}\left(A^{-} A\right)=\operatorname{rank} E_{n}=n
$$

(2)类似(1)证明.

注 若 $\boldsymbol{A} \in C_{n}^{m \times n}$ ,则 $\left(\boldsymbol{A}^{H} \boldsymbol{A}\right)^{-1} \boldsymbol{A}^{H}$ 是 $\boldsymbol{A}_{\mathrm{L}}^{-1}$ 中的一个.若 $\boldsymbol{A} \in C_{m}^{m \times n}$ ,则 $\boldsymbol{A}^{H} \left(\boldsymbol{A} \boldsymbol{A}^{H}\right)^{-1}$ 是 $\boldsymbol{A}_{\mathrm{R}}^{-1}$ 中的一个。

下边研究附加条件的广义逆矩阵。\\
定义8.1.3 设 $A \in C^{m \times n}$ ,使


\begin{gather*}
A A^{-} A=A  \tag{8.1.2}\\
A^{-} A A^{-}=A^{-} \tag{8.1.8}
\end{gather*}


成立的 $\boldsymbol{A}^{-} \in C^{n \times m}$ 称为 $\boldsymbol{A}$ 的自反广义逆,记做 $\boldsymbol{A}_{r}^{-}$.\\
定理8.1.4 设 $\boldsymbol{X}, \boldsymbol{Y} \in C^{n \times m}$ 都是 $\boldsymbol{A} \in C^{m \times n}$ 的广义逆矩阵,则

$$
Z=X A Y
$$

是 $\boldsymbol{A}$ 的自反广义逆矩阵\\
(证略)\\
定理8.1.5 设 $\boldsymbol{A}^{-} \in C^{n \times m}$ 是 $\boldsymbol{A} \in C^{m \times n}$ 的广义逆矩阵,则 $\boldsymbol{A}^{-}$是 $\boldsymbol{A}$ 的自反广义逆矩阵的充要条件是 $\operatorname{rank} A=\operatorname{rank} A^{-}$.

证明 充分性 设 $\boldsymbol{A}^{-}$满足, $\boldsymbol{A} \boldsymbol{A}^{-} \boldsymbol{A}=\boldsymbol{A}$ 且 $\operatorname{rank} \boldsymbol{A}=\operatorname{rank} \boldsymbol{A}^{-}$.\\
欲证 $\boldsymbol{A}^{-} \boldsymbol{A} \boldsymbol{A}^{-}=\boldsymbol{A}^{-}$。\\
由 $\boldsymbol{A} \boldsymbol{A}^{-} \boldsymbol{A}=\boldsymbol{A}$ 知


\begin{equation*}
\operatorname{rank}\left(A A^{-}\right) \geqslant \operatorname{rank} A=\operatorname{rank} A^{-} \tag{1}
\end{equation*}


另一方面有


\begin{equation*}
\operatorname{rank}\left(A A^{-}\right) \leqslant \operatorname{rank} A^{-} \tag{2}
\end{equation*}


由式(1)与式(2)得

$$
\operatorname{rank}\left(A A^{-}\right)=\operatorname{rank} A^{-}
$$

因此,线性方程组

$$
A A^{-} X=0 \quad \text { 与 } \quad A^{-} X=0
$$

是同解方程组。\\
因为 $\boldsymbol{A}^{-}$满足 $\boldsymbol{A} \boldsymbol{A}^{-} \boldsymbol{A}=\boldsymbol{A}$ ,故 $\boldsymbol{A} \boldsymbol{A}^{-}-\boldsymbol{A} \boldsymbol{A}^{-} \boldsymbol{A} \boldsymbol{A}^{-}=0$ 于是

$$
A A^{-}\left(E-A A^{-}\right)=0
$$

这说明 $\boldsymbol{E}-\boldsymbol{A} \boldsymbol{A}^{-}$的列向量是方程组 $\boldsymbol{A} \boldsymbol{A}^{-} \boldsymbol{X}=0$ 的解,因此它也是 $\boldsymbol{A}^{-} \boldsymbol{X}=0$ 的解,此即

$$
A^{-}\left(E-A A^{-}\right)=0
$$

或

$$
A^{-} A A^{-}=A^{-}
$$

必要性 由 $\boldsymbol{A} \boldsymbol{A}^{-} \boldsymbol{A}=\boldsymbol{A}$ 得

$$
\operatorname{rank} \boldsymbol{A} \leqslant \operatorname{rank} \boldsymbol{A}^{-}
$$

由 $\boldsymbol{A}^{-} \boldsymbol{A} \boldsymbol{A}^{-}=\boldsymbol{A}^{-}$得

$$
\operatorname{rank} \boldsymbol{A}^{-} \leqslant \operatorname{rank} \boldsymbol{A}
$$

因此

$$
\operatorname{rank} A^{-}=\operatorname{rank} A
$$

例8.1.5 设 $A \in C_{r}^{m \times n}, A \neq 0, A$ 的满秩分解式 $A=B C$ ,其中 $B \in C_{r}^{m \times r}, C \in C_{r}^{r \times n}$ ,则

$$
\boldsymbol{A}_{r}^{-}=\boldsymbol{C}_{\mathrm{R}}^{-1} \boldsymbol{B}_{\mathrm{L}}^{-1}
$$

是 $\boldsymbol{A}$ 的自反广义逆\\
(证略)

\section*{§8.2 伪逆矩阵}
广义逆矩阵 $\boldsymbol{A}^{-}$满足式(8.1.2),自反广义逆矩阵 $\boldsymbol{A}_{r}^{-}$满足式(8.1.2)与式 (8.1.8).本节介绍的伪逆矩阵 $\boldsymbol{A}^{+}$除了满足式(8.1.2)与式(8.1.8)外,还需满足式(8.2.1),此即定义8.2.1.

定义8.2.1 设 $\boldsymbol{A} \in C^{m \times n}$ ,若 $\boldsymbol{A}^{+} \in C^{n \times m}$ ,且 $\boldsymbol{A}^{+}$满足如下四个公式:

\[
\begin{array}{cl}
\boldsymbol{A} \boldsymbol{A}^{+} \boldsymbol{A}=\boldsymbol{A} & \boldsymbol{A}^{+} \boldsymbol{A} \boldsymbol{A}^{+}=\boldsymbol{A}^{+} \\
\left(\boldsymbol{A} \boldsymbol{A}^{+}\right)^{\mathrm{H}}=\boldsymbol{A} \boldsymbol{A}^{+} & \left(\boldsymbol{A}^{+} \boldsymbol{A}\right)^{\mathrm{H}}=\boldsymbol{A}^{+} \boldsymbol{A} \tag{8.2.1}
\end{array}
\]

则称 $\boldsymbol{A}^{+}$是 $\boldsymbol{A}$ 的伪逆矩阵,上述四个条件称为 Penrose-Moore 方程。\\
显然,伪逆矩阵 $\boldsymbol{A}^{+}$是特殊的自反广义逆矩阵 $\boldsymbol{A}_{r}^{-}$。\\
定理8.2.1 设 $A \in C^{m \times n}, A=B C$ 是 $A$ 的一个满秩分解,则


\begin{equation*}
\boldsymbol{X}=\boldsymbol{C}^{\mathrm{H}}\left(\boldsymbol{C} \boldsymbol{C}^{\mathrm{H}}\right)^{-1}\left(\boldsymbol{B}^{\mathrm{H}} \boldsymbol{B}\right)^{-1} \boldsymbol{B}^{\mathrm{H}} \tag{8.2.2}
\end{equation*}


是 $\boldsymbol{A}$ 的伪逆矩阵。\\
证明 把 $X$ 代人定义8.2.1中四个条件验算即可.\\
推论 若 $A \in C_{r}^{m \times r}$ ,则


\begin{equation*}
\boldsymbol{A}^{+}=\left(\boldsymbol{A}^{\mathrm{H}} \boldsymbol{A}\right)^{-1} \boldsymbol{A}^{\mathrm{H}} \tag{8.2.3}
\end{equation*}


若 $\boldsymbol{A} \in C_{r}^{r \times n}$ ,则


\begin{equation*}
\boldsymbol{A}^{+}=\boldsymbol{A}^{\mathrm{H}}\left(\boldsymbol{A} \boldsymbol{A}^{\mathrm{H}}\right)^{-1} \tag{8.2.4}
\end{equation*}


定理8.2.2 伪逆矩阵 $\boldsymbol{A}^{+}$唯一。\\
证明 设 $\boldsymbol{X} 、 \boldsymbol{Y}$ 都是 $\boldsymbol{A}$ 的伪逆矩阵,即 $\boldsymbol{X}, \boldsymbol{Y}$ 都满足定义8.2.1中的四个等式,所以


\begin{align*}
\boldsymbol{X} & =\boldsymbol{X} \boldsymbol{A} \boldsymbol{X}=\boldsymbol{X} \boldsymbol{A} \boldsymbol{Y} \boldsymbol{A} \boldsymbol{X}=\boldsymbol{X}(\boldsymbol{A} \boldsymbol{Y})^{\mathrm{H}}(\boldsymbol{A} \boldsymbol{X})^{\mathrm{H}} \\
& =\boldsymbol{X}(\boldsymbol{A} \boldsymbol{X} \boldsymbol{A} \boldsymbol{Y})^{\mathrm{H}}=\boldsymbol{X}(\boldsymbol{A} \boldsymbol{Y})^{\mathrm{H}}=\boldsymbol{X} \boldsymbol{A} \boldsymbol{Y} \\
& =\boldsymbol{X} \boldsymbol{A} \boldsymbol{Y} \boldsymbol{A} \boldsymbol{Y}=(\boldsymbol{X} \boldsymbol{A})^{\mathrm{H}}(\boldsymbol{Y} \boldsymbol{A})^{\mathrm{H}} \boldsymbol{Y} \\
& =(\boldsymbol{Y} \boldsymbol{A} \boldsymbol{X} \boldsymbol{A})^{\mathrm{H}} \boldsymbol{Y}=(\boldsymbol{Y} \boldsymbol{A})^{\mathrm{H}} \boldsymbol{Y}=\boldsymbol{Y} \boldsymbol{A} \boldsymbol{Y}=\boldsymbol{Y} . \tag{证毕}
\end{align*}


根据定理8.2.2知,若 $\boldsymbol{A} \in C_{n}^{n \times n}$ ,则 $\boldsymbol{A}^{+}=\boldsymbol{A}^{-1}$ 。\\
伪逆矩阵 $\boldsymbol{A}^{+}$除满足定理8.1.2的性质外,它还有如定理8.2.3的性质。\\
定理 8.2.3 设 $\boldsymbol{A} \in C^{m \times n}$ 则\\
(1)$\left(A^{+}\right)^{+}=A$\\
(2)$\left(\boldsymbol{A A}^{\mathrm{H}}\right)^{+}=\left(\boldsymbol{A}^{\mathrm{H}}\right)^{+} \boldsymbol{A}^{+}=\left(\boldsymbol{A}^{+}\right)^{\mathrm{H}} \boldsymbol{A}^{+}$

$$
\left(\boldsymbol{A}^{\mathrm{H}} \boldsymbol{A}\right)^{+}=\boldsymbol{A}^{+}\left(\boldsymbol{A}^{\mathrm{H}}\right)^{+}=\boldsymbol{A}^{+}\left(\boldsymbol{A}^{+}\right)^{\mathrm{H}}
$$

(3) $\boldsymbol{A}^{+}=\boldsymbol{A}^{\mathrm{H}}\left(\boldsymbol{A} \boldsymbol{A}^{\mathrm{H}}\right)^{+}=\left(\boldsymbol{A}^{\mathrm{H}} \boldsymbol{A}\right)^{+} \boldsymbol{A}^{\mathrm{H}}$\\
证明 容易验证(1),(2),现在只证(3)。

$$
\begin{gathered}
\boldsymbol{A}^{+}=\boldsymbol{A}^{+} \boldsymbol{A} \boldsymbol{A}^{+}=\left(\boldsymbol{A}^{+} \boldsymbol{A}\right)^{\mathrm{H}} \boldsymbol{A}^{+}=\boldsymbol{A}^{\mathrm{H}}\left(\boldsymbol{A}^{+}\right)^{\mathrm{H}} \boldsymbol{A}^{+}=\boldsymbol{A}^{\mathrm{H}}\left(\boldsymbol{A} \boldsymbol{A}^{\mathrm{H}}\right)+ \\
\boldsymbol{A}^{+}=\boldsymbol{A}^{+} \boldsymbol{A} \boldsymbol{A}^{+}=\boldsymbol{A}^{+}\left(\boldsymbol{A} \boldsymbol{A}^{+}\right)^{\mathrm{H}}=\boldsymbol{A}^{+}\left(\boldsymbol{A}^{+}\right)^{\mathrm{H}} \boldsymbol{A}^{\mathrm{H}}=\left(\boldsymbol{A}^{\mathrm{H}} \boldsymbol{A}\right)^{+} \boldsymbol{A}^{\mathrm{H}}(\text { 证毕 })
\end{gathered}
$$

不难验证:若 $\boldsymbol{A}=\operatorname{diag}\left(\lambda_{1}, \lambda_{2}, \cdots, \lambda_{n}\right)$ ,则 $\boldsymbol{A}^{+}=\operatorname{diag}\left(\mu_{1}, \mu_{2}, \cdots, \mu_{n}\right)$ 其中

$$
\mu_{i}=\left\{\begin{array}{cl}
\lambda_{i}^{-1} & \text { 当 } \lambda_{i} \neq 0 \text { 时 } \\
0 & \text { 当 } \lambda_{i}=0 \text { 时 }
\end{array}\right.
$$

例 8.2.1 设 $\boldsymbol{A} \in C_{r}^{m \times n}$ ,则 $\boldsymbol{A}^{\mathrm{H}} \boldsymbol{A}$ 是正定式或半正定的 Hermite 矩阵,故存在 $U \in U^{n \times n}$ ,使得

$$
\boldsymbol{U}^{\mathrm{H}} \boldsymbol{A}^{\mathrm{H}} \boldsymbol{A} \boldsymbol{U}=\operatorname{diag}\left(\lambda_{1}, \lambda_{2}, \cdots, \lambda_{n}\right)=\boldsymbol{\Lambda}
$$

(其中 $\lambda_{1}, \lambda_{2}, \cdots, \lambda_{n}$ 是 $\boldsymbol{A}^{\mathrm{H}} \boldsymbol{A}$ 的 $n$ 个实特征值).\\
试证:


\begin{equation*}
\boldsymbol{A}^{+}=\boldsymbol{U}_{\boldsymbol{\Lambda}}{ }^{+} \boldsymbol{U}^{\mathrm{H}} \boldsymbol{A}^{\mathrm{H}} \tag{8.2.5}
\end{equation*}


证明 因为

$$
\boldsymbol{A}^{\mathrm{H}} \boldsymbol{A}=\boldsymbol{U} \operatorname{diag}\left(\lambda_{1}, \lambda_{2}, \cdots, \lambda_{n}\right) \boldsymbol{U}^{\mathrm{H}}=\boldsymbol{U} \boldsymbol{\Lambda} \boldsymbol{U}^{\mathrm{H}}
$$

不妨设 $\lambda_{1}, \lambda_{2}, \cdots, \lambda_{r} \neq 0, \lambda_{r+1}=\lambda_{r+2}=\cdots=\lambda_{n}=0$ ,则

$$
\begin{aligned}
\boldsymbol{A}^{\mathrm{H}} \boldsymbol{A} & =\boldsymbol{U}\left[\begin{array}{ccccc}
\lambda_{1} & & & & \\
& \ddots & & & \\
& & \lambda_{r} & & \\
& & & 0 & \\
& & & & \ddots \\
& & & & 0
\end{array}\right] \boldsymbol{U}^{\mathrm{H}} \\
& =\boldsymbol{U}\left[\begin{array}{cccc}
\sqrt{\lambda_{1}} & 0 & \cdots & 0 \\
0 & \sqrt{\lambda_{2}} & \cdots & 0 \\
\vdots & \vdots & & \vdots \\
0 & 0 & \cdots & \sqrt{\lambda_{r}} \\
0 & 0 & \cdots & 0 \\
\vdots & \vdots & & \vdots \\
0 & 0 & \cdots & 0
\end{array}\right]_{n \times r}
\end{aligned}
$$

$$
\begin{aligned}
& {\left[\begin{array}{ccccccc}
\sqrt{\lambda_{1}} & 0 & \cdots & 0 & 0 & \cdots & 0 \\
0 & \sqrt{\lambda_{2}} & \cdots & 0 & 0 & \cdots & 0 \\
\vdots & \vdots & & \vdots & \vdots & & \vdots \\
0 & 0 & \cdots & \sqrt{\lambda_{r}} & 0 & \cdots & 0
\end{array}\right]_{r \times n} U^{\mathrm{H}}} \\
& =U \boldsymbol{\Lambda}_{1} \boldsymbol{\Lambda}_{1}{ }^{\mathrm{H}} U^{\mathrm{H}}=B C
\end{aligned}
$$

其中

$$
\begin{gathered}
\boldsymbol{\Lambda}_{1}=\left[\begin{array}{cccc}
\sqrt{\lambda_{1}} & 0 & \cdots & 0 \\
0 & \sqrt{\lambda_{2}} & \cdots & 0 \\
\vdots & \vdots & & \vdots \\
0 & 0 & \cdots & \sqrt{\lambda_{r}} \\
0 & 0 & \cdots & 0 \\
\vdots & \vdots & & \vdots \\
0 & 0 & \cdots & 0
\end{array}\right]_{n \times r} \\
\boldsymbol{B}=\boldsymbol{U}_{1} \in C_{r}^{n \times r}, \quad \boldsymbol{C}=\boldsymbol{\Lambda}_{1}{ }^{\mathrm{H}} \boldsymbol{U}^{\mathrm{H}} \in C_{r}^{r \times n} \\
\boldsymbol{\Lambda}_{1} \boldsymbol{\Lambda}_{1}{ }^{\mathrm{H}}=\boldsymbol{\Lambda} \in C_{r}^{n \times n}, \quad \boldsymbol{\Lambda}_{1}{ }^{\mathrm{H}} \boldsymbol{\Lambda}_{1}=\left[\begin{array}{cccc}
\lambda_{1} & & & \\
& \lambda_{2} & & \\
& & \ddots & \\
& & & \lambda_{r}
\end{array}\right]_{r \times r}=\boldsymbol{\Lambda}_{r} \in C_{r}^{r \times r}
\end{gathered}
$$

根据式(8.2.2)得

$$
\begin{aligned}
\left(\boldsymbol{A}^{\mathrm{H}} \boldsymbol{A}\right)^{+} & =\boldsymbol{U} \boldsymbol{\Lambda}_{1}\left(\boldsymbol{\Lambda}_{1}{ }^{\mathrm{H}} \boldsymbol{U}^{\mathrm{H}} \boldsymbol{U} \boldsymbol{\Lambda}_{1}\right)^{-1}\left(\boldsymbol{\Lambda}_{1}^{\mathrm{H}} \boldsymbol{U}^{\mathrm{H}} \boldsymbol{U} \boldsymbol{\Lambda}_{1}\right)^{-1} \boldsymbol{\Lambda}_{1}^{\mathrm{H}} \boldsymbol{U}^{\mathrm{H}} \\
& =\boldsymbol{U} \boldsymbol{\Lambda}_{1}\left(\boldsymbol{\Lambda}_{1}{ }^{\mathrm{H}} \boldsymbol{\Lambda}_{1}\right)^{-1}\left(\boldsymbol{\Lambda}_{1}^{\mathrm{H}} \boldsymbol{\Lambda}_{1}\right)^{-1} \boldsymbol{\Lambda}_{1}^{\mathrm{H}} \boldsymbol{U}^{\mathrm{H}} \\
& =\boldsymbol{U} \boldsymbol{\Lambda}_{1} \boldsymbol{\Lambda}_{r}^{-1} \boldsymbol{\Lambda}_{r}^{-1} \boldsymbol{\Lambda}_{1}^{\mathrm{H}} \boldsymbol{U}^{\mathrm{H}}=\boldsymbol{U} \boldsymbol{\Lambda}^{+} \boldsymbol{U}^{\mathrm{H}}
\end{aligned}
$$

因此,根据定理8.2.3(3)得

$$
\boldsymbol{A}^{+}=\left(\boldsymbol{A}^{\mathrm{H}} \boldsymbol{A}\right)^{+} \boldsymbol{A}^{\mathrm{H}}=\boldsymbol{U} \boldsymbol{\Lambda}^{+} \boldsymbol{U}^{\mathrm{H}} \boldsymbol{A}^{\mathrm{H}}
$$

例8.2.2 已知

$$
A=\left[\begin{array}{rrr}
-1 & 0 & 1 \\
2 & 0 & -2
\end{array}\right]
$$

求 $\boldsymbol{A}$ 的伪逆矩阵 $\boldsymbol{A}^{+}$。\\
解 易知 $\boldsymbol{A}^{\mathrm{H}} \boldsymbol{A}$ 的特征值 $\lambda_{1}=10, \lambda_{2}=\lambda_{3}=0$ ,当 $\lambda_{1}=10$ 时, $\boldsymbol{A}^{\mathrm{H}} \boldsymbol{A}$ 的单位特征向量为

$$
X_{1}=\left(\frac{1}{\sqrt{2}}, 0,-\frac{1}{\sqrt{2}}\right)^{\mathrm{T}}
$$

当 $\lambda_{2}=\lambda_{3}=0$ 时, $\boldsymbol{A}^{\mathrm{H}} \boldsymbol{A}$ 的单位正交特征向量为

$$
\boldsymbol{X}_{2}=\left(\frac{1}{\sqrt{2}}, 0, \frac{1}{\sqrt{2}}\right)^{\mathrm{T}}, \boldsymbol{X}_{3}=(0,1,0)^{\mathrm{T}}
$$

于是

$$
\begin{aligned}
U=\left(X_{1}, X_{2}, X_{3}\right)= & {\left[\begin{array}{ccc}
\frac{1}{\sqrt{2}} & \frac{1}{\sqrt{2}} & 0 \\
0 & 0 & 1 \\
-\frac{1}{\sqrt{2}} & \frac{1}{\sqrt{2}} & 0
\end{array}\right] \quad \boldsymbol{A}=\left[\begin{array}{ccc}
10 & 0 & 0 \\
0 & 0 & 0 \\
0 & 0 & 0
\end{array}\right] } \\
& \Lambda^{+}=\left[\begin{array}{ccc}
\frac{1}{10} & 0 & 0 \\
0 & 0 & 0 \\
0 & 0 & 0
\end{array}\right]
\end{aligned}
$$

代人式(8.2.5)得

$$
\boldsymbol{A}^{+}=\boldsymbol{U A}^{+} \boldsymbol{U}^{\mathrm{H}} \boldsymbol{A}^{\mathrm{H}}=\left[\begin{array}{rr}
-\frac{1}{10} & \frac{1}{5} \\
0 & 0 \\
\frac{1}{10} & -\frac{1}{5}
\end{array}\right]
$$

在例8.2.1中的公式(8.2.5)经常可以进一步简化,这就是下述例8.2.3.例8.2.3 设 $\boldsymbol{A} \in C_{r}^{m \times n}, \lambda_{1}, \lambda_{2}, \cdots, \lambda_{r}$ 是 $\boldsymbol{A}^{\mathrm{H}} \boldsymbol{A}$ 的非零特征值,且

$$
\boldsymbol{\Lambda}_{r}=\left[\begin{array}{llll}
\lambda_{1} & & & \\
& \lambda_{2} & & \\
& & \ddots & \\
& & & \lambda_{r}
\end{array}\right]
$$

$\boldsymbol{U}=\left(\boldsymbol{U}_{1}, \boldsymbol{U}_{2}\right), \boldsymbol{U}_{1} \in U_{r}^{n \times r}, \boldsymbol{U}_{2} \in U_{n-r}^{n \times(n-r)}$ ,则式(8.2.5)可改写为


\begin{equation*}
\boldsymbol{A}^{+}=U_{1} \boldsymbol{\Lambda}_{r}^{-1} U_{1}^{\mathrm{H}} \boldsymbol{A}^{\mathrm{H}} \tag{8.2.6}
\end{equation*}


\section*{§8.3 广义逆与线性方程组}
\section*{一、矩阵方程}
定理 8.3.1 设 $A \in C^{m \times n}, B \in C^{s \times t}, D \in C^{m \times t}$ ,则矩阵方程


\begin{equation*}
A X B=D \tag{8.3.1}
\end{equation*}


有解的充要条件是存在 $\boldsymbol{A}$ 与 $\boldsymbol{B}$ 的广义逆矩阵 $\boldsymbol{A}^{-}$与 $\boldsymbol{B}^{-}$,使得


\begin{equation*}
A A^{-} D B^{-} B=D \tag{8.3.2}
\end{equation*}


成立。在有解的情况下,矩阵方程(8.3.1)的通解为


\begin{equation*}
X=A^{-} D B^{-}+Y-A^{-} A Y B B^{-} \tag{8.3.3}
\end{equation*}


其中 $\boldsymbol{Y}$ 为任意 $n \times s$ 矩阵。\\
证明 必要性 对于 $\boldsymbol{A}$ 与 $\boldsymbol{B}$ 都有 $\boldsymbol{A}^{-} \in C^{n \times m}, \boldsymbol{B}^{-} \in C^{t \times s}$ 满足

$$
A A^{-} A=A, B B^{-} B=B
$$

现设方程(8.3.1)有解,于是就有

$$
\begin{aligned}
D & =A X B=\left(A A^{-} A\right) X\left(B B^{-} B\right)=A A^{-}(A X B) B^{-} B \\
& =A A^{-} D B^{-} B
\end{aligned}
$$

此即式(8.3.2)。\\
充分性 设 $\boldsymbol{A}$ 与 $\boldsymbol{B}$ 有广义逆矩阵 $\boldsymbol{A}^{-}$与 $\boldsymbol{B}^{-}$,且满足式(8.3.2)现证矩阵方程 (8.3.1)有解。事实上,若命 $\boldsymbol{X}=\boldsymbol{A}^{-} \boldsymbol{D} \boldsymbol{B}^{-}$,则 $\boldsymbol{X}$ 满足 $\boldsymbol{A X B}=\boldsymbol{D}$ ,所以矩阵方程 (8.3.1)有解(且其解为 $\boldsymbol{X}=\boldsymbol{A}^{-} \boldsymbol{D} \boldsymbol{B}^{-}$)。

现在需证方程(8.3.1)的通解是式(8.3.3)。首先形如式(8.3.3)的 $\boldsymbol{X}$ 是方程 (8.3.1)的解,这只需把式(8.3.3)中的 $\boldsymbol{X}$ 代人式(8.3.1)即可(注意到式 (8.3.2), $\boldsymbol{Y}$ 是任意 $n \times s$ 矩阵)。

现设 $\boldsymbol{G} \in C^{n \times s}$ 是式(8.3.1)的任意解,则有

$$
A G B=D
$$

于是

$$
G=A^{-} D B^{-}+G-A^{-} D B^{-}=A^{-} D B^{-}+G-A^{-} A G B B^{-}
$$

这表明式(8.3.1)的任意解 $\boldsymbol{G}$ 都可写成式(8.3.3)的形式,所以式(8.3.3)是矩阵方程式(8.3.1)的通解。

在方程式(8.3.1)中,命 $\boldsymbol{B}=\boldsymbol{E}$ ,则得推论1。\\
推论1 设 $A \in C^{m \times n}, D \in C^{m \times s}$ ,则矩阵方程


\begin{equation*}
A X=D \tag{8.3.4}
\end{equation*}


有解的充要条件是存在 $\boldsymbol{A}$ 的广义逆矩阵 $\boldsymbol{A}^{-}$,使得


\begin{equation*}
A A^{-} D=D \tag{8.3.5}
\end{equation*}


成立.在有解的情况下,矩阵方程(8.3.4)的通解为


\begin{equation*}
X=A^{-} D+Y-A^{-} A Y \tag{8.3.6}
\end{equation*}


其中 $\boldsymbol{Y}$ 为任意 $n \times s$ 矩阵\\
在矩阵方程式(8.3.1)中,命 $\boldsymbol{B}=\boldsymbol{E}, \boldsymbol{D}=\boldsymbol{b} \in C^{m \times 1}$ ,使得线性方程组,则得推论2.

推论 2 设 $A \in C^{m \times n}, b \in C^{m}$ ,则线性方程组


\begin{equation*}
A X=b \tag{8.3.7}
\end{equation*}


有解的充要条件是存在 $\boldsymbol{A}$ 的广义逆矩阵 $\boldsymbol{A}^{-}$使得


\begin{equation*}
A A^{-} b=b \tag{8.3.8}
\end{equation*}


成立。在有解的情况下,线性方程组(8.3.7)的通解为


\begin{equation*}
\boldsymbol{X}=\boldsymbol{A}^{-} \boldsymbol{b}+\left(\boldsymbol{E}_{n}-\boldsymbol{A}^{-} \boldsymbol{A}\right) \boldsymbol{Y} \tag{8.3.9}
\end{equation*}


其中 $\boldsymbol{Y}$ 是任意 $n$ 维向量.

\section*{二、相容方程组 $A x=b$ 的解}
线性方程组 $\boldsymbol{A x}=\boldsymbol{b}$ 有解的充要条件是 $\operatorname{rank}(\boldsymbol{A}, \boldsymbol{b})=\operatorname{rank}(\boldsymbol{A})$ 。一般情况下,方程组的解是不唯一的,利用矩阵广义逆矩阵能写出相容(有解)方程组的通解。

定理8.3.2 设 $\boldsymbol{B}$ 是 $\boldsymbol{A} \in C_{r}^{m \times n}$ 的一个广义逆矩阵,若 $\boldsymbol{A} \boldsymbol{x}=\boldsymbol{b}$ 有解,则其通解可以表示为

$$
\boldsymbol{x}=\boldsymbol{B} \boldsymbol{b}+\left(\boldsymbol{E}_{n}-\boldsymbol{B} \boldsymbol{A}\right) \boldsymbol{z}
$$

其中 $z$ 是任意的 $n$ 维列向量。\\
因为 $\boldsymbol{A x}=\boldsymbol{b}$ 有解等价于 $\operatorname{rank}(\boldsymbol{A}, \boldsymbol{b})=\operatorname{rank}(\boldsymbol{A})$ ,即等价于 $\boldsymbol{b} \in \mathbf{R}(\boldsymbol{A})$ ,所以定理 8.3.2便是定理8.3.1的推论2.

例8.3.1 求线性方程组

$$
\left\{\begin{array}{l}
x_{1}+x_{2}-3 x_{4}-x_{5}=2 \\
x_{1}-x_{2}+2 x_{3}-x_{4}=1 \\
4 x_{1}-2 x_{2}+6 x_{3}+3 x_{4}-4 x_{5}=7 \\
2 x_{1}+4 x_{2}-2 x_{3}+4 x_{4}-7 x_{5}=1
\end{array}\right.
$$

的通解。\\
解 方程组的系数矩阵 $\boldsymbol{A}$ 与列向量 $\boldsymbol{b}$ 分别为

$$
\boldsymbol{A}=\left[\begin{array}{rrrrr}
1 & 1 & 0 & -3 & -1 \\
1 & -1 & 2 & -1 & 0 \\
4 & -2 & 6 & 3 & -4 \\
2 & 4 & -2 & 4 & -7
\end{array}\right], \boldsymbol{b}=\left[\begin{array}{l}
2 \\
1 \\
7 \\
1
\end{array}\right]
$$

用例8.1.1与8.1.2的方法求出 $\boldsymbol{A}$ 的广义逆矩阵.不难求得

$$
P=\left[\begin{array}{rrrr}
\frac{5}{18} & -\frac{1}{6} & \frac{2}{9} & 0 \\
\frac{7}{18} & -\frac{5}{6} & \frac{1}{9} & 0 \\
-\frac{1}{9} & -\frac{1}{3} & \frac{1}{9} & 0 \\
-\frac{5}{12} & \frac{5}{4} & -\frac{1}{3} & \frac{1}{4}
\end{array}\right], Q=\left[\begin{array}{ccccc}
1 & 0 & 0 & -1 & \frac{7}{6} \\
0 & 1 & 0 & 1 & \frac{5}{6} \\
0 & 0 & 0 & 1 & 0 \\
0 & 0 & 1 & 0 & \frac{1}{3} \\
0 & 0 & 0 & 0 & 1
\end{array}\right]
$$

满足

$$
P A Q=\left[\begin{array}{lllll}
1 & 0 & 0 & 0 & 0 \\
0 & 1 & 0 & 0 & 0 \\
0 & 0 & 1 & 0 & 0 \\
0 & 0 & 0 & 0 & 0
\end{array}\right]
$$

只需寻找 $\boldsymbol{A}$ 的一个广义逆矩阵,所以在公式(8.1.4)中命 $\boldsymbol{X}=0, \boldsymbol{Y}=0, \boldsymbol{Z}=0$ ,可得

$$
\begin{aligned}
\boldsymbol{B} & =\boldsymbol{Q}\left[\begin{array}{cc}
\boldsymbol{E}_{r} & 0 \\
0 & 0
\end{array}\right] \boldsymbol{P}=\left[\begin{array}{rrrr}
\frac{5}{18} & -\frac{1}{6} & \frac{2}{9} & 0 \\
\frac{7}{18} & -\frac{5}{6} & \frac{1}{9} & 0 \\
0 & 0 & 0 & 0 \\
-\frac{1}{9} & -\frac{1}{3} & \frac{1}{9} & 0 \\
0 & 0 & 0 & 0
\end{array}\right] \\
\boldsymbol{B} \boldsymbol{b} & =\left[\begin{array}{c}
\frac{35}{18} \\
\frac{13}{18} \\
0 . \\
\frac{2}{9} \\
0
\end{array}\right], \quad \boldsymbol{B} \boldsymbol{A}=\left[\begin{array}{ccccc}
1 & 0 & 1 & 0 & -\frac{7}{6} \\
0 & 1 & -1 & 0 & -\frac{5}{6} \\
0 & 0 & 0 & 0 & 0 \\
0 & 0 & 0 & 1 & -\frac{1}{3} \\
0 & 0 & 0 & 0 & 0
\end{array}\right] \\
\boldsymbol{E}_{n}-\boldsymbol{B} \boldsymbol{A} & =\left[\begin{array}{ccccc}
0 & 0 & -1 & 0 & \frac{7}{6} \\
0 & 0 & 1 & 0 & \frac{5}{6} \\
0 & 0 & 1 & 0 & 0 \\
0 & 0 & 0 & 0 & \frac{1}{3} \\
0 & 0 & 0 & 0 & 1
\end{array}\right]
\end{aligned}
$$

所以方程组的通解为

$$
\begin{aligned}
{\left[\begin{array}{l}
x_{1} \\
x_{2} \\
x_{3} \\
x_{4} \\
x_{5}
\end{array}\right] } & =\left[\begin{array}{c}
\frac{35}{18} \\
\frac{13}{18} \\
0 \\
\frac{2}{9} \\
0
\end{array}\right]+\left[\begin{array}{ccccc}
0 & 0 & -1 & 0 & \frac{7}{6} \\
0 & 0 & 1 & 0 & \frac{5}{6} \\
0 & 0 & 1 & 0 & 0 \\
0 & 0 & 0 & 0 & \frac{1}{3} \\
0 & 0 & 0 & 0 & 1
\end{array}\right]\left[\begin{array}{l}
z_{1} \\
z_{2} \\
z_{3} \\
z_{4} \\
z_{5}
\end{array}\right] \\
& =\left[\begin{array}{c}
\frac{35}{18}-z_{3}+\frac{7}{6} z_{5} \\
\frac{13}{8}+z_{3}+\frac{5}{6} z_{5} \\
z_{3} \\
\frac{2}{9}+\frac{1}{3} z_{5} \\
z_{5}
\end{array}\right]
\end{aligned}
$$

其中 $z_{3}, z_{5}$ 为任意数。\\
读者可用线性代数对增广哭阵 $(\boldsymbol{A} \mid \boldsymbol{b})$ 初等行变换方法验算本例的答案。\\
相容方程组一般情况下,解是不唯一的,在这些解中方程组的最小模解(或称最小范数解)在实际应用中是十分有用的。

定义8.3.1 称相容方程组 $\boldsymbol{A x}=\boldsymbol{b}$ 的所有解 $\boldsymbol{x}$ 中模(2-范

$$
\|x\|=\sqrt{x^{H} x}
$$

数)最小的解是 $A x=b$ 的最小模解。\\
定理8.3.3 设 $\boldsymbol{B}$ 是 $\boldsymbol{A} \in C^{m \times n}$ 的一个广义逆矩阵,则下列两个命题是等价的:\\
(1)对于任给 $\boldsymbol{b} \in R(\boldsymbol{A})$ ,则 $\boldsymbol{x}=\boldsymbol{B} \boldsymbol{b}$ 一定是 $\boldsymbol{A x}=\boldsymbol{b}$ 的最小模解.\\
(2)$(\boldsymbol{B} \boldsymbol{A})^{\mathrm{H}}=\boldsymbol{B} \boldsymbol{A}$ .\\
证明 $(1) \Rightarrow(2)$ .\\
设 $\boldsymbol{b} \in R(\boldsymbol{A}), \boldsymbol{A x}=\boldsymbol{b}$ 有解. $\boldsymbol{B}$ 是 $\boldsymbol{A}$ 的一个广义逆矩阵,由式(8.3.9)可知方程组的通解形式为

$$
\boldsymbol{x}=\boldsymbol{B} \boldsymbol{b}+\left(\boldsymbol{E}_{n}-\boldsymbol{B} \boldsymbol{A}\right) \boldsymbol{z} \quad \boldsymbol{z} \in C^{n}
$$

若 $\boldsymbol{x}=\boldsymbol{B} \boldsymbol{b}$ 是其最小模解,则

$$
\|B b\| \leqslant\left\|B b+\left(E_{n}-B A\right) z\right\|
$$

即

$$
\|\boldsymbol{B} \boldsymbol{b}\|^{2} \leqslant\left\|\boldsymbol{B} \boldsymbol{b}+\left(\boldsymbol{E}_{n}-\boldsymbol{B} \boldsymbol{A}\right) \boldsymbol{z}\right\|^{2}
$$

由于 $b \in R(A)$ ,故对于任何 $y \in C^{n}, b=A y$ ,于是

$$
\|\boldsymbol{B} \mathbf{A} \boldsymbol{y}\|^{2} \leqslant\left\|\boldsymbol{B} \mathbf{A} \boldsymbol{y}+\left(\boldsymbol{E}_{n}-\boldsymbol{B} \boldsymbol{A}\right) \boldsymbol{z}\right\|^{2}
$$

计算上述不等式得:

$$
\begin{aligned}
(\boldsymbol{B} \boldsymbol{A} \boldsymbol{y})^{\mathrm{H}}(\boldsymbol{B} \boldsymbol{A} \boldsymbol{y}) & \leqslant\left\{\boldsymbol{B} \boldsymbol{A} \boldsymbol{y}+\left(\boldsymbol{E}_{n}-\boldsymbol{B} \boldsymbol{A}\right) \boldsymbol{z}\right\}^{\mathrm{H}}\left\{\boldsymbol{B} \boldsymbol{A} \boldsymbol{y}+\left(\boldsymbol{E}_{n}-\boldsymbol{B} \boldsymbol{A}\right) \boldsymbol{z}\right\} \\
& =(\boldsymbol{B} \boldsymbol{A} \boldsymbol{y})^{\mathrm{H}}(\boldsymbol{B} \boldsymbol{A} \boldsymbol{y})+(\boldsymbol{B} \boldsymbol{A} \boldsymbol{y})^{\mathrm{H}}\left(\boldsymbol{E}_{n}-\boldsymbol{B} \boldsymbol{A}\right) \boldsymbol{z}+ \\
& {\left[\left(\boldsymbol{E}_{n}-\boldsymbol{B} \boldsymbol{A}\right) \boldsymbol{z}\right]^{\mathrm{H}} \boldsymbol{B} \boldsymbol{A} \boldsymbol{y}+\left[\left(\boldsymbol{E}_{n}-\boldsymbol{B} \boldsymbol{A}\right) \boldsymbol{z}\right]^{\mathrm{H}}\left[\left(\boldsymbol{E}_{n}-\boldsymbol{B} \boldsymbol{A}\right) \boldsymbol{z}\right] }
\end{aligned}
$$

移项得

$$
\begin{aligned}
& (\boldsymbol{B} \boldsymbol{A} \boldsymbol{y})^{\mathbf{H}}\left(\boldsymbol{E}_{n}-\boldsymbol{B} \boldsymbol{A}\right) \boldsymbol{z}+\left[\left(\boldsymbol{E}_{n}-\boldsymbol{B} \boldsymbol{A}\right) \boldsymbol{z}\right]^{\mathbf{H}} \boldsymbol{B} \boldsymbol{A} \boldsymbol{y}+ \\
& {\left[\left(\boldsymbol{E}_{n}-\boldsymbol{B} \boldsymbol{A}\right) \boldsymbol{z}\right]^{\mathbf{H}}\left[\left(\boldsymbol{E}_{n}-\boldsymbol{B} \boldsymbol{A}\right) \boldsymbol{z}\right] \geqslant 0}
\end{aligned}
$$

由于

$$
\begin{aligned}
& (\boldsymbol{B} \boldsymbol{A} \boldsymbol{y})^{\mathrm{H}}\left(\boldsymbol{E}_{n}-\boldsymbol{B} \boldsymbol{A}\right) \boldsymbol{z}+\left[\left(\boldsymbol{E}_{n}-\boldsymbol{B} \boldsymbol{A}\right) \boldsymbol{z}\right]^{\mathrm{H}} \boldsymbol{B} \boldsymbol{A} \boldsymbol{y} \\
& =2 \operatorname{Re}\left[(\boldsymbol{B} \boldsymbol{A} \boldsymbol{y})^{\mathrm{H}}\left(\boldsymbol{E}_{n}-\boldsymbol{B} \boldsymbol{A} \boldsymbol{y}\right) \boldsymbol{z}\right]
\end{aligned}
$$

所以

$$
2 \operatorname{Re}\left[(B A y)^{\mathrm{H}}\left(E_{n}-B A\right) z\right]+
$$

$$
\left[\left(\boldsymbol{E}_{n}-\boldsymbol{B} \boldsymbol{A}\right) \boldsymbol{z}\right]^{\mathbf{H}}\left[\left(\boldsymbol{E}_{n}-\boldsymbol{B} \boldsymbol{A}\right) \boldsymbol{z}\right] \geqslant 0
$$

该不等式对于任意 $y, z \in C^{n}$ 都成立,因此必须对于任意 $y, z \in C^{n}$ ,都有

$$
2 \operatorname{Re}\left[(\boldsymbol{B} \boldsymbol{A} \boldsymbol{y})^{\mathrm{H}}\left(\boldsymbol{E}_{n}-\boldsymbol{B} \boldsymbol{A}\right) \boldsymbol{z}\right]=0
$$

根据 $y, z$ 的任意性,则有

$$
\begin{gathered}
(\boldsymbol{B A})^{\mathrm{H}}\left(\boldsymbol{E}_{n}-\boldsymbol{B A}\right)=0 \\
(\boldsymbol{B A})^{\mathrm{H}}-(\boldsymbol{B A})^{\mathrm{H}}(\boldsymbol{B A})=0
\end{gathered}
$$

即

$$
(\boldsymbol{B A})^{\mathrm{H}}=(\boldsymbol{B A})^{\mathrm{H}}(\boldsymbol{B A})
$$

故

$$
\boldsymbol{B A}=\left[(\boldsymbol{B A})^{\mathrm{H}}(\boldsymbol{B A})\right]^{\mathrm{H}}=(\boldsymbol{B A})^{\mathrm{H}}(\boldsymbol{B A})=(\boldsymbol{B A})^{\mathrm{H}}
$$

这就证明了(1)$\Rightarrow$(2).\\
不难看出,(2)$\Rightarrow$(1)可以逆向推理。

\section*{三、不相容方程组 $\boldsymbol{A x}=\boldsymbol{b}$ 的解}
若 $\operatorname{rank}(A, b) \neq \operatorname{rank}(A)$ ,即 $\bar{b} \in \mathbf{R}(A)$ 时,方程组 $A x=b$ 无解.这时,称方程组是不相容的。

对于不相容的方程,也希望有方程的"解",并要求所得到的"解"是方程组的最小二乘解与最佳最小二乘解。

定义8.3.2 设 $A \in C^{m \times n}, b \in C^{n}, n$ 维向量 $x_{0}$ 满足对于任何一个 $n$ 维向量 $x$ ,都有

$$
\left\|A x_{0}-b\right\|^{2} \leqslant\|A x-b\|^{2}
$$

则称 $x_{0}$ 是方程组 $A x=b$ 的一个最小二乘解。\\
若 $u$ 是最小二乘解,如果对于任一个最小二乘解 $x_{0}$ ,都有不等式

$$
\|\boldsymbol{u}\| \leqslant\left\|\boldsymbol{x}_{0}\right\|
$$

则称 $\boldsymbol{u}$ 是最佳最小二乘解.\\
定理8.3.4 设 $A \in C^{m \times n}, B \in C^{m \times n}$ ,则下列两个命题是等价的。\\
(1)对于任何 $\boldsymbol{b} \in \boldsymbol{C}^{m \times 1}, \boldsymbol{x}=\boldsymbol{B} \boldsymbol{b}$ 一定是方程组 $\boldsymbol{A} \boldsymbol{x}=\boldsymbol{b}$ 的最小二乘解\\
(2) $\boldsymbol{A} \boldsymbol{B} \boldsymbol{A}=\boldsymbol{A},(\boldsymbol{A B})^{\mathrm{H}}=\boldsymbol{A B}$\\
该定理说明方程组 $\boldsymbol{A} \boldsymbol{x}=\boldsymbol{b}$ 的最小二乘解 $\boldsymbol{x}=\boldsymbol{B} \boldsymbol{b}$ ,其中 $\boldsymbol{B}$ 是 $\boldsymbol{A}$ 的广义逆矩阵 $\boldsymbol{A}^{-}$,且它还需满足 $(\boldsymbol{A B})^{\mathbf{H}}=\boldsymbol{A B}$ ,即 $\boldsymbol{B}$ 是满足 Penrose-Moore 方程(1)与(3),所以有的书上用记号 $\{1,3\}$ 表示。

定理8.3.5 设 $\boldsymbol{A} \in C^{m \times n}, \boldsymbol{b} \in C^{m \times 1}$ ,则 $\boldsymbol{x}=\boldsymbol{A}^{+} \boldsymbol{b}$ 是方程组 $\boldsymbol{A} \boldsymbol{x}=\boldsymbol{b}$ 的最佳最小二乘解.

\section*{习 题}
\section*{8-1 已知矩阵}
$$
A=\left[\begin{array}{ccr}
2 \mathrm{i} & \mathrm{i} & 0 \\
0 & 0 & -3 \\
2 & 1 & 1
\end{array}\right]
$$

写出形如式(8.1.4)的 $\boldsymbol{A}^{-}$通式.\\
8-2 已知矩阵

$$
\boldsymbol{A}=\left[\begin{array}{lllll}
1 & 1 & 0 & 1 & 0 \\
0 & 1 & 1 & 1 & 1 \\
1 & 0 & 1 & 1 & 0
\end{array}\right], \boldsymbol{B}=\left[\begin{array}{llll}
0 & 1 & 0 & 1 \\
0 & 1 & 0 & 1 \\
2 & 0 & 1 & 1
\end{array}\right]
$$

试分别求 $\boldsymbol{A}^{+}, \boldsymbol{B}^{+}$.\\
8-3 设 $\boldsymbol{A} \in C^{m \times n}, \boldsymbol{P}$ 与 $\boldsymbol{Q}$ 分别为 $m$ 阶与 $n$ 阶西矩阵。试证:$(\boldsymbol{P A Q})^{+}=\boldsymbol{Q}^{+} A^{+} P^{+}$.

\section*{8-4 证明下列等式}
(1)$\left(\boldsymbol{A}^{\mathrm{H}} \boldsymbol{A}\right)^{+}=\boldsymbol{A}^{+}\left(\boldsymbol{A}^{\mathrm{H}}\right)^{+} ;\left(\boldsymbol{A} \boldsymbol{A}^{\mathrm{H}}\right)^{+}=\left(\boldsymbol{A}^{\mathrm{H}}\right)^{+} \boldsymbol{A}^{+}$\\
(2)$\left(\boldsymbol{A}^{\mathrm{H}} \boldsymbol{A}\right)^{+}=\boldsymbol{A}^{+}\left(\boldsymbol{A} \boldsymbol{A}^{\mathrm{H}}\right) \boldsymbol{A}=\boldsymbol{A}^{\mathrm{H}}\left(\boldsymbol{A} \boldsymbol{A}^{\mathrm{H}}\right)+\left(\boldsymbol{A}^{\mathrm{H}}\right)^{+}$\\
(3) $\boldsymbol{A} \boldsymbol{A}^{+}=\left(\boldsymbol{A} \boldsymbol{A}^{\mathrm{H}}\right)\left(\boldsymbol{A} \boldsymbol{A}^{\mathrm{H}}\right)^{+}=\left(\boldsymbol{A} \boldsymbol{A}^{\mathrm{H}}\right)^{+}\left(\boldsymbol{A} \boldsymbol{A}^{\mathrm{H}}\right)$\\
(4) $\boldsymbol{A}^{+} \boldsymbol{A}=\left(\boldsymbol{A}^{H} \boldsymbol{A}\right)\left(\boldsymbol{A}^{\mathrm{H}} \boldsymbol{A}\right)^{+}=\left(\boldsymbol{A}^{\mathrm{H}} \boldsymbol{A}\right)^{+}\left(\boldsymbol{A}^{\mathrm{H}} \boldsymbol{A}\right)$\\
(5)如果 $\boldsymbol{A}^{\mathrm{H}}=\boldsymbol{A}$ ,那么

$$
\left(A^{2}\right)^{+}=\left(A^{+}\right)^{2}, A^{2}\left(A^{2}\right)^{+}=\left(A^{2}\right)^{+} A^{2}=A A^{+}
$$

(6)如果 $\boldsymbol{A}^{\mathrm{H}}=\boldsymbol{A}$ ,那么

$$
\boldsymbol{A} \boldsymbol{A}^{+}=\boldsymbol{A}^{+} \boldsymbol{A}
$$

\section*{第九童}
\section*{Kronecker 积}
这一章讨论含有未知矩阵的矩阵代数方程。为此先引入矩阵的 Kronecker 积的概念,并讨论它的一些基本性质,矩阵的列展开与行展开,然后介绍几个简单类型的矩阵代数方程。

\section*{§9.1 Kronecker 积的定义与性质}
在线性代数中曾定义过矩阵 $\boldsymbol{A}$ 与 $\boldsymbol{B}$ 的乘积 $\boldsymbol{A B}$ ,它要求 $\boldsymbol{A}$ 的列数必须等于 $\boldsymbol{B}$的行数,否则 $\boldsymbol{A B}$ 是没有意义的.Kronecker 积是另外意义的 $\boldsymbol{A}$ 与 $\boldsymbol{B}$ 的乘积.

定义9.1.1 设 $\boldsymbol{A}=\left(\boldsymbol{a}_{i j}\right)_{m \times n}, \boldsymbol{B}=\left(b_{i j}\right)_{p \times q}$ ,则称由

$$
\left[\begin{array}{cccc}
a_{11} B & a_{12} B & \cdots & a_{1 n} B \\
a_{21} B & a_{22} B & \cdots & a_{2 n} B \\
\vdots & \vdots & & \vdots \\
a_{m 1} B & a_{m 2} B & \cdots & a_{m n} B
\end{array}\right]
$$

所确定的 $m p \times n q$ 矩阵是 $\boldsymbol{A}$ 与 $\boldsymbol{B}$ 的 Kronecker 积或称 $\boldsymbol{A}$ 与 $\boldsymbol{B}$ 的直积记做 $\boldsymbol{A} \otimes \boldsymbol{B}$ .\\
例9.1.1 设

$$
\boldsymbol{X}=\left(x_{1}, x_{2}, x_{3}\right)^{\mathbf{T}}, \boldsymbol{Y}=\left(y_{1}, y_{2}\right)^{\mathbf{T}}
$$

则

$$
\begin{aligned}
& \boldsymbol{X} \otimes \boldsymbol{Y}=\left[\begin{array}{l}
x_{1} \boldsymbol{Y} \\
x_{2} \boldsymbol{Y} \\
x_{3} \boldsymbol{Y}
\end{array}\right]=\left(x_{1} y_{1}, x_{1} y_{2}, x_{2} y_{1}, x_{2} y_{2}, x_{3} y_{1}, x_{3} y_{2}\right)^{\mathrm{T}} \\
& \boldsymbol{Y} \otimes \boldsymbol{X}=\left[\begin{array}{l}
y_{1} \boldsymbol{X} \\
y_{2} \boldsymbol{X}
\end{array}\right]=\left(y_{1} x_{1}, y_{1} x_{2}, y_{1} x_{3}, y_{2} x_{1}, y_{2} x_{2}, y_{2} x_{3}\right)^{\mathrm{T}}
\end{aligned}
$$

显然,Kronecker 不满足交换律,即一般情况下, $\boldsymbol{X} \otimes \boldsymbol{Y} \neq \boldsymbol{Y} \otimes \boldsymbol{X}$ .\\
若 $A 、 B$ 均为对角阵

$$
\boldsymbol{A}=\left[\begin{array}{cccc}
a_{1} & & & \\
& a_{2} & & \\
& & \ddots & \\
& & & a_{m}
\end{array}\right]_{m \times m}, \boldsymbol{B}=\left[\begin{array}{llll}
b_{1} & & & \\
& b_{2} & & \\
& & \ddots & \\
& & & b_{m}
\end{array}\right]_{n \times n}
$$

则

$$
A \otimes B=\left[\begin{array}{llll}
a_{1} B & & & \\
& a_{2} B & & \\
& & \ddots & \\
& & & a_{m} B
\end{array}\right]_{m n \times m n}
$$

显然,这时 $\boldsymbol{A} \otimes \boldsymbol{B}$ 也是对角矩阵。\\
KronecKer 积运算简单性质:\\
(1)$k(\boldsymbol{A} \otimes \boldsymbol{B})=(k \boldsymbol{A}) \otimes \boldsymbol{B}=\boldsymbol{A} \otimes(k \boldsymbol{B})$\\
(2) $\boldsymbol{A} \otimes(\boldsymbol{B}+\boldsymbol{C})=\boldsymbol{A} \otimes \boldsymbol{B}+\boldsymbol{A} \otimes \boldsymbol{C}$

$$
(\boldsymbol{B}+\boldsymbol{C}) \otimes \boldsymbol{A}=\boldsymbol{B} \otimes \boldsymbol{A}+\boldsymbol{C} \otimes \boldsymbol{A}
$$

(3)$(\boldsymbol{A}+\boldsymbol{B}) \otimes(\boldsymbol{C}+\boldsymbol{D})=\boldsymbol{A} \otimes \boldsymbol{C}+\boldsymbol{A} \otimes \boldsymbol{D}+\boldsymbol{B} \otimes \boldsymbol{C}+\boldsymbol{B} \otimes \boldsymbol{D}$\\
(4) $\boldsymbol{A} \otimes(\boldsymbol{B} \otimes \boldsymbol{C})=(\boldsymbol{A} \otimes \boldsymbol{B}) \otimes \boldsymbol{C}=\boldsymbol{A} \otimes \boldsymbol{B} \otimes \boldsymbol{C}$\\
若 $\boldsymbol{A}$ 与 $\boldsymbol{B}$ 分别是 $m$ 阶单位矩阵 $\boldsymbol{E}_{m}$ 与 $n$ 阶单位矩阵 $\boldsymbol{E}_{n}$ ,则 $\boldsymbol{A} \otimes \boldsymbol{B}$ 是 $m n$ 阶单位矩阵 $\boldsymbol{E}_{m n}$ ,即

$$
\boldsymbol{E}_{m} \otimes \boldsymbol{E}_{n}=\boldsymbol{E}_{n} \otimes \boldsymbol{E}_{m}=\boldsymbol{E}_{m n}
$$

不难验证,若 $\boldsymbol{A}$ 与 $\boldsymbol{B}$ 均是上(下)三角矩阵时,则 $\boldsymbol{A} \otimes \boldsymbol{B}$ 也是上(下)三角矩阵。\\
下述几个定理是 Kronecker 积的重要性质,以后的研究中经常要用到它。\\
定理9.1.1 设 $\boldsymbol{A}=\left(\boldsymbol{a}_{i j}\right)_{m \times n}, \boldsymbol{B}=\left(b_{i j}\right)_{l \times r}, \boldsymbol{C}=\left(c_{i j}\right)_{n \times p}, \boldsymbol{D}=\left(d_{i j}\right)_{r \times s}$ 则


\begin{equation*}
(A \otimes B)(C \otimes D)=A C \otimes B D \tag{9.1.1}
\end{equation*}


证明 $(A \otimes B)(C \otimes D)$

$$
\begin{aligned}
& =\left[\begin{array}{cccc}
a_{11} B & a_{12} B & \cdots & a_{1 n} B \\
a_{21} B & a_{22} B & \cdots & a_{2 n} B \\
\vdots & \vdots & & \vdots \\
a_{m 1} B & a_{m 2} B & \cdots & a_{m n} B
\end{array}\right]\left[\begin{array}{cccc}
c_{11} D & c_{12} D & \cdots & c_{1 p} D \\
c_{21} D & c_{22} D & \cdots & c_{2 p} D \\
\vdots & \vdots & & \vdots \\
c_{n 1} D & c_{n 2} D & \cdots & c_{n p} D
\end{array}\right] \\
& =\left[\begin{array}{cccc}
\left(\sum_{k=1}^{n} a_{1 k} c_{k 1}\right) B D & \left(\sum_{k=1}^{n} a_{1 k} c_{k 2}\right) B D & \cdots & \left(\sum_{k=1}^{n} a_{1 k} c_{k p}\right) B D \\
\left(\sum_{k=1}^{n} a_{2 k} c_{k 1}\right) B D & \left(\sum_{k=1}^{n} a_{2 k} c_{k 2}\right) B D & \cdots & \left(\sum_{k=1}^{n} a_{2 k} c_{k p}\right) B D \\
\vdots & \vdots & & \vdots \\
\left(\sum_{k=1}^{n} a_{m k} c_{k 1}\right) B D & \left(\sum_{k=1}^{n} a_{m k} c_{k 2}\right) B D & \cdots & \left(\sum_{k=1}^{n} a_{m k} c_{k p}\right) B D
\end{array}\right] \\
& =A C \otimes B D
\end{aligned}
$$

推论 若 $\boldsymbol{A}=\left(a_{i j}\right)_{m \times m}, \boldsymbol{B}=\left(b_{i j}\right)_{n \times n}$ ,则

$$
\boldsymbol{A} \otimes \boldsymbol{B}=\left(\boldsymbol{A} \otimes \boldsymbol{E}_{n}\right)\left(\boldsymbol{E}_{m} \otimes \boldsymbol{B}\right)=\left(\boldsymbol{E}_{m} \otimes \boldsymbol{B}\right)\left(\boldsymbol{A} \otimes \boldsymbol{E}_{n}\right)
$$

定理9.1.2 设 $\boldsymbol{A}=\left(a_{i j}\right)_{m \times n}, \boldsymbol{B}=\left(b_{i j}\right)_{p \times q}$ ,则

$$
\begin{aligned}
& (\boldsymbol{A} \otimes \boldsymbol{B})^{\mathrm{T}}=\boldsymbol{A}^{\mathrm{T}} \otimes \boldsymbol{B}^{\mathrm{T}} \\
& (\boldsymbol{A} \otimes \boldsymbol{B})^{\mathrm{H}}=\boldsymbol{A}^{\mathrm{H}} \otimes \boldsymbol{B}^{\mathrm{H}}
\end{aligned}
$$

(证略)\\
定理9.1.3 设 $A \in C^{m \times m}, B \in C^{n \times n}$\\
(1)若 $\boldsymbol{A}, \boldsymbol{B}$ 均为对称矩阵,则 $\boldsymbol{A} \otimes \boldsymbol{B}$ 也为对称矩阵。\\
(2)若 $\boldsymbol{A}, \boldsymbol{B}$ 均为正规矩阵,则 $\boldsymbol{A} \otimes \boldsymbol{B}$ 也为正规矩阵。

\section*{证明(1)(证略)}
(2)因为 $A A^{\mathrm{H}}=A^{\mathrm{H}} A, B B^{\mathrm{H}}=B^{\mathrm{H}} B$ 则

$$
\begin{aligned}
& (A \otimes B)(A \otimes B)^{\mathrm{H}}=(A \otimes B)\left(A^{\mathrm{H}} \otimes B^{\mathrm{H}}\right) \\
& =\left(A A^{\mathrm{H}}\right) \otimes\left(B B^{\mathrm{H}}\right)=\left(A^{\mathrm{H}} A\right) \otimes\left(B^{\mathrm{H}} B\right) \\
& =\left(A^{\mathrm{H}} \otimes B^{\mathrm{H}}\right)(A \otimes B)=(A \otimes B)^{\mathrm{H}}(A \otimes B)
\end{aligned}
$$

所以 $\boldsymbol{A} \otimes \boldsymbol{B}$ 也是正规矩阵。\\
根据定理 9.1.3 可知,若 $\boldsymbol{A}$ 与 $\boldsymbol{B}$ 都是 Hermite 矩阵(或反 Hermite 矩阵或酉矩阵),则 $\boldsymbol{A} \otimes \boldsymbol{B}$ 也是 Hermite 矩阵(或反 Hermite 矩阵或西矩阵)。

定理9.1.4 设 $\boldsymbol{A}$ 与 $\boldsymbol{B}$ 分别是 $\boldsymbol{m}$ 阶与 $\boldsymbol{n}$ 阶可逆矩阵,则 $\boldsymbol{A} \otimes \boldsymbol{B}$ 也为可逆矩阵,且

$$
(A \otimes B)^{-1}=A^{-1} \otimes B^{-1}
$$

证明 因为

$$
\begin{aligned}
(\boldsymbol{A} \otimes \boldsymbol{B})\left(\boldsymbol{A}^{-1} \otimes \boldsymbol{B}^{-1}\right) & =\boldsymbol{A} \boldsymbol{A}^{-1} \otimes \boldsymbol{B} \boldsymbol{B}^{-1} \\
& =\boldsymbol{E}_{m} \otimes \boldsymbol{E}_{n}=\boldsymbol{E}_{m n}
\end{aligned}
$$

故

$$
(A \otimes B)^{-1}=A^{-1} \otimes B^{-1}
$$

定理9.1.5 设 $\boldsymbol{A}=\left(a_{i j}\right)_{m \times m}, \boldsymbol{B}=\left(b_{i j}\right)_{n \times n}$ ,则

$$
\operatorname{tr}(A \otimes B)=\operatorname{tr}(A) \cdot \operatorname{tr}(B)
$$

证明 因为

$$
A \otimes B=\left[\begin{array}{cccc}
a_{11} B & a_{12} B & \cdots & a_{1 m} B \\
a_{21} B & a_{22} B & \cdots & a_{2 m} B \\
\vdots & \vdots & & \vdots \\
a_{m 1} B & a_{m 2} B & \cdots & a_{m m} B
\end{array}\right]
$$

所以

$$
\begin{aligned}
\operatorname{tr}(\boldsymbol{A} \otimes \boldsymbol{B}) & =a_{11} \operatorname{tr} \boldsymbol{B}+a_{22} \operatorname{tr} \boldsymbol{B}+\cdots+a_{m m} \operatorname{tr} \boldsymbol{B} \\
& =\left(a_{11}+a_{22}+\cdots+a_{m m}\right) \operatorname{tr} \boldsymbol{B} \\
& =\operatorname{tr}(\boldsymbol{A}) \cdot \operatorname{tr}(\boldsymbol{B})
\end{aligned}
$$

定理 9.1.6 设 $\boldsymbol{A}=\left(\boldsymbol{a}_{i j}\right)_{m \times n}, \boldsymbol{B}=\left(b_{i j}\right)_{p \times q}$ ,则

$$
\operatorname{rank}(A \otimes B)=(\operatorname{rank} A)(\operatorname{rank} B)
$$

其中 $\operatorname{rank} \boldsymbol{A}$ 表示 $\boldsymbol{A}$ 的秩。\\
证明 设 $\boldsymbol{A}$ 与 $\boldsymbol{B}$ 的标准形式分别为 $\boldsymbol{A}_{1}, \boldsymbol{B}_{1}$ ,即

$$
M A N=A_{1} \quad P B Q=B_{1}
$$

或

$$
A=M^{-1} A_{1} N^{-1} \quad B=P^{-1} B_{1} Q^{-1}
$$

其中 $\boldsymbol{M}, \boldsymbol{N}, \boldsymbol{P}, \boldsymbol{Q}$ 分别为 $m$ 阶,$n$ 阶,$p$ 阶和 $q$ 阶为非奇异矩阵,且

$$
A_{1}=\left[\begin{array}{llllll}
1 & & & & & \\
& \ddots & & & & \\
& & 1 & & & \\
& & & 0 & & \\
& & & & \ddots & \\
& & & & & 0
\end{array}\right], B_{1}=\left[\begin{array}{llllll}
1 & & & & & \\
& \ddots & & & & \\
& & 1 & & & \\
& & & 0 & & \\
& & & & \ddots & \\
& & & & & 0
\end{array}\right]
$$

$A_{1}$ 中主对角线上 1 的个数为 $\operatorname{rank} A, B_{1}$ 中主对角线上 1 的个数为 $\operatorname{rank} B$ ,于是

$$
\begin{aligned}
A \otimes B & =\left(M^{-1} A_{1} N^{-1}\right) \otimes\left(P^{-1} B_{1} Q^{-1}\right) \\
& =\left(M^{-1} \otimes P^{-1}\right)\left(A_{1} \otimes B_{1}\right)\left(N^{-1} \otimes Q^{-1}\right) \\
& =(M \otimes P)^{-1}\left(A_{1} \otimes B_{1}\right)(N \otimes Q)^{-1}
\end{aligned}
$$

故

$$
\operatorname{rank}(A \otimes B)=\operatorname{rank}\left(A_{1} \otimes B_{1}\right)=(\operatorname{rank} A)(\operatorname{rank} B)
$$

定理9.1.7 设 $x_{1}, x_{2}, \cdots, x_{n}$ 是 $n$ 个线性无关的 $m$ 维列向量,$y_{1}, y_{2}, \cdots, y_{q}$ 是 $q$ 个线性无关的 $p$ 维列向量,则 $n q$ 个 $m p$ 维列向量

$$
\boldsymbol{x}_{i} \otimes \boldsymbol{y}_{j} \quad(i=1,2, \cdots, n ; j=1,2, \cdots, q)
$$

线性无关。反之,若向量组 $x_{i} \otimes y_{j}$ 线性无关,则 $x_{1}, x_{2}, \cdots, x_{n}$ 和 $y_{1}, y_{2}, \cdots, y_{q}$ 均线性无关。

证明 设

$$
\begin{array}{r}
\boldsymbol{A}=\left(x_{1}, x_{2}, \cdots, x_{n}\right) \in C^{m \times n} \\
\boldsymbol{B}=\left(y_{1}, y_{2}, \cdots, y_{q}\right) \in C^{p \times q}
\end{array}
$$

显然 $\quad \operatorname{rank} \boldsymbol{A}=n, \quad \operatorname{rank} \boldsymbol{B}=q$\\
因为

$$
\begin{aligned}
& \boldsymbol{A} \otimes \boldsymbol{B}=\left(x_{1} \otimes y_{1}, x_{1} \otimes y_{2}, \cdots, x_{1} \otimes y_{q}, \cdots,\right. \\
& \left.x_{n} \otimes y_{1}, x_{n} \otimes y_{2}, \cdots, x_{n} \otimes y_{q}\right)
\end{aligned}
$$

所以

$$
\operatorname{rank}(\boldsymbol{A} \otimes \boldsymbol{B})=\operatorname{rank} \boldsymbol{A} \cdot \operatorname{rank} \boldsymbol{B}=n q
$$

由于 $\boldsymbol{A} \otimes \boldsymbol{B}$ 是 $m p \times n q$ 矩阵,它们的秩为 $n q$ ,因此 $\boldsymbol{A} \otimes \boldsymbol{B}$ 的 $n q$ 个列向量, $\boldsymbol{x}_{i} \otimes y_{i}(i= 1,2, \cdots, n ; j=1,2, \cdots, n)$ 是线性无关的.

反之,若列向量组 $\boldsymbol{x}_{i} \otimes \boldsymbol{y}_{j}$ 是线性无关的,则 $\boldsymbol{A} \otimes \boldsymbol{B}$ 的列向量组是线性无关的. $\boldsymbol{A} \otimes \boldsymbol{B}$ 的秩为 $n q$ ,即

$$
n q=\operatorname{rank}(A \otimes B)=\operatorname{rank} A \cdot \operatorname{rank} B
$$

因此,必有 $\operatorname{rank} \boldsymbol{A}=n, \operatorname{rank} \boldsymbol{B}=q$ 。此即 $\boldsymbol{A}$ 的列向量组 $\boldsymbol{x}_{1}, \boldsymbol{x}_{2}, \cdots, \boldsymbol{x}_{n}$ 线性无关的. $\boldsymbol{B}$的列向量组 $y_{1}, y_{2}, \cdots, y_{q}$ 线性无关。

定理9.1.8 设 $\boldsymbol{A}$ 的 $m$ 阶矩阵, $\boldsymbol{B}$ 为 $p$ 阶矩阵,则

$$
|\boldsymbol{A} \otimes \boldsymbol{B}|=|\boldsymbol{A}|^{p}|\boldsymbol{B}|^{m}
$$

证明 设 $\boldsymbol{A}$ 与 $\boldsymbol{B}$ 的 Jordan 标准形式为 $\boldsymbol{J}_{1}$ 和 $\boldsymbol{J}_{2}$ ,于是存在 $\boldsymbol{m}$ 阶非奇异矩阵 $\boldsymbol{P}$与 $p$ 阶非奇异矩阵 $\boldsymbol{Q}$ ,满足

$$
P^{-1} A P=J_{1}, Q^{-1} B Q=J_{2}
$$

所以有

$$
\begin{aligned}
A \otimes B & =\left(P J_{1} P^{-1}\right) \otimes\left(Q J_{2} Q^{-1}\right) \\
& =(P \otimes Q)\left(J_{1} \otimes J_{2}\right)\left(P^{-1} \otimes Q^{-1}\right) \\
& =(P \otimes Q)\left(J_{1} \otimes J_{2}\right)(P \otimes Q)^{-1}
\end{aligned}
$$

因为 Jordan 标准形 $\boldsymbol{J}_{1}$ 与 $\boldsymbol{J}_{2}$ 都是上三角矩阵,其主对角线上元素分别是 $\boldsymbol{A}$ 与 $\boldsymbol{B}$ 的特征值 $\lambda_{i}$ 与 $\mu_{i}$ ,故有

$$
\begin{aligned}
|\boldsymbol{A} \otimes \boldsymbol{B}| & =\left|\boldsymbol{J}_{1} \otimes \boldsymbol{J}_{2}\right| \\
& =\prod_{j=1}^{p}\left(\lambda_{1} \mu_{j}\right) \cdot \prod_{j=1}^{p}\left(\lambda_{2} \mu_{j}\right) \cdots \prod_{j=1}^{p}\left(\lambda_{m} \mu_{j}\right) \\
& =\left(\lambda_{1}^{p} \lambda_{2}^{p} \cdots \lambda_{m}^{p}\right)\left(\prod_{j=1}^{p}\left(\mu_{j}\right)\right)^{m} \\
& =|\boldsymbol{A}|^{p}|\boldsymbol{B}|^{m}
\end{aligned}
$$

定理9.1.9 设 $\boldsymbol{A}$ 为 $m$ 阶矩阵, $\boldsymbol{B}$ 为 $n$ 阶矩阵,则存在一个 $m n$ 阶置换矩阵 (有限个初等矩阵的乘积) $\boldsymbol{P}$ ,使得

$$
\boldsymbol{P}^{\mathrm{T}}(\boldsymbol{A} \otimes \boldsymbol{B}) \boldsymbol{P}=\boldsymbol{B} \otimes \boldsymbol{A}
$$

注 对矩阵的 $i$ 行和相应的 $i$ 列施行相同的初等变换在线性代数中称为合同变换.不难验证,若对矩阵 $\boldsymbol{A}$ 作一个合同变换, $\boldsymbol{P}$ 是该初等变换相应的初等矩阵,对 $\boldsymbol{A}$ 施行合同变换以后所得矩阵为 $\boldsymbol{B}$ ,则

$$
\boldsymbol{P}^{\mathrm{T}} \boldsymbol{A} \boldsymbol{P}=\boldsymbol{B}
$$

证明 不难验证,对矩阵 $\boldsymbol{A} \otimes \boldsymbol{E}_{n}$ 施行一系列合同变换可以变成 $\boldsymbol{E}_{n} \otimes \boldsymbol{A}$ ,此即存在一个 $m n$ 阶置换矩阵 $\boldsymbol{P}$ ,使得

$$
\boldsymbol{P}^{\mathrm{T}}\left(\boldsymbol{A} \otimes \boldsymbol{E}_{n}\right) \boldsymbol{P}=\boldsymbol{E}_{n} \otimes \boldsymbol{A}
$$

不难验证,对于这个 $m n$ 阶置换矩阵 $\boldsymbol{P}$ ,还有

$$
\boldsymbol{P}^{\mathrm{T}}\left(\boldsymbol{E}_{m} \otimes \boldsymbol{B}\right) \boldsymbol{P}=\boldsymbol{B} \otimes \boldsymbol{E}_{m}
$$

因为 $\boldsymbol{P}$ 是正交矩阵,因此

$$
\begin{aligned}
\boldsymbol{P}^{\mathrm{T}}(\boldsymbol{A} \otimes \boldsymbol{B}) \boldsymbol{P} & =\boldsymbol{P}^{\mathrm{T}}\left(\boldsymbol{A} \otimes \boldsymbol{E}_{n}\right)\left(\boldsymbol{E}_{m} \otimes \boldsymbol{B}\right) \boldsymbol{P} \\
& =\boldsymbol{P}^{\mathrm{T}}\left(\boldsymbol{A} \otimes \boldsymbol{E}_{n}\right) \boldsymbol{P} \boldsymbol{P}^{\mathrm{T}}\left(\boldsymbol{E}_{m} \otimes \boldsymbol{B}\right) \boldsymbol{P}
\end{aligned}
$$

$$
\begin{aligned}
& =\left(\boldsymbol{E}_{n} \otimes \boldsymbol{A}\right)\left(\boldsymbol{B} \otimes \boldsymbol{E}_{m}\right) \\
& =\boldsymbol{B} \otimes \boldsymbol{A}
\end{aligned}
$$

由于定理9.1.9中的 $\boldsymbol{P}$ 是正交矩阵,故 $\boldsymbol{P}^{\mathrm{T}}=\boldsymbol{P}^{-1}$ 于是可得\\
定理 9.1.10 $\boldsymbol{A} \otimes \boldsymbol{B} \sim \boldsymbol{B} \otimes \boldsymbol{A}$ .\\
对 Kronecker 积也有幂的概念,记

$$
\boldsymbol{A}^{[k]}=\underbrace{\boldsymbol{A} \otimes \boldsymbol{A} \otimes \cdots \otimes \boldsymbol{A}}_{k \uparrow \boldsymbol{A}}
$$

关于 Kronecker 积的幂,有下面的定理。\\
定理9.1.11 设 $\boldsymbol{A} \in C^{m \times n}, \boldsymbol{B} \in C^{n \times p}$ ,则

$$
\left(\boldsymbol{A} \boldsymbol{B}^{[k]}=\boldsymbol{A}^{[k]} \boldsymbol{B}^{[k]}\right)
$$

\section*{§9.2 函数矩阵对矩阵的导数}
设 $\boldsymbol{A}=\left(a_{i j}\right)_{m \times n}, \boldsymbol{B}=\left(b_{k l}\right)_{p \times q}, \boldsymbol{A}$ 中的每一个元素 $a_{i j}$ 是 $\boldsymbol{B}$ 中 $p q$ 个元素 $b_{k l}$ 的函数,这时称 $\boldsymbol{A}$ 是 $\boldsymbol{B}$ 的函数,用 $\boldsymbol{A}(\boldsymbol{B})$ 表示.

定义9.2.1 设 $\boldsymbol{A}=\left(a_{i j}\right)_{m \times n}, \boldsymbol{B}=\left(b_{k l}\right)_{p \times q}$ ,若 $\boldsymbol{A}$ 是 $\boldsymbol{B}$ 的函数,且 $a_{i j}$ 对所有元素 $b_{k l}$ 可以求偏导数,则函数矩阵 $\boldsymbol{A}$ 对矩阵 $\boldsymbol{B}$ 可以求导数,用 $\frac{\mathrm{D} \boldsymbol{A}}{\mathrm{D} \boldsymbol{B}}$ 表示,且

\[
\frac{\mathrm{D} \boldsymbol{A}}{\mathrm{D} \boldsymbol{B}}=\left[\begin{array}{cccc}
\frac{\partial \boldsymbol{A}}{\partial b_{11}} & \frac{\partial \boldsymbol{A}}{\partial b_{12}} & \cdots & \frac{\partial \boldsymbol{A}}{\partial b_{1 q}}  \tag{9.2.1}\\
\frac{\partial \boldsymbol{A}}{\partial b_{21}} & \frac{\partial \boldsymbol{A}}{\partial b_{22}} & \cdots & \frac{\partial \boldsymbol{A}}{\partial b_{2 q}} \\
\vdots & \vdots & & \vdots \\
\frac{\partial \boldsymbol{A}}{\partial b_{p 1}} & \frac{\partial \boldsymbol{A}}{\partial b_{p 2}} & \cdots & \frac{\partial \boldsymbol{A}}{\partial b_{p q}}
\end{array}\right]_{m p \times n q}=\left(\frac{\partial \boldsymbol{A}}{\partial b_{k l}}\right)
\]

其中(请读者自己验证)

\[
\frac{\partial \boldsymbol{A}}{\partial b_{k l}}=\left[\begin{array}{cccc}
\frac{\partial a_{11}}{\partial b_{k l}} & \frac{\partial a_{12}}{\partial b_{k l}} & \cdots & \frac{\partial a_{1 n}}{\partial b_{k l}}  \tag{9.2.2}\\
\frac{\partial a_{21}}{\partial b_{k l}} & \frac{\partial a_{22}}{\partial b_{k l}} & \cdots & \frac{\partial a_{2 n}}{\partial b_{k l}} \\
\vdots & \vdots & & \vdots \\
\frac{\partial a_{m 1}}{\partial b_{k l}} & \frac{\partial a_{m 2}}{\partial b_{k l}} & \cdots & \frac{\partial a_{m n}}{\partial b_{k l}}
\end{array}\right]_{m \times n}
\]

且称 $\frac{\mathrm{D} A}{\mathrm{D} B}$ 是矩阵 $A$ 对矩阵 $B$ 的导数.

$$
\text { 显然 } \frac{\mathrm{D} \boldsymbol{A}}{\mathrm{D} \boldsymbol{A}^{\mathrm{T}}}=\frac{\mathrm{D} \boldsymbol{A}^{\mathrm{T}}}{\mathrm{D} \boldsymbol{A}}=\boldsymbol{E}
$$

不难验证,$\frac{\mathrm{D} \boldsymbol{A}}{\mathrm{D} \boldsymbol{B}}$ 有下列性质:\\
(1)设矩阵 $\boldsymbol{A}$ 与矩阵 $\boldsymbol{B}$ 对矩阵 $\boldsymbol{C}$ 都可求导数,则


\begin{equation*}
\frac{\mathrm{D}(\boldsymbol{A}+\boldsymbol{B})}{\mathrm{D} \boldsymbol{C}}=\frac{\mathrm{D} \boldsymbol{A}}{\mathrm{D} \boldsymbol{C}}+\frac{\mathrm{D} \boldsymbol{B}}{\mathrm{D} \boldsymbol{C}} \tag{9.2.3}
\end{equation*}


(2)设 $\boldsymbol{A}=\left(a_{i j}\right)_{m \times n}, \boldsymbol{B}=\left(b_{i j}\right)_{r \times n}, \boldsymbol{C}=\left(c_{i j}\right)_{p \times q}$ ,且矩阵\\
$\boldsymbol{A}$ 与矩阵 $\boldsymbol{B}$ 对矩阵 $\boldsymbol{C}$ 都可求导数,则\\
(a)$\frac{\mathrm{D}(\boldsymbol{A} \boldsymbol{B})}{\mathrm{D} \boldsymbol{C}}=\frac{\mathrm{D} \boldsymbol{A}}{\mathrm{D} \boldsymbol{C}}\left(\boldsymbol{E}_{q} \otimes \boldsymbol{B}\right)+\left(\boldsymbol{E}_{p} \otimes \boldsymbol{A}\right) \frac{\mathrm{D} \boldsymbol{B}}{\mathrm{D} \boldsymbol{C}}$\\
(b)$\frac{\mathrm{D}(\boldsymbol{A} \otimes \boldsymbol{B})}{\mathrm{D} \boldsymbol{C}}=\frac{\mathrm{D} \boldsymbol{A}}{\mathrm{D} \boldsymbol{C}} \otimes \boldsymbol{B}+\left(\boldsymbol{A} \otimes \frac{\partial \boldsymbol{B}}{\partial c_{i j}}\right)$

其中

\[
\left(\boldsymbol{A} \otimes \frac{\partial \boldsymbol{B}}{\partial c_{i j}}\right)=\left[\begin{array}{cccc}
\boldsymbol{A} \otimes \frac{\partial \boldsymbol{B}}{\partial c_{11}} & \boldsymbol{A} \otimes \frac{\partial \boldsymbol{B}}{\partial c_{12}} & \cdots & \boldsymbol{A} \otimes \frac{\partial \boldsymbol{B}}{\partial c_{1 q}}  \tag{9.2.6}\\
\boldsymbol{A} \otimes \frac{\partial \boldsymbol{B}}{\partial c_{21}} & \boldsymbol{A} \otimes \frac{\partial \boldsymbol{B}}{\partial c_{22}} & \cdots & \boldsymbol{A} \otimes \frac{\partial \boldsymbol{B}}{\partial c_{2 q}} \\
\vdots & \vdots & & \vdots \\
\boldsymbol{A} \otimes \frac{\partial \boldsymbol{B}}{\partial c_{p 1}} & \boldsymbol{A} \otimes \frac{\partial \boldsymbol{B}}{\partial c_{p 2}} & \cdots & \boldsymbol{A} \otimes \frac{\partial \boldsymbol{B}}{\partial c_{p q}}
\end{array}\right]
\]

证明(a)因为

\[
\frac{\mathrm{D}(\boldsymbol{A B})}{\mathrm{DC}}=\left[\begin{array}{cccc}
\frac{\partial(\boldsymbol{A B})}{\partial c_{11}} & \frac{\partial(\boldsymbol{A B})}{\partial c_{12}} & \cdots & \frac{\partial(\boldsymbol{A B})}{\partial c_{1 q}}  \tag{9.2.7}\\
\frac{\partial(\boldsymbol{A B})}{\partial c_{21}} & \frac{\partial(\boldsymbol{A B})}{\partial c_{22}} & \cdots & \frac{\partial(\boldsymbol{A B})}{\partial c_{2 q}} \\
\vdots & \vdots & & \vdots \\
\frac{\partial(\boldsymbol{A B})}{\partial c_{p 1}} & \frac{\partial(\boldsymbol{A B})}{\partial c_{p 2}} & \cdots & \frac{\partial(\boldsymbol{A B})}{\partial c_{p q}}
\end{array}\right]
\]

其中 $\quad \frac{\partial(\boldsymbol{A} \boldsymbol{B})}{\partial c_{i j}}=\frac{\partial \boldsymbol{A}}{\partial c_{i j}} \boldsymbol{B}+\boldsymbol{A} \frac{\partial \boldsymbol{B}}{\partial c_{i j}} \quad(i=1,2, \cdots, p, j=1,2, \cdots, q)$

所以

$$
\frac{\mathrm{D}(\boldsymbol{A} \boldsymbol{B})}{\mathrm{D} \boldsymbol{C}}=\left[\begin{array}{cccc}
\frac{\partial \boldsymbol{A}}{\partial c_{11}} \boldsymbol{B} & \frac{\partial \boldsymbol{A}}{\partial c_{12}} \boldsymbol{B} & \cdots & \frac{\partial \boldsymbol{A}}{\partial c_{1 q}} \boldsymbol{B} \\
\frac{\partial \boldsymbol{A}}{\partial c_{21}} \boldsymbol{B} & \frac{\partial \boldsymbol{A}}{\partial c_{22}} \boldsymbol{B} & \cdots & \frac{\partial \boldsymbol{A}}{\partial c_{2 q}} \boldsymbol{B} \\
\vdots & \vdots & & \vdots \\
\frac{\partial \boldsymbol{A}}{\partial c_{p 1}} \boldsymbol{B} & \frac{\partial \boldsymbol{A}}{\partial c_{p 2}} \boldsymbol{B} & \cdots & \frac{\partial \boldsymbol{A}}{\partial c_{p q}} \boldsymbol{B}
\end{array}\right]
$$


\begin{align*}
& +\left[\begin{array}{cccc}
\boldsymbol{A} \frac{\partial \boldsymbol{B}}{\partial c_{11}} & \boldsymbol{A} \frac{\partial \boldsymbol{B}}{\partial c_{12}} & \cdots & \boldsymbol{A} \frac{\partial \boldsymbol{B}}{\partial c_{1 q}} \\
\boldsymbol{A} \frac{\partial \boldsymbol{B}}{\partial c_{21}} & \boldsymbol{A} \frac{\partial \boldsymbol{B}}{\partial c_{22}} & \cdots & \boldsymbol{A} \frac{\partial \boldsymbol{B}}{\partial c_{2 q}} \\
\vdots & \vdots & & \vdots \\
\boldsymbol{A} \frac{\partial \boldsymbol{B}}{\partial c_{p 1}} & \boldsymbol{A} \frac{\partial \boldsymbol{B}}{\partial c_{p 2}} & \cdots & \boldsymbol{A} \frac{\partial \boldsymbol{B}}{\partial c_{p q}}
\end{array}\right] \\
& =\left[\begin{array}{cccc}
\frac{\partial \boldsymbol{A}}{\partial c_{11}} & \frac{\partial \boldsymbol{A}}{\partial c_{12}} & \cdots & \frac{\partial \boldsymbol{A}}{\partial c_{1 q}} \\
\frac{\partial \boldsymbol{A}}{\partial c_{21}} & \frac{\partial \boldsymbol{A}}{\partial c_{22}} & \cdots & \frac{\partial \boldsymbol{A}}{\partial c_{2 q}} \\
\vdots & \vdots & & \vdots \\
\frac{\partial \boldsymbol{A}}{\partial c_{p 1}} & \frac{\partial \boldsymbol{A}}{\partial c_{p 2}} & \cdots & \frac{\partial \boldsymbol{A}}{\partial c_{p q}}
\end{array}\right]\left[\begin{array}{cccc}
\boldsymbol{B} & & \\
\boldsymbol{B} & & \\
& & & \\
& & & \\
\frac{\partial \boldsymbol{B}}{\partial c_{11}} & \frac{\partial \boldsymbol{B}}{\partial c_{12}} & \cdots & \frac{\partial \boldsymbol{B}}{\partial c_{1 q}} \\
\frac{\partial \boldsymbol{B}}{\partial c_{21}} & \frac{\partial \boldsymbol{B}}{\partial c_{22}} & \cdots & \frac{\partial \boldsymbol{B}}{\partial c_{2 q}} \\
\vdots & \vdots & & \vdots \\
\frac{\partial \boldsymbol{B}}{\partial c_{p 1}} & \frac{\partial \boldsymbol{B}}{\partial c_{p 2}} & \cdots & \frac{\partial \boldsymbol{B}}{\partial c_{p q}}
\end{array}\right] \\
& +\left[\begin{array}{ccccc}
\boldsymbol{A} & & & \\
\boldsymbol{A} & & & \\
=\frac{\mathbf{D} \boldsymbol{A}}{\mathbf{D} \boldsymbol{C}}\left(\boldsymbol{E}_{q} \otimes \boldsymbol{B}\right)+\left(\boldsymbol{\boldsymbol { E } _ { p } \otimes \boldsymbol { A } ) \frac { \mathrm { D } \boldsymbol { B } } { \mathrm { D } \boldsymbol { C } }}\right. & &
\end{array}\right.
\end{align*}


(b)注意到 $\frac{\partial(\boldsymbol{A} \otimes \boldsymbol{B})}{\partial c_{i j}}=\frac{\partial \boldsymbol{A}}{\partial c_{i j}} \otimes \boldsymbol{B}+\boldsymbol{A} \otimes \frac{\partial \boldsymbol{B}}{\partial c_{i j}}$

请读者自己验证式(9.2.5)。\\
推论 1 若 $A$ 是常数矩阵时,有 $\frac{\mathrm{D} A}{\mathrm{D} C}=0$ ,则

$$
\frac{\mathrm{D}(\boldsymbol{A} \boldsymbol{B})}{\mathrm{D} \boldsymbol{C}}=\left(\boldsymbol{E}_{p} \otimes \boldsymbol{A}\right) \frac{\mathrm{D} \boldsymbol{B}}{\mathrm{D} \boldsymbol{C}}
$$

若 $\boldsymbol{B}$ 是常数矩阵时,有 $\frac{\mathrm{D} \boldsymbol{B}}{\mathrm{D} \boldsymbol{C}}=0$ ,则

$$
\frac{\mathrm{D}(\boldsymbol{A} \boldsymbol{B})}{\mathrm{D} \boldsymbol{C}}=\frac{\mathrm{D} \boldsymbol{A}}{\mathrm{D} \boldsymbol{C}}\left(\boldsymbol{E}_{q} \otimes \boldsymbol{B}\right)
$$

推论2 若 $\boldsymbol{A}$ 是常数时, $\boldsymbol{B}$ 对 $\boldsymbol{C}$ 可求导数,则

$$
\frac{\mathrm{D}(k \boldsymbol{B})}{\mathrm{D} \boldsymbol{C}}=k \frac{\mathrm{D} \boldsymbol{B}}{\mathrm{D} \boldsymbol{C}}
$$

推论3 若 $\boldsymbol{A}, \boldsymbol{B}$ 都是一阶矩阵时,则

$$
\begin{gathered}
\frac{\mathrm{D}(A B)}{\mathrm{D} C}=\frac{\mathrm{D} A}{\mathrm{D} C} B+A \frac{\mathrm{D} B}{\mathrm{D} C} \\
\frac{\mathrm{D}(A \otimes B)}{\mathrm{D} C}=\frac{\mathrm{D} A}{\mathrm{D} C} B+A \frac{\mathrm{D} B}{\mathrm{D} C}
\end{gathered}
$$

推论4 若 $\boldsymbol{A}=\left(a_{1}, a_{2} \cdots a_{m}\right)^{\mathrm{T}}, \boldsymbol{B}=\left(b_{1}, b_{2} \cdots b_{m}\right)^{\mathrm{T}}$ ,


\begin{align*}
\boldsymbol{C}=\left(c_{1}, c_{2} \cdots, c_{p}\right)^{\mathrm{T}} . \\
\frac{\mathrm{D}\left(\boldsymbol{A}^{\mathrm{T}} \boldsymbol{B}\right)}{D \boldsymbol{C}}=\frac{\mathrm{D} \boldsymbol{A}^{\mathrm{T}}}{\mathrm{D} \boldsymbol{C}} \boldsymbol{B}+\frac{\mathrm{D} \boldsymbol{B}^{\mathrm{T}}}{\mathrm{D} \boldsymbol{C}} \boldsymbol{A}  \tag{9.2.9}\\
\frac{\mathrm{D}\left(\boldsymbol{A}^{\mathrm{T}} \boldsymbol{B}\right)}{\mathrm{D} \boldsymbol{C}^{\mathrm{T}}}=\boldsymbol{B}^{\mathrm{T}} \frac{\mathrm{D} \boldsymbol{A}}{\mathrm{D} \boldsymbol{C}^{\mathrm{T}}}+\boldsymbol{A}^{\mathrm{T}} \frac{\mathrm{D} \boldsymbol{B}}{\mathrm{D} \boldsymbol{C}^{\mathrm{T}}} \tag{9.2.10}
\end{align*}


(3)设矩阵 $\boldsymbol{A}$ 对矩阵 $\boldsymbol{B}$ 可求导数,则

$$
\left(\frac{\mathbf{D} \boldsymbol{A}}{\mathbf{D} \boldsymbol{B}}\right)^{\mathrm{T}}=\frac{\mathbf{D} \boldsymbol{A}^{\mathrm{T}}}{\mathbf{D} \boldsymbol{B}^{\mathrm{T}}},\left(\frac{\mathbf{D} \boldsymbol{A}}{\mathbf{D} \boldsymbol{B}}\right)^{\mathrm{H}}=\frac{\mathbf{D} \boldsymbol{A}^{\mathrm{H}}}{\mathbf{D} \boldsymbol{B}^{\mathrm{H}}}
$$

证明 设 $\boldsymbol{A}=\left(a_{i j}\right)_{m \times n}, \boldsymbol{B}=\left(b_{i j}\right)_{p \times q}$ .

$$
\begin{aligned}
\left(\frac{\mathrm{D} \boldsymbol{A}}{\mathrm{D} \boldsymbol{B}}\right)^{\mathrm{T}} & =\left[\begin{array}{cccc}
\frac{\partial \boldsymbol{A}}{\partial b_{11}} & \frac{\partial \boldsymbol{A}}{\partial b_{12}} & \cdots & \frac{\partial \boldsymbol{A}}{\partial b_{1 q}} \\
\frac{\partial \boldsymbol{A}}{\partial b_{21}} & \frac{\partial \boldsymbol{A}}{\partial b_{22}} & \cdots & \frac{\partial \boldsymbol{A}}{\partial b_{2 q}} \\
\vdots & \vdots & & \vdots \\
\frac{\partial \boldsymbol{A}}{\partial b_{p 1}} & \frac{\partial \boldsymbol{A}}{\partial b_{p 2}} & \cdots & \frac{\partial \boldsymbol{A}}{\partial b_{p q}}
\end{array}\right]^{\mathrm{T}} \\
& =\left[\begin{array}{cccc}
\frac{\partial \boldsymbol{A}^{\mathrm{T}}}{\partial b_{11}} & \frac{\partial \boldsymbol{A}^{\mathrm{T}}}{\partial b_{12}} & \cdots & \frac{\partial \boldsymbol{A}^{\mathrm{T}}}{\partial b_{1 q}} \\
\frac{\partial \boldsymbol{A}^{\mathrm{T}}}{\partial b_{21}} & \frac{\partial \boldsymbol{A}^{\mathrm{T}}}{\partial b_{22}} & \cdots & \frac{\partial \boldsymbol{A}^{\mathrm{T}}}{\partial b_{2 q}} \\
\vdots & \vdots & & \vdots \\
\frac{\partial \boldsymbol{A}^{\mathrm{T}}}{\partial b_{p 1}} & \frac{\partial \boldsymbol{A}^{\mathrm{T}}}{\partial b_{p 2}} & \cdots & \frac{\partial \boldsymbol{A}^{\mathrm{T}}}{\partial b_{p q}}
\end{array}\right]^{\mathrm{D}}=\frac{\mathrm{D} \boldsymbol{A}^{\mathrm{T}}}{\mathrm{D} \boldsymbol{B}^{\mathrm{T}}}
\end{aligned}
$$

类似可证 $\left(\frac{\mathrm{D} \boldsymbol{A}}{\mathrm{D} \boldsymbol{B}}\right)^{\mathrm{H}}=\frac{\mathrm{D} \boldsymbol{A}^{\mathrm{H}}}{\mathrm{D} \boldsymbol{B}^{\mathrm{H}}}$ .\\
例 9.2 .1 设纯量函数 $\boldsymbol{f}$ 是 $1 \times 3$ 矩阵 $\boldsymbol{X}=(x, y, z)$ 的函数,则 1 阶矩阵 $\boldsymbol{f}$ 关于 $\boldsymbol{X}$ 的导数 $\frac{\mathrm{D} f}{\mathrm{D} \boldsymbol{x}}$ 便是 $\boldsymbol{f}$ 的梯度.即

$$
\operatorname{grad} f=\frac{\mathrm{D} f}{\mathrm{D} X}=\left(\frac{\partial f}{\partial x}, \frac{\partial f}{\partial y}, \frac{\partial f}{\partial z}\right)
$$

例9.2.2 设 $X=\left(x_{1}, x_{2} \cdots, x_{n}\right)^{\mathrm{T}}$ 为向量变量,$\alpha=\left(a_{1}, a_{2}, \cdots, a_{n}\right)^{\mathrm{T}}$ 为给定的向量.$n$ 元函数

$$
f(\boldsymbol{X})=\boldsymbol{\alpha}^{\mathrm{T}} \boldsymbol{X}=a_{1} x_{1}+a_{2} x_{2}+\cdots+a_{n} x_{n}
$$

求 $\frac{\mathrm{D} \boldsymbol{f}}{\mathrm{D} \boldsymbol{X}}$ 与 $\frac{\mathrm{D} \boldsymbol{f}}{\mathrm{D} \boldsymbol{X}^{\mathrm{T}}}$ .\\
解 $\quad \frac{\partial f}{\partial x_{i}}=a_{i} \quad(i=1,2 \cdots, n)$\\
于是

$$
\begin{aligned}
& \frac{\mathrm{D} \boldsymbol{f}}{\mathrm{D} \boldsymbol{X}}=\left(\frac{\partial f}{\partial x_{1}}, \frac{\partial f}{\partial x_{2}}, \cdots, \frac{\partial f}{\partial x_{n}}\right)^{\mathrm{T}}=\left(a_{1}, a_{2}, \cdots, a_{n}\right)^{\mathrm{T}}=\boldsymbol{\alpha} \\
& \frac{\mathrm{D} \boldsymbol{f}}{\mathrm{DX}^{\mathrm{T}}}=\left(\frac{\mathrm{D} \boldsymbol{f}}{\mathrm{DX}}\right)^{\mathrm{T}}=\boldsymbol{\alpha}^{\mathrm{T}}
\end{aligned}
$$

例9.2.3 设 $\boldsymbol{X}=\left(x_{1}, x_{2} \cdots x_{n}\right)^{\mathrm{T}}$ 为向量变量, $\boldsymbol{A}=\left(a_{i j}\right)_{n \times n}$ 为给定的 $n$ 阶常数矩阵,$n$ 元函数

$$
f(X)=X^{\mathrm{T}} A X=\sum_{i, j=1}^{n} a_{i j} x_{i} x_{j}
$$

求 $\frac{\mathrm{D} f}{\mathrm{D} x}, \frac{\mathrm{D} f}{\mathrm{D} \boldsymbol{X}^{\mathrm{T}}}$ .\\
解 $f=\sum_{i, j}^{n} a_{i j} x_{i} x_{j}$

$$
=x_{1} \sum_{j=1}^{n} a_{1 j} x_{j}+x_{2} \sum_{j=1}^{n} a_{2 j} x_{j}+\cdots+x_{n} \sum_{j=1}^{n} a_{n j} x_{j} .
$$

于是

$$
\begin{aligned}
\frac{\partial f}{\partial x_{i}} & =x_{1} a_{1 i}+x_{2} a_{2 i}+\cdots+x_{i-1} a_{i-1 i}+\left(\sum_{j=1}^{n} a_{i j} x_{j}+x_{i} a_{i i}\right) \\
& +x_{i+1} a_{i+1 i}+\cdots+x_{n} a_{n i} \\
& =\sum_{j=1}^{n} a_{j i} x_{j}+\sum_{j=1}^{n} a_{i j} x_{j}
\end{aligned}
$$

因为

$$
\begin{aligned}
\frac{\mathrm{D} f}{\mathrm{D} X} & =\left(\frac{\mathrm{D} f}{\mathrm{D} x_{1}}, \frac{\mathrm{D} f}{\mathrm{D} x_{2}}, \cdots, \frac{\mathrm{D} f}{\mathrm{D} x_{n}}\right)^{\mathrm{T}} \\
& =\left[\begin{array}{c}
\sum_{j=1}^{n} a_{j 1} x_{j}+\sum_{j=1}^{n} a_{1 j} x_{j} \\
\sum_{j=1}^{n} a_{j 2} x_{j}+\sum_{j=1}^{n} a_{2 j} x_{j} \\
\vdots \\
\sum_{j=1}^{n} a_{j n} x_{j}+\sum_{j=1}^{n} a_{n j} x_{j}
\end{array}\right]
\end{aligned}
$$

$$
\begin{array}{r}
=\left[\begin{array}{c}
\sum_{j=1}^{n} a_{j 1} x_{j} \\
\sum_{j=1}^{n} a_{j 2} x_{j} \\
\vdots \\
\sum_{j=1}^{n} a_{j n} x_{j}
\end{array}\right]+\left[\begin{array}{c}
\sum_{j=1}^{n} a_{1 j} x_{j} \\
\sum_{j=1}^{n} a_{2 j} x_{j} \\
\vdots \\
\sum_{j=1}^{n} a_{n j} x_{j}
\end{array}\right] \\
=\boldsymbol{A}^{\mathrm{T}} \boldsymbol{X}+\boldsymbol{A} \boldsymbol{X}=\left(\boldsymbol{A}^{\mathrm{T}}+\boldsymbol{A}\right) \boldsymbol{X}
\end{array}
$$

所以

$$
\frac{\mathrm{D} \boldsymbol{f}}{\mathrm{D} \boldsymbol{X}^{\mathrm{T}}}=\left(\frac{\mathrm{D} \boldsymbol{f}}{\mathrm{D} \boldsymbol{X}}\right)^{\mathrm{T}}=\boldsymbol{X}^{\mathrm{T}}\left(\boldsymbol{A}+\boldsymbol{A}^{\mathrm{T}}\right)
$$

例9.2.4 设 $X(t)=\left(x_{1}(t), x_{2}(t), \cdots x_{n}(t)\right)^{\mathrm{T}}$ 为向量变量,一元函数 $f(t)= f(\boldsymbol{X}(t))=f\left(x_{1}(t), x_{2}(t), \cdots, x_{n}(t)\right)$ ,求 $\frac{\mathrm{d} f}{\mathrm{~d} t}$ .

解 由微积分知识

$$
\begin{aligned}
\frac{\mathrm{d} f}{\mathrm{~d} t} & =\frac{\partial f \mathrm{~d} x_{1}}{\partial x_{1} \mathrm{~d} t}+\frac{\partial f \mathrm{~d} x_{2}}{\partial x_{2} \mathrm{~d} t}+\cdots+\frac{\partial f \mathrm{~d} x_{n}}{\partial x_{n} \mathrm{~d} t} \\
& =\left(\frac{\partial f}{\partial x_{1}}, \frac{\partial f}{\partial x_{2}}, \frac{\partial f}{\partial x_{n}}\right)\left[\begin{array}{c}
\frac{\mathrm{d} x_{1}}{\mathrm{~d} t} \\
\frac{\mathrm{~d} x_{2}}{\mathrm{~d} t} \\
\vdots \\
\frac{\mathrm{~d} x_{n}}{\mathrm{~d} t}
\end{array}\right] \\
& =\frac{\mathrm{D} f}{\mathrm{D} \boldsymbol{X}} \cdot \frac{\mathrm{~d} \boldsymbol{X}}{\mathrm{~d} t}
\end{aligned}
$$

例 9.2.5 已知矩阵变量

$$
\boldsymbol{X}=\left[\begin{array}{cccc}
x_{11} & x_{12} & \cdots & x_{1 n} \\
x_{21} & x_{22} & \cdots & x_{2 n} \\
\vdots & \vdots & & \vdots \\
x_{m 1} & x_{m 2} & \cdots & x_{m n}
\end{array}\right]
$$

$m n$ 元数量函数为

$$
f(X)=\operatorname{tr}\left(X X^{\mathrm{T}}\right)=\sum_{i=1}^{m} \sum_{j=1}^{n} x_{i j}^{2}
$$

求 $\frac{\mathrm{d} f}{\mathrm{~d} \boldsymbol{X}}$ .

解 由题意

$$
\frac{\partial f}{\partial x_{i j}}=2 x_{i j} \quad(i=1,2, \cdots m ; j=1,2 \cdots n)
$$

所以

$$
\frac{\mathrm{d} f}{\mathrm{~d} X}=\left[\frac{\partial f}{\partial x_{i j}}\right]_{m \times n}=\left(2 x_{i j}\right)_{m \times n}=2 X
$$

例9.2.6 $n$ 个变量 $x_{1}, x_{2}, \cdots x_{n}$ 的二次型

$$
\boldsymbol{f}=\sum_{i, j=1}^{n} a_{i j} x_{i} x_{j}=\boldsymbol{X}^{\mathrm{T}} \boldsymbol{A} \boldsymbol{X}
$$

其中

$$
\boldsymbol{X}=\left(x_{1}, x_{2} \cdots x_{n}\right)^{\mathrm{T}}, \boldsymbol{A}=\left(a_{i j}\right)_{n \times n}, \boldsymbol{A}^{\mathrm{T}}=\boldsymbol{A}
$$

则由式(9.2.9)和式(9.2.10)可得

$$
\begin{aligned}
\text { grad } \boldsymbol{f} & =\frac{\mathrm{D} \boldsymbol{f}}{\mathrm{D} \boldsymbol{X}^{\mathrm{T}}}=\frac{\mathrm{D}\left(\boldsymbol{X}^{\mathrm{T}} \boldsymbol{A} \boldsymbol{X}\right)}{\mathrm{D} \boldsymbol{X}^{\mathrm{T}}}=\frac{\boldsymbol{D}\left[(\boldsymbol{A} \boldsymbol{X})^{\mathrm{T}} \boldsymbol{X}\right]}{\mathrm{D} \boldsymbol{X}^{\mathrm{T}}} \\
& =(\boldsymbol{A} \boldsymbol{X})^{\mathrm{T}} \frac{\mathrm{D} \boldsymbol{X}}{\mathrm{D} \boldsymbol{X}^{\mathrm{T}}}+\boldsymbol{X}^{\mathrm{T}} \frac{\mathrm{D}(\boldsymbol{A} \boldsymbol{X})}{\mathrm{D} \boldsymbol{X}^{\mathrm{T}}} \\
& =(\boldsymbol{A} \boldsymbol{X})^{\mathrm{T}}+\boldsymbol{X}^{\mathrm{T}} \boldsymbol{A}=2 \boldsymbol{X}^{\mathrm{T}} \boldsymbol{A} \\
\frac{\mathrm{D}}{\mathrm{D} \boldsymbol{X}}\left(\frac{\mathrm{D} \boldsymbol{f}}{\mathrm{D} \boldsymbol{X}^{\mathrm{T}}}\right) & =\frac{\mathrm{D}}{\mathrm{D} \boldsymbol{X}^{\prime}}\left(2 \boldsymbol{X}^{\mathrm{T}} \boldsymbol{A}\right)=2 \boldsymbol{A}
\end{aligned}
$$

\section*{§9.3 Kronecker 积的特征值}
本节讨论 $\boldsymbol{A}, \boldsymbol{B}$ 的特征值与 $\boldsymbol{A} \otimes \boldsymbol{B}$ 的特征值之间的关系\\
考虑由变量 $x, y$ 组成的复系数多项式

$$
f(x, y)=\sum_{i, j=0}^{l} c_{i j} x^{i} y^{j}
$$

若 $\boldsymbol{A}$ 为 $m$ 阶矩阵, $\boldsymbol{B}$ 为 $n$ 阶矩阵,考虑由下式确定的 $m n$ 阶矩阵

$$
f(A, B)=\sum_{i, j=0}^{l} c_{i j} A^{i} \otimes B^{j}
$$

例如,设 $f(x, y)=2 x+x y^{3}$ ,把 $f(x, y)$ 写成

$$
f(x, y)=2 x^{1} y^{0}+x^{1} y^{3}
$$

于是

$$
f(A, B)=2 A \otimes E+A \otimes B^{3}
$$

定理9.3.1 设 $\lambda_{1}, \lambda_{2}, \cdots, \lambda_{m}$ 是 $m$ 阶矩阵 $A$ 的特征值,$x_{1}, x_{2}, \cdots, x_{m}$ 是 $A$ 的属于 $\lambda_{1}, \lambda_{2}, \cdots, \lambda_{m}$ 的特征向量,$\mu_{1}, \mu_{2}, \cdots, \mu_{n}$ 是 $n$ 阶矩阵 $B$ 的特征值,$y_{1}, y_{2} \cdots, y_{n}$是 $\boldsymbol{B}$ 的属于 $\mu_{1}, \mu_{2}, \cdots, \mu_{n}$ 的特征向量,那么 $m n$ 个数 $f\left(\lambda_{r}, \mu_{s}\right) \quad(r=1,2, \cdots m ; s=$\\
$1,2, \cdots n)$ 是 $m n$ 阶矩阵 $f(A, B)$ 的特征值,$x_{r} \otimes y_{s}$ 是矩阵 $f(A, B)$ 属于 $f\left(\lambda_{r}, \mu_{s}\right)$ 的特征向量,( $r=1,2, \cdots m ; s=1,2, \cdots, n$ ).

证明 由已知条件知

$$
\begin{gathered}
\boldsymbol{A} \boldsymbol{x}_{r}=\lambda_{r} \boldsymbol{x}_{r}, \boldsymbol{B} \boldsymbol{y}_{s}=\mu_{s} \boldsymbol{y}_{s} \\
\boldsymbol{A}^{i} \boldsymbol{x}_{r}=\lambda_{r}^{i} \boldsymbol{x}_{r}, \boldsymbol{B}^{i} \boldsymbol{y}_{s}=\mu_{s}^{i} \boldsymbol{y}_{s}
\end{gathered}
$$

于是

$$
\begin{aligned}
\boldsymbol{f}(\boldsymbol{A}, \boldsymbol{B})\left(\boldsymbol{x}_{r} \otimes \boldsymbol{y}_{s}\right) & =\left(\sum_{i, j=0}^{l} c_{i j} \boldsymbol{A}^{i} \otimes \boldsymbol{B}^{i}\right)\left(\boldsymbol{x}_{r} \otimes \boldsymbol{y}_{s}\right) \\
& =\sum_{i, j=0}^{l} c_{i j}\left(\boldsymbol{A}^{i} \otimes \boldsymbol{B}^{i}\right)\left(\boldsymbol{x}_{r} \otimes \boldsymbol{y}_{s}\right) \\
& =\sum_{i, j=0}^{l} c_{i j}\left(\boldsymbol{A}^{i} \boldsymbol{x}_{r} \otimes \boldsymbol{B}^{j} \boldsymbol{y}_{s}\right) \\
& =\sum_{i, j=0}^{l} c_{i j} \lambda_{r}^{i} \mu_{s}^{j} \boldsymbol{x}_{r} \otimes \boldsymbol{y}_{s} \\
& =f\left(\lambda_{r}, \mu_{s}\right)\left(\boldsymbol{x}_{r} \otimes \boldsymbol{y}_{s}\right)
\end{aligned}
$$

推论 $1 \boldsymbol{A} \otimes \boldsymbol{B}$ 的 $m n$ 个特征值为 $\lambda_{r} \mu_{s},(r=1,2, \cdots, m ; s=1,2, \cdots, n) \lambda_{r} \mu_{s}$ 对应的特征向量是 $\boldsymbol{x}_{r} \otimes y_{s} .(r=1,2, \cdots, m ; s=1,2, \cdots, n)$ 。

推论2 $\boldsymbol{A} \otimes \boldsymbol{E}_{n}+\boldsymbol{E}_{m} \otimes \boldsymbol{B}$ 的特征值是 $\lambda_{r}+\mu_{s}$ ,其对应的特征向量是 $\boldsymbol{x}_{r} \otimes \boldsymbol{y}_{s}(r= 1,2, \cdots, m ; s=1,2, \cdots, n)$ ,

矩阵 $\boldsymbol{A} \otimes \boldsymbol{E}_{n}+\boldsymbol{E}_{m} \otimes \boldsymbol{B}$ 称为矩阵 $\boldsymbol{A}$ 与 $\boldsymbol{B}$ 的 Kronecker 和.

\section*{§9.4 矩阵的列展开与行展开}
定义9.4.1 设 $A=\left(a_{i j}\right)_{m \times n}$ 将 $A$ 的各行依次横排得到 $m n$ 维行向量,称为矩阵 $A$ 的行展开,记为 $\mathrm{rs}(A)$ ,即

$$
\operatorname{rs}(\boldsymbol{A})=\left(a_{11}, a_{12}, \cdots, a_{1 n}, a_{21}, a_{22}, \cdots, a_{2 n}, \cdots, a_{m 1}, a_{m 2}, \cdots, a_{m n}\right)
$$

将 $\boldsymbol{A}$ 的各列依次纵排得到 $m n$ 维列向量称为矩阵 $\boldsymbol{A}$ 的列展开,记为 $\operatorname{cs}(\boldsymbol{A})$ 即

$$
\operatorname{cs}(\boldsymbol{A})=\left(a_{11}, a_{21}, \cdots, a_{m 1}, a_{12}, a_{22}, \cdots, a_{m 2}, \cdots, a_{1 n}, a_{2 n}, \cdots, a_{n n}\right)^{\mathrm{T}}
$$

根据定义9.4.1容易得到

$$
\begin{aligned}
& \operatorname{rs}\left(\boldsymbol{A}^{\mathrm{T}}\right)=(\operatorname{cs}(\boldsymbol{A}))^{\mathrm{T}} \\
& \operatorname{cs}\left(\boldsymbol{A}^{\mathrm{T}}\right)=(r s(\boldsymbol{A}))^{\mathrm{T}}
\end{aligned}
$$

定理9.4.1 设 $\boldsymbol{A}=\left(a_{i j}\right)_{m \times n}, \boldsymbol{B}=\left(b_{i j}\right)_{n \times p}, \boldsymbol{C}=\left(c_{i j}\right)_{p \times q}$ ,则

$$
\begin{aligned}
& \operatorname{rs}(A B C)=\operatorname{rs}(B)\left(A^{\mathrm{T}} \otimes C\right) \\
& \operatorname{cs}(A B C)=\left(C^{\mathrm{T}} \otimes A\right) \operatorname{cs}(B)
\end{aligned}
$$

证明 记 $\boldsymbol{A}$ 的第 $i$ 行向量为 $\boldsymbol{\alpha}_{i}, \boldsymbol{B}$ 的第 $i$ 行向量为 $\boldsymbol{\beta}_{i}, \boldsymbol{C}$ 的第 $i$ 列向量为 $\boldsymbol{\nu}_{i}$ 则

$$
\begin{aligned}
\operatorname{rs}(\boldsymbol{A} \boldsymbol{B} \boldsymbol{C}) & =\operatorname{rs}\left\{\left[\begin{array}{c}
\boldsymbol{\alpha}_{1} \\
\boldsymbol{\alpha}_{2} \\
\vdots \\
\boldsymbol{\alpha}_{m}
\end{array}\right]\left[\begin{array}{c}
\boldsymbol{\beta}_{1} \\
\boldsymbol{\beta}_{2} \\
\vdots \\
\boldsymbol{\beta}_{n}
\end{array}\right]\left(\boldsymbol{\nu}_{1} \boldsymbol{\nu}_{2} \cdots \boldsymbol{\nu}_{q}\right)\right\} \\
& =\operatorname{rs}\left\{\left[\begin{array}{c}
\boldsymbol{\alpha}_{1} \\
\boldsymbol{\alpha}_{2} \\
\vdots \\
\boldsymbol{\alpha}_{m}
\end{array}\right]\left[\begin{array}{cccc}
\boldsymbol{\beta}_{1} \boldsymbol{\nu}_{1} & \boldsymbol{\beta}_{1} \boldsymbol{\nu}_{2} & \cdots & \boldsymbol{\beta}_{1} \boldsymbol{\nu}_{q} \\
\boldsymbol{\beta}_{2} \boldsymbol{\nu}_{1} & \boldsymbol{\beta}_{2} \boldsymbol{\nu}_{2} & \cdots & \boldsymbol{\beta}_{2} \boldsymbol{\nu}_{q} \\
\vdots & \vdots & & \vdots \\
\boldsymbol{\beta}_{n} \boldsymbol{\nu}_{1} & \boldsymbol{\beta}_{n} \boldsymbol{\nu}_{2} & \cdots & \boldsymbol{\beta}_{n} \boldsymbol{\nu}_{q}
\end{array}\right]\right\} \\
& =\left(\boldsymbol{\alpha}_{1} \boldsymbol{B} \boldsymbol{C}, \boldsymbol{\alpha}_{1} \boldsymbol{B} \boldsymbol{C}, \cdots \boldsymbol{\alpha}_{m} \boldsymbol{B} \boldsymbol{C},\right)
\end{aligned}
$$

可以验证

$$
\boldsymbol{\alpha}_{i} \boldsymbol{B} \boldsymbol{C}=\operatorname{rs}(\boldsymbol{B})\left(\boldsymbol{\alpha}_{i}^{\mathrm{T}} \otimes \boldsymbol{C}\right)
$$

因此

$$
\begin{aligned}
\mathrm{rs}(\boldsymbol{A} \boldsymbol{B} \boldsymbol{C}) & =\operatorname{rs}(\boldsymbol{B})\left(\boldsymbol{\alpha}_{1}^{\mathrm{T}} \otimes \boldsymbol{C}, \boldsymbol{\alpha}_{2}^{\mathrm{T}} \otimes \boldsymbol{C}, \cdots,\left(\boldsymbol{\alpha}_{m}^{\mathrm{T}} \otimes \boldsymbol{C}\right)\right. \\
& =\operatorname{rs}(\boldsymbol{B})\left(\boldsymbol{A}^{\mathrm{T}} \otimes \boldsymbol{C}\right)
\end{aligned}
$$

同理可证

$$
\operatorname{cs}(\boldsymbol{A} \boldsymbol{B} \boldsymbol{C})=\left(\boldsymbol{C}^{\mathbf{T}} \otimes \boldsymbol{A}\right) \operatorname{cs}(\boldsymbol{B})
$$

推论 1 设 $\boldsymbol{A} \in C^{m \times m}, ~ B \in C^{n \times n}, ~ X \in C^{m \times n}$ ,则\\
(1) $\operatorname{cs}(\boldsymbol{A} \boldsymbol{X})=\left(\boldsymbol{E}_{n} \otimes \boldsymbol{A}\right) \operatorname{cs}(\boldsymbol{X})$\\
(2) $\operatorname{cs}(\boldsymbol{X B})=\left(\boldsymbol{B}^{\mathrm{T}} \otimes \boldsymbol{E}_{m}\right) \operatorname{cs}(\boldsymbol{X})$\\
(3) $\operatorname{cs}(\boldsymbol{A} \boldsymbol{X}+\boldsymbol{X} \boldsymbol{B})=\left(\boldsymbol{E}_{n} \otimes \boldsymbol{A}+\boldsymbol{B}^{\mathrm{T}} \otimes \boldsymbol{E}_{m}\right) \operatorname{cs}(\boldsymbol{X})$\\
证明 $(1) \operatorname{cs}(\boldsymbol{A} \boldsymbol{X})=\operatorname{cs}\left(\boldsymbol{A} \boldsymbol{X} \boldsymbol{E}_{n}\right)=\left(\boldsymbol{E}_{n} \otimes \boldsymbol{A}\right) \operatorname{cs}(\boldsymbol{X})$\\
(2) $\operatorname{cs}(\boldsymbol{X B})=\operatorname{cs}\left(\boldsymbol{E}_{m} \boldsymbol{X B}\right)=\left(\boldsymbol{B}^{\mathrm{T}} \otimes \boldsymbol{E}_{m}\right) \operatorname{cs}(\boldsymbol{X})$\\
(3) $\operatorname{cs}(\boldsymbol{A} \boldsymbol{X}+\boldsymbol{X} \boldsymbol{B})=\operatorname{cs}(\boldsymbol{A} \boldsymbol{X})+\operatorname{cs}(\boldsymbol{X} \boldsymbol{B})=\left(\boldsymbol{E}_{n} \otimes \boldsymbol{A}+\boldsymbol{B}^{\mathrm{T}} \otimes \boldsymbol{E}_{m}\right) \operatorname{cs}(\boldsymbol{X})$\\
推论 2 设 $\boldsymbol{A} \in C^{m \times r}, ~ \boldsymbol{B} \in C^{r \times n}$ ,则

证明

$$
\begin{aligned}
& \operatorname{rs}(\boldsymbol{A B})=\operatorname{rs}\left(\boldsymbol{A}\left(\boldsymbol{E}_{m} \otimes \boldsymbol{B}\right)\right)=\operatorname{rs}\left(\boldsymbol{B}\left(\boldsymbol{A}^{T} \otimes \boldsymbol{E}_{n}\right)\right) \\
& \operatorname{rs}(\boldsymbol{A B})=\operatorname{rs}\left(\boldsymbol{E}_{m} \boldsymbol{A B}\right)=\operatorname{rs}\left(\boldsymbol{A}\left(\boldsymbol{E}_{m} \otimes \boldsymbol{B}\right)\right) \\
& \operatorname{rs}(\boldsymbol{A B})=\operatorname{rs}\left(\boldsymbol{A B} \boldsymbol{E}_{n}\right)=\operatorname{rs}\left(\boldsymbol{B}\left(\boldsymbol{A}^{\mathrm{T}} \otimes \boldsymbol{E}_{n}\right)\right)
\end{aligned}
$$

\section*{§9.5 线性矩阵代数方程}
这一节讨论形如


\begin{equation*}
\boldsymbol{A}_{1} \boldsymbol{X} \boldsymbol{B}_{1}+\boldsymbol{A}_{2} \boldsymbol{X} \boldsymbol{B}_{2}+\cdots+\boldsymbol{A}_{p} \boldsymbol{X} \boldsymbol{B}_{p}=\boldsymbol{C} \tag{9.5.1}
\end{equation*}


的线性矩阵代数方程,通常称这样的方程为Sylvester线性矩阵方程,其中 $A_{j} \in C^{m \times m}, B_{j} \in C^{n \times n}(j=1,2, \cdots p), X, C \in C^{m \times n}$ .

对方程(9.5.1)可以构造一个对应的线性方程组


\begin{equation*}
\boldsymbol{G} \boldsymbol{x}=\boldsymbol{c} \tag{9.5.2}
\end{equation*}


其中 $\boldsymbol{x}=\operatorname{cs}(\boldsymbol{X}), \boldsymbol{c}=\operatorname{cs}(\boldsymbol{C}) \quad \boldsymbol{G}=\sum_{j=1}^{p}\left(\boldsymbol{B}_{j}^{\mathrm{T}} \otimes \boldsymbol{A}_{j}\right)$\\
定理 9.5.1 矩阵 $\boldsymbol{X} \in C^{m \times n}$ 是方程(9.5.1)的解的充要条件是 $\boldsymbol{x}=\operatorname{cs}(\boldsymbol{X})$ 是方程(9.5.2)的解。

证明 对方程(9.5.1)实施列展开得

$$
\begin{aligned}
\operatorname{cs}(\boldsymbol{C}) & =\operatorname{cs}\left(\sum_{j=1}^{p}\left(\boldsymbol{A}_{j} \boldsymbol{X} \boldsymbol{B}_{j}\right)\right) \\
& \left.=\sum_{j=1}^{p} \operatorname{cs}\left(\boldsymbol{A}_{j} \boldsymbol{X} \boldsymbol{B}_{j}\right)\right) \\
& =\sum_{j=1}^{p}\left(\boldsymbol{B}_{j}^{\mathrm{T}} \otimes \boldsymbol{A}_{j}\right) \operatorname{cs}(\boldsymbol{X}) \\
& =\boldsymbol{G} \operatorname{cs}(\boldsymbol{X})
\end{aligned}
$$

此即

$$
\boldsymbol{G} \boldsymbol{x}=\boldsymbol{c}
$$

所以方程(9.5.1)的解与方程(9.5.2)的解相同.定理得证.\\
推论1 方程(9.5.1)有解的充要条件是 $\operatorname{rank}(\boldsymbol{G}, \boldsymbol{c})=\operatorname{rank}(\boldsymbol{G})$\\
推论2 方程(9.5.1)有唯一解的充要条件是 $\boldsymbol{G}$ 为非奇异的。\\
下面讨论方程(9.5.1)的两个重要的特殊情况\\
方程


\begin{equation*}
A X+X B=C \tag{9.5.3}
\end{equation*}


定理 9.5.2 方程 $\boldsymbol{A} \boldsymbol{X}+\boldsymbol{X B}=\boldsymbol{C}$ 有唯一解的充要条件是 $\boldsymbol{A}$ 与 $\boldsymbol{B}$ 的特征值满足

$$
\lambda_{i}(\boldsymbol{A})+\lambda_{j}(\boldsymbol{B}) \neq 0 \quad(\forall i, j) .
$$

$(i=1,2, \cdots, m ; j=1,2, \cdots, n)$ ,其中 $\boldsymbol{\lambda}_{i}(\boldsymbol{A})$ 表示 $\boldsymbol{A}$ 的第 $i$ 个特征值,$\lambda_{j}(\boldsymbol{B})$ 表示 $\boldsymbol{B}$的第 $j$ 个特征值.

证明 由定理9.5.1知方程(9.5.3)对应的线性方程组是

$$
\begin{aligned}
\operatorname{cs}(\boldsymbol{C}) & =\operatorname{cs}(\boldsymbol{A} \boldsymbol{X}+\boldsymbol{X} \boldsymbol{B}) \\
& =\left(\boldsymbol{E}_{n} \otimes \boldsymbol{A}+\boldsymbol{B}^{\mathrm{T}} \otimes \boldsymbol{E}_{m}\right) \operatorname{cs}(\boldsymbol{X})
\end{aligned}
$$

由推论 2 知,该方程组有唯一解的充要条件是矩阵 $\boldsymbol{E}_{n} \otimes \boldsymbol{A}+\boldsymbol{B}^{\mathrm{T}} \otimes \boldsymbol{E}_{m}$ 是非奇异的,即矩阵 $\boldsymbol{E}_{n} \otimes \boldsymbol{A}+\boldsymbol{B}^{\mathrm{T}} \otimes \boldsymbol{E}_{m}$ 没有零特征值,由推论2知,矩阵 $\boldsymbol{E}_{n} \otimes \boldsymbol{A}+\boldsymbol{B}^{\mathrm{T}} \otimes \boldsymbol{E}_{m}$ 的特征值是 $\boldsymbol{\lambda}_{i}(\boldsymbol{A})+\lambda_{j}(\boldsymbol{B})$ 。于是有 $\boldsymbol{\lambda}_{i}(\boldsymbol{A})+\lambda_{j}(\boldsymbol{B}) \neq 0,(\forall i, j)$ 。

推论1 矩阵代数方程

$$
A X+X B=0
$$

有非零矩阵 $\boldsymbol{X}$ 的充分必要条件是对于某一个 $i$ 与 $j$ 有

$$
\lambda_{i}(\boldsymbol{A})+\mu_{j}(\boldsymbol{B})=0
$$

$(i=1,2, \cdots, m ; j=1,2, \cdots, n)$\\
推论2 若 $\operatorname{Re} \lambda_{i}(\boldsymbol{A})<0$ 且 $\operatorname{Re} \lambda_{i}(\boldsymbol{B})<0(i=1,2, \cdots, m)$ ,则方程 $\boldsymbol{A} \boldsymbol{X}+\boldsymbol{X} \boldsymbol{B}= \boldsymbol{C}$ 有唯一解,其中 $\operatorname{Re} \lambda_{i}(\boldsymbol{A})$ 表示特征值 $\lambda_{i}$ 的实部。

方程(9.5.3)在某些特殊条件下可得其解的表达式。\\
定理9.5.3 若矩阵 $A \in C^{m \times m}, B \in C^{n \times n}$ 的所有特征值只有负实部,则方程

$$
A X+X B=C
$$

有唯一解,且可以表示为


\begin{equation*}
\boldsymbol{X}=-\int_{0}^{\infty} \mathrm{e}^{A t} \mathrm{Ce}^{B t} \mathrm{~d} t \tag{9.5.4}
\end{equation*}


(证略)\\
定理 9.5.4 设 $A \in C^{m \times m}, B \in C^{n \times n}, X \in C^{m \times n}$ ,则\\
矩阵代数方程


\begin{equation*}
X+A X B=C \tag{9.5.5}
\end{equation*}


有唯一解的充要条件是

$$
\lambda_{i}(A) \mu_{j}(B) \neq-1
$$

$(i=1,2, \cdots, m ; j=1,2, \cdots, n), \lambda_{i}(\boldsymbol{A})$ 与 $\mu_{j}(\boldsymbol{B})$ 分别为 $\boldsymbol{A}$ 与 $\boldsymbol{B}$ 的特征值.\\
证明 把方程(9.5.5)两边按列展开有

$$
\begin{aligned}
\operatorname{cs}(\boldsymbol{C}) & =\operatorname{cs}\left(\boldsymbol{E}_{m} \boldsymbol{X} \boldsymbol{E}_{n}+\boldsymbol{A} \boldsymbol{X B}\right) \\
& =\left(\boldsymbol{E}_{n} \otimes \boldsymbol{E}_{m}+\boldsymbol{B}^{\mathrm{T}} \otimes \boldsymbol{A}\right) \operatorname{cs}(\boldsymbol{X})
\end{aligned}
$$

于是方程(9.5.5)有唯一解的充要条件是矩阵 $\boldsymbol{E}_{n} \otimes \boldsymbol{E}_{m}+\boldsymbol{B}^{\mathrm{T}} \otimes \boldsymbol{A}$ 的特征值不全为零。由定理9.3.1的两个推论得

$$
1+\lambda_{i}(A) \mu_{j}(B) \neq 0
$$

即

$$
\lambda_{i}(A) \mu_{j}(B) \neq-1 \quad(\forall i, j)
$$

$(i=1,2 \cdots, m ; j=1,2, \cdots, n)$

\section*{符 号 说 明}
\begin{center}
\begin{tabular}{|l|l|}
\hline
R & 实数集合、实数域 \\
\hline
C & 复数集合、复数域 \\
\hline
$R^{n}$ & $n$ 维(元)实向量集合(空间) \\
\hline
$C^{n}$ & $n$ 维(元)复向量集合(空间) \\
\hline
$R^{m \times n}\left(C^{m \times n}\right)$ & $m \times n$ 实(复)矩阵集合 \\
\hline
$R^{n \times n}\left(C^{n \times n}\right)$ & $n$ 阶实(复)矩(方)阵集合 \\
\hline
$\boldsymbol{R} \boldsymbol{(} \boldsymbol{A} \boldsymbol{)}$ & 矩阵 $\boldsymbol{A}$ 的值域、矩阵 $A$ 的列空间 \\
\hline
$N(\boldsymbol{A})$ & 矩阵 $A$ 的核空间、矩阵 $A$ 的零空间 \\
\hline
$\operatorname{dim} V$ & 线性空间的维数 \\
\hline
$\operatorname{span}\left\{\alpha_{1}, \alpha_{2}, \cdots, \alpha_{m}\right\}$ & 由向量 $\boldsymbol{\alpha}_{1}, \boldsymbol{\alpha}_{2}, \cdots, \boldsymbol{\alpha}_{m}$ 生成的子空间 \\
\hline
$\operatorname{rank} A$ & 矩阵 $\boldsymbol{A}$ 的秩 \\
\hline
$F_{r}^{m \times n}\left(R_{r}^{m \times n}, C_{r}^{m \times n}\right)$ & 元素在数域 $F$(实数域 $\mathbf{R}$ 、复数域 $\mathbf{C})$ 中秩为 $r$ 的 $m \times n$ 矩阵集合 \\
\hline
$R(A B)$ & 线性变换 $\mathscr{B}$ 的值域 \\
\hline
$N(\mathscr{A})$ & 线性变换 $\mathscr{A}$ 的核空间 \\
\hline
$V_{\lambda i}$ & 特征值是 $\lambda_{i}$ 的特征子空间 \\
\hline
$\operatorname{tr}(\boldsymbol{A})$ & 矩阵 $\boldsymbol{A}$ 的迹 \\
\hline
$U^{n \times n}$ & $\boldsymbol{n}$ 阶酉矩阵集合 \\
\hline
$U_{r}^{m \times r}$ & $r$ 个列向量组是标准正交向量组的 $m \times r$ 矩阵集合 \\
\hline
$U_{r}^{\times n}$ & $r$ 个行向量组是标准正交向量组的 $r \times n$矩阵集合 \\
\hline
$E^{n \times n}$ & $n$ 阶正交矩阵集合 \\
\hline
$\lambda(\boldsymbol{A})$ & 矩阵 $\boldsymbol{A}$ 的谱,矩阵 $\boldsymbol{A}$ 的所有特征值的集合 \\
\hline
$\rho(\boldsymbol{A})$ & 矩阵 $\boldsymbol{A}$ 的谱半径 \\
\hline
$\operatorname{det} \boldsymbol{A}$ & 矩阵 $\boldsymbol{A}$ 的行列式 \\
\hline
$\boldsymbol{A}^{-}\left(\boldsymbol{A}_{\mathbf{L}}^{-}\right)$ & 矩阵 $\boldsymbol{A}$ 的广义逆矩阵(矩阵 $A$ 的自反广义逆矩阵) \\
\hline
$A^{+}$ & 矩阵 $\boldsymbol{A}$ 的伪逆矩阵 \\
\hline
$\boldsymbol{A}_{\mathbf{L}}^{-1}\left(\boldsymbol{A}_{\mathbf{R}}^{-1}\right)$ & 矩阵 $\boldsymbol{A}$ 的左(右)逆 \\
\hline
$\boldsymbol{A}^{\boldsymbol{*}}, \operatorname{adj} \boldsymbol{A}$ & 矩阵 $\boldsymbol{A}$ 的伴随矩阵 \\
\hline
\end{tabular}
\end{center}

\begin{center}
\begin{tabular}{|l|l|}
\hline
$\boldsymbol{A}^{\mathrm{T}}\left(\boldsymbol{A}^{\mathrm{H}}\right)$ & 矩阵的转置(矩阵 $A$ 的共轭转置,矩阵的转置共轭) \\
\hline
$V_{1} \oplus V_{2}$ & 子空间 $V_{1}$ 与 $V_{2}$ 的直和 \\
\hline
$V_{1}$(1)$V_{2}$ & 子空间 $V_{1}$ 与 $V_{2}$ 的正交和 \\
\hline
$\boldsymbol{S}_{\perp}$ & 子空间 $S$ 的正交补 \\
\hline
$\operatorname{Re} \lambda(\operatorname{Im} \lambda)$ & 复数 $\lambda$ 的实部(虚部) \\
\hline
$\|\cdot\|$ & 范数 \\
\hline
$\boldsymbol{A} \otimes \boldsymbol{B}$ & 矩阵 $\boldsymbol{A}$ 与 $\boldsymbol{B}$ 的 Kronecker 积(直积) \\
\hline
$\mathrm{rs}(\boldsymbol{A})$ & 矩阵 $\boldsymbol{A}$ 的行展开 \\
\hline
$\operatorname{cs}(\boldsymbol{A})$ & 矩阵 $\boldsymbol{A}$ 的列展开 \\
\hline
$\frac{\mathrm{D} \boldsymbol{A}}{\mathrm{D} \boldsymbol{B}}$ & 矩阵 $\boldsymbol{A}$ 对矩阵 $\boldsymbol{B}$ 的导数 \\
\hline
\end{tabular}
\end{center}

\section*{参 考 文 献}
[1]TAHTMAXEP $\boldsymbol{\phi}$ .P.矩阵论[M].柯召,译.北京:高等教育出版社, $\mathbf{1 9 9 5 .}$\\
[2]黄琳.系统与控制理论中的线性代数[M].北京:科学出版社, 1984.\\
[3]须田信英,等.自动控制中的矩阵理论[M].曹长修,译.北京:科学出版社,1979.\\
[4]王耕禄,史荣昌.矩阵理论[M].北京:国防工业出版社,1988.\\
[5]王朝瑞,史荣昌.矩阵分析[M].北京:北京理工大学出版社, 1989.\\
[6]谢国瑞.应用矩阵方法[M].北京;化学工业出版社, 1988.\\
[7]Bellman.R.Introduction to Matrix Analysis(Second Edition)[M].The Rand Corporation, 1970.\\
[8]Horn R A.and Johnson C R.Matrix Analysis[M].Cambridge University Press, 1985.\\
[9]P Lancaster and M.Tismenetsky.The Theory of Matrices(Second Edition)[M]. Academic Press,Inc.,1985.\\
[10]黄延祝,钟守铭,李正良.矩阵理论[M].北京:高等教育出版社,2003.\\
[11]戴华.矩阵论[M].北京:科学出版社,2001.\\
[12]程云鹏,等.矩阵论(第二版)[M].西安:西北工业大学出版社,2003.\\
[13]董增福.矩阵分析教程[M].哈尔滨:哈尔滨工业大学出版社,2003.\\
[14]丘维声.高等代数(上,下册)[M].北京:高等教育出版社, 1996.\\
[15]方保镕,周继东,李医民.矩阵论[M].北京:清华大学出版社,2004.\\
[16]林升旭.矩阵论.学习辅导与典型题解析[M].武汉:华中科技大学出版社,2003.

责任编辑:张玉荣\\
封面设计:庚辰年代


\end{document}